%# -*- coding:utf-8 -*-
%%%%%%%%%%%%%%%%%%%%%%%%%%%%%%%%%%%%%%%%%%%%%%%%%%%%%%%%%%%%%%%%%%%%%%%%%%%%%%%%%%%%%


\chapter{大事年表 }

20世纪50年代初——吴家盗血尸墓。

1952年——裘德考回美,老九门衰落。

1956年——考古队广西上思张家铺遗址考古。

1963-1965年——张大佛爷领衔,老九门悉数参与史上最大盗墓活动。

1970年——得力于大金牙的翻译组织完成了对张家古楼的研究。

1974年——陈皮阿四倒斗镜儿宫,裘德考解开战国帛书,并组织了对龙脉的首次探索。

1976年——原考古队巴乃考古,实为送葬。

1977年——吴邪出生。

1978年左右——原考古队被掉包。

1979年后——解九爷的队伍走投无路,投靠杭州的吴邪爷爷,最后吴邪爷爷以金蝉脱壳之计将那具尸体藏于南宋皇陵之内。

1982年左右——吴三省抢在裘德考队伍之前,单枪匹马再探血尸墓。

1985年左右——考古队进入西沙海底墓,中招后被囚禁于疗养院。解连环与吴三省首次联手。

1990年——组织封存巴乃考古资料,解除疗养院的监视。文锦一行仍然以疗养院为基地,继续研究,并建立录像带机制。

1993年——通过对海底墓中带出的资料的研究,文锦等发现了长白山的线索,并决定前往。

1993年6月18日——在长白山云顶天宫看,文锦见到了终极。

1995年——文锦一行找到了传说中的西王母国。此行之后,霍玲开始尸变。

1995年——1999年霍老太收到神秘录像带。

2000年左右——小哥回到广西巴乃,不料失忆症发作,被当做肉饵放入古墓中钓尸,被陈皮阿四所救。

2003年2月1日——大金牙带着战国帛书找到吴邪,吴邪的盗墓之旅拉开序幕。

2003年2月——七星鲁王宫。

2003年3月——西沙海底墓。

2003年秋——秦岭神树。

2003年冬——云顶天宫。

2004年5月——蛇沼鬼城。

2004年8月——阴山古楼。

2004年A月——铁三角大闹新月饭店。

2004年B月——邛笼石影。

2004年C月——吴邪胖子深入张家古楼,救出闷油瓶。

2004年D月——吴邪发现三叔家的地下室,之后收到一封信。

2005年立秋——闷油瓶千里赴杭与吴邪道别,再次前往长白山。

2015年——十年之后……

{\fzqiti\hfill (《盗墓笔记大事年表》完)}
