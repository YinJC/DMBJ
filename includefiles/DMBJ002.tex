\part{引子}
\chapter{盗墓笔记}

八月的杭州气候宜人,虽然近几年来,夏天的温度越来越高,但是在西湖边上,你还是能感到当年“水光潋滟晴方好”的意境。

我靠在铺子的躺椅上,翻阅这几个月来我整理的东西。从格尔木回来已经有三个多月了,我似乎一直没有缓过来,最后发生的事情实在超出了我的承受范围,我没有想象到事情会以这么一个事态收场。

这三个月,我始终无法走出当时的梦魇,我每天晚上都会做梦,梦到无数经历的画面。

可是,我真的能摆脱了吗?我真的很怀疑,我心中的郁结,并没有随着那些秘密的解开而少任何一点。

别人拼命想掩盖的,必然是你不希望看到的,所以,追寻别人的秘密必然要承担知道秘密的受过。

这是我最后领悟出来的话,可是,就连闷油瓶都无法逃脱那种宿命,我又能如何呢?又有多少人,可以把满腔的疑问在心里放上一辈子呢?

回来之后,我将这一年来的所有的事情,全部写了下来,从我爷爷的笔记开始,一直到现在,一件一件的事情。但是我知道,我终究会有忘记的那一天,犹如三叔的面具,戴的太久,就摘不下来了。时间总是能改变一些东西,我现在只希望这一天能来的更早一些。

在整件事情中,还有很多我不了解的部分,比如说,我真正的三叔在哪里?闷油瓶的真正身份,小时的文锦到底去了哪里?终极到底是什么?那地下的巨大遗迹到底是谁修建的?文锦那批人到底是什么身份,他们到底在进行着怎样的计划?

这些东西仍旧是一个一个的谜团,本来最让我上心的是后者,不过放到现在看来,这些问题也并不怎么重要了。

闷油瓶回来之后,我们将他送去了北京大学第一医院,做了全身的检查。他的身体基本上没有问题,就是神智还不是很清醒,我们将他留在医院里,找了专人照顾。但这不是长久之计,我问过长沙的一些人,想了解闷油瓶的一些背景,让他们去帮我打听,可是到现在还没有任何一个人回复我。

胖子说他有办法,也许有回音,看样子,要了解闷油瓶背后的事情,远比我想的要难,现在也只有寄希望于他能够早日好转,提供一些有用的东西给我们。如果不能,那只能是由我们养他一辈子,对于他来说,也许倒不是一件坏事。

很少有人能有忘掉一切的机会,而幸运的忘掉的人,却又不顾一切地想记起来,这种轮回简直是一个任性的悖论。私底下说起来,我倒真不怕他永远记不起来,反而怕他记起了什么,却又不清楚。

潘子被送到医院,他能活下来简直是奇迹,我总感觉有些不可思议。他其实受伤并不重,很快就康复了。

长沙那边现在一片混乱,潘子告诉我,之前老伙计还在的时候,三爷就算不在,那边的局面也好控制,但是现在不行了,树倒猢狲散,到处是风言风语,他也不知道怎么办。好在三叔的产业被陈皮阿四斗得缩了不少,否则还要难处理,他只有走一步是一步,实在不行,那也只能拆伙,他这些年攒的钱早已经不愁吃不愁穿,现在也许是该退休的时候。

我告诉他让他快点找个姑娘成个家,三叔的产业也就别操心了,三叔年纪也大了,他又无儿无女,这事迟早会发生,积垢已久,靠我们是没法力挽狂澜的。

潘子没什么反应,三叔生死未卜,我想他永远也不会安心,可能还会一直的找下去,我只有祝他好运。

胖子分手的时候回了北京,他是最没感觉的人,回去照常开张做生意,按照潘子的说法,这人的城府非但不浅,而且还很深,不过我是实在看不出来他深在哪里。胖子临走说了一句套话:青山不改绿水长流,后会有期。说的挺有感觉,若不是这么多时间相处下来,又出生入死的人,很难体会到这种套话里的意思有多么婉转凄凉。

扎西在格尔木就和我们道别了,如果不是他,我们肯定走不出塔里木,所以当时我们想筹点钱给他,扎西说这件事情对于他来说是一次业,能把我们活生生地带出来,已经是菩萨保佑,他不能再要我们的钱,后来我把我的手表送给了他,留个纪念。

阿宁死了,裘德考的公司我暂时没了联系,发了几个E-MAIL给熟人,都被退了信,也不知道他们是否还要继续下去。无论如何,这一次的失败,那老鬼也应该死心了,如果还执着下去,那也只能自求多福。

尘归尘,土归土,所有人的生活好像都回到了正常的轨道上。那时我刚回到杭州,继续过我朝九晚五的小康生活,坐到那藤椅上,打一个小盹,一觉醒来,百无聊赖地翻开我爷爷的笔记,忽然就感觉时光倒流,恍如隔世。

庄周梦蝶,醒后不知道自己是在做化人之梦的蝴蝶,还是在做化蝶之梦的凡人,以前我听着玄乎,现在我一下子就明白了他的感触。只觉得这一年来的一切,好比梦幻,一闪而过,又感觉自己还在蛇沼之中,眼前的悠然,可能是自己临死前的臆想。

不管是哪个,我都想欣然接受了。有的时候,一件事情结束比得到这件事情的结果更加让人期待。

然而在我心底的最深处,我十分明白,这件事说结束还早的很。

\chapter{讨论}

两个星期后,闷油瓶出院,我去北京和他们碰头,顺便商量之后的事情。

回来之后我最棘手的事情,就是如何处理三叔留下的烂摊子。我这一年来的事情,虽然轰轰烈烈,却都是在暗中进行,家里人完全不知道我这边发生的巨变,三叔如今是真正的下落不明,可能永远不会出现,这边的事情如何解释是一回事。

另一面就是闷油瓶,如今他真的变成了拖油瓶,随着他意识的恢复,我必须面临如何和他重新认识的问题。

他没有亲人,在这世上干干净净,也不知道老窝在哪里,问了不少人,什么消息也没有,正如他自己所说,他几乎和这个世界没有一点联系。他的随身行李全部丢在盆地里,没钱没任何证件,这时候放任他不管,恐怕他只有去路边当流浪汉。

胖子混得相当不错,在琉璃厂也开了堂口。我们在他的新店里碰头,几个月不见,闷油瓶已经恢复了之前的气色,除了眉宇间对这个世界的陌生,其他倒是给我熟悉的感觉,这让我多少有点心宽。见到他的时候,他靠在窗口,也没有看我,眼神如镜,淡得比以前更甚,好比心思已经根本不存在于人世之间。

我先说了点客套话,他毫无反应,就问胖子他的情况如何,医生是怎么说的。

胖子摇头:“不就是那样,据说是回忆起一些片段来,医生说是受了强烈的刺激,得精神刺激才有可能好转,不然每天炖猪脑都没用。”

我叹了口气,也不知道在那陨石之内,在最后时刻到底发生了什么,能让他变成这个样子。

“你有什么打算没?”说了点无聊的,胖子就问我,“我这儿就四十多个平方,可实在局促,你要让他住在这里,我连相好都不敢找,别人一看我藏着个小白脸,还以为你胖爷我是兔儿爷。”

“你这人真没良心,人家可是不止一次救过咱的命,你担心这不靠谱的干啥?”我没好气道。

“他又不住你那儿你当然站着说话不喊疼,你要我出钱给小哥找个房子,那咱是一句话,他要住四合院我都给他拿下,和我住一起就不行,这和救命不救命没关系。”胖子道:“你看要不这样,我掏钱租房子,你掏钱找保姆,咱们把他安顿在这附近,给他好吃好喝,没事周末过去探望一下。”

“你这他娘的整得好象金屋藏娇一样。”我道,“他又不是傻子,你得问问他自己的想法。”

于是胖子便看向闷油瓶:“小哥,你自己说怎么着吧,今后有什么打算?”

闷油瓶闭了闭眼睛,似乎在思考,隔了很久才道:“我想到处去走走。”

我道:“走走?到哪儿去走走,有目的地吗?”

他淡然道:“不知道,到你们说的那些地方,长沙、杭州、山东,看看能不能记起什么东西来。”

我心里咯噔了一声,这是我最不愿意听到的——他想记起点什么东西来,现在他脑海里基本是一片空白,他的过去是一个巨大的谜题,但是谜题越大,对人的折磨就越小。然而如果他在游历过程中,记忆开始复苏,在他脑海里浮现出的情感片段对于空虚的人来说是诱惑力极大的,一点点的提示都会变成各种各样的线头,让他痛苦不堪。

我理解,对于失去记忆的人来说,人生的所有目的,应该是找回自己的过去。这一点无论如何也无法回避,但是我实在不想他再走上那条老路。

胖子看我脸色有变,知道我心里有个疙瘩,拍了拍我,提醒我道:“顺其自然,咱们不是说好的吗,你想把他硬按在这里也不现实。”

我叹了口气,如果这样,只有实行第二个方案了,就是和他一起琢磨这些事情,看着他,我们到底是过来人,很多东西可以避免他走极端。

他的想法我也想过,我曾经有计划带他到长沙,让其他人看看,不过现在长沙形势混乱,我都不知道去找谁好。这时候我忽然想到一件事,问胖子道:“你上次不是说你有办法能知道这小哥的背景,怎么后来就没消息了?”

“别提了,这事儿说起来就恶心。”胖子道:“你胖爷我当时计划是找那些夹喇嘛的人问问,他们当中间人的消息广,这小哥竟然能被你三叔联系到,肯定曾留一些信息在夹喇嘛的地方,咱们可以通过这个下手。”

我一听心说这是好办法啊,怎么就恶心了?胖子继续道:“没想到这些人个个都摇头,说什么不能讲。你说这批人平日里干的就是拉皮条的勾当,这时候给我充什么圣人君子。”

我哦了一声,是这么回事,行有行规,这倒不能怪他们。他们这些人可能就指望着这些信息吃饭,一旦透露出来,恐怕不止混不下去,还有可能被做掉。

“这些人口硬得不得了,这条路也是死路。”胖子道:“你那边怎么样?”

我叹了口气,说要是我三叔在,也许还能打听点什么出来,现在我接触的人资历不够啊,那些老瓢把子品性古怪,现在都盯着我这边的状况呢,我特地去接近他们,还不给他们吃了。那不是我这种人能干的事儿。

“那你就别琢磨了,我看还是按照小哥说的来,咱们给他报个旅行团,准备点钱,让小哥自己出去走走,”胖子道:“要不咱干脆替他征婚,把他包给一富婆,以小哥的姿色,估计咱还有得赚,以后就让他们自己过去,你看如何?”

这不是扯淡吗,我心说,摇头不语,琢磨起胖子刚才的说法,总觉得那是个好办法,胖子还没想到点子上。想着就想到一个人:“不对,你刚才找夹喇嘛的办法,也许还不是死路。”

“怎么说?”

“那些人不肯说,无非是怕得罪人,又或是不知道,怕说出来露短,但是有一个人,就没这个顾虑,也许咱们可以从这个人身上下手。”

“哪个人?”胖子问。闷油瓶也转过头来。

“去长白山的那次,替我三叔夹喇嘛的,是一个叫楚哥的人,你还记得吗?”

“你是说那个光头?”

我点头,楚哥楚光头,是三叔合作的地下钱庄老板,被陈皮阿四买通后,被雷子逮了,现在不知道在哪里坐牢。他联系了闷油瓶和胖子,肯定知道他们的信息,而且他现在身在牢房,也没什么顾虑,只是不知道怎么找到他,还有怎么让他开口,毕竟他说也没顾虑,但是不说也没顾虑。

胖子一击掌:“哎呀,还真是。”点头理解了我的想法,道,“这我倒没想到,不过,咱要是去找他,他把我们举报了怎么办?”

“这种人精明得很,他手里信息很多,他要是有心吐出来,长沙一片倒,他忍着没说就是因为知道不说才对自己有利。”我道,“他现在落难,求人的地方很多,我看套出话来不难。”说着心里已经知道应该怎么办了。其他事情不能麻烦潘子,这事倒是不敏感,可以托他去问问情况。

这就决定还是帮闷油瓶查吧,我们插手好过他到处乱跑。不过这事情我没法一个人干,我这边没事得要命,而且局势混乱,让闷油瓶跟着我到处跑肯定不行,他那种人我又制不住,万一他突然想起什么来,突然又溜了,我去哪儿撞墙都不知道,得拖胖子下水。

和胖子商量一下,胖子也只好同意,他道:“别的不说,最好是能找到小哥住的地方,那咱们可以省很多力气。”

于是就这么约定,我去托潘子办事,闷油瓶先和胖子住在一起,有眉目了,我们再一起商量后面的情况。反正以三个人的关系,这事情怎么样也脱不了身,不如当自己的事情做,算是还闷油瓶的人情。

常言道,好言难劝该死鬼,这一拍板,这是非就跟着来了。

我回杭州后给潘子打了电话,讲了来龙去脉,潘子也是讲义气的人,一口答应,他效率很高,三天后,我就接到他的电话。

我以为有了眉目,问他情况如何。

他叹了口气,对我道:“麻烦事,找是找到了,我问了他,你想知道的事情他确实知道,不过他不肯白说,有条件。”

“什么条件?”我问道。这是医疗中的事情,我在他这样的情况也会提条件。

“他要十万块钱,还要你去见他,他要亲自和你说。”

“见我?”我愣了一下,有点意外,心说:钱好说,见我干什么?听着感觉有点不妥当。

“该不是他想把我引出来,好戴罪立功?”我心寒道,耳朵边一下听到了铁锁链的声音。

“我也觉得有可能。”潘子啧了一声,“不过,他让我给你带了一样东西,他说你看了这东西,必然会去见他。”

“是什么东西?”我好奇道。

“是一张老照片。”潘子顿了顿,“很老的照片,是我那辈人年轻时候的那种黑白照片。”

我忽然起了一身鸡皮疙瘩。第一反应就想到了三叔西沙出海前的合影,那张狗屁的照片,误了我多少时间。心里琢磨,难道楚哥也知道这事的隐情吗?不过他现在用这件事情来谈条件,未免有些晚了。

想着问道:“上面拍的是什么?”

潘子嗯了半天,道:“我不敢肯定,感觉上,那拍的应该是一个‘鬼’。”

\chapter{第二张老照片}

照片通过E-MAIL发了过来,潘子对此一窍不通,我教了他半天,收到的时候,离我和他打电话,已经过了一个小时。

那真是一张很老的照片,发黄,上面有褪色的痕迹,即使如此,我还是能看到照片上的东西,也理解了为什么潘子不能肯定,以及“鬼”是什么意思。

那张老照片应该是在一间老宅中拍摄的,背景是一面屏风,照片发白得厉害,细节都看不清楚,却能够看到在屏风后面,直直站着一个人影。

光从屏风后透过来,人影相当的清楚,让人毛骨悚然的是人的姿势,平常人站立,总是会有一个重心的偏移,但是这个人影几乎是直立在那里。而且,整个人肩膀是塌的,一看就不正常。我第一感觉,这人是吊在半空的。

屏风后面吊着个死人?

我心里有点不舒服,但是想不出这照片哪里能引起我的兴趣。看了这照片就会去找他?没有这种感觉。

再往下看,地板是木头的,照片左边边缘是一个深景,是屏风后的走廊,一半被屏风遮了,一半能看到,那个地方已经皱了起来,粗看看不清楚,但是仔细看,我就看到走廊一边有几道门。

一下我就觉得这场景有点熟悉,这种古老陈旧的感觉,加上这样的房间排列,肯定在哪里看见过,而且印象还比较深。

我拖动鼠标,E-MAIL里还有照片背面的扫描,上面写着楚哥的手记,显然是写给我的。上面道:1984年,格尔木解放军疗养院。

我倒吸一口冷气,恍然大悟,啊,这是格尔木的那幢废弃的疗养院里拍的照片。我脑子里一下子闪出了当时的情形,这不知道是几楼的走廊。

那疗养院是文锦他们为了躲避三叔的追查而选择的藏身之地。文锦一行人背景诡秘,按照三叔的说法,他们不知道在进行什么研究。在这个废弃的疗养院里,他们拍摄了大量的录象带,监视着自己的一举一动,里面甚至还有个极度像我的人存在,这方面的事情完全是一团乱麻。

楚哥怎么会有那地方的照片,难道他也牵涉其中?

不像,我一想,他和三叔关系非常好,会不会是三叔有什么东西在他那里?或者托他办过什么事情……所以他知道一些内幕。

这确实很有可能,如果他真的知道在那疗养院发生过什么事情,对于我来说是一个意外之喜。不过话说回来,这张照片拍的是什么呢?

对于普通人而言,拍照必然会有主观的目的,要么就是留影纪念,要么就是保存资料,不可能毫无意义地就去拍一张照片。当时,在那个疗养院里,显然是有了什么契机,使得有一个人拍下了这张照片。

留影纪念我看是不太可能,屏风很普通,那简陋的走廊处于照片的边缘,肯定不是为了拍这些而照的。那么,这个人要拍的,必然是这屏风后的那个影子。

这是一件相当诡异的事情,一方面这个影子让人毛骨悚然,另一方面,这样的拍照方式,确实让人觉得,这可能是在拍“鬼”,因为这看上去有点像网络上的鬼照片了。而且我心里很清楚,这不可能是个鬼,一定是有什么东西在这屏风后头。而拍照的人,基于某种理由,隔着屏风拍了这张照片。只是我们不在当场,只看到了一个结果,所以觉得匪夷所思。

那幢疗养院实在隐藏了太多东西,他们把自己的一举一动拍了下来,现在又出现我这样的照片,到底他们在里面干了些什么?

想了想也没有办法顾虑这么多了,看来确实是有必要见一下这个人,于是给潘子打了电话,说明了我的想法。潘子想了想就答应了,说他来安排,安排妥当后再通知我。

书说繁简,很快,我在坪塘监狱就见到了楚哥,过程比我想的要顺利。潘子带我进去,这是我第一次进监狱,一路过来直冒冷汗,过了几道铁门,我在休息室里看到了他。

这家伙明显瘦了一圈,光头都不亮了,看上去老了好几岁,皱着眉头瑟瑟发抖,我递给他烟,他抽了几口才有点放松。想想当初见他油光满面的样子,我不由感慨,混这行的暴富暴穷,活成了这个样子也得认命。

见面局促了片刻,我也不知道和他说什么好,反倒是他先问我:“你三叔什么情况?”声音都沙哑了不少。

我草草说了一下长沙的情况,就道三叔音信全无,场面上看不到人,不知道跑到什么地方去了。

“报应,走这行就是这报应!”他狠狠地吸了一口烟,似乎有点走神,想了想抬眼盯着我看了看,又问道,“你在打听哑巴张的事情?”

“哑巴张?”我愣了一下,才反应过来,“你是说那小哥?你们叫他哑巴张?”

“道上人都这么叫他。”他此时已经把烟抽完了,速度极快,我看他手又抖了起来,把我的烟和打火机都递给他。他立即拿出来又点了一根。“因为他不喜欢说话,你打听他的事情干什么?”

我心说关你屁事,一下子不知道怎么回答,潘子就在一边道:“你他娘的问这么多干吗?”

楚哥抽了几口,瞄了潘子一眼,也是有恃无恐:“老子都这样了,问一声能怎么样?”

潘子本来见他就恨得慌,啧了一声想说狠话,我把他拦住了。楚哥现在算是最落魄的时候,说狠话没用,所谓已经没有任何东西可以失去了,你骂他几句又能如何?我道:“楚哥,你在江湖上混得比我长多了,知道有些事情我真不好说。”

“哟嗬,小三爷也和我玩场面话了,行啊。”他点头看着我,有点酸溜溜地说。

我倒是不吃他这一套,只是看着他,他哆嗦着似笑非笑了一会儿,发现我毫无反应,也有点无趣,忽然就对着潘子说:“潘爷,你钱付给我老爸了吧?”

潘子掏出一东西,那是一张收条,大概是潘子拿十万块替我付了,甩到楚哥面前。楚哥拿过来看了看,道:“果然是三爷的人,够爽快!”

“钱我们也付了,人你也见着了,现在你能说了吧?”潘子悻然道。

楚哥点头,就对他道:“那请潘爷你回避下,这是我和你们小三爷的事情。”

潘子皱着眉头就有点火,我忙给他打了个眼色,意思就是顺着他吧,他能有什么办法。潘子暗骂一声,起身出去。

楚哥看着他离开,直到门关上,才转头看着我。我发现他脸色变了,他猛吐一口烟,就对我道:“小三爷,你不能再继续查下去了。”

我吃惊地看着楚哥,没想到他会这么说。

“为什么?”我脱口而出。

他叹了口气:“你看看我,我的下场,你三叔的下场,哑巴张的下场,所有人的下场,你都看到了。”他站起来:“从这之后的东西太惊人了,不是我们这种人接触的。”

我坐直了一些,想起了那张照片,问他道:“你到底知道些什么?”

\chapter{同病相怜之人}

楚哥这样的说法,让我感觉他知道相当多的事情,不由让我紧张起来,于是出言催促,唯恐他和三叔一样,说到一半又不说了。

这一下不由就露了怯,楚哥看着我笑了笑道:“你别急,我会把我知道的都告诉你,不过你要先答应我几件事情。”

“是什么?”我问道。心说:该不是要临时加价?

他看了看门口,发着哆嗦道:“你不能对别人说,这些事情是我告诉你的,毕竟,能告诉你哑巴张的事情,我也能为了钱告诉你其他人的事情,搞不好有人听到这个消息,想不开找人把我做了。我也不是无期,还是要出去的,而且这里也没我想的那么安全。如果我不是走投无路了,我也不会卖这些消息。”

我点头,这我可以理解,所以他才让我来见他,还要把潘子支开,这种事情越少人知道越好。

“我和你三叔是多年的朋友,所以早年有很多的事,都是我去实施的,比如说,调查陈文锦。所以,我知道的事情,比你想象的多得多。”他哆嗦道,“你知道这后面的水有多深。你可能不知道,你三叔经常提你,所以我知道你的事情,你不是道上人,所以我才敢卖消息给你。”

哦,我心里一阵翻腾,这倒是可以解释为什么他会有那张照片。问他道:“这究竟是怎么一回事?”

我继续道:“我不知道你三叔有没有和你说过,那些人的事情?”

“你指那支考察队?”我道,脑海里响起了三叔的话:他们都不正常。“说过一些,但是不多。”

“你三叔这辈子,一直在调查那批人的行踪,我之前跟他混的时候,经常听他唠叨,但是越查,他就发现这批人越不正常。”楚哥又吸完一根烟,拿出一根来对上继续吸,“这些人,好像都是独立的,独立于这个世界,和这个社会一点联系也没有。他们来自哪里?是什么人?到底在考察什么?谁也不知道。”

“这些我知道。”

“但是我劝他放弃,他对我说,他绝对不相信,这个世界上会有这种人存在。那几年我们几乎用光了所有的办法,一直没有进展,最后你三叔还是听了我的,死心了。我以为这事情就这么完了,没想到一年前,你三叔、你、还有哑巴张那几个人去山东回来之后,你三叔忽然告诉我,那哑巴张也是那伙人之一,而且一直没老。惊讶之下,我们马上开始调查,目标自然是哑巴张。”

我坐了坐直,看到楚哥又点了一支烟,这不知道是第几支了。他还是深深地吸了一口。“哑巴张当时是四阿公的人,是你三叔从四阿公那里借来的,我就找人过去打听他的身世,结果听到了一些难以置信的事情。”

他顿了顿,“据说,四阿公第一次见到哑巴张的情形相当奇特,那事情发生在四年前,在广西的一次捕尸当中,你听说过捕尸吗?”

我点头,捕尸是旧社会的事情,一般发生在出现某种灾难的时候,有僵尸传说的地方比较盛行,打旱魃就是其中一种。这种时候往往会挖坟翻尸,也有真的闹尸变的时候,村民挑出胆子大的,用套索套粽子拖出古墓,在太阳下暴晒除害。

陈皮阿四的人和楚哥讲的捕尸却和这个不同,楚哥道,“这要从陈皮阿四在广西的生意说起。”

广西历来是一个各民族文化荟萃的地方,文物古迹众多,不过因为文化差异与中原太大,中原人那一套在广西完全没用,在广西活跃的一般都是淘家或者是古董倒家,都往村寨民间去收古董。因为广西和越南接壤,久而久之,有一些越南人就发现这个生财之道,这些人结伴越境到中国来盗掘一下古墓。广西有岭南文化,古墓众多,而且很多都是明葬,越南人不懂盗墓,乱挖乱掘,但还是能搞到一些东西的。

中原一代在长沙、陕西这些地方的生意其实已经很难做了,你说斗没有吧,确实还有,有很多油斗,盗了十几次,里面还有东西剩下,进去总不至于空手。但是有真东西,有龙脊背的真的太少了,要开一个新斗几家都蹲着抢货,这样的局面,肯定得求变,所以有很多瓢把子都在打外省的主意。有一段时间,黑龙江挖金国坟的也有不少,广西也是一条线。

陈皮阿四的盘子大,所以和广西的越南人也有联系,那一次派人去广西,就是因为听那边的说,有一批越南佬发现了大斗,不知道是什么来历,看上去规模相当大,要这边派人去“指导”,他们不知道哪些东西值钱哪些不值钱。

当时去了三人,他们跟着越南人进了雨林,第一次看到了越南人是怎么办事的。越南人是全副武装,估计这批人不仅干这一种买卖,还抬着一个筐子,问他们装的是什么,他们说里面是“阿坤”,陈皮阿四的人懂越南话,也不知道是什么意思。

在中越边境的林子里穿行了三天,他们才到达那个地方。古墓几乎是敞开的,他们用芭蕉叶盖住发现的入口,好像是一个地窖,就在他们要进入的时候,越南人拦住了他们,对他们做手势,意思大概是“小心”。

说着有一个越南人把筐子里的东西搬了出来,这时候他们才发现,筐子里装的竟然是一个浑身赤裸的男人。

那人的手脚被绑着,披头散发,浑身是泥,越南人就扛着他从入口吊了进去。

入口下面就是墓道,一路是向下的石阶,越南人都拔出了刀,陈皮阿四的人也准备起了黑驴蹄子,走着就发现这古墓规模极大,走了十分钟才到了墓室,下到底下就闻到了腐臭味。他们寻着臭味,发现墓室的中央有一个脸盆大的方井,味道就从下面传出来的。

这是一个两层墓,而且是岭南国的群葬墓,手电照下去,井下是相当矮的墓室,大概只有一点五米高,能看到排列的木棺侵在积水里,从底下弥散出浓烈的恶臭。

越南人直接把那个被绑住的男人推了下去,然后垂下绳套,用手电照着,似乎在等待什么猎物。

陈皮阿四的人一看就知道了,这古墓里肯定有问题,也许他们第一次进去已经死了人了,所以这一次,他们带了人进来。这个人可能相当于鱼饵,他们想要用活人把里面的什么东西引出来,然后放绳套下去套住吊起来。这确实是一种捕尸的做法。

听着这未免也太残忍了,盗窃文物无非是求财,弄得要夺人性命这事情就变质了,但是那边的事情,有历史原因,很难一概而论。陈皮阿四的人知道了越南人都是亡命徒,这种事情不能干涉,否则不知道他们会干出什么事来。

不过他们等了半天,一点动静也没有,越南人非常奇怪,在那里用越南话商量了一会儿,领头人就逼着一个越南人下去查看。

那人下去之后看了一圈,就招手,意思是没事了,另几个越南人也下去,开始往上面吊东西,陈皮阿四的人当时也大意了,没有跟着下去。结果没吊上来两件,突然下面就起了变故,听到有人惨叫,血都从井里溅了出来。

这些越南人相当彪悍,立即就有人往上逃,还真给逃上来两个,接着,一下就有一只指甲奇长的尸手从井下伸了出来,差点把领头的抓下去。他们吓得半死,没有办法,只好用石头把井口封了起来,垒了十几块大石头,然后仓皇而逃。

这个事情后来被陈皮阿四知道了,对于这种经验丰富的瓢把子,不可能因为里面有几只粽子就放弃这座古墓。于是陈皮阿四亲自带人回到广西,到达那座古墓的时候,已经是一个星期后了。他们搬开石头之后,就发现下面一片狼籍,满是残肢,恶臭四溢。

陈皮阿四以为人已经全部死光了,下去之后,却看到墓室的一边倒着十几只粽子,脖子全部被拧断了。一个浑身赤裸的人坐在粽子中间的棺材上,正面无表情地看着他。

楚哥道:“这个人,就是那个之前被越南人当鱼饵的‘阿坤’,也就是现在的哑巴张,当时就是他们第一次见面。”

我吸了口凉气:“这也太戏剧性了。”

“这里面肯定有夸张,这行里容易传神。”楚哥说着这件事,似乎也挺享受,可能是回到了坐牢前的时候,“据说,那帮越南人是在广西一个村子里发现哑巴张的,当时他神智不清,他们当他是傻子,把他绑去当饵。不过事情的大概应该就是这么回事,夸张的可能是粽子的数量之类。之后,他就成了四阿公的伙计,这事情在四阿公手下几个得力的人里面传的很广,不过对外他们什么都不说。”

“那这之前的事情?”

“没有人知道,哑巴张相当厉害,四阿公相当看重他,不过,我想四阿公恐怕也不知道他的来龙去脉,道上有规矩,这种事情也不会有人多问。”

我心说,陈皮阿四知道也没用啊,他自己现在在哪儿都不知道,我哪儿问他去。

“虽然这件事情只是一个传说,但是至少给了你三叔一个方向。”楚哥道。“不过,事情急转直下,你三叔着急去西沙,我就代他去了广西,拿着哑巴张的照片去那一带问消息。那他妈的根本不是人干的活,老子整整花了两个月时间,才在一个叫巴乃的小村,得到一些线索……”

那个村是山区,靠近中越边境,那里就有人认出了哑巴张,当地的名字就叫阿坤,并且带楚哥到了阿坤住的地方。

我啊了一声,实在没想到:“你是说他住在广西的农村里?”

“相当偏僻,但那个地方是陈皮阿四在广西的堂口,越南人很多,他应该就是住在那里,不过我不敢百分之百肯定。去长白山夹喇嘛,我是通过四阿公联系他的,他的大部分时间应该都在外面下地,看得出来屋子没怎么住人,也许,当年他离开广西就没回去过。”

“他那屋子是什么样子的?”我问道。我有点好奇,闷油瓶的家会是什么样子的。

“很普通,那是一幢高脚矮房,就和当地少数民族住的土房一样,里面就是床板和一张桌子,在那桌子上有玻璃,下面压着不少照片,我是偷偷进去的,因为那是四阿公的地盘,我也不敢放肆,没敢把东西带出来,就只是在里面翻找了一下,拿了其中一张照片出来——就是我给你的那张,准备等和你三叔商量了再决定怎么办。不过我没想到陈皮阿四老早就盯上我了,还没出巴乃,就被人给逮了个正着,之后的事情你也知道了。”他顿了顿,又道,“我自己的感觉,我在长沙打听哑巴张的时候,四阿公就已经注意到我了,他可能多少知道一些事情,所以我一到巴乃就被盯住了。我当时没有别的选择,只能和他一起来对付你三叔。”

我问道:“那你刚才说的,这个后面的大秘密是什么?”

楚哥看着我,又发抖起来:“这个我不能说……”

我最讨厌有人给我打哑迷,道:“什么不能说,你是不是嫌钱不够?”

楚哥哆嗦着:“小三爷,实不相瞒,你三叔在的时候,最忌讳的就是你寻根问底。现在他生死未卜,难保有一天他突然出现,这些事情你自己查到也就罢了,要是他知道这些事情是我告诉你的,我恐怕小命难保。你三叔做事也不是善男信女,我卖过他一次,但那算是情有可原,只是这件事如果再出卖他,在道义上也说不过去。你也说了,道上的事情有道上的讲究,你想知道这个,你到那房子里,看看那桌上玻璃下面压的其他照片,自然就会明白为什么我让你收手。我只能告诉你这些,具体的内容,绝对不能从我嘴巴里说出来。”

他还想点烟,但是烟已经没了,咳嗽一声,眼神茫然,竟然和闷油瓶的眼神有点相似。

\chapter{再次出发}

广西的山村,村里的哑巴,这他娘的越扯越没边了。不过那楚哥说的搞的我心痒难耐,闷油瓶的房间里他到底看到了什么,怎么问他都不说了,追问了多遍,他嘴硬的利害。我看他的样子,感觉有点异样和做作,十分的古怪,最后守卫都进来问是怎么回事,到这份上,再逼下去恐怕会出事,于是只好作罢。

潘子相当的郁闷,道,要不他找人教训他一顿,让他吐出来。我说不用做得这么绝,我看他的样子有点虚,有可能是自己也不知道。

“为什么?”潘子问。

“这叫做虚张声势,他可能只是知道那房间里有桌子,上面有照片,但是他并不知道照片里面确切是什么,虚张声势,这种卖消息放债的,都会这一套。”我道,“他当然是去过,才敢说的那么肯定。”

这只是我的推测,其实想这些都没有意义,无论如何,还是要亲自去一趟,到时候自然会知道他说的是不是太夸张。

从楚哥那里拿来巴乃的地址,去广西的计划就基本上确定了。

巴乃是一个瑶寨,处于广西十万大山山区的腹地,被人叫做广西的西伯利亚,早些年是一个相当贫苦的地方。看那个地址,恐怕还不是巴乃村里,可能还是村四周山里的地方。

陈皮阿四是老派人,可能喜欢选这种报了警都要两天才能赶到的地方做堂口,有什么不妙往山里一走就没关系了,不过这可苦了我们。

胖子和闷油瓶先到了杭州会合,胖子说也好,可以趁这个机会会会南蛮的堂口,也多点货源,这年头生意难做,他都断粮好久了。于是我们休息了几天,便由杭州出发,飞到南宁,然后转火车进上思。

这不是倒斗,什么东西都没带,我们一身轻松,一路上乱开玩笑,一个车厢睡了六个人,两个是外地打工回上思的,还有一个是导游,那导游教我们打大字牌,和麻将似的,好玩的紧。

靠近上思就全是山了,火车一个一个地过山洞,远处群山雾绕,导游说,那就是十万大山的腹地。

广西的山叫做十万大山,几百公里的山脉铺成一片,森林面积五百多万亩,其中心是几十万亩的原始丛林无人区,山峦叠嶂,森林苍郁,瀑布溪流,据说是一处洞天福地,是群仙聚会之所。不过这种地势也造成了交通的极度不便利,我们选择火车也是因为这个原因,平原地区的人,坐汽车进广西腹地,可能会吐成人干。

我看着那大山,心情非常异样。以往,看到这种情形,往往意味着我之后就要深入到这崇山峻岭之中,去寻找一些深埋在其中的秘密。然而这一次,我们的目的地只是山中的一个县城。

这种感觉很奇怪,不知道是失望还是庆幸,看着远处青色的花岗岩山峰和茂密的林海,我总觉得有点起鸡皮疙瘩。

到了上思,转去南平再进巴乃,坐一段车走一段路,正值盛夏,一路风光美得几乎让人融化,我和胖子看的满眼生花,连闷油瓶的眼睛里都有了神采。

这样在路上就耽误了比较长的时间,到了巴乃已经是临近傍晚,我之前问几个驴友拿过资料,知道瑶寨那里可以住宿,一路询问过去,问到一个叫阿贵的人那里,才算找到地方。

阿贵四十多岁,有两个女儿一个儿子,年纪都不大,有两间高脚的瑶族木楼,一座自己住,一座用来当旅馆,在当地算是个能人,很多游客都是他从外面带过来的。他看闷油瓶,我原以为他会认出来,没想到他一点反应都没有。胖子和他说了我们的来历,他出手阔绰,也没怎么讨价还价就住了下来。阿贵相当习惯我们这些人,颇有农家乐老板的派头,表示住在他这里,他什么都能帮我们搞定。

一路舟车劳顿,我也想不出来有什么需要他搞定的,只觉得肚子饿得慌,就对他说先把晚饭搞定吧。

阿贵就让他的两个女儿去做饭,他带我们安顿下来。我在木头地板上放下行李,用泉水擦了一把身子,坐在高脚木头的地板上,十分凉爽舒服,浑身都软了,再看着两个窈窕的瑶家女孩弄着饭菜,我忽然觉得这才是我想要的生活。

趁着饭没好的当口,闷油瓶就向阿贵询问楚哥给我们的那个地址是在什么地方,他有点急切。

阿贵说就在寨子里,不过在寨子的上头。胖子就让他别急:“虽说是你自己的房子,但是这么晚让别人带你去,你又没钥匙,很容易给人怀疑,咱们到了这里,有大把的时间,明天再去也无妨。”

我也赞同,闷油瓶点头,我相信这种耐心他是绝对有的。

晚饭是炖肉和甜酒,瑶寨人还有打猎,吃的据说是松鼠的肉,感觉很怪,但是甜酒相当OK,入口是甜的,而且当地水好,入口非常清冽。胖子喝多了,舌头大了,直劝阿贵说自己是大老板,他不想走了,让阿贵把两个女儿都许配给他,他会好好种地的。

我怕他乱说话得罪人,忙把东西扒完,帮他两个女儿收拾,让胖子自己一个人待着吹吹凉风清醒一下。

一边洗一边和两个小姑娘聊天,问瑶寨的情况。两个小姑娘告诉我,以前这里很穷,连饭也吃不饱,后来有人来旅游之后,情况才好起来,像他们阿爹带了人过来住家里,赚的钱就够吃喝了,他也不用上山打猎,可以买其他人打来的东西,这样他们一家就养活了好几家人。

我特地问了陈皮阿四的情况,又问她们是不是这里有越南人。

她们说越南人是有,不过不是在巴乃,还要往山里。这里现在来的人多了,她们也分不清楚是不是有长沙人在里头。

收拾完我甩着手,心说看来陈皮阿四还真小心,连村子都不敢待。

想来,他们可能是化装成观光客到巴乃,越南人直接走林子,他们在山里汇合交易。如此说来,这里交易的东西,恐怕比我们想的要多得多,至少陈皮阿四非常看重。这些关系,可能也是他以前在广西逃难的时候种下的人脉。

想着,走到饭堂里,准备问阿贵讨点水果吃,这时候看到一身酒气的胖子正盯着一边的墙上看。

我以为他喝多了,脑子入定了,没想到他看到我,就把我拉住了,对我道:“小吴,你过来。”

我走过去,问他干嘛?他用眼神给我打了个方向,我看到在吃饭的房间的木墙上,挂了一个相框,里面夹着很多的相片。他用下巴指着其中的一张相片,对我道:“你来看,这是谁?”

\chapter{继承}

那是一张有点发棕色的黑白照,和楚哥给我看的那一张相当的像,夹在很多的像片之中,不容易分辨。上面是两个人的合影,我吃惊的发现,其中一个人竟然是陈文锦!

这张照片比楚哥给我看的那张要大很多,所以看得相当清楚。照片里的另一个男人穿着瑶族的民间服饰,表情紧张,文锦则笑得很灿烂。除了这两个人之外,还有一个小孩子在背景处。

这是怎么一回事?文锦的照片怎么会出现在这里?我起了一身的鸡皮疙瘩,立即问阿贵:“这张照片是什么时候拍的?”

阿贵过来看了看:“几十年前。”他指着那个穿着民间服饰的男人,“这是我的阿爸,这个女的是考察队的人。”

“考察队?这里来过考察队?”我几乎跳起来,“这是怎么一回事?”

“我不清楚,好像是说那边的山里发现了什么。”阿贵指了指一个方向,“搞了好几年,后来忽然就没下文了。”

我心中暗叫,踏破铁鞋无觅处,得来全不费功夫,这一趟还真给我来值了!立即就拉阿贵坐下,让他马上和我讲讲这考察队的事情。

阿贵觉得莫名其妙,大概觉得这人怎么回事?怎么一听到这事这么兴奋?

胖子就道:“我们几个人就好这个,你别介意,您就说给我听听,我们给钱,给稿费,千字三十。”

阿贵一听有钱,立刻就来劲了,忙招手叫他女儿过来数着字,把事情和我们从头到尾说了一遍。

事情发生的时候,阿贵只有十几岁,当时巴乃非常的贫穷,几乎与世隔绝,所以考察队的出现,让他印象深刻。

他记得考察队有十几个人,由一个女人带队,是跟着外面赶集的人回寨子里的,因为他的阿爹当时是村子里的联络员,所以就去接待。

那个女人就告诉他的阿爹,他们是城市里来的考古队员,要在附近进行考古考察,希望他父亲能够配合。他们有政府的红章子文件,这在寨子里算是件大事,阿贵的父亲不敢怠慢,帮忙安排了住宿和向导。

考察队在这里就待了六、七个月,不过,这期间,大部分时间就在外头山里跑,寨子里的人基本上都没有和这支考察队接触。和他们关系最紧密的,就是阿贵父亲所安排的向导。

后来考察队的人走了,他们就问向导,这些人到底在山里干什么?向导也说不清楚。这几个月几乎走遍了附近的山,最后似乎才找到要找的地方,不继续在山里跑就不需要向导了,他就没随着队走。那女人只让他隔三天去报到一趟,还特别提醒他,不要早也不要晚。

后来,出了个听起来挺邪门的事情。

向导一开始都是三天去一次,没什么大问题,有一次他要帮亲戚打草,想着提早了一天去也没关系,结果去了,发现那支考古队的营地里一个人也没有,不知道到什么地方去了。他吓坏了,以为是遭了祸害,又不敢说,自己一个人去找,找遍了附近的山都没发现。

他胆战心惊的回村,一晚上没睡觉,第二天再去,却发现那些人又出现了,营地里热热闹闹,好像什么也没发生过一样。他当时就觉得不正常,以为是山神作怪,也没敢讲,等考古队走了,才说给村里人听。

考古队离开的时候,带走了十几箱东西,据说都是从那一带找到的。谁也不知道里面是什么。这张照片是临走的时候,那个女领队和他父亲照的合影,在城里冲印出来寄回来的。就因为这件事,他父亲后来成了村官,所以把这当成自己的光辉历史,挂到墙上。

阿贵说完,胖子已经按耐不住兴奋,又问阿贵:“是哪一年的事情,你记得么?”

阿贵用他的烟杆指了照片后面背景中的小孩:“这就是我,太小了,年份搞不清楚,当时没有书读,不过肯定有人会记得,你们要想知道的更详细,我明天去帮你们问问。”

我道了谢,心里翻腾起来,看样子这里的事情确实不那么简单,考察队在这里出现过,那闷油瓶住在这里,就不是什么偶然的事,背后肯定有渊源。虽然阿贵的资讯并不多,但是已经可以肯定,他们在山里,确实是进行了一系列的考古活动,这显然应该和他们的计划有关系。

我看向那山,又问阿贵道:“你是本地人,那山里,你们当地有没有什么说法?能有什么东西?”

“那一带叫羊脚山,我还真不知道那地方会有什么,其实我也挺好奇的。后来我也问过一些人,据一些老人说,那山沟里原先有个老寨子,不知道是什么时候的,后来皇帝打仗,起了山火,被烧了大半,烧死好多人,就荒废了,也许他们在研究那东西。”

阿贵道:“怎么?你们也感兴趣?”

“相当有兴趣,”胖子诚恳道。“那山有点远,路不好走,而且很奇怪,野兽很少,我们一般不去,不过那里有一道河谷,可以抓鱼,可这个季节下雨很多,会有危险,我建议你们还是不要去哪里玩。”

“你去过没有?”闷油瓶忽然问。

“我也没去过,我爷爷去过,说那山火非常厉害,地面上能看到的东西都没了,土里也许还剩点地基椿子,好多年的事了,”阿贵道,“你们想知道那考古队的事情,不如我明天带你们去找当时的那个导游问问,他一定知道的比我多,山里最好就别去了。”

闷油瓶并不理会,只道:“如果一定要去,应该怎么过去?”

“要顺着溪走,路很难走,你们要过去,我可以帮你们找个带路的,两百块,怎么样?不过明天去不了,起码得过两天,现在猎户都没回来。”

闷油瓶看了看我,我点了点头,无论如何也要去看看,两天的时间正好,我们可以先在寨子里好好打听一下闷油瓶的事情,然后再去山里,时间上不冲突。

阿贵就嘀咕了一声,道:“问题是,那地方什么都没有,就是林子,你们去了看不到什么。”

胖子对他道:“就是去踩踩也好。”

阿贵苦笑着摇头叹气:“那路可真难走,你们城里人也不知道是怎么回事,喜欢花钱买罪受。”

说着又突然想到了什么,问我们道,“对了,你们打听这些干什么?你们该不是盗卖文物的?”

胖子喝多了,一听骂道:“什么倒卖文物?说的那么土!告诉你,其实我们是倒……”

我赶紧戳了他一下,接着道:“是导游!有个团要进来,这里没地陪,我们先来打听一下,在找景点。”

阿贵一听很有兴趣:“那好,人带来我帮你们安排,这里好玩的地方多的是,那山里不好玩,你们自己去就算了,客人肯定不喜欢。”

我点头堆笑答应,心里暗骂胖子。

胖子也知道自己失言,不再啰嗦,自顾自去放尿。

我还想问阿贵一些详细的情况,不过他说真的不记得了,看得出他可能出去打工的时间比较长,对村子的过去也不是太了解,我只好作罢,等着明天找其他人打听。这事情就这么拍板了,接着我们坐在外面露天乘凉,继续商量细节。

胖子看阿贵离开,立即压低声音道:“他娘的那帮考古队神出鬼没,白天不见人,临走还带走那么多东西,明显这羊角山一带有一座古墓啊!这真是瞎猫碰上死耗子,咱们旅游来的,却得了这个消息,怎么样?两位,咱们是不是该顺应天意,顺手就把这斗给倒了?”

我对胖子道:“我就知道你肯定得提这个!那山里有古墓,现在只是你的推测,要到了那儿实地看才知道。而且那批人进的古墓,每一个都诡异异常,我是真不想进去。”

“这次肯定没事,你没看他们都安全出来了嘛!”胖子道,“而且还带了好几箱子明器,他娘的,这得值多少钱啊?”

“说起来也奇怪,听阿贵的说法,这批人显然没有采取考古队大揭顶的工作方式,看样子竟然也是打盗洞下去的,真是少见。”我道。如果不是确定这批人的政府背景,我绝对会以为他们是伪装成考古队员的职业盗墓者。

“这就是你孤陋寡闻,在条件不成熟的时候,考古队也会使用盗洞抢救一些文物,我看,可能这古墓的规模相当大,以当时上思的条件,没法进行挖掘。”胖子道,口水都下来了,“那小阿妹不是说,越南人还在山里,我想他们恐怕也是听过这件事,在找这古墓。我们就算不为钱,也不能把这便宜让给那批连洛阳铲都不会用的越南佬。”

我叹了口气,心说我是真的不想再下地了,你再怎么说我都不会听的,不过,如果那里真有古墓,那么必然和考古队在追查的东西有关,不进去似乎又不甘心。

这有点难办。

胖子继续在我耳边唠叨,我就行缓兵之计,让他别激动,我们两天后去实地看了再说,就是真有古墓,那地方这么大,也不一定能找得到。不过如果真找到了,他要下去,我们也会帮手,他这才肯罢休。但是他已经无法按捺了,阿贵一回来,就立即拉着问东问西。

我本来怕他露馅,但是心里很乱,也就没心情管这些,让他去了。自己靠到柱上,一边学闷油瓶看月亮,一边琢磨怎么办。

晚上有点湿热,我们扇着扇子,吹着山里刮来的带着树木清新的凉风,很快酒劲上了头,我有点晕乎,胖子在和阿贵聊什么,有点听不清楚,脑子也转不起来,只觉得这里看天上的星星,好像回到了小时候在乡下的感觉,十分的自然美满。

恍惚间,忽然注意到,另一边,阿贵自家木楼的窗户里,似乎有一个人正看着我们这里。

那边没有开灯,只能看到有一个模糊的古怪影子,我揉了揉眼睛,发现那影子肩膀完全是塌的,就像楚哥给我的照片上,那屏风后的影子一样。

\chapter{影子传说}

夏天的山风吹过挂在房前的灯,灯泡和四周大量的虫子一起晃动,光影斑驳,我以开始以为自己看错了,但是风过后,那影子还是在哪里。

我看这,刚开始几眼还没有什么感觉,后来越看,背就凉了起来,难道阿贵家里有人上吊了?

于是强忍住恍惚的感觉坐了起来,揉了揉眼睛仔细去看。

再一看,那影子却消失了,窗子后面一片漆黑,什么也没有。

是错觉?我用力皱了皱眉头,就问阿贵:那个房间后面住着什么人?

阿贵看了看道:“是我的儿子。”

哦,我脑子里闪了一下,但是什么也没闪起来,只觉得又晕起来,心说那肯定是他儿子在看这边,我喝多了,看的东西不正常起来。

天色也晚了,阿贵看了看自己的房子,就说要回去休息。

胖子付了千字三十的消息费,我们和他打了招呼,也进了屋子,进屋子胖子就郁闷:“我靠,就这么一两句话的事,这龟儿子竟然能讲掉我三百块钱,劳动人民的智慧真是无穷的。”

我说谁叫你充大款,在穷乡僻壤露富是最没流儿的行为,你他娘还后悔,没流儿中的没流儿。

胖子嘀咕了几句,说我假道学,伪君子,我也没精神理他。普通人进广西晚上没那么容易睡着,我们前几晚就睡的不踏实,不过今天晚上喝了酒,人相当迷糊,很快就睡着了。

这一觉相安无事,一直睡到了第二天十一点多才起床。吃了阿贵给我们做的中饭,我们就跟着他女儿往楚哥给我们的地址走,走了不到十分钟就到了。

那是一栋很老的高脚木楼,黑瓦黄泥墙,只一层,比起其他的木楼看上去小一点——说起来这里的房子好像都是这个样子的——看上去似乎没有住人,混在寨子的其他房子里,十分的不起眼。

阿贵的女儿很奇怪我们到这里干什么,我们假装拍照,胖子给了她点钱把她支开,看四周没什么人,我们就尝试着爬进去。

木楼建在山坡上,后面贴着山,窗户全破了,门锁得很牢,上面贴着褪了色的门神画,推了两把连门缝也推不出来。

对这木楼有印象吗?我问闷油瓶。

他摸着这些木头的柱子和门,摇头,我叹了口气,这时候胖子已经把一边的窗户翘了开来,对我们招手:“快,这里可以进去。”

“这么熟练,你他娘的以前是不是也干过?”我骂道。

“你胖爷我是什么人物,触类旁通你懂不?盗墓和盗窃就一个字的区别。”胖子一边说,一边催我们。

我们一人望风,偷偷从窗里爬进去,然后把窗关好。进去之后我的心竟然狂跳,感觉极端的刺激,连裤子被钩住了,差点就光腚,心说这偷活人就比偷死人心理压力大多了。

木楼里面有点暗,不过结构很简单,我先是看到了一个像阿贵一样的吃饭的大房间,和灶台连在一起,墙上挂着很多工具,都锈了。

“小哥,真看不出来你原来是个种地的。”胖子拿起一边的锄头道:“锄禾日当午,我是锄禾,你是当午。”

我们没理他,看到一边有木墙隔着,木墙后应该就是楚哥说的他找到的房间。这种木楼只有一间房间,肯定没错。

没有门,只有一块相当旧的帘子,上面的灰尘都起了花,闷油瓶皱着眉头,看了一圈四周,似乎有点犹豫,不过只过了几秒,他就撩起了帘子走了进去。我也有点紧张,这个似乎漂浮在虚空中的人,终于找到了一个自己的落脚点,却一点也不记得,也不知道老天爷是不是在玩他,不过没时间细想,胖子就把我推了进去。

一进房间,就是一股霉味,里面非常暗,什么也看不清楚,勉强看着胖子想去开窗,却发现这房间竟然没窗。

没想到会有这种情况,没人带手电,我们只能把帘子打了一节,让外面的光照进来。在暗淡的光下,可以看到房间很局促,一圈架子靠墙放着,我们想事看到了一些书和一些盒子,架子上空空荡荡,地上散落着泥巴,除了这些东西,就剩下一边的一张板床和一张木头桌子。桌子是老旧的学生课桌。所有的东西上都有一层薄尘。

这山中的空气非常干净,所以灰积的不多,如果是在大城市里,恐怕这里的灰可以铲去种地了。这也说明这里确定很久没有人进来过了。

“这就是你的房间?”我有点吃惊,看着这个房间,感觉有点太普通了,这就是闷油瓶住的地方?像他这种人,房间不是应该更加古怪一点吗?

但是一想,似乎具体的古怪法我也想不出来,他到底也是一个人,人总是睡床,总不会是睡棺材。线索也不能写在墙壁上,应该是在这些摆设里。

我们走进去,胖子走近那些柜子,发现基本上没有什么东西,自言自语道:“看不出你还是一个非常穷苦的种地的。”

房间里的东西虽然不多,但是看上去相当乱,那些盒子和书放的并不整齐,可能是楚哥来的时候被翻过了。我随手拿起一本书,发现书潮的厉害,是一本老版本的线装书,我翻了翻,里面都有点发霉了。心中奇怪,怎么会有这种书?

唯一看上去像点样子的,就是床和桌子,我想到这个,就立即朝那只写字桌走去,去找楚哥说的那些照片。

走到桌子旁边,我就看到了桌子上蒙着灰尘的玻璃,下面依稀能看到很多的照片,看样子楚哥没有骗我。

\chapter{照片的谜团}

这时候胖子捏了我一下,让我看闷油瓶。

我转头去看,看到闷油瓶还是一言不发,小心翼翼地摸着那些书,但看他的神情,似乎是有点什么疑惑。

“是不是想起什么来了?”我心中一动,问他道。

他没再理我,只是张了张嘴巴,欲言又止,眉头皱得更紧了。

我心道:难道有门?不敢出声打扰他,就在后面静静地看着。只见他侧着头,在房间里转了一圈,忽然道:“好像不对。”

“什么不对?”胖子奇怪。

他捏住自己的眉心,似乎在用自己所有的精力去回忆:“不对,这个房间,给我的感觉就是不对。”

“难道这不是你的房间?”

他摇头,忽然,他的目光集中向了那张床。他立即蹲了下去,去看床下。

我也趴了下去,床下一片漆黑,闷油瓶回头,胖子非常识相地马上把打火机递给他。他打起来,往床下伸去。

下面什么都没有,只有很多的蜘蛛网。但是他不死心,还是往里面爬,并开始在木头地板的缝隙中模,摸着摸着,忽然见他手指一钩,竟然抓住了一块地板,将它掰了起来。闷油瓶的力气惊人,就听到一声恐怖的断裂声,整条的木地板被他掰下来一块。他把掰下来的部分一扔,继续去掰,动作之大简直是疯狂了。

我和胖子都累了,一时间不知道要干嘛,胖子叫道:“小哥,就算不对,你也不用拆房子啊。”

但是没用,我们反应过来的当口,闷油瓶已经在床下的地板上掰出一个大洞,这时候我才忽然意识到什么,只见他把手伸到这个洞里,竟然从里面拉出一个黑色的铁皮箱来,用力往外拖。

他娘的,原来是这样!我兴奋起来,忙也爬了过去,就见木地板下面,竟然有一隔层,显然是精心设计的暗格。

看来找到关键了,我心说,立即帮闷油瓶拉住这只箱子,用力地拉出来。这箱子沉得要命,就这么拉出来,我已经一身是汗。胖子帮着我们把箱子抬起来,放在床上。

“我靠,这是什么?”胖子道,“这么沉,难道是小哥的私房钱?”

“怎么可能?”我说,吹掉上面的灰,仔细去打量。

这是一只黑色的铁皮箱,相当大,1×0.5宽,看上去能放进去一个人,上面布满了已经生锈的花纹,似乎年代相当久远。“看上去像以前地主人家的东西,可能还是个古董。”我看了上面老式扭锁,这箱子可能是民国时候的东西了,很有可能是大户人家用来放衣服的,或者是戏院放戏服的箱子。

闷油瓶喘着气爬了出来,我们看向他:“这是怎么回事?”

他没回答,眼神一片迷茫,自己也有点迷惑。

看来他只是想起一些片段,不过他能想起来这件事,说明这箱子是他自己藏起来的,看来里面有相当重要的东西。可能就有他背景的线索。我们都很振奋。我对胖子道:“快打开看看。”

胖子立即去拧那箱锁,没想到还没动手,闷油瓶一手按住箱面,叫道:“千万不要打开!”

\chapter{档案}

我们给他吓了一跳,只见他脸色苍白,似乎非常的紧张。

“怎么了?”我问道。

他皱着眉头,看着这个箱子,好久才道:“不要打开,我的感觉……很不好。”

“你想起来什么了?你想起来不能打开这个箱子?”

闷油瓶点头:“我不知道,只是有非常不好的感觉,开这个箱子,肯定要出事。”看着他的脸色,我发现他冷汗都下来了,不由自己后背也冒了冷汗,他都能紧张到这种地步,这箱子里到底是什么东西,难道是个炸弹?立即就让胖子把拧锁的手收了回来。

胖子道:“我靠,小哥你也别吓我。你到底记起什么了?”

闷油瓶捏住自己的额头,有点痛苦:“我没法形容这种感觉。”

胖子就啧了一声:“难不成这箱子,不是普通的开法,里面有机关?咱们这么一开,可能会射出毒针,或者会流出毒液?”

我一想很有可能,闷油瓶对机关了解相当深,这铁皮箱子是他的东西,似乎又放了相当重要的东西,很可能是设了机关,不知道窍门,开启会有很大的危险。

这一下可麻烦了,我是心痒难耐,但是在这种情况下,我又不可能咬牙说拼死开一下看看,这时候我有个念头,要是刚才胖子手快点可能就没这种麻烦事了,但是一想,刚才如果胖子手快点,可能我们这一辈子就都没麻烦事了。

我让胖子小心翼翼地帮忙把这铁皮箱子放到桌子上,仔细去看它的锁,这种老式的扭锁其实不是一种锁,而是一种普通的搭扣,只要轻轻一拨就可以打开,以我们的水平,怎么看也看不出这扭锁后面会不会有问题。

“那怎么办?”胖子也郁闷。

“看来只有先把这个东西带回去,找几个高手看看,然后在这里的其他地方找找,有没有什么值得注意的地方。”我道。看着四周,现在也只有这么个办法。

胖子敲了敲铁皮:“我靠,那得什么时候才能把这东西打开,说不定得半年。要么咱们干脆点,找阿贵去要把刀来,从铁皮上撬进去。”

我还没摇头,闷油瓶已经摇头了,他道:“不对,应该不是机关的问题。”说着他用他奇长的手指,按住那扭锁,稍微拨动了一下,没有机括的感觉,锁没有问题。

不是机关,那为什么不能打开?

闷油瓶摇头。我沉思道:“难道是这箱子里面的东西有问题?”

“这能有什么?难不成里面是条毒蛇?关了这么多年,早就成蛇干了。”胖子有点补耐烦了,道:“要不这样,你们全部退下,胖爷我来,老子命硬,我就不相信我能被一箱子干掉。”

“万万不可,不说是活物,里面可能有什么剧毒的东西,你一打开,不仅连累了我们,可能整个寨子里的人都会受你牵连。”我道。

胖子骂了一声,就一下子坐在床上:“这也不行那也不行,那送炼钢厂溶了吧。咱们都假装没这回事。”

我感觉这气氛有点搞笑,又有点诡异,我们从大老远赶到这里,确实是找到了闷油瓶的房子,也找到了重要的线索,但是因为闷油瓶一个似有似无的感觉,我们连放着线索的箱子都不敢打开,这确实郁闷。但是,在这种环节上冒险,确实也是不值得的。

我拍了拍胖子让他稍安勿躁,不如再敲敲地板,看看这下面是否还有夹层。看闷油瓶掰断地板的方式,这夹层做的时候使用了整条木板钉死,说明短时间内他不准备取出这个箱子,这种隐藏夹层的做法工程浩大,可能不止一个。

于是我们开始东敲敲,西弄弄,不过这房子是架空的,怎么敲我们都觉得这木板下面有东西。

高脚木楼的地板不是工业铺装,只是用长木条简易搭起来的,木板之间的缝隙很大,胖子就趴在地上,用眼睛往下面瞧。下面一半是用来养鸡的地方,能看到泥地。

胖子还真是不怕脏,一点一点看过来,搞的浑身是泥,但毫无收获,似乎暗格只有那么一个。

我们反复找了三遍,里外每一块地方都查过了,确定无疑,胖子就拍着衣服道:“行了,该找的找不到,该开的开不了,咱们收拾收拾东西先撤吧,免得阿贵他们起疑心。给一破房子拍照不可能拍这么久。”

我一想也是,就去搬那箱子,胖子就阻止道:“这东西不能见光,现在搬出去,阿贵见我们空手出来,搬这么大一东西回去,恐怕不好解释。如果事情传出去,可能会传到陈皮阿四的耳朵里。我看,我们还是把箱子放回原处,临走的时候再找个晚上搬出来。”

胖子想的周到,我点头,于是胖子爬到床下,把箱子再次推进那个洞里,然后把那些木板草草盖上去,把那洞掩上。

接着我收拾了照片文件放进包里,准备回去好好查看,正收拾着,忽然又听见敲地板的声音。

我就对胖子道:“别敲了,你不是说要走了吗?”

胖子在一边抽烟,举了举双手,表示自己没敲,我再以看闷油瓶,他正在将那些盒子和书一样一样放整齐,显然也听到了敲地板的声音,看向了我们。

咦?我愣了一下,那是谁在敲地板?

我们凝神静气,仔细去听,就发现那声音来自于床下,“笃笃笃”,很轻微,但是很急促。

胖子和我对视一下,掐掉自己的烟头,小心翼翼地弯下腰去看床底下,我也蹲了下去。

床下肯定没人,这不用说,我们贴近地板,发现感觉不到地板在震动,这个声音不是敲地板,而且听起来,有点遥远,感觉不出具体是在床下的哪个角落。

胖子做了个手势,意思是:在地板下面!

我点头,心说:难道有老鼠或者鸡跑到这高脚木楼的下面去了?忽然我就看到,盖着那铁箱的木板碎皮,竟然动了一下。

嗯?这他娘的怪了,我目瞪口呆,难道是那只铁皮箱子在动?

\chapter{老鼠}

我脑子的第一反应,就是有老鼠。

这种山村里,老鼠是相当常见的,废弃的木屋,简直是老鼠的天堂。但是,刚才翻动物品的时候,并没有发现老鼠的痕迹,所以感觉有些意外。可能是被敲地板给惊吓到,爬出来的。我们到处乱敲,唯独没有敲床下,所以就躲这里来了。

这样的情况我没有想到,倒不怕那铁皮箱被咬坏,不过如果老鼠乱啃,拨开扭锁,可能会发生危险。

我有点担心,立即朝那暗格爬去,一边用力拍了两下地板,想让老鼠逃跑。

果然,我一拍地板,那边好像受了惊吓一样,一下动静大了起来,但就是不见老鼠从木板下跑出来。这种和人类生活在一起的动物都精的厉害,会自己判断形势,看样子可能认为躲在里面比跳出来逃跑要稳妥。

我不喜欢老鼠,特别是这里的老鼠应该是山鼠,是比较凶猛的一种,可能会主动咬人,一下子也不敢贸然掰开那些木板,就等胖子过来处理。

胖子完全不在乎,刚才憋着一股闷气,这下正好发泄,嘀咕了一句:“太岁头上动土,也不打听你爷爷我是属什么的。”一边让我调整位置,挡住那老鼠可能逃跑的方向,自己小心翼翼地拨开一块木板,弓起身子,单手做鹰爪样。

我和他对了一眼,表示做好准备,胖子深吸一口气后发难,猛地拨开木板,抓了下去,连抓两下,激动中脑袋往后仰,一下撞上床板,疼得他马上缩了起来。但是他相当敬业,叫疼前还先叫我快抓!

那暗格里一阵扑腾,我怕老鼠惊了之后,真的会碰掉扭锁,也顾不得这么多了,伸手下去一阵乱摸,就想把它逼出来。没想到一抓,突然抓住一条碗口粗细的东西。那东西立即挣扎,顿时我脑子就嗡了一下。靠!难道不是老鼠,是蛇?

这下可给胖子害死了,这可是广西,中国毒蛇最多的地方!刚想放手,胖子就冲过来帮我,一下子握住我的手,道:“抓住了,别放手!”

我脸都绿了,就这样让它握住我的手,硬生生把那东西给拉了上来。一边道:“他娘的也算有收获了,等一下给阿贵炖——我操!这是什么东西?”

胖子一下放了手,我看到,从那暗格里拉出来的,竟然是一只灰色的人手!

我惨叫一声,立刻把那手甩掉,心说怎么回事?那手猛地缩回道暗格里,抓住铁皮箱子就开始扯动,动作极大,扯了两下扯不出来,接着就去扳四周的木板。

我和胖子都看愣了,好久胖子才反应过来,大叫:“我靠!釜底抽薪!贼啊!”

我也反应了过来,有人在地板下面,想偷这只箱子。胖子立即就怒了,大骂一声,一下抱住那铁箱子,从暗格里拖出来。此时看见暗格一边的木板已被扳断,那手就是从此洞里伸进来的,只不过洞口太小,箱子拉不出去。

那手一发现箱子被抱走,马上就往洞口缩去。胖子哪肯?赶上去抓,一下抓住那手腕,叫我帮忙,可我还没伸手下去,那手已挣脱,消失在那洞里,接着就听到地板下一阵撞击声,那人显然狂爬而去。

胖子忙爬出来,对闷油瓶大叫:“小哥,去外面截住他!”

抬头一看,闷油瓶早就破窗而出。胖子来劲了,跟着对我道:“小吴,你看着这箱子!”说着抖起肥肉也冲了出去,边跑边大叫:“小哥,左右包抄!”

我拉着箱子从床下出来,只感觉心简直要跳出来,这他娘的是怎么回事?那狗日的到底是谁的手?怎么会这么恐怖?我靠!真他娘的吓死我了!

喘了半天,不知道是这里湿热的气候还是什么,还是没喘明白,就拉着箱子靠到一边,听到外面传来胖子的大叫:“他娘的,怎么人呢?遁地了?”声音越来越远,显然是跑开了。

我想深深呼吸几口,去帮他们,突然听到床下又发出木板断裂声,我愣了一下,哎呀一声,意识到不妙。我靠!难道他没走?调虎离山?

忙低头往床下看,只见从那暗格中钻出一个人,正朝我爬过来。

\chapter{面人}

我的第一反应就是快跑,抱起那箱子,就想跑出去。但是箱子实在太沉了,我一个人根本没办法抬动,硬是推着挪了几步,手忙脚乱加紧张,箱子却不知道为什么,卡在地板上动不了。

回头看,那人已经从床下爬了出来,浑身是泥,简直好像从泥沼中爬出的文锦。

我突然反应过来,这又不是粽子,是人啊!我这么害怕干什么?想起胖子刚才玩的锄头,立即跑出去,拿上就冲回去。

回去一看,那人已经抱起了铁皮箱,跌跌撞撞朝我冲过来。我抡起锄头便打,他一猫腰一个翻身躲过去,接着用肘部用力一顶我的后背。我一阵剧痛,差点扑到在地。他头也不回一下就冲出了门去。

我虽然不常打架,但内心里也是一个相当固执的人,有着土夫子的血统,当即火冒三丈,抄起锄头追了出去。

一出门,感觉眼前一亮,胖子正在一边蹲着,往高脚木楼下面看。那人力气极大,抱着铁箱跌跌撞撞就朝他身后跑了过去。

我对胖子大叫:“拦住他!”

胖子还不知道怎么回事,回头看我,我再吼道:“那箱子给抢走了!”

胖子也算反应快,这么一瞬间就反应了过来,立刻拉了一下,正好拉住那人的衣服。

箱子太重,那人一下子失去了平衡,摔倒在地,箱子被摔了出去。他爬起来去抢,胖子不是我,哪有那么容易就让他得逞?又一个泰山压顶,将他再次滚倒。

我此时已经冲到箱子边上,一把就抱住。

这是一个很严重的错误,这时候,我首先应该帮助胖子将这个人制服才对,因为抓住了那人,箱子自然就没危险了。可是形势太急,我没有想明白。结果胖子没有把他压住,他一看抢箱子再没指望,连滚带爬的站起来就跑。

胖子吼了一声“别走”,立马追过去,我随即跟上,却发现那人跑得极快,冲进村子,很快就跑得没影了。寨子里房屋纵横交错,都由青石小道相连,不是本地人很容易迷路,根本不知道他是往哪里跑的。

胖子喘气,奇怪这人怎么从楼里跑了出来,就问我是怎么回事?我把事情一说,他大骂一声,后悔莫及。

看着那人消失的方向,我只觉得莫名其妙,这到底是什么人?为什么会突然出现,来抢这只铁皮箱子?我们现在应该没什么对手了,来这里也没多少人知道啊!难道是普通的毛贼?不过,这毛贼的手法也太新奇了。

胖子骂骂咧咧,这时闷油瓶赶了过来。他刚才给胖子支到另一边蹲点去了,如果有他在,我估计那家伙肯定逃不了。

走回屋子里,那铁皮箱子给摔在泥地里,沾了一大块泥。胖子道:“幸亏老天保佑,箱子没散开,否则还真不知道会出什么事。”

我道:“现在看来,这东西不能放回原处去了,我看还是带回阿贵家里,给他点钱,他自然知道怎么做。”

胖子点头称是,说:“虽说最危险的地方就是最安全的地方,不过还是放在自己身边实在。”二话没说就去搬箱子。可扣住箱缝,刚往上一提,突然就听到“哢”的一声,扭锁竟然和箱体断开。

箱子摔在地上,翻了开来,里面的东西一下滚了出来。

{\fzqiti\hfill (《引子》完)}

\part{阴山古楼 }
\chapter{起源}

为了帮助闷油瓶寻找失去的记忆,我们来到了十万大山的腹地,被称为广西的西伯利亚的巴乃。

我一直认为这种失去记忆、寻找记忆的情节不太可能会发生在现实中,所以最初还是感觉到有一丝异样。旁人的过去也许稀松平常,但是闷油瓶背后的故事,应该会有所不同,就像看一本悬疑小说,并且自己参与了进来,心中很有些忐忑和兴奋。

闷油瓶一如既往的沉默寡言,像他这种人的心中是否会有常人的纠结我不敢肯定,至少,他表现出来的这种耐心让我佩服。我也有过一些犹豫,帮他寻找过去,相当于把他从目前的平静中拉回现实,不知道到底是好事还是坏事。

进山的过程不再赘述,我们按照楚哥给我们的线索,找到了闷油瓶以前住的高脚楼,并且在破败的床下暗格中,发现了一只铁箱。之后发生了一连串事情,有人竟然想从高脚楼的楼板下把铁箱拽走,好在我们及时发现了,但是那人显然非常熟悉村子的环境,迅速逃入了村中小路,不见踪影。

就在我们莫名其妙,还没反应过来刚才发生了什么时,胖子抱着的古老铁箱子的搭扣竟然断了,箱子摔到地上一下子翻了开来。

事情发生得十分的快,三个人都没有反应过来,箱子已经在地上了,箱盖大开,一块拳头大小的东西从里面滚了出来,定格在胖子的脚下。

闷油瓶之前说过,说他对这箱子有一些模糊的记忆,说箱子里的东西可能十分危险,让我们绝对不要打开,所以箱子刚掉到地上,我下意识就抬手缩腰,做了个防御的动作。

胖子没有时间做更多的反应,也只是缩了一下脖子,我们两个人一下都定在那儿不敢动。

我原本以为会爆炸,当时也没有时间多考虑,一切都是条件反射,然而咬牙缩着脖子等了几秒,却什么都没发生。没有爆炸,也没有暗器飞过来。

我小心翼翼地睁开眼睛,看向胖子脚下,摔出来的东西好似一块木头,长满了疙瘩,我从来没有见过,但似乎不是什么危险物。胖子渐渐放松了下来,走远了几步,我也慢慢放下手,心生奇怪:难道是闷油瓶记错了?还是因为时间太久,以至于过了保质期没了危险性?

看向闷油瓶,他并没有什么特殊的表情,但是显然也吓了一跳。

这就好比是一只爆竹哑火,谁也不敢第一时间去看是怎么回事,我们僵了片刻,刚才还信誓旦旦说自己命硬的胖子才凑过去。我也跟过去,看到那东西形状有点像葫芦,大概有广口杯那么大,表面有一些脓包一样的疙瘩,好像癞蛤蟆的皮让人觉得很不舒服。仔细看后发现,这只癞皮“葫芦”的脓包里夹杂着金属锈迹的光泽,竟然像是铁的。

胖子想用手去拿,闷油瓶制止了,他从边上折下一片南瓜叶,包住“铁葫芦”拿了起来。

从他拿“葫芦”的手感来看,确实是铁的,而且重量还不轻。那些铁疙瘩像是被强酸腐蚀过或者铸的时候夹了大量的气泡,红色和黄色的脓斑是铁锈的痕迹,这东西就是一葫芦状的铁坨子,但能看到上面有一些古代的花纹,已经非常模糊了,隐约能感觉这是件古物。

胖子看着纳闷道:“什么玩意儿?跟炮弹似的,难道是古代的手榴弹?”

我立即摇头:“别瞎说,你把手榴弹埋床下面?”

明朝的火器已经非常发达,“震天雷”和“国姓瓶”的杀伤力很大,我经手过一些,但都是掏了馅儿的——也就是没火药——(谁也不能交易一个实心的,那等于交易军火)。这些火器最早都是福建渔民从海里网上来,然后被古董商用日用品换走,但这铁疙瘩不像海货,所以应该不是火器。更何况把这东西埋在床下,要是赶上天干物燥的时候爆炸了怎么办?闷油瓶绝对不会做那么缺心眼的事。

闷油瓶颠了颠,闻了闻,也摇头。我问他刚才危险的感觉是否还在?他没说话但是神情异样,看着那铁葫芦停顿了一会儿,道:“这东西只有一层皮是铁的,真正的东西被包在铁皮里了。”

我愣了一下:“何以见得?”

闷油瓶道:“重量太轻。”

胖子惊讶道:“你他娘的能掂量出来?”

这不奇怪,一般经手古董的人,这种手艺都是必练的,而且掂量过纯铁或者做过模具的人都会知道,一块铁的重量和普通人的预期是不同的,铅笔盒大小的铁块,力气一般的人用两个手指可夹不起来。

我对胖子道:“你们半路出家的基本功不行,像这种手头上的功夫,我们或多或少都要练几家子。”

胖子呸了一声:“胖爷我花这么多闲工夫练这个干吗,买只电子秤才多少钱。”

我做了个鄙夷的表情,接着问闷油瓶道:“什么东西要被包在铁皮里保存?你有没有什么想法或者印象?”

闷油瓶摇头,胖子就道:“以前有一种铁包金,运输的时候金块外面包上铁皮,不显眼,不过这东西的铁皮看样子是铸上去的,而且重量还轻了,里面肯定不是黄金。”

“铁包金”这我倒没听说过,我只知道有一种叫铁包金的藏獒,爷爷有过一只,因为水土不服一直养不起来,后来被村里的牛踢死了,胖子说的事不知道是胡吹的还是他真见过。

让我在意的是那上面模糊的花纹,既然有花纹那么这东西至少有装饰作用,不会是单纯的铸件。它肯定有确实的用途。

“会不会是什么铁器的部件?”胖子又道,“比如说铁香炉的脚,或者以前车轱辘上的装饰品?”

我心说也有可能,我对铁器的认识不深,铁器易生锈,在古墓中很难保存,所以市面上流传得远不如铜器和瓷器。铁器的价值一般也不高,所以大部分搞古董的人都不熟悉,我实在一点头绪也没有。

不过既然是古物,还藏在闷油瓶的床下,那么这东西肯定有点来历,应该和他在这个村子里经历的事有关。

我想起胖子昨天的想法,心里有一个推测,胖子说羊角山附近可能有一个古墓,那么事情的经过也许是这样:闷油瓶当年可能在文锦的考古队里,这“葫芦”可能是他们从那个古墓里带出来的东西。但是因为某种原因,小哥把这“葫芦”藏了起来,否则很难解释其来历。

胖子皱了皱肥眉:“我也推测是这样,那么当年小哥把东西藏起来,显然是在提防什么,当时的情况恐怕非常复杂。”

有提防必然有敌对,说明考古队在这里发生的事情,不会像阿贵说的那么单纯。

三人沉默了片刻,我感觉有点舒坦又有点郁闷,开心的是这里得到的信息比我想象的要多很多,郁闷的是这些信息都只能大概勾勒出“一个事件”的大体样子,没法触到细节。

文锦在这里出现,阿贵在照片上的年纪只有十七八岁的样子,现在阿贵肯定有四十出头了,那么就是二十多年前的事情。那时候正好是西沙事件发生前后,那么文锦在这里出现的时间应该是在西沙出事前没多久——他们离开这里之后才去的西沙——我没有看到照片上有其他人,文锦是跟着另外一支队伍还是和西沙考古队来的这里就不得而知了。

闷油瓶在这里被越南人绑了当阿昆,时间应该是五六年前,中间差了十五年,这十五年他在干什么?我感觉很有问题,以他的身手那几个越南人定然不是对手,就算对方有枪,我想要逃脱总不是问题,何至于被捆着当猪崽?难道他和陈皮阿四的见面是他设计好的?这些都是疑问。

“刚才抢咱们东西的人,会不会和这件事情也有关系?”胖子望着那人消失的方向问。

我想起这茬儿来,就问他们道:“你们刚才有没有看清楚他的脸?”

“干,那家伙跑得比兔子还快,别说脸了,连屁股都没看清楚,只看到这人蓬头垢面的,体形和你差不多,一溜烟就没影了。”

我心说这人是谁呢?我们到这里来基本上不会引人注目,这是一个单纯尾随我们的小偷,还是局内人?这点让我意外,有点被如影随形的感觉,如果他不是单纯的偷窃犯,那他必然和这件事情有关联,那么我们现在的处境就有点糟糕,晚上得关门睡觉了。

“等下咱们问问阿贵,那人像疯子一样,指不定他知道什么。”胖子道,“现在怎么办?咱们拿这个铁葫芦也没辙,要不等下找个铁匠看看能不能熔开一部分。”

我道不然,劳动人民的智慧是无穷的,这种东西我知道有一种处理方法,可以使用硫酸一点一点把铁壳子溶薄了。你看这些烂铁疙瘩,估计有人已经这么干过,不过由于某种原因没有成功就停止了。

说不定这么干的人就是闷油瓶。我有一个感觉,他对于这东西有危险的印象,可能正是他在溶解铁封时发现的,当时他可能忽然发现了什么危险的迹象,让他印象非常非常深刻,使得他立即停止了作业。现在他虽然什么都忘记了,但是那印象还留在脑海里,让他觉得不安。

当然这是一个完全的推测。即使我感觉很有这种可能。

胖子跃跃欲试道:“硫酸好办,我去化肥站要一点来。”

我心说那玩意儿还是不要轻易去动的好,对他说悠着点,等一下可以带到阿贵那里仔细琢磨琢磨,让闷油瓶仔细看看。

闷油瓶将铁葫芦放回到铁箱子里,翻上盖子,胖子立刻抱起来:“得,今天算是有收获了,这玩意儿现在我得贴身看着,你们赶快再进去翻翻,那闺女等下就回来了,抓紧时间。”

我想起楚哥和我说的照片还没看呢,心说那才是正事,就立即起身往窗户走去。

刚站起来还没走两步,闷油瓶忽然发现了什么,一下拉住了我。我看他的眼神,立即感觉有点不对,忙顺着他的视线一看,顿时一愣。我看到一边高脚楼上方的山坡上,站着几个村民,不知道什么时候出现的,正满脸阴霾地看着我们。

\chapter{古怪的村子}

闷油瓶拉住了我,我当时心里咯噔了一声,第一反应是:他们什么时候站在那儿的?

我们生活在城市中,习惯于平视一切,到了这里一般不会想到去注意山头,所以最早来的时候,这山坡上有没有人我一点印象也没有。如果他们一早就在上面了,那么我们爬进高脚楼肯定就被他们发现了,这就有点不妙了。

而且看他们几个的表情,似乎都很不善,有点冷目观望的感觉。好像以前黑白电影里,老百姓看汉奸的表情。

我有点不知所措,一时间也停下来和他们对视。这几个人都在四十岁到五十岁之间,山民生活艰辛普遍显老,所以实际年龄可能更小一点。有两个人挑着扁担,好像刚从山上收了什么东西下来。这几个人没有任何举动,只是直勾勾地看着我们。

我在长沙老家并不受欢迎,以前也经历过这种场面,知道这种表情,意味着他们对我们有很大的警戒心,但还拿不准我们是什么人。看来我们刚才的举动有可能都被看到了。

在山村里,绝对不能得罪当地人,否则后果不堪设想,轻则被赶出去,重则直接被扭送进派出所。长白山一行被楚哥出卖的事情让我们的案底都不干净,也不知道有没有被通缉,进了派出所他们一查网络,难保不会出更大的事。

这时候再爬进去就是找打了,胖子在我们后面打了几个“哔”的音,暗示我们快走,别和他们对着看,这有点挑衅的意思,当心把人家惹毛了人家冲下来。

本来做贼我的心里就有点阴影,这时候心跳更快了,一下紧张起来,感觉有一股压力从山上压下来。但我看了看那高脚楼,又觉得不能走,这唾手可得的东西却不能得到,好比看小说眼看谜底就要揭开,作者却又绕起圈子一样,太让人难受了。

一时半会儿我没有挪步,胖子就架住我,对我轻声道:“晚上再来,差不了这几个小时。”一边拖着我往后拉。

我们三个绷着身子,尽量自然地离开,走入村中,走了好一段距离才回头,看后面村民没有跟来,才松了口气。

这情景有点像小时候我和老痒去果园偷橘子,偷完出来正好碰上园主,两个人兜里全是橘子心里怕得要死,只好佯装路过。那种紧张感让你的脚都迈不开,现在当然没有小时候那么害怕,但是感觉也不好受,而且还有点好笑。

凭借着记忆,我们绕了几个弯路回到了阿贵家里,阿贵不在,他的大女儿在编簸箕,看到我们就问怎么这么快就回来了。我道太热了吃不消了。

胖子径直回到房里,将铁箱子藏到床下后,我们才安下心来,感觉这事情就过去了。胖子道:“吃一堑长一智,以后咱们白天别那么猴急,得先观察环境。同时,我看我们也得在阿贵那儿打点一下,他是地头蛇,咱们得拉他进伙,关键时候咱们好有个人替我们说话。”

我心说恐怕也没用,这浑水怕他也不肯蹚。而且,我猴急是有原因的,事情到这种程度,任何节外生枝都有可能产生蝴蝶效应,所以能急一些还是急一些好。

说完话胖子出去讨水喝,我惦记着那没有看到的照片,只觉得浑身燥热心神不宁,就躺下来逼自己静心。没多久听到胖子在问阿贵的女儿,那木楼后面的山路是通到哪儿去的,平时走的人多不多?阿贵女儿说那儿是山里的瓜田,夏天了,西瓜熟了,所以有人经常上山去摘西瓜。那老木楼老早就在了,以前听说有个老太婆住过。

我看了看闷油瓶心说老太婆?难道闷油瓶以前是和一老太婆同居的?他那空白的十五年搞不好是在那里被关着当性奴,那未免太悲惨了。接着又诧异自己不知道哪里来的龌龊念头,大概是一路过来听胖子的黄色笑话听多了。

不过阿贵女儿说的以前,时间跨度不明确,说不定是更早以前,也说不定是闷油瓶离开了之后。

之后,胖子问了阿贵女儿那个蓬头垢面男的事,一问之下还真有这么一个人。这疯子从她刚出生就在了,也不知道是谁,村里人都叫他“阿玉儿子”,好像以前也是个猎户,不知道怎么的就疯了。这人住在山上的一间破屋子里,有时候看到他下来捡一些剩饭吃,现在不怎么看得到了,可能老了走不太动了。有老人可怜他,会把吃的东西放到山口用一只缸罩起来,他晚上会把缸搬开,把吃的东西带回去。

我听了觉得奇怪,今天看到的那人狂奔如牛,一点也不像老人,难道我们城里人的体质连山里的老疯子都不如?

也确实有可能,因为说是老了,也不知道到底多老,说不定只有四十几岁,因为没吃没喝风吹雨打所以显得非常老,但就冲着常年在山上生活,他的体质肯定异于常人。

胖子拿着水杯进来又对我道:“听到没有,现在是收西瓜的季节,那边人太多,你得沉住气,这里不比荒郊野外,你想怎样就怎样,与其冒那个风险,咱们不如稍微等等。我看咱们等到后半夜最合适,小不忍则乱‘大便’。”

我算了一下,心说不行,如果确实是个疯子,那他的行为是不可预测的,难保他不会爬回去看看。对于他来说爬到一幢村里的废弃老屋里不算什么大事,谁知道他会在里面做什么。就说我等不及,待会儿吃了中饭我还得去转转,能进去我就进去把这心事了了。

胖子就苦笑,不愿意和我多谈了,就说随我。

长话短说,吃了中饭,我和闷油瓶又去了老屋外头,发现门口的大树下,竟然坐着几个老鬼在纳凉。

故事和现实生活的区别就是,故事你总能在关键时候加快节奏,但是现实生活总他娘的会出意外。我蹲在一边的树下,等那几个老头离开,等到脑门油都晒爆了,那几个老头反而越聊越欢快。

我很难形容那种堵在胸口的焦虑,又不想回去被胖子笑话,就在忐忑不安中度过了几个小时。胖子后来找了我们,他看我们这么久没回来,以为我们被逮住了。

我此时已经逐渐冷静下来,或者说是“热”静,因为烈阳高照,空气中翻起潮湿热浪,我们拿着芭蕉叶扇凉也不顶用,给蒸得都发泡了,热得没了动力。那些焦虑全从毛孔晒了出去。闷油瓶真是让我佩服,即使这么热,他也岿然不动,一点也看不出烦躁,但是同样浑身汗湿。冰山一样的酷哥同样挡不住广西的大太阳。

胖子奚落了我一顿,我也没力气反驳他,他在北京待久了,完全没法习惯这里的湿热,更是难受,便对我们道:“走走走走,别干等着,咱们出去走走,找条溪泡着,否则我非馊了不可。”

绕出村外有一条山涧,我们来的时候见过,不宽但是水挺急的,当时看见就觉得那儿肯定是个避暑的好地方,只是不知道从寨里怎么走才能到达。

我也实在吃不消了,一听就感觉合意,就爬起来三个人一块过去。沿途问了几个村民,村民给我指了路,胖子摘了芭蕉叶挡在头上,一路骂太阳一路七拐八拐走出了寨子。

寨子和溪涧基本相邻,山区的寨子基本都建在溪涧的旁边,寨子和溪涧之间是石头滩子,下大雨的时候水会漫上来,这些石卵可以起到一个缓冲的作用。我们在埂上眺望了一下,发现戏水的人还不少。看来当地人也不是不怕热。

碧弯弯的溪涧水比我们在下游看到时平静,走到溪边就感觉一股凉意扑面而来。在游玩的大部分是孩子,十五六岁的女孩子都不穿内衣只穿着衬衫,湿透的衣服贴在身上显出了曼妙的身材。胖子一下就来劲了,三两下脱掉衣服就往溪水里冲,好像猪八戒看到蜘蛛精一样。

我感觉自己穿着三角裤不雅观,就穿着运动短裤下了水,阳光下的溪水有点暖和,我走到石头下的阴凉处。闷油瓶没有下水,坐在一边的树下纳凉。

泡了片刻,暑意就全消了,一种悠闲的惬意扑面而来,胖子在和女孩子们嬉戏,闷油瓶打起了瞌睡。我抬头往寨子望去,能看到闷油瓶的高脚楼就在不远的地方,这比在阿贵家里干等要舒服多了。

好比发榜的考生,在发榜的墙前等着,比在家里等着要舒坦一点。刚才的焦虑让我都觉得有点可怜自己,于是告诉自己,不要紧张,这一次我们不是倒斗,在这里什么都不会发生,不会有粽子,慢慢来就行了。

于是我躺了下来,把身子浸在水里,闭上眼睛,舒展身体。

也不知道躺了多久,我有点蒙眬的时候,忽然就听到有人叫我。我逐渐苏醒,刚坐起来,一溜水就拍到我的脸上,把我一下泼清醒了。我起来后发现戏水的孩子都跑回了岸上,朝着一个方向叫着跑去。胖子一边泼我一边叫着:“醒醒!”

我站起来,看到远处的寨子里的某处,竟然冒起了青烟,问怎么回事情?胖子道:“好像有房子着火了。”

我看向那个方向,那是闷油瓶高脚楼所在的地方,顿时觉得不妙。

\chapter{火灾}

此时我还只是有不祥的感觉,但我的内心还是告诉自己,不可能这么巧合,这种天气里木制的老房子发生火灾的概率很高,但是心中不祥感渐渐强烈到让我有点窒息。

跟着小孩子跑,冲向着火的地方,越靠近我就越觉得不好。等到我冲到跟前,我几乎不敢相信眼前的情形,只见闷油瓶的高脚楼里冒出了滚滚浓烟,火势极大,热浪冲天,根本没法靠近,一看就知道已经烧得没法救了。高脚楼后面的山也烧了起来,灌木丛一片焦黑,火还在往上蔓延。

村民正从四面八方赶来冲到山上去扑火,我们经历过山火,知道山火一旦烧起来,那种可怕的后果是难以想象的。所以先救山火绝对是正确的。

这火的源头似乎在山上,闷油瓶的高脚楼就在山脚边,于是受到了殃及,但我呆立在那里,知道肯定不是这么回事。

火势太大了,我们到溪里去才多少时间,就算被雷劈中也不可能烧得这么快。最明显的是,空气中弥漫着一股浓烈的煤油味。

这里没有消防栓,所有的救火设备只有桶,但是桶的数目有限,他们又是从水缸里舀水,等山火扑灭的时候闷油瓶的房子肯定已经烧得一点也不剩了。我情急之下想冲进去,胖子一把把我拉住,说已经没办法了,进去太危险了,犯不着把命丧在这里。

我脑子里一片混乱,跪倒在地上,这时忽然边上人影一闪,我们还没反应过来,就看到闷油瓶冲了过去,冲到火房前,往高脚楼底下的隔空处滚了进去。

胖子和我都大惊失色,要知道在这样毫无保护的情况下冲进火场,绝对是重度烧伤,没一点情面可讲。不是说你不碰到火就没事了,火场中心的温度高达上千度,在里面待着一瞬间就熟了。

胖子马上大叫救人!我和他立即冲过去,挨近房子五六米处,滚烫的热浪就扑面而来。我的汗毛立即就被烤卷了,眉毛头发发出啪啪的声音。我咬牙忍住皮肤的灼痛,冲到房子边上,蹲下去,立刻发现根本不可能进去,里面的高温犹如火龙的呼吸涌出,趴下去勉强看,地下有潮湿的泥巴,闷油瓶裹了一身湿泥,正在往里爬。

再想仔细看已经不行,我们被热浪烤得没法睁开眼睛,只得连滚带爬地退出来。旁边救火的人赶紧冲上来把我们拉住。

刚被扶起来,就听到火场里传出一声东西垮塌的巨响,接着闷油瓶也从高脚楼的隔空处滚了出来。他浑身都冒着白烟,跌跌撞撞爬起朝我们跑来,旁边马上有人上去往他身上泼水,边上有人说疯了疯了。

我冲过去,只见他浑身裹满了房下的烂泥,不知道有没有烧伤,但能看见左手有几处全是黑灰,显然他豁出去用手掏了。我大骂:你不想活了!胖子扶起他就问道:“怎么样?”

他面无表情,只冷冷道:“全烧没了。”说着看了看忙着救火的人们,“全是煤油味,连地板都烧穿了。”

这动作的意思不言而喻,胖子也看了看救火的人,面色不善地看了看我:“小吴,看来这村子有点问题。”

我看着闷油瓶的伤心里没空琢磨这些,边上有人对我叫道:“快带他到村公所找医生吧,烧伤可大可小,那房子没人住,学什么救人啊。”

我们找了一个围观的小孩带路,带闷油瓶到村公所后,那小孩让我待着,他去叫医生过来。我想起刚才还是后怕,忍不住埋怨闷油瓶。胖子让我别烦人了,小心被人听到。我才闭嘴,心里堵得有点喘不过气来,也不知道说什么好。

闷油瓶似乎根本没在意身上的伤口,只是在那里发呆,不知道想些什么,气氛凝固了。

这种郁闷我都不想形容,谁也没有想到会发生这种事,早知道这样我宁可当场被逮住打一顿也要先进去看了再说。现在说什么都晚了。

四个小时后才把大火扑灭,很多人都烧伤了,不久后来了一个赤脚医生,用草药给伤员处理伤口。闷油瓶一检查倒还好,大概是因为地下的淤泥隔热,他的烧伤虽然多但都不严重,只有左手烧伤得有点厉害。赤脚医生似乎见过大风大浪,也不紧张,慢吞吞地给他们上了草药,说只要坚持换药,一点疤都不会留下。这里夏天山火频发,村民自古对于烧伤就有很多的经验。

我们几个都不说话,回到阿贵家里一清洗,我的眉毛头发都焦得直往下掉。简直惨不忍睹。

闷油瓶彻底陷入了沉默,房间里满是烧伤草药奇怪的味道,很难闻。我有点责怪胖子,对他道如果不是他说先回来,当时我们头皮硬一下直接进去把照片拿出来,就不会有现在这事了。

胖子就火了,道这怎么能怨他,既然有人放火那咱们肯定早被人盯上了,出事是迟早的。这次烧的是老房子,如果咱们看到了照片,那他们烧的可能就是我们了。而且当时那种情况,是人都不会硬着头皮进去,光天化日之下你爬到人家房里,胆子也太大了。

我也是有股闷气没处发,确实怨不得胖子,可是胖子这么说我就一肚子无名火,硬是忍住和他吵架的冲动,用头撞了几下墙壁才稍微缓和了一点。

胖子啧了一声,对我道:“我看这事咱们就是没办法,我估计他娘的早就设计好了,不然我们不可能这么倒霉。偷箱子那疯子,我看可能是别人装的,也是放火人那一伙的。你想他偷箱子的时候动静那么大,还故意敲了地板引起了我们的注意,肯定就是把我们引出去。”他顿了顿,“然后他的同伙在外面,我们一出去看到他们,就肯定不敢再进去,等我们一走他们就放火烧房子……他娘的,肯定是这么回事儿。”

有道理,我点头,这么说来,他们应该是临时发现了我们,情急之下把我们引了出去,如果早知道我们的计划,他们早该采取措施了,不会这么急切和极端。

如果真是这样,那放火的很有可能就是当时在山坡上看着我们的那几个村民……他们是什么人?我从来没有见过他们,他们也不应该会认识我。

“他们肯定不知道我们在找什么,如果他们知道我们在找照片,只要把照片拿走烧掉就可以了,不需要把整栋房子烧了。”胖子道,“不过这些人也不聪明,露了脸了,我就不信我们拿他们没辙。你还记得他们长什么样子吗?”

我有些模糊的印象,不过那么远的距离也实在不能认全,肯定会有些困难,于是不由得叹气。

如果闷油瓶没有突然想起那只箱子,我们会直接看到照片,也不会出现现在这种情况,但是这样一来,这只箱子就将埋在烧焦的废墟下面,永无出头之日。错有错着,我们并没有完全失败,想到这里,我倒有些释怀。天无绝人之路,而且这房子一烧,我就知道了一件事情:这村子里肯定有人知道什么,而且不会是普通的事,不管怎么说,这算条线索。

只是,不知道是否那批人接下来还有行动,会不会对我们有所行动,胖子说应该不会来害命,否则没必要烧房子,直接杀了我们就行了。不过咱们还是要小心,以后必须多长个心眼。

就算是这么想,胖子还有点放心不下,去阿贵的院子里里拿了几把镰刀回来藏在床下防身,还搞了几只杯子,挂在门窗上,门窗一动就会掉下来发出声响。

我这时候总觉得心神不宁,有一种预感——既然有人在阻挠我们,阿贵帮我们找当年那个老向导的事情也会出变故。有人不想让我们继续查下去。

\chapter{变故}

山火最后不了了之,听阿贵说起来,好像是天气太热的原因,具体怎么烧起来的还不知道,反正这里每年夏天都会有山火,只是离村子这么近还是第一次,幸亏烧了的是废弃的屋子,没有太大的损失。

我心中暗骂,我的损失可大了,这样一来,楚哥对我们说的线索就全断了。现在唯一的办法就是出去后想办法逼楚哥开口,这肯定不是容易的事,而且必然要使用胁迫的手段,我并不太能接受。不过,不是完全没戏,所以我倒没有极端的郁闷——只要楚哥不被烧掉就可以了。

和胖子说了说,看来我们在这里待不了多少时间,找了老向导之后,如果没有特殊的理由,我们可能就得回长沙,因为留在这里已经没有意义。所谓的羊角山倒斗,可能得下回分解。

胖子也很无奈,虽然有点舍不得,但是我们这一次过来什么工具都没有带,要去羊角山也不是很现实。但他还是坚持要去山里看看再回,于是最后就定了个再议。

之后我一直忐忑不安,总觉得老向导的事情肯定也会出岔子,想着作最坏的打算,以便到时候真的发生,我能好受一点。

出乎意料的是,老向导的事情非常顺利,阿贵回来后告诉我们他已经约好了,明天我们就可以到老猎人家找他。那老头脾气有点怪,他和那老猎人说我们是政府的人,老头可能会积极点,让我们到时候别露馅就行。

胖子一看就不是当政府官员的料,一商议,就让他别去了。他说他去化肥店想办法讨点硫酸,看看能不能溶掉那只“铁葫芦”,看看其中是什么东西,再去烧掉的废墟里扒扒,说不定还能够扒出点什么来。

我觉得分头行动也不错,但还是千叮万嘱,硫酸讨回来后千万别轻举妄动,要等我们一起的时候再琢磨,这“铁葫芦”还是有点危险。胖子满口答应,说自己又不是小孩。

商议妥当后我们便去睡觉,一夜无话,各怀心思。到第二天天亮我们分头行事,我和闷油瓶由阿贵带着去找老猎人,胖子直奔化肥店。

本以为不会出岔子了,没想到到了之后老头却放了我们鸽子,说是昨天晚上进山去了,现在还没回来。

猎人打猎那是满山游走,根本无处寻踪,我心说这是怎么回事,怎么约好的突然就进山了,难道还是被我料中?老头的儿子也有点不好意思,就说老头老糊涂了,两年前突然就开始有点不正常,时不时不打招呼就进山,也不知道去干吗。谁说了都不听,说去就去,第二天多重要的事情都不管,你看猎枪都还在墙上挂着,肯定不是去打猎,等等就能回来。

我心说那也没有办法,只能等等了。刚在他家坐下来,忽然从门口又进来一个人,进来就问:“盘马老爹在吗?”

盘马老爹就是老向导在这里的称呼,看来还不止我们一个人找他,让我惊诧的是,这人说话一口的京腔。

我们朝外望去,就见一个五短身材的中年人绕进来,我一看他的脸就感觉有点异样,这人长得肥头大耳,但是收拾得很整齐,晒得黝黑但看不出一点干体力活的样子。

盘马老爹的儿子立即就迎了上去,阿贵对我道:“这是盘马老爹的远房侄子,听说是个大款。”

我听他的口音,京腔纯正,心说这远房亲戚也够远的。

那中年人似乎对这里很熟,也没什么犹豫径直就入了院里。给老爹的儿子递了根烟,他已经看到了我,面露疑惑之色,呀喝了一句:“有客人?”

老爹的儿子用乡音很重的普通话说:“是,也是来找我阿爹,这两位是政府里的——”

那中年人似乎对这个不感兴趣,立即打断他问道:“老爹呢?”

老爹的儿子面露尴尬,又把他老爹行踪不明的事情说了一遍。中年人啧了一声,点头:“老爹这是什么意思?又不在,老让我吃瘪,我和老板那里怎么说啊。”说着看了看我们,面有不善道,“你这孙子该不是嫌钱少,又另找了主顾,想诳我?”

老爹的儿子忙说不是不是,说我们真是找老爹的,政府里的人。

中年人又看了我们一眼,半信半疑的模样,走到我们跟前:“你们是哪个单位的?这镇里的人我还都熟悉,怎么就没见过你们?”

这就问得有点不客气了,我抬头看了看他,也不好发作,道:“我们是省里的,我们找老爹做个采访。”

“省里的?”他怀疑地看着我们,不过看我们确实像机关单位的,就嘀咕了一句,转头对老爹的儿子道:“得,那你再劝劝你老爹,我老板开的价不低了,留着那玩意儿,生不带来死不带去的,有什么用对吧?别固执了,卖了绝对合算,拿点钱老头子享几年清福多好。”

他儿子不停地点头。

中年人又道:“你们有客人,我扎堆在这儿不好,我先撤了。”说着又笑了,“事情成了,我带你们去风光风光。多用点心,晚上找我喝酒去,我先走了。”

说着出了院子,头也不回,风风火火地走了,我看着莫名其妙,就问他儿子这人是谁啊?他想干什么?

老爹的儿子看他走远了就松了口气,叹气道这人是他们的一个远房亲戚,说是老爹的侄子,他的堂兄弟。这人是个地痞流氓,一直在北京混日子,他们早就不来往了。这人不知道最近跟了哪个老板,跑到广西来收古董,到处让他介绍人,这人自来熟,特别虚,他们又不敢得罪。

我问道:“听他的意思,他看中你家什么东西了,想收了去,难道你家还有什么祖传的宝贝?”

\chapter{巡山}

老爹的儿子唉了一声,对我道:“说这事我就郁闷,我家老爹手里有块破铁,一直当宝贝一样藏着掖着,说是以前从山里捡来的,是值钱东西,以前一直让我去县里找人问能不能卖掉,我也就当他发神经。不知为什么前段时间这事被那远房亲戚知道了,他还真找到人来买,出的价钱还不低,结果还真是有病,老爹来了劲了又不卖了,惹得那小子就是不走,一直在这山沟里猫着整天来劝,给他烦死了。”

我看了看闷油瓶,心中有所触动,看来那老头爽约不是因为我们,而是为了避开那远房侄子。铁块?难道那老头手里也有我们从闷油瓶床下发现的东西?

阿贵在一边抽烟笑道:“你就不能偷偷从你老爹那儿摸了去,换了钱不就得了,以后政府来收可一分钱都不给。”

那儿子道:“不是我不想,这老头贼精,我有一次说要把那东西扔了免得他魔怔,他就把那东西给藏起来了,那时我就找不到了。哎,想想真想抽自己一巴掌,没想到那块破铁真的值钱,要是真能做成这买卖,那是天上掉下的金蛋,我儿子上学的事就不用这么发愁了。”

我听着暗自感叹,表面上看这儿子有点不像话,有点腻歪老人的意思,但是我看得出这家人确实是有困难,这种家务事上我们也不能插嘴。

这时闷油瓶忽然问道:“你父亲把东西藏起来,是不是在两年前?”

他儿子想了想,点头道:“哎,你怎么知道?”

我立即明白了闷油瓶的意思,接着道:“你父亲肯定是把东西藏到山里去了,老人心里不放心,所以隔三差五去看看,这就是你父亲反常的原因。”

他一听,哎了一声说有道理,阿贵道:“那你老爹对这事还真上了心了,你还是再劝劝吧,要真把它偷了,你老爹非拿枪毙了你不可。”

儿子道:“那是,我老爹那爆脾气,我也懒得和他吵,实在不成也就算了。就是我那远方亲戚实在是缠人,我怕依他那秉性,这算是挡了他的财路,我们家以后就不得安宁了。”

我们一边闲聊一边等着盘马老爹回来,他儿子对我说了不少盘马老爹的事情,也让我对这个老头有一个了解。

盘马是当地的土着,在这片土地上繁洐了好几代,是现在硕果仅存的老猎人之一,他们的下一代大部分汉化了,一般只在农闲的时候打打猎,更多时候都出外打工,女孩子也都嫁到外地去了。后来这里的旅游业发展起来了,情势又有了变化。

说起来,盘马老爹在当地也算是个名人,枪法好,百步穿杨,而且身手利落,爬树特别厉害。以前逢年过节盘马都是大红人,都得靠他打野猪分肉,后来经济发展了,他年纪也大了,也就慢慢不被人重视,所以开始有点愤世嫉俗,为人又特固执,后来和子女都处不好。

这种老人像是一个经典样本,我知道的就有不少。我以前的邻居是个老红军,也经常念叨世风日下,不屑与我们这些不懂事的年轻人为伍。这是典型的和自己过不去。想想自己也是,好像人最大的本事就是折腾自己。

聊着聊着,我们在老头家里傻等到下午,老头还是没回来。我再怎么掩饰也无法压住我的焦虑,一方面怕有什么节外生枝,一方面是等得太久了。

老爹的儿子很不好意思,对我们说他去找找,不料一去之下也没回来。我们一直待到傍晚,实在等不下去了。

阿贵很没面子,嘴里骂骂咧咧说这两父子太不像话了,一起走出来,却正好碰到老爹的儿子急匆匆地路过,后头还跟着一批人,也没跟我们打招呼,直往山上去了。

我看到老爹的儿子面容不善,阿贵很纳闷,抓住一个人问怎么回事,那人道:“阿赖家的儿子在山上发现了盘马老爹的衣服,上面全是血,老爹可能出事了,我们正找人去发现衣服的地方搜山。”

“是在哪儿发现的?”阿贵忙问。

“在水牛头沟子里,阿赖家的儿子打猎回来,路过发现的。”

“这么远?”阿贵非常惊讶。

我对于这里的地名一点方位感都没有,就问道:“是什么地方?”

“那是周渡山和羊角山前面的山口,要走大半天才到。”阿贵对我们道,“你们先回去,我得去看看。”说着就跟了上去。

我和闷油瓶对看一眼,感觉难以言喻,心说真的被我料中了,这事也出了岔子。

闷油瓶面色沉寂,看不出一丝波澜,但是脚步却跟了上去,我快步跟上,心说此事实在蹊跷,我们有必要去了解清楚。

\chapter{水牛头沟}

我们想要去帮忙搜山,阿贵一开始并不答应,我们好说歹说才跟了过去。阿贵的小女儿叫云彩,阿贵让他的女儿跟着我们,别走散了。村民们聚合起大概二十人,举着火把和手电,带着猎狗往水牛头沟走。

山路四周漆黑一片,我们一边叫喊一边让猎狗闻着衣服去。

这里的林场都被砍伐过一遍,前路并不难走,只是这里雨水充沛,山上多有积水坑,里面全是山蚂蟥。我们一直走到保林区,路才难走起来,不过这些山民全是猎人,经验丰富,走起来一点也不吃力。对于我们来说,这样的山路和塔木托比起来实在像是散步一样。一行人就这么往大山的深处走去。

我一边走一边问云彩,水牛头沟一带是什么情况,老爹是否会有什么危险?

云彩回头道:“那里是大保林区和我们村护林区的边界线,羊角山在大保林区,周渡山在护林区,中间就是水牛头沟。羊角山后面就是深山老林了。林场的人都在山口立了牌子的,让我们不要进去,所以除了以前的老猎人,我们一般都不去羊角山,羊角山后面的林子更是没听说有人进去过。”

阿贵在我后面道:“村子里对羊角山最熟悉的,恐怕只有盘马老爹。后面的林子据说以前只有古越的脚商才敢走,古时候越南玉民为了逃关税,从林子里穿一个月的路过来卖玉石,不知道多少人被捂在这些山的深处。”

玉石买卖是古中越边境最暴利、最残酷、最具有神秘色彩的商业贸易,我听说过越南和缅甸玉帮之间惨绝人寰的斗争,一夜暴穷、一夜暴富在这里平常到不能再平常,在那种以一搏万的巨大利益下,人性完全没有任何容身之所。

阿贵说这里离玉石交易最盛的地点不远,从巴乃到广西的玉商,都和广东的一些老板做小生意,是最苦的一批玉民,所以也特别的凶狠。特别是清朝的时候,越南人半商半匪一批批过来,那是当地一害。

我心里想着如果是这样,如果能在林子里发现那些越南玉民的遗骸,说不定能找到他们带来的玉石原石。这年头玉色好的原石十分稀有,玉石价格高得离谱,当年的玉石质地比现在高出好多,如果找到一两块好的,那比什么明器都值钱。不过转念一想,那些越南玉民当年对这些玉石看得比自己的命还珍贵,如今如此截取,是很大的不义,这和盗墓不同,恐怕会招来不祥之事。

走到前半夜头上我们才走进沟里,发现血衣的人指了指一棵树,就说衣服是树上发现的,他先看到有血粘在树干上,抬头看才发现衣服,刚开始以为是被野猫咬死的夜猫子,后来才发现不是。

手电照到树上,这种铜皮手电简直没有什么照明能力,但是能确定上面没有其他东西,显然是盘马老爹爬上树后,将血衣留了下来。

老爹快八十岁了,虽然以前爬树是高手,但按道理不可能无缘无故爬到树上去,显然是遇到了什么危险。我问云彩,这里有什么猛兽?云彩说很久以前听过有老虎,现在在山里,最厉害的东西可能是豹子。

我一听,心说老虎现在绝对没了,豹子是爬树的好手,如果真是豹子那就麻烦了,而且豹子有把食物挂到树上藏起来的习性,搞不好老爹已经遇难了。

不过阿彩又道豹子都在深山里,这里的山不够深,遇到豹子的概率太小了。老爹没有带枪,到这么深的山里来干吗?

我想起小兵嘎子把缴获的手枪藏在鸟巢里的情节,心说难道盘马老爹也学的这一招,但是树上并没有鸟巢。

我们在树的四周搜索了片刻,没有任何所得,只能勉强看到一些血迹,几个方向都有。带来的几只狗派上了用场,猎手们都带着枪,子弹上膛后兵分几路往远处去找,我跟着阿贵那一路往羊角山的方向走。

水牛头沟很长很深,没有人走到尽头过,沟的中段就是羊角山和周渡山相接的山口,呈现出一股热带森林的势头,和塔木托的感觉很相似,让我很不舒服。我总是忽有忽无地听到“咯咯”声然后起一身冷汗,但是也没有办法,自己要来的,只得硬着头皮跟着。

山狗相当剽悍,站起来比我都高,虽然全是杂种狗,但是训练有素,很快就闻到了味道,一路引着我们往山谷深处走去。

一路无话,走到后半夜月牙顶在头上,狗似乎找到了目标,我们在羊角山山口附近停下下来。那是山腰上的一个斜坡,因为泥石流的关系树木很稀,斜坡非常陡,而且泥土湿滑,松软得好比雪层。我们用树枝当拐杖,才能保持平衡,时不时踩错了地方,整片的泥就那么一路滑下去。

猎狗拉着我们,艰难地半爬着来到一处树下,之后就不再徘徊,而是对着树后的一大片草丛狂吠。

云彩有些害怕,我的心也吊了起来,如果老爹遇到了豹子,那么草丛里的东西可能惨不忍睹。

阿贵上前用树枝拨开草丛,手电照射之下却发现里面没有尸体,只看见一块大石头。我们过去后发现那是一块年代久远的石碑断片,有些年头了,风吹雨打的痕迹很明显,表面都磨蚀干净了。

阿贵他们拨开四周齐腰的杂草寻找,忽然一个猎人哎呀了一声,人一下矮了下去。

我们忙冲过去将他拉住,就见草丛里隐蔽着一个泥坑,好像是被雨水冲出来的,坑里还有烂泥。往坑底一看,我和闷油瓶对视一眼,心里都咯噔一下,坑里隐约可以看到几截烂木头裹在烂泥里,看形状我基本能肯定那是一只已经支离破碎的棺材。

这是一个被冲出来的简陋古墓。

\chapter{古坟}

月光惨白照在山腰里,四周什么都看不见,但能听到坡下沟里密林深处发出各种各样奇怪的声音,这个坑让阿贵他们怔住了。山民迷信,看到棺材总认为不吉利,他们互相看看,阿贵没有什么想法,自言自语道:“大半夜的看到棺材,回去要洗眼睛。”

另一个人趴下来看了看,道:“这是谁的坟,怎么挖在这么深的山里?”

没人回答他,云彩吓得躲在闷油瓶身后。

我能肯定这肯定是一个荒坟,不是大户人家的墓,年代应该是明清,因为这样质量的棺材,在雨水这么充沛的地区能够保存到现在,时间不可能太早。看棺材里的烂泥里也有草长起来,那么棺材被雨水冲出暴露在野外至少有几个年头了,里面的尸骨肯定已经被毁了。

坑不大,用手电照照,我们找不到里面有盘马老爹的踪迹。人肯定不在,但我感觉这里可能就是盘马藏东西的地方,因为它确实十分适合藏物。盘马儿子说的铁块可能就在下面。

狗还在叫,引得人烦躁,阿贵把狗拉远,让它们在四周晃荡,接着拾来树枝在里面翻找。

他们也不敢下到坑里,对于棺材普通人都会忌讳,但是狗的反应告诉我们这洞里肯定有东西。这样找肯定是找不到的。

我看了看这里的山势,就是我这个只知道风水皮毛的人也能看出来,这里绝对不适合葬人。这里是山口,山上所有的水都会往这儿来汇聚,在这里葬人不出几天就霉了。这个墓不会是胖子推测的在羊角山中的大墓,只可能是普通的荒山古墓,应该没什么危险。于是我就让阿贵别搅了,我和闷油瓶下坑去翻。

我下盗洞都轻车熟路,更不要说是翻个棺材,何况闷油瓶还在身边。阿贵却非常惊讶,觉得我这样的城里人怎么胆子这么大,云彩更是眼巴巴地看着,有点反应不过来的样子。

两个人一前一后下到坑里,因为坑在斜坡上,坑壁一边很浅,一边很高,能看到山坡塌陷形成的断壁,半截棺材嵌在断壁内,个头还不小,看上面的残漆是一只黑色老木棺,沉入墓底的淤泥有半尺——不是这里土质沉降,就是这老棺奇沉。

这种简陋的葬法也不是一般百姓能用得起的,棺材看似是上路货色,可能是以前这里地主的买办。墓里头已经破得不成样子,四处全是烂泥。

不知道是不是被胖子传染了,看到棺材我的心跳也开始加速,我告诉自己,这时候必须表现得外行,否则很容易被阿贵他们怀疑。

闷油瓶接过手电,拨开那些杂草,只看了一圈,我们就看到棺材的不显眼处,有一些手印的血迹。闷油瓶让我帮他照着,伸手对着比画了一下,那个棺材上的手印,应该是俯身平衡身体的时候粘上去的。闷油瓶也蹲下去,下面就是棺材的裂缝,他想也不想,直接把手伸到裂缝内,开始在烂泥里掏起来。

听着淤泥搅动的声音,我觉得后背发毛,他只是在烂泥中摸了几把就将手拔了出来,手里拿着一块粘满烂泥的东西。甩掉上面的泥,那是一只塑料袋,上面也有血迹,但闷油瓶抖了几下,我们发现塑料袋是空的。

“怎么会这样?”我奇怪道,“东西呢?”

“血迹是新鲜的,他把东西拿走了。”闷油瓶看了看四周,淡淡道,“时间不长,肯定就在附近。”

“这么说他是受了伤之后,才来这里拿的东西?”我松了口气,从受伤的地方到这里有段距离,既然能走过来,那么伤得不会太重。

闷油瓶又摸了一下,没摸出什么来,我们爬上去,我对阿贵把情况说了说。一个没有枪的老猎人,虽然强悍而有经验,但是绝不可能逃过一只豹子的攻击,而且奇怪的是,在受了伤之后他为什么还要来这里,他应该立即回村才对。他一路流了那么多血,过来将这铁块拿走,是什么原因,难道他觉得铁块放在这里会有危险?

我们把狗叫了回来,以古坟为中心,几个人各自到四处去找。一拨人往山上去,一拨人顺着山腰,我们两个跟着阿贵父女向谷底找去。我问云彩,除了豹子,林子里还有什么会攻击人的东西?

云彩说以前太多了,现在都给吃光了,以前蟒蛇有很多,现在好久都没看到了,会攻击人的,可能是野猪。不过野猪胆子很小,只有被激怒的时候才会攻击人,盘马老爹经验丰富,不可能在没有武器的情况下去激怒野猪的。

我心说有可能,但还是无法解释盘马老爹到这里来把东西拿走的原因。这时候我心中隐隐怀疑,是不是盘马老爹遇到的危险不是动物,会不会是烧了房子的那几个神秘人袭击了他?正琢磨着,忽然就听到远处另一拨人的方向传来一阵急促的狗吠。

\chapter{老头}

我们立即停下来回头,同时又有谁惊叫了一声。

这一声惊叫犹如厉鬼,我们只看见那边乱做一团,也不知道发生了什么。我们愣了一下,立即抄起家伙往惊叫的地方跑去。

相隔不远,只听狗在狂吠,树影婆娑中也看不出他们为什么大叫。阿贵喝问:“出什么事了?”

“当心!草里面有东西!”前面的人叫道。刚叫完一旁的林子忽然有了动静,好似有什么东西正快速穿过灌木,动静很大,看来是只大型动物。

阿贵端起他的枪开了一枪,打在哪儿都看不真切,炸雷一样的枪响把远处的飞鸟全惊飞了,那动物一阵狂奔,隐入了黑暗中。

我们冲到他们跟前,山上的几个也冲了过来,手电往林子里四处扫去,只见到灌木一路抖动,阿贵马上大叫:“放狗出去!”

几个猎人打了声唿哨,猎狗一下就冲了出去,那气势和城里的宠物犬完全不同,一下前面就乱了套了,灌木摩擦声,狗叫声,不绝于耳。阿贵他们立即尾随而去,几个人应该都有打猎的经验,用当地话大叫了几声,散了开来跟着狗就往林子里跑。

我们想跟过去,阿贵回头朝云彩大叫了几声,云彩把我们拦住,说不要跟去,他们顾不了我们。黑灯瞎火的,猎人不能随便开枪,那野兽逼急了可能伤人。野兽,特别是豹子一类的猛兽非常凶狠,被抓上一下就是重伤,所以要格外的小心,我们没经验很容易出事,而且我不懂怎么围猎,去帮忙也是添乱。

我自然是不肯,心说要论身手,闷油瓶还会给你们添乱?往前追了几步,却发现她说的添乱是另一回事。

猎狗训练有素,三只分开摆出队形,冲到了那东西前面,那东西遭到围堵立刻掉转往回跑,而后面就是围上去的几个猎人。狗和人一前一后,正好形成一个包围的态势。这需要包围圈每个人都有经验,否则猎物就可能找到突破点逃出去。

阿贵他们不停地叫喊,让猎物搞不清状况,不知道该往哪个方向逃,只能在包围圈里不停地折返。同时猎人们都举起了猎枪,不停地缩小包围圈。这是猎野猪的方法,我见过以前老家有类似的情形,猎稍微大点的动物都用这种方式。

太久没看到打猎的真实情形,我们屏息看着,阿贵他们越逼越近,很快猎物已经进入猎枪的射程范围内,只是猎物不停地动,手电光无法锁定。这里的猎狗都是中型犬,猎得最多的是野鸡和野兔之类的小动物,所以也不敢贸然上去。如果是北方猎狼的大狗,在以一对三的形式下,早就冲上去肉搏了。

磨蹭了半天阿贵他们也没有开枪,一般的猎物在这种时候都会犯错误,会突然冲向某个方向,一旦靠近准备着的猎人,猎人近距离开枪就十拿九稳,之后猎狗再追过去,这东西就基本逃不掉了。但是这一只不仅没有立即突围,反而逐渐冷静了下来,没两下就潜伏在草里不知道藏在哪个位置了。这样一来阿贵他们反而不敢靠近。

我看着这些十分诧异,心说厉害啊,反客为主,这到底是什么东西,这么狡猾,难道是只大狐狸?

但是狐狸要多大才能袭击人啊,难道这只是狐狸中的施瓦辛格?

阿贵照了几下实在拿不准,这批猎人不是以前那些一辈子在山里讨生活的山精,经验到底欠缺一些,也没有好办法,就吆喝云彩拿石头去砸,把猎物砸出来。我们捡起石头刚想过去,却被闷油瓶双双拉住,我抬头看他,发现他不知何时面色有变,眼睛没有看着围猎的地方,而是看着阿贵的身后,叫了一声:“当心背后!”

我跟着看去,竟然发现阿贵身后的草泛起了一股波纹,好像是风吹的,但是四周又没有风,又像是有东西潜在草里在朝阿贵逐渐靠拢。

阿贵立即回头,那波纹一下就停止了。

“什么东西?”我惊疑道,“还有一只?”

“不是。”闷油瓶看着四周,冷然道。我把手电扫向周围,一下就发现四周远处的草丛泛过好几道奇怪的波纹,正在向我们聚拢而来。

这里的猎人哪里见过这样的场面,一个个瞠目结舌,还是云彩这丫头第一个反应过来,立即打了个唿哨,把狗叫了回来。

我大叫让他们聚拢过来,几个人聚在一起,仔细去看四周的动静,就见那些波纹犹如草中的波浪一样,忽隐忽现。

三只猎狗比我们更能感觉到情势的诡异,不停地朝四周狂吠,烦躁不堪。几道波纹在不规则的运动中,逐渐靠近我们,我虽说不害怕,但是不可避免地紧张起来,心如擂鼓。

“到我们中间去。”阿贵对云彩说了一句,也搞不清到底是什么状况。不过山民剽悍是真的,竟没有一个害怕的,几个人都把枪端了起来,此时也顾不得我们,我拿了块石头当武器,看了看四周的环境,道:“这里草太多了,我们退到山坡古坟那边去。”

几个人立即动身,一边警惕一边快速往山上走,没想到我们一动,那几道波纹立即就围了过来,在离我三十多米的时候,又一下子消失了。我们几乎没有时间紧张就直接慌张了,正道也不走,直接顺着坡直线往上。

山泥全是湿的,几个男的上去了,一下云彩就崴了脚,滑下去好几米。我拉了一把结果自己也脚下一滑,脚下的泥全垮了。

闷油瓶和阿贵停下来拉我,一下队伍的距离就拉开了几米。山坡上杂草密集得好比幔帐,我此时就听到四周的草丛里全是草秆被踩断的声音,十分密集,顿时心中燃起了强烈的不安。

被拉起来后我去找云彩,云彩崴了脚已经疼得哭了起来,我冷汗冒得腿都不听使唤,咬牙拨开草好不容易把云彩扶到山坡上,那边的烂泥已经又垮出了一个坑。我在她的小屁股上推了一把,上面的闷油瓶单手就把她拉了上去。

我爬了几下,发现我体重太大,没人在屁股后面推我的话,那泥吃不消我的重量还得垮,于是企图往边上绕上去。没想到人背喝凉水也塞牙,没走几步,脚下的烂泥又垮了,我一下摔在山坡上滑落了好几米。挣扎着爬起来,我听上头阿贵大叫:“跑开!快跑开!”

听声音我本能地知道他肯定看到了什么,立即往左一动,又听到阿贵大叫:“错了!不是那边!”一下我看到面前的草丛一阵骚动,接着我看到一只小牛犊般大小,吊睛白额,似豹非豹的动物从草里探出上半身来,两只碧绿的眼睛放着寒光,一张脸狰眉狞目,好似京剧脸谱里的凶妖一般。

我一和它对视就知道这玩意儿是什么东西了,心中无比的诧异——这竟然是一只猞猁。

猞猁是一种大猫,比豹子小,比猫大得多,这种猫科动物的脸好比妖怪,邪毒凶都在上面。猞猁和豹子最明显的区别是猞猁的耳朵上有两道很长的粗毛,像京剧里的花翎。

这种东西智商极高,虽然喜欢独居,但在食物匮乏的时候也会协同捕猎,是除了狮子外能唯一能成群合作捕猎的猫科动物。在西藏,大型猞猁被称为“林魔”,据说会叼年轻女性回巢交尾,但因为皮毛的关系,近代几乎被捕杀干净了。怎么它会出现在偷猎这么严重的广西?

如果是猞猁,倒可以解释盘马老爹为什么被袭击而没有死,猞猁像猫,喜欢将猎物玩得精疲力竭再杀死。而且性格极其谨慎,不会轻易贴身肉搏。

心念电转之间,在我的另一边,又是一只猞猁探出头来。这一只更大,同时头上掉落烂泥,闷油瓶已经从上面下来,滑到了我边上。阿贵的猎刀在他手里。闷油瓶下来后立即拉住我,“踩着我的背上去。”他斩钉截铁道。

“啊,那多不好意思。”我一时没反应过来。

“上来!”上面的阿贵大叫,满头冷汗。

猫科动物最喜攻击猎物的咽喉,一击必杀,我缩起自己的脖子,心说我就不客气了,扒拉了几下烂泥,踩到闷油瓶的肩膀上,闷油瓶猛地一抬身子把我送了上去。上面的阿贵拉住我的手,我乱踢乱蹬好不容易在山坡上稳住,忽然听到云彩一声惊叫,从下面的草丛里猛地蹿出一只庞然大物,纵身跳在山坡上借力。我就那么看着一只“巨猫”踩着飞溅的泥花,几乎是飞檐走壁般飞到我的面前。

阿贵条件反射下放了手,我一下就摔了下去,凌空被咬住了。

幸好猞猁的体形还是太小,没法把我直接压到地上,我摔进草丛里滚下去好几米,随即狠狠踢了它一脚,将它踢了出去,起来一看我的肩膀几乎被咬穿了。

四周所有的草都几乎在动,被我踢飞的那一只刚落地就已经恢复了攻击的姿势,再次朝我猛扑过来。

我完全没有任何时间去害怕和恐惧,这几年的探险生涯让我具备了极强的求生本能,我护住咽喉一下就被撞倒,索性一个翻身顺着山坡翻了下去,疾滚而下。

这一滚真是天昏地暗,爬起来后我也不管三七二十一,跌跌撞撞就跑。后面的阿贵他们已经放枪了,我也分辨不清方向,一直往山谷里的深处冲去。跑出没几米就听到背后一阵疾风,我知道它来了,绝对不能把自己的后脑让出来,脑壳会被直接咬穿的,于是我立即转身。

几乎是刚转身就看到一个黑影以迅雷不及掩耳之势追了过来,根本就没法估计速度,转眼就到了我面前。我心说完了,这一次将我扑倒之后我绝对没有时间再做防御,条件反射下我闭眼等死。

眼睛都没完全闭上,转眼之间,忽然我身边的草丛分了开来,接着寒光一闪,一个人影闪电般从草丛里扑了出来,一下和黑影抱在一起。

黑影来势极凶,两个影子撞在一起后翻出去好远,我愣在那里完全反应不过来,好像做梦一样。只听到猞猁的吼叫和呻吟声,草丛里乱成一团。

不知过了多久,草丛里安静了下来,从里面站起来一个黑影。我松了口气,那人影走了出来,走到了月光下,我才发现那是一个干瘦的陌生老头,浑身都是血,手里提着一把瑶苗特有的猎刀,那只大猞猁被扛在背上,似乎已经断气了。

他走到我跟前,看到我后愣住,用当地话问了我一句,我也不知道他说了什么,只是下意识地摇头,心说这天神爷爷是谁啊?而下一秒我看到了更加让人惊讶的画面——我看见老头的身上,竟然纹着一只黑色的麒麟。

鹿角龙鳞,踩火焚风,和闷油瓶身上的如出一辙。

\chapter{盘马老爹}

老头很瘦,和肩膀上肥大的猞猁一比就更显瘦削,但是仔细看能看到他身上已经萎缩的肌肉仍精练如铁条,可以想象在壮年的时候会是何等雄伟。月光下老头的眼睛炯炯有神,有一种让人说不出的感觉。

他把猎刀收回到腰后的鞘里,又打量了我一下,把猞猁换过到自己的另一只肩膀上,接着用当地话让我跟他走。

四周的草还在动,但老头熟视无睹,背着猞猁一路往前。很快,四周的动静逐渐远去了,林子深处传来了它们的悲鸣声。猞猁都是临时组成的狩猎团体,这一只可能是其中最强壮的,负责最后的扑杀,它一死狩猎团体就瓦解了,猞猁生性十分谨慎,绝对不会再冒第二次险。

老头一边叫喝,一边往古坟的方向走,手电光闪烁不定,但始终定在山上,显然阿贵这家伙不厚道,没下来救我。

只有一只手电朝这里来,我们迎上去,看到闷油瓶少有的有些急切,看到我没事后似乎松口气,接着他看到了老头。

闷油瓶的手上也全是血,阿贵的猎刀被反手握着,两个人对视了一眼。闷油瓶看到老头的文身,顿时就愣住了,但是老头好似没有注意他,径直就从他身边走了过去。

我心说我靠,好酷的老头,有闷油瓶的风范,难道这家伙是瓶爸爸?

闷油瓶想上去询问,我将他拦住,说这老头不是省油的灯,而且显然语言不通,问他也没有用,先回去再说。

途经我摔下来的地方,看到地上也有一具猞猁的尸体,脖子被拧断了,显然是闷油瓶的杰作。老头示意我们抬起来,闷油瓶将尸体过到肩上,一起爬上山坡,上面的人立即跑了过来,看到老头后显得很惊讶。

老头和他们用当地话唧唧呱呱说了一通,我完全听不懂,我就偷偷问云彩,这老头是谁啊。

云彩道:“还能是谁,他就是你们要找的盘马老爹。”

“他就是盘马?”我不由得吃惊,不过之前也想到了这一点。都说盘马老爹是最厉害的猎人,除了他还有谁能这么老的年纪徒手杀死一只这么大的猞猁。要知道单只的猞猁可以猎杀落单的藏狼,猫科动物是进化到了顶点的哺乳动物捕食者,不是极端熟悉它们的习性不可能做到。

刚才盘马老爹肯定是被猞猁袭击了之后,一直和猞猁周旋到了这里,然后蛰伏下来等待时机。娘的,最后那一下必杀我看就是闷油瓶也不一定能做得那么干脆,就是稍微晚个一秒,我和老爹之间肯定就死一个。

阿贵看了看我的伤势,向我们介绍了一下双方,老爹似乎对我们不感兴趣,只略打了个招呼就开始擦身上的污秽。

擦掉身上的血,我发现他的文身在血污中非常骇人,而且造型确实和闷油瓶的几乎一样,老爹的后脊梁骨有新伤口,深得有点恐怖,可能是猞猁偷袭所致。

几个人嘀嘀咕咕的,述说着进山的经过。自己半猜半琢磨,加上云彩的翻译,我听懂了大概,前面的和我猜的差不离,确实是因为他儿子的事情才进的山,不想怎么会遇上猞猁这种东西。好在老爹进山有一个习惯,就是在背上搭一条树枝,一来可以当拐杖,二来在平地的时候可以防着后面的罩门被偷袭。这都是古时野兽横行时留下来的规矩,一辈子都没派上用场,不料就是这一次救了命,衣服给扯了去,但后脖子没有被咬断,真是险之又险。

猞猁已经多少年没露面了,在这里又突然出现,可能是因为前几天连降大雨,深山里出了异变才被迫出来,人多的地方老鼠多,于是它们被食物吸引到了村寨边上。

老爹的神情很兴奋,似乎是找回了当年巅峰时的感觉,我寻思现在也不适宜多问问题,阿贵吆喝着回去,说村里人该急死了,老爹和我的伤口都有点深,必须尽快处理。

几个人把两具猞猁的尸体烧了,此时天色都泛白了,于是我们踩熄了火立即出发。

猞猁的皮毛价值连城,就这么烧了实在太可惜了,不过阿贵说,不能让其他人知道这里出现了猞猁,否则,不出一个星期偷猎的人就会蜂拥而至,这些人贪得无厌就算打不到猞猁也肯定要打点别的回去,这里肯定会被打得什么都不剩下。

一路无话,回到村里天都大亮了,几个村里的干事都通宵没睡,带着几个人正准备进山,在山口碰上了我们。

我们在村公所里吃了早饭,烙饼加鸡蛋粥,我饿得慌吃了两大碗,村里和过节似的,不停有人来问东问西。

我的肩膀几乎被咬了个对穿,消毒后打了破伤风针,又敷了草药。盘马老爹的背上缝了十几针,那赤脚医生也真下得去手,好比家里缝被褥一样,三下五除二就缝好了,期间老爹一直沉默不语,就听着那些村干部在不停地啰唆。

这些烦琐事情不提,处理完后我们想先回去休息,等缓过劲来再去拜访老爹。不料老爹临走的时候,却做了一个手势,让我们跟他回家。

我和闷油瓶对视一眼,心说这老头真是脾气古怪,两个人站了起来连忙跟了上去,走出没两步,盘马老爹又摇头,忽然指了指闷油瓶说了一句什么。

我们听不懂,不禁看向跟来的阿贵,阿贵也露出了奇怪的神色,和盘马老爹说了几句,盘马就用很坚决的语气回答他,说完之后就径直走了。

我不知道出了什么事,很茫然地看着阿贵,阿贵有点尴尬,我问他老爹说了什么?阿贵对我道:“他说,你想知道事情就你一个人来,这位不能去。”

我皱起眉头,心说这是什么意思,看了看闷油瓶,阿贵又道:“他还说……”

“说什么?”

“说你们两个在一起,迟早有一个会被另一个害死。”

\chapter{坐下来谈}

听了那话,我一下就愣了,这没头没尾的,盘马老爹忽然说了这么一句,我一下没反应过来。但是,同时我脑子咯噔了一下,感觉到这一句话听着有点瘆人。

还没细想闷油瓶已经追了上去,一下赶到那老头前面将他拉住。“你这么说,你认识我?”他问道。

盘马老爹抬头看着他,脸上毫无表情,没有回答,闷油瓶一下脱掉自己的上衣,露出了自己的上半身:“你看看,你是不是认识我?”

两人黑色的文身无比清晰,似乎是两只麒麟正在对决相冲,而他们目视着对方,十分的奇特。

对峙了片刻,盘马仍旧什么都没有说,而是漠然地从闷油瓶身边走了过去,完全不会理会他,面部表情也没有任何的波澜。

我无法形容那时的感觉,很奇特,如果一定要用文字形容,我只能说我仿佛看到了两个不同时空的闷油瓶,瞬间交合又瞬间分开。

“闷油瓶终于遇到对手了。”我当时心里出现了一个奇怪的想法,如果不是时机不对的话我还真有点幸灾乐祸。一直以来,我认为世界上不可能有人比闷油瓶更难搞的人,原来不是,果然很多时候需要以毒攻毒,以闷打闷。

闷油瓶没有再次追上去,他静静地看着盘马扬长而去,就这么几秒钟的时间,刚才那种时空错乱的感觉又烟消云散。

阿贵不知所措,看看我,看看远去的盘马,看看闷油瓶,面色有点撮火,显然搞不懂这故弄玄虚的是唱的哪一出。我怕他出现腻烦情绪,忙拍了拍他,走到闷油瓶身边,和他说让他回去,别急,既然盘马让我去我就去,问完了就立即回来告诉他。

闷油瓶不置可否,点了点头,还是看着远去的盘马,不知道在思索什么。

不知为什么,这时,我觉得他的眼神忽然变得有些不同了,好像少了什么东西,同时我又感觉,这眼神我之前在什么地方见过。

刚才他们四目交汇的时候,一定发生了什么,盘马的这种表现,是一种极强烈的暗示,他肯定知道一些事,而且他肯定知道闷油瓶是谁,甚至和他有过比较深的渊源,但看他的态度,似乎这种渊源一点都不愉快。

我迫不及待地追了上去。

跟阿贵再次来到盘马家的饭堂里席地坐下,我脑子里一直在琢磨盘马的话是什么意思,以及应该如何有效地和盘马这样的人交流。

“你们两个在一起,迟早有一个会被另一个害死。”

盘马突然说出这么一句话,本身就让人摸不着头脑,如果他不是知道什么,他一个山里的猎人是不会无缘无故耍花枪的。但他的态度又很奇怪,而且很明显,他不是很喜欢闷油瓶。

我实在想不出个中关系。这可能是一句很普通的话,也可能带有什么隐喻,我一直告诉自己让自己别多想,也许盘马老爹的意思是我的身手太差,闷油瓶的身手又太好,所以我总有一天会连累他。但是我的直觉告诉我,这句话从承前启后来看,被警告的人似乎是我,我是那个迟早被害死的人。

但是闷油瓶可能把我害死吗?如果没有他,我现在早就是几进宫的粽子了,即使他要害死我,我也只能认栽了,这似乎也完全说不通。

盘马的儿子打来水给我们洗脸洗身体,盘马因为伤口在后背,就由他儿子代劳,他自己点起水烟袋,抽他们瑶族的黄烟。

我闻着味道发现烟味和闷油瓶的草药味有点类似,看来那些草药里也有这种成分。于是我想着能不能以这个当切入口先缓和一下气氛,却完全找不到话头。

天色一下沉了下来,似乎又要下雨,广西实在太喜欢下雨了,盘马的儿媳妇关上窗户后席地而坐,风从缝隙中吹进来,气温一下凉爽了很多,老头这才给我行了一个当地的礼仪,我也学着还了一下。

此时我才能仔细打量盘马的样貌。盘马五官分明,脸上满是和山民一样黝黑的皱纹,非常普通的样貌,这时很难想象当年他天神老爹的派头,真是人不可貌相。这个五官绝对和闷油瓶不会是一个谱系的,想到这里我稍微放心了一点。

阿贵在一边把我的来意说了一遍,还说我是官面上的人物,盘马看着我说了一句话,阿贵翻译道:“老爹说,你到底是什么人他大概也能猜得到,他也早就料到有一天会有人问起这件事。你想问什么就问吧,问完就赶紧走,不要来打扰他。”

我又愣了一下,感觉老爹话里带着什么意思,好像他误会我是什么人了。

可是我又无法清晰地感觉出他误会的原因,想着想着我立即反应过来,知道现在根本不应该去琢磨,当成自己也没发觉是最妥当的,等再有点苗头了,再说清楚也不迟。

我正了正神,心里理了一下,于是对老爹道:“就是想和您打听一下以前那只考古队的事情,我想您能把当年的情况和我大概说一遍。不过,在这之前,我想知道,您刚才的那句话,是什么意思?什么叫我们两个,一个肯定会被一个害死——”

盘马吸了一大口烟,忽然露出一个很奇怪的表情,摇头说了几句话,阿贵翻译道:“老爹说,他刚才那句话的意思很明白,你的那个朋友你完全不了解他是怎样的一个人,和他在一起,你绝对不会有好下场。”

“您认识他?”我立即追问道。“为什么这么说?”

盘马老爹看着我,顿了顿,好久才道:“脸我不认得,但我认得他身上的死人味道。”

\chapter{味道}

阿贵翻译这句话用了很长的时间,显然他也觉得非常奇怪,这是什么意思?我更加不明白了。

“死人味道”是什么味道?尸臭?

我还想继续追问,没想到盘马摇了摇头,让我不要问这个问题:死人味道,就是死人味道。你想知道其他的事就快问,这件事情,他只能说到这里,信不信,他都不管。

我自然不肯就这么放弃,但是盘马的态度很强硬,我求了他几声,他连一点表情都没有,甚至不作回应。

这边阿贵就给我打了几个眼色,让我别追问了,怕问烦了盘马翻脸,我才停了下来,心中不由得暗骂死老头他娘的太不识抬举。

我看得出盘马心里肯定有很多东西,虽然表面上他没有任何表现,但是话里无一不是在告诉我,他知道很多东西。但是他似乎又有点遮遮掩掩,显得态度很矛盾,从他对闷油瓶的不动声色来看,这老头子绝对见过大世面。

我脑子转了一下,换位思考,什么时候人会有这种表现?

一种是有东西待价而沽的时候,我以前和一些掮客打交道,都是这样放一句,收一句。但这老鬼不是很像那些掮客。

另一种是自己心中藏有一个秘密,绝对不能说,但是他看到了一个现象和他的秘密有关,如果他不说可能会导致某些严重的事情发生,在这种矛盾中他只能提供一些模棱两可的说辞。比如说有一个特务已经被人怀疑了,这时候他看到一个小鬼在玩一个铁圆盘,他知道铁圆盘是地雷,但他如果和那个小孩说了,他的特务身份就可能暴露,这时他就会对那个小鬼说:“你和这个东西玩,迟早会被这个东西害死。”

我觉得这种可能性很大,我刚开始来这里只想知道文锦他们进山的一些细节和时间,但他看到了闷油瓶之后,表现出的一些细节让我想得更多。也就是说,推理出他认为闷油瓶是一只会炸死我的地雷,他心中有一个秘密使得他知道闷油瓶是地雷,但是他并不愿意说。

有意思!我忽然就不内火了,他娘的不怕你不泄密,就怕你没秘密。这老鬼会提醒我,说明他良知未泯,至少可以说,他对我的印象应该不坏。现在骂人也没用,耐心一点说不定还能套出来点什么。

不过,一开始就表明自己的窥探想法会让他心生警觉,所以我决定先不动声色,转移一下注意力。于是我点头道:“算了,这个您不想说,那我也就不勉强了,您能和我说说那支考古队的事情吗?”

阿贵听了之后松了口气,显然他怕我们吵起来,不给任何我再问的机会,迅速把这个问题翻译了过去。

盘马这才抬起头来,却又摇摇头,说了一句话。阿贵也立即翻译回来道:“老爹说,你弄错了,那不是考古队,那些人,是当兵的。”

“当兵的?”我一开始以为我听错了,阿贵又翻译一遍。我没听错。

琢磨了一下,我感觉一定是盘马老爹搞错了,当时的人都穿着绿军装,他可能把那些人都当成当兵的了。

(接下来的对话,都有阿贵在其中翻译,为了叙述方便不再一一说明。)

“当时形势很紧张嘛。来了好些个兵,都背着冲锋枪,说是要到羊角山里,找人给他们带路,阿贵的爹当时就找了我,我就给他们带到山里去了。”老爹继续道。

我皱起眉头,忽然想起那时和越南的边境纠纷,上世纪七十年代这里一直在零零星星地打仗,我倒没有想到当时这里正是战区,形势更加的复杂。

这真是我没想到的情况,我一下就陷入了沉思,脑子里很多东西开始闪现出来。

当时那种环境下,肯定不可能会有考古队来这里考察的,那事情就奇怪了……文锦他们还真是神通广大。难道当时的项目是国家派下的项目,有枪就说明真的有当兵的保护。看来盘马老爹说的也不全是假的。

什么项目能够让国家往战区里派进一支考古队呢?难道羊角山里真的有一个价值很大的古墓?

“那些人的背景非常深……”三叔的话在我脑海里一闪而过,让我打了个寒战。

\chapter{盘马的回忆}

之后,我和盘马老爹的对话持续了三个多小时,我不停地提问题,一边了解事情的经过,一边试图试探出那个秘密。

谈话内容十分的分散,老爹讲话加上阿贵翻译,有时候还要互相解释概念,非常花时间。而且老爹并不十分配合我的问题,也或许是阿贵的翻译有一些偏差。所以谈完之后,我的脑海中完全是一片支离破碎的景象。

文锦他们进山的年份,大概是在1976年,老头没法很精确地说出时间。

当时带队的应该就是文锦,但是我拿出西沙的合照让老爹看的时候,他却无法分辨出其他人。时间太久人也太多,当时那种环境下,所有的人都一个发型一种衣服,他只记住了唯一的一个带队,非常合理。

前面的事情平淡无奇,当时这里边境冲突频繁,村里出现部队太平常了,要知道在1978年前后,上思一带几乎都是解放军,山里的路大部分都是打对越反击战时挖出来的,部队要进山里找向导,那是属于军事任务。

盘马拿了部队的津贴,当时他还是壮年,打猎的时候他一个人走得最远、最深,自然是当向导最合适的人选。

他们在当天的清晨出发,部队的任务他不便多问详情,只是将部队的人引到了羊角山里,之后便是跟着部队走。他的心思放在了记路上,羊角山他去得也不多,他必须保证能安全返回。

他们走了相当长的时间,在山里过了一夜,来到了山里的一处湖泊。

那个地方盘马只到过一次,那还是他三十一岁那年,他娶老婆要打几只獐子回去请舅爷。那年山里太不太平,野兽都躲到深山里去了。他一路带着狗找进来,找到了这个湖,在湖边上埋伏了一天,猎到了一只野猪。之后他再没有深入过那里。

那种湖泊自然没有名字,也许除了盘马外,村里人都不知道那里有湖。湖是一个死湖,没有溪涧,底下有没有连着其他地方他就不知道了,部队的人在湖边上扎营立了帐篷,之后盘马的任务就完成了。

接下来,他负责每隔几天送给部队一些给养,部队自身的补给很充足,所以他每次进山只带一些大米或者盐巴。阿贵说的那一次奇怪的事,就发生在其中一次。在此期间没有人知道那支部队驻扎在那里是干什么。

在这个过程中盘马是很好奇的,但是他也知道在那种年月里,窥探这些东西的代价太大,所以他忍住了自己的好奇心。后来队伍开拔的时候,多了很多盒子,大约有三十个,每个都是鞋盒大小。当兵的很小心地带了出来。

他好奇,曾经想拿过一个,但被一个当兵的很婉转地制止了。当兵的说这盒子里装的东西很危险,他寻了个机会拿了一下,只感觉入手十分的重,不知道装的是什么。

我听到这里,脑子里大概有一些印象,这种鞋盒大小的盒子,叫做收纳盒,外号叫做骨董盒,是考古队用来存放出土整理后的文物碎片的。这种盒子一般都被严格编号,有大有小,但是大部分都是鞋盒大小(出土的文物一般较重,鞋盒大小所容纳的重量最适合搬运)。

盘马非常纳闷,因为湖的边上并没有什么特别的东西,盒子里的东西是哪里来的?他当时的想法是这盒子里肯定装的是石头,因为湖泊的边上是大片的石滩,有很多很多的石头。

不过,他很快就发现不对劲,因为在山中行进了一段时间后,盒子里开始散发出一股奇怪的味道,非常难闻并且无法形容。

\chapter{心理战}

我的第一反应是腐臭味,但盘马说不是,常年打猎的人经常和肉食打交道,腐臭味他绝对能分辨出来,那种味道,确实无法形容。

对于气味的形容一般基于物件,比如说“像茉莉花一样香”或者“和臭袜子一样臭”,盘马老爹无法形容,必然是他没有闻过的味道,这种味道甚至连相似的都找不到。

我想问他这种味道是不是就是“死人的味道”,但终究忍住了,如果这个话题他不想说,中途提出来对我并没有好处。

盘马的好奇更甚,但之后那些人开始对他有所提防,他一直没有机会再接触到那些盒子。回到村里之后,这一批人很快就走了,从此再也没有出现过。这件事对他的影响很深,他进山打猎,总是会想起那支军队,他们进山是什么目的,他们在湖边干什么,那些盒子里是什么东西,又是从哪里来的?

当时他就预感到,这件事必然以后会有人打听,但是没有想到,我们来得这么晚,过了近三十年我们才出现。

我问他湖的形态,他告诉我,湖是长的,像一把弯刀。四周全是石头,有的很大,比人还大,有的和鹅卵石差不多。湖现在还在,不过因为气候的变化,湖的水位下降得很厉害,三年前他去过一次,湖已经比原来小了一半。

听到这里我陷入了沉思。盒子中装的大有可能就是我们在闷油瓶的高脚楼里发现的那种铁块,如果是三十多盒,整盒整盒往外搬的话,数量必然不少,还真有可能是如胖子说的,是什么东西的碎片。

这些东西是从哪里来的呢?之前胖子在有限的条件下推测,这羊角山中有一个古墓,但是我现在听来,感觉会不会是从那个湖底捞上来的?

难道他们在那个湖底发现了一只大型的铁器之类的东西,然后他们将其就地分解,一块一块带出去?

不太可能,这样一来这东西就等于废铁,而且如果是这样,不可能用鞋盒那么小的盒子来装。

我不禁也好奇起来,心中已经同意了胖子的想法,无论如何得去羊角山里去看一看。

盘马老爹也有一块铁块,说是山里捡来的,而且他认为价值连城,显然考古队走了之后,盘马老爹肯定还做了一些什么。他不知道我知道他有这块铁块,所以只字未提,这让我更加确定他瞒着很多事。

不过,他现在和我说的,应该也不是谎言。铁块、“死人的味道”是和危险连在一起的,他肯定经历了一件事情,让他把这三者联系了起来。闷油瓶的记忆中,铁块是一个十分危险的东西,而盘马老爹的回忆中,那个当兵的也和他说过铁块很危险,这些都很吻合。

我琢磨着怎么让他开口,要说坏水,虽然我本性比较安分守己,但是和潘子、胖子他们混久了,要挤也能挤出少许来。这种时候,我能利用的就是老爹还弄不清楚我的身份,可以诈他一下。

诈人的诀窍就是让别人以为你基本上都知道了,从而在整个对话的形式上,把询问变成一种质问。

这就到关键时候了,我静了一会儿,脑子里有了一个大概的想法,就又问道:“那么,你后来再回到湖边的时候,是怎么发现那块铁块的。”

这完全是我猜测的,因为铁块既然是从山里找来的,就不太可能是其他地方,我赌了一把,反正猜错我也完全没有损失。

盘马老爹一下人就僵了,我知道自己猜对了,但是他除了那极快的一点僵硬,并没有继续表现出什么来,而是看向我。

我知道这时候要下点猛料,又继续道:“你放心,我只要知道那时候的事情,另外那件事情,我不感兴趣。”

盘马老爹这下脸色就变了,放下烟斗,就问道:“你到底是谁?”

我心中松口气,几乎要出冷汗。这后面一句话,是在上一句猜测的成功上继续加码,死人味道,铁块的危险,闷油瓶的事情。我料想能让老爹保守秘密的,必然是有一个事故,这个事故一定非常的惊险,很可能有人死,我本来可以说:“他的死我就不过问了。”但是我不知道到底死了多少人,所以换了一个更加稳妥的办法。

心虚之人,除非知道我的底细,否则必然会露出马脚。

我心说反客为主的时候到了,立即装出一副高深莫测的表情——我在和客户砍价的时候经常如此——淡淡道:“你还是不要问的好,这整件事情你只要原原本本告诉我就可以了。”说着我摸着口袋抓出一叠钱来,这是本来预备给盘马的资料费,本来打算给个两三百,但是为了视觉效果我把口袋里的一叠都掏了出来,放到自己面前。“我知道一些事情,但是并非完全清楚,所以你不要担心,只要照实说出来,你拿你的钱,之后什么事情都没有,也不会有人知道我们在这里说过什么。”

盘马看着我,露出了心神不定的神色,我用一种非常镇定但是充满逼迫的眼神看着他,等他发飙或者投降。

“你是怎么知道那些事情的?”他问我道,“你倒说给我听听。”

啧,我骂了一声心说这老鬼还真顽固,这怎么说得出来,我表面不动声色,但是脑子立即狂转。

那就是一秒内的反应,我几乎顺口就道:“难道你们就不知道,有人跟着你们吗?”

我话一出,自己还没回过味来,就发现盘马的表情明显松了下来,心中咯噔一下,我心说糟糕了,被揭穿了。

盘马看着我道:“虽然我不知道你是谁,不过我也不是老糊涂,你回去后不要来找我了,你什么都不知道,我也不会告诉你。”说着就要来撵我。

我迅速地回想,心说哪里被他发现了,是他能确定觉得没有人跟着他,还是当时的情况不可能被人跟?我想着怎么补救却发现没什么好办法,一下就沮丧了下来。

他的儿子来开门,意思是让我们出去,门一开光线一亮,我正想起身,忽然就发现老爹的脚,竟然有一些轻微的抖动。

我猛地看向老爹,发现他正看着我,虽然脸上镇定得一点波澜也看不出来,但是脸色坏得吓人,显然处于极度的紧张中。

我一下就明白了,他也在诈我!

我立即将我起身的起势化成一个伸懒腰的动作,然后重新坐定,用不容辩驳的语气道:“不要嘴硬,我拿事实说话,我没有多少耐心。”

盘马看着我,他儿子也看着我,我信心十足,能感觉出自己当时的表情确实阴险不可捉摸得要命。

对峙良久,盘马一下崩溃了,他低下了头,向他儿子打了个眼色,他儿子和阿贵说了几句什么,阿贵就半拉半扯地被拉了出去,他儿子进来,坐在了阿贵的位置上,门重新被关上。

盘马老爹向我行了一个十分大的礼,抬头的时候道:“不管你是谁,希望你说话算话,如果要算老账,就全算我的头上。那些人全是我杀的,其他几个人只是帮我抬东西。”

\chapter{那是一个魔湖}

我诧异于这话是什么意思,但是盘马很快就把整件事情说了出来,只听了几句,我就遍体冰凉,一下明白了死人味道的来历。但是这件事情实在太恐怖了,太出乎我的意料了,我听完之后,首先感觉到的不是疑惑,而是恶心。

我实在无法想象竟然会有这种事情,也无法理解他当时的目的,更无法想象当时的人心为什么会是这样。如果盘马说的是真的,那么他身上背负的就不是什么秘密,而是巨大的罪孽。

前面的过程和盘马说的完全一样,关键的问题就出在盘马所说的,他进山却发现考古队消失的那一次。

盘马说了谎,他那一次进山,考古队并没有消失,而且他也不是一个人进山,他带了自己的四个兄弟替他背东西,这样他们回来的时候还能打猎。

送完粮食之后,他们没有离开,因为在营地里待到傍晚可以吃到一顿白米饭,这对于他们来说简直是皇帝一般的待遇。但是考古队不允许他们待在营地的内部,他们一直在营地外吹牛打屁,要一直等到傍晚开饭。

在这个过程中,四个兄弟中的其中一个人,看着考古队的军用补给,突然起了歹心。

当时十万大山的贫困程度是现在的人无法想象的,连年的边境冲突,野兽都逃进了深山里,小孩子没有肉吃,只能吃一些米穗和野菜,都发育不良,白米饭更是当糖来吃的东西。部队的补给对于他们来说诱惑太大了,那几袋大米他们可以吃一年。

因为让村民帮忙运粮绝对会中途被掏掉一些,所以部队收粮都要过秤,如果发现少了虽不会追究但是以后就要换人。他那个兄弟就盘算着,等着他们过完秤,他们入夜睡了,他们偷偷进去,掏几碗出来,这样不会丢了活儿也能让家里人吃到甜头。

这本来是一件非常单纯的事情,盘马不同意,他的手艺好,家里算不错,没有苦到饿死孩子的分上,但是其他四个人都动心了。

盘马只得让他们去,他在外面等着,没有想到,这四个人进去后出了事。

他们从每一袋大米中舀了三碗米,出来的时候正好被一个进帐篷检查的小兵碰到了。那时是军事状态,人的神经都是绷紧的,小兵马上举枪,但是他没有看到躲在他身后还有一个人。情急之下,后面的人一下把小兵按住,他们四个人用米袋把小兵活活给捂死了。

杀了人之后,四人怕得要死,杀人罪,特别是杀军人,如果让人发现,肯定直接就枪毙。他们逃出去,和盘马一说,盘马立刻心说糟糕了。

这件事情他无论如何也脱不了关系,因为考古队请的是他,而几个兄弟是他请来帮忙的,所有的责任他一分都逃不掉,而且在这种敏感时候,说他没参与也没有人会信。

他当即想了一个办法,必须把那小兵的尸体从里面拖出来,当成失踪,否则他们肯定会被调查。

他们潜回去,把米全部还上,然后把小兵的尸体拖出了帐篷,结果没拖多远就被放哨的人发现了。放哨的人一路追过来问他们在干吗,盘马他们一时慌神之下尸体就被看见了,哨兵立即举枪,但是当时提出偷东西的伙计早就准备好了,一下就把那人的喉管割断了。

几乎没有什么考虑,他们走火入魔般连杀了两个人。盘马一下感觉事情已经完蛋了,说逃吧,但是杀人的那个兄弟却杀红了眼,说已经杀了两个人,杀两个是杀,杀光也是杀,如果让他们回去通报军部,我们这辈子都要猫在山里了,与其如此,我们把这些人都杀了,就说他们不见了,其他人肯定认为是越南人干的。

这是在一种诡异的气氛下突如其来的冲动,考古队的人数不多,那时大部分都在酣睡,想到那些白米、冲锋枪和之后的事情,盘马竟然也无法抑制地起了歹念。

之后的过程让人恶心,他们拿着冲锋枪和匕首,偷进一个又一个帐篷,把里面的人全部杀死了。

杀完人后,他们把尸体、枪和弹药,还有物资全部都抛入湖中,把白米和吃的偷偷背回了村里,藏在床下。一些他们能用的,但是背不动的日用品等东西也藏了起来,等风平浪静后再拿,同时几个人约好,以后决死不提这个事情。

盘马当时心虚,思前想后的,就开始在村里宣称考古队都不见了的怪事,想为以后的事情做一个铺垫。因为当时边境冲突频繁,有队伍在越南边界失踪,一般都会认为是越南特工干的。

几个人认为万无一失,谁也没有想到,这却是他们噩梦的开始。

三天后,盘马再次进山,回到了湖边,想去那些东西里翻翻,先把值钱的东西拿回去。那一晚的疯狂让他心有余悸,所以他先是远远地看了一下,让他毛骨悚然的是,他竟然看到湖边又出现了一个营地,竟然还有人在活动。

有其他的军队?尸体被发现了?他胆战心惊,好久才缓过来,等鼓起勇气偷偷靠近去观察的时候,他却瞠目结舌,发现之前的考古队竟然又出现他在面前。

盘马完全不知道自己的感觉,他有点闹不清到底是怎么回事,看着在营地中忙碌的那些人,好像身在幻影之中。那些人似乎根本不知道之前发生的事情,纷纷都和他打招呼。

他以为自己在做梦,捏了好几下才发现都是真的,那些脸虽然不熟悉,但都是考古队里见过的,他甚至看到了几个亲手被他勒死的人在那里谈笑风生。

他仓皇赶回到村里,失魂落魄,急忙把事情和其他人一说,他们去看了之后发现果然如此。他们都吓坏了,琢磨这到底是怎么回事,难道那是一弯魔湖,能让里面的死人复活?

但是那些人都是活生生的,一点也不像僵尸。

盘马百思不得其解,村里人很迷信,觉得这一定是山神湖鬼在作怪,吓得魂不附体。盘马琢磨了很久,鼓起了勇气,再一次回到湖边给他们送粮食,试探性地问起了那一天的事情,然而,所有人都回答没事,那表情没有任何异样。

一天好像就被翻过去了,天神把那一天的事情全部抽走了。或者是,那几个行凶者在当天都做了一个同样的梦,他们根本没有去杀人。

盘马并不是一个就此认命的人,他不相信自己是做了一个梦,但是他又怎么想也想不明白,之后一直留心着这一批人,想知道他们到底是人是鬼——可是,无论怎么看,他都看不出一丝破绽来。

唯一让他感觉到有点奇怪的是,他闻到那批人身上,出现了一种奇怪的味道,是之前没有的。

\chapter{中邪}

那种味道,就是盘马从后来的盒子里闻到的味道,只不过盒子里发出的更加的浓烈。

对于盘马来说,那就完全是死人的味道。那些不知是人是鬼的恶魔,他们身上的味道肯定是从地府里带出来的。

“你的那位朋友身上,也有那种味道,如果不是被草药的味道盖住,我第一次看到他的时候,就会闻到。”盘马老爹看着我,“他和他们一样,也是湖里的妖怪!”

闷油瓶身上有什么味道?我对味道这种东西不是很敏感,我也不是猎人,没有极好的嗅觉,所以对此半信半疑——下次要偷偷去闻一下。

如果事情到此为止,也许这事就会过去,过上一段时间,人会自己怀疑自己的记忆,对于没有解释的会自动抹掉。但是,我知道事情肯定没有结束,因为光是这样,盘马老爹不会得出闷油瓶会害死我的结论。

果然,盘马继续说了下去,他说之后发生的事情,让他一辈子都无法忘记这种味道。

这件怪事发生之后,盘马老是感觉心神不宁,虽然那些人似乎和之前一模一样,但是,盘马总感觉他们的眼神和神情有一丝妖异,这种感觉没有任何事实依据,完全是一种心理作用。盘马有一种预感,村里会出事情。

几天后,村里发生了一件事,让他开始毛骨悚然。

和他一起行凶的,还有四个人,他们说起来都有血缘关系,远近略有不同,其中一个人叫做庞二贵,胆子最小,忽然就不见了。盘马和其他几个人心里有秘密,一下心就提了起来,谁也不敢说。村里人去山里找了两天,最后,盘马他们硬着头皮回到湖边,竟然发现那个庞二贵在营地里,和那支考古队里的人谈笑风生。

他们莫名其妙,把他领了回来,盘马拉住他的时候,就闻到从庞二贵的身上,竟然也传来了那股神秘的味道。

盘马看着庞二贵大白天就开始起鸡皮疙瘩,他一下就感觉庞二贵的表情和以前不一样了,好像变了一个人。

那种恐惧是无法形容的,他感觉庞二贵肯定被鬼迷了,回到村里,他叮嘱了庞二贵的媳妇,让她如果发现她男人不正常,立即和他说。

但是她媳妇没有机会去发现了,第二天,他媳妇起来后就发现庞二贵吊死在床边上。整个屋子里,弥漫着那股奇怪的味道。

村子里以为是庞二贵想不开,或者是被狐仙迷了,盘马心里明白,惶恐不安的他更加确定那些人是妖怪,肯定是庞二贵中了邪了。

庞二贵的媳妇被吓坏了,再也不敢住那个房子,搬回了娘家,那房子就荒废了下来。其他几个人吓得要命,两个搬出了村子,盘马和另外一个留了下来,晚上根本都不敢睡觉,借了好几只狗,唯恐下一个就是自己。

但是狗也没有用,一个星期后,和他一起留下的另一个人也失踪了。两天后,一个小孩在庞二贵家废弃的房子里发现了他,他吊死在和庞二贵一样的位置上。

盘马生性刚烈,自小和大山为伴,所以非常的坚强,恐惧到极点之后,他反而豁出去了,带着枪就赶向湖边,心说反正是死,死也要死个明白,绝对不会坐等。但是他进山之后,正巧考古队开拔。

盘马是在半路上遇到的队伍,似乎他们不再需要向导,盘马之前已经想得很决绝,但是一见到他们一下就软了,他胆战心惊地随着队伍出了山。

如盘马之后所说的,考古队带着散发出奇怪气味的盒子离开了村子,再也没有出现,一直到现在。逃到另外两个村的人没有出事情,盘马胆战心惊地过了一年,才逐渐放下心来,相信他们真的走了。

这一件事犹如噩梦一样一直缠绕着盘马,那种恐惧我可以想象。军队走后半个月,为了弄清到底发生了什么,他再次回到了湖边。绕着湖边走了一圈,他发现了有一件衣服不知道怎么被冲到了岸上,在那件衣服里,他发现了那块奇怪的铁块。

这块铁块的发现,让他肯定了这些人肯定是从湖里爬上来的,因为铁块在衣服里,绝不可能被湖水冲到岸上。那块铁块散发着让他毛骨悚然的味道,他自觉非同小可,所以一直放在身上。早年生活贫困的时候,他想把它卖掉,现在生活逐渐好起来了,想起当年不禁有些后怕,就想保住这个秘密,带进棺材算了。

之后,我们出现了。

盘马的秘密,到此就结束了。

听完之后,我陷入了久久的沉思中,少有的,我没有感觉到更加的迷惑,我第一次感觉到,我似乎找到了一条链条,能把我心中的疑团串联起来。

这些谜团都好比一根根双头的螺纹钢管,连接的地方都是一个疑团,但是把其中两个疑团连起来,那么四个谜团就会失去两个,把所有的钢管连接起来,那么这么多谜团,可能只剩下首尾的两个。所以疑团一个一个连接起来,让人很有快感。

如果是以前的我,我一定会抓狂,但是现在我学会了不去看问题的本身,我清楚地意识到了这件事情的真相,这件事情需要去求证,如果我的想法是正确的,那么,三叔,或者说解连环一直疑惑的问题,就有了答案。

而要求证这件事情,必须要到那座湖边去。

盘马老爹拿出了那块铁块给我看,那东西果然和闷油瓶床下发现的那块一样,同样的铁疙瘩,上面有着古朴的花纹,不过盘马的这一块略大。我特地闻了一下,果然闻到了一股奇怪的味道,非常的淡,几乎无法分辨。老爹说,刚发现的时候味道很浓,逐渐的,这味道一点一点消失了,铁块放在家里,家里什么虫子都没有。

我对这东西暂时失去了兴趣,心里充满了我的推测。

盘马不肯再去那个湖边,我想着让阿贵另找向导,把钱给了盘马,便起身告辞。

到门口的时候,我忽然想了另外一件事,回头问道:“对了,老爹,你身上的文身,是怎么来的?”

盘马看着我,有些诧异我忽然问这个,他的儿子替他解释道:“这是防蛊的文身,是小时候一个路过的苗人巫师替他文的。当时我的爷爷救了他的命,他给我爹文了这个答谢,据说有这个文身,到了苗寨可以通行无阻,没有人会为难你。”

\chapter{计划}

阿贵一直在门口等我,蹲在地上郁闷地抽烟,显然不知道盘马他们在搞什么鬼。见到我后立即站了起来,我对他道:走,咱们回去。

在路上我问他,知不知道盘马说的那个羊角山的湖泊?阿贵点头,说以前听说过,不过他自己没去过。我道我出高价,帮我尽快找一个猎人,带我们过去。

阿贵满口答应,试探性问我,盘马到底和我说了什么?不过阿贵问得很小心,我心说告诉你就是害了你,随口便敷衍掉了。

急匆匆回到阿贵家里,我着急想把我的发现告诉闷油瓶,却发现家里只有云彩和她的姐姐在烧灶台,胖子和闷油瓶都不在。

我心说奇怪,问云彩人呢?云彩道那位不怎么说话的老板回来后看到胖老板还没回来就问我,我告诉他胖老板一晚上没回,他就急匆匆去找了。

我本来心里很兴奋,一下子兴奋劲就压了下去,心说胖子一晚上没回来?

山村不像城市有娱乐场所可以让他去逍遥,他一晚上没回来有点不正常。我对胖子的秉性很了解,想到他之前说的要去弄点硫酸的事情,一下就有不祥的预感。

相信闷油瓶和我一样,也立即想到了这个可能性,所以才会立即去找。

我马上让阿贵带我去村里的村公所,如果胖子有什么意外,肯定会在那里。走出去没几步,却正碰见胖子和闷油瓶回来了,胖子脸上还蒙着纱布,一边走一边骂,好像受了伤。

一问才知道原来胖子买硫酸回来的路上,看到一只马蜂窝,来了兴致,结果错误估计了自己的身手,中弹了,而且还挺严重,在村公所挂盐水,结果睡了一晚上。胖子说这里的马蜂和他以前碰到的不一样,之前他碰到的马蜂都是捅了才发飙,这一次他才靠近马蜂就突然围了过来,凶得不得了。

我说你别找客观原因,你得承认你就是老了,老胖子不提当年勇,捅马蜂窝这种事情你以后还是少干,免得别人笑话。

回房给胖子换药,换药显然极其疼,要不是为了在云彩面前表示自己的男子气概,他肯定叫得像杀猪一样。

云彩倒是很镇定,蜻蜓点水一样在他脸上消毒,我发现他的下巴上有几块指甲大的地方全肿了,云彩用竹签子先把肿的地方划破再上药,那简直就是活剔肉,难怪疼死他了。

弄完后胖子吃饭都艰苦,好不容易吃完饭,天色暗了下来,我们在高脚楼延伸出的走廊上乘凉,我把在盘马家听到的一切全部复述了一遍。

听完之后,两个人都皱起了眉头,胖子问道:“还有这种事情,娘的这都赶上我小时候吓唬姑娘家的鬼故事了,这事情能是真的吗,你说你的假设是什么?”

“我认为,盘马绝对没有说谎。”我道,“这件事情绝对是真的,但是,他的真,不是那种意义上的真。”

“你是什么意思?”胖子道。

“咱们考虑最合理的可能性,不去考虑什么魔湖啊,妖怪啊,你觉得这件事情最可能的情况是什么?”

胖子摇头道:“少来这一套,我的脑细胞全给马蜂叮死了,我不来猜你的,你直接说就是了。”

我苦笑,好容易想表现一下,胖子还不配合,道:“好,咱们把一切不可能的因素都去掉,没有什么有魔力的湖泊,没有什么死人复活,也没有妖怪,但是事情必须是合理的,盘马说的话必须成立,那么这件事情唯一的可能性其实很明显——人不可能复活,那么进山的考古队和出山的考古队,就肯定不是同一支队伍。”

胖子顿了顿,领悟道:“你是说,死的人没复活,走出来的,是另外一批人?”

“盘马他们杀了的那一批人,确实是死了,盘马并不了解那支队伍,如果有另外一支队伍易容之后,我觉得并不需要多么高深的化装,就可以骗过盘马。”

“可是,为什么他们要这么干?这不是耍他嘛。”

“我仅仅是推测,通过那支队伍的情况和盘马的情况,我感觉这事可能有些误差。咱们假设这是一场蓄谋已久的阴谋,那么,可能计划中,就在盘马杀死考古队的那一天,这一支考古队就已经被设定会被抹掉,但是,这个计划可能出现了偏差。也许来杀死考古队的杀手,在林子中遇到了什么意外,没有到来,反而由盘马完成了这个任务,之后替换的冒牌队伍来到这里,以为是杀手完成了任务,于是就按照计划开始了伪装。那么,不知情的盘马才有了魔湖一说。”我道,“这是一种合理性的推测,事实可能完全不是这样,但是这证明了有可能这事会出现。”

“哎,这个听上去好像有点靠谱,不过胖爷我好像在哪儿听过这样的桥段?”胖子道,“你有什么证据?”

“只有一些细节,比如说,考古队是盘马带进去的,但是出来的时候,并没有等盘马进来带他们出去,而是自己出发了。说明后面的队伍,熟悉这里的地形,他们有出去的本领。之后发生的事情,可能是因为考古队发现了什么蛛丝马迹,对庞二贵他们进行了杀人灭口。”我道,“我现在不知道是否这一考古队就是去西沙的那一支,但是我感觉,即使不全部是,肯定其中也有几个人是。如果是这样,那么你说会不会,有人为了进这个考古队去西沙,而进行了这一次调包。”我的思路很成熟。

胖子道:“他娘的,但是你怎么证明呢?”

“最直接的方法,咱们应该去羊角山的那个湖里看一下,现在湖变小了,我觉得可以潜水下去看看下面有什么,有没有当时抛入湖中的尸体。”

“他娘的这个有点困难吧,现在快过了四十年了,有尸体也早就烂没了。”

“骨头肯定还在。”我道:“盘马他们没有船,抛尸的地方肯定是湖边,我觉得我们可以去碰碰运气。”

\chapter{似曾相识}

胖子觉得我的说法很玄乎,但是也承认这是事件合理的唯一可能性。他本来就是羊角山一日游的积极分子,如此我一说要去,自然是满口答应。

接下来我们商议了一些具体事项。因为这一次是旅游性质,什么装备都没有带,所以有点棘手,万一碰到有开棺掘冢之类需要家伙的事就只能干瞪眼。

地方偏僻,在这种地方也不可能买到现成的装备,胖子说道,有些东西倒是没有必要,咱们可以买点替代品,虽然用起来不会那么称手,但是这一次离村子还算近,对质量的要求也不用太高。

他说的是野外生存用品,猎人有自己的一套,肯定不需要我们背着固体燃料和无烟炉,不过见识了野兽的剽悍,我觉得武器还是要准备一些的。

把阿贵叫来和他商量这些事情,阿贵自己也打猎,有三把猎枪,都是被改装过的不知道名字的老枪。三把枪年代就不同,最老的一把是阿贵从鸡棚里拿出来的,虽然枪管子的成色还可以,但枪膛里头全锈了,谁也不敢用,也没处去找火药去。另外两把都是打子弹的,看得出是战争年代留下来的。

前几年禁枪,但是这里的人都靠打猎为生,吃饭的家伙当然都不肯交出去,上头也知道情况,睁一只眼闭一只眼,就是现在子弹不好弄,阿贵说得村干部去县里批才买得来。

阿贵自己打猎已经属于业余活动,所以家里存弹不多,胖子把两把枪检查了一下,道:“阿贵的那把绝对没问题,另一把太久没用了,但是枪保养得还可以,要开一枪才知道还能不能用。”

我们以五十块一发的高昂价格,在阿贵隔壁几户邻居那里买来了五十发子弹,我看那黄铜的圆柱状子弹就知道是小作坊里手工做出来的,这东西要五十块他娘的有点让我心疼。胖子说别这么小肚鸡肠,五十块钱可能就救了你的命,绝对值。

开山的砍刀阿贵家就有,阿贵特地去磨锋利了,其他的东西我们写了条子,让他去乡里看看有没有替代品,没有爬山的绳子就用井里的麻绳,没有大功率的手电就拿几只手电捆起来用,没有匕首就用镰刀。

阿贵对我们建议道,现在雨水多,山里蚊虫毒蚁也多,特别是湖泊边上,蚊子都跟马蜂一样大,要带蚊香和蚊帐,把蚊香甩在篝火里,否则我们几个城里人肯定吃不消。我心说有闷油瓶在,这个不需要担心。

安排妥当,阿贵说那些东西得一两天时间准备,反正打猎的人也都没回来,他准备好了再出发。

在此期间,胖子说可以想办法用他带回来的硫酸,看看那铁块中包着什么东西,这需要精细的操作,要挑一个好一点的场地。

我想起盘马的叙述,觉得不妥当,这铁块中散发出一股气味,而且这气味随着时间的推移逐渐变淡,说明里面有一种挥发性的物质,鬼知道这种物质对人体会不会有害。我觉得要溶开这东西的时间未到,到了那边,查到一些蛛丝马迹之后,再判断是不是要冒这个险比较靠谱。

胖子的好奇心烧得他受不了,但是我说的绝对有道理,闷油瓶也同意我的看法,想到可能连累到其他人,他也只好作罢。

接下的时间胖子兴致勃勃,一是他的古墓说他深信不疑,二是他很久没打猎了手痒得厉害,一晚上不顾脸肿得像被马踢过一样,一直和我们唠叨他以前打猎的事。我也睡不着,但脑子却想着湖边的事情,闷油瓶一直没有说话,我看他一直看着阿贵隔壁的楼,看着那个窗户出神。

我想起前天晚上在那个楼里看到了影子,不过现在那个窗户里一片漆黑,什么也不看见,阿贵的儿子似乎不是很愿意见人,深居简出的。我怀疑是不是有什么疾病,所以只能待在家里。农村里经常有这样的事情。

一个晚上没睡,加上一天剧烈的思想活动,很快我就晃神听不清胖子在说什么,闷油瓶靠在那里打起了瞌睡。在这里外面比屋内凉快得多,闷油瓶在四周一只虫子也没有,我们就这么躺下睡着了,醒来已经是第二天的中午。

这一天各自准备不说,第三天准备得当,阿贵带我们出发。

让我郁闷的是,我没有看到传说中的向导,一起出发的竟然是阿贵自己和云彩。

我问怎么回事,阿贵你不是说你没去过吗?怎么是你自己带我们去?

阿贵道这猎人进了山里,不知道是不是遇到了什么阻碍,几队都没回来,其他人都没去过,他能找到的人就是他女儿云彩,云彩以前跟着爷爷去过那里几次,知道怎么走。他带着我们,加上云彩认路,还有狗,问题应该不大。否则我们几个语言不通,恐怕会出麻烦。

我心说糟糕了,看来我出价太高了,阿贵舍不得让别人赚这个钱了。胖子立即说不行,咱们是去干事,带着个小丫头这不开玩笑嘛,要是受点什么伤的,你这个当爹的不心疼我还心疼呢。

阿贵一个劲说没事,这里的小丫头片子也都是五六岁就摸枪了,要论在山里,她比我们有用,而且这山她比他都熟悉,不用担心。

说着云彩就从屋里出来,我和胖子一看,眼睛都直了。只见云彩完全换了一个人一般,一身的瑶族猎装,猎刀横在后腰,背着一把小短猎枪。瑶族姑娘本来身材就好,这衣服一穿,那小腿和身上的线条绷了出来,真是好看得紧。加上英姿飒爽中带着俏皮的表情,带着十七八岁年纪那种让人不可抗拒的味道,一下子就把胖子给征服了。

她走到我们边上,挑战似的盯着我们,道:“几位老板,瞧不起人是不是?”

“没有没有!完全没有!”胖子立即道,“大妹子,你不要误会,你胖哥哥我主要是怕你辛苦,其实在我们心里,你绝对是最佳人选。”

我立即皱起眉头,踢了胖子一脚,低声骂道:“你怎么变卦得那么快,怎么着,就你这年纪了,还想老牛吃嫩草?”

“我年纪怎么了,胖爷我这说起来叫做人到壮年,是壮牛,不是老牛。”他低声道,“你都让潘子去找个婆娘,怎么就容不得我?”

我也不知道他是真的动了心还是只想吃点豆腐,对他道苗瑶一家,女家都厉害,你小心人家真动了情把你下蛊绑了,那你就得上门在人家家里种一辈子田,如果变心逃跑,一发蛊那就是万虫穿心,一身的神膘都喂了蛊虫。

胖子显然见多识广,不以为然,说牡丹花下死,做鬼也风流,最好全瑶寨的美女都向他下蛊,那他就留在这里做村长。

嬉笑中我也只好接受了这个现状,看云彩那种气度,我感觉阿贵说的没错,而且这一次估计不会有太大的危险。

唯一让我在意的是,我们打包东西的时候,胖子老是找云彩调侃,把云彩逗得哈哈笑。但是我能看出来,云彩时不时偷偷看着闷油瓶,看得很小心,总是看一眼立即转回眼神,但在那清澈的眼睛里,我是能看出一点东西来的。

我们按照当时找盘马老爹的路线原路出发,对于这路线我已经有少许了解,一路比晚上搜索盘马老爹时轻松多了。胖子简直是被迷住了,围着云彩就转,就差趴下来给她当马骑了,云彩也确实可爱,蹦蹦跳跳的。

她问我们到底是干什么的,肯定不是导游,哪有导游会到这种地方来的,胖子故作神秘,说我们是有秘密任务的大人物,如果她肯亲他一口他就偷偷告诉她。

我还真怕云彩亲他,那太浪费了,还好云彩还是有审美能力的,坚决不上当。不过闷油瓶没有为我们的气氛所感染,他的脸色一直没有任何变化,在轻松的气氛中,只有他仍旧沉在阴云里。

当天晚上到了山口的古坟处,我们深入进去一两公里稍事休息,天亮后继续,在山中走了两天,才来到了那处湖边。

远远我就在山脊上看到了那湖,大概是连日暴雨的缘故,湖泊比我想象的要大一些。果然如盘马说的四周全是石头,湖四周是莽莽群山,高大陡峭的山峰连绵不断,山体巨大入云,一点也不像丘陵,完全是险恶的大山大水。山中植物分布得非常厚实,连山间的断崖都是墨绿色的,十万大山果然名不虚传。我不由得庆幸,此地离村子尚且不远,再往里走,这深山中的腹地恐怕比塔木坨还要险恶。

经过一条已经完全被植被覆盖不可见的山路,我们来到湖滩上,完全看不出当年这里有人驻扎过的痕迹。湖水非常清澈,倒映着天空中的云彩相当漂亮,甩掉包裹,我们到湖水里去洗脸,水是凉的,说明湖底通着地下河,在三伏天里冰凉的湖水让人精神一振。

洗完脸我仰头看向四周,湖水倒映着天空和四周的山,忽然就发现这里似曾相识般熟悉。我看了一眼,边上的闷油瓶也是一脸的疑惑。

\chapter{脑筋急转弯}

这种一刹那的熟悉感以前我也有过,每每都让我起了一身鸡皮疙瘩。书上说这是一种错觉,但是这一次却不同,因为我看到闷油瓶的脸色也起了变化,同样一脸疑惑的表情,不知是否和我是同样的感觉。

是哪里呢?我在哪里看到过这里的情景,或者是看到过与这里类似的情景?

我努力回忆,从脑子里翻来覆去思考,但是想不起来,只记得这情景我应该刚看到不久。而且,与这种相似的感觉一起来的,还有一种“不对劲”的感觉。显然我记忆里的印象,和这里仍然有少许的不同。

胖子没心没肺,直接脱得只剩下裤衩就在水里游泳了,阿贵让他小心点,山里的湖里都不吉利,不要太折腾。胖子什么场面没见过,朝阿贵泼水让他闭嘴。

回到岸上,我们脱掉了湿掉的鞋和裤子,胖子帮阿贵搭起了雨棚,阿贵去砍柴,云彩帮忙烧饭,我喝着水,这才想起这山势在哪里见过。

这山的形状和感觉,竟然和我们在村子溪边戏水时看到的山景非常相似,山的线条、走势,都如出一辙。只不过当时我们是在溪涧里,现在我们是在湖泊里。所以这水里的倒影和山的样子,一下让我吃了一惊。只不过这里的山上树木茂密,而在寨子边上,树木都被砍伐过了,所以才有少许的异样。

我闭目养神的时候,仔细观察过溪涧四周的风景,闷油瓶别看心不在焉的,一切他肯定也看在眼里,胖子的注意力在当时那些小姑娘身上,难怪不察觉。

这还真是有趣,大自然真是鬼斧神工,不知是纯粹的巧合,还是因为什么地质原因形成。好像有一种风水地势就是如此,这种地形叫做“鱼鳞岙”,所有的山好像鱼鳞一样,一层一层的,山势都十分的相像,这种风水不适合葬人,因为据说鱼鳞下是藏污纳垢的地方。从地理上说鱼鳞状的特别容易水土流失,也是积水特别严重的地形,我们在山口看到的古坟就一个例子。不过,如果在“鱼鳞岙”里有一泉湖,那就完全不同了,那叫“鱼来自得水”,水在鱼鳞里,出水而不亡,那这就不是鱼,而是一条未化的小龙,如果有早亡的年轻人,应该葬在这里。

如此说来,这里有个古墓的可能性真的很大,可惜我不知道这种山势的殓葬细节,在我看来四周的山上都不是很适合葬人。

云彩他们搭完窝棚,开始收集一边的柴火,我和胖子、闷油瓶不需要帮忙,开始环湖搜索大概观察四周的环境。

湖泊只剩下两个足球场大小,一下就走完了,我走在岸边看着湖内,感觉湖底似乎也全都是石头,而且湖底的落差很大,稍微浅一点的地方能看到水底,再往下湖底就迅速隐入了黑暗,看来水下可能极深。湖滩上全是大大小小的石头,如盘马所说大小差别很大,让我在意的是,湖滩非常干净,什么杂物都没有,也许是被连日大雨冲进湖里了。

我对于极深的湖泊总是怀有一种莫名的恐惧,俗话说浅水不藏龙,水深必有怪,水一深代表湖的容纳范围没有我们从湖面上看到的那么小,就有可能有一些奇怪的东西在里面。世界上很多有水怪的大湖,湖面不大但都极其深,即使没有什么古怪,水极深的地方也容易有一些大鱼。有些大水库清库底的时候,总会发现一些长得无比巨大的鱼。

绕了一圈没有看到明显的尸骨痕迹,不过湖滩大部分石头都很细碎,四十多年来这里水位不断变化,山石不断滚落,那些尸骨也许被压在了石头的下面。

我们判断着当时的过程,按照一般的情况考古队应该和我们一样扎在湖的南面,另一面是山,会有落石和泥石流的危险,那么我们要搜索的区域应该是湖的南面。

这是个大工程,还好带了几只狗,不过也不知道能不能派上用场。尸体被水泡了这么多年,肯定白骨化了,和石头不见得有什么区别。

吃过中饭阿贵去四周转转,看看有什么东西好打,我们开始划区域寻找,云彩给我们洗汗臭的衣服。湖边的区域很大,我和胖子、闷油瓶三个人每人一大块地方开始了行动。

我们要做的就是徒手把石头一块一块搬开,这里石头的情况,应该是离岸最近的不停地往湖中心滚落,但是这里的水位是逐渐下降的,而且石头累积本身就有防雨水冲刷的作用(雨水会浸入石滩下层汇聚成地下水,而不会在石滩上形成水流,都江堰的一部分就是这种原理)。湖底的坡度很陡,当年盘马不可能走入湖中太深,那么抛尸的地方肯定离岸很近,而且水位下降了很多,尸骨不会在湖里,而是在岸上。

胖子说尸体丢下去后如果没有什么东西捆扎,会先变成浮水尸,然后沉底被鱼虾吞食,骨头应该是散的,脑袋在这里,屁股可能就在一百米外,这么找肯定找不到。而且如果尸体没有被抛入很深的地方,那么也有可能被动物拖上岸分食。

我道无论怎么说,不太可能一点蛛丝马迹都不剩下,毛主席说过,世界上怕就怕认真二字,咱们先找着,真找不到再来分析原因。

三个人就这么一直翻到夕阳西下,仍然没有结果,几只猎狗在湖边嬉戏,完全不理会我们,也不想帮忙。湖边的太阳很毒,晒了一天,我的天灵盖都火辣辣的痛。阿贵的枪在林子里响了两声,带回来一只野鸡,很快烤鸡的香味就让我们按捺不住了。

胖子不禁有些沮丧,我们休息的时候靠到一起抽烟,胖子就说看来够戗,你还是看看这里什么地方可能有肥斗比较保险,死人可能找不着了。我知道他惦记着他的古墓说,安抚他道反正要待好几天,慢慢来吧,真要找不到死人,我就替他去找那肥斗。

难得我心中没有多少急躁,喝了点米酒,我们围在湖边的篝火旁休息,既是湖边又是山中,凉爽得要命。云彩也换了衣服,穿了轻薄的T恤,洗了头感觉和城市里的女孩很像了。吃了饭她还跳舞给我们看,瑶族的舞蹈有很多转圈和后踢小腿的动作,瑶族姑娘的小腿又特别的好看,胖子看得下巴都掉了下来,一定要去学,但是他完全像跳大神,我笑得人仰马翻。

太久没有笑得这么舒畅了,我最后都笑不动了,但是转眼看到闷油瓶,却见他靠在石头上,一点放松的表情都没有。乍一看都察觉不到他的存在。

我心说到这里来找他的过去也不知是不是一个错误,就目前收集到的线索来看,显然策略上我们是来对了,对于我们来说,这一路过来是轻松的,但对他来说,遇到的东西无一不是在敲击他过去的心门,让他轻松起来真的很难。

这人又是典型的自我放逐型人格,心在桃园外,兀自笑春风,谁也进不了他心里。

想想有些不忍,我拿了一块小石头丢他,对他道:“别琢磨了,告诉你,我有经验,怎么琢磨都没用,咱们现在做的就是拼图,在所有的片找得差不多之前,少琢磨一些。”说着递给他米酒。

闷油瓶默默接过,放到一边,我有点多了,叹了口气道:“你就不能喝一口?”

他摇头,看向一边的黑暗。

我只得把注意力转回到胖子身上,胖子正出脑筋急转弯给我们猜,问云彩,什么战斗是:杀敌一百,自损三千?

我怕胖子出黄色笑话给小姑娘猜,小姑娘很纯啊,这种东西感觉说出来都是污染,就喝了他一下。胖子说放心吧,这个脑筋急转弯绝对正经。

阿贵也喝多了,咯咯直傻笑,猜来猜去都不对,最后答案公布,原来是屁胡和十三幺的战斗,打麻将放炮,赢下家一百,但是输给中炮三十番。

瑶寨里不兴这个,云彩根本听不懂,我骂道你这不是欺负人吗?有没有乡土气息一点的脑筋急转弯。

胖子就道有,问我们道:再猜,什么战斗是杀敌一个,自损三千的。

“马蜂!”云彩立即举手道。

胖子啧道:“臭丫头,你存心刺激我是不是?”

我们大笑,我说那肯定是骑兵和坦克的战斗,胖子道如果是骑兵和坦克,自损一万都杀不了一个。

接着我们猜,有猜打扑克的,有猜蚂蚁的,有猜吃鲍鱼的,胖子都说不对,得意扬扬,好像在凌辱我们的智商。

我怒道,你他妈的说那是什么战斗?如果牵强我就揍你。

胖子道:“这个太容易了,哎,胖爷我真是天赋异禀,和你们这些凡夫俗子怎么都有差距,我告诉你你听好了,杀敌一个,自损三千,是香蕉和大象的战斗。”

我听了大怒,骂道,你胡说什么,香蕉和大象的战斗,这是什么玩意儿,你倒说说香蕉和大象打怎么可能杀敌一个,自损三千?

胖子道:“大象被撑死了呗。”

我们一下笑成一团,云彩都笑得无法呼吸了,但是笑了几声,我们就慢慢收敛了下来,因为我看到闷油瓶在我们人仰马翻的时候,默默地站了起来,往湖的方向走去,然后远远地坐在篝火勉强能照到的地方。

云彩的眼神里有一丝惶恐,她看了看我们:“他是不是嫌我们太吵了?”

胖子叹了口气,吸了一口黄烟叶,安慰道:“没事,别理他,他是去拉屎。”

我看着闷油瓶,刚想站起来,云彩却抢先朝他走了过去。

\chapter{虹吸效应}

云彩坐在闷油瓶身边,远远的也不知道有没有和他说上话,胖子直直地看着,我调侃道:“你失恋了,节哀顺变。”

胖子不以为然道:“你不是也一样!”

“一你妈个头!”我怒道,“我可没你那么变态,我对小女孩没兴趣。”

胖子拍拍我:“我相信小哥,绝对是够义气的人。”说着把酒递给我,自己也起来放尿。很快后面传来长篇大尿的水声,源源不断,也不知道他憋了多久。

我不禁莞尔,笑得也累了,静下来,看着远处月光下的湖面,忽然感觉来这里也许是一种缘分。

独看这里湖光山色,谁能想到当年发生了那么诡异的事,又看我们笑声豪迈,谁又知道其实我们背负了这么多东西。世界上的一切都很简单,而人似乎是最复杂的,这种复杂又是他们抗拒却又逃避不了的。

庸人自扰,都是庸人自扰。我闭上眼睛,深吸了一口气,想自己以前的那种心境,又想想现在的这种心境,觉得以前那个在那么多谜中到处碰壁的形象真的有点可笑。

胖子放完水,哆嗦着走回来,看云彩还在那边,就奇怪道:“那丫头还没碰一鼻子灰回来?毅力可嘉啊。”

我道:“别说,也许小哥正喜欢这种类型的呢,他们现在都在交换定情信物了。”

胖子说道:“那不成,他们离我们这么远,万一有个妖怪什么的从湖里出来把他们拖了去,我都不好救,我去保护他们一下。”说着就要过去。

我拉住他,说不要打扰了,闷油瓶现在可能已经很烦了,他现在肯定满脑子都是问题,这种时候我也经历过,让他一个人待着比较好。你仔细听听,云彩也没有说话,说不定只是陪着他看天。

胖子坐下来,仔细听了听,却听到一边云彩正在唱歌。我和胖子都静了下来,微弱的湖风带来了轻灵的歌声,是瑶族的歌曲,唱得很轻,但是很清晰。

再没有人说话,我心说云彩这丫头真不错,于是坐下来,看着天上的繁星听了下去。

天上薄云飘过,我的心境很快如湖水一般平静,慢慢地,在空灵的歌声中我进入了恍惚的状态。

迷迷糊糊的,不知道过了多久,忽然歌声就停了,一下我心境动荡了一下,睁开了眼睛。一边的闷油瓶已经站了起来看着湖面,一边无聊地趴着的几只狗也都抬起了头看着相同的方向。

胖子还在闭目养神,阿贵也感觉到了异样,我拍醒胖子,就听到风从湖面的方向带来“吧嗒吧嗒”的声音,好像有好几只脚掌很大的腿,正在湖泊的浅滩上往岸上走来。

狗全都站了起来,警惕地盯着那个方向,这些猎狗训练有素,没有一只发出吠叫。胖子和我对视了一眼,我朝他龇牙,他指了指一边的手电,让我递给他。阿贵却一边让我们安静地坐下,一边摆手让我们别紧张,他轻身道:“没事,好像是野兽在舔水。”

“是什么野兽,听动静个头挺大啊。”胖子轻声问。

阿贵拿起猎枪,让我们待着别动,赤脚往黑暗中摸去。云彩跟在后面,胖子一看要打猎了,立即按捺不住,给我们打了个眼色,我也想去看看,于是隔了几米,偷偷尾随过去。

走到闷油瓶边上,依稀看到一些湖面的情况,我们寻找想象中的野兽,但是没找到。可能这只野兽只是喝水的动静大,个头不大。我们用手电扫射,循着声音寻找,找着找着,却发现这种声音来自四面八方,而且有节奏,不像是动物发出来的。

“不是野兽,是什么声音?”胖子自言自语。

“潮声。”闷油瓶道。

我们面面相觑,这么小的湖会有潮水?难道今天的月亮特别大?抬头看看,月亮根本看不清楚。

阿贵放下枪,我们朝湖边走去,走到吃水线附近,果然,湖水在有节奏地波动着,像海浪拍打沙滩,不过幅度不大,那动物舔水的声音,是水撞击石头发出来的。

我看着脚下的石滩,发现水位下降了,脚下都是湿的,也就是说刚才我们吹牛打屁加上云彩唱歌的时间,这湖泊的水位就在不停地下降。从湿线开始一直走到水边,我发现起码有十几步,水位降得很厉害。

“怎么回事?难道湖底漏了?”胖子搭手眺望。

我对地理很熟悉,知道这是一种地理现象,对他道:“这大概是虹吸效应。”

“虹吸是什么?虹吸二锅头?”

“这湖看来确实和地下河相连,附近可能还有一个更巨大的湖与之相连,被潮汐或者气压影响,这里的湖受到连动,比如说小湖和大湖都是磁铁,而假设虹吸效应是月亮引力引起的,那么月亮也是大磁铁,肯定大湖受到的吸力大,于是大小湖就产生压力差了,小湖中的水会被抽到大湖中去,小湖的水位就会降低。”我抬头看看了天,忽然就意识到了什么。

难怪我们找不到一点尸体的痕迹,如果这里存在虹吸效应,每天晚上有虹吸潮,那么当年的尸体可能会被虹吸潮吸到湖中心去。就好像抽水马桶的原理一样。

不光是尸体,所有在湖里的东西都会被抽到湖的中心去,难怪我感觉湖边上除了石头,一点东西都没有。

这湖的湖底落差很大,非常陡峭,只要往下滑落就不会在涨潮的时候被推回来,如果当时没有用石头压住,那么肯定留在湖中心最深的地方了。

想到这里我不由得有些沮丧,不知道这湖有多深,我们没有带水肺,如果湖水太深,那么我们这一次可以说是无功而返了。

不过,再一想又振奋起来,徒手潜水的人能潜到一百多米深的地方,虽然我们没有那种专业技能,但是潜个二三十米也应该问题不大。如果湖水没有深得离谱,我们还是可以下水去找找的,就是需要水性好的人。

来这里一次不容易,不管怎么样我们都得试一试,游到湖中间倒没什么难度。

想着我问他们道:“你们憋气都能憋多久?”

\chapter{湖底}

我们几个中,胖子、闷油瓶和我都有点水性,阿贵能游泳,但是他们一般在溪涧中,没有长时间踩水的习惯,所以恐怕帮助不大。云彩倒是水性很好,可是没有泳衣,我们总不能让她穿着小背心帮我潜水,那胖子恐怕就没心思干事了。

要说憋气时间还真没个准,胖子说他肺大,能憋五分钟,我说不可能,你体积那么大,潜到水下受到的压力比我们大得多,一般能憋到三分钟的人已经是神仙了。千万别逞能,这玩意儿不是开玩笑的。

胖子道他倒不是很担心这个,咱们下去肯定会在浅的地方先试试水,问题是我们没脚蹼,往下潜水很慢,可能没到底就没气了。

我点头,其实自由深潜也不是完全的徒手,也是有相关的装备和保护措施的,而其中最重要的是人的心理素质。我在西沙的时候,听那几个潜水员和我们说过,深水潜水最关键的恰恰是心理素质,所有的深水潜水,特别是自由深潜的潜水员都会做瑜伽的入定训练。在水深的地方,四周一片漆黑,犹如身在一片虚无中,这时人会不自觉地恐慌。在水下,一恐慌就没法定神了,很容易出事情。有水肺的时候,耗氧量也会大幅增加,如果没有水肺就可能直接心理窒息了。

可惜西沙的那片区块海水都太浅,而且水太清,我没有体验到那种感觉,也不知道实际碰到会是如何。

不过自由潜水对于装备并不苛刻,我们可以找到一些替代品,比如说胖子提出的问题,我们只要用石头加速我们下降就可以了。这里的湖原先可能很深,但是这些年水位下降不可能还有一百多米,我看五十米深已经是极限了,当然在潜水之前我们也得先探一下。

我们详细讨论了一些细节,三个人都很兴奋。第二天我们起得很早,趁着太阳没出来,我还是继续在岸边进行最后一次搜索,确定自己昨天的印象。湖四周有一层薄雾,但是只到湖的外延为止,云彩他们都习惯了早起,早早就烧好了早饭。那是很薄的稀粥,胖子一个人都能喝十碗,不过云彩烧的,他怎么也不会说不好喝。

吃完后,胖子也来帮忙搜索,这一次带了狗,胖子逗那些狗,说找骨头,找骨头,找到骨头给你们配母狗。狗却自顾自到湖边喝水嬉戏,完全不理会他。

等到日头出来,我已经又转了一圈,确定是不太可能找到了。我和他们合计,确定得下水,时间定在下午水稍微暖和一点的时候,于是按照昨天计划的,开始收集和准备很长的绳子、一个小浮筏、几块重量合适的石头。

阿贵和云彩帮我们编草绳,不需要太结实,只要能用来测量深度就行了,但是要尽量长。胖子拿着镰刀割了不少草,然后铺开来晒,但是并不是所有的草都适合编,一大半都不能用。

我和闷油瓶用编好的绳子扎了两只八仙桌大小的小浮排,然后找等同大腿大小的石头,绑上草绳做压仓物。

草绳编了三截,只有十多米,两个人一个上午能有这样的成就就很了不起了,因为没有经过很好的加工处理,很粗糙,但是我也不管了,反正没指望能用上几个月,能撑住几个时辰就行了。

另外把胖子的尼龙包裁掉,把里面的尼龙线扯出来盘了个线圈,上面绑个小石头当成小锚,用来探测深度。

准备妥当之后,我们把这些东西全部堆到小浮排上,然后脱得只剩下裤衩缓缓走入湖中。闷油瓶的内裤是胖子买的,上面有两只小鸡,把云彩笑得差点晕过去。

此时已经是下午两点左右,湖水的表面还是冰凉,肯定与活水相连。要是没有太阳,这么大的温差,说不定我们下水还会抽筋。

一路踩水,很快脚下的水的颜色就变深了,这有点让人心虚,看不到底的地方总让人感觉不安全,不过经历了大风大浪,那种感觉一闪就过。湖也不大,我们很快就踩水到了湖中心的位置。

湖风非常凉爽,暑意全消,在湖中心,踩水需要更用力才能保持身体的平衡。胖子用手抹了一把脸,问道:“天真无邪船长,先干什么?”

“先测水深。”我道。

胖子拿起系着小石头的尼龙丝,往水里丢去。石头拉着丝线往下不停地沉,丝线圈在胖子手里不停地转动。很快,只剩下线能看到,石头沉入了黑暗之中。

等了一分多钟,线圈才停止转动,胖子把线头拉断,把线一点一点拉上来,一边数绕的圈数,最后确定水深有三十三米多。

我吸了口凉气,虽然和我估计的差不多,但是真听到还是有点觉得可怕,并且这也不一定是最深的地方,这种石头湖,最深的地方不一定在湖的正中央。

“三十三米,大副,咱们得潜十多层楼这么深啊。”

“我靠,怎么一听到三十米立马就给我降官阶了?”我骂道,一边硬撑,“十层楼一般般,他娘的,怕个鬼。”

说着就和闷油瓶用泥塞住耳朵,先浅浅地潜了几下适应了水温,让胖子暂时先在上面看着,他胖不那么好潜,我们争取一次搞定就不用他了。说着用绑着大石头的草绳系在腰上,拿好镰刀、装在塑料袋里的手电,我就和闷油瓶打了个眼色。

我们深深吸入一口气,在气到极限的时候,一下把石头从木筏上推入水中,石头缓缓沉下,带动我们直接往水里沉去。

在苏丹,出轨的酋长夫人就是这么被处死的。我抬头看着水面,没有潜水镜,所有的情形都是迷蒙的,模模糊糊能看到胖子的下半身和木筏的影子,还能看到太阳在水面上的光晕。但是这些情景很快就远去了,一下四周便进入了绝对的寂静。再往下看,下面是一片漆黑的深渊,只能看到闷油瓶的手电,他头朝下灵活得像一只水蝙。

这种情形不会持续太久,我告诉自己。随着四周光线的急剧下降,同时出现的是巨大的水压,我的耳膜和胸口开始非常难受,使得我不得不吐出肺里的空气。

很快,我的手电照到了水下的情形,那是青蒙蒙的一片石头,逐渐朝我靠近。随着我的下沉,水底也越来越清晰,我发现水下的石头有深有浅,显然并不平坦,而是一处斜坡。

也就几乎在这个时候,我有点锁不住气,看了看表,才下水不到三十秒。我开始感觉一股压力直冲我的鼻子,很想很想吸气。

另一边闷油瓶还在不断下潜,我抬头看了看头顶,天哪,头顶一片模糊,只在很远处有一点光晕,你可以想象,你在一个漆黑一片并有三十米高的大礼堂里抬头看碗口大小的天窗的感觉,不由得恐惧顿生乱了手脚,感觉没法坚持了。

于是拔出腰里的镰刀想割断拉住我的草绳,没想到的是,浸了水的草绳很韧,我割了两刀,草绳只断了一半,另一半怎么也割不断了。

我一下就慌了,条件反射下告诉自己深呼吸镇定,结果一呼吸一口水直呛进肺里,我整个人咳嗽得曲了起来。

好不容易把肺里的水憋住,从绳子的一头传来一阵震动,石头已经落到底了。我努力稳住自己朝下望去,水底果然是一大片单调的陡峭石滩,和岸上的石滩一样,都是大大小小的石头。不过这些石头经年累月泡在水里,上面覆盖着一层水糜,让我感觉异样的是,这些石头完全是“干净”的,不像我以前看到的水底,石头上都会长一些藻类和螺丝。

石滩很陡峭,我的“负重石”卡在石滩的几块石头里,没有往陡坡下滑,但是石滩下面一片幽深,好像还有得潜。

我不知道现在的深度是多少米,另一边闷油瓶下潜的地点肯定比我深得多,因为我已经看到他的手电光沉了下去,好比黑夜中一个模糊的信号弹。

我肺里的气已经吐光,人也开始往水底沉去,很快就趴在了水底,这时反而感觉自己还能憋上一段时间。刚才的紧迫感可能是水压压住我的胸口导致的,我撑了一下,把我的“负重石”从卡住的地方搬了起来,往斜坡下方丢去。

负重石头滑了下去,再次带动我下潜,又滚下去了七八米,石滩的坡度变缓,石头又停住了。

我抓住绳子再次沉下去,还想搬起石头,这时我忽然发现我斜坡下方深邃的青灰色的水中,出现了一个巨大而模糊的影子,好像一只鳄鱼的脑袋。

水下的视线十分的模糊,我只能看清楚大概,不由得吓了一跳,心说这种湖里都会有水怪?

手电照下去,却看到那影子其实是一间样式古老的木楼,垮塌在我脚下的深沟内,只有一个大概的架子,上面覆满了棉絮一样的沉积物。我拽住绳子稳定自己的姿势,靠近那木楼再转动手电,看到这种木楼不止一间,下面还有不少交错的黑影,甚至还有破败的瓦房。顺着这深沟的坡度望下去,石阶,篱笆什么都有,所有的这些都静静地沉在湖水中。

天哪,我惊呆了,我看到的,竟然是一座瑶族的古寨。

\chapter{湖底的古寨}

幽深青色的湖底给过我很多想象,但是我从来没有想到,我会在湖底看到这些东西。

这些木楼被沉积物完全覆盖,很像沉船的一部分,在这种光线下我无法仔细观察,但还是能肯定,我眼前应该是一座沉在湖底的瑶族古寨。

更深处的坡下一片黑暗,下面黑影幢幢,肯定还有东西,我猜测都是这种高脚木楼。

这是怎么回事?为什么湖底会有这些?难道这里发生过大面积的山洪,导致山体崩塌,把原本是村庄的地方淹没了?

看着这幽冥一般的青色古楼,我整个头脑都混沌了,连四周的环境都忘记了,只是呆呆地看着眼前的情形。

正在发呆,忽然浑身一震,我开始往上浮去,一扯脐带一样的绳子,发现原来被我死死拽住的绳子终于断了,这时候才再次感觉到令人窒息的水压扑面而来,再也顾不上眼前的情形,奋力向上挣扎着游去。

那是一种让人很难形容的感觉,有了浮力的帮助我上升得非常快,四周是黑暗,上方是逐渐明亮的光圈,我的大脑开始缺氧,只感觉光圈越来越迷蒙,像在游向天堂。

淹死的人最后看到的大概也是这种场景,我心说,最后的几秒我的气已经到了极限,脑子一下空白,眼前一片白光,之后猛地感觉脸一松,四周的白光收缩了,同时我听到了水声和其他无法分辨的声音,看到了水光潋滟的湖面。

我几乎就没有力气吸那第一口气,那一下呼吸几乎是用全身的力气爆发出来的,等我终于让肺部充满空气的时候,我差点晕了过去——天哪,活了几十年,从来就没有觉得呼吸是那么舒畅的一件事情。

接着我开始大口喘气,几乎是恐怖地吞咽空气,逐渐地四周的一切舒缓过来。

等我完全清醒,抬手看了看表,发现从我潜水下去到我浮出水面,才过了一分钟多一点,我却感觉过了好几个小时一样。水底的环境和看到的情形太让我震惊了,以至于感觉都失常了。

而在平时我的憋气时间没有这么短,看样子游泳池和深水湖泊完全是两回事,我想得太天真了。

胖子和筏子在离我三十米处,可能是我最后冲出水面的时候用错了力气,偏离了方向。我朝胖子游去,游回到筏子边上,胖子就问我怎么这么快就上来了。

我刚想说话,忽然感觉上唇很烫,一摸,竟然流鼻血了。接着耳朵和全身都开始疼起来,人开始晕眩,差点就从筏子上脱手沉下去,恍惚间感觉被胖子拽住了,隐约听到他对我道:“我操,你上浮得太快了,血管爆掉了!”

还好晕眩稍纵即逝,很快我就缓了过来。我不是专业潜水员,看来身体的构架确实不适合这种自由潜水。我再次趴到筏子上,看着源源不断的鼻血贴着我的脸流到我的下巴,然后滴到水里,我隐隐有些担心是否自己的内脏也受了损伤。

胖子给我用他的手绢暂时堵了一下鼻孔,就问我怎么回事?怎么上来得这么急。

我仰起头让鼻血回流,同时把我看到的一说,胖子听得目瞪口呆,随后他还不相信,这种事情,不是自己亲眼看到,不知道是个什么情形。他说他也要下去看一下,我把他拦住了,告诉他这下面绝对不止我们测的那么深,一个人下去太危险了。

这时候又是一声水声,闷油瓶也浮了上来,大口地吸了一口气。他出现的地方离筏子只有两米多,显然比我镇定得多。

我看了看表,比我多潜了一分钟左右,他吃力地游到筏子边上,单手扶上来,胖子刚想问情况如何,闷油瓶另一只手忽然从水里哗啦提上来一个东西,甩到了筏子上,水花一下溅了我们满脸。

我还没看清楚,胖子就惊叫起来:“我操,这是什么鬼东西!”

\chapter{捞起来的怪物}

大概是胖子的叫声给了我预判,我顿时感觉到心里发毛,忙抹开脸上的水去看。

我的第一感觉就是闷油瓶可能找到了那些尸体,我已经做好看到一具惨白尸骨的准备。

可惜我猜错了,我看到被甩到筏子上的好像是一具登山包大小的死动物。仔细一看又发现那“沉尸”的四周竟然还长了一团腐烂的发黑的触手,“沉尸”被水泡胀了,好像一只球一样,看样子在水里已经腐烂了很久。

看过发大水湖里漂过的死猪死狗的人都知道这种尸体有多恶心,我顿时感觉到一股反胃,忙翻身蹬出去远离那筏子,心说闷油瓶捞这东西干什么?

游出去一米多我立即用湖水洗去溅到我脸上的腐尸水,感觉黏糊糊的,胖子已经在那里开骂了,“小哥,我操,你他娘的真是下得去手,什么恶心你捞什么。”

闷油瓶却不以为意,一下趴到筏子上,手直接压在那腐尸上,顿时尸水被挤了出来,顺着筏子流到湖面上。接着他开始把那些触手从尸上撕下来,抛到水里。

我刚开始几乎要吐了,但随即就发现不太对,因为我没有闻到强烈的腐臭味,接着看到胖子似乎发现了什么,也在招手让我过去。

我再次游过去,闷油瓶甩出来的“触手”还漂浮在筏子四周,我忍住恶心捞起一条看了看,发现那不是什么触手,而是一种奇怪的像水草的东西。再仔细看那黑色的“沉尸”,我这才知道自己看错了。那具“沉尸”鼓起的肚子已经瘪了下去,这么一看就不像尸体,反倒像是一个瘪掉的皮球,而四周的触手都是那种奇怪的像水草的东西。

我上去帮着闷油瓶从那“沉尸”边上把水草除下,终于看清了那东西竟然是腐烂发黑的老式牛皮包,牛皮已经被水泡得全黑透了,表层都烂没了,只剩下薄薄的一层底衬。

这是以前装大行李的大包,里面有铁丝的架子,所以没散开,否则肯定烂没了。

“这是……?”胖子失语。

闷油瓶道:“在我潜下去的地方,有一层篱笆,有很多沉到湖底的包和杂物卡在篱笆上,散落了一大片,我看到有步枪、皮包和帐篷,我只捞了一个上来。”

我立即意识到了这是什么:“这肯定是盘马说的,他们杀完人后和尸体一起沉到湖里的枪和装备,看来我说的没错,确实这些都被虹吸潮吸往湖底沉挂在篱笆上了。”

闷油瓶点头,显然同意我的说法。

“篱笆?他娘的,这湖底真有个村子?”胖子还是不相信。

我脑子里乱成一团,心说我骗你干什么,要不是亲眼见到我也不信。

水下的古寨看规模不小,这种一锅端被湖泊淹没的情形十分特别,一般是大型水利工程牺牲性的蓄水造成的,比如三峡大坝蓄水,好多低水位的村子甚至名胜古迹都被淹没了。也有地震导致的山体破坏,水库随着湖泊中的大水流入山洼淹没村子,或者整个村子的地基因为地震而垮塌,村子陷入地下后又被水淹没。

但这里的地形不像是发生过地震的样子,这个石头湖也非常的奇怪,水底全是碎石头不知是怎么产生的。

他娘的这村子肯定和这整件事情有关系。当年的考古队显然来到这个湖边,是为了打捞在湖底的铁块,而这些铁块显然存在于湖底的那个古寨中。这些因素之间到底有什么渊源?这里发生过什么事情?

看来水里深藏的事情肯定超出我的想象。

“先别管这些,先看看包里是什么东西?”胖子急着想开包,但是这包很大,筏子又小,我们三个人扶着不好操作,胖子弄了几下没找到开包的诀窍,筏子却感觉快翻了。我心乱如麻,没心思琢磨这些,拦住了他道:“别急于一时,等下翻了就白捞了,我们先回岸上。”

“不行,”胖子道,“咱们不知道里面有什么,要是个死人或者什么不能让阿贵看到的东西,难道你也杀人灭口?咱们得在这儿先看了。”

我一想也对,让他们知道太多终归不是好事,于是让他快点。

包的整个形还在,我们扯动那薄薄的烂牛皮时发现还有很大的韧性,当时军工产品的质量真是让人神往。这种包一般都用铁皮搭扣,我们在筏子上小心翼翼地把包翻了个身找到了背面的搭扣,翻的时候感觉里面的东西软软的,好像一团棉絮。

这种包本来就是放衣服或者衣料多一些,我心说不要翻出来是床被子,那就搞笑了。

翻开之后看到了已经锈成铁疙瘩的两个搭扣,已经开不动了,胖子拔出镰刀,直接在包上划了一道口子,露出了里面的铁丝框。

我以前看过一本很老的国产警匪电影,里面也有这种包,当时是用来抛尸的,里面装的是尸块,还是有点心理阴影,胖子也很小心,用镰刀把牛皮翻开来。果然,里面是一团几乎已经腐烂的棉絮,这是被水泡烂的毯子的残余物。胖子用刀在里面搅动,很快,我们在棉絮的底部发现了一些东西,拨弄了一下,胖子像考古一样把这些东西全部勾了出来,那完全是一个女人的生活用品。

让我下这个结论的,自然是其中的三把梳子,男人也会带梳子但不会带三把,而且其中一把的齿特别大,那肯定是用来梳长发的。

还有两只发卡,一枚毛主席像章,还有一只木头镜框和一只百雀羚的雪花膏,另外还有一只茶叶罐。

百雀羚雪花膏和茶叶罐都是铁皮的,锈得非常厉害,不过因为湖底的状态稳定可以看出铁锈到了一定程度就停止了。

我最感兴趣是那只木头镜框,里面有照片,但已经完全被水浸烂,只剩下一团团的色条。只要把镜框后面的盖子拧开,里面的东西肯定全都烂掉了,即使不烂掉,从色条上也完全看不出拍的是什么东西。

茶叶罐子摇动后有声音,显然里面是密封的。胖子想打开但是锈死了。他不信邪,用镰刀当榔头敲击罐底,但是筏子不能承受那种敲打,他只好一边仰泳一边把罐子放在自己胸口上敲,清脆的打鼓一样的声音在湖面上回荡,好像一只肥大的水獭。

我看着好笑,但是确实管用,很快罐底就被敲破了,他从里面倒出了一块黑色的东西,立即就惊呼了一声。

我一看心就一沉,那竟然是一块小铁块,和我在闷油瓶床下发现的非常类似。

胖子嘟囔道:“又是这种东西,看来这只皮箱确实属于当时的考古队,盘马没骗我们,他娘的这玩意儿到底是什么?”

我接过铁块仔细看了看,摇头不语,因为我发现这铁块和闷油瓶的那一块相比,有少许不寻常。

\chapter{铁块}

这块铁块比我们之前看到的小了很多,大概只有大拇指的大小,让我觉得意外的是,这块铁相对的光滑,虽然也是锈迹斑斑,但比闷油瓶的那块要干净很多,上面的花纹还清晰可辨。

我曾经想过,闷油瓶床下的铁块那副丑陋的样子是不是因为有人用酸处理过,现在看来果然如此。这种铁块原来应该是这种样子的,而不是闷油瓶那块那样全是看上去像癞蛤蟆,而且从上面非常精美的装饰花纹来看,显然属于一件非常高超的艺术品。

小铁块也有不规则的断面,显然并不是整体,应该是另外一件东西的碎片,这些铁块应该来自于一件或者几件大型的铁器。

我一边踩水,一边脑子飞快转动,感觉事情在此时已经基本连成一线了。现在问题开始清晰起来,指向了大概两点。

我的推测是否正确,是否这里发生过考古队被调包的事件,我们还得继续去寻找那些被他们抛入湖里的设备、踪迹,我想那些尸体很可能也会在附近,这看来并不是难事了。

再有就是湖底古寨的事情,深山中的湖泊底部怎么会淹着一个寨子呢?这些铁块来自于这个寨子,它们原本是什么东西又有什么用处?为什么考古队会知道这件事情要把它们打捞起来,他娘的这之后的猫腻可能就多了,我们现在完全无从想起。关于湖的事情只能大概的向阿贵打听,不过,我感觉他不会有太多的信息给我们。

这两点的答案,都在水底。我叹了口气,明白接下来应该做什么,我们必须仔细观察湖底,并且把下面能找到的东西都捞上来查看。看样子,得在水里泡上很长时间。

可惜,我们身上的草绳都已经酥了无法再用,我的体力也不足以再次潜水,否则我真想立即下去再看看。

我们在这片水域用尼龙绳加浮漂做了一个记号,三人先回到岸上休息,云彩看到我的样子吓坏了,急忙给我处理。我鼻子里塞了两个布条,蹲在草丛里换好衣服,感觉骨头好像从里面裂开了,疼得我一点力气都用不出来。胖子和闷油瓶把筏子从水里拽到岸上,像使用担架一样抬起筏子,连同筏子上的烂牛皮包一路抬到岸上干的地方。

云彩他们非常好奇我们从水里捞上来了什么,因为里面没什么特别的,所以胖子也就让她去看,真看到了她就觉得恶心。

太阳毒辣辣的,内裤甩在石头上自己就会干,我们吃了几棵野果子补充糖分,胖子一边吃一边问阿贵知道不知道淹村的事情?阿贵一头雾水,完全没有任何概念,说他从来不知道这湖下面还有一个寨子。

刚才我在水中视线一片模糊,大多看不分明,无法说出更多的细节,但是凭借上面那种沉积物的厚度,我就知道这村子沉在湖底肯定有年头了。我让阿贵再想想,附近的寨子有没有关于这件事的传说,哪怕是很老的传说,只要搭边都行。阿贵还是摇头,发誓肯定没有,他道:“其实,我也觉得有点奇怪,我们所有人都知道这里有个湖,但是这湖到现在连名字也没有,老人也不是经常提起。”

我和胖子面面相觑,我预料到他不会知道得太多,因为到底是传说,能不能流传下来要看运气,但是我没有想他会说得这么绝对。

羊角山有很多的传说和怪事,因为这里自古是深山和猎区的分界线,人类的活动痕迹到这里就基本不延伸了,所以有传说是很正常的,可是羊角山中这么大一个湖泊,理应也有传说,但却像绝缘了一样,没有任何故事,让我感觉有点奇怪。

胖子道:“这会否就是你们说的被山火烧过的老村寨,说你们的老寨子也是在羊角山被山火烧光了,其实是被淹在这湖下了?所以你们都说在地面上看不到一点痕迹了。”

阿贵摇头:“年代太久了,就是那烧毁的老寨子的传说,也是大明皇帝的时候,两者间有什么联系,我真就没法说了。”

我看阿贵就知道他不是在说谎,于是躺下来抽了根烟,用手指按摩自己爆痛的太阳穴,心说果然得靠自己。

胖子遥指着湖面我估计出的湖底最深的位置道:“这湖底是怎么个德行,我看像被钉锤敲出来的一样,你说是怎么形成的?”

我道:“这不是形成的,这种落差一般只在山与山之间的峡谷河流中才会产生,这湖应该是个堰塞湖,可能是在几百年前形成的。”

“是因为地震吗?”云彩在边上好奇道。

我摇头:“水下的村子保持得相对完好,如果是大地震我们肯定看不到这么整齐的石头路和篱笆,说明村子被水淹没是在相对温和的情况下。”我指了指胖子刚才指的最深处,说出了我的推测,“有可能是因为地质运动,或者什么另外的原因,在几百年前我们对面的那些山体中,突然出现了一条连通着附近地下水系的暗河,因为这个村子正好地处低洼地带,所以突如其来的大水就将整个村子全部淹没了。”

为什么说是地下水系的水,是因为我没有听阿贵说过附近有更大的湖泊,十万大山中我也没有听说过有大湖,但是这里的喀斯特地下河是很有名的。这里接近热带,降雨十分频繁,这些水肯定得有地方去。地面上走的河流水,最后也是汇入地下的大江大河。

昨晚的虹吸潮肯定也是因为这个口子。

胖子道:“看来我说得没错,那我们要找的东西,一定就在最深的地方,我们不可能找到了。”

我摇头:“非也,这些木楼就好比过滤网,被虹吸潮水吸入湖底的东西,大部分都会在古村的外沿被那些篱笆和木楼卡住,所以我们只需要搜索这一圈就基本会有收获,否则,我觉得可以承认失败了。”

这一圈的深度并不太深,我估计只有二十米多,只要有点耐心,我们肯定能发现什么。

胖子看了看太阳,一下又来了兴致,道:“今日事今日毕,咱们这就下水。”

我立即摇头,那是不可能的,从刚才我们潜水的经历来看,徒手潜水实在有点勉强,要想仔细从容地调查水底的古寨,肯定得用专业的潜水用具。我们绝对没法马上进行,得先回到县城里,然后通过关系把装备运过来。

这是一个大工程,潜水器械很重,可能得雇十几个人用骡子拉进山里来,这就不符合我们低调的初衷。而且,这种东西不是那么好弄的,除了氧气瓶,我们还得准备充氧气用的氧气泵,那玩意儿可不是什么小家伙,骡子可能都拉不进来,得分解后再运输,那时间就更久了。

我心中很急,让我要再等一段时间,我恐怕会被折磨死。

胖子也是不愿意回去,但他比我理性,他想了想道:“这个不用想,想要完全探索我们肯定得回去带水肺过来,没什么其他选择。不过从刚才潜水的情况来看,只是潜入水底简单搜索的话也没有必要用水肺,我们可以分头办事,一个人回去置办装备,另外两个在这里先开始打捞那些沉物。这两件事情可以同时进行。”

“那谁回去?”我问道。

“从关系上来说,当然是你回去最合适,你的关系最多,我和小哥在这里打捞。你想你认识这么多伙计,直接找几个伙计帮你置办,可以交代完了就回来,比我们方便多了。”

我骂了一声:“我靠,那还不是一样,我还是得憋死。”

“一个人憋死总好过三个人一起憋死,而且你想,让小哥去肯定不可能,我的关系在北京,比你不方便很多,我去办的话你等的时间更长,在这种地方看看风景是不错,你待上一个月你也难过,所以听胖爷我的没错,你回去置办是最理想的。”胖子冠冕堂皇道。

我看着胖子的表情,那叫一个欠揍,但是仔细一想,他说的确实有道理,我只要给潘子打一个电话几天内事情就能搞定,还能把王盟和三叔铺子里的几个伙计都带过来帮忙。胖子这不靠谱的,他出去办事我还真不放心。我只好点头,当下一合计,也别磨蹭了,明天一早就回去,力求速战速决。

于是和阿贵约好,明天由阿贵带我回去,云彩在这里守着胖子和闷油瓶,我一想阿贵这么来来回回也辛苦,而且现在我们还真缺他不可,得笼络一下他,于是开了个大价钱。

接下来的时间我就瘫了,几乎就没站起来过,胖子和闷油瓶又去潜了两次,又带上来一些东西,但都已经高度破败了,都是垃圾,没什么价值。其中有一支当时的冲锋枪,烂得好比烧火棍一样,胖子爱惜枪,直叫可惜。

胖子也看到了沉在水下的寨子,不由得吃惊竟有这么大规模,他道可惜没有潜水镜,否则可以看得更加清楚一点,也不会尽捞些垃圾。接着他就满世界找替代品,搜遍了所有的装备,最后终于找到了一个东西,那就是手电筒的筒头,但是这玩意儿不太可能密封,胖子就作了一个非常离谱的决定,他把手电筒的筒头贴在自己的眼睛上,缝隙粘上胶布和油脂,然后用力压住,这样可以保证一只眼睛能在水下远视。胖子潜入到浅水中试验,却立即被水压压进筒里,这方法是行不通的。无奈之下他只好让我记得,阿贵和我回县城里,随便找个体育用品店先带点普通的装备过来顶顶。

当夜无话,第二天我早上我就离开了羊角山,走的时候,天空乌云密布,似乎要下大雨,我挥手和他们告别,接着走上山路。走到山腰再次看向湖面时,看到那片乌云,我忽然有一种奇怪的预感,似乎有什么事要发生。

\chapter{流水行程}

长话短说,回到巴乃后我先是吃了一顿好的,之后马不停蹄去到了附近的一个县城,先买了一些游泳用的东西,嘱咐阿贵带回去,然后坐上中巴驶出十万大山。

一路颠簸,心里又急于是十分的煎熬,在车上我还看到了盘马老爹那个满嘴京腔的远方亲戚,看得出他有很重的心事,一路都没说话光在琢磨事情。他也没认出我来。

回到了防城港,定下酒店就开始操办。以前置办过东西,知道其中的猫腻和困难,所以我做得十分有条理,先给潘子打了电话,让他运一些装备,他熟门熟路,效率最高,然后让王盟立即飞过来帮忙,我需要一个人蹲点。

潘子听到我要装备后有些担心,我骗他说别人托我办的,他才答应下来。东西和人都是在五天后到的,我在防城港租好车,一路将东西直接运到了巴乃。盘山公路陡峭非凡,我只能开C驾照车,这一次硬着头皮开大头车,惊险万分,几次差点都冲下山崖,因为全都贴着一边的峭壁开,车头的两边都撞变形了,王盟下车的时候腿都软了。

巴乃的路都是扶贫砂石路,最后一段实在开不进去了,天又下大雨,只好下来换小车。大车的装备装了三车皮的拖拉机才拉进村子里。至此一切顺利,但从我离开到再踏上巴乃的村头,已经过了两个星期时间。

本来和阿贵约好在村口接应,先把东西运到他家里去,到了村口卸掉货却不见他的人。

我当时已经筋疲力尽,不由得有点恼怒,让王盟在村口看着东西,自己去阿贵家找他。我们住的用做客房和吃饭的那栋楼家门紧闭,我敲了半天没反应,只好去他住的那栋木楼。

木楼的门倒是开着,这是云彩他们住的地方,大堂和我们那边差不多,因为厨房不在这里显得干净了很多,角落里堆着他们编织的一些彩框,是卖给观光客的。墙上贴着一些年画,她们两姐妹的闺房在里屋,阿贵睡在边屋,还有一只木梯子通向二楼。

这里民风淳朴,大门都不锁,里面的房间都安的帘子,我叫了几声,小心翼翼进去后发现都不在,又对着楼上吼了两声,还是没有人,似乎都不在家。

我心里就骂开了,他娘的这个阿贵怎么回事,约好了等我的,人怎么找不到了?难道他进山去了?那就要了命了,我在这里就认得他一个,等他回来不知是什么时候了。

我当时血气上涌,并不信邪,怕他也许在上面干活听不到,于是快步上楼,扯开喉咙继续叫。

一楼和二楼之间有块竹子编的门一样的东西,是压在楼板上的,我一下就推开爬了上去。上面是个走道,走道尽头通向一边的木阳台,板竹墙有点年头了,看起来都是从那种废弃的老木楼上偷过来的。两边各有一个房间,一边是堆东西的,里面全是编好的框子和绷起来风干的兽皮,另一边门关着,我敲了半天没反应,好像人确实不在。

我喘了几口气让自己冷静下来,发火也没用处了。这时候忽然想到这门后面,好像是阿贵说的,他儿子住的房间。

他儿子只在他嘴巴里说说,我从来没见过,我感觉可能有些什么残疾所以不太见人,怎么今天也不在?我不由得好奇,透过门缝往里看了看,发现里面非常昏暗,只能看到墙上挂着非常多东西,看不清是什么,好像都是纸片,但确实没人,而且,我没有看到有日用器皿,空空荡荡的。

我心说奇怪,他儿子就睡在这种房间里?这房间怎么住人?想推一把门进去仔细看看,门却纹丝不动,好像里面有什么闩子闩住了。

我没时间考虑这些,收起好奇心下楼,找邻居问了一下,却说阿贵很久没出现了,好像两个星期前进山后就没出来。不过他们也不敢肯定,因为阿贵经常要到外地接客人。他的小女儿因为连日大雨,去邻村的爷爷家去了。

我骂了一声,两个星期前就是我离开这里的时候,看样子他再次进山之后就没出来,很可能他根本就没记得我和他说他得出来接一下。

于是只好自己掏钱,叫了几个村民帮忙,先把那些装备搬到阿贵那里,让王盟先看着。然后又想通过那邻居的帮忙,再找一个向导进山,自己先带着一些力所能及的装备往山里去,到了之后,换阿贵出来找人把装备运进来。

一问就却立即知道了为什么阿贵不来接我,原来我走了之后连下了好几天的雨,山里全是泥石流和烂泥,不要说徒步出来,就是现在带着十几个人拉着骡子进山,全军覆灭也是几秒钟的事情,阿贵他们很可能被困在山里了。

这真是始料未及的事情,我一下子不知所措。不过那邻居对我说不用太担心,阿贵知道怎么应付,他们只要待在湖边,最多被雨淋一下,不会有大的危险。不过我要再进山的话,最起码还要等上一个星期,如果雨不停的话可能更久,这种天气没有任何一个猎户肯帮忙。那不是钱的问题。

一个星期,我一盘算这事就不对了,阿贵如果一直没有回来,那他们都两个星期没有补给了,吃的东西很可能已经耗光,就算阿贵能打猎,在这种大雨下有没有猎物还是个问题。

其实即使他们撑得住,我也等不及再耽搁一个星期。于是开出了三倍的高价想找个要钱不要命的,最后那邻居被我问烦了,就对我说,现在这种天气,敢进山的只有一个,那就是盘马老爹,你要不去求求他看吧。

\chapter{心理战2}

我回到阿贵的房子里,王盟浑身湿透正在把衣服里的水拧出去,我也脱了衣服,不再客气,去阿贵屋里把他的酒拿了出来喝了几口去湿,接下来就琢磨该怎么办。

说实话,我真的一点也没有想到过这种情况,完全是始料未及,这让我想起以前我的导师和我说过的一个概念,叫做“去先入为主表格”。这是一个物流里的概念,后来被应用到很多行业里,就是说在任何环节都必须完全重新考虑所有的条件,不能有任何想当然。在物流里考虑的特别多,包括天气、宗教、习惯罢工周期,所有的细节在任何一个港口都得完全考虑才能保证顺畅。

我就是对这里的天气先入为主了,不知道广西的雨季有多恐怖,才会没有把这个因素考虑进去。

如今事情变得非常棘手,听他们说的,雨什么时候停完全无法预测,而且就算停了,很长时间内山里还是非常危险,所以什么时候能进山,最短是一个星期,最长可能有一个月还多。我不能盼老天开眼,所以现在进山是最正确的。

但如果现在去找盘马老爹求助,我实在是把握不大,我之前讹他的时候和他说过不会再去找他,现如今又去求助,和之前我给他那种背后势力很大的印象不符合,一下就穿帮了,穿帮后他不揍我就不错了,更不要说帮我。

想着想着,我告诉自己不能退缩,既然找盘马是唯一的办法,那只能硬着头皮上了。必须有一个非常巧妙的说法让他上钩。

盘马是只老狐狸,有他们那代人特有的智慧,怎么引他入局,实在是件麻烦事。

我想来想去没个好辙。这事情他娘的真难办,我突然出现,求他带我进山,这事本身就没有任何说服力。因为如果我连进山的能力都没有,那同样也没有威胁他的本钱。

首先,我能明确的是我的态度不能是求,我得是威胁,或者是逼迫,我宁可让他认为我是一个出尔反尔的强大的坏人,不择手段想要达到目的,也不能让他看出我是空架子。

其次,我得把注意力转移,无论我找什么理由来让他带我进山,进山还是进山,我用这个理由找他就表示我没有能力进山,强大的坏人可以在其他地方没能力,但是不可以没能力进山。我必须把我的目的掩藏起来,让他以为我需要他干的是其他事情,进山只是这件事里必须做的工作。

我第一要逼迫,第二,我没有能力办到需要求助于他,不能代表我的无能。这件事会是什么样的呢?

救阿贵和云彩?

不可能,太善良了,我既然是一个冷酷无情不择手段的人,这种善良的品质不能出现在我的身上。而且,盘马本身有一种天生的邪性,我一旦表现出善良,他立即就能压倒我反过来威胁我,我不能表现出人性的弱点。

说要让他到那边当面辨认什么东西?

好像有点牵强,没有他一定成行的说服力。而且这么干我想装也不知道应该以怎样的腔调去装。另外,就算他同意了,看我一个人和他上路,他难免不起疑心,我那种身手在他眼里肯定就越看越孬种,说不定遇到危险还要靠他救我。一来二去,我肯定又没法控制。

想到后来头都大了,感觉这事和套话不一样,套话好比商务谈判,你只要在谈判的时候混过去就行了,这件事谈完了我还得和他上路,一路在这么恶劣的条件上都得装。难度太高了。

我揉着太阳穴,想把坏水全倒出来,他娘的,换个思路,如果靠装不行,能不能来点狠的。

绑架?我一下脑子一跳:把他打晕了然后装驴车上?

但是我立即想起了盘马的身手,再看王盟和我,马上放弃了,我靠,绑架,说不定被他当场就砍死了。

绑架不行,那么直接上大钱,我狠点,直接拿个二三十万出来拿钱砸他。

想到盘马家很困难,加上他儿子的那种态度,我一下脑子里有了一个剧本,就说我要那种铁块,这几天就要,一块多少钱,让盘马去捞,捞上来一块我就给一万,这样,也许他们为了钱就可能自己进山。

发现这个有点靠谱,我开始掏身上的东西,二三十万不是什么大数字,不过我随身不可能带那么多,我把身上的现金杂物全理了出来,数了一下,只有四万,卡里还有钱,但要到镇上去取。估计了一下感觉大概够,刚想让王盟出发,忽然又脑子一闪。

不对,这不是万全之策,虽然我估计盘马很可能会答应,但到底不是百分之百肯定,他万一拒绝了呢?

他一拒绝,我就再没有第二次机会了。爷爷和我说过,做事情可以失败,但不可以在没有第二次机会的时候失败。

“一个办法可以没有百分之五十的成功率,甚至可以只有百分之十的成功率,但是必须留有余地,这样其实就拥有了后续的无数个百分之一百。”

我一下又颓了,挠着头看着我那些信用卡,心说他娘的,还真是难。爷爷只说了做事情要留余地,我也想留,但是怎么留啊。

我有点焦虑,站了起来,想到外面的大雨里冲冲,把脑子里那些废想法全部甩掉,于是收拾我的那些卡,把杂物都理起来。我一下摸到了一包东西,就是在闷油瓶床下发现的那块铁块。

原本胖子让我先带回城里去,找个地方存起来,我给忘记了。我拿起铁块,解开外面的报纸看了看,忽然灵光一闪,想起了爷爷说过的另外一句话:“与人斗,直攻其短。”

和别人斗智,直接攻击对方最薄弱的地方。

盘马最薄弱的地方是什么?我一想,又看到手里的铁块,脑子里有了一个万全的策略。

仔细一过,发现天衣无缝。我不由得一身鸡皮疙瘩,自己的这些想法让我觉得有点恐惧,从来就没有这么处心积虑算计过人,经历了这些事情,我发现自己变了,竟然能自然而平静地考虑这么深的阴谋。但是一想到胖子和闷油瓶的处境,我也没法顾虑太多。

事不宜迟,我立即开始准备,先让王盟给我找了一个香炉,里面填满了热炭,然后把铁块和香炉包在一起烤。

盘马说过这种铁块会散发味道,但随着时间的推移,味道会越来越淡,我知道肯定是里面的某种东西在挥发,而依据一般的规律,一加热,这种淡淡的挥发会再次加剧。

不出我所料,缓缓地,铁块开始散发出一股奇怪的味道,越来越浓郁。

我是第一次闻到这股味道,感觉确实非常怪,无法形容,一定要形容就是一股化学味。混杂着烫铁的杂味。这种味道如果给盘马闻,他确实无法辨认出是什么。

我把东西用毛巾松松地包好,放进背包里,然后在镜子前练了一下高深莫测的妖异表情,之后打着伞,朝盘马家走去。

盘马看到我出现时的表情,很难形容,说不出是惊讶,是恐惧,还是厌恶。

但等我进到屋子里,坐下来,满屋开始弥漫我身上的异味之后,他的脸上只剩下了惊恐。接着,他立即就崩溃了。

我从容地坐下来,看着浑身发抖的盘马,第一句话就是:他们回来了。我来接你。

\chapter{风雨无阻}

我本来准备好了很多的说辞,打算在这场合将他这种恐惧加深,但是完全没有了必要,我只说了没几句话,他就崩溃了,完全丢了魂儿。

与人斗,直攻其短。

盘马的短,就是心中的恐惧,什么都不用说,从心理上我完全摧毁了他。

但是,事情并没有我想的那么顺利,因为他实在太恐惧了,几乎破门而逃,事实上,可能他宁可死也不愿再去见到那些人。

我一点一点将他说服,最后给他的概念是,他必须把这个事情了结了,否则他的儿子孙子都会倒霉,才逼得他就范。当时他也是心一横,抱着必死的心跟我进山。至于进山干什么,我什么都没说,他也根本没问。

当然,名义上是让他跟我进山,但是实际上,是我跟着他,在山里走反正我走在后面前面都没有关系。

看到他这个样子,让我起了深深的负罪感。本来,为了我自己的利益,把一个老人吓成这样就是不义之举,况且我还得逼他跟我到那么危险的山里,这种行为让我觉得恶心,我忽然发现我血管里可能真的流着我三叔他们的血液,那种凶狠狡诈的家族本能。

长话短说,我们整顿了半天就出发了,出发的时候我在前盘马在后,看上去是我在带路,其实我完全不认得。

这一路几乎毫不停歇,又是瓢泼大雨,山路非常难走,好在在防城港我养足了力气,所以还熬得住。盘马一路上完全不说话,我也基本上不和他交谈,就是闷头猛走。

不日便回到了湖边,远远一看,我的娘啊,湖水的水位几乎涨了起码五六米,湖面一下子大了很多,和我临走那水光潋滟相比,现在的羊角山大雨磅礴,山坡上泥水飞溅,面目十分的狰狞。

现在在山上太危险了,我们赶着骡子立即蹚着泥水,由小道直下到石滩湖边。

在山中雨水打在树叶上的声音已经震耳欲聋,不要说到了湖边,瓢泼大雨打在湖面上发出频率一致的声音,几乎充斥了整个天地,让人根本无法对话。盘马的几只猎狗非常的烦躁不安,也不跟随过来,盘马只好任由它们躲在石滩边缘的树下。

没有了树冠的遮挡,雨帘直挂,能见度极其低,我们硬拉着骡子往以前搭的雨棚走去,很快就在雨帘中看到一个模糊的影子一闪而过,好像是胖子。

我知道叫也没用,就算是面对面,现在这种时候也没法说话,便继续往前。这时不知道为什么,骡子忽然都停住了,我回头一看,原来盘马拉住骡子看向我,显然他认为到目的地了,要等我的指示。

经过这么多天,我看他也似乎想通了,并没有像之前那么害怕。而且看眼神,他似乎下定了什么决心,整个人阴沉得不行,我都有点害怕。

人就是这样,一天两天可以吓到半死,天天吓就皮了。

到了这里我就不用再装了,其实到了路途最后我也没有装,因为太累了,我反而开始琢磨如何和盘马解释他将看到的情形。如果让他知道我完全在讹他,恐怕他会杀了我,但是继续骗下去又很难,而且也太不人道了。

我不知道怎么和盘马说,这件事其实只要阿贵他们一出现就立即会穿帮。我想必须先和胖子商量一下,或者我干脆躲起来,等他火发完了再出来,于是让他站住别动,自己放下缰绳先过去找胖子他们商量,顺便通知他们帮忙卸货。

没走几步,看向前方的雨帘就发现刚才的人影又闪现了出来,这时候我才发现那影子有点奇怪。还没等我仔细去分辨是谁,突然后脑就一疼,接着我眼前一黑摔倒在地,好歹没晕过去。

就地一滚坐起来,我看到盘马老爹脸色铁青地站在我背后,另一手的猎刀已经拔了出来,眼里全是杀意。

“你干什么?”我骂道,一下就看到他把刀举了起来,一下朝我劈来。

我靠,我大惊失色,立即就地一滚躲了过去又爬起来,只见盘马的刀在雨中画出了一道优美的弧线,直切向我的脖子,我的下一个趔趄正好避过,坐倒在地,才发现他下的是杀手。

我看着那眼神,想起路上他不变的表情,忽然心说不好,妈的,这家伙在路上是想通了,他娘的他想通的是先下手为强,要和我们拼了,把我们全杀了。

我操,这事情麻烦了,我立即想逃,逃了几步盘马老爹已经绕到我前面,横刀就劈了过来。我大叫我错了,我骗你的!没事情,他们他娘的没回来。狗日的他根本听不进去。

我一路奔波早就跑不起来,在雨中和他周旋了没多久就向雨棚跑,没想到没几下脚踩进一道石头缝里倒了下去,盘马立即逼了上来,我胡乱抄起石头朝他扔去,但都被他躲了过去。他反手拿刀正要压上来,忽然身形停了停,好像发现了什么,看向了另外一边。

我乘机爬起来继续跑,一下发现四周的雨帘中出现了很多人影,将我们围在了中间。

\chapter{雨中魔影}

那几个人影飘飘摇摇,时而出现,时而在雨帘中消失,一看就知道不怀好意,似乎正在仔细观察我们,伺机而动。这种幽灵一样站在雨帘后窥视的影子让人觉得不寒而栗。

怎么回事,这里突然出现了这么多人?

我脑子里第一个念头,就是我之前推测的:村里有人暗中在阻碍我们,现在他们终于动手了。这些人可能要在这里截杀我们。

这可乱了,他娘的,一边是疯马,一边是截杀的大队伍,狗日的。这次他娘的死定了。

我粗略看了一圈,这里大概有七个人,他娘的,不知道他们想干吗,看来这算是在这里设伏了。

我抹了一把脸,把雨水抹掉,但是雨太大,瞬间还是雨水打满眼睑,那些人影还是模模糊糊看不清楚,不知道他们带着什么武器。

也看不清楚盘马脸上的表情,我和他保持着距离,他顿了顿,忽然就朝着其中一个影子疾冲过去。

我一开始吓了一跳,但是随即明白了他的想法:我操,他以为这些人影是那些人了。

在这种环境下,谁也无法从容地设伏或者截杀别人,所以与其等对方看明白了,不如一下冲过去,这么几个人在这么混乱的环境下,只要一乱就会把我和自己人认错,他就有可乘之机。

我不知道这对我算好事还是坏事,我也管不了那么多了,立即就跟着盘马跑去,他们把我团团围住,盘马一和他们起冲突,肯定就会有缺口,我可以借机逃出去。

雨棚也不能回去了,如果这些人早在这里了,那阿贵和闷油瓶他们的情况不知道怎么样,毕竟闷油瓶和胖子身手再好,一人一枪也就挂了,何况还有阿贵他们拖累。

在又滑又不平的石滩上跑步好似耍杂技,我跑了几米膝盖全磕破了,我远远跟着盘马冲到了其中一个影子跟前,因为距离一变动,四周的影子全都不好辨认了,也搞不清楚它们有什么动作。盘马直朝那个影子冲过去,手中瑶刀切过雨帘那阵势一点也看不出那是个八十岁的老人。

奇怪的是,那个影子岿然不动,似乎毫不在意盘马凌厉的冲击。十秒不到我们就冲到了那影子跟前,盘马老爹刀锋一转没有砍上去,却一下停住了,接着他忽然发出一声惨叫,刀掉在地上。他开始往后狂退,接着被石头绊摔在地上。

我从边上绕过去一看,发现了影子的真面目:那竟然是一具站立着的骷髅。让人毛骨悚然的是,这具骷髅身上还穿着已经腐烂成黑色条丝的军装和武装带,背着生锈的冲锋枪。

我头皮一奓也立即退了一步,心说我靠,他娘的这是什么东西!难道那些死人真的从水里爬上岸来了!

但是我的心理承受能力要比盘马好上很多,随即一阵雨打下来,我就看到那骷髅的头骨随着风摇摆,像灯笼一样,好像是挂在上面的。

定睛一看我发现,这具骸骨是用树枝架起来的,背后有一个树枝架子。

我靠,这里怎么会有死人,难道他们找到湖底的尸体了?我吸了口凉气,仔细一看那骨骸,果然不差,那些被水腐蚀的衣服和武装带,这肯定是一个当兵的。看样子我的想法没有错。

不过,我看着这骨骸立在这里的样子,又觉得诡异异常,暗骂了一声,这算什么?吓唬人?

盘马老爹吓得够戗,我回头看的时候,已经看不到他在什么地方,我心想这是胖子的恶作剧?

我立即冲回到骡子那里,还是不见盘马老爹,我头疼得要命,走向另外那些影子,发现都是同样的死人,我能找到的一共是七具骨骸,在其他地方还有没有就无法肯定了。那个疯子不见踪影,似乎躲藏了起来。

这么大的雨,我没法去找盘马,于是准备先去和阿贵会合,告诉他们这里还有其他人。骡子似乎是害怕这些死人,怎么驱赶也不动,我把它们拴到石头上,然后绕过那死人直走到之前的雨棚里。雨棚明显已经经过加固,在这么大的雨中岿然不倒,我冲进去,只听四周顿时一安静,环顾了一下,发现他们不在里头。

我再次暗骂,心说下这么大的雨,难道他们还在下水?还是他妈的出了什么事情。只见雨棚内堆着大量的东西,都是从水下打捞上来的,我不在的这两个星期,胖子和闷油瓶成果斐然。

这些东西凡是金属的都锈得一塌糊涂,我看到水壶、步枪手枪、望远镜、一些匕首、砍刀,都是当时的武装配备,可以想见当时这里的战争气氛。另外还有很多生活用品,甚至有饼干盒,非常细致,什么都有,可能是从一些大件的打捞物里找出来的。

我想着自己没有东西防身,捡起一把当时的56式三棱刺刀,这是很有名的刺刀。当时刺刀其实并不多用,毕竟是近八十年代,单兵兵器的火力都很强大,刺刀一般只在执行特种任务的时候才用,丛林战里越南人是不会跟对手拼刺刀的。

因为本身的材料问题,刺刀并没有腐朽得很厉害,我听说这种刺刀上通常喂过毒,所以也特别小心,反手握着。我心里琢磨该怎么办,妈的,主要是这么大的雨,叫也听不见,看也看不清楚。

想着自己在雨棚里目标太大,搞不好盘马杀进来,于是重新冲进雨里,跑到湖边,看阿贵他们是不是在。来回绕了几圈,忽然看到有个人在湖滩上拖着木筏子往岸上走。

我冲过去,发现那人是阿贵,单薄的背影一个人拖着筏子往岸上走,我出现在他面前的时候,他看到了我,一下子呆住了,脸色苍白得吓人。“怎么只有你一个人?他们人呢?”我问道。

阿贵呆呆地立在湖水中,神情有些呆滞,他就这么盯着我,我又问了第二遍,他还是没反应。

我看着那些木筏,以为阿贵是刚从湖里回来,心说我靠,果然这些人他娘的疯了,这么大的雨还在打捞。这时候我忽然意识到不对,为什么阿贵拖着筏子回来了?他应该在湖面上等着他们,否则在大雨中游泳是非常危险的,更何况水位已经上升了那么多,而且阿贵的表情十分的不对劲。

我走近阿贵,想再问清楚,越走近就越意识到不对,阿贵的表情无比呆滞,似乎经历了什么让他极度受刺激的事情,他整个人在离魂状态。我上去就抽了他一个巴掌,大吼道:“出了什么事情?”他一下就反应了过来,这才忽然泪流满面,大哭道:“他们……他们都死了!”

\chapter{魔湖的诡异}

“死了?”我脑子嗡的一声,心说,怎么可能?

阿贵说完这句话,一下子情绪就完全崩溃了,整个人几乎是瘫倒在湖里。我只好先把他扶了起来,扶回到雨棚里。又到骡子那里拿了几罐米酒给他灌下去,他才舒缓过来。但情绪还是极度的低迷,语无伦次。

我一边听一边组织,最后终于明白这里发生了什么。

原来跟着我离开之后,再次返回时,阿贵找了几个人帮运食物和东西到湖边,看看没什么事,云彩就跟着那些人回家干别的了,这里只剩他自己看着。

当时闷油瓶和胖子已经打捞上来了很多的东西,并且他们已经发现了可能藏匿着那些尸体的地方。但是那时雨已经没完没了地下了起来,水位开始升高,使得他们的打捞陷入了僵局。

这时,他们在整理打捞物的过程中,发现了一整套打捞设备,包括潜水服、牵引绳,当时使用的是重装潜水的设备,由气管连着水面,用麻绳牵引。胖子说他们肯定是使用这套设备在这个湖底古寨里打捞那些铁块的。

整套设备在水下泡了很长时间,大部分部件都已经不能用了,但当时的潜水头盔使用了非常耐腐的材料,打包在装备包里竟然没有透水,里面还是干的,只有在外面的一层橡胶脱落得斑斑驳驳。

胖子当时突发奇想,想利用这个头盔和一部分橡胶做一个简易的潜水设备,头盔里的空气可以供他呼吸七到八次,因为人呼出的气体中同样含有大量的氧气,所以这点空气还是很可观的,运用得好可以让胖子在水下待的时间延长到五分种。

对于潜水来说,这从容的五分钟和那一分钟可是天壤之别。他们就是利用了这套设备,找到了水下的骸骨。当时的过程是,他们使用了两条绳索,一条拴在胖子的腰上,因为头盔很重,光靠胖子的力气可能会在上浮的过程中出危险,此时需要他们将他拉上来。另一条绳索上全是用铁丝弯的钩子——铁丝是从皮箱的龙骨里拆出来的——胖子潜下去后,把打捞上来的东西全部都挂到钩子上,这样一趟下去能捞不少东西上来。

骸骨全部已经散落,分布在那条篱笆的东端,他们将其打捞起来,根据其中的位置,将他们用树枝拼合起来以确定人数,操作十分简便顺利。

等他们把所有能看到的骨头都打捞起来之后,拼接的时候发现了一个问题。

所有骨头拼成了大概的人形,他们惊奇地发现,所有的骨骸中,竟然都没有右手掌。

按照骸骨统计的方法,头骨和盆骨是判断人数最重要的依据,因为其他骨骼太零碎,有所缺失不稀奇,但是,一只右手掌都没有实在是太奇怪了。这应该不是偶然。

胖子和闷油瓶开始琢磨是什么原因造成了这种情况,到底是抛尸的时候有什么特别的情况使得右手掌都缺失了,还是被人为地砍掉了?

盘马和我说的过程中,完全没有提过他们砍掉这些尸体的手掌,而且他们也没有理由这么干。结果百思不得其解释,胖子还奇怪那些人难道都是狗熊,熊掌被人剁了炖秘制菜了。

最后,还是阿贵得出了一个结论,他说会不会这些人本身就没有右手,所有人的右手都是假的用木头做的,结果抛入湖中之后木质的义手都腐烂了。

我听到这里,却完全不这么想,因为所有人都没有右手这个前提太诡异了,我实在想不出有什么情况会这样,我反而感觉是否这些人的右手上有什么特征,有人为了隐瞒这些人的身份于是将手剁掉了。或者是,好像战利品一样,这些人的右手被人收集走了。可是盘马又没有提过这件事,难道当年他们抛尸之后,尸体还被捞上来重新处理过?但这个想法随后也被证实不可能,因为在阿贵的叙述中,胖子也想到了这一点,但看那些人的手腕骨,都没有被刀切过的痕迹。那些人的右手掌好像都是自然脱落的。手腕部分的关节都在。

在盘马老爹的叙述中,考古队那帮人都是有右手的,显然右手的缺失是在他们死了之后,他们实在想不出理由,于是再次潜水去寻找线索。

他们在篱笆附近再没有发现什么,胖子怀疑那些骨头沉入到篱笆内的古寨之中了。

之前他们刚开始潜水的时候就有一个默契,就是绝对不进入湖底的古寨之中,只在环境比较简单的外围活动。因为寨子内比外围又深了好几米,而且这种湖底探险危险性很大,湖底的环境谁也没有测试过,说不定有的古寨已经十分的脆弱,一碰就坍塌,需要更加完备的潜水设备。

胖子等不及,认为就是过去看看没什么大不了,所以这时就有了一些矛盾,但是我不在,闷油瓶又不会说什么闲话,阿贵也不可能反驳老板,所以胖子就潜下去了。

这一次,却出现了意想不到的变故。

当时的绳子是阿贵从县城里带回的尼龙绳,非常结实,而且买了有三百米,所以胖子一点也不担心,他可以潜到更深的地方。胖子潜下去之后,逐渐深入,和以往一样,阿贵也没有太担心,他看着时间,预备着到点之后,再用劲把胖子提上来。

他们约好的时间是四分半钟,因为大概需要三十秒到一分钟的时间上浮,上浮太快会出现潜水病。

在水下潜水,其实四分钟给人的感觉是很漫长的,而在水上是稍纵即逝,不久阿贵就开始扯动绳子,没曾想这拉了几下,忽然绳子就绷直了,而且怎么拉也拉不动,好像下面被什么东西咬住了。

当时第一个念头就是可能挂在篱笆上了,之前也遇到过这种情况,那些篱笆被水泡了不知道多少年,全都像旺仔小馒头一样酥软,只要用力拉就可以了。阿贵用力扯了几下,果然绳子动了。

阿贵快速拉升,可是这一拉,他就发现手感不对了。绳子吃的力气变小了很多,拉起来非常轻。

这种感觉说起来有点恐怖,很像钓鱼时鱼儿咬钩之后,和鱼僵持了几秒线却松了,这代表着饵被咬掉了,鱼却脱钩了,而现在,饵就是胖子。

阿贵当时冷汗就下来了,越拉他感觉越不对,离水面越来越近,手感也越来越轻。随着逐渐可以看到的水下黑影,他几乎就窒息了,等到那影子拉出水面,他发现胖子竟然不见了,他拉上来的,只是个头盔。

他一推测,很可能是这绳子钩在什么地方了,胖子一看形势不对,立即把头盔脱了,然后自己浮上来。脱了之后,不知怎么的钩住绳子的东西又松脱了。这样说来,胖子很快就会浮上来。

可是,等了一分多种,没有任何东西浮上来。

他感觉有点不妙了,这不同于其他状况,在水下待了一分钟,普通人肯定溺死了。

当时闷油瓶在岸上,阿贵逐渐就慌了,本来挺好的生意能赚钱不说,在这里只要会游泳就能轻松打发老板,现在一下子出了状况,那是要负责任的。在山里这种小地方,出点这种事情可能会被人传一辈子。

他一边脱掉衣服,一边朝岸上呐喊,看闷油瓶往湖里跑过来后,他跳入了湖中,抱着石头潜水下去,可惜他实在没经验,沉了几米石头就脱手了,又挣扎着浮上来。正好闷油瓶赶到,阿贵把情况一说,闷油瓶立即戴好捞上来的头盔,也跳了下去。

阿贵拉着绳子求神保佑,他没有想到的是,一直等了五分钟,不仅胖子没有上来,连下去的闷油瓶也没有任何动作,那绳子就那么垂在水里。

他拖起绳子,熟悉的手感又传了过来,等他拽出水后,发现同样的情况再次出现,绳子的另一头,闷油瓶也不见了,只剩下了潜水头盔。

我听完后就蒙了,脑子里乱成一团,内心并不接受这些事情,感觉太扯淡了,这种事情怎么可能发生,但同时我又清楚地知道阿贵不可能说谎,那这事对于我来说,简直太可怕了。

我问阿贵这是什么时候发生的事情,他道离现在已经快两个星期了。事发之后他在湖面上等了一天,什么东西都没有浮上来。

两个星期?就是鲸鱼,在水里闷两个星期也死透了。难怪阿贵说他们死了,不管是什么原因导致他们在水里脱下了潜水头盔,死亡是可以确定的。

那天之后,阿贵每天都要到湖面上看一圈,想看看有没有尸体浮上来,但是一直没有尸体。他一度以为湖底有什么怪鱼把他们吃了,但很明显的也没有任何血迹和被攻击的痕迹留在那个潜水头盔上。

我看了一下头盔,发现胖子做了很有趣的改动,而这种改动使得头盔在水下很难脱下,这就变成了一件“存在问题”的事情。我潜入过水底,知道水底的情形是怎么样的,虽然进入到古寨中有潜在的危险,但也不会让他们花那么大精力脱掉头盔啊。

我怀疑是否是潜水病,因为潜水到更深的地方后,吸入氧气的比例似乎要经过调制,否则会形成醉氧,但是醉氧不是醉酒,不会醉到脱衣服的。

在水下肯定发生了什么事,使得他们非脱掉头盔不可,而且,闷油瓶也脱掉了头盔,说明这肯定是个不可选择的过程。闷油瓶不会像胖子那样突发奇想。

那么在水下脱下头盔之后,他们为什么没有再出现呢,难道他们遇到的这件事最后还是导致了什么意外吗?

我长途跋涉,身心俱疲,一下遇到如此棘手的情况,真的有点手足无措。但我绝对不承认他们已经死了,我们一起经历了那么多事情,可以死在任何地方,但我们都绝处逢生了,怎么可能死在这么一次半旅游半调查的旅途中。

即使话是这样说,我一仔细琢磨这个事情,心还是揪了起来,让我立即放弃侥幸。因为我知道,意外是不和你讲道理的,就算你以前遇到过再大的危险,该到你死的时候你怎么也逃不过。历史上很多大英雄都是风云一生最后死在小人物手里,难道上帝玩我,他们两个真就这么没了?

想了想,我的内心还是无法接受,人烦躁起来,心说当时已经在下雨,在湖面上的视线肯定不好,他们也许当时已经上浮但离阿贵的位置很远,所以阿贵没有看见,之后又因为什么原因,他们独自上岸了。

不管怎么说,有件事我是必须做的,无论他们是否出了意外,我必须潜水下去看个究竟,活要见人,死要见尸。

\chapter{独自下水}

雨还是那么大,像疯了一样,在杭州这么大的雨是坚持不了这么长时间的。

阿贵已经无法再帮忙,我猜他是怕我和他们一样,他再也经不起这种刺激了,我和他说了盘马带我来的事,让他小心盘马,虽然我觉得这一次盘马可能真的崩溃了。

他想去撤掉那些死人,我说不要,那些死人在,可以防止盘马回来,看盘马的样子,已经是很难说服他了。我真没有想到这人凶悍到这种地步。

回到骡子边上,我从上面取下带来的那一套水肺,便急匆匆往湖里走。我一分钟也等不下去,必须去查证一下。

穿上全套装备,在海南我已经对潜水非常熟悉,所以此时并不紧张,推着木筏就冒雨往湖中心游去。

因为戴着脚蹼,我很快就游到了湖中心的位置。暴雨拍打着湖面,千万条雨线带出的是振聋发聩的雨声,这种无法言喻的声音反而让我平静了下来。我四处寻找当时我们留下的浮标,发现在这种环境下根本无法寻找,只得找了一个大概的方位,然后戴上潜水镜,沉入水中。

有了上一次的经验,这一次我稍微从容了一点,因为我知道这种潜水方式绝对沉不到最底部,所以准备就在沟的上方悬浮一段时间,借以观察大概的情况。

潜到之前的位置,我再次切断绳子,吐光肺里的气,这样我便不会迅速上浮,同时划动手脚使得自己悬浮在一个固定的深度。

有了潜水镜水下的一切非常清晰,可惜,现在光线暗淡了很多,我用双脚保持平衡,一边尽量沉得更低一点,一手划动探灯,开始往深处照去。不久,一个灰青色但轮廓分明的湖底世界比较清晰地出现在了我眼前。我划动双脚开始往前游去。

因为手电只能一部分一部分地探照,我无法看清全貌,只有凭借记忆在脑海中将我看到的东西连成一片。好在我是学建筑的,有一种特殊的记忆方式,能够把看到的部分在脑海里形成一个整体。

这是一个单色的世界,一切都是暗青的湖水色,往前游了一小段,发现果然如我所想,沟口一直到沟底非常暗的部分,这么一条陡峭的斜坡,都是覆盖着沉积物的木楼。湖底竟然完全不平坦,而是一个很深的不规则的水下峡谷,寨子就依山而建在峡谷的南坡。

接下来的时间,我不停地上浮和下潜,变换着自己的位置,在短暂的一分钟内观察水底的情况。

更多的细节出现在我眼前,幽冥一般的水下古寨,规模应该和我们来时的瑶寨不相伯仲,有五十六户人家,大都是高脚楼。但能从细节上看出,这些古楼不是近代所建,非常的古朴,细节上瑶族的特征非常明显。不像现在有很多高脚楼都是土不土洋不洋的。

对于我们原先下潜的位置我还有一些印象,胖子也提过有篱笆的地方。在那一带搜索,很快,我就找到了那细小的浮标,同时看到了那些篱笆。我立即沉了下去,水下什么都没有,看不出一点他们存在过的痕迹,也没有任何的异样。

\chapter{老树蜇头}

胖子和闷油瓶应该就在这个地方遭遇了什么事,因为某个我还不知道的理由,他们解开了连着水面的绳子,然后,就在这几十米深的湖底消失了。

没有水肺,他们在水下只能坚持一分钟,这一分钟他们能走到哪里去呢?我不愿意相信什么被水鬼吞噬的诡异说法。按照现实推断,他们在水下最多只能行进二三十米,也就是说,除非当时水下有一艘潜水艇在接他们,否则,他们什么都干不了,也没有任何地方可以去。他们应该就在这附近。

但是四周什么都没有,寂静的湖底坦坦荡荡。

其中最奇怪的部分,是脱掉那只潜水头盔和解开绳子这两点细节。一方面,这只头盔穿戴起来十分的麻烦,它的拉链在背后而且非常长,即使给你从容的时间,要脱掉它可能也得十秒到二十秒,加上解开绳子,最快也得加上五秒。这二十五秒还是闷油瓶的时间,如果是胖子,他的那种体格和心理素质,恐怕需要更长时间。另一方面,这头盔并不影响他们的行动,被攻击时还能作为防具,所以,于情还是于理,他们都没有必要脱掉头盔。

到底是发生了什么事,使得他们升起脱掉头盔这个念头呢?

从闷油瓶也同样脱掉了头盔来看,这件事肯定不是突发奇想,他的性格非常靠谱,所以脱掉头盔应该是一个非常必要的举动。

我觉得肯定不会多危险,他们从容脱掉头盔,必然他们遭遇的事不是急迫的瞬息万变的,比如被动物攻击,或者遇到了怪事之类,反而应该是一件让他们能从容思考,并且作出“可以脱掉头盔,不会有危险”或者“可以脱掉头盔,危险在控制范围内”这样判断的事情。

那么,能肯定的一点,这件事一定发生在附近。

一步一步的分析让我逐渐沉静了下来,看了看石坡下方幽深的水下古寨,忽然感觉到有一股妖异的寒冷从那片废墟中透了出来——他们会不会是在这湖底古寨的里面?但是,从这里到达古寨在一分钟内是不可能办到的。他们疯了才会脱掉头盔游到那里去。那等于自杀。

我尝试还原当时的景象,看看四周有没有什么地方是必须解开绳子才能过去的,又或者是必须拿掉头盔才能通过的。

四周都是干净的石滩,我缓缓游动,发现这里的情况非常的简单,在强力探灯和潜水镜的视野下一目了然。唯一有可能的地方是石坡下方,靠近寨子边缘的地方,那里有好几根沉底的巨大朽木。

这几棵朽木肯定是当年村外的大树,现在所有的嫩枝和叶子全部腐烂成泥,只剩下粗大的树干还未完全腐烂。无数从它身上掉落的枝丫堆积在周围,形成了一大片枯萎灌木丛般的树枝堆,大量的树枝纵横交错,并被水中的石灰质覆盖得犹如岩石一般。

如果胖子在其中发现了什么东西,他可能会解开绳子才进去,因为绳子很容易缠在这些枝丫里,而笨拙的头盔也会让他无法把头部靠近去查看。

想着,我忽然有了一阵寒意,脑子里忽然有了一个很恐怖的念头:当时,也许胖子在这堆枝丫中发现了什么,他解开头盔和绳子去看,结果困在其中,然后闷油瓶为了救胖子也脱掉了头盔,结果也困在了里面,两人于是都溺毙了,所以才会出现不见尸体的诡异结果。

如果真是这样,那我将面临极其恐怖的景象——我会在树枝堆里看到他们两个在水下泡了两个星期的遗体。他们的尸体之所以没有浮上水面,可能是被困在这些鬼爪一样的枝丫中了。

我不敢过去了,但随即我逼自己划动脚蹼,因为现在已经无法逃避。

保持距离,我漂浮到那些朽木上,探灯往下照去,看到下面有一个篮球场大小的区域里,全是白花花的树枝,如同铁丝网般纠结成一片。光线透过那些树枝照下去,下面一层又一层,要是卡在这里面,就是大罗神仙也逃不出来。

而在这些树枝纠结中,确实有一些很大的缺口,似乎是有人强行掰开这些树枝造成的,在其中并没有看到胖子和闷油瓶的尸体。

我找了一圈发现确实没有,才略松一口气,咬着钢牙逼自己沉下去靠近树枝的表面。

贴近这些树枝我屏息一看,立即发现这片树枝肯定困不死人,很多的树枝都被掰断了,我从断口处看到这些树枝其实内部已经腐烂得犹如泥粉,用手一碰就断成好几截。它们能保持树枝的形状只因为外面有层薄薄的石灰质在支撑,好比一根根非常薄的石灰管。那东西吃不了力,即使被困住了,稍微挣扎一下就全碎成小的石灰片了。

在这些个缺口中,确实有无数的石灰片和断掉的“石灰树枝”凌乱地堆在四周,也许是胖子在这里搜索过那些骸骨造成的痕迹。我把探灯凑近往下照了照,却不见什么异常。显然他们什么都没有找到。

我不由得苦笑,如果不是这个原因,我真的想不出这里到底发生了什么,为什么在湖底忽然就消失了,难道真如阿贵说的,这里有什么湖鬼作祟不成。

那一刹那,我甚至有了一个想法,我想把自己的潜水服也脱掉,看看到底会发生什么,好歹才忍住没有做出这种荒唐事。

这几根朽木的下方就是古寨,我位于俯视的视角,看到的全是瓦顶而看不到内部,探灯打到最大也没用,那一点灯晕透去,反而让古寨显得更加安静幽深。

我收敛心神,准备继续去四处搜索,探灯就划动了一下,就在转开头部那一瞬间,我忽然感觉到古寨之中好像起了变化,便又将头转了回去。在古寨深处的某处,不知何时出现了一点诡异的绿光,似乎是一盏晦暗的孤灯被人点亮。

深水下,青色冰凉的光晕似乎是从幽冥中亮起的磷火,朦朦胧胧,我的大脑顿时一片空白,中了梦魇一般,心跳加速,压得我胸口无法呼吸。

我操,这是怎么回事,这是什么光?难道这古寨中有人?

难道闷油瓶和胖子在这座古寨里,他们不仅还活着,而且还在活动?

可这是在几十米深的湖底,淹没了近千年的古寨,他们没有氧气,怎么可能在水下活这么长时间?就算是手电两个星期也早就耗尽电池了。而且这种光,有一种无法言喻的鬼魅质感,不是手电的,也不像火光。

窒息的感觉越来越强烈,这似乎是当年死在湖底的冤魂还没有成佛,一直在这湖底的废墟中徘徊?这是当年瑶家的灯火,穿越了幽冥和人间的隔阂,在指引亡灵回归鬼域的方向?

一种莫名的冲动,让我不自觉就想朝灯火游去,好比在冰冷黑暗的湖底,好比迷路的人在山中看见灯光一般。但是在那一刹那,我忽然灵感一闪,突然意识到,会否胖子和闷油瓶在当时,在我所在的位置,也是看到这一点光才导致了他们的失踪?

这难道就是关键所在?接下来,会发生什么?

我不由得收紧心神,观察四方,怕有什么突然发生的事件。

然而环视了一圈,四周无比的安静,探灯照去,看不出一丝的异动。

转回头去,那一盏孤灯的绿光越来越晦涩。忽然,一股毫无来由的恐慌,开始在我心中蔓延。

\chapter{水底的灯光}

湖底古寨深处的孤灯不知是从古寨中的哪个位置亮起来的,是在深处还是某幢古楼的窗户之中?孤灯的颜色实在无法形容,灯光非常之不通透,似乎是被人蒙在一层青暗色的罩子里,朦朦胧胧不像人间的灯火。

这座诡异的湖泊已经给了我太多的惊讶,这片清幽之下的寂静之地隐藏着太多的秘密,这里到底发生过什么,使得所有的一切都像被诅咒了一样?

在这种幽冥环境下,我孤身一人潜入深山中的湖底,没有任何支援,没有任何帮助,第一次感觉到无比的恐慌和孤寂。这种无助的绝望感比死亡还要让我恐惧。我有一刹那的错觉,想到了深海的一种以灯光捕食的丑恶鱼类,这古寨给我的感觉,似乎是一种巨大的生物,正在使用那灯光吸引猎物自投罗网。

我看了看氧气表,心脏的狂跳使得氧气耗费得很快,那种毛骨悚然的梦魇感始终挥之不去。我强压住自己的恐慌,心中默念道:“如果要弄清真相,恐怕必须得以身犯险,如果胖子和闷油瓶现在还活着,那么他们现在肯定陷在一种非常诡异的情形中,我可能是他们唯一的希望。我既然来到了这里,其实根本就没有退路,这青色的孤灯,不论是凶是吉,都是召唤我的指路灯。”

这近乎是自我催眠,但在当时的环境下,我真的不知道从哪里去鼓起勇气继续深入。我念了三遍,才感觉那种恐慌稍微减轻了一点,将刺刀拔出反手握着——虽然不知道这东西能对幽灵有什么用处,但总算是壮胆。

我划动脚蹼,贴着湖底的石滩开始往古寨潜去,潜不了多久,古寨中的幽光就因为我角度的下降,逐渐被古楼遮挡,很快便看不到了。四周的黑暗逐渐回笼,深处的古寨再次回到幽冥之中。

我逐渐镇定了下来,奇迹般地恐慌开始退却,看来这恐慌似乎完全来自于青色的幽光。心中不由得暂时松了口气,以我的性格,眼看着灯光逐渐靠近会把我逼疯的。

我所处的位置和古寨的边缘并不远,逐渐靠近后我发现古寨边缘的地方,石滩斜坡上还有不少朽木,有些还立着,有些已经倒塌横亘在湖底。显然这个古寨在被淹没之前,四周大树林立。此处果然风水俱佳。

下潜不到片刻,我便来到了古寨最上端的地方,最近的高脚木楼顶部离我最多只有两三米的距离。因为是从坡上往坡下潜,此时的水深可能已经超过七十米,水压让我相当的不适应。“不识庐山真面目,只缘身在此山中。”到了此处,我完全看不到寨子的全貌,只看到密集的大楼盖子,隐约能看到寨子之中的青色幽光就在不远处。同时我看到,在我的脚下,寨子边缘的一处地方,立着很多犹如墓碑一样的石碑。

我略微下潜用探灯去照,发现石碑上结满了水锈,显然这些石板中本来有石灰岩的成分,在水中溶解了,把石头泡得坑坑洼洼全是孔洞。已经完全看不清楚上面的字,但不是墓碑,是瑶苗特有的一种石碑。

古瑶有石碑定法的传统,瑶民在遇到一些需要集体讨论的事项时,会开“石碑会”,会后立一石碑于寨中,称为石碑律。这好比是瑶族的法典,所有人包括瑶王都必须遵守,瑶族人把这种石碑叫做“阿常”。

这种律令的神圣程度超乎我们汉人的想象,瑶人认为“石碑大过天”,不少古时的汉瑶冲突就是因为汉人想动摇石碑律而产生的。而每块石碑都有一个管理人,叫做石碑头人,权力很大。

这里石碑很多,如果是石碑律,那上面肯定记载着许多十分重要的事情,可惜字迹已经看不清了,而且很多石碑律牵扯到瑶寨晦涩的古老秘密,所以大都使用无字碑,全靠当事人的自觉来维持上面的规定。

我想如果能够看到这些石碑上的字,也许就能知道这个古寨到底发生过什么事情。

越过石碑群,我再次来到寨子的上方悬浮着,因为距离挨得很近,湖底那些破败的高脚木楼和木楼间的小道,变得无比的清晰。此时,青色的幽光再次显露了出来,看不到光源但是暗淡的光晕就在前方。我的头皮再次开始发炸,心跳得更加厉害,恐慌感几乎没有任何削弱,一下又充斥进了我所有的感官中。同时,我感觉到这种恐慌非常异样,它似乎来自我最原始最深层记忆中的恐惧,无法形容,更无法驱除。我到底在怕什么?

在这种高度鸟瞰一座千年古寨,世界上和我有同样经历的人恐怕不到一百个。看着就在身下垂手可触的破败腐朽的木楼,你能感觉像是漂浮在古道的半空中闲庭信步。千年前四周的景象不可避免地在你脑海里形成,但随即又被水流和某些情性带回到现实,这种交织让人感觉很不真实。

这是第一次近看这个湖底古寨,我发现整个寨子和巴乃很相似,高脚木楼修建得十分之密,两到三层的木楼中间有一些三人并进的青石小径和石阶穿插着。这些腐朽的木楼都往一边倒去,看上去随时会坍塌,有些房顶滑塌在一边的另一幢楼墙上,形成一道“门”的样子。我在这些门的上方悬浮着游动,看着自己吐出的气泡冒上去,心不由自主揪了起来。如果潜入寨中,只要有一点意外,这四周的木楼就可能倒塌,如果逃脱不及就会被活埋。在水底被活埋意味着一点获救的机会都没有了。

掠过几幢破败的高脚楼顶,灯光的所在越来越近,我的心跳窒息感也越来越强,看灯光和高脚楼之间的角度,我判断那光来自其中一幢古楼之内,可能是映着窗口透出来的。正要咬牙硬着头皮潜下去,忽然一暗,那光消失了。

我的精神高度紧张,这一下把我惊得几乎晕厥过去,呼吸管都脱嘴了,但在那一瞬间,我已经看到了灯光的所在。

那像是一幢非常巨大的复合式高脚塔楼,由好几幢高脚楼组合在一起,大概是瑶族大家族的塔楼,一般是寨子中最富裕的家族聚集形成的。刚才那一瞬间太快,我没来得及看到灯光是从塔楼的哪个窗口透出的。

我缓缓下沉,探灯照下去,一下就愣了,天哪,这是什么楼?

这塔楼果然有点不同,它的外沿竟然是石头结构,而且,那瓦顶的飞檐,竟然是徽式的。

这不是瑶族的塔楼,而是汉人的建筑。

我愣了,这是怎么回事?为什么在瑶族的古寨里会有一幢汉式的楼宇?

\chapter{瑶家大院}

苗瑶自古和汉家不两立,分群而居,对自己的隐私和血统非常在乎,特别是南瑶,从古到今就是少数民族冲突最多的地方。古时候有三苗之乱,解放前还有客家人村门,为了一口井,一条河沟,汉瑶、汉苗之间,甚至瑶寨与瑶寨之间,都能杀得无比惨烈,以至于直接催生了太平天国运动,可以说当时民族之间的猜忌和隔阂是势同水火的。

所以瑶汉混居是完全不可能的事情,即使有瑶族人肯接受汉人在寨子中定居,那汉人也必然得住在瑶房内,绝对不可能有瑶王会允许汉人在瑶寨里盖这种耀武扬威的大塔楼子。

我完全无法理解,这简直好比在高粱地里发现一颗西瓜!

缓缓下沉,静静地看着这一幢古楼,又发现了更加蹊跷的地方,这座汉式的古楼完全被包在四周的高脚楼内,而且楼顶的瓦片颜色一模一样,似乎是被高脚楼刻意的保护起来,从外面看,根本发现不了里面有一幢这样的古楼。

再看这汉楼的规模,非常奇怪,呈口字形状,口字中间是天井,四周是三层的楼宇,底座和外墙全部用条石修建而成,学建筑的明眼人一眼就看得出来,此乃明清时南方大户人家沿街大宅的风格,一般都是当地望族修建的家族院落,有好几近深,后面还有园子和更多的建筑,巨大条石则是防土匪响马用的,这种无比结识的建筑,能保护深宅大院里几百号人自锁自持的生活。

也就是说,这幢古楼应该是一幢幽深大宅子的前脸,它的门对着的是正中街道外面的高墙,围住整个古宅,四周有大门、小门、照壁,有些门让下人进出,有些可能是沿街做生意的店面。大门进来后,有复杂的回廊通往后进的宅院。最典型代表就是杭州的湖青鱼躺。

然而这里只有这么一幢独楼,好像之后的部分被一刀切断了,整个古寨就剩下一个脑袋。

我绕着楼缓缓游了一圈,确实如此,后面就是青石板街道,四周都是瑶家的高脚楼,没有任何其他汉式建筑的样子,不可思议至极。

类似情况也不是没有见过,解放后,一些大宅被分到穷人手里,一个楼里住着几十户人家,后面院子的通道就被堵了起来,前后本是一个宅院的屋子,由此变成许多个独立的单元。但这里的状况显然不同。

我读了这么多书,尤其对中国古典建筑有深刻的记忆,脑海中无数的概念闪过,却始终无法找到任何自认能过关的理解。外行人可能会觉得小题大作,对于我来说,却是如鲠在喉,他娘的这楼是谁盖的?为什么要盖成这个样子?

青色灯光就来自于这幢汉式的古楼内,在我到来之时忽然熄灭,难道是宅子中的“人”发现了不速之客?又或是想告诉我,这就是我的目的地?我甚至想着,这是汉式的寨子,其中的鬼魂应该也是汉人,那么也许能念在同族的情分上放我一马。

不管怎么说,我都必须进入这古楼中一探究竟,无比的疑惑甚至让我不那么害怕了。

浮在天井上方,下面犹如一个巨大且黑黝黝的井口,把探灯开到最亮,往下照了照,既没有看到能发光的东西,也没有杂物。

我不再给自己恐慌和想象的时间,强逼着定了定神,翻转身子,头朝下挥动脚蹼,往天井潜下去。

空间一聚拢,光线就亮了起来,很快调了光度,使得眼睛能够适应,完成之时,人已经降到了天井院内。

感觉一下就不同了,四周漂浮的白色颗粒,全是因为我下降鼓动水流而飘起来的,下面确实满是沉淀物的石桌石椅。探灯往四面照,天井的四角都有大柱子,中间两边各有两根,一共十二根,往内是木石的回廊,再后头就是房间,都是雕花的窗花,腐朽坍塌,全被覆盖成白色,看上去无比残旧。

木门木窗脱落腐朽,但奇迹般的,这里的房屋结构竟然还算完整,可能当时使用了相当上乘的木料。

转动探灯,四面都有门,前面是通往前堂的后门,后面是通往进院子的门,两边则是通往侧厢。门口的柱子上都挂着对联,对联的木料不如木柱子那么好,扭曲且长着真菌一样的木花儿,其中两个门的对联更有半截掉在地上烂了,只有前堂后门的保存较完好。

挥动脚蹼,把前堂后面对联上的附着物擦掉,是这么两句联:

〖已勒燕然高奏凯
犹思曲阜低吟诗〗

这是很普通的对联,但我看得出其联语的意思,表明了这座楼的主人有军功在身。楼的主人是当兵的?而且看规模,应该是个军官。

前堂的后门已经坍塌成一团烂泥,一处窗框裂出几条大缝,手一碰就成片碎成齏粉,在水中如烟雾般翻腾,好似随时会烟消云散。手电筒从缝隙里照进去,里面无比杂乱,都是坍塌的木梁和一些无法形容的杂物,可见内部被破坏得十分厉害。

隐约能看到中间的回壁,那是房间中央立着的一面墙,风水中,气从前门进来,不能让它直接就从后门出去,中间必须有一块墙壁挡一下,叫做绕梁,使气走得不至于太快,从而多在屋内盘踞,还有一说是这样一来,后面的开口就从南北向变了东西向,更利于走财位。

这其实是有道理的,万一你正在进行什么阴谋活动,肯定躲不掉,有块回壁,给了人周转的空间,就是有强盗进来,也多少有时间躲一下。

我小心翼翼游了进去,之所以先进前堂,是因为对联让我想到一件事情……广西、广东大户人家的前堂,大部分有牌匾和灵牌阁楼。那里的牌匾必然和主人的身份有关系,所以决定先去看看,找找线索。

进入里面,猛地一看,我却傻了眼。

探灯四处一照,发现前楼内部已经完全腐烂,木质的地板全部坍塌,往上看没有天花板,能直接看到最高的楼顶,尚未腐烂的只有石头部件和一些巨大的粗木梁。大量的杂物掉落在楼底,一片残破。整个楼的内部空间,犹如路边拆迁得只剩骨架的老楼房,又或者是一个巨大而简略的脚手架。

我悬浮着把探灯往回壁的上端扫,基本上都烂没了,上方只能可拿到一幅牌匾,也腐烂得非常厉害,我游上去,小心翼翼抹掉上面的附着物,里面的颜色彻底褪没了,只剩下土色突起的轮廓,隐约能分辨出四个字:樊天子包。

看不懂什么意思,落款却让我眼皮一跳,是……张家楼主。后面为年月日款印。

这种牌匾有可能是别人送的,别人如果不送,主人本身又是大儒或者风雅人士,便会自己写。这边的瑶寨之内,不太可能有瑶人会写汉语,还写得如此漂亮的一手毛笔字。这是十分漂亮的瘦金体,我做拓本这么多年,能看得出其书法功力十分深厚。这个张家楼主,很有可能就是古楼的主人。

“张家楼主……”我心中自言自语,“张家?”

张起灵,张张张张,是巧合吗?

脑子里浮想起之前发生的一切,这里找到的大量线索,似乎都和闷油瓶有若隐若现的联系,难道真和他有某种关系?

有意思!牛人做牛逼的事,奇怪的古楼,该不会是闷油瓶的老宅?这个张家楼主是他的祖宗?想想还真有可能性。

这个张家楼主能在山中修这样的大宅,显然家底雄厚,又能写一手书法,对联内容又极度附庸风雅,怎么看也应该是自比儒商大家的胡雪岩一类的做派,可这样的人家,为何会在偏远的瑶寨之中,修出一幢如此古怪的楼?是遭人迫害来此隐居,还是另有所图?

我忽然有一点小兴奋,觉得古楼之中一定发生过大量的故事,如果真和闷油瓶有关系,这一次就来值了!可惜再无其他可看之物,前堂之中应该陈列了很多的字书,现在肯定全部腐烂,要是有更多的文字就好了。看来只有一个房间一个房间看过去,找找所有的蛛丝马迹了。

瞧了瞧氧气表,还剩一半,要抓紧时间。我准备先退到天井,再想想去哪个房间最合适。

正想摆动脚蹼,突然后脑一激灵,背后亮起一团幽冷的绿光。

\chapter{绿光}

我几乎是条件反射,靠身体的第一本能就转过了身去。透过前堂的后门,就看到天井对面的后堂里,亮起一团诡异的绿光。光线从腐朽的雕花窗透了出来,朦朦胧胧地在水中“弥漫”。

绿光诡异非常,和之前如出一辙。现在距离如此之近,可以发现那光线有一些非常难以察觉的抖动。这种抖动让整个天井都青惨惨的,鬼气森森,似乎一下子进入了另外一种空间。

我咽下一口唾沫,遍体冰凉,心中的恐惧难以形容,就连脑子也有点不太好使了。该来的还是来了,想躲也躲不了!

我尽量镇定下来,一边朝那后堂靠近,一边告诉自己,既然到了这里,就已预见到这种情况。之前类似的情况也遇到不少,不是照样平安无事吗?我就不信这次能比之前的可怕到哪里去。

从前堂出大门过天井到后堂,只要二十步不到,不知是因为我浑身僵硬,还是时间感觉错误,足足游了五分钟才到。

后堂大门紧闭,窗户那里有几处雕花扇完全塌落,里面绿光弥漫,但是看不清楚。小心翼翼地往里照了一下,光扫过的那一刹那照出的一团阴影,几乎让我的心跳在瞬间停止。

本以为会是一张青色的女人脸,结果只是一个影子。

后堂和前堂完全是一样的情形,除了地面上堆积的腐烂坍塌物,几乎空空如也。后堂的中间也有一块回避,森然的绿光就从那横壁之后隐隐约约地偷出来。

这景象很像聊斋故事中的情节,破败的古宅,点着油灯的书生正在夜读,女鬼飘然而至,在宅外看着屋内的灯光。只不过现在换了个位置,书生在外看着屋内的火光,屋内还真有可能是一个当时被淹死的女鬼。

我将这后堂上上下下打量了一遍,弄清楚了大概的结构,以便万一发生冲突能够迅速跑路。正准备从窗户进入,青色光团却又迅速暗淡下去,直至熄灭。

我心中一紧,好像被人掐住了脖子一样,顿时屏住呼吸。

它察觉到我了?

脑子里闪过非常多的画面,猜测回避之后是什么样的情形,那只“水鬼”既然察觉到了我的到来,肯定会潜伏起来,准备发动突然袭击。

不对!自己完全没有任何胜算,就这么过去,万一真是水鬼,岂不是找死?

我现在孤立无援,也没有人知道我在这里,不说这后面真是水鬼,就是忽然脚被卡主,或者氧气耗尽,都肯定得死在这里,而且几百年都不会被发现。真的就这么豁出去了吗?是不是应该再仔细想想?

我一下就泄气了,刚才的勇气烟消云散,又不敢进去了。

自己是不是被恐惧弄昏了头?

现在这种情况,是否该先退回去寻找后援?

可是,如此一来,之前我所做的事情就都白费了。闷油瓶和胖子他们完全没有痕迹,就这么消失在湖底,此时如果上去,还有可能再次下水吗?就算再来,我还有勇气重复一遍刚才的过程吗?恐怕没了。那么,也许闷油瓶和胖子,就真的从我的生命中消失了。

这时我忍不住开始想念潘子,如果他在这里,会是多么大的推动力?我和他们这些人果然不同。原以为自己的经验已经够丰富,但勇气这种东西,好像和经验没有多大关系。

人在天井里,只要退开几步,摆动双腿,一直往上,不出几分钟就可以脱离古怪的湖底古楼,眼前的一切都不用再考虑。我却定在那里,犹豫不决,因为内心清楚知道,无论是往前还是往后,只要第一步迈出去,就不可能停下来了。

这时,眼睛瞄到一个东西,一只清晰的手印。

手印就印在窗框上,由于刚才实在太紧张,竟然没有发现。

这地方到处是沉淀物,这个手印如此清晰,显然是不久前才印下的。是我的吗?凑过去比了一下,见手印中有两只手指非常的长,是闷油瓶留下来的。

我先愣住,接着按手印的位置比画了一下,正好是掰开窗框的动作——闷油瓶在这里掰开过窗框?

从这里到我最初下来的地方有几百米距离,他脱掉了头盔,在没有样子的情况下,怎么肯呢过行进如此长时间?难道他也成了水鬼?

心中的不可思议越来越甚,可想到闷油瓶,心理忽然就一定。不是答应过要帮他的吗?如果他变成了水鬼,大不了我死了也变成水鬼,那水鬼三人组也不会太寂寞。要不是他过去几次救我,我早就死了,如今只是为他冒一下险,有何不可?我的命就这么值钱?

我勉强镇定了下来,说实话,这么说并不能让恐惧减轻,甚至还更加害怕,浑身几乎不受控制地颤抖,根本无法抑制,但心中的信念如此强悍,使得我及时当着这种恐惧,还是从窗户里游入了后堂内。

一进入,我立刻想着,这样是不是不太礼貌?是不是得先敲个门?这样人家兴许会念在我知书达理的分上,放我一条生路。想完随即就抽了自己一嘴巴子,让自己镇定点。

后堂和前堂里的情形一摸一样,一点一点地绕过那回避,绿光没有再亮起来。眼看几乎要看到回避后的情形,我却停了停,因为手抖得连探灯都快拿不住了。

颤抖无法抑制,灯光随着节奏抖动,使得面前的回避看着像要倒下来,只好用另一只手帮忙,强自迈出最后几步。

那一瞬间,全身的神经高度紧张,内心已经做好看到任何恐怖情形的准备,随着后面的情形真正映入眼中,甚至感觉到脑子里的血管都要崩断了。

然而探灯照去,只有一片白色的坍塌物,其他什么都没有。

我操!我有一种被人戏弄的感觉,人在极度的紧张下,并没有因为什么都没看到而立即放松,反而持续绷紧。

环看四周,发现整个内堂是完全封闭的,后面空空荡荡,应该通往后进大院的地方只有一道大门。刚才在外头看过,外面就是大街。

如果发出绿光的东西先前在这里,现在肯定还在,一定是躲起来了。

我屏息游了过去,做出防御的动作,望向坍塌物的下方,看看是否压着东西,但由于太过杂乱,辨不清楚。看着看着,突然瞄到唯一立着的东西,后堂回避后的角落里,有一道屏风。

屏风不知是用什么材料制作的,竟然没有腐烂,但是其中的枢纽已无法支撑,歪歪扭扭地倾斜,没了正形。探灯照去,头皮一点一点麻了起来。在屏风之后,印出一个古怪的人影。

\chapter{成真}

我一下子就僵住了,双脚发软,整个身子都脱力了,不敢再动一下,目光也不敢离开,探灯就一直照着那个方向。

在强力探灯的穿透下,人影相当清楚,让人毛骨悚然的是那人的姿势,这个人的姿势非常怪异,整个人几乎是直立在那里,整个肩膀是塌的,我第一感觉是这人和我一样浮在那边,但似乎那人影纹丝不动,只有窨尸才会那样。

当时的那种窒息感已经到了极限,这可能是我到现在遇到的最匪夷所思的事情,这要是在陆地上,能有无数种解释,可这是在湖泊的水底,水深六七十米的地方,这个影子悠悠地站在那里,一动不动,绝对不是什么潜水员。

这到底是什么东西,是妖怪,还是水鬼!

没有人能不用氧气瓶在水下生存,也没有人可以在水下这么站立。我心里发毛,这次他娘的真的撞了大运了,给阿贵说准了恐怕真是只水鬼,由不得我不信了。

想到水鬼,我立即就想到了之前我们在寻找的那些尸骨:这是考古队的那些人死了之后在水里尸变的粽子?那是之前这村子被淹之后的亡灵?闷油瓶和胖子的失踪,是中了这些东西的招?

如果是粽子还好办,我全副装备怎么也不可能比它跑得慢,要是鬼魂,我恐怕就得要做他的替死鬼了。胖子他们如果遇难,也不知道会不会出来帮我。

我完全不知所措,不敢前进又不敢转身,因为怕一转身,这东西立即扑过来,我宁可看着它把我杀了,也不想忽然感到背后有异,只能死死盯着那影子。

然而,我僵直了片刻,却发现那影子纹丝不动,那种不动非常奇怪,犹如石雕,连一点移动都看不到。同时,我有了一种更加奇怪的感觉,我感觉这影子,他娘的好像在哪里见过。

这种感觉奇迹般的越来越强烈,似乎是潜意志在指引我,我鼓起勇气,那影子在屏风上的形状却开始一点一点变化。

冷汗又不可抑制地下来了,我看着那影子的变化,那种似曾相识的感觉越来越浓,甚至一度压过了我的恐惧。走了大概七米的距离,这种感觉已经到达一个极点,就在那一瞬间,我想了起来。

我的老天,这个影子,这个屏风,不就是楚哥那张照片里的那个影子吗!

在我来巴乃之前,我收到了一张照片,照片是三叔的老朋友楚哥寄给我的,上面拍摄的是一幢古老建筑内部的情形,里面就有一道屏风。而屏风的后面,也有一个人的影子。回忆起来,这人影,竟然和我现在看到的一模一样。

因为那照片后面写了格尔木的鬼楼,我当时判断那照片是拍摄了格尔木鬼楼里的情形,现在看来我错了,难道那照片后的注释不是注释那张照片本身的,那张照片难道拍摄于这里?

但是当时那张照片并没有任何水下的痕迹,也就是说,如果拍摄的是这样,那么照片拍摄的时候,这水下的古寨还没有被淹没。

那种照片应该最早也得是三四十年代的东西,难道这个古寨被淹没的时间,其实并没有我想的那么久远?

照片……影子……水底……难道楚哥给我的那张照片蕴含这我不了解的深意,而我只是把它简单地当成了一张信纸?他给我那照片,就是想我来寻找这照片上的影子吗?

我的脑子一下清明,随后又被无数的诡异年头充满。

让我脑子一片混乱的是那个影子,那张照片中,那影子的姿势如此怪异,但是现在这个影子,几乎和那照片中没有丝毫差别。

如果那照片拍摄的是这里,那就是说在拍完照片后,这影子没有任何移动,一直在这里?那就不可能是水鬼,因为当时这里还没有沉在水里呢,这影子应该是个死物。

我愣在那儿,忽然就来了一股勇气,找了一块砖头,摆动脚蹼,一下就朝屏风游了过去。快到屏风的时候我把砖头往屏风上一砸,心说去你妈的。但还没说完,我就后悔了。

屏风已经被水泡得根本吃不了力,石头砸在屏风的柱上,屏风一下子倒跨了,腐蚀物像雪花一样飘了起来,朝我扑面而来。我立即后退,拿着探灯去照,但是一眼看去全是漂浮物。我用手拨开把台灯往前照去,混乱间,从漂浮物中伸出一个东西来,一下子朝我扑来。

我立刻就炸了,挣扎着往后退,同时拿着军刺就开始乱刺,刺了十几下,什么都没刺到,嘴巴里的呼吸器反而掉了。

我手忙脚乱地抓回来,眼前的漂浮物已经被水流冲得散开了,我面前只是一根白色的浮木。

我骂了一声,一脚踢开,用探灯去照屏风后影子的位置。

那影子还立在那里,漂浮物逐渐稀薄了一点,它的真面目已经或多或少显露了出来。

那是一个人形的东西,有头,有手,有脚,站立在那里。浑身是白色的附着物,呈现着一个非常僵直的动作,好像是一具被僵化的死人,被吊了起来后,不知怎么蜡化了,尸体被包裹了起来。又好像是石像,非常难以形容。

它的面部完全被覆盖,也不知有没有表情,但看着确实是个死物,因为它如果能动,身上的附着物肯定不会积得如此之厚……

这是什么玩意儿?我心中的疑惑更甚。

\chapter{影子的真面目}

我看着那人形,莫名其妙的鸡皮疙瘩掉了一地。

第一眼的感觉,它其实是石像,但随即就意识到不可能,因为形态太过于逼真了,感觉真像是一个被固化的吊死的人。那个年代,就算有人要雕刻这种惊世骇俗的东西,也不会雕得如此写真,南蛮地区虽然有很多邪神,但多走夸张路线,也没有写实的。

一路过来的怪事如此之多,让我不敢轻视,搞不好刚才发出绿光的就是这东西,位置看上去也正好。

小心地靠近那人形,游近之后,蜡化死人的感觉更加明显,另一方面,我发现它的右手自手腕处断开,整个手缺失。不是一开始就铸成这样的,而是被破坏的。

小样!想学维纳斯没学到家啊!我迟疑了一下,小心翼翼地利用军刺刮掉上面的白色沉淀,想看看它本来的颜色。

刮掉一块一看,我吃了一惊,这东西本身居然是黑红斑斓的花色,但不很鲜艳,暗淡地纠结在一起。好比霉垢一样。再继续刮,就发现黑黑红红的斑驳霉花原来都是铁锈,这东西竟然是铁的。

不会吧!是具铁俑?壮着胆子用手捏了一下,果然是实打实的铁,有些地方可能淬炼得好,还没有腐烂,甚至能看到上面雕刻着非常精致的花纹,其他表面则完全生锈,都是暗红色的斑点。

我逐渐意识到了什么,立即将所有的附着物都从它身上刮落下来,一具造型非常其特的铁俑很快出现在面前。

我不由有点惊呆,因为刚才这东西给我的印象,是造型逼真但表面简陋,可现在再看,它的表面原来经过打磨抛光,虽然现在锈得不成样子,但能肯定之前非常精致,浑身都是优美的花纹,是一件艺术品。用手去摸,感觉到这些花纹和在闷油瓶床下发现的铁块花纹完全一样。

我明白了!那些考古队在水下打捞的东西,就是这个!那些铁块,就是这种铁俑的碎片。

这东西算是文物吗?有考古的价值吗?

转念想到闷油瓶说过这些铁块非常危险,我留了个心眼,不再去触碰,保持距离,仔细观察。

我对铁器毫无研究,但对鎏金铜器的认识颇深,铁俑在古玩市场见过,属于锡铁器,都是小件,从来没看过这么大的。一来古时候的铁很贵,这么大的铁俑,不说其他,就是耗费料斗非常惊人,二来铁器不容易保存,太容易生锈,有非常多的明代铁佛,其实都是中空的。

如果这东西整体的做工都和闷油瓶那铁块一样,基本就是实心的,里面可能包着东西,但也不会太空,可能非常重。如此重的东西,难道是佛教的大铁法器,锁什么妖用的?

我胡思乱想,但也知道怎么也不可能想出个所以然来,所有事情没有一条牵头的线,怎么琢磨都不会有用处。

本想看看铁俑身上的花纹,可锈得实在太厉害,根本看不到整体,其他地方也瞧不出名堂来。盘马曾说铁块很多,难道这里不止一个铁俑?

但四周空空荡荡,啥也没有,这种东西这么大,也不可能被压在那些坍塌物下面看不到。考古队带走的那些铁块,是从什么地方打捞起来的?

难不成这里的每间瑶寨之中都有同样的铁俑,分布于整个寨子中?还是说,藏在古楼内的其他地方?

下意识转头,看到后边的大门。

回想那张照片,屏风的一边,有一条走廊,我调整了一下自己的位置,发现照片上的走廊所在,在这里就是后堂的后门。

普通的老宅中,这道门后应该是第一进大院,可这里只有一幢古楼的前脸,所以这道门之后就出去了,外面是古寨的青石板街道,不可能是走廊。

然而记忆里,照片中的门框和这里的一模一样,毫无疑问了,拍摄地点就在这儿。怎么会出现偏差呢?难道在拍摄照片的时候,这里有走廊,但后来被拆了?

我的时间观念完全混乱了,看来那照片的拍摄时间,这古寨沉没的时间,都必须重新考虑。

游近去看,雕花的门完全没有腐朽迹象,拉了一把,发觉它外表仿木,其实是铁门。再用探灯去照,顿时一愣,没看到外面的青石路,这门后面,真的是一道走廊。

走廊不是平的,而是倾斜往下,通向地下深处,两边的情形,和照片上一模一样。

我愈加肯定照片上拍摄的地点就在这里,心中一个激灵,心道不会吧,如果是这样的结构,这后堂的后门连着走廊,走廊通往地下,难道这古宅是有后进的,但这后进的大院,是修建在地下?

\chapter{后半部分在地下}

我的概念完全被颠覆了,这幢古楼不光位置不太对,连结构都如此的诡异,通往后进的门后,竟然是一道往地下的走廊。难不成后面的整个大宅子全都修建在底下?设计者显然刻意做了手脚,可能后堂实际的长度,和房间内部的长度不一样,别人进来,看到这门就以为是后门,其实它离真正的后门还有一段距离,中间做了隐密的走廊。

大门开在地面上,其他部分修在底下,这还算是宅子吗?简直是老鼠窝。设计者真的太有想象力了。

忽然就想起了一句话,是三叔很久以前和我说的,深山里盖别墅,不是华侨就是盗墓。这儿算是深山了吧?这深山中的古宅,莫非是个盗墓的假楼?好比经常听说有人在古墓上头修一猪圈,然后来掩护盗墓一样。

表面上看,实在太切合这种说法了,从走廊下去,可能就是他们正在墓掘的古墓,这些铁俑是从古墓中挖掘出来的陪葬品。

但再仔细一想就知道不可能盗墓贼的脾气我了解,哪怕是最有实力、性格最古怪的盗墓贼,也不可能为了盗墓而修建一道如此结实的走廊!这一看就是非常有经验的工匠所修建的永久性石街,而非临时起意。

况且,为什么要在瑶寨里修汉式楼宇?假楼的存在是为了隐蔽,不让人注意盗墓活动,在瑶寨中搞一个汉楼,那不是更加显眼?

依这种思维,最好、最有效率的办法,应该是在此地修建一个瑶族高脚楼,然后在晚上直接挖个洞下去,修建一幢如此高大结识的汉式古楼,耗费的时间和金钱,可能远远大于盗挖一座古墓的价值,也太张扬了,完全没有必要。

非要这个说法可行,只有一个可能性,就是下面的东西价值大得无比惊人,而且极难进入,可能要二十年、三十年以上的经营。但我也基本能肯定,这下头不可能有什么大墓,因为此地正好位在山区低洼处,所有的地下水全往这里走,根本没法修过大的墓葬。

从我学建筑的一些知识来说,还能肯定一件事情……这座建筑似乎是为了某种特殊的用途儿特意修建的,所有的特徽都在为这种用途服务,目前不知道这个用途是什么,所以无从判断,但这用途的核心部分,应该就在地下。

看了看氧气表,所剩无多了,最多还能坚持十五分钟,没有时间再耗着,再看这道走廊,好像并不太深,十几步之后就放缓了,下头是青砖的地面。

青色的光没有再出现,也没有任何的危险气息。我想,就算是水鬼,似乎也没有什么恶意,而且好像在刻意指引着我进行这一步又一步。如果真要取我性命,我恐怕早就死了。

之前的经历让我觉得自己有点窝囊废,于是定了定神,小心翼翼打开那扇门,朝漆黑一片的下方游去。

来到底部,拿探灯一照,我立刻就吸了口凉气。

下面是一间砖头砌成的地下室,不大,非常的狭长,长度很夸张,我在这里看不到另一端的尽头。

砖室的两边摆着很多的铁架子,上面一具一具地平躺着无数铁人。

这有点像龙羔子,两边的铁俑好比刚烧好的瓷器,全部陈列开来,在黄色的探灯光下,铁俑又好像一具具尸体,大有国外大教堂,秘藏地下室的感觉。稍微一估计,最起码有六百具。

难道这里以前是一个铸铁人的工厂?

这地方的沉淀物少了很多,很多铁锈斑斓起了锈鳞,看着像腐烂的的黑色尸体。

一路过去,我发现铁俑的动作都不一样,更诡异的是,所有的铁俑都没有右手,所有的右手都被破坏掉,撕口很不规则,似乎是人为的。

之前的极度恐慌已经让我麻木,警惕着四周,继续贴着地面往前。一直到房间的尽头,并没有见到想象中的地下庭院,而是一面封闭的墙,只在尽头的砖石地面上看到一口井。

在地下室挖一口井,而且是在水源充足的广西,那是脑子烧坏了的做法。再看井旁修有凸陷的、便于攀爬的阶梯,立刻就明白了下面有东西。

此时,先前的预判开始动摇。这太像假楼盗墓的迹象了!也许底下真的是一个古墓,也许就是有这么一个老瓢把子,性格非常古怪,喜欢花大价钱在盗墓上面盖超级豪华的房子,甚至盖得比下面的墓还豪华,还希望把房子造得极度与众不同,让别人越注意越好。

也许还真有一皇陵修在了地下水超级丰富的地区,他娘的海里都有人修呢!凭什么就不许人家泡在水里?

我拿探灯往井里照去,如果这是一个盗洞,如此的结构足以确定,古墓非常难以进入,又需要修筑一条走廊,以便大型机械火很多人同时工作,墓应该是在别人的房子下面,他们只好探取迂回的办法,而非直上直下。如此这般,这伙人肯定不完全是专业的盗墓贼,很可能是一个人非常多而且龙蛇混杂的队伍,如此想来,很像过去那些盗墓军阀的作风。

军阀在当地视力及其庞大,想在瑶寨里修个楼,没有人敢说不,同时,和瑶苗的关系又很紧张,万一让瑶人知道他们在寨子里盗墓,难保会民族矛盾激化。

一方面要快,一方面要藏,如果地下的坟墓巨大,为了节约时间,的确可能修一条结识的走廊,便于打量人员进出。再对照上头的那对联,这张家楼主有军功在身,还真有这个可能。

想得觉得自己挺厉害,再见井下幽深,看不出什么名堂来,我背着氧气瓶,没法下去,便准备把身子撤回来。

这时,井下幽幽地亮起了绿光。

我心里咯噔一下,有来了!这一次能看到光离井口很近,只有两三米。想用探灯照,没想到还没动作,那绿光先动了,瞬间朝我冲过来。

我立即举起军刺,心说动真格的了!但那绿光来势太快,猛一下便如流星闪过耳边。

那一闪间,什么都没有看到,但我立马肯定这不是幽灵水鬼,更像是一只发着绿光的动物。

急忙转身,只见绿光闪入了边上一只铁架子里,一下子就灭了。用手电筒去照,只一闪又亮起来,像在和我的手电筒光呼应。

终于,我看到了那东西的真身。那好似一只无比肥大,犹如四脚蛇一样的灰色东西,有我的胳膊长短,正趴在一个铁俑的头上,身上好似绑了什么东西,定睛一看,居然是一只手电筒,正幽幽地发着绿光。

\chapter{胖子的小聪明}

我不知道这东西是什么,好像是一种大个儿的娃娃鱼,以前在老家吃过,但从没见过这么打的,看着非常瘆人。至于那手电筒,我一见太阳穴就一跳,正是之前裸潜的时候用的老黄皮手电筒。

这肯定是胖子他们带下来的,看那娃娃鱼身上的线,也肯定是人绑上去的,难不成是胖子他们的杰作?

我脑子一转,一下就明白了怎么回事。没有人会莫名其妙地这么干,胖子很可能是想让别人注意到这只娃娃鱼。

难道他们被困在了某个地方,只好通过这种方式求救?

打死我也没想到,那青光是这种东西发出来的。他娘的!手电筒光怎么变成绿色的了?

一下放松下来,人顿时瘫软,浑身都松了劲。看来我想的没错,他们在水下真有奇遇,现在很可能还活着,只是被困在某个地方,不得不透过这种方式求救。这个地方很可能有空气,但是为谁所隔断。

虽然不知道胖子和闷油瓶在水下到底经历了什么,又是怎么到达那地方的,但能摘掉他们很可能还活着,感觉太好了!

以胖子那种鬼精性格,娃娃鱼上面可能还有关于他们近况的线索,得把它逮住才行。可人在水下手脚很不方便,再看那东西游动的速度,恐怕够呛。

娃娃鱼是水中一霸,咬人非常厉害,而且这个头也太大了,一口下去,恐怕我的手指都得交代。

不管了!再戗也得试试。

我举起军刺,缓缓地游过去,尽量地慢,但只靠近了一米多,嗖的一下,那东西猛地一摆尾巴,闪电一般游出去六七米,停到了砖石的另一边。

靠!这东西就算在岸上用鱼叉都不一定能叉中,更不要说我现在得在水里用手捉了。好在它看似温顺,没来攻击。

我还想尝试,继续缓缓地靠过去,这一次几乎挨近它了,但就在伸手的哪一刹那,它又迅速地闪到了另外一个地方。

我立刻意识到自己在水下不可能抓到这种东西,它滑动尾部形成的水流很有劲道,不难想象爆发力有多大,即使抓在手里,凭我的力气很可能也制不住。

氧气灯发出警报,我有些急躁,用手电筒四处去照,想找找有没有可利用当工具的东西。偏偏周围什么都没有,那些铁俑重的要命,就算有用我也举不起来。

念头一转,想到自己带下来的那根军刺。这可说是我的精神支柱,虽然从来没用过。

我实在不想伤这条娃娃鱼,怎么说它也是一个生命,但到了这个时候,心中无比急切,再管不了那么多了。人的恶性一旦上来,什么怜悯都是空话。

我再次游过去,举起军刺就像把它钉死,就算一下钉不死,至少让它受伤,没法再游得这么快。

它停在了铁架子的脚下,趴在上面的青砖上,我屏住呼吸,浮尸一样缓缓漂过去,一点一点地靠近。眼看来到离它只有半个胳膊的地方,犹如电影的慢镜头般极度缓慢地举起手里的军刺,挪动到差不多的位置,便想刺下去。

可能是我的杀意被娃娃鱼感觉到了,它嗖地往前挪出几公分,与之同时,我心一狠,军刺一扎,刺在了它的尾巴上。

那东西尾巴上全是肉,疼得卷了起来,力气果然非常大,军刺几乎脱手。我追上去,一把抓住上面的手电筒,但在水下阻力太大,一下没抓实,娃娃鱼竟把尾巴直接挣断,飞也似的游出去六七米,这一次不再停下来,往砖室的另一头逃去。

没了尾巴,它的速度明显慢了下来。

我摆动脚蹼往里追,好几次他都差点被抓到,但人在水里,这样一抓的精确度实在太低,总是在自认肯定能得手的情况下被它逃脱。如此连追几十米,我先前已在水下潜了这么长时间,体力就跟不上了。

我死死地咬住呼吸器,用手拉住铁架子借力,勉强跟着。青光忽然一个转弯不见,顺着消失的弧度扑过去,就见墙壁上的青砖空出一个洞,伸手进去,立刻摸到手电筒,但却怎么抓也抓不出来。

娃娃鱼一定死死抓住了里面的砖壁。

我蹬起双脚。顶住砖石的两边,用全部的体重往后翻,就觉手上猛然一松,手电筒被拔了出来。整个人紧接着一个跟斗甩翻出去,撞在后面的铁架子上。

好不容易稳住姿势,往手里一看,绑着手电筒的绳子,原来正是胖子旅行包上的尼龙丝,那东西吃不了力气,断了。

再用探灯照了照洞,娃娃鱼窝在里面,看样子是不肯出来了,我也懒得再理,急忙把手电筒放在探灯下,想看看胖子是否另外做了手脚。那上面果然刻了几个字:SOS,跟着虹吸潮。

翻过来,后面还有一行小字,但已看不清楚了。

\chapter{玉脉}

这几个字刻得非常粗糙,字形丑陋,但是极其用力和清晰,手电筒都被刮得变形了,可能就是它老是一明一暗的原因。

手电筒的玻璃罩上,糊着厚厚的一层防水胶布,颜色是绿色的,青光他娘的就是这么来的!我不由得暗骂,死胖子把我魂都下没了,就是有搜索队看见,恐怕也会吓死。

这几个字的意思非常的明白,就是告诉我,他们还活着,但是需要救援,找到他们的线索就是虹吸潮。

这种手电筒最多的用电时长不会超过十小时,现在还能发光,光线这么亮,绑到娃娃鱼身上的时间就不会长,他们一定还活得很好。

胖子这家伙真是不得了!这娃娃鱼到现在才出现,显然是他判断出我就会在这个时候下水。可是,这里离失踪的地方起码一千多米,他们是怎么不用氧气瓶而到达井下的?

算了,我不愿意细想,只觉得整个人都清明了,一块隐隐约约地打石头终于沉了下来。能知道他们肯定还活着,其他的也就不重要了。

现在最重要的,是把人救出来。

之前在岸上看到虹吸潮现象的时候,推测这湖底可能与地下河有相通的口子。现在再看,推断是正确的,而他们受困的地方,就在口子附近。

胖子说顺着水流,但虹吸潮还没有开始,怎么可能有水流?

我甩掉手电筒,想用手去感觉四周的水流,可冰凉的湖水让我的手一片麻木,感觉粗糙的东西还可以,敏锐地感觉水流完全不行,而且就目测,水流是静止的。

又想了想,有了一个办法,抓起一把铁人上的沉淀物,让它们漂散在水中。

探灯的光线下,白色的悬浮颗粒一下扩散开来,我仔细看着,它们在水中渐渐平静,然后,极度缓慢地,开始朝井口移动。

果然!这里有着非常非常缓慢地水流,向着井的下方。

虹吸潮还是存在的,只不过微弱到肉眼无法察觉。再看方向,现在另一边的氧压可能很低,使得这里的水流在往那里反吸。

看了看氧气表,还有一些时间,我只带了这一套氧气设备,如果这一次找不到人,可能要等阿贵把其他设备运进来才有第二次机会,就是两到三天后,我必须确认他们能不能坚持那么多天。如果有可能,这么短的距离,我希望能够把他们一次带出来。

估计了一下时间,氧气表为零之后,里面的压缩空气还可以坚持二十分钟。只要把回程的时间控制在十分钟左右,我能用来探索的时间,最少还有十分钟。

事不宜迟!我解开身上的氧气瓶,用手提着先沉入井中,然后一头栽下去。

井内非常狭窄,好在挖得笔直,一路往下沉去,看着高度表,很快氧压已经超过七个大气压,深度快接近九十米了。

头朝下,身体的不适感达到极限,之前是精神非常紧张才没有感觉出来,现在只稍微轻松了一点,令人极度窒息的压力所带来的恶心,立刻开始在喉口泛滥。

这时决不能吐,我体内的器官里有气体,一吐之下,受到压力的影响,积物反而可能全冲入气管,我只得硬生生忍住,几乎是用上全身的力气,把注意力转移到探灯光的光斑处。

不久后,青砖消失,露出了岩石的脉络,显然他们的工程只做到这里,底下就是单纯的挖掘。也在这时,我开始感觉到不妙,听到一种奇怪的声音从井的深处发出来,水流速度则在一点一点地变快。

越听越感觉不对,好像是非常湍急的水流声,正想停住好好听一下,下面的氧气瓶忽然被一股力量拔动,抖动起来。

我是用牙齿咬住呼吸器,让呼吸管挂着氧气瓶的,本来就很吃力,这一抖动,一下没咬住,呼吸器就从嘴里脱了下去,往深处沉去。

我立即冲上前去抓,好在做了保险措施,有条带子挂在脖子上,便想拉着带子吧氧气瓶拉上来。没想到氧气瓶沉下去一米多不到,竟然以极快的速度消失在我的视野里。

那一瞬间,我看到了井的底部,原来井道下面是一条与井垂直的水道,当中的水流非常湍急,一下就把氧气瓶吸走。刚想大骂,氧气瓶连着脖子的带子先被抽紧,力道之大,几乎要把我的脖子勒断。

眼见自己整个被扯着往激流里去,我牙一咬,想用脖子的力量(想)把氧气瓶拉出来,但是只坚持了几秒就知道不可能,而且因为颈部的血管被卡住,脑子开始供不上血了。

我心中臭骂胖子,怎么没把这个写出来?又想单手把带子解开,但解开了不也得死?此时我已经快无法思考了,干脆手脚一松,往下一沉,先顺着水流再说,反正胖子也让我顺着虹吸潮前进。

还没等有什么感觉,人便被一股极大的力量往下拉,半秒钟后,已经被拽进了水道里,打着转儿被水流带着走。想保持住姿势,肩膀却连连撞着四周,好在水道本身有两三个人宽,而且常年被激流冲击。十分光滑,否则要有什么犄角旮旯,这两下肯定皮开肉绽。

也巧,氧气瓶在水里打转,也转到了边上,稍微一个迟缓就和我撞在了一起。我此时已经气短,几乎坚持不住,不管三七二十一就拽住它。

好不容易在湍急的水流中找到那条蛇一样的呼吸管,急忙塞回嘴里。还没吸上一口,却到了一个急泻而下的下坡,我直接几个大翻转,脑袋一路像弹珠机一样弹着洞壁就下去了。

这一摔直接把我摔懵了,好长时间不知道是怎么回事,就是本能地死死咬住呼吸器,也不知道又往前被带了多久,忽听一声巨响,前面的氧气瓶撞到了拦着水道的什么东西上。

我清醒过来,想保护脑袋却没时间反应,随即也撞到那东西上,一声闷响,撞得七荤八素。我也没有氧气瓶那么有弹性,一撞之后,只能被水流死死地按在那儿。抬手一摸。发现是个铁栅栏,用劲全身的力气转过身再摸着,没有缺口。

这里难道就是目的地了?抬头一看,四下却没有任何通路,全是结实无比的岩壁,死路一条。

我不相信,调整了一下姿势,用探灯仔细去照,确实没有。

他娘的!真奇了怪了!胖子说顺着虹吸潮就能找到他们,怎么现在是死路?

再一想,顿时出了一身冷汗,心道糟糕,难道胖子的所在地是水道中段?我刚才被撞得浑浑噩噩,已经错过了?

方才速度太快了,我根本没想过去看四周的情况,而且也不可能有这么快的反应速度,在这种情况下发现什么出口,然后立刻进去,好在我感觉自己被冲下的时间不长,那个入口如果真在通道里,应该距离不远。

这地方比较宽敞,我背上氧气瓶,开始尝试顶着水流抓着岩石往回走。可才走了两步,便意识到有点要命,水流太快了!就是有手抓的地方也得用尽全身力气才能移动,更何况岩石壁还光滑得要命。

我用尽各种办法,尝试各种角度,结果都是失败。最成功的一次大约离开铁栅栏十步,但脚一打滑,立即被打回原形,且彻底筋疲力尽。

被水流压着,我越来越感觉到不妙。

这儿看着似乎很普通,但绝对是个绝境,我等于被困在了一个没有牢笼的地方。胖子怎么没把这细节写下来?

要真出不去,这次就被他害死了!

看了看氧气表,数值已经无法显示了,显然随时都可能用完。我有点慌,把住铁栏杆用力摇动,想看看能否拆下来往后走,却发现全都是用铁浆浇在石头缝里的,结实得要命。

后面一片漆黑,探灯照去,就见水道急剧下降,水流更加湍急。也许正是为这个原因才在此地修起铁栏杆,怕人被卷入到更加狭窄的水道里去。

一时间我真的慌了,连呼吸器都有些咬不住,连忙深呼吸,告诉自己镇定。

以前总是能在被困的时候想出什么办法来,因为人是一种只要有一点希望就能发挥出巨大潜力的生物。我开始迅速思考,同时不停地看,不停地摸,想找到一丝灵感。

一开始我还信心满满,认为天无绝人之路,但让我绝望的是,这一次和以往都不同,虽然是开放式的环境,但十分的单纯,摸了半天,只是更确认了自己不可能战胜水流,也不可能拆掉铁栏杆。

尽管继续思考,但我心里已隐约出现一个念头:这次逃不掉了!

必死无疑。

\chapter{奇洞}

接下来的那几分钟,我这辈子都不会忘记。

一片漆黑的水道中,没有任何怪物,没有任何恐怖的东西。虽然我的主意识海不想承认,但潜意识已经很明确地知道,自己在很短的时间之后必然死亡,真真切切的死亡,这一次逃不掉了。

这种感觉的可怕,言语根本无法形容。

我忽然对自己之前做的所有决定感到后悔,一方面又想告诉自己不能放弃,要争取到最后一刻,但内心已经完全绝望,脑子不受控制地出现各种各样的念头。接着开始走神,一下想着当时如果浮上水面,现在会是什么情形?一下想着如果我死了,我的家人会怎么反应?后悔和恐惧让脑子一团混乱。

氧气表早就没有了数值,无法确定什么时候会窒息,只能一边尽最后的努力,一边等着那一刻的到来。

到最后关头,我几乎是期待着那窒息的感觉一点一点地出现。随着能吸入的氧气越来越少,一切都被拉长。恐惧让我痛哭流涕,根本无法镇定,脑子里就只有一个念头:我要死了。

很快,氧气完全耗尽,我还是不停地吸着呼吸管,但是什么都没有了。憋着最后一股气,一直憋到极限,在剧烈的痛苦下,我下意识地用了嘴呼吸,一股酸呛猛地冲进肺里,整个人顿时抽搐起来。

这是在水下,我没有第二口气来呛出肺里的水,呛过几下之后,那种酸麻便弥漫到整个肺,只觉胸口像要炸开。

我无法形容之后的感受,也根本不知道自己挣扎了多久。缓缓地,这些感觉都远去了,四周安静下来,眼前的光慢慢缩小,耳边听到了一些奇怪的声音,好像有人在说话,又好像是水声。

下一瞬间,一切都暗了下来。

那一刻,我以为自己死了,再没有任何的转机。不是死在粽子手里,反而是淹死的。爷爷说的真的很对,既然死在粽子手里也是死,淹死也是死,为何要怕粽子而不怕水呢?人真是讽刺的动物。

好在最后的平静感还不错,如果所有人死时都能这样安详宁静,那么,对死亡本身便不需要多恐惧,反倒是死亡前的那段时间比较难熬。

当再次苏醒,我最开始感到一丝诧异,但有很长一段时间,思考能力是无法运作的,所以这种诧异我无法理解,根本不明白这代表着什么。

逐渐、逐渐地,意识才恢复过来。

首先来找我的是疼痛,剧烈的疼痛一开始出现在手上,然后慢慢扩展,最后倒肺部。好像肺里有一张铁丝网,一呼吸就感到人又要死过去。

我吧所有的精力都放在抵御疼痛上,也不知道过了多久,才发现自己适应了。接着,其他的感觉逐渐复苏。

之前经历的一切这时才开始出现在脑海里,从防城港回来、下水、湖底古寨中奇怪的青光、奇怪的汉式古楼、铁俑、井下、最后的窒息……等等等等,一点一点都想了起来。随即心中就奇怪,自己当时必死无疑,怎么又醒了过来?

有一刹那,感觉那些好像是梦,我说不定一直都在这里睡觉,淹死的情形只是一场恶梦,但浑身的疼痛让我知道这不可能,自己应该是由于什么原因获救了。

尝试着动一下手,发现非常艰难,但能感觉大四周的潮湿,像在一块湿润的岩石上,耳朵和眼睛开始有了反应,听到耳边有声音并且逐渐清晰,有人在哼歌,而且……

是胖子的声音!

歌唱得极其难听,但我一下子就激动起来,立即用全身的力气想转头去看,结果疼得叫起来。

歌声瞬间停止,胖子叫:“醒了醒了!”接着眼前亮起来,一张长满了胡渣肥脸出现在面前。同时,我也看到了闷油瓶,站在胖子身后,举着火把。

我看着这两个猪头,起初还不敢相信。胖子开始说话,我的脑子仍不能很好地理解他说了些什么,但能清楚地意识到,这不是幻觉,我真的看到了他们!一下就百感交集,之前怀抱的剧烈恐惧、希望、担忧等各种情绪终于放开了,不知道该怎么表达才好,眼泪想流下来,却不由自主笑起来。

一个人经历了那么多的事,无比孤寂之中的剧烈恐慌,从死亡边上擦身而过的绝望,再然后发现自己安然无事,这种狂喜是能让人疯狂的。但我之所以百感交集,却不是为这个,我心里想的是:不管现在是什么情形,终于又和他们在一起了,终于不是一个人了!这种感觉太好了!

一边抽搐一边笑肯定非常奇怪,胖子显然以为我抽疯了,立刻把我扶起来,二话不说就抽了两个耳光,一双大手跟着用力敲我的背,说道:“喘气!喘气!深呼吸!”

他下手极重,我的闹戏嗡了一声,自谦的失控情绪一下就被打没了,再被他一敲,忽然就觉得急剧地恶心,开始呕吐和咳嗽,也不知道吐出来些什么。

吐完后,我艰难地转头看向他们,视力越来越清楚,各种各样的声音变得更有层次感。

“怎么样?还难受吗?”我听到胖子问。

我怕他再敲我,马上摆手,但说不出话来。

他明显松了口气道:“谢天谢地,你醒过来了。他娘的!老子以为你这次肯定得成植物人,那老子就罪过大了。”

“这到底怎么回事?我怎么没死?”我下意识就问。

“这你得去问阎王爷。”胖子道,说着把我扶起来靠在石壁上,让我放松。

我已经很清醒了,又看向他们,两个星期不见,两个人都好像在小煤窑当黑工一样,只穿着内裤,非常的狼狈,一脸胡子,而且瘦了不少。让我松了一口气的是,虽然他们的样子很狼狈,但是气色不错,显然没有受伤。

转头看左右,远处亮着小小的篝火,不知道是用什么搭的,照出了环境。这里是一个开凿出来的扁平的洞穴,大概有三十平方米打,站起来脑袋可以顶住洞顶,四处在渗水,像下雨一样,地面上都是湿的。岩石呈现出一种墨绿相间的颜色,在探灯的照耀下很漂亮。另一边还有一个半人高但很狭长的洞口,像被刀捅出来的,不知道通向哪里。

“我操!这里是哪里?你们出了什么事情?把我担心死了,还以为你们挂了。”我骂道。

胖子咧嘴道:“这说来话长,本来还担心你找不到我们。怎样?你是不是看到我那通讯员才找到这里的?”

说起那“通讯员”我就有气,恨不得一下掐死胖子,但心有余而力不足,只好作罢,骂道:“你那通讯员太他妈不敬业,差点把我搞死!”

“靠!我能找到那玩意儿就算不错了。”胖子问道,“你快说说,你是怎么到这儿来的?”

我听了好不来气,心说你问我,我怎么知道?“我不知道,我不是你们救上来的吗?”

胖子本来很兴奋,听我一说,突然面色就凝固了,“我们救了你?”

“是啊!”我于是把自己找到那娃娃鱼,随后下到井里的经过,全部说了一遍。

胖子听后露出很古怪的表情,回头看闷油瓶,闷油瓶坐在他后面的石头上,面色阴晴不明。

我奇怪道:“怎么?有什么问题?难道不是你们救了我?”

胖子缓缓摇头道:“你是怎么到这儿来的,你完全不知道?”

我一头雾水:“知道什么?”再看他们的表情,忽然感觉不妙,立即问,“到底怎么了?我身上出了什么事?”

胖子颓然坐到地上,骂了一声娘,似乎一下就被击倒了,叹气道:“你不知道,我们就更不知道了。”

我不由得恼怒,骂道:“到底怎么回事?你他娘的玩什么哑谜?快告诉我。”

胖子打了个手势,让我问闷油瓶。我看向他,就听他道:“大概五个小时前,你出现在你现在躺的地方,深度昏迷,几乎没有知觉。我们对你进行了简单的抢救,然后,过了五小时,你醒了过来。”

我等着闷油瓶说下去,他却闭嘴了。

“没了?”我诧异问。

“没了。”他闷声道。

“你没说你们是怎么救到我的。”我道。

胖子看着我,“你没听清楚重点,我们根本没有救到你。五个小时前,你出现在你现在躺的地方。”他一字一顿,“出现,也就是说,原来那地方什么都没有,突然你就躺在了那里。”

我皱起眉头,花了一些时间才明白他的意思,问道:“你是说,我是自己出现在这里的?”

胖子点头,“我和小哥一直在另外一个洞里,那里比较干燥,但是我隔一段时间会到这儿来取水。发现这个洞里忽然多了一个人的时候,我吓了个半死,但你胖爷我立马就认出了你,把小哥叫来,一起把你抢救了回来。你当时已经咽了气了,所以真要找个救命恶人,你胖爷我还是有资格客串一下的。后来怕你身上有什么骨折,我们一直不敢移动,就在这里等你醒过来。”

我看胖子的眼神,知道他不是胡扯,顿时陷入了沉思。

还真是没有想到的发展,我本以为昏迷之后有什么奇遇,被胖子和闷油瓶及时发现,然后获救,现在看却不是这样。然而我不可能在昏迷的过程中自己到达这里,也不可能透过瞬间移动来到这儿。这是怎么回事?

难道,救我的另有其人?有另外的人把我救了起来,送到这里?

那里是湖底的废弃井道,不可能有人打酱油路过,也就是说,有人在跟着我。

我和胖子说了我的想法,问他有没有这方面的痕迹,但他和闷油瓶没有任何的反应,似乎不认同。

胖子苦笑起来,拍了拍我,大声发泄道:“狗日的!这是不可能的,如果有人能把你带到这里来,那么他娘的,它首先肯定不可能是‘人’。”

“为什么?”我问。

他又苦涩地笑了笑道:“你能站起来吗?我带你在这个洞里走一圈,你自己看,就知道问题出在哪儿了。”

\chapter{洞里的问题}

胖子神秘兮兮的,而一边的闷油瓶始终没有说话。

我不知道胖子到底在搞什么鬼,但闷油瓶的态度告诉我,他并不否定胖子的说法。我心中的疑惑到达了顶点,决定先不去计较这些,看看再说。

想是这么想,但心有余而力不足,我被胖子扶着,哆哆嗦嗦的,要死死勾住他的脖子才能不摔倒。

所在的这个洞只有三十平方米,其实没有什么看头,火把转了一圈,都是人工开凿的痕迹,其他什么都没有。唯一特别的上面墨黑色的痕迹,不知道这里的岩石中含有什么矿物。

我跟着他蹚水猫腰,通过那一道好比刀砍出来的通道,走到另一边的洞里。

这里别有洞天,比先前呆的那个洞起码打了两倍,里面堆满了东西,都是一些生锈的工具,木头的架子背篓,还有堆起来的青砖,边上有很多我不认识的石磨一样的玩意。

让我吃惊的是,这个洞的角落里摆着几只高达洞顶的架子,上面就躺着那种铁俑。洞里的洞顶和墙壁上布满墨绿色的条纹,在探照灯的照射下更加清晰,散发出琉璃一样的光芒。

洞穴的中间,有一只倒放的罐子,上面是一个神像,不知道是什么神,前面还有几点的香炉,很简陋。

“这他娘的到底是什么地方?”我诧异道,看着好像是一个还在挖掘中的石室,工程只做到一半,工具盒、原料堆了一堆。

“我们猜测,这里应该是一个矿坑。”胖子道。

“矿坑?”我看着周遭,“什么类型的矿坑?”再看向那些铁俑,问道:“难道是铁矿?”

胖子摇头:“他娘的比铁矿可值钱多了!你来看。”他指向上面墨绿色的条纹,“你能摸出这是什么石头吗?你想想,这附近最盛产什么?”

我不是很明白他的话,摸了摸石上的纹路,感觉它出奇的温润光滑,简直像女孩子的脸。他没有瞎说,确实不一般。再一想,脑子里闪过一个概念,“我靠!难道这些石头是……翡翠?”

胖子点头:“我不内行,但依我看,就算不是翡翠,也不会是太差的玉石。这样该是一条非常好的玉脉。”

我啊了一声,脑子一跳,想起了之前在湖底石寨看到的各种奇怪现象。

这个山洞看来也是那奇怪古楼地下的一部分,之前一直怀疑这里的山中有什么,感觉可能最大的是古墓,没想到会是玉石矿。

这是没有想到,不过至此也想通了。

这里有一个隐蔽的玉矿,和古墓时差不多的道理,可玉矿的价值,完全不是古墓可比的。黄金有价玉无价,拥有一个玉矿,富可敌国。

这么一来,上面种种严密的布置,一下就完全和理了——如果是为了偷采玉矿,不说盖一座楼,就是盖一座城堡都不亏。

在这里盖这座古楼,甚至可能连瑶王都有股份,并用特权实施保护,玉矿的价值太大,没有任何政权能放弃这种诱惑。

至于为什么要藏起来?很简单,如果被任何其他地方的势力知道,肯定立刻发兵来打。这东西换成钱,能买多少鸦片烟土啊?

“这里发生的事,我看恐怕都和玉矿有关系。为了这东西,在恐怖的阴谋诡异也不算离奇,价值实在太大了。”胖子道。

“那这些东西是怎么回事?”我看着角落里放置的铁架和上面十几具横躺的铁俑,问道,“难道这些也是工具?他们嫌工头太苛刻了,所以把锄头修成工头的样子,然后天天砸?”

胖子半笑不笑,似乎没什么力气开玩笑,道:“我不清楚,不过你看这些东西,都是铸铁的工具,边上还有铁托子,我认为这些铁俑和我们走大货一样,是用来运东西的。矿石挖出来,直接封到铁俑里拉走,到当地再熔开。当时兵荒马乱的,这样做一来能防止路上出现意外,把玉石敲碎,二来上面有雕的花纹,防锈了再打碎,可以说是收来炼铁做子弹的。”

“哦!”我吸了口气,心说原来是这样。

蛇有蛇路,他们这种人一看就明白。我先前还觉得无比的纳闷,不由得有点失望,原来以为这铁俑背后还有更深的故事。

转念一想,又觉得有点不对。那些考古队打捞这些铁块,难道就是为了打捞其中的玉石?

不太可能,玉石的价值虽然大,但以当时的国力,应该不至于穷到让考古队去打捞,难道这些东西还有其他用处?

胖子只是笑笑,表情并不轻松。贴着洞壁缓缓走了一圈,我继续道:“不过,看这个矿洞的规模,他们好像没有挖掘出多少,开采的广度不高啊!”

“玉矿规模本来就不会很大,这不是问题的关键,”他将我扶的正一点,“你胖爷我想让你看的,不是这些东西。”

我转头继续看四周,并没有看到其他能吸引注意力的地方,便问:“你要让我看的是什么?”

胖子举起火把,问道:“你没有发现吗?这里没有任何出口。”

我陡然一震,前一秒还抓住他的意思,后一秒就明白过来。急忙环视整个洞穴,一看,冷汗就下来了。

确实,这两个洞都不大,刚才一路看来,没有见到能出去的地方。

隔壁那个三十多平方米的小洞非常简单,肯定没有出口,这里稍大一些,可同样也没有任何洞口。

我脑子有点乱,立即转身,胖子扶着我又将两边的洞穴走了一遍。这一次彻底专注在找出口,看完之后,只觉遍体生寒,几乎无法说话。

胖子说的没错,这里没有任何出口。所有洞壁都是整块的岩石,连一条缝隙都找不到。

“这怎么回事?”我看向他,“怎么会这样?”

他一脸的苦涩,不说话。

我下意识去看洞穴的顶部,洞壁没有,就有可能在洞顶。

洞顶非常矮,伸手就能碰到,环视一圈,和岩壁一摸一样,什么都没有,完全是整块的岩石。

胖子叹了口气,摆手道:“不用看了!这里里外外上上下下,所有的每寸每毫我们都找过了,这两个洞是完全封闭的。”

我无法接受:“怎么可能?”

胖子叹气道:“我不知道,但这确实是事实。这个洞,好像……”他顿了顿,语气有点迟疑,“是完全封闭的,好像是从内部被挖掘出来的。”

我呆了一呆,摇头道:“绝对不可能的,如果是这样,我们是怎么进来的?”

他让我靠在山岩上,看了看随后跟过来的闷油瓶,摇头道:“不知道。”

\chapter{封闭空间}

闷油瓶表现得和之前不同,有点古怪,一直不怎么动,靠在角落里,转头看向我,淡淡地说了一句:“我没有印象,但是我知道,事情才刚开始。”

在无比诡异的气氛中,胖子和闷油瓶把经历的事情跟我说了一遍,原以为会听到一个非常复杂的故事,没有想到,他们说得无比简单。

我离开之后,他们的行动和我预计的差不多,开始用阿贵带来的简易器械进行打捞。岩上的那些尸骨,是在枯树的枝桠里找到的,猜想可能是虹吸潮的关系,大的尸体最后都卡在了枝桠里,而抛入水中的装备在另一个地方,所以被挂在那片篱笆上。

失踪前最后一次下水,胖子是第一个。当时他已经准备上浮了,却看到有东西在手电筒的照射范围里闪了一下,似乎是某种金属。

下水本来就是为了打捞东西,他自然马上被吸引过去,可等游到那里,却发现那边什么都没有,只有一些大块的石头。

头盔里的氧气差不多耗尽了,他也不能仔细看那些石头的缝隙,以为闪光是小块的金属或者玻璃,于是没有在意,准备上浮。

就在这时,忽然感觉到“有什么东西咬了他一下”,手上立即一阵麻痹,几秒内就传遍全身。他心说糟糕,想冲上水面,但已来不及了,下一瞬就昏了过去。等醒来,已经躺在了这个山洞里。

闷油瓶的状况比他稍微复杂一点,但也差不离。他是去找胖子,所以下水很急,入水没多少时间,突然感觉到有点不对,想回头却晚了,在水下,他的身手再好毕竟也有限。

他的原话是:“我感觉到背后有东西动了一下,要回头已经晚了,醒过来的时候我也出现在这个地方。”

我心说奇怪,怎么可能发生这种事情?

一下就失去了知常见,然后醒来,发现自己出现在另一个地方,这好像是外星人干的事情。难道这里是飞碟内部?

再次看向石洞,四边全是岩石,如果真是飞碟,也是石器时代的。

我感觉到事情越发不靠谱起来,他娘的!胖子和闷油瓶被什么东西“咬”了一下,失去知觉,如果是中了某种生物的毒,就该淹死了,但他们反而出现在这地方,怎么看怎么不是神秘现象,太像是人为的了,应该是有人把他们迷晕了然后搬到这儿来。

但,如果是人为的,又怎么解释现在的处境?这是一处完全封闭的山洞,什么人能把我们穿透岩石塞进来?刘谦?

胖子想着那时的情形,还带着疑惑,“我很想不通,当时在水下视野不错,被扎之后到昏迷之前还有一小段时间是清醒的,我立即四处看了,什么都没有。”

“也许是一种虫子或者鱼,个子比较小,只要贴在你的背上,你就发现不了,你身上有伤口吗?”我问道,不可能平白无故地疼一下,若是被东西刺了,肯定有痕迹。

“刚醒我就看了,没有任何痕迹。”胖子让我看他被刺的地方,确实什么都没有。“我觉得不太可能是虫子。你想,连小哥都中招了,什么虫子敢咬他?”我啧了一声,这事情太邪门了,讲不通啊!所有的情节都讲不通,完全不像“人”能做到的。真是湖神在耍我们?

胖子继续和我说,这里唯一能出入的地方就是外面洞穴顶上的一条手腕粗细的裂缝。那支娃娃鱼就是从那儿发现的。大量的渗水从那裂缝而来,他们这两个星期基本上什么都没吃,就靠喝水活着,他瘦了大概六公斤,皮都挂了下来。为了不消耗体力,几乎都是静坐着不动。

外面另外一边还有一些过去开凿剩下来的木头架子,可以用来烧火,每天只烧一点,好在氧气不成问题。

之前我突然出现,他们以为我是看到了娃娃鱼身上上标志,因而找过来,并且知道了进出的方法。没想到连我也不知道自己是怎么进来的,害得胖子空欢喜一场。

我吸了口气,想起一件事情,问道:“既然你们是突然昏迷的,为什么会让我顺着虹吸潮走?你们怎么会认为顺着水流就能到这儿?”

胖子道:“是声音。我不知道这个没事所在的位置,但我知道肯定在虹吸潮的口子附近,因为到了晚上,外面的渗水就会有规则收缩,声音非常明显,好像呼吸一样。只有离虹吸潮非常近,才会有如此大的幅度。如果你发现娃娃鱼,被引到虹吸潮的口子附近,就可能会发现通往这里的裂缝。”

我不禁暗骂,原来是这么回事,也太理想主义了!

胖子的想法完全没有依据,事实证明顺着虹吸潮是死路一条,但我既然没死,也不想再埋怨什么。

听完之后,我颤颤悠悠站了起来,虽然绝对相信胖子,但内心的强烈冲动还是让我想自己看看这个洞穴,仔细贴着这些石头看看。

胖子看着就叹气,摇头道:“别浪费体力了。天真,你想想,他娘的我和小哥在这里困了两个星期了。这两个星期,我们能干什么?胖爷我刚开始也完全不信,一直认为可能有暗道,一直找,一点一点找,你知道把一块石头看一千遍是什么感觉吗?我看到最后几乎要吐了,但是,没有就是没有。”

他的表情非常的痛苦,我能想像出那种上感受,但不自己看过,心里就是感觉空空的,就让他别管。

吃力地扒着岩石壁走了一圈,这次看得非常仔细,胖子说得一点也没有错,岩壁确实完全是整体,偶然有细微的裂缝也是自然形成的,连刀都插不进去。最大的裂缝是外面洞穴的没洞顶,但也只有胳臂精细,源源不断的水从上面流下来,地上全是大大小小的水坑,这些水又顺着底下的岩石缝隙流下去。

这个洞穴的位置会在哪里?会不会在我溺水的地点附近?看这些凿痕,和那井下部的岩石痕迹很相似,肯定是同一批工匠凿出来的。

那么,我们就是在湖底下山脉的岩层中了。我到底不是学地质勘探的,只知道一些力学知识,其他的完全没概念。

敲击岩石,发出的都是无比沉闷的声音,似乎也不可能有暗道。而且闷油瓶在这里,如果真有暗道,他应该早就发现了。

又去瞧堆积在一旁的东西。刚才相看之下,角落里似乎有几只石磨一样的东西,走近了仔细看,好像是铸铁的炉子,里面还有铁渣滓,一边是放着大量工具的架子,稳妥得不成样子。

另外就是一尊大概只有啤酒瓶高的泥塑神像,是关公,又是别的菩萨,过往从来没见过,或许是少数民族的神灵。

尝试着手动了一下,不知道是因为我身体完全无力还是它太重了,纹丝不动。胖子就道他早就搬过了,下面没有通道。

走回胖子那里,终于确定他说得没有错,虽然之前便相信了他,但此时的确定是发自内心的。心里升起一股焦虑感,这是人对于封闭空间本能的反应。

\chapter{假设}

我一边脱掉身上的潜水服,企图尽快恢复体力,一边就问胖子,他们在这里这么长时间了,有什么推测?

他摇头:“我自己觉得最靠谱的推测,就是我们都死了,穿透岩石进入这个洞穴的,是我们的鬼魂。”

我苦笑,这话的意思我明白,并不是真的认为我们都死了,他想说的是,其他的推测比这个更不靠谱,这是没有前因后果的事。推测需要线索,但现在什么线索都没有,一切只能假设。

胖子道:“如果那作怪的东西,目的不是想杀死我们,那么,不管接下来发生什么,咱们总不至于送命吧!如果要杀,何必换个地方?”

我苦笑,不送命,那么是什么事情?难道这里会突然出现个大汉把我们强暴?我摇头道:“这没有什么必然的关系,现在活生生的未必是好事。你吃醉虾不也是图个新鲜吗?”

胖子吸了口凉气,想着确实悚人,就有点郁闷,骂道:“老子最恨这种摸不着,想不明白的东西了!你说咱们三个人是不是八字犯冲,怎么碰一起老走这种窑子?狗日的实在是魔障!还有那阿贵也真是的,啥也不知道,否则有点提示,也能提防点儿。”

我暗暗皱眉,胖子说得很对,这件事之所以一点头绪都没有,甚至无从推测,就是因为这样,现在的处境是莫名其妙就发生了的,在我们的已知里,肯定缺少了某一样非常关键的东西。

调查从村子开始,一点一点衍生,所有的讯息都是由上一级的讯息带出的,现在知道了铁块的来源就是那些铁俑,知道文锦来过这个湖畔,也确定了考古队被人掉了包,并晓得了湖下古寨的一些秘密,虽然其中的线索有些还没完全连上,比如说这些铁俑到底是怎么回事,但只要继续调查下去,我相信一切都会连起来。

但是目前在这里发生的事情,眼下的困境,却和这些讯息都没有关系,也就是说,我们在村子中了解到的多种线索中,完全地缺失了一块。

是在哪里漏掉了呢?

刚才我问胖子他的推测时,发现这件事没法推测,没有人噩耗可以佐证的因素。想着这些,我对他和闷油瓶说:“我们应该把知道的东西从头完全理一遍。这个地方和这整件事情肯定有联系,从头完全都列出来,说不定能找到点提示。”

胖子吹气,指了指地上,上面有他用石头刻字的痕迹,“我之前理过了,实在想不出来。你要理也好,你读的书多,应该比我好一些,我理到后来头都痛了!”

我看着那些字,正是他专用的枚举法,把所有的可能性全部都写下来,包括所有的线索,然后在那里画圈,找到其中的联系。

我道:“这一次和以往碰到的不同,所有的讯息都是碎片,你这么写,只会越写越乱。我先理一下,然后我们从一个概念开始,看着能不能搭积木一样把整条线搭出来。”

我捡了一块石头,在另外的地上写上了几个关键字。从进村开始,陆续发现的东西和后续的部分全部连起来。

〖铁块——铁俑的碎片——湖底的村子——不知是何用处——到处都有——似乎有危险——散发奇怪的味道
照片——烧毁
盘马的说法——考古队被调包——尸体找到——打捞铁块——目的?
水下的古寨——汉式古楼——地下通道——大量铁俑——玉矿?
封闭的矿洞——铁俑——同样的凿痕
A、B——刺痛——昏迷
C——窒息——昏迷〗

写完后,把那些已经确定的东西全部划掉,表格就变成:

〖不知是何用处——似乎有危险——散发奇怪的味道——目的?
——大量铁俑——玉矿?
封闭的矿洞——铁俑——同样的凿痕
A、B——刺痛——昏迷
C——窒息——昏迷〗

这样一来,我们能确定和不能确定的东西,全部都列了出来。

接着,我们始将其中一些因素连起来,道:“首先,我们先肯定,古寨里的汉式古楼的主人姓张,暂时叫他张家楼主。”我看了闷油瓶一眼,“这人有军功,而且是个国学大家,可能是当地的军阀,当然也可能是其他背景,和事情的核心没有太多关系。”

“在某年某月,这个叫张家楼主的人,因为某种原因——同样,这种原因我们不需要知道——发现着寨子底下有一个玉矿。在巨大利息的诱惑下,他伙同了这里的瑶王强挖,在瑶寨中修建一座结实的汉式楼宇,供手下使用。楼宇修得这么坚固,显然他们在这里的强挖时间非常长,可能准备几代人干下去。”

“我们现在所处的这个洞穴,看开凿的痕迹,应该就是他们挖掘的矿洞,至少是其中之一。”

说完我看向胖子,问他有什么要补充的?他摇头,我又道:“好,事情到这里一切正常,也都符合常理,可这就和我们现在的处境有了矛盾。显然目前所处的矿洞是全封闭的,所以我可以这么说,从一切正常到现在的处境,这之间的时间内,发生了一件事情,使得矿洞发生莫名其妙的变化。”

胖子点头道:“别说的这么文绉绉的,他娘的就是这洞后来出了事情。”

这一部分是最初的假设,也比较确定,我将其作为起点写下来,然后在边上画了一个问号,“这里出了什么事情?肯定不会是突然封闭,因为若是这样,会有人被困死。”

“非也,你想,我们进来都是莫名其妙的,他们说不定后来找到了出去的办法。”胖子道。

我摇头,那个年头的矿工是什么文化素质?他们能想到办法,我想不到?而且即使能想到,也不会太快,那么以他们当时有工具、有体力的状况看,应该会先有“砸”出去的想法,并在地面留下大量的碎石痕迹。

不过,我毕竟当时不在现场,不好下肯定的论断,就没有反驳胖子。我们咬着嘴唇,开始想各种往里套的假设。

还没想上两圈,闷油瓶就开口了,淡淡道:“矿洞中的神像,是瑶族的雷王神,是凶神,一般不会公开供奉,除非发生过什么可怕的事。”

我们都愣了一下,胖子道:“我靠!你怎么懂这玩意儿?”

闷油瓶不回答,继续道:“这东西在里面,说明事情不是突然发生的,而且发生后,还能从外面拿来石像在这里供奉,代表这件事虽然很可怕,但是不至于把他们吓跑。”

我想了想,觉得有道理:“设立神像,表明他们还想继续挖掘下去,所以用这个神像在这里镇压什么,事情虽然可怕,但只是心理上的恐慌,还没威胁到生命安全,咱们想想,换位思考,如果我们是矿工,在什么情况下也会这么做?”

胖子吸了口冷气:“这听上去怎么这么耳熟?难道,他们在这里挖到了不吉利的东西?”

我也点头,似乎在同时冒出同种念头,过去经常在老家听到这种传言,什么工厂动工,结果地基一挖,挖到了乱葬的死人骨头,就摆个关公镇一下。

“这里是岩层,这种狗屁地方能挖到什么?”胖子道,“难道是霸王龙的化石?”说完哎了一声,显然感觉自己的说法挺有可能的,“你想,他们挖着挖着,突然挖到这么个史前怪物,肯定吓个半死,以为挖到妖怪的骨头了。”

我拍了拍他:“同志,有空多读点书,恐龙化石的年代和玉的年代差了好几亿年,这里挖出恐龙化石,就好比肯德基全家桶一样。”

“那你说是什么?”胖子不服气道。

我们想了想都摇头,其实根本没法想,这种岩脉里能有什么既合理存在,有让他们觉得不吉利的东西?我真想不出来。这里合理存在的东西只可能是石头,难道是一块让他们觉得不吉利的石头?如果说不合理,那么什么都有可能。

胖子走到那神像面前,问闷油瓶道:“小哥,这累王神凶到什么程度?是不是和咱们的钟馗一样,是抓鬼的?”

闷油瓶摇头:“雷王,是专门克制邪神的。”

瑶苗神话和汉族的不同,其中很多邪恶的东西都是神,能和正义的神平起平坐,普通的神干不动他们。

胖子啧了一声:“也就是说,钟馗只是公安,这雷王是纪委会书记。”在一边的篝火里检出两根细柴,插进香炉里,拜了拜,“雷书记,不好意思,小弟们之前有眼不识泰山,一直没认出您来。这点东西不成样子,但也算是个形式,就当是张白条,要咱们能出去,小弟们一定把香油补上。我知道您搞纪委工作,很多东西收了不方便,回头您把您夫人电话告诉我,咱们跟您夫人联系……”

我心说这家伙也太不靠谱了,道:“你也不是瑶人,人家怎么可能会保佑你?别浪费你的柴火了。况且只有上级给下级打白条,哪有下级给上级打白条的?”

胖子道:“你懂个屁!你在杭州交税,去北京就不交税了?我这不叫白条,叫期权。咱们这叫先打个招呼,好过以后后悔。”

说着他转身,不想那细柴因为头重脚轻,一下子带动香炉倒了下去,根部翘了起来,香灰全翻出来。

胖子立即回身扶住,我笑道:“你看,人家清正廉明,不收。”

胖子再啧了一声,把细柴掰撕一半,重新插进去,然后把洒出来的香灰用脚擦平,擦了几下,随着香灰被涂开,我忽然看见,他脚下的岩面上,出现一些奇怪的线条。

\chapter{挖出来的是什么}

我感到莫名其妙,立即靠过去,把胖子的脚拨开仔细一看,果然,有一部分香灰嵌入到石头表面细微的缝隙中,形成一些线条。而且很明显,这些线条非常圆润,不是石头表面本身的纹路。

我是搞拓印的,知道这是一种拓印原理,用非常细腻的粉末来显示出地上浅痕的方法,类似于很多间谍剧里必用的,用铅笔涂抹便签纸得到写在上一页的讯息,显然有人在这神龛前的岩面上,刻过什么东西。

我兴奋起来,一下把香炉翻倒,把里面的香灰全部倒在地上、岩面上,开始用双手涂抹。很快,地面及岩壁开始出现更多细微的线条。

“这是……”胖子也发现了异样。

“应该是挖掘这个洞的工匠刻下的。”我道。

“我看,雷书记这么快就显灵了!”胖子道,“效率比咱们人间高多了。”

“你先别说的那么快。”我道,把灰全部都抹均匀。

他蹲下来帮忙,闷油瓶也凑了上来,我们把香灰涂满了一大片区域。很快,一片歪歪扭扭的文字出现在面前。

这些字每一个都有象棋大小,全部是繁体,刻得无比的潦草,有些几乎模糊不清,但数量颇多,有三、四十个,大大小小的。

看笔记,应该是一个人所刻。

文字是汉字,但其中有些字我从来没见过,应该是方言发音。

胖子疑惑道:“难道之前的工匠和我们一样,也在这石岩上讨论过东西?”

我摇头说不是,这些文字是连篇的,显然刻的人写的是一整段话,不过刻痕非常浅,和我们一样,应该也是用石头简单地在岩壁上划出来的,没有用到雕刻工具。

是什么样的一个人,出于什么目的,在这神像前写下这些字呢?无从想起,但关键应该在文字中。

我辨认了一下,文字是竖着读的,出去认不出来的,仔细地一个字一个字辨认,然后用石头重新刻在一边。

是一段很简单的话。

十一月又七日。

东墙,自左七尺,有十六。

西墙,自左三尺,有七。

北墙,自左五尺,有十。

南墙,自左六尺,有四。

细数,须三日内掘出复工。

“这是……采矿计量的记录?”我迟疑道。

看整个语感,好像是一处留言,一个工头离开之前,留给其他人的一点提示,并且有一个嘱咐:细数。似是上级写个下级的。

“东南西北?”胖子看了看四周,“是不是玉脉的分布记录?”

我摇头,玉脉的走向完全是自然形成,一点规律也没有,只在一个剖面上定什么左几尺没有任何用处。“有十六”,“有七”,“有十”,“有四”,好像是一种计数量的标记,他在数墙上的东西。

看了看东墙,上面什么都没有,只有玉脉和岩石自然地皴皱,深色的玉脈之复杂,简直有如岩石的血管,根本无法用“十六”这么小的数字来表示。而且他最后有一句:须三日之内挖掘出复工,好像是说那“十六”、“七”所代表的东西,阻碍了继续开采。

是什么呢?难道是石脉种无比坚硬的岩精?但是岩精坚硬的要命,且重达百吨,怎么可能在三日内掘出?

我们都站了起来,走到东面洞壁的最左边,用手指量了七尺的距离,看看那部分有什么东西。

七尺之后,还是岩石的表面,无数墨绿色的痕迹,什么都没有。

我和胖子面面相觑,其实,这里的岩面我们看的非常仔细,就算不这么看,也知道表面上瞧不出什么来。

“他上面写的东西,会不会已经被掘出来了?”

有这个可能,但再想了想,脑子里有了一种很奇怪的念头。

我回到神龛前,把地上的香灰收拢起来,放回香炉里,然后拿着到那块岩壁前,抓了一把,在上头涂抹。

一开始什么都没有,但等涂了几圈,果然,上面出现了线条,好像是某种东西的轮廓。

“哎?”胖子惊讶道,“你怎么知道的?”

“那种留言太含糊了,是汇总式的最后留言,肯定会在岩面上也留下记号。”我道,一边继续涂抹。

很快,一个不规则多边的轮廓在石头上显现了出来,我从身上解下我的强力探灯,打开。轮廓非常明显,好比画画打草稿的时候,先用直的短线条勾勒出物体大概的外形一样。

然而,我们并没有从岩石的脉络上,看出任何和这轮廓有联系的形状,好像是随意画在岩壁上的,用来做切割时的参考。

可即使如此,我还是感觉遍体发冷,脑子里很多碎片开始自发的进行各种各样的组合,内心已经知道,这岩壁里肯定有东西,否则,这轮廓不可能刻在这里。他们要把这里的东西挖出来,所以做了大概的标记。

为什么看不到?难道是方法不对?

想着,我问:“你们谁知道,他们采玉矿的时候,有什么特别的过程?”

胖子摇头道:“不是用炸药吗?”

闷油瓶却道:“先用火烧,然后用冷水泼,使石头自然裂开。”

“用水泼?走!去打水!”我立即道,也不知道自己到底想证明什么,但心中有一股极强的直觉,碰到关键了!

我冲到另一边的洞里,把脱下来的潜水服裤管打上结,然后往里面装水,再背回去,和胖子两个抓着往岩壁上泼。

如此连泼了十几次,岩石的颜色因为渗水而变深。

退后几步再看,由于泼了水,岩石表面玉脉的部分变得模糊,其他部分也变得光滑通透。原来这些石头也是玉石,只不过含量不同,所以被那些墨绿的翡翠称得像普通岩石。

同时,我们看到了,那块岩壁中,透出一个若隐若现的影子。

是一个人影。

\chapter{石中人}

刚分辨出的那一瞬间,还以为那是我自己的影子,动了一下,却发现那影子并不跟着我动。

我们三个犹如掉入冰窟中,看着那玉脉中的人影,都有点站立不住。

“那是什么玩意儿?”我轻声道。

“鬼才知道。”胖子用同样的语气回答,顿了顿,“好像……好像是个人?”

“怎么可能是人?如果是人,他是怎么到这岩石壁里去的?”我道。

胖子看了看我,哆嗦着问:“你有没有听说过石中鱼的传说?”

他才说完,我身上就冒出一连串的鸡皮疙瘩。

石中鱼是志怪小说中经常出现的故事,说一块完整的山石,被人打开之后,发现里面是空心的,不但有水,水中还有一条活鱼。

没有人知道这鱼是怎么进到石头里的,也没有人知道这鱼是怎么活下来的,石头中没有任何的食物。

这种现象往往被认为是神迹,石中有鱼,既然不是从外面进去的,那就是石头自己产生的。传说吃了这种石鱼能长生不老,但也有人说吃了即刻毙命。

石中鱼的传说很广泛,各在都有,似乎不是杜撰的,胖子现在突然提起,我当然知道他指的是什么意思,但知道归知道,实在无法相信那种说法能用到这里。

“不可能。”我道。

“既然石中可以有鱼,为什么不能有人?”

我吸了一口凉气,看着那石中的人影,还是摇头,“不可能,这肯定只是看着像人的阴影。”

“是不是,继续泼就知道了。那地上写的,这东西不止一个。”胖子道。

我们立即故技重施,很快把四面墙上全部泼满水。

随着所有的岩石都被浸湿,我毛骨悚然地发现,这附近的岩石里,真嵌满了人形的影子,有各种不同的动作。

洞壁的内部,竟然好像全镶嵌着人。

数了一下,和地上记载的完全一样。

“真是见了鬼了!”胖子重新坐下来,“难怪要雷书记出马,这他娘的是怎么回事情?”

“难道是昆仑胎?”我想起以前听说的天地生精的说法,难道这是个宝穴,翡翠在某种神秘的力量下人化了?

胖子摇头:“昆仑胎到底只是个传说,而且据说都是非常大的山体,这些影子形状诡异,我看不是什么好东西,而且……”他看向一边那个躺着铁俑的架子,“我刚才可能判断错误了,你看这些影子的动作,是不是和那些铁俑非常像?”

我已经惊讶的无法说话,胖子接着面色惨白道:“我知道这很惊悚,不过我看这里的这些工具,都是铸铁的工具,忽然就想到了这种可能性。”

我看着那些人影,“你是说,这些铁俑不是运输工具,而是用来封他们挖出来的这些影子?”

“恐怕不止这么简单。”胖子纠正道,“这些铁俑,大概是他们处理过后的东西。他们可能先在岩壁上面打孔,然后住里面灌入铁浆,把里面的人冻住,最后再砸出来。”

我想到在古楼的地下室里看到的无数铁俑,浑身都是鸡皮疙瘩,如果是这样,这里得挖出了多少这种东西?强笑道:“这都只是我们的推测。”

胖子的面色依然苍白,显然自己都觉得自己的想法很恐怖,又道:“其实有一个办法,就是现在把这块石头砸碎,看看里面这影子到底是什么东西。”说着,指了指一边的石工锤。

我摸着面前的岩壁,非常厚实,不是那么容易打裂的。忽然想起以前的镇妖传说,古代不是老是说,老天镇妖,喜欢把妖怪镇在山下?

我操!难道这些影子是妖怪?

要是这样,把它们放出来,岂不是找死?

我生起了剧烈的好奇心,伴随着那种悚然,同时摇头:“以前的工匠用那么费劲的方法来处理,显然这些人影的真身非常骇人和不祥,甚至非常危险,还是不动为妙。”

胖子听我这么说,把头转向闷油瓶,像是想征求他的意见。

闷油瓶死死地盯着那些影子,没有回答他,而是对我们道:“我们和它们……其实一样。”

\chapter{这里的石头}

“为什么这么说?”我纳闷道,但刚问完就明白了闷油瓶的意思。

在某种意义上说,我们和这些石头里的人影,处境是一样的,只是他们的空间更小,被困在石头中,就好比那些活在石头中的怪鱼,不过可以肯定,如果若干年后我们被发现,绝对不会是活蹦乱跳的。

想到这个,我心中有些凛然,道:“多少还是有些不一样,至少我们现在有这么大的活动空间,而且还活着。活着就有无限的可能性。”

闷油瓶淡淡道:“我不是这个意思。”

我啊了一声,有点意外。以前一直感觉和他们有一种默契,但是在这里,我有点跟不上他的想法了。他想到的东西好像比我快得多。

我问:“你是不是有什么想法?直接说出来吧!”

他看着我,“你们有没有想过,如果这里没有被挖出这么一个矿坑,我们现在是什么处境?”

我想了想,感觉大脑有点迟钝,还是不明白他的意思,但胖子的面色马上白了,骂了一声:“我操!”

随即我也明白了,后脑的头皮炸了起来。

如果这里不是一个矿坑,那么,会是什么?

这里就是岩壁,大山的内部。如果我们以同样的方式被莫名其妙地带到这里,那么现在,就可能是嵌在岩壁中,和那些影子一模一样。

我不寒而栗。这是一种什么感觉?如果我醒过来,发现自己被镶嵌在大山深处的岩壁中,动弹不得,必须这样直到死亡,那太恐怖了。

闷油瓶道:“反过来想这件事情,也许,我们现在活着,完全是一种巧合。”

我默默点头。这怪事也许是这山中的一种神秘现象,在山里可能不是第一次发生。就算当年没有人在此地挖矿坑,事情同样会发生,而我们现在的处境将更加的匪夷所思。

胖子咽了口唾沫,看着那些人影,道:“那么,这些就是我们的前辈?是以前碰到同样事情的受害者?”

“这也只是一种可能性。”闷油瓶道,“不过,我宁可相信是这样。”

我明白他的意思。如果这是一种奇怪的自然现象,他之前的推断就可能是错误的,那么不管我们的处境多么不利,至少暂时是安全的。

胖子就问道:“天真,你读的书多,你推测推测看,这可能是怎么回事?要是如小哥说的那样,可能是什么情况?”

我失笑道:“这种事书读得再多也没用,你要用读书能学到的东西来解释,就是物理学的概念,我们可能掉进了两个空间之间的裂缝,一下子从一个地方塞到了这里。不过在现实中,这是不可能的,就算真让你进入到天然形成的空间裂缝,再次出现的地方会是另一个宇宙,出现在同一个区域的可能性少到无限接近于零。”

世界上有很多这种事的传说,在一些非常特别的地点,比如百慕达,都说有这种现象。但我不相信这里是这种情况,胖子和闷油瓶在湖底失去意识的过程,完全不像是被“自然现象”搞定,太像是被人使用什么东西暗算。所以,我很赞同闷油瓶之前的看法:带我们来这里的力量,绝对是有意义和目的的。

胖子却不以为然,他道:“可能性少到无限接近于零,也不等于零。”

我道:“用科学来解释,就只有这一个解释。如果不是这样,我们面对的情况就完全是另一个范畴了。”

胖子陷入了沉思,自言自语道:“咱们老祖宗留下来的传说里,有没有这种事情?”

我想了想,过去从来没有在任何笔记小说中,看到岩石里出现人影的记录,当然,也许是我涉猎还不够广。

胖子接着道:“传说刘伯温墓附近的山里,有人只走了一天,出来的地方距离进山的地方相距一百多公里,好像在一瞬间就从一个地方被带到了另外一个地方。他们把这种现象叫做‘山鬼背’,以为自己是被山鬼背着走,所以不知道走了多少路。也有人叫‘走山’,说是山在走路,你说,会不会这里也有类似的现象,不过走的方向不一样?”

我摇头,这说法不成立。他们是在山的表面,我们现在在山的内部,不是什么背和走,是被山吞了。

而且,这事有一点蹊跷的地方,特别难理解,就是这矿洞是封闭的,四周没有任何崩塌,但这矿洞本来肯定有入口,哪儿去了?就算碰上‘山背鬼’或者‘走山’这种可能非常特殊的什么自然现象,也不会连入口都消失掉。

这里发生的事情要更加复杂,而且透着一股非常奇怪的感觉。

想到这里,我又想起了盘马的说法,他说这个湖里有魔鬼,我此时竟然有点相信了。好像只有魔鬼才能做出如此匪夷所思的事情。就算没有魔鬼,我看这山或者湖,总归有点不太平常。

水分逐渐蒸发,那些影子逐渐淡去,很快就看不清楚了,我用脚把先前在地上刻的‘铁俑’画掉。接着又琢磨了半天,还是没有任何结果。

岩壁恢复了原样,我们的感觉却变了,知道岩壁的五六拳之后有东西嵌在里面,我有一种强烈的被注视感,让人心神不定。这种感觉刚才没有,显然是心理作用,但无法驱除。

三个人都闷声在想,都不说话,偶尔胖子蹦出一个想法,都被我否决掉。

我想了很多的可能性,但也都不靠谱。最后,我开始把刚才想的事情又从头琢磨了一遍,包括所有的细节,看看还能否带出什么来。

如胖子说的,这些铁俑的作用是封这些影子,那么考古队的动机倒是可以解释,他们要找的东西,就是这些影子的遗体碎片,只是不知道这东西对他们有什么用处。

矿工在开采玉矿的时候,挖到这些人影,能肯定地是,开采并没有中断,对于玉石的渴望使得他们一边祭祀雷王,一边继续挖掘。

之后,到了某一天,有某个人在雷王的神像前留下信息。

看留言的内容和石壁中人影的情况,显然他的指示没有被执行,可能他离开之后,开采就终止了。使他们终止开采的可能性非常多,可能是战乱,可能是灾害,当然也可能是这个矿洞的入口莫名其妙的消失,甚至可能,那些矿工也和我们遭遇一样的情况——这里说不定不只有一个矿洞,他们被困在了其他地方。可以有任何的可能性,唯一能肯定的是,玉矿开采的故事,到这里就结束了。

之后,是我们的故事。

乍一看,非常的清晰及合理,但在仔细的想,会发现其中出现了一个很难察觉的矛盾。矛盾来自逆向思维,如果采矿的所有活动都没有发生呢?那么,这里会发生什么?没有人发掘玉矿,就没有矿坑,那么胖子和闷油瓶在水下,是否也会遇到事情?

如果,采矿活动不发生,那么我们现在所处的位置,就是实心的岩壁,如果把我们带到这里的力量是一种自然现象,那么,即使这里是岩壁,事情同样会发生,因为力量本身是自然的,我们只是奇怪现象的受害者之一。

但在反之,如果不是自然现象呢?如果这矿洞并不存在,这件事情,还会不会发生?

我感觉可能就不会发生了,因为闷油瓶和我都认为,这件事情背后有着某种意识,目的肯定不是杀死我们,带我们到这里来的这种行为背后,必然有着还不被知道的目的,而实现的前提,就是要有这个坑。则我们被困死,等于被杀死,对于“它”没有意义。根据以上推断,把事情分解开,首先能知道,那个意识,知道有这个矿洞的存在。另一方面,这个矿洞并不是经过规划的,它存在于这里是个偶然,那也就可以证明一点,那个意识的神秘目的,产生于这个矿坑行成之后。先有了这个矿洞,才有这个目的。那么,事情就很牵强,有点讲不通了。

假设这股力量,我们称其为魔鬼,某天溜达的时候,突然发现这里出现了一个矿坑,经其琢磨,发现可以利用,就兴起了一个目的,然后使用某种手段,将我们抓来,困在这里,以便实现目的……

如果是这样的过程,那他的目的,怎么看也不会是什么正经事,而且这种行为,起承转合,有板有眼,目的性和操作性太强,简直和人的思维完全一致。我并不排斥世界上可能有某些神秘力量存在,但我认为这种力量肯定是超然的,不会如此功利和浅薄。

但如果这个力量不是魔鬼,是一个人,那就不一样了。

有一个人知道这里有个矿坑,发现其可以利用,便设计了一个阴谋,使用某种手段将胖子和闷油瓶在湖底迷昏,再用一种非常巧妙的方式带进这里。以便实现他的计划。这听起来就非常的合理,我们非但不会觉得此人不靠谱,还会认为他如此处心积虑,必然之后有更大的阴谋。

有一个哲人说过一句话:当所有的不可能都排除后,再不可能,也是事实。这正是我一直感觉这件事情很奇怪的原因。身在其中,我闻到了一股浓浓的“阴谋”味道。

也就是说,弄不好,我们也就是在一个“人”设置的阴谋里。只是这个阴谋太巧妙了,无法理解。我看向了闷油瓶,他一定早就意识到了这一点。所以根本就不来参与我们的假设,但他没有进一步的行动,因为这终归只是一种感觉,无法证实。

\chapter{异变}

接下来的几天,一切都没有变化。我刚开始无法适应,饿得天昏地暗,但三天之后,人体自动转入体内消耗,逐渐就精神起来。

没有任何事情发生,时间好像凝固了。武侠小说中,很多痴男怨女都会被困于绝境,等他们重返外界,回忆过去,往往会发现,绝境内的时间,才是最快乐和安详的。

然而实际情况完全不是这样,篝火压到最低,四周只有不断的水声,火光下的岩壁呈现非常暗的黄色。身在山洞中的封闭感,让人无时无刻不觉得焦虑。我得学闷油瓶每天打坐才勉强熬得下去,否则非疯了不可。

胖子那种性格更是待不下去,我都不知道之前那两个礼拜他是怎么熬下来的,但他几乎每天都会想个新花样出来。

另外,我们在这几天里,用香灰一点一点把石壁都抹了一遍,希望找出一些别的痕迹。

确实,地面上有很多划痕,看来先前的人休息之余经常会在地面上画一些东西。我们看到了简易的棋盘,还有很多的字,但都没有任何价值,只有其中一条让我觉得有点意思,那是在洞壁之前的地上,大概是一个矿工休息时刻的,刻了好几个同样的名字,叫赵翠姐,估计是相思所致。看着这个,不由得想起地面上的阿贵,估计他更崩溃了。

到了第三天,我不由自主地对自己的想法产生了怀疑,想着,这么漫无天日地待下去,会不会最后什么事情都没发生?又或者,那个魔鬼已经把我们忘记了?

闷油瓶还是老样子,我的军刺被他拿去,横插在了腰间。人几乎不动,一整天都靠在篝火边上,看不出有一丝的焦虑。

虽然他之前就一直是这副样子,但我感觉这一次他镇静得有点过分,有时候甚至有错觉,他知道即将发生什么事情。

平静一直持续到第五天的半夜——应该是半夜,如果我的手表还准的话,忽然就起了变故。

我醒过来放尿,浑浑噩噩的,突然发现闷油瓶不再原来耳朵位置上,惊了一下,下意识往四处去看,发现他站在一边的岩壁前,正看着什么。

胖子在一边打呼噜,我感觉到不妙,看了看表并将他踢醒,两个人走了过去。

走到岩壁前一看,我们都愣住,人影竟然又出现了。

我心说,闷油瓶半夜看这种东西干嘛?再一瞧,却发现岩壁没有被打湿,而且,那诡异的人影,看着和之前有些不一样。

拿来矿灯,打开,把整块岩壁照亮,下一刹那我就吸了口冷气,岩壁中所有的影子,现在居然都能清晰地看到。强光下,这些影子离岩壁表面的距离,竟感觉比之前近了很多。

“我操!怎么回事?”我骂道。

闷油瓶道:“它们在朝我们移动。”

\chapter{怪物}

墙壁中的影子确实在向我们靠近,而且连动作都有奇怪的变化,头往前诡异地伸着,好像努力想从石壁中探出来。

“移动?”胖子没睡醒,还没弄明白。

“之前它们埋在岩壁中三尺左右的地方,现在只有一尺不到了。”闷油瓶道,做了一个手势,“五天时间,它们朝我们前进了两尺多,再有一天半……”

他顿了顿,没有说下去。

我知道他的意思,再有一天半,这些影子就可能从岩层中出来了。

“难道它们是活的?”我不由毛骨悚然。

闷油瓶摇头,直勾勾地看着影子,那动作,似乎在和影子对视一般。

我的睡意在一瞬间消失无踪,拿着探灯照了一圈,见四周全部都是影子,鸡皮疙瘩都暴了起来。这些影子到底是什么东西?如果它们从墙壁中出来……想着,头皮直发炸。

走了一圈,我突然意识到了什么,立即骂道:“我靠!难道这就是那个东西的目的?”

“什么目的?”胖子还是迷迷糊糊的。

“我不清楚,但也许是一种仪式,我们是祭品,或者,这是一种饲喂,我们是食物,或者这是种捕猎,我们是诱饵……总之,我们是为这些影子准备的。”

胖子皱了皱眉,终于醒悟过来,呆了呆,骂了一声:“我操!不会吧!”

我说什么不会?看那些影子诡异的形状,肯定不会是F罩杯的美女,那么它们被我们吸引,绝对不会是好事。

我登时就心乱如麻,不知道应该怎么办,看向闷油瓶,却见他入定了一样,不知道在想什么。

胖子忽然从一边的工具堆里掏出一把石工锤,丢给我。

“干嘛?”我问。

“先下手为强。”他沉声道,“打到它们连妈妈都不认识。”说着就要去砸。

我一把抓住他,“这些是什么东西都不知道,你砸几下不一定砸得死,反而把它从里面放了出来,到时候看你怎么收拾!”

胖子骂道:“我真受不了你这个笨蛋!你不会砸条缝出来先看看?”

我还是感觉不妥,再看闷油瓶,他仍旧不理我们。

胖子以为这是他也同意,举起石工锤,朝一个人影就砸下去。

他好几天没吃饭,体力不支,第一下只砸出个小凹坑来,但这里的石质非常脆,一下就裂出了细缝。

他呸了几口,随即又是一下,顺着那墨绿色的玉脉,竟然裂进去一条深缝。

瞬间,一股非常浓烈的气味从石头里传出来,几乎无法让人呼吸,我们都不由自主地后退了几步。

胖子还想再砸,我再次把他拉住,因为我看到,裂缝深处露出了一团东西。

我们捂住口鼻,等那气味稍微消散了一些便靠过去。

胖子拿起矿灯,往里头照。

起初只看到墨绿色的一团,好像也是岩石,但无法辨别那是人影的哪个部分。本来也没有多么害怕,但当凑近的刹那,那团东西转动了一下,接着,一双只有眼白的眼睛从裂缝后面转出来,看向我。

那一瞬,我几乎窒息。

那双眼睛没有任何感情,也没有任何的定向,但你就是能知道,它在看着你,从裂缝中看着你,这情形实在太诡异了!

我和胖子不由自主吸了口冷气,两个人都炸了,并且立即确认,这东西不是人!

不敢再看,我猛然把头转开,胖子也不知道该怎么办了。

我看着他,心道你不是要打得它连妈妈也不认识吗?他却猛摇头。

刚想说点什么,突然从裂缝里传出一声婴儿般的叫声,无比的尖厉,同时,一双极细的爪子猛地伸了出来,抓住我的脖子。

这速度太快了,谁也来不及反应,我已经被扯向裂缝,狠狠地撞在岩壁上。

闷油瓶这时反应比胖子都快,一下扑过来抓住我,另一手的军刺就朝裂缝捅进去,刺到那双爪子的手腕上,连刺三下,那东西才放手。

我摔出来,迅速被胖子拉离。

那双爪子很快又伸出来,连抓几下都抓空。胖子抡起锤子砸了几下,也不知道有没有砸到,爪子又缩了进去。

我们惊魂未定,喘了半天粗气。胖子道:“我操!他奶奶的是个狠角色!”

一边闷油瓶已经头也不回地走到篝火旁边,拿起一个筐子,抄起一盘火炭,道:“帮忙。”

\chapter{火炭}

我一看便知道闷油瓶想干什么,还没等仔细去想是否妥当,他已经把一盘子火炭全倒进砸出来的那条缝隙中。

缝隙离里面的东西还有些距离,胖子紧随其后,又是一盘子,后灌入的火炭把已经在缝隙中的往里推了进去。

顿时,石头中传来一阵阵声音,酷似婴儿哭啼,尖锐的要命,凄惨无比。

按道理说,把这种恐怖的东西弄死,应该不会有太大的心理压力,但我听着,还是感觉心被揪起来,相当的不忍,到底它现在完全处于弱势,完全只能任人宰割。

闷油瓶面若冰霜,毫不犹豫地继续灌。

空气中弥漫着一股奇怪的味道,我十分熟悉,那就是之前铁块中的“死人味道”,想不到它确实代表了死亡,石壁中的影子起先不停地抖动,逐渐停了下来,凄厉的叫声变得模糊不清。

我自幼心软,虽然刚才差点被抓住,但这么活生生地把一个人形的东西弄死,心中还是无比的难受。

胖子倒没有我这么迂腐,虽然也有点犯嘀咕,但并不扭捏,干笑几声道:“来生投人胎,别投错地方了。”

最后,那个影子一点动静都没有了,只剩下石头上的缺口,仍在冒青烟。

我颓然坐倒在地,长出了一口气,刚想缓一下,闷油瓶却道:“还没有结束。”

我抬头一看就明白了他是什么意思,另一边的岩壁上,还有三个人影已离表面非常近了。

“我们一定要这么干吗?”我问。

闷油瓶没有回答,看了一眼胖子。他点了点头,举起锤子和凿子,走向另外一个人影,我不想再看,就坐在那儿没动。胖子念了几声阿弥陀佛,又动手开凿,很快,刚才发生的事情便重演了一遍。

等转到第三个的时候,胖子也受不了了,满头是汗地在那影子前站了很久,问闷油瓶:“小哥,咱们能不能歇歇再干?”

闷油瓶摇头,看了看四周,冷冷道:“不用再干了,没有时间了。”

跟着转头一看,顿时凛然,不知道什么时候,岩壁中的人影,已经全部贴着壁面显现了出来,一眼看去能数的清的,又多出了起码十具,而且能用肉眼看见。

它们正向石壁的表面缓慢移动。

这是怎么回事?难道它们发现了我们的企图,加快了速度?

我又站了起来,闷油瓶拿起的我军刺,反手握住,胖子操起石工锤,我手无寸铁,看了看,从地上操起一根钎杆,三个人背对着背,注视着四周。

胖子已经兴奋了起来,他这种人如果真的要干仗,才不会管对方是阿诺还是石头妖怪。就听他骂了几声,道:“狗日的!也好,他娘的我真受不了在这儿待下去了,饿死不如这么死光荣,咱们大干一场!”说完又想起了什么,一脚把那神像踢飞,“他娘的不给面子!老子拜你不如拜个鸡巴!”

我心跳的极快,不由自主地颤抖,但出奇的并不是害怕,对胖子道:“这么死有什么光荣的?他娘的谁知道你是怎么死的?”

刚说完,忽然脖子后面一凉,有什么东西落到了我脖子上,我吓得赶紧跳开一摸,一看,是一些岩石的碎片。

我脑子一跳,心说我靠,忘记了头顶也是石头,抬头便看到离头顶不到两拳的岩顶已经开裂,缝隙中出现一个浑身绿色的东西。

我们立刻让开,岩顶几乎在同时裂开,一团绿影猛地从上面挂下来,之后是一阵凄厉的叫声。

探灯光下,我根本没有看清那东西的全貌,只知道一个影子摔下来,在探灯光圈里停留了半秒,一下就闪开,撞在了篝火上。

篝火被撞散架,火星和炭火被撞得到处都是,集中的光线完全被撞散,四下顿时一片漆黑,只能看到无数小的火点在燃烧。

这变化始料不及,我用探灯追着那东西照,但只能扫到残影。

胖子反应最快,抄起地上一根还燃烧着的柴火,可才拿起来火就熄灭了,剩下一截暗红色的炭。

“狗日的——”他大骂,“的”字还没完全吐出就变成一声闷哼,人好像被什么东西扑倒在地,接着是一连串扑打的声音。

循声把探灯照去,见胖子和一只东西扭打在一起,转开去照闷油瓶,手电筒一转,没找到他,却一下照到一张无比狰狞的面孔。

我转探灯有一个惯性,所以那脸只在面前出现一瞬,那样的冲击力却远大于直接看到。我顿时吓得屁滚尿流,条件反射下连连后退,大叫:“又出来一个!”

害怕归害怕,手上的钎杆朝那个方向就扫过去,闷响中敲到了什么,但没有吃到力气。钎杆是全铁的,非常重,我凭单手无法再打第二下,只好抽回来,再用探灯去照。

还没照清楚,背后被猛地一撞,整个人便摔了出去,直接滚到地上。探灯一下脱手,不知道飞到哪儿去了。

我爬起来便知道糟糕,什么都看不见,麻烦了。此时就听闷油瓶大喊一声:“趴在地上,不要动!”接着又是一阵凄厉惨叫,一团东西重重摔在我身边。

我抱头缩到一边,身边几拳的地方嘶声连连,然后暗中听到“咔嚓”的颈骨折断声音,惨叫声戛然而止。

另一边,胖子那里还没结束,听他一下接一下用力锤着,“操!敢偷你胖爷的桃!敢偷你胖爷的桃!”锤一下就是一声惨叫,如此连锤四下,那边也没了动静。他用力呸了一口。

看不清那里的状况,周遭一下安静了。

我问道:“都解决了?”

边上闷油瓶厉声道:“别说话,听!”

我立即屏气,听到黑暗里传来爬行的声音,数量之多,无法估计。

\chapter{有三十五个}

忽然感到肩膀上不大对,刚才被闷油瓶按住的地方,竟然全是血。另一边传来胖子撕心肺的惨叫,不是占据上风的,而是被逼入绝境的怒吼,听得人魂飞魄散。

虽然我什么都看不到,但能想像四周是什么情形,那些石头中的人影,肯定已将我们团团包围了。

回忆一下先前在地上看到的话:十六、七、十、四,一共是三十七。刚才那两个已经被烧死了,那么,我们要面对的,有三十五个。

我看不见周围的情形,不知道胖子他们有没有挂彩,所以没有多,同时也没有精力胡思乱想,死死地抓着钎杆,注意力全集中到了耳朵上。

胖子离我们很远,很可能已经被隔开,身边没人,他有点测定不住气,呼吸声非常紧张,但同时又很卑鄙地压低自己的呼吸,心说都去找他。

没有僵持多少时间,果然胖子那里先炸起来,他一声闷哼,然后大叫:“我操!开干!”

呼的一下,不知道他砸到了什么,那边一片混乱,有东西叫了起来,同时四下好比惊飞的鸟群般响起嘶叫声,乱成一锅粥,全部朝他去了。

我抡起杆子想上骈帮尽快,上前两步不到就撞到一团东西上,滑腻腻的。没等反应过来,黑暗中一场尖啸,劲风四起,人一下被撞翻在地上,身上几个地方立即传来剧痛。

用手一抓,抓到一支爪子,但是立刻脱手。匆忙用手乱挡,很快手就被抓得一塌糊涂。不过没几下就听一场闷响,那东西被人踹了出去。

我手尽快脚乱地爬起来,却被身边的闷油瓶按住肩膀,他轻声喝道:“不要说话,你不要动!”说完如一道劲风朝胖子去了。

我心中的感觉很怪既想上去帮忙,又感觉闷油瓶的话不能不听。忽然感到肩膀上不大对,一摸之下,刚才被他按住的地方,竟然全是血。

那种血量不会是自己划开的,肯定是受了重伤。我心下凛然,方才那阵搏斗,黑暗中听着似乎他占尽了上风,但显然也没讨到多少便宜。

另一边传来胖子撕心肺的惨叫,不是占据上风的,而是被逼入绝境的怒吼,听得人魂飞魄散。有很多时候,我会忍不住想像,我们三个人中的一个,如果出现意外,会是什么情形?但想归想,只要闷油瓶在,总感觉不可能出现这种事。然而现在,这种感觉烟消云散了,胖子很可能就会在这里被干掉。

“退到墙边上去!”

决瓶的声音出现在胖子的位置,随着话音落下,状况变得更加混乱,惨叫声、倒地声,胖子的叫骂声,混成一团。

我脑子里一片空白,此时已无法思考,抱着钎杆无法动弹,只能听着那边的动静,自己上去也没有用,情况之混乱不是我可以理解的,如果不是身手极好的人,凑上去甚至会被胖子谋杀。

也不知道这种状态了多久,忽然,境消失了,一片寂静。

我仍不敢动弹,不知道这是什么情况。他们都死了?还是所有的石中人都被干掉了?又或者,两者都是?

仔细地听了一会儿,突然“啪”的一场,探灯在一边竟亮了起来。转头一看,是闷油瓶,一手架着胖子,一手拿着我的探灯。

我想了口气,看着他一瘸一拐地和胖子走到我身边,把胖子放下,自己也坐了下来,两个浑身都是口子,淌着血。

在几乎遍布全身的血污中,麒麟纹身又出现了。这一次不仅是肩膀,他的上半身几乎已经燃烧起来,蔓延到全身。

我目瞪口呆,他却把探灯递给我,按着抓着我的手,把探灯指向墙壁上的一个口子,那些石中人出来的裂口。

“这是这种东西活动形成的通道,我刚才看了一下,这个通道也许可以通到外面。”他道,“你带上工具,快点离开。”

我立即点头,“你先休息一下,我帮你检查一下伤口,如果没事,我们马上走。他娘的,我还以为这次我们凶多吉少了。我真服了你,没想到你厉害到这种程度。”

他往后面的石壁上一靠,淡淡道:“我和他,走不了了。”

“你在说什么胡话?”我骂道。

他忽然朝我笑了笑,道:“还好,我没有害死你……”

我愣了。他一阵,吐出一大口鲜血。

“你——”我的脑子嗡了一声。

他仍微笑着看我,头缓缓地低了下来,坐在那里,好像只是在休息。但是,四周完全寂静了。

\chapter{脱出}

看着他安静地坐在面前,我心中的滋味无法形容。

我不知道自己脑子里想了什么,肯定有无数的念头在涌动,但是,我什么都感觉不到。

愣了片刻才醒悟过来,立即哆哆嗦嗦地去摸他的手腕,伸出这支手,几乎用了自己全部的力气。

还好,还有一些体温,脉搏非常的微弱,几乎感觉不到。

转头去看胖子,发现他的肚子破了一个大洞,肠子都挂在外面了,脉搏更是微乎其微。

他们身上的伤口还在流血,都是划伤,显然是那种东西的长爪子划的,十分密集,可以想见是无比惨烈的搏斗。

流血过多,心力衰竭,死亡几乎是无可逆转的。我有一些绝望、无助、懊恼、悔恨,无法形容的感受一起涌了上来,眼泪几乎要从眼眶冲出来。

可不知道为什么,不知道从哪里来的魄力,我在下一瞬间把这些感觉都推了出去,突然就冷静了下来。

我自己都被这种突如其来的冷静吓了一跳,像是心中有另外一个自己,暂时否决掉要来的情绪。不晓得在经历这种时刻时,其他人是否也有同样的体会,但就在此时,我的脑子里忽然无比的清晰。

——他们还没有死去,我自然不可能撒腿离开,但又不能在这里眼看着他们死。我必须做点什么,做我最后的努力。

我站了起来,开始琢磨怎么办。

首先找来了香灰,把他们最深的伤口全都抹上,把血暂时止住,然后把胖子的肠子一点一点的塞回到肚子里。那种感觉我不想记录下来。

弄完之后,拿来潜水服,撕成几条绑成绳子,拿来一旁的木框,绑了一下,做成一个拖曳式的单架,把两人绑了上去。

“就是死,你们也给我死在地面上。”我咬牙道。

弄完后,我拿好探灯,拿起一旁的军刺,看了看四周。地面上全是绿色的液体,也许是那种东西的血液,更多的是血肉模糊的人体,一片狼藉。

我没有细看,也不敢细看,转向四面的岩壁,想找闷油瓶说的洞口,只一眼就呆住了——石壁之内,竟然还隐隐约约地透着影子,而且比刚才看到的更多,但远比刚才看到的要小,都是一些小孩的影子。

我看了一圈,不禁毛骨悚然,当即不敢耽搁,拖着他们,朝着闷油瓶说的那个口子探了进去。

胖子本身就极重,加上闷油瓶的重量,我费了九牛二虎之力,这才把两个人拖进来。

果然如闷油瓶说的,那口子里是条通道,那些东西好像可以腐蚀这里的玉石,在玉中慢慢移动。四周全是上好的玉脉,如果有任何玉商在这里,肯定会疯掉。

但,它们如果是玉中自然形成的,那这条通道应该是封闭的。我用力拉了片刻,发现通道很长,同时,看着通道的岩壁,感觉很是不对,岩壁中不时出现一张张模糊的面孔,好像是岩石中的人正聚拢过来,看我爬行。

好在我的神经已经是怕到勒极点,索性不管,咬牙拖着胖子和闷油瓶,只顾自己爬着。

这个通道没有任何分岔,但是非常的曲折,有些地方甚至是垂直的,我足足爬了十几个小时,几乎累昏过去仍然没有到头。

也不知多久之后,探灯的光都快灭了,忽然,我听到了水声。

我几乎是发了狂似地往前爬,猛然手下一空,没按到想象中的地面,人差点摔下去。

探灯勉力一照,面前竟然出现了一个断层,是一道不规则的山体裂缝,不宽,两只脚撑开就能保持平衡。裂缝上方,水如瀑布一样跌落下来。

我喝了几口水,探灯往前照,前头再没有通道,这里好像是这个通道的起点。那些玩意儿可能是从这裂缝爬下去的。再上下左右照了照,好家伙!裂缝断层的表面全是像被蛀出的洞,而且全在同一面,这些东西跟山里的蛀虫一样。另一面什么都没有。

我放下胖子和闷油瓶,也没法管他们到底现在情况怎么样了,攀着那些洞一个一个爬下去,看看哪个可能通往外面。

其实完全不知道怎么辨别,只能一个一个地探。突然感到似乎哪里有风吹进来,我心中一喜,立即循着感觉找去,果然找到一个有空气流通的洞口。

有门儿!我心说,又爬了回去,解开一条绳子,把他们一个一个地送下去。

我饿了好几天,其实没什么体力,这一路极端的煎熬,到中途时,经常以用力就觉得天旋地转,并且开始干呕。这是体力极度透支的迹象,我觉得自己随时都可能晕过去。

最起码又用了六七个小时,这么几步路的距离才完成,我缩了进去,之后,又是天昏地暗的拖曳和爬行。

我能肯定,这段过程中,四周肯定发生了很多事情,因为耳边到处是奇怪的声音,但是,我没有任何的心理波动,麻木得一塌糊涂。就是这个时候死了,我可能也就这样了。

不知道爬了多久,前面忽然出现光。这时候我连加快速度的力量都没有了,只是继续行尸走肉般爬着、爬着。

然后,一瞬间,我听到了风声和水声,看到了久违的地面。我几乎反应不过来,还没等辨别出这是什么地方,就看到几个人出现在周围,抬头一看,是面色阴鸷的村民模样的人。

他们将我从洞口拽出来,可我一个也不认识。

湖滩另一面的一座山坡上全是人,入耳全是长沙话。

我的身体极度虚弱,一被拉出来就头晕目眩的,接着有个人带着一群人朝我过来。看天色是晚上,四面灯火通明,全是汽灯。还有人拿着对讲机在不停地叫喊:“找到了!找到了!”

带着一群人向我走过来的人,很快就到了视野内,我远远地看着,惊讶地发现,那竟然是我的二叔,后面跟着潘子。

他们都一脸急切,可没等他到跟前,我就失去了知觉。

\chapter{二叔}

醒过来的时候,我发现自己已经回到了阿贵的房间里,云彩在一边照顾我。外面非常嘈杂,我是被吵醒的。

我并没有受什么伤,只是体力不支,所以这一觉睡下去,人已经没有大碍了。我坐起来,云彩看到,立即给我递了水,然后到外面去叫人。不久,潘子走了进来,问我感觉怎么样?

我没有看到二叔,也没回答他的问题,劈头就问胖子他们怎么样了?

潘子告诉我,已经在第一时间把他们送到医院去了,现在还没有消息。他让我放心,如果他们死不了,那就是死不了,如果不幸挂了,那也没有办法。

我听乐稍微安了一下心,送医院去了,至少还有希望。

接着,我们这是怎么回事?他神秘兮兮的什么也不说,只说是我家二叔不让他和我多谈这些事,而是现在还在湖边,等他回来会亲口告诉我,然后让我多休息,说完就出去了,似乎外面非常的忙。

阿贵家附近的几个高脚楼都被二叔包了下来,我看到很多二叔。三叔以前的伙计,足有二十多个,在想起先前在湖边看到的,估计这次来了几百人,阿贵早就从崩溃中走了出来,穿针引线地忙活,但问他情况,他什么都不知道。

我没有办法,只好照办,一直在阿贵家休息了两天,身体大概复原之后,二叔才从湖边回来。

和二叔一起出现的还有好些人,竟然都是长沙的几个表叔,有几个是跟着三叔混的,都是我们家族里有头有脸的人物。

我心说怎么回事?怎么吴家人都到这儿来了?

我没敢问,因为二叔和那些亲戚的脸色并不好看,寒暄了一下,发现他们看我的眼神都很古怪。

二叔的气色很差,折腾了一番后亲戚们散了,二叔看了看我,勾住我的肩膀,问我身体没事了吧?

我点头说没事,这才低声问他是什么情况。他看了看我,叹了口气,拍了拍我的肩膀,示意跟他去逛逛。

我们来到村旁的溪边,一路逛来他也没说话,一直走到那幢被烧毁的老房子前,他才道:“你的E-MAIL,我已经看到了。”

我心中已然感觉到,这可能和那封E-MAIL有关系,便看着他,等着他继续说下去。

他顿了顿,才道:“你相信你在信里写的内容吗?”

“这叫我怎么说呢?我想不信,但又不敢不信,因为我想不出别的可能行了。”我道,“你和三叔相处了这么久,有发现什么异样么?”

二叔点起烟,看着我,皱着眉头不说话。

我道:“这是别人说的,三叔没亲口否认,所以,我不是没有怀疑。”

二叔仍看着我,几口就把烟吸完了,顿了顿,忽然道:“你不用怀疑了,我告诉你,这确实是真的。”

“确实?”我道,“你怎么确实?”

他慢慢道:“这件事情,我们早就知道了。”

我呆立在那里,不相信自己的耳朵。

二叔继续道:“小邪,有些事情没有你想的那么简单,但也有很多事,没有你想的那么复杂。”

“如果你们知道,你们怎么让这事发生了?”我问。

他站着不语,然后做了个手势,让我继续走,顺手递过来一张东西。

我接过来一看,是一张照片,“这是?”

“烧掉那栋房子之前,我留了一张。我想,现在给你看,比在当时给你看,要合适得多。”他道。

我愣了,一下懵了,房子?烧掉?我操!不会吧!当即就道:“二叔,那是你干的?”

还想说话,但他摆了摆手,让我看那张照片,“那些事情,我们就不提了。”

那是一张非常普通的黑白照片,也是一张合影。再仔细一看,上面是一个陌生的中年人,正和文锦说着什么,后面是考古队的其他人。中年人不是以往见过的照片中的人。他非常白,非常消瘦。但是我看着有些熟悉。

“这就是楚光头想让你看的照片。”二叔道,“我找一张最能说明问题的留下来,想着如果最后还是没办法,还得让你知道的话,物证会比我的嘴巴更能说明问题。”

“就是这个?”我不无法理解,“这照片有什么问题?”

“你不是认识这人吗?”他道,指了指那个陌生人。

我看着那个白而消瘦的人,忽然就想了起来他是谁,不由得“啊”了一声,“怎么会是他?他不是……”

这个人和我们的故事没有联系,但却不是无关紧要的人,如果他们和文锦那一队出现在一张照片上,那这只考古队的规格,就不是我想的那种地位了。

我们继续逛,二叔道:“我不能告诉你细节,但我可以给你讲个故事。小邪,有些时候,有些事情,他就是一个故事,仅仅是一个故事,你要不要听?”

\chapter{开心}

我点头,二叔又点了一根烟,道:“你读的书不比我少,秦始皇的本纪你读过吧?”

我点头,《史记》是搞古董的必修,自然读过。他继续道:“《汉书》呢?”

我又点头,他道:“你有没有发现?我们中国古代的这些皇帝,都有一个惯例,无论是大皇帝、小皇帝,草头天子还是正统皇室,在功成名就、寰内太平之后,他们都必然会有一种行为,就是求长生。”

“追求永生是帝王的终极梦想,并不奇怪。我要是一辈子不愁钱花,想杀谁就杀谁,想娶哪个女人就娶哪个女人,那我唯一的追求,恐怕就是将这种生活再继续下去。”我附和道。

二叔没有理会,只是继续说道:“如果翻开史书,你会发现,真的,这种惯例太难打破了,而且越是开国皇帝,越是变本加厉,秦始皇、汉武帝、唐太宗……”他顿了顿,“一代一代下来,几十国号变了,称号变了,更甚至,连皇帝的称呼都不用了,惯例还是没有打破。”

我点头,确实是这样。人性是传承不变的,不管你站在什么位置,到了一定的时候,一样会看到死亡向你靠近。

“但是,所谓长生秘诀和传说,越靠近现代越模糊。很多帝王都认为,长生术的线索存在于古代方士的墓葬里,所以,自然会出现一些队伍帮帝王进行实地勘探。这种队伍往往挂羊头卖狗肉,以一些现有的编制做掩护。”他看着我,笑了笑,“而这些队伍里的人,当晚是民间最厉害的高手。自古土夫子,南北地仙、摸金校尉,有不少都被招安吃起了公粮。在某些时候,强权压下,也由不得你不效忠,为了家里老小,只能低头。”

“不过,这种事情始终见不得光,所以历代这些人最后都没有什么好下场。另外,在这种队伍中,总有人想摆脱那种无孔不入的控制,而且长生这种事,不仅对帝王将相有吸引力,对这些寻找者也是巨大的诱惑。当他们真的发现一些线索时,心中不免会有自己的想法。”

“这些想法,他们往往会告诉自己的兄弟或者家人,这些家族的成员都是见过风浪、刀尖上滚的人,胆子都很大,于是,就会产生一些计划,以实施这些想法。这些计划有些失败了,有些成功了,有些也不知道是失败还是成功,但能肯定的是,一旦被发现,那么,这些人的末日就到了。”

他停了下来,勾住我的肩膀道:“不过,有些计划能瞒很长时间,甚至改朝换代。当这时候,双方已经达成了某种共识,没有人希望它被捅出来。”说着,他又看了看我,“特别是‘它’。”

我不敢说我完全听懂了二叔的故事,但是,我明白了他想说什么。

说实话,我哦完全没有想到,事情的背后回事这种范畴的东西,难怪楚哥会和我说,不能再查下去了。沉默中,把二叔说的和我之前的一些推测连起来,居然发现,很多事情一下就变得合理了。

我问道:“那么,这里的事情,也是‘它’所进行的活动中的一处?”

二叔点头:“恐怕是,所以我很早就知道这个村子的存在,一听潘子说你到了这里,就觉得不妙,立即叫他带着人过来。凡是那批人去的地方,必然凶险万分。”

“那你知道不知道,这里到底是什么情况?那些到底是什么东西?”我问二叔。

他想了想,道:“那些,可能是密洛陀。”

“密洛陀?那是什么玩意儿?”

“密洛陀是瑶人的祖先,在他们的神话里,他们的第一个女神,是从山中产生的。我估计,责众怪物就是密洛陀的原型。”他从口袋里掏出一块铁块,“这个女神第一次造人,造出来的就是铁人,但是铁盒女神的神力相克,没能成功。当时那些矿工用铁封石中人,显然都是听过这种传说的瑶人,你的估计应该差不离。”

我点头。二叔继续说道:“至于这东西是怎么产生的,恐怕没人知道。听你的描述,这件事很像一件宗教仪式,你们被当成祭品,等在那里。那些东西存在于山底很深的地方,要弄下去得花很长时间,我感觉,你们碰到的事,可能是别人安排的。”

他也有同样的感觉,证明我的直觉没错,但是我道:“可是,我说了,那个矿洞没有任何的出口。”

他想了想,拍了拍我道:“我以前和你说过,已经发生的事,不管你看到的现象如何,它就是发生了。你既然进去了,那必然就有入口,找不到不能说没有,入口肯定就在那里。”

我苦笑,之前胖子说的时候,我也是这种想法,但找不到就是找不到。

二叔的对讲机突然响了,他接起,只嗯了几声就挂掉,我继续问,他来找我为什么带这么多人来?这也太夸张了!他们现在在湖边干嘛?

二叔面色铁青,只道:“是有一些事情,这一次,还亏得有你,否则我们真找不到这里。至于来这里的目的,我现在还不能告诉你,等事情证实了,你自然会知道。”他看着手表,“这里的事情才刚刚开始,而且,我们的时间不多了。”

“是和三叔有关吗?或者,和‘它’?”我问。

二叔笑笑,“别急,到时候你就会知道,你所经历的这些事情,其实是多么微不足道。现在不要问,也不要去打听,你要找那小哥的过去就尽管去找,但我这里,你少来你那套。我和老三不同,我不会让你乱来的。”

二叔没有和我再说什么,和三叔不同,我不会和他磨什么嘴皮子,那完全没用,他会说到做到,说事情证实了会告诉我,就绝对不食言。

他说他还要在这里待一段时间,我可以在这儿等,去其他地方走走也行。不过,以后要随时报告行踪,不让我再乱跑了。

因为惦记着胖子和闷油瓶,我在一个星期后离开村子,去了防城港的医院。云彩和阿贵带着我找到了他们的病房,两人都没事。

一声说,其实两个人受的伤都不算致命,只是失血太多并且发生感染,好在他们的体质都非常好,我用香灰止血也同时又隔绝细菌的作用,所以只输了血就救了过来。那些香灰真的非常关键,如果他们再流多一掌那么多的血,可能就是大罗神仙也管不过来了。

用香灰止血是我听单田芳的评书学来的,没想到真的管用,看样子评书还真得多听听。

看到胖子的时候,我几乎老泪纵横。就这么几天不见,他的身体又肥回去了,一点也不像刚从阎王殿走了一遭的样子。

胖子看到云彩来了,一下又找不着北了,就要下床标榜自己的不死之身。

他们大概问了我之后的情况,我把我怎么把胖子的肠子塞进去,怎么把他们从那里拖出来都说了一遍。

胖子听完后一愣一愣的,说难怪他最近总觉得自己的肠子走向不对,一想大便就打饱嗝,说你别给我塞反了。

说着这个,我们开始聊这整件事情,我拿出一张纸给他们看。先前在阿贵家,我按照记忆,吧古寨的平面图画了下来。

但是如此讨论也没有什么结果,胖子就闹着要带我们去吃病号饭。

等了片刻,却不见云彩有动静,回头一看,发现她正看着那张湖底平面图发怔。没有一点反应,显然被什麽吸引了。

我有点意外,那平面图画的很容易,其实没什麽好看的。和胖子对视了一眼,胖子问她道:怎麽了,大妹子。

云彩嘟起嘴巴,抬头道:“两位老板,你画的这个湖底寨子,和巴乃好像啊。”

\chapter{很像的寨子}

巴乃就是阿贵他们住的那个寨子,也是一个典型的瑶寨,不过我们才住了没几天,对村里的地形没什么概念。云彩这么一说,我真有点意外。

“哪儿像了?”胖子把那图接过来,“你们这儿的村子,不是都差不多吗?”

云彩也不敢说死,把图递给了阿贵,说道:“阿爹,你看看。”我们也立即凑了过去。

阿贵看了看,一开始似乎也不理解,云彩把图换了个方向,然后和他用当地话说了几句,他才恍然大悟,挠了挠头道:“咦!还真是有点像。”

我来了兴趣,到底偶们不是本地人,对于很多细节,不及世世代代生活在这里的人敏感。而且女人又特别的细心,就让她也指给我们看。

本来我以为,可能单纯因为湖里的山势和巴乃四周的山势很像,所以导致村子的一些倚山建筑比较相似,但云彩一说,我就倒吸了一口冷气。

被指出的相似的地方竟然是路和篱笆。

云彩告诉我,她看这图的第一眼,就很明显地发现,我画的这个“湖底古寨”中的道路和篱笆的走势,和他们的寨子一模一样。这让她意识到异常,然后才开始发现村子的其他部分,也有很多地方是非常相似的。

我不可能回忆起巴乃寨子全部的青石路和台阶走向,但对阿贵房子附近的路有记忆,一参照,果然如此。只要把平面图换一个方向,立即就能找到阿贵家边上的几条小路,交叉方式和图上的真非常接近。

我的悲伤一下就全是冷汗,这就有点过了。这张平面图描绘的是一个沉在湖下的寨子,距今可能有几百上千年的时间了,但现在却发现,湖底的寨子和一座现实存在的寨子,有着无数的高度相似点,这他娘的是什么事啊?

虽然努力压制那种莫名的毛骨悚然,还是不可避免地打起哆嗦,直觉告诉我,这里可能有大问题。

吸了几口气把鸡皮疙瘩按回去,然后让云彩把所有的相似点都指出来,我必须判断哪些相似点可不可能是因为某种特殊的合理原因而形成。

可能当时我的面色有点吓人,云彩看我这么认真,害怕起来,不敢说话。胖子拍了一下,让我不要吓到小阿妹,我才意识到自己失态了。

我们从村口说起,一直说到村尾,越说我的心底越凉,意识到这不可能是任何的偶然可以做到的。从村口几个装饰牌坊的位置,到里面的大量青石路,篱笆,还有房子的排列,真的极为相似。

要造成这样的情况,只有一个可能,就是这个湖底的古寨和巴乃,是由同一个设计师设计的。

可是,村子怎么可能由设计师来设计?村子都是自然形成的,由千年来所有的村民自发进行调配,寻找最适合建房的地方,寻找最合理的路线,从而慢慢形成道路和房屋的布局。

最让我在意的是道路的高度相似。村子一旦形成,特别是山村,道路是在很长时间内都不太可能改变的东西,因而有“古道西风”一说。对于道路,村民做的最多是返修,不可能把整条路去掉,重新开一条。我们在很多山村里走的道路,大部分在两晋的时候就存在了。即使在杭州,那些山上的石道,也是很早的时候由寺里的和尚修造,现今政府做的,只是不断地返修。

所以,巴乃村子里的古道和湖底古寨的道路高度相似,本身是极不正常的,甚至可以说是诡异。对于我这个学建筑的人来说,更是煎熬,脑子里各种以前看过的东西在不停地翻滚,却不知道自己想找什么。

胖子还没有意识到我想得有多深,问:“天真,你以前听说过这种事吗?”

我摇头让他别问,这不是单纯的“听说过”,出现两个相似结构的建筑群,历史上,这种事情只有一个人干过,就是汪藏海。他负责设计的曲靖城和澳门城市完全一样的,但那是城市级的范畴,城市是可以规划的,村庄则完全不同,我从来没有听说过哪里有两个完全相同的村子。

而且,如果两个村子都存在,还可以说是奇观,或者是某个隐世高人的恶趣味,然而现在,一个存在,一个居然沉在湖底。

不管我怎么告诉自己,不要往复杂的方向想,但直觉总是告诉我,这里发生的事情,绝对不是单线的。我现在手里掌握的碎片,只不过是那颗“真相”洋葱的最外层。

胖子见我没什么反应,又去问闷油瓶。闷油瓶也没回答他,似乎对这个不敢兴趣,只是看着图发呆。

阿贵闪闪躲躲道:“咱们传说过,都说村子原来不在那地方,而在羊角山里。说不定真像和胖老板说的,这下面得寨子就是我们的古寨,村子不是被火烧的,是被水淹了,然后咱们的老祖宗就道外面相似的地方,再按照原来的格局修了一个村子,反正这里的山和我们外面的山差不多啊!”

我对他道:“除非你们的老祖宗对于堪舆学友很深的学问,否则,就算有意仿照,也很难仿照到这种程度。”

要达到这种相似,必须在原村没有被淹没的时候就进行精确的规划测量,当时的瑶民还处于未开化阶段,不可能有如此造诣。

云彩嘟嘴道:“老板,你凭什么看不起瑶民?说不定就真有那么一个人呢!”

我苦笑,不是我想这么想,而是如果真这样,那么这事就复杂了,于是答道:“即使有这么一个人也说不通,因为没有任何必要。瑶文化对于建筑的规划并不苛刻,何必非要搞得和以前的村子一样呢?这个村子的布局,本身并没有什么特别的隐含意义。”

中国有很多村子都是高人建设的,比如说浙江有个俞原村,就是刘伯温根据星象而造,整个村子时一个巨大的星盘。但现在这个湖底古寨的平面图非常没有规划,凭我的阅历,看不出蹊跷。

“你怎么想?”胖子问我道,“你肯定有点什么想法。”

确实,我有一些推测,但推测其实是没有用的,于是摇头:“我只是能肯定地说,这是故意而为的,而且花了大力气,因为普通的人,就是想修也修不到这种相似度。”我感到最难以理解的是,这样明显的事,当地竟没有传说。

阿贵他有好几代的记忆,他们的村子年代也非常久远了。也就是说,这种COPY行为发生的时间在更久以前。但从张家楼里的一些迹象判断,玉矿开采的时间不会太晚。湖水的倒灌,应该是在玉矿开采之后,否则矿坑不可能修起来。

三件事情,COPY、湖水的倒灌、玉矿的开采,按发生的时间顺序排,应该是COPY早于玉矿开采,湖水的倒灌最后。

也就是说,在玉矿开采之前,那个湖是不存在的。村子没有被淹没,即使已经荒废了,它也在那里。

那么,当地人应该就会知道,有两个一模一样的村子。就算其中一个后来被水淹了,但年代如此之远,至少会有传说。

而且,这种传说的辐射范围会很广,就是往外几十里的其他村里,也免不了有流传。

阿贵却说没有任何传说提到过湖底的寨子。这个隐秘的古寨就好像是一个意外,在历史的行进中,完全地被人遗忘。

\chapter{镜像阴谋}

当然,这种遗忘可以是偶然的,事实上,不知道有多少传说湮灭在历史中,但这种湮灭一般都是大规模的,不会单单只有一个传说消失。传说断代,必然随同某一段历史时期的完全空白,没有任何讯息。

我觉得这其中有猫腻,寨子里的传说和老故事不少,不存在明显的断代,却单单没有任何“本来有个一模一样的古寨,但是被水淹没”的相关传说,是否有人不希望这个传说流传?

关于这种COPY,我心中其实有种非常强烈的直觉,认为是出“镜像阴谋”。“镜像阴谋”是日本推理小说中的一种常见诡计。诡计的核心就是隐瞒,也就是说,阿贵他们的村子是假的,是为了不让别人发现真实的村子已经消失,因而特意建造的。

生出如此强烈直觉的根本原因,是盘马说的魔湖的故事。我当时的推测其实也是一种镜像诡计,老的考古队被抹掉,一只来历不明的新队伍神不知鬼不觉地替代,就是“镜像”。

这也可以解释,为什么会有大火烧了老寨的传说,当时的寨子肯定不全是在峡谷的坡上,山里的村子会有很多零星的楼房,分布在离存在较远的地方,这些寨子水淹不掉,但一定要毁灭,就可能使用了山火,为了掩盖山火的痕迹,最终使用了这种说法。

这种诡计的背后,就是大阴谋了,并且可能极度的血腥,原本村里的人必然会被全部屠戮,杀人者很可能假扮成村民,住入了假村之中,实行他们的计划,这个诡计发生在很久以前,若干年后,又发生了玉矿的事情,之后,村子被淹没。

一个地方发生这么多的事,显然,这里的村落山川河流中,隐藏着什么。

一切听着实在太玄乎,感觉不太可能,我很抗拒往这个方向思考,反正也无法求证,现在只能压制疑问,等待之后进一步的调查结果。

大脑完全不够用了,刚想喝点东西透透气,胖子却又发出一声啧,指着图问我道:“天真,你这样看看,你画的图像什么?”

凑过去,就发现他拿笔涂黑了一些地方,很快我的平面图就变得斑驳,等他拿起来放到太阳光下面,我就愣了。

被他稍微一加工,整个村子的平面图,竟然变成了一只动物的样子,有眼睛和爪子。再仔细一辨认,立即认了出来,那是一只麒麟。

“越来越好玩了。”胖子喃喃道。

我浑身的毛都立了起来,直接能看到的是,那麒麟的样子,和闷油瓶身上的很像。我靠!难道真的来对地方了!我心说,脑子里几个概念不停的闪动,麒麟、纹身、平面图,忽然就有了一个横空出世的念头。

拿着图走向闷油瓶,他正在发呆。

我上去对他道:“快快!把衣服脱了!”

他愣了一下,面露不解,我把手里的图给他看,这样那样不停的解释,他仍是不理解,但还是按照我的意思把衣服脱了下来。

我看着他磨叽的动作真是心痒,真想一脚把他踹翻,马上贴上去看,等他脱了衣服我才想起来,他身上的纹身,平时是看不见的。

我问闷油瓶这是怎么回事,他告诉我,这种纹身是用一种带刺植物的汁液纹出来的,平常是透明的,只有体温超过一定温度才会变成黑色。古时候苗人多有湿热病,这种纹身可用来检测小孩子的体温。

当然,要体温超过一定的温度,也可以是因为剧烈的运动,所以闷油瓶在剧烈的搏斗或者激动地时候,纹身便会显现。也由于不同的浓度,对于温度的敏感不同,只有极端剧烈的搏斗,能让所有的纹身图案显现出来。

胖子弄来热水袋,我们逼着闷油瓶烫他的胸部,果然,黑色的纹身慢慢显现。

胖子就道:“我靠!这招好啊!我以前作弊怎么不知道这个。”我则开始仔细看他的纹身和我画的地图。

“你看看这古楼的位置。”胖子道,指了指塔边上路径的走向,“如果巴乃和这个村子是一样的,那么这湖底古楼的位置,正巧在小哥那高脚木楼的位置上,如果贴在小哥身上,就是麒麟的眼睛。”

“哦?”我心中一动,细细一看,果然如此,心说胖子果然心细。

这有什么深意吗?

胖子又道:“这样看来能肯定一点,就是小哥,你肯定和这个有渊源。”

我切了一声,说这不是废话吗?

胖子道非也,这对于我们,指导意义重大。以前只是估计,大概这里会有一些线索,现在可以确定了。估计和确定是两个完全不同的概念,我们今后的做法也会改变。

我点头,这倒也是,而且,这个村子的事情才刚刚开始,有的搞了。

胖子接下来和我们讨论了一些指导方向,“这事算是有眉目了,也不用那么急,反正村子不可能忽然又没了,我们肯定得继续待着,做个系统的调查。另外,周围的村子也得一个一个去打听,看看能问出什麽来。这是个很长的过程。我看,得在这里呆上一段很长的时间。整理一下,先回去带点东西过来,接下来可能要常驻。”说着对云彩就咧嘴笑:“丫头咱们相处的时间长着呢!”

云彩也笑笑,眼神却不自觉的晃像闷油瓶。

\chapter{不速之客}

接下来的事情其实没有必要记述,但和之后的发展有些关系,所以也提上一提。

二叔在五天后离开,我不知道他们在那里是否还找到了什么,总之他什么都没有告诉我但和我约定回杭州后好好聊一次。

胖子和闷油瓶其实没受到多严重的伤,得到救治之后,没两周就出院了。我们没有立即回杭州,而是再次去了巴乃。胖子断定闷油瓶和那里有联系,没有得到更多线索之前,可能要在那里长住。

我们在四天后又去了那个湖边,在湖中心祭拜了那些骸骨,立了土黄丘。

盘马再也没有出现,这让我很是内疚,但想到他的罪巷,感觉也是一种命数。拿着我的专业打捞设备,继续进行细致的打捞,期望得到更多的线索。更多的东西被陆续捞了上来,但没有发现什么特别关键的。

接下来,我们准备进入古寨中,仔细地查看那座张家楼情况。但就在这个节骨眼上,所有的氧气瓶都耗尽了,必须去更换。

也巧,最后一天潜水完成,准备上岸返程的时候,湖边出了变故。

当时我们还在湖中心,刚浮上来胖子就出声招呼,抹了一把脸,指向岸边。我朝岸上看去,发现不止云彩他们,还出现了好多人,竟然正在搭建帐篷。

“我操!怎么回事?”胖子奇怪道,“这里变旅游景点了?怎么又来人?”

我喘了几口,仔细地观察,发现来人中有很多是寨子里的村民,云彩正在和他们聊天,其中另有一些人穿得很城市化,不知道来历。更多的人正从我们来时的小路下来,牵着好多的骡子,上头全是包裹。

这批人我一个都不认识,约翰不是二叔又回来了。

慢悠悠地游回到岸上,我越发觉得事情有点古怪,因为那些人带着好多只骡子,大包小包的好多东西。几个大帐篷已经搭了起来,石滩上一片忙碌,几个人只是略带惊讶地看过来,没有谁过多地理会这几个穿着裤衩从水里出来的人。

我们完全不知道该如何反应,走到云彩和阿贵边上,我忽然看到一个人,在盘马老爹家里碰到的那个满嘴京腔、五短身材的家伙,正在吆喝那些当脚夫的村民干这干那,一脸飞扬跋扈的样子。

这种人我在道上见得多了,想起当时听到的,他应该是跟着一个北京老板来这时原,那么这些人可能都是那个北京老板带来的。难道他们也问出了盘马老爹的故事,准备到这里来找东西?人也太多了点吧!

他看到我们,也算是见过一面,就打了招呼。我也懒得多想,回了礼,从他身边经过,到云彩那里,问这是怎么回事?

她轻声说听几个村里人告诉她,有一个大老板雇了他们搬东西到这里,具体情况那些人也不清楚。

这局面比较尴尬,我不希望事情有这么发展,但这湖是公家的,你也不可能说不让别人来。这批人的目标是那种几块,我不知道他们是知道铁块的真相,还是单纯就是为了救赎,没法做出对策。

他们的人源源不断,六七顶帐篷支了起来,所有的人都是一口京腔,让我恍惚间觉得来到了后海边上。

坐下来,一边休息一边警惕地看着他们做事。这其实挺郁闷的,好比你在球场上打球,打着打着忽然来了一堆人,全都人高马大而且人数比你多几倍,这时候你只能乖乖下场休息。

我一边暗骂一边仔细观察他们运来的东西,看看能否发现什么线索。不看不知道,一看心就直往下沉。那些大包裹里,竟然有好几只水肺,好多物资看起来都像潜设备。

“人家是有备而来的。”胖子哼了哼,“他们知道水下面有东西。”

我脑子转了一下,对胖子道:“会不会是北京有什么老瓢把子来这里淘货了?那些人你认不认识?”

胖子道:“北京多的是掮客倒爷,潘家园里没几个是亲自下地的,我想可能性不大。这些人不会是四九城里混的,我看也许是咱们不知道的人。这年头,各地都有新势力。”

“你在北京人脉广,你看,有一两个认识的吗?”我再问。

胖子摇头,“我怎么看没有一个脸热的,你让我再仔细看看,不过这些人的京腔有点怪。你等等,你胖爷我打听一下,看看能不能问出他们老板是谁。”

胖子朝忙碌的营地里走去,用北京话和其中一个人打招呼,不过那人没搭理他。胖子是什么人物?立即跟了过去,他们就走远了。

我想着我能干些什么,要么到他们营地里逛逛,看看有什么,或者干脆去找他们的老板?

最终我什么都没干,因为潜水后的净利润痛让我站不起身,眼睛和耳朵也非常难受,特别是耳朵,又痒又疼,听声音都非常奇怪,看来这样潜水对身体的伤害很大。

正思索着该怎么办,忽然身后的闷油瓶捏了我肩膀一下。

捏得恰到好处,我舒服得一缩脖子,心说这家伙良心发现要给我按摩,却听他轻声道:“你看。”

我把注意力重新投回到营地里,想看那里有无异样,却发现另一边的林子里又来了一队人,有一个人被人从骡子上被扶下来。那五短身材的很快迎了过去。

仔细观瞧,发现那人年纪弓箭有点大了,下来之后走路踉踉跄跄的,连腰也直不起来。他四周有好几个随从,前前后后朝我们走了过来。

站起来想过去,闷油瓶却按住我。我转头,发现他矮身在我后头,漆漆地盯着来人,对我道:“不要让他们看到我。”

“怎么回事?”我心里一个,挺直了身子将闷油瓶挡住,看着他们越来越靠近。被搀扶着的那个像大人物的人,是一个高大但体形无比消瘦的老头,看得出年轻时肯定非常魁梧。因为被若干人拥簇着,我没能看清他的面孔,只觉得这人非常苍老,走路完全没有力气,应该已是风烛残年。

边上一干人等,有男有女,更加混杂,那个五短身材一路似乎在做介绍。几人边说边走,并没有走到我们面前,拐入了一顶帐篷里。

等他们走进帐篷,闷油瓶才松开捏着我肩膀的手。我被他捏得气血不畅,揉了几下,问他道:“怎么?你认识这个人?”

他点点头,脸色铁青道:“裘德考。”

“裘德考?”我一下愣了,“这老头就是裘德考?”接着几乎跳起来。我靠!这些人同样是阿宁公司的队伍,这老头竟然亲自出马了。

一时间我不知该如何反应。裘德考在我心中有一个既定的形象,既确定又不确定,是一个长着斯文赫定那样一张脸的传教士,但又有些像马可·波罗那个大骗子。而在童年时代,爷爷和我说的故事里,裘德考是一个最坏的坏蛋,我还曾经把他想像成一只大头狼脸的妖怪。真没想到,他本人会是如此形容枯稿的一个老人。

这种预判让我觉得非常古怪,十分的不真实。爷爷的故事就相当于我小时候的童话书,现在,童话书的人物忽然从爷爷的笔记本里走了出来,一时之间,很有错乱的感觉。

他来这是干什么呢?看这阵势是知道湖底下的事的。蛇沼之后,他和我们一样没有放弃追查,也追到这里来了?

可是,我们的调查方向完全是随兴而为,他们和我们没有相同的基础,怎么会碰到一起?难道他们一直跟踪着?

想想又觉得不像,如果是跟踪,他们不可能做出比我们更周全的准备。我们就完全想不到这里需要潜水设备,他们却带来了,肯定知道得更多,至少要知道得比较早。我既有点兴奋,又有点害怕。这老头亲自出现在这里,肯定非同小可。他这样的年纪不适合长途奔袭,这次出现,必然是孤注一掷。

下面到底有什么东西?

转念一想,现在的局面麻烦了,我们和他们的关系太复杂了。我的爷爷和裘德考是世仇,虽然现在我没有任何报仇的想法,但这层关系让我不可能对他们有任何好感。而三叔和裘德考之间的恩怨,更是剪不断理还乱。

我们两方之间即使没有敌意,也有极强的竞争关系,在敌强我弱的情况下,得好好想想该怎么来处理关系。

得走一步是一步。

我压下毛刺刺的心跳,又想起了一件事——闷油瓶不是失忆了吗?怎么会认识裘德考?而且他躲什么?

转过头,我就问他。他还是看着帐篷的方向,答道:“我在医院的时候,见过他一次。”

“医院?是北京还是格尔木?”我们是被裘德考的人从柴达木接出来的,不过不记得碰到过他,他当时受的打击应该比我们更大。

“北京。”他回道,“就在上上个月。”

那就是在北京治病的时候。靠!裘德考见过闷油瓶?胖子怎么没告诉我?

再一想,他娘的胖子这个人要说义气绝对是够义气,但要他照顾人他肯定是不行的。我在杭州时,让他看着闷油瓶,想必是做一半放一半。而且闷油瓶这种人,单独和任何人相处都很困难,没有我在其中溜须打屁,胖子那没溜的性格肯定和他是大眼瞪小眼。闷油瓶见到裘德考的时候,他不知道在哪里溜达,所以不知道。

想起这个我就想骂人,闷油瓶是我们手中的一张大牌,怎么他见过裘德考我们都不知道?也就是说,如果裘德考狠点,闷油瓶被他接走都有可能,那我们上吊都不缺的。胖子真是太不上心了!闷油瓶也真是,什么都不说。

“他找你干嘛?”我问闷油瓶,“你怎么没和我说啊?老大。”

他没有回答,闪回了我身后。

回头一看,裘德考被人搀扶着从帐篷里出来,向四周望了望,戴上了帽子,朝一边的树阴走去。

“你躲什么?”我又问,“被他看到又怎么样?可能他早就知道你在这里了。”

闷油瓶摇头,对我道:“我们不能让他们抢先,必须斤他们的时间。”

“你想干嘛?”我问。

他指了指一边堆着的潜水器械,“我们去抢水肺。”

\chapter{使坏}

我立即明白了闷油瓶的意思,脑子里灵光一闪,只想了个大概就不由得叫好。

我们没有水肺,如果裘德考他们有任何行动,都只能干看。而回去拿水肺再返回的时间里,人家说不定早就搞定开路了。若这水下有什么关键之处,我们绝对没有任何机会获得先机。

确实如闷油瓶所说,这可能是唯一的机会了。

在他们还没有反应过来的时候去抢水肺,然后使其报废,这样没有了氧气瓶,他们有压缩空气机也没有办法。这是典型的先下手为强,在别人完全没有想到的时候就行动。

不过,现有的条件下是否能抢到?我抱有疑问。水肺放在河滩上靠湖比较远的地方,过去拿了就走,就算闷油瓶能一个打十个,他也不到我们,冲到湖里之前,我和胖子肯定就被按住抽死了。

想了想,我道:“你说得有道理,但这事急不来,人家这么多人,咱们不可能现在就挺着个肚子上。等到晚上,偷偷摸过去偷出来。”

闷油瓶摇头:“我们没有晚上了,一旦安定下业,他们会立刻下水,你看。”

他指向一个方向,那里已经有好几个人在湖边打充气筏,还有人走入了湖中,显然是潜水夫在观察环境。

“他们为什么这么急?”我很奇怪。

闷油瓶顿了顿,忽然就道:“也许,没有时间了。”

我愣了一下,这句话在他嘴里说出来很有深意,不过目前没工夫细琢磨。

小跑过去把胖子叫了回来,他一听我们的计划,啊了一声,摇头道:“我靠!刚和他们套了近乎就去抢劫,胖爷我的名声不得臭了?”

我说道:“这水下如果有明器,他们下水后可就全摸走了。你是要明器,还是要名声?”

胖子想了想道:“真奇了怪了,我觉得天真你的话特别容易说服人。那咱们就先不管名声了,你说怎么做?”

我再想了想,硬抢肯定是不行,便让胖子去准备小木排,重新上满石头。我们不可能背着负重的铅块冲进湖里,那么只能用石头来负重。之后,必须想一个办法吸引那些人的注意力,以便迅速地拿到水肺。

放水肺处到岸边的距离,如果全速奔跑,大概只需要三十秒。但在这条路上有很多在人忙碌,只要略一停顿,就会被人追上。在这么多人的眼皮底下偷东西,需要相当的技巧和心理素质。

这个我很不内行,怎么想也觉得不可能。而且经闷油瓶那么一说,觉得特别的紧张,感觉自己马上就要没机会了。

这时候还是胖子有办法,他看了看那些人,又看了看水肺的位置,突然道:“你们会骑马吗?”

“怎么?”我问。

他指了指一旁的骡子,打了个眼色:“看过蒙古骑手夺羊吗?”

我一下理解了他的意思,皱眉道:“骡子和马不一样,骡子跑不动啊!”

“我靠!我们又不赛马,只要它跑几十米。这东西这么大个子,跑起来谁敢拦?问题只有一个,中途千万别摔下来。”

有门儿!我狂点头。胖子马上就去准备。我们先把木排扒到湖里,然后回来,抑制出钱找到了看骡子的人,说想借去运点东西。

那人先前在村里见过我们,有钱当然赚。

胖子问:“骡子什么时候跑得最快?”

那人道:“发情的时候,拉也拉不住。”

胖子道:“这个难点,有啥需要避讳的?骡子最怕什么?”

打点妥当,我们拉着骡子,慢悠悠地走到他们忙碌的营地里。靠近放水肺的地方,互相看了看,我已经紧张得全身冒汗了。

三个人率着骡子,感觉特傻,跟墨西哥那些农夫一样。不过,倒没有多多突兀,因为四周好些骡子都在那里卸东西。

水肺里在一个大帆布包里,就几个包是连在一起的,胖子把骡子赶了赶,走近了点,给我打个眼色,让我去解绳子。

我看了看,没有人注意我们,刚想动手,却听到后面有人咸了一声:“喂!你们是干什么的?”

我条件反射下猛然回头,看到一个女人正朝这里走来,在树下纳凉的一行人也都站了起业。我一下就慌了,心说怎么办?被发现了!

那一刹那,胖子一个箭步,抓起水肺就大叫:“上骡子!”

我一下,也抓起了水肺。三个人立刻上了骡子,胖子用力一抽骡子屁股,大叫道:“骡子疯了!”

受到惊吓的骡子扬开四蹄,狂奔起来。

别看骡子平时走路慢腾腾的猛地一跑我差点没坐住,加上胖子和我的水肺是连在一起的,我们两个互相拉扯,好像玩杂技一下,十分危险。

所有人的注意力都被吸引了,后面的女孩子迅速反应了过来,大叫:“拦住他们!”

胖子估计得一点也没有错,这骡子跑起来声势惊人,往前狂冲而去,把前头两个正在搭遮阳棚的人吓得闪开,甚至摔倒在地。

胖子还在叫:“让开!当心!”

三个人狂冲向湖边,后面那女孩的喊声被尖叫完全淹没,而且这种情况谁敢上来?被骡子踩上一脚可是伤筋动骨的事情,一时间,湖边鸡飞狗跳。

我还没反应过来,骡子已经冲到湖边。它们怕水,一个急转身,我们几个都摔了下来。

我的额头磕在石头上,随后被胖子扶起来,骡子继续狂奔。回头一看,那女人带着几个人追了过来,我们连忙转身往湖里冲。

到了湖边,一下就占了优势。这湖的水位下降得非常快,冲入湖里,几下就到了脚够不着地的地方,我们拖着水肺往深水里去。游出好几十米后再回头看,那几个人也下水了。

游到小木排那儿,抱起石头,胖子大叫:“沉!”三个人一个猛子往水里一压,迅速往下沉去。

在水下,只见上面几个人已经游到了上方,差一点就要被他们拽住。有几个人潜水下来捞了一圈,但很快都浮了上去。

我们从容地套上水肺,戴上潜水镜。到底是专业设备,一下四周就清明了。我用鼻排水把潜水镜里的水排出去一半,负上水肺,戴上脚蹼,他们也已穿戴整齐。

裘德考的装备果然是高级货,腰带上还有一条工具带,里面有led lenser的潜水手电筒,潜水匕首和单体氧气罐,一罐可以坚持三分钟。把这些东西运到山里需要大量的手续,此人看来背景不浅。

全部检查完毕,我已经沉到了湖底,有了水肺能潜到两三百米,这点深度我完全不放在眼里。关键是对手没有水肺了,根本不用担心有人下水来撵。

胖子做了手势,指了指前方。这里离之前下水的位置还有一段距离,水深相对较浅,前方幽深一片,古寨就在那里。我们必须离开这个位置,这湖说大不大,说小不小,只要游了开去,在另一个地方上岸,他们就只能干瞪眼。打开手电筒,跟着胖子开始前进,最後到达谷寨上方,将铅快和氧气瓶都沉下去,看着它们掉入寨子的中央,然後一路潜泳到达湖泊另一边。

偷偷上岸的同时,就见湖对面一片气急败坏。

后来阿贵和云彩在山中接应了我们,我们心中暗笑,潜伏而回。

{\fzqiti\hfill (《阴山古楼篇》完)}