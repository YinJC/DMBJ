%# -*- coding:utf-8 -*-
%%%%%%%%%%%%%%%%%%%%%%%%%%%%%%%%%%%%%%%%%%%%%%%%%%%%%%%%%%%%%%%%%%%%%%%%%%%%%%%%%%%%%

\part{七星鲁王}
\chapter{血尸}

50年前,长沙镖子岭。

四个土夫子正蹲在一个土丘上,所有人都不说话,直勾勾地盯着地上那把洛阳铲。

铲子头上带着刚从地下带出的旧土,离奇的是,这一坏土正不停地向外渗着鲜红的液体,就像刚刚在血液里蘸过一样。

“这下子麻烦大喽。”老烟头把他的旱烟在地上敲了敲,接着道,“下面是个血尸嘎,弄不好我们这点儿当当,都要撂在下面噢。”

“下不下去喃?要得要不得,一句话,莫七里八里的!”独眼的小伙子说,“你说你个老人家腿脚不方便,就莫下去了,我和我弟两个下去,管他什么东西,直接给他来一梭子。”

老烟头不怒反笑,对边上的一个大胡子说:“你屋里二伢子海式撩天的,指不定什么时候就给翻盖子了,你得多教育教育,咱这买卖,不是有只匣子炮就能喔荷西天。”

那大胡子瞪了那年轻人一眼:“你崽子,怎么这么跟老太爷讲话,老太爷淘土的时候你他妈的还在你娘肚子里吃屎咧。”

“我咋说……说错了,老祖宗不说了嘛,那血尸就是个好东西,下面宝贝肯定不少,不下去,走嘎一炉锅汤。”

“你还敢顶嘴!”大胡子举手就打,被老烟头用烟枪挡了回去。

“打不得,你做伢那时候不还是一样,这叫上梁不正下梁歪!”

那独眼的小伙子看他老爸被数落了,低下头偷笑,老烟头咳嗽了一声,又敲了那独眼的少年一记头棍:“你笑个啥?碰到血尸,可大可小,上次你二公就是在洛阳挖到这东西,结果现在还疯疯癫癫的,都不知道着了什么道。等一下我先下去,你跟在我后面,二伢子你带个土耗子殿后,三伢子你就别下去了,四个人都下去,想退都来不及退,你就拉着土耗子的尾巴,我们在里面一吆喝你就把东西拉出来。”

年纪最小的那少年不服气了:“我不依,你们偏心,我告诉我娘去!”

老烟头大笑:“你看你看,三伢子还怯不得子了,别闹,等一下给你摸把金刀刀。”

“我不要你摸,我自己会摸。”

那独眼老二就火了,一把揪住老三的耳朵:“你这杂家伙跟我寻事觅缝啰,招呼老子发宝气喃?!”

那年纪最小的少年看样子平日挨过不少揍,一看他二哥真火了,就吓得不敢吭声了,直望着他爹求救,怎料他爹已经去收拾家伙去了。他二哥这下得意了:“你何什咯样不带爱相啰,这次老头子也不帮你,你要再吆喝,我拧你个花麻鸡吧!”

老烟头拍拍老二的肩膀,大叫一声:“小子们,操家伙啰!”说完一把旋风铲已经舞开了。

半个小时候后,盗洞已经打得见不到底了,除了老二不时上来透气,洞里连声音都听不清楚了,老三等得不耐烦起来,就朝洞里大叫:“大爷爷,挖穿没有?”

隔了有好几秒,里面才传来一阵模糊的声音:“不……知道,你……待在上面,拉好……好绳子!”

是他二哥的声音,然后听到他那老烟头咳嗽了一声:“轻点声……听!有动静!”

然后就是死一般的沉寂。

老三知道下面肯定有什么变故,吓得也不敢说话了,突然他听到一阵让人毛骨悚然的咯咯声,好像蛤蟆叫一样的从洞里发出来。

然后他二哥在下面大吼了一声:“三伢子,拉!”

他不敢怠慢,一蹬地猛地拽住土耗子的尾巴,就往外拉,刚拉了几下,突然下面好像有什么东西咬住了,竟然有一股反力把绳子向盗洞里拉去,老三根本没想过还会有这种情况,差点就被拉到洞里去,他急中生智,一下子把尾巴绑在自己腰上,然后全身向后倒去,后背几乎和地面成了30度角,这个是他在村里和别的男孩子拔河的时候用的招数,这样一来他的体重就全部吃在绳子上,就算是匹骡子,他也能顶一顶。

果然,这样一来他就和洞里的东西对峙住了,双方都各自吃力,但是都拉不动分毫,僵持了有十几秒,就听到洞里一声盒子炮响,然后听到他爹大叫:“三伢子,快跑!!!!!!”就觉得绳子一松,土耗子嗖一声从洞里弹了出来,好像上面还挂了什么东西!那时候老三也顾不得那么多了,他知道下面肯定出了事情了,一把接住土耗子,扭头就跑!

他一口七跑出有二里多地,才敢停下来,掏出他怀里的土耗子一看,吓得大叫了一声,原来土耗子上勾着一只血淋淋的断手。他认得那手的模样,不由哭了出来,这手是分明是他二哥的。看样子他二哥就算不死也残废了。想到这里,他不由一咬牙,想回去救他二哥和老爹,刚一回头,却看见背后蹲着个血红血红的东西,正直勾勾地看着他。

这老三也不是个二流货色,平日里跟着他老爹大浪淘沙,离奇的事情见过不少,知道这地底下的,什么事情都有可能发生,最重要的不是大惊小怪,而是随机应变,要知道再凶险的鬼也强不过一活人,这什么黑凶白凶的,也得遵守物理定律,一梭子子弹打过去,打烂了也就没什么好怕的了。

想到这里,他把心一横,一边后退,一边腰上别着的一支匣子炮已经拽在手里,开了连发,只要那血红的东西有什么动静,就先给他劈头来个暴雨梨花。谁知道这时候那血红的东西竟然站起来了,老三仔细一看,顿觉得头皮发麻,胃里一阵翻腾,那分明是一个被剥了皮的人!混身上下血淋淋的,好像是自己整个儿从人皮里挤了出来一样。可是这样的一个人,竟然还能走动,那真是奇迹了,难道这就是血尸的真面目?

想着,那血尸一个弓身,突然就扑了过来,一下子老三就和他对上眼了,那血淋淋的脸一下子就贴着他的鼻子,一股酸气扑面而来,老三顺势向后一倒,同时匣子炮整一梭子子弹全部近距离打在了那东西胸膛上,距离过近,子弹全部都穿了过去,把那东西打的血花四溅,向后退了好几步。老三心中暗喜,再一回手对准那东西的脑袋就一扣扳机,就听喀嚓一声,枪竟然卡壳了!

这把老匣子炮是当年他二爷爷从一个军阀墓里挖出来的,想来也没用了多少年月,可惜这几年跟着他爹爹到处跑,也没工夫保养,平时候开枪的机会也少之又少,谁知道竟然在这节骨眼上卡壳了。那老三也真不简单,一看枪不好使唤,轮圆了胳膊用吃奶的力气把枪给砸了过去,也不管砸没砸到,扭头就跑。这次他连头也不敢回,看准前面一颗大树就奔了过去,寻思着怎么着它也不会爬树吧,突然脚下一绊,他一个狗吃屎,整张脸磕在一树墩上,顿时鼻子嘴巴里全是血。

老三狠狠一巴掌拍在地上,心里那个气啊,妈的怎么就这么背。

这时候听到后面风声响起,知道阎王爷来点名了,心一横,死就死吧,索性就趴在地上不起来了。没成想,那具血尸好像没看到他一样,竟然从他身上踩了过去,那血淋淋的脚板马上在他背后印下一个印子,这血尸出奇的重,一脚下去,老三就觉得嗓子一甜,只觉胆汁都被像踩吐了出来,而且背上那被踩过地方马上一阵奇痒,眼前马上朦胧起来,他马上意识到自己可能中毒了,而且毒性还非常的猛烈,恍惚间他看到不远处的地方,他二哥的那只手里好像还握着什么东西。

他用力眨了眨眼睛,仔细一看,原来是一块古帛片。他心想,自家二哥拼了命都要带出来的东西,肯定不是寻常东西,现在又不知道他们怎么样了,我好歹得把东西收好,万一我真的死了,他们找到我的尸体,也能从我身上找得着,那二哥的这只手也不算白断了,我也不至于白死。他一边这么想着,一边艰难地爬过去,用力掰开二哥紧握的手把那帛片从掌心里拿出来,塞进了自己袖子里。

这个时候他的耳朵也开始蜂鸣了,眼睛就像蒙了一层纱一样,手脚都开始凉起来。按他以往的经验,现在他裤裆里肯定大小便一大堆,中尸毒的人都死得很难看,他现在最希望的是不要给隔壁村的二丫头看见自己这个样子。

他就这么混混着胡想,脑子已经不怎么听他使唤了,这时候他又开始隐隐约约地听到他在盗洞口听到的那种咯咯怪声。

老三隐约觉得一丝不对,刚才和血尸搏斗了这么些时候,也没听它叫过一声,现在怎么又叫起来了?难道刚才的那只并不是血尸?那刚才看到的又是什么东西呢?可惜这个时候他已经基本无法做思考了,他条件反射地抬起头看了一下,只见一张巨大的怪脸正俯下身子看着他,两只没有瞳孔的眼睛里空荡荡地毫无生气。

\chapter{五十年后}

50年后,杭州西泠印社,我的思绪被一个老头子打断了,我合上我爷爷的笔记,打量了一下对方。

“你这里收不收拓本?”他问,看样子就是随便问问的,我做这行挺有天分的,也就敷衍他:“收,不过价钱收不高。”意思是,你没好东西就滚吧,别耽误大爷看书。

做我们这行,三年不开张,开张吃三年,平日里清闲惯了,最讨厌伺候那些一知半解的客人,演变到后来,只要看到那些过路客,就直接放哀乐赶人。不过最近空闲的也有点过分了,眼看旺季快过了,也不见什么好东西进来,所以也有点耐不住。

“那我想打听一下,这里有没有战国帛书的拓本?就是50年前,长沙那几个土夫子盗出来,又被一美国人骗走的那一篇?”那人一边看着我柜台里的藏品,一边问。

“你都说被美国人骗走了,哪里还有。”我一听就火了,“找拓本当然是去市场里淘,哪有指定了一本去找的,怎么可能找得到?”

他压低了声音:“我听说你有门路,我是老痒介绍来的。”

我警惕起来,心里一惊,老痒不是前年就进号子里,怎么,把我供出来了?心里一急,背上冷汗就出来了,“哪……哪个老痒,我不认识。”

“我懂我懂,”他呵呵一笑,从怀里掏一只手表,“你看,老痒说你一看这个就明白了。”

那手表是老痒当年在东北的时候他初恋情人送给他的,他把这表当命一样,喝醉了就拿出这表边看边“鹃啊,丽啊”的叫,我问他你那老娘们到底叫什么,他想半天,竟然哭出来,说我他娘的给忘了。这老痒肯把这表给这个人,说明这人确实有些来头。

可我怎么打量这人都觉得面目可憎,但人家找上门来了,还是爽快点说话好,于是直接一抬手:“那就算你是老痒的朋友,找我什么事情?”

他露牙齿一笑,露出一颗大金牙:“我一个朋友在山西带回点东西,想让你给我看看,那是不是真东西。”

“看你一口京腔的,你北京的大土靶子到南方来找我咨询,太抬举我了吧,北京多少好手,恐怕你醉翁之意不在酒啊!”

他嘿嘿一笑:“都说南方人精明,果然不假,看你年纪不大,倒也看得很通透,说实话,我这次来,确实不是找您,我想见见你家里老太爷。”

我的脸色一下就变了:“找我爷爷,你什么居心?”

“你老太爷当年在长沙镖子岭盗出战国帛书以后,是否留有一两份拓本?我朋友只想知道,与我们手上这一卷是否一样?”

他话没说完,我对着边上打瞌睡的伙计吼道:“王盟,送客!”

那金牙老头急了:“怎么遭说着说着就要赶人呢?”

“你说的是不错,可惜你来太晚了,我老爷子去年已经西游,你要找他,回去割脉吧!”我心道:“当年那事情,连中央都惊动了,那是大事情,哪能给你把旧帐翻出来,我家里还能有好吗?”

“我说你个小孙子,说话就怎么不中听呢。”大金牙老头一脸贼笑,“老爷子不在了也不打紧,我也没说怎么着啊,好歹,你也看一看我带来的东西,你也卖卖老痒的面子不是?”

我看了他一眼,这皮笑肉不笑的,看样子不看他一眼他还真不肯走,心说就当卖老痒个面子,他出来的时候也不用被他埋怨,于是点头:“看看就看看,是不是我可不敢说。”

其实这战国帛书有20多卷,每卷各不相同,我爷爷当时拓下来的那一篇只是其中很短的一部分,但是又极其重要,现在也就是我有几份拓本当压箱底的宝贝,世面上有钱也买不到,只见那金牙老头从怀里掏出一张白纸,我一看就来气,靠,还是个复印件。

“那是啊,那宝贝那能到处揣着跑啊,一抖就碎。”他说,还故作神秘的压低声音,“要不是我路子广,这东西早跑到国外去了,也算是为人民服务。”

我呵呵一笑:“看你那样子不就是个倒斗的吗,我看你是不敢出手,这是国宝,你脑袋不想要了!”

一句话被我揭穿,老头子脸就绿了,可他有求于我,还得忍着,说:“也不能这么说,每一行都有每一行的道道,想你老爷子当年在长沙做土夫子的时候,那也是威名远播……”

我脸色肯定很难看,咬着牙:“你要再提我爷爷,我就不看了!”

“好好,咱打住,你快给我瞅瞅,我也好快点跑路。”

我展开那白纸头,一看就知道,这是一篇保存完好的战国帛书,但并不是我爷爷当时盗出来的那一份,这一份虽然年代也比较久远,但是应该是后几朝的赝品,也就是说是古董赝品,这是个身份很尴尬的东西。于是我一笑:“这应该是汉代的赝品,怎么说呢,你说它是假的,也不是假的,说它是真的,也不是真的,鬼知道这是照本摹的还是胡编的?所以我也不知道怎么说好了。”

“那这是不是你爷爷盗出来的那一份?”

“实话和你说,我爷爷盗出来的那份他自己都没来得及看上一眼就被那美国佬骗过去了,你这问题我实在回答不了你。”我心想,忽悠你还不容易,表情上还装出特诚恳的样子。那金牙老头还真信了,叹了口气:“那真是不凑巧,那看样子不去找那个美国人,恐怕还真没指望了。”

“怎么,你们怎么就这么在意这一卷?”我问道,这太奇怪了,这古籍的收藏都是看缘分的,想把一套20卷战国时期的古籍都找到,那也未免太贪心了。

“小兄弟,不瞒你说,我还真不是倒斗的,你看我这身子骨,哪够折腾啊,不过我那朋友的确是行家里手,我也不知道他卖的是什么关子,总之,人家有人家的道理。”他呵呵一笑,摇摇头,“咱也不好多问,对吧,先走一步了。”说完头也不回的就走了。

我低头一看,他那张复印纸还在我手里呢,突然,我在那纸上发现一个图案,那是个狐狸一样的人脸,两只没有瞳孔的眼睛很有立体感,好像从那纸上凹了出来一样,看得我吸了口凉气,这一份帛书我从来没见过,应该是一份珍品。我琢磨着等老痒出来,就用这复印件做几块假的拓片也够我乐的。忙急急跑到门外张了一眼,只看到那金牙老头正往回赶。

我心想他肯定是回来拿这张东西,忙跑回去,拿起数码相机把它给拍了下来,然后拿起纸头走出门外。迎面碰上大金牙老头的鼻子:“你东西忘了。”我说道。

我爷爷是长沙土夫子,也就是一般说的“盗墓贼”。

我爷爷入这行的原因一点也不出奇,用现在的话说那就是世袭的行当。我太公的太公13岁那年,华中一带闹旱灾,那年代,一闹旱灾就起饥荒,你有钱也买不到东西吃,那时候长沙边边角角里啥都没有,就是古墓多,于是靠山吃三,靠墓吃墓,全村人一起倒斗,那几年不知道长沙一带有多少人饿死,可就他们那村一个都没死,还一个一个都吃个油光满面的,那可全是靠着用挖出来的东西跟洋人换粮食吃才能这样的。

再后来时间长了,盗墓这东西和其他东西一样,也有个文化的积累,到我爷爷那辈,已经有行规、门派之分,那个时候盗墓的分南、北两派,南派就是我爷爷那派,擅长洛阳铲探土,高手只凭一个鼻子就能断定深浅朝代,现在很多小说里描写动不动就洛阳铲,其实北派是不用洛阳铲的,他们精于对陵墓位置、结构的准确判断,也就是所谓的“寻龙点穴”。但是北派的人有点古怪,怎么说呢,按我爷爷的说法那就是他们不实在,花花肠子太多,盗个墓还搞这么多名堂,进去东西拿了就走呗,还要一扣二扣的,搁现在就叫官僚主义得很。而南派规矩就不多,且从不忌讳死人,北派人骂南派是土狗,糟蹋文物,倒过的斗没一个不塌的,连死人都拉出来卖,南派骂北派是伪君子,明明是个贼还弄得自己跟什么似的,后来更是闹到要火拼的地步,甚至还有“斗尸”一类的事情发生,到最后两派终于划长江而分,北派叫倒斗,南派就叫淘沙或是淘土,洛阳铲还是分了之后才发明出来,北派人根本不屑使用。

我爷爷他不认识字,后来进了扫盲班,那时候他只会淘沙,学个字差点把他折腾死,也亏了他有了文化,才能把他的一些经历记录下来,在长沙镖子岭那老三,就是我爷爷,这些事情都他是一个字一个字记录在他那本老旧的笔记本上,我奶奶是个文化人,大家闺秀,就是被他的这些故事吸引,最后我爷爷就入赘到杭州来,在这里安了家。

那笔记算是我家的家传宝贝,我爷爷的鼻子在那次的事情后就彻底废掉了,后来他训练了一只狗来闻土,人送绰号“狗王”。这是真事情,现在长沙做过土夫子的,老一辈的人都知道这名字。

至于我爷爷后来怎么活下来的,我的二伯伯和太公和太太公最后怎么样了,我爷爷始终不肯告诉我,在我记忆里面,我也没有看到过一个独眼独臂的二伯,估计真的是凶多吉少,一提到这个事情,我爷爷就哭,就直说:“那不是小孩子能听的故事。”无论我们怎么问,怎么撒娇,他也不肯透露半个字。最后随着我们年龄的增长,也逐渐失去了童年的好奇心。

傍晚,店子打烊,又是无聊的一天过去了,屁东西也没有收进来,我打发掉伙计,这个时候,一个短信息发过来。

“9点鸡眼黄沙。”

是家里三叔发过来的,这是暗话,就是说有新货到了,紧接着,又是一条:“龙脊背,速来。”

我眼睛一亮,三叔的眼光出奇的高,这龙脊背就是有好东西的意思,连他都觉得是好东西,我真要见识一下。

我关好店门,开着我的破金杯车就直奔我三叔那里,一方面想看看他所谓的好东西是什么,另一方面,也想让他看看我今天拍到的那份帛书上的图案到底是什么?到底他是我们这一代人中唯一还和土夫子有接触的人。

我车刚开到他楼下,就听他在上面叫:“你小子他娘的,叫你快点,你磨个半天,现在来还有个屁用!”

我靠了一声:“不是吧,好东西也留给我啊,你也卖得太快了。”

正说着,我看到一个年轻人从他正门里面走了出来,身上背了根长长的东西,用布包得结结实实的,一看就知道应该是一把古兵器,这东西的确值钱,要是卖得好,价格能翻十几倍上去。

我指指那年轻人,我三叔叔点点头,做了无可奈何的个手势,我心里一阵悲哀,心想难道我的小摊子今年真的要破产了?

我上了楼,自己搞了杯咖啡,把今天那金牙老头跑来刺探事情和三叔一说,本以为他会和我同仇敌忾,没想到他好像变了个人一样,沉默不语,直接把我数码相机里的东西打印了出来,放在灯下一看,我马上看见我三叔脸色变了。

“怎了?”我问道,“这东西有什么蹊跷?”

他皱起眉头,说道:“不会吧,这张好像是张古墓的地图啊!”

\chapter{瓜子庙}

我看看上面满是文字的帛书打印件,又看看三叔的表情,不像是开玩笑啊,怎么难道三叔叔已经超脱到能从字里看出画来的地步了?怎么看这平日里吃喝嫖赌的老不正经也没什么仙根啊。

三叔兴奋得不住得发颤,一边自言自语:“这些人从哪里搞来这么好的东西,怎么我就从来碰不到,这次真是造化了,看样子他们还搞不清楚这是什么,我们可以赶在他们之前把这拨沙子给淘了。”

我大大迷惑:“三叔,也许我是笨了点,可您真能从这么小的字里看出地图来?”

“你懂什么,这叫字画,就是把那地方详细的地理位置用文字写出来,这东西,如果是别人还真看不懂,幸亏你三叔我还有点阅历,这世界上,能看懂这玩意的除了我之外恐怕不超过10个人。”

我三叔没什么其他本事,但是从小对那些稀奇古怪的非正统的古代文字和暗语非常得有研究,一句话概括,就是什么东西生僻他就研究什么,像什么西夏的五木书图,女真最早期的牙字,他都能说出个道道来。所以他能知道这个什么劳什子的字画,我倒是一点也不惊讶。

不过他这个人是得了便宜便卖乖的那种类型,在他面前还得装笨,不然他一句话就把你打发了,于是我装出很憨的表情,问他:“哦,那上面是不是写着向左走然后向右走,看见前面大树向右拐,看见一口井然后钻下去?这样?”

三叔叹了口气:“儒子不可教也,你的悟性这么差,看样子我们家到你这一代就玩完了。”

我看他这个样子,还叹的真是真切,似乎是心里话,不由觉得好笑:“那你说是怎么样的?我爹又不教我,这东西又不是天生的。”

他得意地嘎嘎嘴,说道:“这种字画,其实是种密码,它有严格的格式,只要把里面写的东西按照它的格式画出来,就是一幅完整的地图了,所有你不要小看这区区几个字的帛书,不知道里面的信息有多复杂,说不定连哪里用了多少块砖都标得很清楚。”

我一听就来了兴趣,心说我从小到大,家里也没让我出去倒个实斗,这一次必然要让三叔带我去见识一下,摸几个宝贝也好度过我的经济危机。这么一边想着一边就问他道:“那你能不能看出里面写着是谁的墓,或者是不是比较有来头的主?”

三叔得意地一笑:“我现在不能完全看懂,不过这个墓穴应该是战国时期鲁国的一个贵族的,关看他的墓穴所在被人用这种隐秘的字画方式记录在这张帛书上,说明此人的地位应该相当高,而且这个墓地必然十分隐秘是个好斗,一定值得一去。”

我看他眼睛里直放光的样子,就觉得稀奇,这老家伙平日里门都懒得出一步,难道这次竟然想亲自出马?那真是千古奇闻了,忙问他:“怎么?三叔,你真的打算亲自去淘这拨沙子?”

他拍拍我的肩膀:“这你就不懂了吧,和你说,唐宋元明清,那斗里面是有宝贝,但那最多只能说是巧夺天工,但是战国的时候,那时期的皇族古墓,年代过于久远了,你永远也估计不到那里面有什么东西,那战国墓可是出神器的地方,那可都是人间没有的东西!你说我能不想见见嘛?”

“你就这么肯定?说不定里面啥都没有呢?”

“不会,你没看这图案吗?”他指了指那张诡异的狐狸脸:“这是鲁国最早人牲时候祭祀带的面具,这墓里埋一定是什么身份很特殊的人,可能比当时的皇帝还要尊贵。”

我脱口而出:“皇帝他爹。”

三叔瞪了我一眼,就想把那张打印纸收起来,我一把按住,朝他一笑:“三叔,你别急着收起来,怎么说这东西也是我搞来的,这次你怎么样也要带我去见识一下。”

他大叫:“不行,淘这沙可不是这么简单的,那地方可没空调,还机关重重的,随时可能要歇菜。你是你爹的独苗,你要是有个三长两短,我非让你爹给扒了皮不可。”

我也大叫:“那拉倒!就当我没来过!”说着把那纸头从他手里猛地抽了出来,转头就走。我知道三叔这人,一旦遇到自己喜欢的东西,就一点原则也没有,看到古董这样,看到女人也这样,我就吃准他这一点,果然才走了几步,他就投降了,追上来,一把拉住我手里的纸:“好好好,你厉害,不过咱可说好了,我们下盗洞的时候,你可得待在上面。这样总行吧?”

我顿时心花怒放,心说:到时候我要下去你还能拦得住我?忙点头道:“一句话!出门在外,就全听你的,你让我干吗我干吗!”

三叔无奈地叹了口,说:“我们两个人还不成事,我明天再调几个有经验的伙计过来,这几天我就解这张字画,你得帮我去置办些东西。”说着他迅速写了张条子给我,对我说,“千万别买了假货,还有,准备套旅游的行头出来,不然还没到地方,我们就先拘留了。”我点头答应,就各自分头去忙。

三叔要的东西都比较刁钻,我觉得恐怕是故意为难一下我,因为这单子里的东西一般店里还真没有,比如什么分体式防水矿灯,螺纹钢管,考土铲头,多用军刀,折叠铲,短柄锤,绷带,尼龙绳等等,我才买了一半就花了将近一万块钱了,心里那叫这个心疼啊,直骂这老狐狸,妈的这么有钱还这么吝啬。

三天后,我还有我三叔的两个老淘沙的伙计,还有那天买了我叔叔那手龙脊背货色的小伙子,五个人到了山东瓜子庙再往西100多公里的地方。

说起这地方,该怎么说呢,真只能说这就是一个地方,什么都没。我们先是长途汽车,然后是长途中巴,然后是长途摩托,然后是牛,我们最后从牛车下来的时候,前看后后左看右看还是什么都没,然后就看到前面跑来一只狗,我三叔一拍请来的向导,“老爷子,下一程咱骑这狗吗,恐怕这狗够戗啊!”

“不会,”老爷子大笑,“这狗是用来报信的,这最后一程啊,什么车都没,得做船,那狗会把那船带过来。”

“这狗,还会游泳?”

“游得可好咧,游得可好咧,”老头子看着那狗,“驴蛋蛋,去游一个看看。”

那狗还真有灵性,真跳到河里游了一圈。上来抖抖毛,就趴地上吐舌头。

“现在还太早,那船工肯定还没开工,咱们先歇会儿,抽口烟。”

我一看表:“下午2点还没开工,你这船工是什么作息时间啊?”

“我们这里就他一个船工,他最厉害,他什么时候起来什么时候开工,有时候一天都不开工,能把人急死。”老头子笑笑,“没办法,这河神爷只卖他面子,别人,只要一进那山洞洞就肯定出不来,就他没事。要是你们会骑骡子,我们就能从山上翻过去,再一天也能到,不过你看你们这么多东西,我们全村的骡子也不够你们用的。”

“哦。”三叔一听到山洞,马上来劲了,拿出翻译好的地图,这地图他一直当宝贝一样,看都不让我看一眼,他一拿出来,我们马上凑过去看,只有那个小伙子还是一言不发坐在一边。

说实话,我二叔两个伙计很好相处,都是实在人,就这人像个闷油瓶,一路上连屁都没放过一个,只是直勾勾看着天,好像忧郁天会掉下来一样,特讨厌!我一开始还和他说几句话,后来干脆懒的理他,真不明白三叔把他带来干什么。

“有山洞,还真是个河洞,就在这山后面。”三叔说,“怎么老人家,这山洞还能吃人?”

老头子呵呵一笑:“都是上几代留下来的话了,我也记不清楚了,那河道没通的时候,村里都说里面有蛇精,进去的人一个都没出来过,后来有一天,那船工的太爷爷就从那洞里撑了个小船出来了,说是外面来的货郎,你说这货郎哪有扛着只船到处跑的?大家都说他是蛇精变的,他太爷爷就大笑,说船是他隔壁村里买的,不信可以去隔壁村问,他们跑去一问,果然是这样,别人才相信,还以为那洞里的妖怪已经没了,结果胆子大的几个年轻人去探洞,又没出来。从那以后只有他家的人能够直进直出,你说古怪不?后来他们家就一直做这一行,一直到现在。”

“那狗没事情吗?”我奇怪了,“不是用它报信的吗?”

“这狗也是他家养的,别人家别说是狗了,牛进去都出不来。”

“这么古怪的事情,政府就没人管?”

“那也要说出去有人信才行。”老头子在地上敲敲旱烟管。

三叔眉头一皱,拍拍手:“驴蛋蛋,过来。”

那狗还真听话,屁颠屁颠就跑过来了,三叔抱起他一闻,脸色一变:“不会吧,难道那洞里有这东西?”

我也抱起来一闻,一股狗骚味道呛得我一阵咳嗽,这狗的主人也真懒,不知道多久没给这狗洗澡了。

他一个叫潘子的伙计哈哈大笑:“你想学你三叔,你还嫩着呢。”

“这死狗,怎么这么臭!”我恶心得直咧嘴。

“这狗小时候就吃死人肉长大的。”三叔说道,“那是个尸洞,难怪要等时间才能过,那船工,小时候恐怕也是……”

“不会吧!”我吓得寒毛都倒立起来,这句话一出,连那闷声不响的小子的脸色都变了。

我三叔的另一伙计是一个大汉,我们叫他阿奎,看他块头都和拉车那牛差不多大了,胆子却很小,轻声问:“那尸洞到底是什么东西?进去会不会出事情?”

“不知道,前几年我在山西太原也找到这么一个洞,那里是日本人屠杀堆尸的地方,凡是有尸洞的地方必有屠杀,这个是肯定的,那时候看着好玩就在那里做实验,把狗啊,鸭子的放在竹子排上,然后架上摄像机,推进去,那洞最多一公里多点,我准备了足够长的电缆,可是等到电缆都拉光了,那竹排子都没出来,里面一片漆黑,不知道漂到什么地方去了,后来就想把这竹排子拉出来,才拉了没几下,突然竹排子就翻了,然后就……”三叔手一摊,“最后只看到一半张脸,离得屏幕太近了看不出是狗还是什么东西。要过这种洞,古时候都是一排死人和活人一起过去的,要是活的东西,进去就出不来!不过,听说山西那一带有个地方的人从小就喂小孩子吃死人肉,把尸气积在身体里,到了长大了,就和死人没什么两样,连鬼都看不到他。老爷子,你那船工是不是山西过来的?”

老头子的脸色微微有些变化,摇摇头:“不晓得哦,那是他太爷爷那时候的事情了,都不是一个朝代人。”说着看了看天,对那狗叫了一声,“驴蛋蛋,去把你家那船领过来!”那狗呜的一声,跳进水里就游往山后面游去。

这个时候,我看见,三叔叔对潘子使了个眼色,潘子偷偷从行李里取出一只背包背在身上,那个一边坐着的年轻人,也站了起来,从行李堆里拿出了自己的包,潘子在走过我身后的时候,轻声用杭州话说了一句:“这老头子有问题,小心。”

\chapter{尸洞}

这一路过来,凶险的事情遇到不少,这几个伙计,非常厉害,我对他们非常信任了。所以,潘子一说这话,我就心里有数了,大个子阿奎也朝我使了个眼色,意思是你就缩后面,什么动静都别探头看。我不由苦笑,我凭什么探头啊?你一个阿奎一拳就能把一头牛打蒙掉,潘子就不用说了,退伍老兵,一身的伤疤,俺们三叔从小就是打架不要命的角色,还有那闷声不吭的拖油瓶,怎么看也不像个善类,而我,自古书生最无用,三叔硬塞给我的军刀我都觉得手感太重,怎么用怎么别扭。

正想着我该带个什么东西防身,驴蛋蛋扑通扑通游了回来,老头子把烟枪往裤管上一拍,“走!船来了。”

果然,两只平板船一前一后从山后驶了出来,前面那船上站着个中年人,一边撑船一边对着我们吆喝,这船还真不小,看样子装我们几个加上装备是绰绰有余了,老头子拍拍牛脖子:“各位,行李就不用拿下来了,我把牛和车一齐拉上第二只船,我们就坐第一只船里。省点力气。”

潘子一笑:“有些东西见不得水,还是随身带着好,等一下那牛跳水里去,那我们不歇菜了嘛?”

老头子笑着点头:“你说的也是个理,不过俺这牛也不是水牛,绝跳不到水里去。要跳下去,我老汉帮你们都捞上来,一件也少不了你们的。”

说着牵着牛就先走到渡头上去了,我们几个各自背着自己的随身行李,跟在后面。那中年人船撑得很麻利,几下就到岸了。

在老头子赶牛上第二只船的时候,我打量了一下那撑船的中年人,皮肤黝黑黝黑的,极其普通,但是不知道是心理作用还是什么,总觉得这人看上去鬼鬼的。又想起三叔说起的吃死人肉的事情,突然觉得那人越看越恐怖。

“等一下各位到洞里的时候,千万小声说话,不要惊动河神。”那人说,“特别是不要说河神的坏话。”

大概多少时间能过那个洞,我三叔问他。

“快的话,5分钟就过去了,里面水很急的,快昨很。”

“怎么还有慢的时候?”

“是,有时候这水是逆流的,你看我刚才是顺流出来的,那现在我们肯定逆流进去了,那时间就长了,估计要个15分钟,有几个弯还挺险。”

“那里面亮不?”

那人嘿嘿一笑:“黑灯瞎火的,怎么可能会亮,可以说是漆黑一片,”不过他指了指耳朵,“我撑了十几年的船了,这几篙子,用耳朵就行了。”

“那我们打个手电行不?”潘子扬了扬他手里的矿灯,“总不碍吧?”

“不碍事,”那人说,“但是千万别照水里,吓死你们!”

“怎么?”我三叔一笑,“有水鬼啊?”

“那水鬼算个啥,这水里的东西,我也不敢说是什么,你们要胆子真大,待会儿自己看一眼,记得,看一眼就得了。你们要运气好,就看到一团黑水,要运气不好,看到的东西能把你们吓疯过去。”

说着,我们已经能看到那洞了,这洞藏在山壁后面,我们在岸上的时候一直看不到,总把它想象成一个大洞,但是实际一看,不由叫了一声不好,没想到这洞这么小,小到刚比这船大了10个公分,最恐怖的是它的高度,人坐着都进不去,要低下身子才能勉强进去,这么大的空间,如果里面的人要暗算我们,我们根本活动不开手脚。潘子怪叫了一声:“靠,这洞也忒寒碜了点吧?”

“这还算大的,里面有一段,还要低呢。”后面的老头子说道。

三叔看了潘子一眼,潘子造作的一笑:“啊,这么小的洞,要是里面有人打劫我们,不是想逃都逃不掉?”

这话一说,我看到撑船的中年人做了一个很不明显的手势,老头子脸色一变。我心说,果然有问题啊,这时候我们就听到一阵呼啸,船已经进洞了。

潘子打开了矿灯,这洞刚进去还段还光亮,但是很快所有的光线就只剩下这矿灯了。

“三爷,这洞不简单啊。”阿奎说道:“这是盗洞啊!”

“水盗洞,古圆近方,你看这些痕迹,这洞有年头了,看样子,这洞里应该另有乾坤。”

“哦,这位看样子有些来头,说的不错。”那中年人猫着腰单膝跪在船头,单手撑篙,一点一划,但是奇怪的,他的篙子根本不沾水,他人更是大气都不喘,接着说道:“听说啊,这整座山啊,就是座古墓,这附近这样大大小小的水盗洞还真不少,就这个最大,最深,你也看到了,恐怕那时候这水还没有这么高,那时候应该还是个旱洞。”

“哦,看样子你也是个行家啊。”三叔客气地递过去支烟。他摇摇,说:“什么行家,我也是听以前来这里的那些个人说的。听得多了,也就能说上两句了,也就知道这么点浅显的。你可千万别说我是行家。”

潘子和大奎的手都按在自己的刀上,一边和那几个人说笑,气氛看上去十分的融洽,其实每个人都不知道有多紧张。我心说,我们有五个人,他们只有两个人,要真的动起手来,也不见得会输,但是他们既然敢动手,那肯定是有什么周全的准备在。

正想着,突然那闷油瓶一摆手,“嘘,听!有人说话!”我们马上屏气息,果然听到窸窸窣窣声音从洞的深处传来,我仔细想分辨他们在说什么,可总觉得能听懂又听不懂,听了一会儿,我回头想问那中年船工这洞里是不是经常会有这个声音,竟然发现他人已经不见了!再一回头,靠,那老头子也不见了。

“潘子,他们到哪里去了?”三叔急得大叫。

“不知道,没听见跳水的声音,”潘子也慌了,“刚才一听到声音,人突然就走神了。”

“遭了,我们身上没尸气,不知道会发生什么事情!”三叔懊恼起来,“潘子,你在越南打过仗,你有没有吃过死人!”

“开玩笑,三爷,我那时候在炊事班天天刷盘子!”潘子一指阿奎,“胖奎,你不是你说家里老早是卖人肉包子的,你小时候肯定吃了不少。”

“放屁,我乱盖的,再说了,这人肉包子也是卖给别人吃的,你见谁卖人肉包子自己拼命吃的?”

我忙打了暂停的手势:“你们三个人加起来150多岁,丢不丢人啊!”

我话刚说完,船突然抖动了一下,潘子忙拿起矿灯往水里一照,我们借着灯光,就看到水里一个巨大的影子游了过去。

胖奎吓得脸都白了,指着那水里,下巴咯哒了半天,愣没说出一个字来。三叔怕他背过气去,猛扇了他一巴掌,骂道:“没出息!咯哒啥呢,人家两小鬼都没吭声,你她妈的跟了这么多年,吃屎去了?”

“我的娘啊——三爷,这东西也忒大了!咱几个恐怕还不够开饭的。”胖奎心有余悸地看着水里,他本来是坐在船舷上的,现在屁股已经挪到船中间来了,好像怕水里有什么东西突然蹿出来把他叼去。

“我呸!”三叔狠狠瞪了他一眼,“我们这里要家伙有家伙,要人有人。我吴家老三淘了这么久的沙子,什么妖魔鬼怪没见过?你没事少在这里给我放屁。”

潘子也吓得够呛,不过对于他来说,与其说是恐惧,更不如说是震撼,在这么狭窄的一个空间里,水里下掠过这么巨大的一个东西,一时间所有人脑子都抽筋了,这也不奇怪。潘子看了看四周说:“三爷,这洞里古古怪怪的,我心里瘆台阶慌,什么事情咱出去了再说,如何?”

胖奎马上表示同意,其实我心里也巴不得出去,但我到底是三叔的本家,怎么样也要等他表态了再发言。

三叔这个时候竟然望向那个闷油瓶,好像在征求他的意见,以三叔的个性,天王老子都不放在眼里,如今却好像对这个小子非常的忌讳,我不由奇怪,转过头去看他怎么表态,却发现他根本没在听我们说话,而且本来木然得像石雕一样的表情已经不见了,两只眼睛直盯着水里,好像在聚精会神地找什么东西。

我想问问三叔这人到底是什么来头,现在场合也不合适,只好偷偷问潘子,潘子也摇摇头说不知道,只知道这人有两下子,他特别用下巴指了指那人的手,说:“你看,这手,要多少年才能练成这样?”

这之前我还真没注意过那人的手,这一看,发现还真不寻常。

他的手,中指和食指特别的长,我马上联想到古时候发丘中郎将的双指探洞的工夫,我在我爷爷笔记上看到过相关的记载,那发丘中郎将里的高手,这一双手指,稳如泰山,力量极大,可以轻易破解墓穴中的细小机关,而要练成这么一手绝活,非得从小练起不可,其过程必然是苦不堪言。

我还在想着,到底他这手有什么能耐,就见他抬起右手,闪电般插进水里,那动作快的,几乎就是白光一闪,他的手已经回来了,两个奇长的手指上还夹着一只黑糊糊的虫子,他把这虫子往甲板上一扔,说:“刚才就是这东西。”

我低头一看,不由松了一口气:“这不是龙虱吗!这么说刚才那一大团影子,只是大量的水虱子游过去?”

“是。”那人用他的衣服擦了擦手。

虽然还不是很相信,但是我们已经松了口气。胖奎突然一脚把那虫子踩扁,“妈的,吓得老子半死。”

但是我转念一想,不对啊,怎么可能有这么多龙虱同时活动的?而且这水虱,个头也太大了!那闷油瓶也好像不是很释怀的样子,看样子也在思考这个问题。

胖奎还在用脚踩那虫的尸体,已经稀烂了,估计是想挽回点刚才失态的面子,三叔捡起一只断脚,放在鼻子下闻了闻,骇然道:“这不是龙虱,这是尸蹩。”我们一呆,都觉得不妙,这名字听上去就不吉利。

“我的姥姥,这东西是吃腐肉的,有死物的地方就特别多,吃得好就长得大,看样子这上游,肯定有块地方是积尸地。而且还是了不得的大。”三叔看着那黑漆漆的洞。

“那这东西咬活人不?”大奎怯怯地问。

“如果是正常大小的,那肯定不咬人的,但是你看这只的个头,它咬不咬人我还真不能肯定。”三叔纳闷地看着,“这东西一般都待在死人多的地方,不会经常游来游去,怎么现在这么一大群一起迁移呢?”

那闷油瓶突然把头转向洞穴的深处,“我看,恐怕它们刚才是在逃命。”

“啥?逃命?”胖子一个激灵,“那这洞里头……”

闷油瓶点点头:“我总觉得里面好像有什么东西正在朝我们过来,而且,块头不小。”

\chapter{水影}

“哟,我的小爷爷,你也别吓我,我块头大,最怕这说不出名堂的东西来,你说就是一帮马贼,我大奎也不放在眼里,这东西,是啥都不知道,你看我这腿都软了。”

我心想,在这里待下去也不办法,而且一种很不舒服的预感在我心里一直时有时无的,不知道是这压抑的洞穴给我的心理作用还是什么,于是说:“别管是什么,现在最重要的还是快点出去,现在我们是逆流,要往回走,肯定比来的时候快,我想我们进这个洞才十分钟不到点,出去肯定不是问题。”

“对,对,小三爷说的对。”大奎忙附和,“三爷您就说句话,大不了我们出了以后翻山过去,东西都我来扛,我力气大,耽误这一两天的工夫,也差不了多少啊?咱盗洞打的快一点,不就补回来了吗?”

三爷又看了一眼那闷油瓶,问到:“小哥,你怎么看?”

闷油瓶淡淡道:“现在想出去,恐怕已经来不及了,那两个人既然能放我们进来,就肯定有十分的把握我们出不去。”

“不出去,难道在这里等到老死?”潘子看着他,那闷油瓶看了他一眼,竟然把头转过去闭目养神起来。潘子吃了个闭门羹,只好对三叔说:“我看这样,往前咱们是万万不能,你看阿奎,非吓死不可,我们就往后退,这进来的路不复杂,说不定能出去,要真遇上什么奇门遁甲的,我们再想办法!”

“也只有这个办法了,”三叔点点头,对潘子说,“前后都打一矿灯,你把那几杆猎枪都装起来,我和阿奎用来撑篙,潘子和大侄子盯着后面,小哥你就帮我指路。”我们各自答应,潘子又拿出一只矿灯,对着我们身后一照,那第二只船上的牛被这光一照,叫了一声,潘子骂了声娘:“三爷,得把这牛赶到水里去,不然这篙没办法撑啊。”

因为刚才矿灯是打向前面的,所以我们根本就没注意,早把后面还拉了只船给忘记了,现在看到,不由骇然,看样子这两老贼考虑得真是周详,这洞的高度,那牛根本站不起来,不要说把牛赶到水里去,那一车的装备加上这牛,吃水已经很深了,我们人再上去,不仅篙子撑不动,还有可能会沉。这样子,这后面的这拖船,就像一个塞子一样把我们给堵住了。

这个时候,我隐约又听见了洞的深处传出了怪声,而且,明显比上一次近了很多,那声音,好像无数小鬼的窃窃私语一样,让人极端的不舒服,所有人都静了下来,气氛一时间诡异到了极点。我突然间全部的注意全部被这声音吸引了,几次想收回心神,却马上又被吸引了过去,心叫不妙,这声音有蹊跷!虽然知道,但是却怎么也回不了神,一时间满脑子都是这种声音。就在这个时候,不知道谁狠狠地踢了我一脚,我一个不稳就掉到水里去了。

马上,脑子里的声音全没了,几乎是同时我看见潘子也掉了下来。然后是三叔和大奎,最后那闷油瓶带着一只矿灯也跳了下来,在水里那声音模糊了很多,我们都没什么影响,但是用肉眼在水里看东西非常的模糊,我眯起眼睛也只能看到个大概,闷油瓶向我们指了指水下,然后用灯一照,水并不很深,能够看到水底一层白沙,他扫了一圈,既没什么植物,也没有鱼虾之类的,我实在憋不住气了,探出水去吸了一口,刚把眼睛上的水甩掉,突然发现一张血淋淋的脸从上倒挂下来,两只眼睛死死瞪着我。

我就这样盯着他,他也这样盯着我。

我认出这个人就是给我们撑船的那中年人,一抬头,发现他只剩下上半身,洞顶上一只黑色的大虫子正在啃咬他的肠子,不时还甩一下。我顿时就吓蒙了,这不是只巨大到可怕的尸蹩吗?我的老天,这得吃多少死人才能长这么大啊?!

正在这时候,潘子的头也在另一边冒了出来,可惜他没我走运,还没等他明白怎么一回事情呢,那虫“吱”地叫了一声,把尸体一甩,直接一下就扑到他头上,仰起一对大螯“唰”地卡进了潘子的头皮里。

那潘子也算是个人物,这种情况下见他左手一翻,不知道什么时候军刀已经在手上了,直接把刀往那虫子的螯根下一翘,直接把它一只螯给挖了出来。要是我,挨了大虫子这一下子估计就得去阎王那里报到了。那虫子不知道从哪里发出“吱”的一声惨叫,另一只螯吃不住力气,被潘子顺势一拳推了出去,这一连串都是电光火石一般发生的,那潘子也没看见我,却直接把那虫子按在我脸上了。

我心里大骂,这潘子太不厚道了,平日里说如何如何罩我,现在一有情况,直接把这要命的东西往我脸上扔。你说你还有把军刀,老子就一双手,这下子要完了。那虫子还真不客气,直接就用它锋利的爪子割去我脸上的一块皮,我一咬牙,想把它甩开,没想到它几个爪上都有倒钩,牢牢地钩住我的衣服,有几个都直接钩到我肉里去了,疼得眼泪都出来了。

这时候,那闷油瓶也浮出了头,一看我快顶不住了,赶忙冲过来,一下子把两根手指插进那虫子的背脊,一发力,一扯,一条白花花的通心粉一样的东西被他扯了出来,可怜那虫子刚才还占尽上风,一秒都不到就歇菜了,我把那虫尸往船上一扔,感觉像做了场梦一样。

那大奎对着闷油瓶举起大拇指:“小哥,我大奎服你,这么大一虫子,你愣把他肠子扯出来了。不服不行!”

“去,”潘子头上破了俩血洞,还好口子不大,一边撕牙一边说,“瞧你那文化,这叫中枢神经,人家这一家伙,直接把那虫子搞瘫痪了!”

“你是说这虫子还没死?”大奎半只脚已经爬到船上去了,一听这,又把那脚放回到水里。

闷油瓶一个翻身上了船,把那虫子踢到一边,“还不能杀它,我们得靠它出这个尸洞。”

“你说刚才那声音,是不是这虫子发出来的?”三叔问他,刚才听这虫子叫了几声,好像不像。

闷油瓶把那虫子翻过来,我们看到在它的尾巴上,有一只拳头大的六角铜制密封的风铃,不知道什么时候植进去的,已经铜绿得一塌糊涂了,那风铃的六面,都刻着密密麻麻的咒文。潘子一遍绑上绷带,一边用脚踢了一下,那六角铃铛突然自己动了起来!

发出的声音和刚才听到一样,不过刚才听到的非常空灵,好像幽明里飘来的一样,现在这个听起来就很真切,看样子这个铃铛就是那个声音的来源,但是一定要和空旷的回声配合才有蛊惑人心的作用。这六角铃铛里必然有十分精巧的机关,而且还能经历千年而不腐,估计是金银一类的东西。但是它何以能够自己响起来?

我正在纳闷,这铃铛越发放肆地响起来,好像里面有个关不住的冤魂想逃出这封闭他的神器。可惜这东西太小,反而让我觉得有些可笑。潘子自顾自包扎完伤口,熟练得好像每天都会伤这么一回似的,那铃铛劈里啪啦的响,他听得心烦,就一脚想把它踩住,没想到这青铜的外壳其实已经老化得不成样子了,那铃铛啪一声,竟然被他踩裂了。从里面飚出一股极其难闻的绿水。

三叔气坏了,一拳就想敲潘子的头,一想他脑袋刚被插了两个洞,他再一拳,恐怕就和这铃铛一样了,只好作罢,改打为骂:“你小子脚就不能给我放老实点!这东西少说也是个神器,你就这样一脚给我糟蹋了!”

“三爷,我哪知道这东西这么不结实啊。”潘子还觉得委屈。三叔气得直摇头,他拿军刀拨开青铜的碎片,里面是一个又一个像蜂窝一样的大小和形状都不一样的小铃铛,这些小铃铛都附在一个很精致的空心球上面,那球上面打满了孔洞,如今球已经被踩裂了,里面一只青色大蜈蚣,头部已经被踩扁,那绿水就是从这手指粗的蜈蚣体内被踩出来的。

三叔叔用刀尖把那空心球翻过来,发现这球上有一个管子,直插到与那巨大尸蹩连接的部分,说道:“恐怕这蜈蚣肚子饿的时候,就通过这根管子钻到尸蹩肚子里去吃东西。这样的共生系统,到底是怎么想出来的。”

那半截船工的尸体飘在水上,一沉一沉,三叔叹了口气:“这叫做自作自受,他们肯定是想把我们放单在这尸洞里,等我们死了,再来捞我们的东西。不晓得今天遇上了什么变故,竟然自己死在这大尸蹩手里,真是活该!”

“这叫做无巧不成书,看样子我们运气还不错。”我说道。

潘子摇摇头,说:“那东西的爪子力气恐怕不可能短时间内把一个人撕成两半,要是它有这力气,我的脑浆都已经给它挖出来了,我说这东西肯定不止一只,这一只肯定是在分尸后把那尸体叼过来想自己独食。”

大奎本来已经很放松了,听他这么一说,不由咽了口唾沫。

“别慌,刚才这小哥不是说了嘛,我们得靠这东西出这个洞!我们就把这大尸蹩放在船头上,让它给我们开路,这东西一辈子吃尸体,阴气极重,是那些什么僵尸啊的客星。在尸洞,估计它们就是这里的霸王。有它在我们船上,我们肯定能出去。”三叔说,“来,我们也不退出去了,我倒要看看,前面到底是什么地方,竟然能生出这么大只虫子来。”

听我三叔一说,我也觉得有理,算算在这洞里已经待了不少时间了,这连头都抬不起来的地方太压抑了,我们几个从后面的行李里取出折叠铲,用来当船篙,撑着石壁就向前驶去。

我一边划一边研究这边上洞壁,突然有了个疑问,于是问三叔:“你看这些都是整块的石头,古时候的倒斗先人到底怎么挖出来的啊?就算是现在,没几百人恐怕也挖不出这么深的洞穴。”

三叔说:“你看这洞这么圆,年代十分久远了,估计当年挖这个洞的,肯定是官倒,就是专门倒斗的军队,看样子,我们要找到那地图上所标的墓穴,恐怕没想的那么容易。”

“三爷,你怎么就这么肯定这墓还在呢,你看人家一个军队来,挖了这么长的洞,难保这东西已经给人家搬光了!”大奎说,“我看,说不定我们进去的时候,连块棺材板都没有。”

我三叔闷哼一声,说道:“如果这斗在几千年已经被人盗了,那我们也无话可说,但是你要知道,这洞穴在那地图上是确确实实存在的,这说明这个盗洞在墓主人下葬的时候已经有了,这盗洞的年月,应该在我们要找的古墓之前。而且这一带肯定不止一个墓穴,谁知道这个盗洞是盗哪个的时候挖的。”

“那就是说,”我已经感觉到我三叔这番话有着令人不寒而栗的意味,“我们现在所遇到的一切,包括巨大的尸蹩,六角青铜风铃的年月,他们的主人可能比战国还要早?”

三叔摇摇头:“我更关心的是,为什么我们的这位墓主人,要把自己的墓地设在一个已经被盗墓穴周围,这个,不是犯了风水的大忌吗?”

闷油瓶突然一摆手,让我们不要说话,指了前面,我门看到矿灯光打不到的洞穴深处,有一团绿色的磷光。三叔叹了口气:“积尸地到了!”

\chapter{积尸地}

我们停下船,这应该是这段水洞里最凶险的一段,如果没做好准备,实在不应该贸然就闯进去,三叔看了看表,说:“这尸洞,就是走得进出不来的洞,咱们掏了这么久的沙子,还是第一次闯进这种地方来。我觉得,有可能这洞,真的有古怪!”

潘子低声插了一句:“靠,那还用说。”

三叔瞪了他一眼,接着说:“但是,这只是那老头子的一面之词。这洞到底是不是只有那船工领着能走过去,其他人都过不去,我们已经没办法知道,如果这个洞,”他加重了语气,“真的是个尸洞,那么前面必然会有危险,至于会遇到什么,我们根本没办法知道,也许会鬼打墙,船开到哪里都不知道,也许会有几百个水鬼来掀我们的船板。”

大奎倒吸了口冷气:“不至于吧。”

“总之什么情况都有可能发生,我们这次淘沙倒斗,连墓地都没到就遇到这么多凶险,实在是运气不好,但不管怎么样,淘沙就不怕鬼,怕鬼就不淘沙,既然干了这一行,不遇些古怪事情也没多大意思。”三叔一边招呼潘子从背包里取出双管猎枪,“咱们现在有高科技在手上,比早年的前辈们有利得多,要真有水鬼,也是它们倒霉!”

那大奎吓得浑身发抖,我对三叔说:“你这战前动员怎么说的和鬼故事一样?反而有反效果。”

三叔一拉枪拴,“这家伙这次真把我脸丢光了,没想到这么没用,他妈的来之前吹得大力金刚似的。”然后把枪递给那闷油瓶,对他说,“一共能打两枪,打完了就得换子弹,这些都是散弹,所以距离一远就没什么威力了。瞄准了再开枪。”

我对双管猎枪还是十分熟悉的,小的时候玩打飞碟还得过奖,于是端起来,三叔和大奎一手拿着军刀,一手用折叠铲撑船,潘子、我和闷油瓶端着枪,慢慢向那发着绿光的积尸地划过去。

在矿灯微弱的发散光照射下,我发现这洞竟然越来越大起来,那绿光越来越近,我先听到边上的闷油瓶冒了句洋文出来,然后又听到潘子骂了声娘,然后我就见到让我这辈子都忘不了的景象。

这洞到了绿光这一段,豁然开朗,变成了一个十分巨大的天然岩洞,那水道也变成了岩洞里的一条河水,这水道两边的浅滩上,全是绿幽幽的腐尸,是人还是动物的根本没办法分辨,可以看到最靠近里面的一排一排的骷髅十分整齐,应该是人为堆在这里的,而在外面的就比较凌乱了,特别是河道边上的,什么动作的都有,还有很多没有完全腐烂的尸体,这些尸体上,无一例外地都有一层灰色薄膜一样的东西,就像保鲜膜一样紧紧包在他们身上。不时有几只巨大的尸蹩从尸体里破出来,这些尸蹩都比我们船上这只个头小很多,但是比普通的已经大上四五倍了,一些小尸蹩想来分一杯羹,刚一爬到尸体,那大尸蹩就一敖把小的咬死,吃下去。

“这些尸体大部分是从上游飘下来,然后在这里搁浅的,大家小心,看看四周有什么奇怪的东西!”

“你们看!”大奎眼尖,一指一边的山壁,我们转过头去,竟然看到一只绿幽幽的水晶棺材,镶嵌在这几乎垂直的洞壁的半空。里面似乎有一具穿着白色衣服的女尸,但是这距离实在太远,我们根本看不清楚。

“那边也有!”潘子一指另一边,我们一看,果然,在另一边的山壁同样的位置上,也有一具水晶棺材,但是,这一具,却是空的!

三叔倒吸一口冷气,“这具尸体到哪里去了?”

“难道是个粽子?”大奎问,“三爷,这地方不应该有粽子啊?”

“你们都注意点,如果看到有动的东西,什么都别问先放一枪。”三叔说,一边警惕地看着四周。

这个时候,河道的方向一转,我们绕过了一堆尸骨,大奎哇一声,吓得倒在船里,我们定睛一看,只见一个白色羽衣的女人,正背对着我们,黑色的长发一直披到腰,我看她衣带的装饰,断定是西周时候的。不由咽了口吐沫,说:“尸体在这里呢——”

“停——停——”三叔叔擦了擦脑门上的汗,“大奎,把包里的黑驴蹄子拿过来!这恐怕是千年的大粽子了,拿那只1923年的蹄子,新的怕她不收。”

说了两遍,那大奎都没有动静,我们回头一看,他已经口吐白沫,在那儿抽搐了。要不是环境不允许,我恐怕都要笑出来了。

“潘子,你去拿,妈的,下回我要还带他出来,活该我给粽子吃掉。”三叔接过黑驴蹄子,在手上吐了两口吐沫,说:“瞧瞧吴三爷的手段,大侄子看清楚了,这千年的粽子可是难得见到的,要是我没得手,你就朝我天灵盖开一枪,让你三叔叔死得痛快点!”

我一拉他:“你到底有没有把握?”其实我并不是特别害怕,到底以前并没有碰到过这种事情,总觉得这一身素衣,身材苗条的女人的背影,有一点哀,但是平时恐怖片里,那长头发白衣服的女人转过来都不怎么样。心理作用在这里,心还是跳得很厉害。

这个时候闷油瓶也按了一下三叔的肩膀,说:“黑驴蹄子是对付僵尸的,这家伙恐怕不是僵尸,让我来。”他从包里取出一杆长长的东西,我认得是他从我三叔叔那里买走的那个龙脊背货色,他松开东西上的布,里面果然是一把乌黑的古刀,看样子竟然还是乌金做的。

他把古刀往自己手背上一划,然后站到船头,把自己的血往水里滴去,刚滴了第一下,“哗啦”一声,所有的尸蹩就像见了鬼一样,全部从尸体里爬了出来,发了疯似的想远离我们的船,一下子我们船四周,水里的、尸体里的尸蹩全部都跑得没影子了。

那闷油瓶的手上不一会儿便滴满了血,他把血手往那白衣女子一指,那女子竟然跪了下来。我们看得呆掉了,闷油瓶对三叔说:“快走,千万不要回头看!”

虽然我很想看看那女人长什么样子,但是一想到回头看到的可能是张干尸的脸,还是决定不冒这个险,三叔和潘子两个人拼了命地划,终于看到前面一个逐渐变小的洞口,和我们进来时候的洞差不多,看样子,这个洞是在这个山的中心的,两边挖通之后才有了这条水道,这样就变成一个两边进出口都很窄的毛细孔结构,就算两边水面把洞给没了,这里面还是能保持干燥。

我们渐渐地驶进盗洞,又不得不低下头,在进入盗洞前,我留了心眼,不是说不能往后看吗,我看水里倒影好了,看看她有没有跟在后面,不看还好,一看差点背过气去,在水中的倒影里,一只不知道是什么的东西正趴在我的背上,我正想大叫出来,已经控制不住想回头了,就觉得后脑被一下重击,眼前一黑就什么都不知道了。

\chapter{一百多个人头}

也不知道过了都久。我反反复复做了很多乱七八糟的梦,朦胧中,我好像看见一个白衣女子背对着我,我想看她的脸,跑到她前面去,却还是看到她的背,于是反复地跑,可是怎么跑都只能看到她的后背,正纳闷怎么回事情呢,突然发现,她竟然是两面都是后背,我大叫一声醒了,眼睛一睁开,就望见血红的晚霞和天空!

“醒了?”潘子一张大脸朝我笑。

我眯了眯眼睛适应光线,潘子一指天:“看到没,妈的,我们终于出来了!”

我摸摸后脑勺:“你小子,是不是你揍我!”

“不揍你行不?叫你别回头,你小子差点害死我们。”

我记忆一下子恢复,吓得猛一摸后背,想看看后面那东西还在不在。潘子哈哈大笑:“放心吧,已经走了。”

“那是什么东西?”我心有余悸。

“那小哥说,那东西叫做傀,其实就是那白衣女粽子的魂魄,她不过是借了你的阳气,出那个尸洞而已,不过具体的情况那小哥也没告诉我们,才说了几句就晕过去了。”三叔一边划一边说,“不过看样子那小哥来头不小啊,那千年的粽子就这样给他下跪,不知道什么道行了!”

我坐起来,看闷油瓶和胖奎并排靠在那里,都睡得很香,一笑,这来的时候没觉得怎么样,现在看到这天,就觉得特别舒服,问道:“他到底是什么人啊?”

三叔摇摇头:“这我真的不清楚,我让我在长沙的朋友介绍个有经验的帮手过来,他们就介绍了他,我只知道他姓张,一路上我也试探了不少次,这人不是睡觉就是发呆,我也不知道他什么来历,不过介绍他的那个人,在这道上很有威望,他介绍的人,应该可以放心。”

我一听,越加觉得这个人很神秘,但是既然三叔都这样说了,我再问也没意思了,看了一眼前面,问潘子:“能看到那村了吗?”

“好像就在前面了。”

三叔指了指前面的已经星星点点的灯火:“看样子,那村子没我们想的那么破,好像还有电灯光。”

一想到有村子,我马上就想起热水澡,爆炒的野味,村里大姑娘的大辫子,不由越发激动起来。这个时候,我借着夕阳,看到我们左右山顶上有一队人影子,他们骑着骡子,看样子应该也是进村的,因为这山也不高,我依稀可以辨别出这几个人都不像是本地人。

我们上了渡头,村里一小娃娃看到我们,突然大叫:“有鬼啊!”

我们纳闷,但那小孩子跑得飞快,我们也没办法。那牛就乖乖待在后面那只船上面,一点脾气都没有,真是头好牛,潘子在老家放过牛,就充当了赶牛的角色。上岸的时候,大奎醒了过来,还以为自己刚才是在做梦,先是被我三叔一顿揍,然后潘子又去补了几脚。

那闷油瓶子好像失血过多,一直没醒过来,我把他扶到牛车上,这人也真是的,身子软的像个女人似的,好像没什么骨头一样。我把他安顿好,三叔抓住个过路人问哪里有宾馆,那人像看神经病一样看着我们:“你们以为这是什么地方?我们村一共就三十几户人家,还宾馆,想找地方住,去村里的招待所吧。”

我们只好找到那鬼屋一样的招待所,没想到里面还不错,至少通了电话和电,还是水泥的房子,最可贵的是,有热水,而且铺盖很干净。在这村里,应该是属于五星级标准了。

我们各自洗了澡,那个舒服,一身的尸臭都洗掉了,然后到大厅里吃炒菜,那闷油瓶子总算是醒了过来,精神很不好,我们给他点了盘猪肝让他补补血,也没问他什么。到底他算是救命恩人,有些话,还是得等到人家康复了再说。

我们点了啤酒,明天还要开工,所以也不能喝太多,一边吃一边和那女服务员调笑:“我说大妹子,你这里不错啊,你看都水泥地,外面也是水泥路,怎么你们这些水泥都是那些骡子一担子一担子从山头上背过来的?”

“哪能啊,这要背到什么时候去,我们这里老早是通了公路的。那些解放汽车都能过来,后来前年山体塌方,把那路给埋了,山里还塌出个大鼎,省里来了好多人,一看,说这是战国时候的东西,是国宝,就把那鼎给拉走了,也不管这路了,你说气人不?后来村里说自己修,修什么啊修,没钱,修修停停,一年了,还在修呢。”

“那水路呢,你们这里不有渡头吗?”

“那都是解放前时候的东西了,多少年没拉过船了,现在要还有人让你走水路,肯定是来谋财害命的,你们外地人一定要当心。这水摊子很邪乎,这些年淹死个把人,一具尸体都没捞上来,俺们家老人偷偷说,那是给山神爷爷给吞了。”

我看了一眼三叔,心说你妈的找的什么向导啊,看样子就是找了个贼,三叔也不好意思,面子上下不去,忙喝了口酒。问:“对了,这里外地人多吗?”

“您别看我这招待所小,我可告诉您,只要是外地来的,都住我们这里,这些时间,自从那鼎挖出来后,我们这里外地人就越来越多,还有人在山那头准备造别墅呢。”

三叔呼一声站了来,大叫:“操,不至于吧!”这荒山野岭的造别墅,不是华侨就是盗墓啊。

那大妹子吓了一跳,潘子忙一拉三叔:“三爷,您一把年纪了,别一惊一乍的,”然后对那女的说,“没事情,三爷大概是觉得不可思议。”

我听到三叔低声骂了一句,然后不好意思地一笑,问:“哎,你们有什么名胜古迹没有,有什么地方好玩点的?”

那服务员笑盈盈的,突然低声说道:“几位看来不像是来玩的,怎么,估计是来倒斗的吧?”

看到我们都不说话,她坐到我们边上:“实话说,来这里的外地人,哪个不是来倒斗的,你们要真的是来观光旅游的,这一车的装备岂不是累赘?”

三叔看了看我,给那大姑娘倒了一杯酒:“这么说,您也是行家?”

“咳,我哪行啊,我是听我爷爷他们说的,这些年来这里来了不少倒斗的,摸去不少好东西,但是我爷爷说,那厉害的东西,还在更里面的地方,那是一个神仙墓,里面不要说金银珠宝,那些东西和神仙的宝贝比起来,那就是个屁。”

“哦,”三叔非常有兴趣,“这么说,你爷爷进去过?”

那大姑娘抿嘴一笑:“看你说的,我爷爷也是听他爷爷说的,这个传说都不知道什么时候留下来的,那神仙听说是玉皇大帝派下来的,变成一个大将军,帮当时的皇帝打仗,当时功成圆满就飞升了,他的肉身和他打仗时候用过的宝器葬在一起了。那墓穴,比皇帝的还要好,不然怎么叫神仙啊。”

“既然这么说哦,肯定有很多人去找这个墓了?”三叔紧张地问道,“有人找到过没?”

“哎,你不知道,那地方,现在已经根本进不去了,前年山体塌方的时候,那地方也塌了,您猜那山里头塌出什么来了?”

“什么,总是一个鼎什么的。”胖奎说道。

“什么啊,要真是个鼎,早被人拉走了,我和您说,你可别告诉别人,”那大妹子喝了口啤酒说,“那地方挖出了100多个人头!”

\chapter{山谷}

三叔一皱眉头:“就光是人头?没身子?”

大妹子说:“是啊,你说可怕不?自从那地方塌方之后,就没路可走了,骡子都进不去,你们要想去那儿,只能一脚一脚爬过去,我看就算到了那地方也只能干看看。前面有几批人马都去过那地方,那几个老爷子一看那山塌成这样就直摇头。”

三叔看了一眼闷油瓶,看他懒洋洋的一点反应也没有,就问那服务员:“那山塌了之前,总有人进去过吧?”

“有是有,不过我看他们进去几天,最后也就这样出来了,啥也没带出来,来的时候都开开心心的,出来的时候那衣服都跟要饭的一样,臭得要命,我爷爷说他们可能连斗在那里都没找到。怎么,你们几位也想去试试啊?”

“瞧你说的,来了总要去看看。不然不白来一趟。”三叔呵呵一笑,也没再说什么。

那服务员厨房去给厨房催菜,潘子就说:“看样子我们要去那大斗应该就在那地方没错了,可听这大妹子说的,我们这一车的装备,恐怕很难运到山里去。”

“有装备有有装备的倒法,没装备有没装备的倒法。这战国墓,一般是直土坑,直上直下,没有墓室,不知道这个是不是一样,这我们还得到现场看,这墓有多大,埋的有多深,恐怕和我们以前倒的那些还真不一样。你看那山里塌出的人头,那就是我们老祖宗说的鬼头坑,那里肯定是以前他们人牲的陪葬坑。”三叔拿出地图,一指上面的一个圆圈,“你们看,就是这个地方,这地方离那主墓还远着呢,以前来的那些人,如果按照寻龙点穴的说法,肯定到这里就得停住,这里就是龙头,一般情况,墓肯定在这个下面,但是你们看,再往里走点,这个地方,是个葫芦口,你不往里走根本不知道里面还有洞天,这才是真正的龙头所在,设计这个墓的人,肯定非常了解寻龙点穴,特地在这里设了个套让他们钻。如果我不出所料,这假龙头的下面,必然是个机关重重的虚冢!”三叔看我们听得入神,得意地继续说,“要是没这地图,就是我们老祖宗来了,恐怕也得着了道儿。明天啊,我们就把必须要带的带上,轻装上阵,先去踩一下点,如果实在不行,我们就回来搬东西。”

我们点头称是,再吃了一下子酒就都回房间去了。

然后就是拆装备,这年头当然不用传统的洛阳铲子了,三叔拿出一把考古探铲,这铲子是用钢管一节一节拧起来的,你要多少就上多少根钢管,比那木把子的洛阳铲隐蔽多了,这战国墓一向都是十几米以下,所以省不了,这钢管收拾起来,每个人背十根,每人配一个铲头。潘子有把短头步枪,平时用皮套包得结实,现在也已经拿出来,这枪比那些黑市上买来的双管枪短了很多,可以放在衣服里别人也看不出来,他把这些连同几把子弹一起塞进他的背包里,三叔说,下去用双管枪根本连转身都没办法转。潘子这把短枪实用多了。我准备了只个码相机,一把泥刀,想想也没什么东西要带,本来俺不就是个实习土夫子嘛。

一夜无话,一天的舟车劳顿,我睡得不知道多香,醒来的时候就觉得关节都酥了,我们匆匆吃了早饭,带上点干粮就出发了。那大妹子挺热心的,叫了她村里一个娃把我们带过去,走了两个多小时的山路,那光屁股孩子一指前面:“就那儿!”我一看,果然,很明显前面的山勾勾是被泥石流冲出来的,我们现在就站在一条山脉和另一条山脉之间,这峡谷很长,雨季的时候应该是条河,但是给泥石一冲,又加上这几个月干旱,就剩下中间的一条浅溪。

这两边的山都很陡,根本不能走人,而前面的河道已经被山上塌方下来的石头堵住了。

我拍拍那光屁股娃的头,对他说:“回去玩去,帮我谢谢你姐啊!”

那娃一伸手:“来张50的!”

我一楞,那娃也不说话,就伸手盯着我,我说,什么50的?

三叔哈哈大笑,掏出100块钱来给他,他一把抢过来,蹦蹦跳跳的就跑了。

我这才恍然,也笑了:“现在这山里的小子也这么市侩。”

“人为鸟死——”大奎念叨道,潘子踢了他一脚:“有文化不?为鸟死,你去为鸡巴死啊。”

我们二话不说就开爬,这石头还不算松动,一会儿工夫我们就翻了过去,没那大妹子说的这么恐怖,倒是没看见她说的那些人头,这塌坡后面刚开始是一片峡谷,到后面就慢慢都是树了,到了远处,是一片茂密的森林,也不知道这样的生态是怎么产生的。

这个时候我们看到那塌坡下面的峡谷里,有一个老头子正在打水,我仔细一看,妈的,不就是那领我们进洞的死老头嘛。那老头子猛然看到我们,吓得一下掉溪里去了,然后爬起来就跑。潘子笑骂了一声,叫你跑,掏出他那短枪一枪打在那老头子前脚的沙地里,那老头子吓得跳了起来,又往后跑,潘子连开三枪,每一枪都打在他的脚印上,那老头子也算机灵,一看对方拿他玩呢,知道跑不掉了,一个扑通,就跪倒在地上。

我们跑下坡,那老头子给我们磕头:“大爷爷饶命,我老汉也是实在没办法了,才打几位爷爷的注意,没想到几位爷爷神仙一样的人物,这次真的是有眼不识泰山!”

说着一把鼻涕一把泪的,三叔问他:“怎么,我看你这中气足的,你什么东西没办法啊?”

“实话不瞒您说,我这身子真的有病,你别看我这好像很硬朗,其实我每天都得吃好几副药呢,你看,我这不打水去煎药嘛。”他指了指一边的水筒。

“我来问你,你这老鬼,怎么就在那洞里一下子就不见了?”

“我说出来,几位爷爷就不杀我?”那老鬼看着我们。

“放心,现在是法制社会,”三叔说,“坦白从宽,抗拒从严。”

“是,是,我坦白,”那老头子说,“其实也没什么大不了的事儿,你们别看那洞好像就一根直洞,其实洞顶上有不少窟窿,那些窟窿都打得很隐秘,要不是你存心去找,根本发现不了,我就乘几位不注意的时候,站起来钻那窟窿里去了。等你们船一走,我再出来,那驴蛋蛋听见我的哨子,就会拉一只木盆过来,我就这样出去,事成之后,那船工鲁老二就会把我那份给我,其实我拿的也不多。”他突然想到什么,“对了,鲁老二呢?想必也栽在几位爷手里了吧。”

潘子做了杀头的手势,“已经送他报到了。”

那老头子先是一呆,然后一拍大腿:“死的好,其实我也不想干那事情,那鲁老二说如果我不干就连我一起做了,各位,你看我也是没办法,您就放过我吧。”

“你少来这一套,”三叔说,“你住什么地方,怎么在这里打水?”

“我住在那里头,”老头子指指边上一个山洞,“你看我一个老头子,有没田地,我儿子又死的早,又没房子住,现在也就是等死了,可怜哦。”

“那你对这一带很熟悉喽,正好,要我们放过你也可以,你得带我们去个地方。”三叔一指那森林,老头子顿时就吓得脸色一变,“我的爷爷,敢情你们是来倒斗的啊,那斗你们不能倒啊!那里面有妖怪啊!”

我一听,就知道有戏,这老头子肯定知道什么,三叔就问他:“怎么,你见过?”

“哎呀,前几年,我也带一队人去那里,说是去考古,我一看那就是去倒斗的,但是这帮家伙和其他人不同,我以前见到的那些小毛贼都是看墓就倒,那一批人,不瞒你们说,那气度,一看就不是一般人物,他们边上这些墓连看都不看,就直说要进这山沟沟里面,那时候我们村里就我一个人去过那地方,那些人阔气得很,一下子就给我十张大票子,我看到这钱就不争气了,带他们进了这林子,一直走,走到我以前到过的那地方,他们还要往前走,我就不肯咧,我说你十张大票子也不能买我的命啊,他们就说再给我十张,我说再给我一百张我也不干,他们那头头就翻脸咧,拿枪顶着我的头,没办法,只好再带他们往里头走。”

他挠了挠头,继续说:“后来他们就说到地方了,这些人乐得啊,然后就在那里捣鼓什么东西了,说什么就在这下面,那天晚上我就喝多了,我们就找了个地方扎帐篷,我睡下去就一点知觉都没了,可等我醒来一看,你猜怎么地,这些人全不见了,东西都还在,火还没熄呢。我就害怕啊,就到处叫,可是叫了半天也没有人理我,我就觉得出事情了,心想反正他们也不在,我就溜吧,于是撒腿就跑。”

那老头子好像回忆起看到什么恐怖的景象一样,眯起眼睛,说:“才跑了没几步,我就听到有人叫我,我头一回,看见一个他们队里的女的在朝我招手,我正想骂呢,怎么一大早就跑得一个人都没了,突然我就看见她身后有一棵大树,张牙舞爪的,往树上一看,还了得,我看见这树上密密麻麻的吊满了死人,眼珠子都爆了出来,我吓得尿都出来了,跑了一天一夜才跑回村里。您说,这肯定是个树妖啊,要不是老汉我从小吃实心肉长大的,我肯定也被这妖怪勾了魂魄啊。”

三叔叹了口:“你果然也是个吃实心肉的!”然后挥了挥手。潘子会意的把这老家伙绑起来,有他带路,我们能省很多事情呢。

这老头子一百个不愿意,也没有办法,按他的说法,到他说的那个地方要一天时间。大奎在前面开路,我们加快了脚程,边走边看地图,希望凭着地图和那老头子的记忆,能在天黑前赶到那里,我们走了有半天时间,一开始还能说话,后来就觉得怎么满眼的绿色绿的眼睛发花,人不停地打起哈欠,直想睡觉。突然,那老头子停住不走了。

潘子骂道:“你又玩什么花样?”

老头子看着一边的树丛,声音都发抖了:“那……是……什么东西?”

我们转过去一看,只见那草丛里一闪一闪的,竟然是一只手机。

\chapter{古墓}

那手机应该是刚丢下不久,我捡起来一看,上面沾着血水,就觉得不妙:“看样子这里不止我们一批人,好像还有人受伤了,这手机肯定不会是从天上掉下来的。”

我打开手机的电话本,看到里面就几个号码,都是国外的电话,其他就什么信息都没有了,三叔说:“不管怎么样,我们不可能去找他们,还是赶路要紧。”我看了看四周,也没有什么线索,只好开路继续走。但是在这荒郊野外看到一只这么现代化的东西,总觉得有点不可思议,就问那老头子,除了我们最近还有人进过这林子吗?

那老头子呵呵一笑:“两个星期前有一拨人,大概十几个,到现在还没出来呢。这地方凶险着呢,几位爷爷,咱现在回头还来得及。”

“不就是个妖怪嘛,”大奎说,“告诉你,我们这位小爷爷,连千年的僵尸都要给他磕头,有他在,什么妖魔鬼怪,都不在话下,对不?”他问闷油瓶,闷油瓶一点反应也没有,好像根本当他是空气一样。大奎碰了个钉子,不由不爽,但也没办法。

我们闷头走到天昏地暗,下午四点不到,终于到达了目的地。

我们看到了十几只几乎还完好的军用帐篷,这种帐篷质量非常好,虽然现在上面积满了腐烂的落叶,但里面还是非常的干燥和干净,帐篷里有不少生活用品,我们随便翻了翻,有很多零散的装备,没有人的尸体,那老头子应该没说谎。

我们甚至找到了一只发电机和几桶汽油,发动机用油布包着,不过大部分的零件都烂得不成样子了,胖奎试着发动一下,结果一点反应都没有,不过汽油还OK。我翻了一下,发现所有的东西上都被撕掉了标签,连帐篷和他们背包上的商标都没有,心说奇怪,看样子这些人不想让别人知道是从哪里来的。

我们在这营地里生了火,简单吃了一顿晚饭。那老头子一边吃还一边警惕地看着四周,生怕妖怪突然冲出来,把他也吊死,那压缩食品的味道实在是不好吃,我几乎就喝了几口水。

闷油瓶一边吃一边看着地图,他指了指地图上一个画了那狐狸怪脸的地方:“我们现在肯定是在这里。”

我们全部都凑过去,他接着说:“这里是祭祀的地方,下面应该是祭祀台,陪葬的祭祀可能就在这下面。”

三叔蹲到地上,抓起一把土,放到鼻子下面闻了闻,摇摇头,又走了几步,又抓了一把,说,“埋的太深了,得下几铲看看。”

我们把螺纹钢管接起来,把铲头接上,三叔用脚在地上踩出几个印子,示意这里就是下铲的位置,大奎先把铲头固定,然后用短柄锤子开始下铲,三叔就把一只手搭在钢管上,感觉下面的情况,一共敲上十三节的时候,三叔突然说:“有了!”

我们把铲子一节一节往上拔,最后一把带出来一拨土,大奎卸下铲头,走到火堆边上给我们看,我和三叔一看,脸同时白了,就连闷油瓶也啊了一声。原来那土就像是在血里浸过一样,正滴着鲜血一样的液体。

三叔拿到鼻子前一闻,皱了皱眉头,我和三叔都看过关于血尸的记载,但具体是怎样一个情况,从我爷爷的笔记里也无法准确地推断出来,不过既然泥里带血,那下面的墓肯定是非同小可。

我看着三叔,想看他怎么决定,他想了想,点上一只烟,说:“不管怎么样,先挖开来再说。”

一边潘子和大奎没有停下手,大奎又下了几铲,然后把铲头都拿给三叔,三叔每个铲头都闻了一下,用泥刀开始在地上把那些铲洞连起来,我看他们忙活着定位,一会儿的工夫,地上就画出了古墓的大概轮廓。

探穴定位是土夫子的基本工,一般来说,上面什么样子,下面的墓肯定就是这个样子的,很少有土夫子会弄错的,但是我看着这个轮廓,就觉得不对劲,大部分的战国墓是没有地宫的,可这个下面明显有,而且还是砖顶,真太不寻常了。

三叔叔用手指丈量,最后把棺材的位置基本确定了下来,说:“下面是砖顶,我铲头打不下去,只能凭经验标个大概的位置,这地宫太古怪了,我不知道那里的砖薄,只能按照宋墓的经验,先从后墙打进去看看。如果不行还要重来,所以手脚要快一点了。”

我三叔他们打了十几年的盗洞,速度极快,三把旋风铲子上下翻飞,一下子就下去了七八米,因为是在这荒郊野外,也没必要做土,我们就直接把泥翻到外面,不一会儿,大奎在下面叫道:“搞定!”

大奎已经把盗洞的下面挖得很大,并清理出一大面砖墙,我们打上矿灯,下到里面,闷油瓶看到大奎在拿手敲砖墙,忙把他按住了:“什么都别碰。”那闷油瓶眼神极其锐利,吓得大奎一跳。

他自己伸出两根手指,放在那墙上面,沿着这砖缝摸起来,摸了很久才停下来,说:“这里面有防盗的夹层,搬的时候,所有的砖头都要往外拿,不能往里面推,更不能砸!”

潘子摸了摸墙,说:“怎么可能连条缝都没有,怎么可能把这些砖头夹出来?”

闷油瓶自顾自,他摸到一块砖,突然一发力,竟然把砖头从墙壁里拉了出来。这土砖是何等的结实,光靠两根手指要把一块砖从墙里拔出来,不知道要多大的力量。这两根手指真的非同小可。

他把砖头小心地放到地上,指了指砖的后面,我们看到那后面有一面暗红色的蜡墙,说:“这墙里全是炼丹时候用的礬酸,如果一打破,这些有机强酸会瞬间浇在我们身上,马上烧得连皮都没有。”

我咽了口唾沫,突然间想到了爷爷看到的那只没皮的怪物,心里非常震惊,难道那不是血尸,而是被浇了礬酸的太爷爷?那爷爷那几枪岂不是打在了太爷爷的身上?

闷油瓶子让胖奎往下面又挖了一个五米的直井,然后从自己的包里拿出一只注射针头和一条塑料管子,他把管子连上针头,然后把另一端放进那深坑里。潘子打起火折子,把那针头烧红,闷油瓶小心翼翼地插进了蜡墙里,马上,红色的礬酸便从管子的那一头流进直井里去。

很快,暗红色的蜡墙就变成了白色,看样子里面的东西已经全部都流光了,闷油瓶点点头,说:“行了!”我们马上开始搬砖。很快,就在墙上搬出了个能让一个人通过的洞,三叔往洞里丢了个火折子,借着火光,观察了一下里面的环境。

我们从墓的北面打穿进来,看见这地上是整块的石板,上面刻满了古文字,这些石板呈类似八卦的排列方式,越外面的越大,在中间的越小,这墓穴的四周是八盏长明灯,当然已经灭了,墓穴中间放着一只四足方鼎,鼎上面的墓顶上刻着日月星辰,而墓室的南边,正对着我们的地方,放着一口石棺,石棺后面是一条走道,似乎是向下的走向,不知道通到什么地方去。

三叔探头进去闻了闻,然后招了招手,我们一个接一个地钻了进去。

三叔看着地上的字,对闷油瓶说:“小哥,你看看这些字,能不能看出这里葬的是什么人?”

闷油瓶摇摇头,也没说什么。

我们打起好几个折子,扔到长明灯里,这整个墓室就亮了起来,我想起爷爷笔记上最后看到的怪物,好像还有爷爷反复提到听到咯咯的怪声,心里就直发毛,这时候潘子竟然爬到那鼎上去了,想看看里面有什么东西。突然,他欢呼了一声:“三爷,这里有宝贝!”

我们都爬了上去,看到那鼎里有一具无头干尸,衣服已经烂光了,那干尸身上还有些玉制的首饰,潘子也不客气,直接就摘下来带到自己手上去了。

“这个应该是人牲完了之后剩下来的人的躯干,他们把头砍掉祭天,然后把身体放到这里祭人,这些应该是战俘,奴隶手上不可能有首饰的。”

潘子一下子跳进鼎里,想看看下面还有什么东西,闷油瓶想要阻止也来不及了,他回头看看那石棺材,幸好没反应,三叔大骂:“你小子,这鼎是人家放祭品用的,你小子想被当祭品啊?”

潘子呵呵一笑:“三爷,我又不是大奎,您别吓唬我,”他从里面摸出一只大玉瓶来,“你瞧,好东西还真不少,我们把这鼎反过来看看还有啥吧?”

“别胡闹,快出来!”三叔说,他看到闷油瓶的脸色已经白了,眼睛死死盯着那石棺,知道可能出事情了。

这个时候,我就听到了“咯咯”的声音。我转头一听,不由一阵发寒,那声音不是从棺材里传出来的,竟然是那闷油瓶发出来的。

\chapter{影子}

我开始还以为他存心想吓唬我,可是看他的表情和他为人,又不像是那种人。那闷油瓶不停地发出“咯咯”的声音,又不见他嘴动,我们四个人看着他,那个寒啊,心说不至于吧,难道闷油瓶竟然是个无间道粽子?

三叔看到看他表情这么恐怖,一把把潘子拉了出来。突然,闷油瓶不出声了。墓室里静得一点声音也没有,不知道过了多久,我有点不耐烦了,刚想问他怎么回事,棺材板突然向上翻了一下,开始剧烈地抖动起来。然后从石棺材里发出来了阴森得让人不寒而栗的声音,那声音和我爷爷笔记里描写的非常相似,真的好像是青蛙叫的声音。

大奎见状,吓得一屁股坐地上了。我也脚一软,几乎就要坐下去了。我三叔到底见过世面,虽然脚开始抖起来,但是竟然没摔倒。

那闷油瓶听到声音后,脸色非常难看,一下子跪倒在地上,朝那棺材重重地嗑了一个头。我们一见,马上学样子,全部跪倒磕头。那闷油瓶抬起头来,又发出一连串的怪声,好像在念什么咒语一样。三叔冷汗都出来了,轻声说:“他该不是在和它说话吧?”

那石棺终于稳定下来不抖动了,闷油瓶又磕了一个头,然后站了起来,对我们说:“我们天亮前必须离开这里。”

三叔擦了擦汗,问:“小哥,敢情您刚才那是在和这个粽子爷爷讨价还价呢?”

闷油瓶做了个不要问的手势:“不要再碰这里的任何东西了,这棺材里的主极厉害,要是把这个放出来,大罗神仙也出不去。”

潘子还不知好歹,笑着问:“我说这位小哥,你刚才说的那门子外语呢?”

闷油瓶也不去理他,指了指棺材后面那通道,说:“轻轻过去,千万别碰到那棺材!”三叔定了定神,说实话,有这么一个人在边上,我们胆子大了很多,于是收拾一下家伙,三叔打头,闷油瓶在最后,我们打开矿灯,直下到棺材后的地道里去。大奎走过那棺材的时候背死死贴着墙壁,尽量保持距离,样子非常好笑,但是我这个时候完全没有笑话他的兴趣了。

这墓道是向下倾斜的,墓道两边都雕着铭文,还有一些石刻,我看了一下,也不懂什么意思。其实我做拓本和古玩生意,对这些还是有一定研究的,我能看懂几个词。

但是我可以这么说,就算我全都看明白这些字,因为根本没标点,要明白里面的意思也非常困难。古人讲话非常简洁,而且非常有技巧,比如说,一个:“然”,我记得一个齐国的国君问他的军师一个问题,那军师点头一笑,说:“然。”那国君就回去琢磨了半天想这个“然”到底是同意还是反对,结果就积劳成疾了,弥留之际就把自己考虑的答案和军师说了,问军师当时是不是这个意思,那军师呵呵一笑:“然。”那皇帝立马就断气了。

三叔走得很小心,每一步都要走很长时间,矿灯的穿透力不是很强,前面黑漆漆的,后面也黑漆漆的,这种感觉和我们在水洞一样,我觉得非常的不舒服。走了大概有半个小时,地道开始向上,我们知道应该已经走完半程了,这个时候,我们看到了一个盗洞,三叔不由一惊,他最怕别人捷足先登了,忙过去查看。

这盗洞肯定是不久前挖的,连土都比较新,我问三叔:“老头子说,两个星期前有帮人进了这个山谷,会不会是那帮人挖的?”

“我看不出来,不过这洞挖得很匆忙,看样子,不像是为了进来而打的洞,倒像是为了出去而打的!恐怕我们真的被人抢了先了。”

“别泄气,三爷,要是他们倒的好,肯定是从原路出去的,看样子肯定出变故了。我看,宝贝怎么也应该在。”潘子安慰道。

三叔点点头,那我们继续走,既然有人替我们趟过雷了,我们也不需要这么婆婆妈妈的了。

我们加快了速度,又走了十五分钟,我们到了一处加粗的回廊,这一段比我们来的那一段宽了一倍多,装饰也考究了很多,看样子到了主墓区了。这个回廊的底部,是一扇巨大的玉门,非常的通透,而今已经大开,想必是有人从里面打开的,那玉门的边上,有两个雕像,是两个饿面鬼,一个手里拿着一只鬼爪,一个手里举着一只印玺,浑身漆黑。

三叔检查了一下玉门,发现上面的机关已经被破坏掉了,我们从门缝里进去,里面空间很大,而且一片漆黑,矿灯的电源已经不足了,照不很透彻。

但是我们已经大概可以看个梗概了,这应该就是主墓了,潘子拿他的矿灯一扫,就叫了一声:“怎么有这么多棺材!”

在没有强光源的情况下,要看清楚这墓里有什么的确十分困难,我眼睛扫了一下,果然墓室的中间摆着很多的石棺,而且一眼就能看出,似乎是按照什么次序排列的,并不是非常正规整齐的排列,墓室的上面是个画满了壁画的大弘顶,四周都是正块的石头板,上面密密麻麻都是字。我把矿灯放到一边的地上,潘子把他手里的那只也放到和我交叉的方向上,照了个大概,我们看到墓室边上还有两个耳室。

三叔和我走到第一个石棺边上,打起火折子,那石棺和我们下盗洞时候看到的那只档次完全不同,这一只上面雕满了铭文,我看了一下,竟然能看懂一部分!

上面的文字,记述这了石棺里主人的生平,原来,这墓主人是鲁国的一个诸侯,这个人,天生就有一只鬼玺,能够向地府借阴兵,所以战无不克,被鲁国公封为鲁殇王,有一天,他突然求见鲁国公,说,自己多年向地府借兵,现在地君有小鬼造反,必须回地府还地君的人情债(当然原句不是这样写的),希望鲁国公能够准他回地府复命。鲁国公当时就准奏了,那鲁殇王磕了个头就坐化了。

鲁国公以为他还会回来,就在这里给他设了这个地宫,把他的尸体保存起来,希望他回来的时候能够继续为他效命,云云,非常啰嗦。里面还详细描述他打的战役,几乎都有他鬼玺一亮,地下就杀出大批阴兵掠走人的魂魄。潘子听了我的解说,感叹:“这么厉害,幸亏他死得早,要不然统一六国的就是鲁国了。”

我大笑:“那可不一定,古代人很会吹的,你鲁殇王会借阴兵,那齐国的谁谁谁还能借天兵呢,我记得还有能飞的将军呢,山海经你总看过吧。”

“不管怎么样,总算知道我们在倒谁的斗了,不过,这里这么多棺材,哪个才是他的?”潘子问。

我又看了其他几个棺材上的铭文,大都差不都,都是相同的内容,我们数了一下,一共有七口,正好是北斗七星,七口棺材上没有任何可以提示的记录。正在我研究其他一些我看不懂的铭文的时候,大奎在一边鬼叫道:“你们看,这个石棺已经被人开过了。”

我走过去一看,果然,棺材板并不是完全和棺材密封的,而且棺材上有很多地方都有很新的撬杆撬过的痕迹。三叔从包里取出我们的撬杆,一点一点,把那棺材板撬开,然后拿灯往里一照,潘子发出一声怪声,看了看我们,一连的迷惑:“怎么里面是个老外?”

我们一看,里面果然是个老外,不仅是个老外,而且还非常新鲜,死了绝对不到一个星期,潘子想伸手进去掏东西,那闷油瓶一把抓住他的肩膀,看样子用的力气极大,疼得潘子一咧嘴巴,“别动,正主在他下面!”

我们仔细一看,果然,那老外下面还有一具尸体,看不清楚是什么样子,三叔掏出黑驴蹄子,说:“应该是个黑毛,先下手为强。”

这个时候,大奎在我身后拉了拉我的衣服,把我拉到一边。

他平时颇爽快,我感觉奇怪,问他怎么了,他指了指对面的墙上我们几个被矿灯投射出来的影子,轻声说:“你看,这个是你的影子,对吧?”

我没好气道:“怎么,现在连影子也怕了?”

他的脸色不是很好,听我这么一说,嘴巴也哆嗦了一下,我心想,不会吧,真的怕到这种程度?他摆摆手,让我别说话,然后又指着那些影子:“这个是我的,这个是潘子的,这个是三爷的,这个是小哥的,你都看到了吧?加上你的一共是五个吧?”

我点点头,突然好像也发现了什么,大奎咽了口唾沫,指了指不和我们在一起的另一个孤零零的影子,几乎要哭出来地问:“那这个影子是谁的啊?”

\chapter{七星棺}

我仔细一看那影子,正赶上那影子一低头,那头在抬起来的时候,变得十分巨大,几乎比他的肩膀还要宽,这种恐惧真是无法用语言来表达出来,我就觉得头皮发麻,不受控制地大叫了一声:“有鬼!”

所有的人转头来看我,我根本没办法停下大叫,一边指着那影子,一边转过头,几乎同时我就看见了那影子的主人,那是一个脑袋巨大的怪物!手里拿着一只奇怪的兵器,在半黑暗中,那畸形的大脑袋,比任何你能想象到的怪物都要可怕得多的多。那闷油瓶拿起他的矿灯一照,我们看清楚了这怪物的真面目,它就像……就像一个人把一大瓦罐套在头上面……靠,你爷爷的。

我的极度恐惧马上变成极度愤怒,原来那果然是一个人,头上套着个大瓦罐,手里拿着一只手电筒,还摆了一个埃及人的poss,瓦罐上还有两个窟窿,两只贼眼透过这洞望向外面,十分可恶。

场面一时间非常尴尬,我们也搞不清这人是敌是友,同时也是被这家伙吓蒙掉了,脑子还没反应过来,最后还是潘子骂了一句:“X你妈的,一枪毙了你!”说完就去掏枪,那家伙一看把我们惹毛了,叫了一声:“我的妈呀!”也闪得极快,直接就往我们来时候的那过道里跑了过去,潘子老实不客气,举枪喀嚓上膛,然后就是一枪,把那人头上的瓦罐打碎了,就剩下个圈套在他脖子。那人边跑边大骂:“你他妈的找死,看你爷爷我回来怎么收拾你。”说着脚下像抹了油一样,一下子就不见了。

闷油瓶一看,说了一句不好,“不能让他到我们盗洞那边去,他要是碰到那个棺材就完蛋了!”说完,从他包里里刷地抽出那把黑金古刀,也不提一个矿灯,就这么几步就追到黑暗里去了。

潘子想追去帮忙,三叔一把拉住,说:“你过去能帮个屁忙,快去看看那两个耳室,看他是从哪里出来的。”

我忙走到右边的耳室里,看见一个盗洞从石壁里直接挖了下来,角里还有一只蜡烛,那蜡烛燃在那里,正发着幽幽的绿光,我哦了一声,原来那家伙是个摸金的,我看见地上还有个包,看样子也是他丢在这里的,打开一看,里面是一些工具,几个电池,还有一张这个古墓的草图,虽然非常的潦草,但是我一眼就能看出来,里面的几个方块是代表这七个棺材,这草图边上,写了很多的字,都是不同的笔记,看样子应该是几个人在这里讨论的时候写上去的,在这个草图边上写了一个很大的问号,然后写了几个字——七星疑棺。

我不由一紧,这七星疑棺我好像在哪里看到过,一想就想起来,爷爷的笔记本里提到过,这七星疑棺,除了一个是真的之外,其他的里面,不是有机关,就是设了极其诡异的手段,总之如果你开错一个,这疑棺里的机关或是法术就会击发,必然是凶险万分。看那个老外,应该是不明就里,以为每个棺材里都有宝贝,结果着了道了,不知道被什么东西拖进棺材里去了,而他的伙伴,估计是看到同伴遇害,恐慌之下,逃出了这个墓室,然后在那走道里另挖了一个盗洞仓皇逃了出去。

分析到这里,我自己觉得十分的有道理,拿着这地图就想去和我三叔说,等我一走出去,才发现外面只剩下了一只矿灯,这只在尸洞里进过水,现在时明时暗,非常不好用,而我三叔和大奎他们,竟然不见了!我又到了另一个耳室看了一下,也不见他们的人影,于是捡起那矿灯,喊了一嗓子:“三叔!!”

按道理他们不可能丢下我一个人,自己先走掉的,我先是怀疑他们出了什么事情,可是,刚才也没有打斗的声音啊,以潘子他们的身手,无论遇到什么怪物,惨叫的能力还是有的啊!

可是除了回音,根本没人回答我,这黑幽幽的墓室,七口冷冷的棺材,一具陌生的尸体,马上把我逼回到现实里,我突然间想起自己其实不是一个专业的土夫子,我一个人是根本无法待在墓室里。就算没有什么妖怪,但是我的想象已经可以逼死我了!

我又大叫了一嗓子,真希望马上有人能回答我,可还是一片寂静。这个时候,我手里的矿灯突然闪了一下,好像要熄灭的样子,我出了一身冷汗,脑子开始混乱起来。

如果是一直这么安静,那么我有可能还能慢慢地冷静下来,但是非常的不巧,这个时候我突然听到了石头棺板喀哒了一声,不知道是这七个里的哪个发出来的,我就觉得一阵晕眩,心跳到嗓子眼来了,我退到墙边上,突然,什么东西一闪,我转头一看,原来是隔壁耳室里的蜡烛灭了。

我哀叹一声,心说我也没拿你什么东西啊,你怎么就给我吹了灯了,再回头看看那几口石棺,那口已经被打开的石棺里的古尸,竟然已经坐了起来,那老外的尸体也连着被它带了起来,好像两具尸体一起坐了起来一样,好歹没回头看我。

我不敢再看,闭上眼睛,迈着发抖的腿,小心翼翼地贴着墙挪动,然后一窜,猫进了那个耳室里。

我爷爷在笔记上写过他练胆子的心诀,就是看不到就当没发生过,我想也是,不然看着具坐着的千年古尸,我根本没办法思考问题。我把矿灯放到角落里,尽量让光不要照到外面,然后拼命翻那胖子留下来的包,看看里面还有什么东西,摸了半天,又摸出几块压缩饼干,还有另外一些纸,上面也密密麻麻地写了很多东西和图画,看样子重要的家伙他都带在身上呢。因为外面现在一点光线也没有了,一片漆黑,我也不知道那尸体在搞什么,如果它只不停地坐起来,躺下去,锻炼腹肌,我倒也不怕它。就怕它不知道好歹走过来。

这个时候,一阵风从那盗洞里吹进来,我马上灵光一闪,心想对了,这洞肯定是通到外面的,要不然也是通到别的地方去的,不管哪里,总比在这里好,我在那洞边上刻了个记号,让三叔如果回来看到,可以知道我进洞里去了,然后拿起矿灯,收拾了一下那胖子的包背在身上就钻了进去。

我一边爬着,一边回忆我爷爷小时候和我说的那些常识,什么古圆近方,秦岭汉坡,九浅一深,哦不对,呸,他妈的。我摇摇头,发现我脑子里关于这方面的东西其实非常少。我看了看这盗洞,似圆非圆,似方非方,也不知道到底是什么时候挖的,心理琢磨着,刚才头上带瓦罐那小子要是自己掘了这个地道进来,那么他敲墓砖的时候要么就是触动机关,要是高手,那起码也会发出点声音,但他进来的时候我们几乎没有注意到,那肯定这个洞老早就在了,那就是说,这个洞肯定是另一伙人挖的,或者他老早就挖好了。我推断,要不就是被这个小子从别人的盗洞下来,要不就是他打的盗洞和这个洞撞在一起了。

爬了一会儿,果然出现了一个分叉口,看这两个洞手法完全不同,肯定是两拨人挖的,心想无论哪个都是通到外面的,随便找一个就行了,为了让三叔能找到我,我在我选的那个洞上也画了个记号,然后就爬了进去。

这个时候我已经憧憬着一阵清新的空气,一轮明月,最好是我探出洞去,就能看到一个火堆燃着,他们在上面接应的人看到我,把我拉上去,把我让进帐篷里,然后就是吃点干粮,睡个好觉,然后三叔他们找到我,一起回家,倒个屁的斗啊,我真受够了,别人倒一辈子斗就遇到个别白毛黑毛,我第一次倒斗,走到哪里都是粽子,连口气也不让我喘,我容易嘛。想着,最好那在上面接应的还是个女的,然后还能给我按一下肩膀什么的。

想想就干劲十足,于是加快了动作,不久我就看到了火光出现在前面,我大喜,黎明前的黑暗啊,于是四肢齐用,猛探出了头去,真想猛吸一口气,一看!呆了。

真是希望越大,失望越大,我面前又出现了一个墓道,跟我来的时候经过的那个墓道非常相似,看样子这个墓非常的复杂啊!

我不由骂了一声,一边用矿灯照了照四周,一仔细看我就傻了,这里不就是我来的同一条墓道吗?怎么,原来这个盗洞和那边那个是通的,当初我们还以为有人挖了这个洞想逃出去。

我真的一头雾水,实在想不出,挖这个洞的人,到底是什么目的。

\chapter{门}

我想起那吓唬我们的小子的包里有很多纸上画了一些地图一样的简图,也许上面会有线索,这个时候病急乱投医了,往前有七星疑棺,后面是个连闷油瓶都要磕头的怪物,哪边都不能去,这里最安全了,我坐到地上,摊开那些纸,乱翻起来。其中一张我看得出是他们打盗洞前的设计图,下面写了很多设想,特别是关于血尸墓的设计的推测,我看不太懂,写得非常凌乱,就看到几个琉璃顶之类的字。看样子他们为了破血尸墓的机关,花了非常多的心思,不知道最后有没有实施。然后还有一张,上面画了一个张牙舞爪类似于树,又像是一只鬼爪的东西。

我又把那些纸翻过来看,终于让我看到一张有点意义的东西,上面是一个墓穴的鸟览图,我看到湖底墓道,然后又是放置七星疑棺的地方,画得非常清楚,然后我们下来的那个墓室没有画上去,看样子他们还没到过那里,我还看到了我刚才爬过的那个盗洞,那个分叉口也标得很清楚,我看到如果我选择另一个口子,到了一个地方竟然断掉了,边上写了个字:“塌。”

意思已经很明确了,我想通过盗洞回地面的愿望已经破灭了。我再看,这图上最离奇的是,在我现在站的这个地方的左边,没有任何道路可以连通的地方,竟然还画了一个墓室,而连通这个墓道和那墓室之间的,是条虚线,这个墓室好像是在另一个空间一样的感觉。我不由去摸了摸我后面的墙壁,难道这墙后面有个秘道?

我仔细观察起这个墙壁来,回忆了一下爷爷笔记里那些石头暗门构造。一般来说,如果要这个机关能够千年不腐,必须使用石头和水银来击发,那击发装置的触发器必须是一块平板,这墙壁上都是一块一块的铭文雕刻,如果真有暗门,其中必然有一块能够活动,但是这一块又必须位于非常难于被注意到的地方。

按照这样的思路,我伏下身子,去看石壁和地板处的位置,果然,有一块四方的衔接石板非常可疑。我一按,没反应,但是有松动,再一按,还是没反应,于是就有点毛了,站起来一脚,这下子就听到咕噜一声。

我那一刹那以为,按照一般外国片里,那墙会翻转,把我带到隔壁去,要不就是墙像门一样打开,所以我脚下的地板突然一空的时候,我一点防备都没有,整个人就掉了下去。这种设计哪里是叫暗门啊,明明是个陷阱!我暗叫一声不好,可能要歇菜!这下面不知道是什么东西,说不定是几把锉骨钢刀。

这是电光火石一般,我还没想完呢,就一屁股坐在地板上,还没来得及庆幸没摔死,手上抓的矿灯啪一声砸在地上,电池砸了出来,灯灭了,我顿时陷入了一片黑暗之中。

在现在这种情况下,这矿灯就和我的命一样重要,要是没有光线,在这根本不可能有光源的古墓里,根本就是死路一条。我赶紧扑过去,想把那矿灯摸过来,那矿灯的位置我记得很清楚,一下子就摸到了,那电池应该在左边,我随手往左边上一摸,突然摸到了一只冰凉的手。

\chapter{02200059}

我大叫一声,反射般把手抽了回来,在黑暗中摸到自己没法解释的东西是最让人讨厌的,而且摸到那手的一刹那我感觉到这手的主人必然已经死去了,因为那冰凉和浮肿的皮肤,感觉不到一点生气。

我突然想起自己身上还有一些火折子,忙打一只,借着火光,我看到那地方躺着一具尸体,他的肚子上有一个很大的创口,创口上围着很多尸蹩,这些尸蹩每只都有我的手掌大,颜色是青色的,不时还有一些小点的尸蹩从他的嘴巴和眼洞里爬出来。

我感到一阵恶心,这个人看样子已经死了有一个星期左右了,应该又是上一个盗墓队伍的牺牲品,难道他也是因为发现了那个机关,所以才死在这里的?我想到这里,忙借着马上要熄灭的火光找到电池,往矿灯一里一装,竟然又亮了,我松了口气,那老板说这矿灯可以受三米以上的撞击,看样子还真没骗我!

有了灯,我照了一下四周,这个地方什么都没有,非常的简陋,是一个四方的地窖,四周都是不规则的石头累起来的石墙,墙上有很多排气孔一样的洞,黑黝黝的不知道通到什么地方,不时从那些洞里吹来一些凉风。

我随即检查了那尸体,那是一个中年人,四十岁左右,腹部被撕裂了,看样子是致命伤。他身上穿着迷彩服,口袋鼓鼓囊囊的,我从里面掏出了一只钱包,里面有一些钱,还有一张车站寄存的纸条,我又继续摸,在他的皮带扣上,我发现了一个钢印,上面刻了一行数字:02200059。其他竟然没有任何能证明他身份的东西。

我把他的钱包放到自己口袋,打算出去后自己再研究一下。

这里的建筑风格,很像西周时候的古墓,又有点像一条临时的逃生通道,我想不太可能会有人把墓修在别人的墓地上面,可能这里就是造墓的工匠给自己留的后路!

古时候,特别是战国的时候,你要是参加了修贵族墓穴的工程,那就等于死,不是被毒杀就是和尸体活埋在一起,但是劳动人民的智慧是不容忽视的,大多数工匠都会给自己做一个秘密的通道,好让自己逃出生天。我用灯一扫,果然看见一个非常狭小的门在一边的墙上面,但是这个门离地面还是有点高度的,下面有一个木头梯子,已经烂光了,我估计了一下高度,我不可能跳得上去,这个时候我看到有一张脸突然从那通道里探了出来。

我一看,不由大喜,叫到:“潘子!是我!”

那潘子吓了一跳,也看到了我,可是他不但没有露出喜悦的神情,反而好像看到了什么恐怖的东西一样,几乎从那通道里掉下来。

我正奇怪呢,潘子突然掏出枪,枪口直对着我,我一看不好,怎么难道潘子把我当成粽子了,这下子冤死了!我大叫:“是我,潘子!你他妈的干什么?”

那潘子就像跟本没听见一样,一声巨响,那枪声在这地洞里出奇的响,那子弹几乎贴着我的耳朵呼啸了过去,不知道打到我身后的什么上,一泡腥臭的东西溅了我一后脑勺,我猛转过身,就看见好几只青色的大蹩趴在墙上,几只大敖杀气腾腾地仰着。有几只已经爬到我头顶上的天花板上,离我的脑袋只有十几公分。

我刚想后退几步,离这些大虫子远一点,突然,两只墙上的虫子像弹簧一样飞了过来,几乎一下子就到了我面前,就在同时,又是两声巨响,两颗子弹从我的头顶飞过,凌空把这两只虫子打爆,那真的是打爆,我一脸都是虫子爆出的体液。这个时候,我听到潘子叫道:“我快没子弹了,你妈的还傻站在那里干什么,快点跑过来!”

有了潘子这个靠山,我心里踏实多了,转头就跑,潘子又放了一枪,估计又打爆了一只,我这个时候已经到了墙根了,潘子把手伸下来,我一跳正抓住他的手,还好这石壁非常粗糙,我的脚有地方着力,潘子只一拉我就上去了,还没站稳,潘子那把短枪从我裤裆下面伸出去,又是一枪,那弹壳直接跳出来打到我的裆部,我惨叫一声,几乎晕过去,大骂道:“你爷爷的,想阉了我啊!”

潘子骂道:“妈的,鸡巴和命当然是命重要啦!”

我突然发现矿灯不在我手上了,我回头一看,发现掉在下面,那光源的四周爬满了大大小小的尸蹩,青幽幽的一大片,不知道是从哪里爬出来的,我问潘子:“你还有多少子弹?”

他摸了摸口袋,就掏出一颗来,不由苦笑:“还有一颗光荣弹。”话音未落,一只尸蹩已经跳上石道,对着我们发出“吱,吱”的声音。

潘子到底是当过兵的人,这应变的本领是不在话下,直接变枪为锤,拿着枪管,把那木头枪托当锤头,一下就把那虫子敲扁,踢了下去,但是这根本不是长久之计,更多的虫子爬了上来,我们连踢带敲,还是有几只爬到我们身上,那带倒钩的爪子一下就带去一快皮肉。

我对潘子说:“我们跑吧,这么多根本没办法挡。”潘子问,跑哪里去?我一指后面,说,“这后面肯定是个出口呢,你看这个坑道,绝对是古时候的修墓工匠逃命用的,只要沿着这个跑,肯定就能出去。”

潘子大骂:“屁,我说你们这些书呆子就是以为书上说的都对,我告诉你,这道我都走遍了,根本是个迷宫,我好不容易走到这个地方算有点起色,要是再往后退,不知道要转悠到什么时候!”

我一惊,心说难道我猜错了,但是现在这个情况,也没办法再去细想,眼看虫子越来越多,我大叫道:“那总比在这里喂虫子强!”

这个时候,突然又是咕噜一声,又从上面的暗门掉下一个人来,正压到那些虫子身上,这突如其来的撞击,吓得那些虫子退了开去,那人骂骂咧咧地站起来:“我的屁股耶,妈的,这是什么门,怎么还往下开的。”他拿手电一照四周,大叫,“靠!什么玩意!怎么这么多虫子!!”

我们一看,真是冤家路窄,这不是刚才在主墓吓唬我们的那个摸金贼。

那些尸蹩已经又围了过来,非常迅速,这人也算厉害,把那手电当狼头用,一敲一只,但是根本不顶用,马上他背上就爬满了虫子,他杀猪似的叫起来,手伸到后面想把那些虫子扯下来。这个时候,潘子突然一把掏出了他怀里的全部火折子,全点上,然后一个纵身就跳了下去,我连拦的时间都没有。

他就地一个打滚,就翻到了那小子的边上,那尸蹩怕火,一只只全跳了开去,可是火折子根本不是长久的点火工具,而且刚才一连串动作,那火就非常小了,潘子大叫:“你这里还有没有!”我一摸我怀里,竟然还有几个剩下的,把心一横,心想,妈的,豁出去了,也学潘子那样一个纵身,跳了下去,可惜身手不济,直接一个狗吃屎。手里的火折子就脱手了,一下子就掉到尸蹩堆里去了。潘子大骂:“我的爷爷,你这不是要我的命嘛!”

我忙爬起来,跑到他们边上,那些尸蹩忌讳着火,一时间也不敢扑上来,但是随着那火光越来越暗,它们的包围圈也越来越小起来,我不由咽了口唾沫,心里想:“看来要歇菜了。”

\chapter{闷油瓶}

那小子咳了一声:“同志们,我连累你们了,看样子我们要去见马克思了,我胖子真的什么也没怕过,可也真没想到会这么死。”

他穿着一套黑色老鼠衣,所以在黑暗中看不出他的体形,我仔细一看,果然是个白白胖胖的人。真不到这么肥的人也能做摸金贼。

潘子大骂,“死胖子,你他妈的到底哪里冒出来的,我他妈的真想抽死你!”

我看着火折子已经快不行了,几乎要哭出来了,说道:“你们快想想办法,不然不管谁抽谁都是虫子占便宜!”

潘子看了看四周,把短枪递给那胖子,然后把火折子递给我,说:“本来我们把衣服烧了还能撑点时间,可是这火折子火太小了,可能还没点着我们就已经挂了,我数到三,我来吸引这些虫子,你们就拼命跑到墙根那里,做个人梯爬上去,时间肯定够,我动作快,等你们上去了,我再跑过来,时间一刻都不能耽误!”

还没等我拒绝,那潘子猛的一跳,就扑进那尸蹩堆里。马上,那尸蹩潮水一样涌了上去,我们面前果然有了条路。我大叫一声想去救他,那胖子一把拉住我,说:“上去!”

他硬拉着我连跑几步,一托,我借势就爬了上去,然后伸手把他也拉了上来。

我一看下面,那潘子身上满是尸蹩,疼得在地上打滚,我几乎要哭出来了,那胖子大叫:“快爬起来,就几步路!快!”可是潘子已经不可能爬起来了,他的嘴巴里都已经开始有尸蹩钻进去,几次想站起来,都被扑到地上,我真的没想到这些虫子攻击性这么强,潘子蜷起身子,看着我们在上面大叫,他苦难地摇了摇头。

最后他的脸都被尸蹩盖满了,我看到他伸出了手,做了一个枪的手势,那手上已经全是伤口,我知道他是要我们把他打死。

那胖子不忍看下去,一咬牙,大叫了一声:“兄弟,得罪了!”

就在这个时候,突然那顶上又是一声机关响,又一个人从上面跳了下来,注意,这个人是跳下来的,不是摔下来的,所以他落地的时候很稳,但是落地的分量非常重,他一躬身缓冲,单手撑地,呼了口气,那些尸蹩先是一愣,突然间就像疯了一样到处乱撞起来,拼了命的想远离这个人,原本像潮水一样涌过来的这些大虫子,这个时候同样像潮水一样退了下去,消失在墙壁上的几处沟穴深处。

我仔细一看,不由大喜,这人不就是闷油瓶吗?那胖子也惊叫了一声:“天哪,这家伙竟然没死!”然而我定睛一看,又觉得不妙,只见他上身的衣服已经悉数破光了,浑身上下都是血,看样子受了比较严重的伤。闷油瓶瞥见地上已经奄奄一息的潘子,忙上去一把把他背了起来,我们一看有救了,赶紧伸手下去,一人拉住潘子,一人拉住闷油瓶,把他们拉了上来。

这真是沧海变桑田,绝境逢生,刚才还是十死无生的境地,现在就突然形式逆转。我们想检查潘子的伤势,然而闷油瓶一摆手,说:“快走,它追过来了。”

虽然我还没有领会他话的意思,但是那胖子已经跳了起来,看样子非常的感同身受,他一把背起潘子。我捡起潘子的矿灯在前面开路,四个人就直接往石道的深处跑去。

不知道跑了多久,我已经分不清到底转了几个弯,闷油瓶拉住胖子,说:“行了,这里的石道设计有些古怪,它短时间应该追不过来。”我们停下来,才发现自己已经汗流浃背,我忙问他们说的那个是什么东西,闷油瓶子叹了口气,也不回答我,直接把潘子平放在地上,我一想对,现在最重要的是看看潘子的伤势如何。

潘子这次真的是伤得非常严重,几乎浑身都是口子,如果用绷带把他包起来,就算有足够的绷带,他也变成个木乃伊了。我看了看,幸运的是,大部分的伤口都不深,但是他脖子和腹部有几处几乎可以致命,看样子这些虫子非常善于攻击人柔软的地方,我想起先前让我摸到手的那尸体,也是腹部被咬得最厉害。

闷油瓶用手按了按他的腹腔,抽出了他腰间的黑金古刀,说:“帮我按住他。”

我大惊,有一股不祥的预感,忙问,“你要干什么?”

他盯着潘子的肚子,就像一个屠夫在看他的牺牲品,他用他那两只奇长的手指在他伤口附近划动,一边对我说:“他肚子里钻进去了一只。”

“不会吧。”我怀疑地看着他,然后看了看那胖子,那胖子已经按住了潘子的脚:“从你们的表现来看,我相信他多一点。”

我只好按住潘子的手,闷油瓶一刀挑起他肚子上的口子,然后用他手指以闪电般的速度插进他的伤口,一探,一钩,夹出一只青色的尸蹩,这几个动作速度已经是非常的快了,但潘子还是痛得整个人弓了起来,他力气极大,我几乎按不住他。

“这只窒息死在他肚子里。”闷油瓶把虫尸一扔,“伤口已经太深,如果不消毒,可能会感染,非常麻烦。”

胖子从枪里取出那颗光荣弹,说:“要不我们学学美国人民的先进经验,把这颗光荣弹用到真正需要它的地方,我们把子弹头拧下来,用火药烧他的伤口?”

潘子一把抓住胖子的脚,痛得咬着牙骂道:“我又不是中枪伤!你他妈想……想我烧断我的肠子啊?”他从他裤子口袋里取出一捆绷带,上面还有血迹,看样子是他头上的伤口拆下来的,说,“幸亏没仍掉,先给我绑上,绑紧点,这点伤不算什么!”

胖子说:“这年头不时兴个人英雄主义了,同志,你肠子我都看见了,你就别死撑了。”说完就要动手,我和闷油瓶忙拦住他,我说:“别乱来,子弹烧到他的内脏就完了。还是先包起来。”

胖子一想也对,我们手忙脚乱地帮潘子包好伤口,然后又撕了我衣服上的几快布,在外面又裹了一层,潘子疼得几乎要晕厥过去了,我看他靠在墙上喘气,不由非常感动,要不是我把那个火折子弄掉了,他也许就不至于弄成这样了。

这个时候,我想起一件事情,问胖子:“对了,你他妈的到底是谁啊?”

那胖子刚想说话,闷油瓶做了个不要发出声音的手势,我马上就听到了一声让人毛骨悚然的咯咯声,从走道的一边传了过来。

\chapter{屁}

胖子举起那只有一颗光荣弹的短枪,示意闷油瓶,意思好像是:要不,咱就和它拼了?闷油瓶一摆手,不同意,然后让我们学他的样子,捂住鼻子,他自己一手捂住潘子的鼻子,一手关掉矿灯。

马上,我们陷入了绝对黑暗之中,四周除了那恐怖的咯咯声,就是我自己急促的心跳。这一段时间里,我所有的注意力都放到那声音身上,我听到他越来越近,空气中也出现一股非常奇特的腥臭。

我害怕得几乎要窒息,听着声音越来越清晰,就觉得自己好像是一个在等死的死刑犯一样,突然,在我一个恍惚间,那个声音突然听不见了!我心里一抖,难道它发现我们了?

过了足足有五六分钟,一声极其阴森但是清晰的咯咯声突然出现在我们身边,那么的真切,我的老天,几乎就在我的耳朵边上!我顿时头皮发炸,死命按住自己的嘴不让自己叫出来,冷汗几乎把我的衣服都湿透了。

这几分钟真是极度的煎熬啊,我脑子里一片空白,不知道最后等待我的是死还是活,过了又大概三十秒,那声音终于开始向远处移动了,我心里一叹,我的姥姥,终于有一线生机了。突然,“扑”一声,不知道哪个王八蛋竟然在这个时候放了个屁。

那个声音突然就消失了,与此同时,矿灯光亮,我马上看到了一张巨大的怪脸几乎就贴在我鼻子上,两只没有瞳孔的眼睛直勾勾盯着我的眼睛,我吓得一个趔趄,倒退出去好几步,这个时候,闷油瓶大叫一声:“跑!”胖子看似笨拙,其实非常灵活,一个就地打滚把潘子背起来,撒腿就跑,我跟在他后面,一边大骂:“死胖子,是不是你放的屁!”

胖子脸通红,“靠!你哪只眼睛看见胖爷放屁了!”

我真是懊恼,“我说,你他妈的真是个灾星!”这个时候,突然就听到前面的胖子大叫:“啊……”

我一惊,刚想问他啊什么,突然脚下一空,也啊的大叫了一声,原来刚才没有拿矿灯,又转了几个弯,基本上看不到东西,这个时候脚下的路好像突然间没了,我看不到下面,不知道有多深,就觉得好像正掉向无底的深渊。

不过那种感觉很快就被屁股上的巨痛取代了,正晕眩间,突然一阵闪光,胖子打亮了他的狼眼手电。我一看,这里又是一个石室,非常的简陋,和我们刚才大战尸蹩的那个非常类似,但是因为大小不同,我知道绝对不是同一个。不过胖子这个时候非常紧张,说:“真是冤家路窄,该不会这里又招虫子咬吧?”

我想有闷油瓶在,至少虫子不用怕,回头一看,靠,他竟然不见了!难道和我们跑岔了路了?我急忙回忆了一下,发现原来刚才混乱间,我根本就没注意他是否跟着过来。我转念一想,那怪物不知道是什么东西,怎么能任由我们跑掉,肯定是他在后面帮我们挡了一下,不知道他是不是凶多吉少了。

心里越想越觉得非常不妙,这样下去,迟早是个死啊,那胖子检查了一下四周,然后把潘子放到角落里,自己也坐了下来,揉着屁股说:“对了,我得问你件事,你们是不是也来找鬼玺的?”

我一听莫名奇妙,“难道,真的有这个东西?”

胖子仔细听了听,似乎并没有东西追过来,轻声对我说,“怎么?你们什么都不知道,竟然敢下到这个墓里?你知道不知道,这个鲁殇王,他是干什么的?”

我一听,似乎能从他嘴巴里掏出点什么来,便问:“他不就是个小诸侯王吗,只是听说能借阴兵打仗。”

“屁,”胖子很轻藐地看了我一眼,“我和你说,这个所谓的鲁殇王和那所谓的借阴兵打仗,其实都是一个弥天大谎,这个古墓里暗藏的玄机,如果我不告诉你,你猜破了头也猜不到。”

\chapter{小手}

我这几年做古董和拓本生意积累了不少看人的经验,这一行最考你眼力,既要会看东西,又要会看人,我一看这个胖子,就不是个实在人,想从这种人嘴里打听消息,说好话不如激他,于是装做根本不相信他的样子,说:“说的和什么似的,你要真知道,你能像个没头苍蝇一样在这里乱撞?”

胖子果然就范了,拿电筒照了一下我的脸,说:“你小子还不信?我胖爷来之前可是实实在在做了一个多月的准备工作,你们知道这鲁殇王是干什么的吗?知道借阴兵是怎么回事吗?知道鬼玺有什么用吗?”看我不说话,他得意地一笑,“我告诉你,这鲁殇王,说的好听是个将军,其实说白了和我们一样,就是个倒斗的。”

我忽然想起,三叔也说过类似的话,但我不是非常能理解,他们到底是怎么看出来的,胖子继续说下去:“可是人家比我们厉害,倒斗倒得都封王了,帛书上有记载,那鲁殇王的部队,大多数都是白天休息,夜里行军,而且经常一下子整支部队就消失了,然后又突然在另一个地方出现,而且他们去过的地方,经常是‘坟多破败,问之,则曰阴兵尽出也’,你说我们这些唯物主义的无产阶级革命工作者,怎么可能会相信世界上有阴兵这种东西啊!他们必然是到处挖坟盗墓,如果被人发现坟土被动过,就说是鲁殇王借了这些墓主的魂魄,于是借阴兵一说便四传开来,那个时候的人非常迷信这些,后来就传得神乎其神了。”

我不是非常相信,说:“你们就凭这些信息就做这个结论,未免太武断了吧。”

胖子瞪了我一眼,怪我插嘴,说:“当然不止这么点证据,最直接的证据就是,这七星疑棺,历史上记载,首先就是盗墓贼使用的,因为他们自觉盗墓无数,惶恐死后遭到相同的命运,于是凭借他们的经验,设计了这个虚棺之局。他们认为,无论机关再精巧,也拦不住盗墓贼,唯一的办法,就是让他们犹豫不决,无法下手!这七个棺材,除了一个真正的主棺之外,其他六个,无论哪个被误开,都是九死一生,里面不是暗弩就是设了邪术。到了宋代以后,这个局才逐渐被一些能人巧士发扬光大,这种设计出自不光彩的职业,普通人家是觉得不吉利的,而且一个墓穴里放七个棺材,花费也太高。”

我看这胖子看上去十分粗枝大叶,没想到竟然有这么渊博的知识,不由觉得一敬,但我看他应该还没说完,于是问:“照你这么说,那有没有办法分辨出哪个是主棺?”胖子拍拍我,大概看出了我的态度变化,非常得意:“看你小同志还挺好学,那我就学孔老二悔人不倦好了,你听好,要分辨这七星疑棺,并不是没有办法!但是,我们行有行规,一般人倒斗遇到七星棺,都会叩几个头自觉退出去,老祖宗不会怪罪。以前兵荒马乱的年月,一些搬山道人衣食无靠,实在没有办法,终于破了规矩,那时候有个高人,就想出一个办法,破了这个局,那就是用两根撬杆,棺材翘起一角,然后在棺底凿穿一个小孔,用一个铁钩探入,看看钩出来的东西是什么,这样一来,就可以判断这棺材里到底是什么。”

我不由感叹,这盗墓者和设计者之间的斗智,真的是可以写一部书了,那胖子突然很神秘地凑过来,对我说:“但是这里的七口石棺,恐怕都是假的,恐怕这个鲁王墓,都是假的。”

他又用狼眼照了照我们刚才掉下来的那个石道口,看看没有什么东西爬过来,才继续道:“本来我是怎么也想不通这一点,但是当我掉到这个石道迷宫里的时候,我突然间发现,这里竟然是一个西周墓。”我大吃了一惊:“难道这里不是那些工匠挖的逃生通道?”

这个时候潘子在角落里骂了一句:“我早和你说了,这里怎么可能是逃生通道,你见过谁把逃生通道挖得像迷宫一样?谁会有这么好的兴致?”我大大的迷惑,心里似乎想到什么又抓不住重点:“怎么可能有人会把自己墓穴修在别人的墓穴上面?这不是想断子绝孙吗?”

胖子摸了摸嘴巴,说:“你也是个倒斗的,自然知道风水这些说法,我们这些倒斗的人是最不屑的,这风水除了指导我们倒斗外,我真看不出还有什么其他用处。这风水是门学问,但是古人的学问,死人的学问,和我们这些社会主义大好青年是不相干的。”他特地拍了拍自己的胸脯,“而且,这把自己葬在别人墓里的,风水也有这么一说,好像是叫……叫……叫什么……藏龙穴,反正就是类似一个名字,这些肤浅的名字我们就不要去管它,反正把自己葬在别人的墓穴里,只要你命理配合,布置得当,也是非常有可能的,所以,那鲁殇王的棺材,必然就藏在这西周墓里,绝错不了!”

潘子听了他这话,扑哧一声笑了出来,说:“怎么,就你这熊样,你也能懂风水?”

那胖子大怒:“什么懂不懂的,如果我不懂……我怎么能知道这么多东西?”潘子哈哈大笑,但是一笑伤口就疼了,不由捂着肚子,说道:“也不知道你哪里听来的这些胡说八道,你要是真懂风水,你带我们走出这个迷宫去?我可以是转了七八个圈都找不着路。”

我听潘子说起来,便想起了一件事情,问道:“对了,当时你们怎么丢下我自己跑掉了,你知道我几乎被吓死!三叔他们呢?”

潘子艰难地直了直身子,说:“我也不是很清楚,那时候那小哥去追这个死胖子,虽然三叔让我不要追过去,但是我心想那小子如果紧张起来,必然是有重要的事情,而且,有件事情我没和你说,我总觉得这小子跟着我们过来,目的不单纯,我不是很相信他,也想去看看,所以我就跟上去了。”他皱起眉头,很迷惑地说,“我跑了几分钟,突然看见前面的墓道里有什么东西,我拿灯一照,那东西就嗖一下不见了,我有点紧张起来,就走到那个地方,这个时候,我看到了,那石头和石头的缝隙里,好像夹着一只五指一样长的人手。”

胖子一惊,嘴巴动了动,好像想说什么,但是他最终没发出声音来。

潘子回忆着那个时候的一切细节,说道:“于是我就凑过去看,你知道我这人就是控制不住自己的好奇心,大便也想尝一把,现在想想还真有点后怕,我真没想到那只像手的东西,竟然突然就冲了出来,一把就卡住我的脖子,那力气大的,几乎要把我卡窒息了,我那个时候也不知道怎么办,幸好身上还有把军刀,我一边手脚乱登,一边去割那手,发现这手的手腕细得吓人,几乎就比那手指粗一点点,也不知道它的力气是哪来的,我一刀下去,就划了一道很长的口子,那手马上就松手了,缩回到墙缝里去了。”潘子摸摸脖子,“我想他妈的,这墙后面肯定有蹊跷,就去查这墙,我左敲敲,右踢踢,突然不知道按了什么东西,妈的整个人就掉下去!”他拍了拍墙,“以后你们也知道了,我掉到和这里一样的一个石头室里,然后发现了石道,幸亏老子身手好,跳了半天,终于跳了上去,要不然还真不知道什么时候才会碰到小三爷。”

“那这么说,你也不知道三叔他们的下落?”我叹了口气,潘子显然也刚刚知道三叔他们失踪了,也露出了非常忧虑的神色。我转向胖子,问他,“死胖子,那你是怎么下来的?你给我说实话,那鬼东西是不是你招惹了出来的?”胖子说道:“哎,你要这么说那我真是比苏三还冤了,我跑到那地方时,那个不知道哪里冒出来的老头子已经把那怪物弄出来了,跟在我后面那小子看到了,叫了声糟糕转头就跑,我一看,如果要我和那怪物拼命,估计也不是没有胜算,但是革命的火种还得保存啊,而且组织上给我的任务我还没完成呢,于是我也转头就跑。跑了一会儿,我看见那小哥在我前面停下来,叫我站在那里,我还没明白怎么回事儿呢,他一脚踢了一下墙壁,我就掉下来了,我还以为他要救我呢,没想到下面这么多虫子,娘的。”说到这里,他看了看四周,好像惶恐又有虫子爬出来咬他一样。

潘子看了我一眼,说:“你看,这小子好像对这个古墓非常的了解,非常的不简单。肯定有问题。”我一直觉得那闷油瓶不错,因为只要有他在,我就觉得很有安全感,但是潘子这么一说,我也觉得,这一路上来,那家伙好像知道的太多了,好像什么他都能料到一样,不由也怀疑起来。在我包里还有胖子那里找来的几块压缩饼干,我想起来也很长时间没有吃东西了,于是拿出来大家都吃了一点,潘子吃的很少,说万一他肠子已经穿了,吃多了也是漏出来,还是留给我们吃,因为不知道什么时候能出去。他这么一说,虽然胖子很想吃也不好意思吃多了。我又把我碰到的事情和他们说了一遍,人也逐渐放松了下来。

我们沉默了一段时间,又聊了点别的,胖子说这么干坐着也不是办法,要不我们还是进那个石道碰碰运气,潘子也这样想,于是我们决定再休息一下,然后出发。

我迷迷糊糊地打了个盹,半睡半醒之间,突然看见胖子在朝我挤眉毛弄眼睛,我本来就觉得这个胖子非常的不靠谱,有点精神分裂的感觉,你说谁能在个古墓还能想出来头上套个瓦罐吓唬人?这种人不是胆子太肥就是脑子太瘦。现在我们这里一个人身负重伤,三个人不知去向,这种环境下他竟然还能有兴致朝我做鬼脸,要是我还有力气,必然冲上去给他一下子。

但是,这个时候我发现就连潘子也在朝我挤眉弄眼起来,我想:吓,神经病也能传染?就见他们两个人不停地拍自己的左肩膀,嘴巴一动一动,好像在说:“手,手!”我看他们头上冷汗都下来了,觉得奇怪,于是看了看自己的手,没什么异样啊,难道是我的肩膀,我很随意地转过头去,突然发现我肩膀正搭着一只绿色的小手。

\chapter{洞}

那只小手,五只手指都一样长,手臂极细,和潘子形容的一模一样,十分的恐怖,胖子一个劲地向我做手势,叫我不要动,我其实并不是非常害怕,如果一个人一下遇到突发事情太多,反而会变得冷静起来,我这个时候反而觉得有种在被恶作剧的感觉。突然间觉得非常厌烦,真想一手抓住那手狠狠地咬一口。

当然理智还是让我待在那里不要动,胖子用潘子的枪,去挑那只手,想把那手挑下我的肩膀,枪刚伸过去,那手就像一条蛇一样,一把就缠上了那枪,直接就往后拉去,胖子哪肯放手,大屁股一抖,和那手拔上河了。

我忙上去帮手,胖子一个人劲就很大,再加上我,竟然也只能和这细细的手臂打个平手,眼看我们快坚持不住了,潘子一扬手,把军刀扔给胖子,胖子骂了一句,刀子从下往上狠命一割,从那手上刮下一块皮来。那手突然放开,狂甩着逃进了黑暗中,那动静,我竟然觉得非常像一条蛇。这一下子我和胖子双双吃不到力,都摔了个四脚朝天。

胖子一个肥猪打挺跳起来,追过去一看,原来那里有一条非常深的沟缝。他使劲往里面挤了挤,虽然里面还挺宽敞,但是入口太小了,他的体形根本爬不进去,他丧气地一挥手,恼怒地用手去掰那些石砖,没想到,这石头墙壁看上去非常结实,竟然这么容易就给他掰了下来,他忙说:“快看,原来这里有个大洞!”

我们凑过去,胖子用狼眼一照,里面果然是别有洞天。这洞黑糊糊的,不知道通到什么地方去,我们真是没有想到,这墙壁的黑暗处,竟然藏着一个非常小的通道,难怪上次那些尸蹩可以神出鬼没。

潘子摸了摸那洞的表面,纳闷地说:“看样子是人工挖出来的,难道是给那些尸蹩活动的通道?”

“你说这些尸蹩就在里面?”胖子本来想钻到那个洞里去看看,一听潘子这么说,不由犹豫起来,潘子轻声说:“不用怕,刚才那小哥给我处理伤口的时候,我把他身上的血抹在自己手上了,你看,”他指了指手上一块血污,“你们用点口水往自己脸上也涂点,肯定管用!”

我不由失笑:“你他妈的也太缺德了,人家至少还救了你的命呢!”

潘子不好意思地笑笑,说:“那时候也不知道为什么,看到他的血滴到地上,总觉得不要浪费。”胖子也听不懂我们在说什么,问:“怎么,那小兄弟的血这么厉害?”

我们两个都点头,把在尸洞里的情形和胖子一说,胖子顿时对潘子手上的那块血非常有兴趣,赞叹说:“那敢情好,以后我去倒斗,也可以威风一下,妈的,谁要是敢吹我的蜡烛,我就让他跪在棺材板上。”说着,好像恨不得把潘子手上那块血剜下来一样。

潘子对我说:“这小洞不知道开这里到底是什么用意,不过既然我们走不出那石道迷宫,我想这里也是个希望。要不我们进去看看?”我看了看这个阴风阵阵的小洞口,只能容纳一个人,觉得毛骨悚然进去有点不妥当,但是如果没有行动,那也只能在这里等死,于是点头表示同意。那胖子把自己的皮带脱下来,绑在自己脚上,对潘子说,“你就拉住这皮带,我在前面开路。”

说完二话不说,一猫腰第一个进了洞,然后潘子拉住那皮带,也进了去,我看他们消失在黑暗中,咽了口唾沫,叫了声上帝保佑,然后心一横,也钻了进去。

胖子在前面爬得极慢,有的地方他几乎就过不去,一定要先运一下气,把屁股缩小了,才能通得过,潘子在后面被拖得也辛苦,而且直接对他的屁股,对胖子说:“你可千万别再放屁了。”

胖子在前面喘着粗气,也没力气回答,我看他这么贫的人也不吭声了,就知道他确实是累得够戗,就这样我们像三只虫子一样,一挪一挪的,也不知道爬了多久,突然胖子轻声叫了一声:“有光!”突然间就加快了速度,潘子一下伤口被拉紧,疼得直叫悠着点!胖子爬的极快,看样子他这样的体形,要在这么个洞爬出这个速度已经是奇迹了,我看到那光也越来越强烈,心想难道真给我们碰到怎么好的运气,这个小洞竟然是通到地面上的?终于,胖子第一个爬出了这个洞,他刚出去,我就听到他吓得大叫了一声:“我操!!!这里到底是什么地方?”

\chapter{大树}

我小心翼翼地爬出这个洞口,外面只有一小块突起的地方可以让我站立,再往外就是悬崖了,往下最起码有十五米的高度,而且风非常大,我只有紧贴着崖壁来观察这个地方。

我真的不知道怎么来形容我看到的地方,在我眼前,是一个巨大的天然岩洞,粗略估计有一个足球场的大小,洞顶上有一道大裂缝,月光从这个裂缝里照进来,正好可以勾勒出整个洞穴的轮廓。我现在的位置,就在是靠西边的洞壁上,上下都没有可以攀爬的东西。我扫视了一下,发现我们周围的洞壁上,也密密麻麻的全是洞,足有成千上万个,那密集的程度,就好像这个洞壁被不同口径的超级机关炮扫过十几遍一样。

而最让人感觉到震撼的是,这个洞穴的中间,有一棵几乎十层楼高、十人环抱也不一定能抱起来的大树。而那棵大树上,还盘绕着无数条电线杆一样粗的藤蔓,这些藤蔓纵横交错,几乎缠绕了所有可以缠绕的东西,它们的分支如柳条一样从树上垂下来,有些挂在半空中,有些已经垂到了地上,甚至还有些藤蔓干脆从洞壁的孔洞里伸了进去,举目可以看到的地方,几乎都有蔓延过来的藤蔓,就连我们这个洞口的边上,也爬着一两根。

如果仔细去看,还可以看到靠里面的树枝上还挂着很多东西,一开始我还以为是果实,但是看着这些东西的轮廓又似乎不是,这些东西藏在浓密的藤蔓后面,不时还给风吹得抖动几下,十分的诡异。

而这个天然洞穴的底部,有一条石头的围廊,从一个祭祀台一样的小型建筑开始,一直通到树冠下面,我依稀可以看到,那围廊的终点,是一处有十几级台阶的石台,上面放置有一张玉床,上面竟然好像还躺着个人!距离实在太远,除了一个轮廓之外,其他什么都看不清楚。我不敢下定论。

胖子非常兴奋,直叫:“妈的,还真给老子找着了,这里肯定就是那个西周墓的主墓室。躺在那玉台上的,必然是鲁殇王的尸身。这鲁殇老儿也真够缺德的,雀占鸠巢,把人家的斗倒掉,自己住进来。今天我胖爷就来替天行道,收拾收拾你这个没职业道德的,让你知道倒斗就是这个下场!”他说得兴起,也没想自己是干什么的,连自己也一道骂进去了。

这个时候潘子突然说道:“你们最好不要轻举妄动,这鲁殇王十分的邪门,我想这里必然还是另有玄机。我看我们还是想办法从上面的裂缝先回到地面上去。”

我抬头看了看上面,不由咋舌,要爬到顶上已经不容易了,还要在顶上倒挂着很长一段距离才能到那裂缝口,我们又不是蜘蛛人,怎么可能做得到?于是转过头去想问胖子的意见,只见他已经半个身子探到悬崖外面去了,根本没把潘子的话放在心里。我见他身手十分敏捷,也就没有去阻止他,他几下子就爬下去两米多,到了另一个洞口上,刚想继续往下爬,那洞里突然伸出了一只手,一把抓住了他的脚。

胖子吓得一个激灵,猛踢那只手想把那手踢掉,就听从那洞里传来一个男人的声音:“别动!你再走一步就死定了。”我一听,竟然是三叔,不由一喜,叫了一声:“三叔,是不是你?”

下面那人惊讶道:“大侄子,你他妈的跑到哪里去了!他娘的担心死我了!你没事情吧?”

我一听果然是三叔,心里松了口气,叫道:“没事,不过潘子受伤了!都是这胖子害的!”说着想探出头去看看,可是下面这个洞就在我现在这块突起的死角里,我只能看到胖子的半条腿。只好作罢。就听那个胖子大叫了一声:“同志,我请你不要抓我的脚好吗?”

三叔大骂:“你这胖子到底是哪里冒出来的,他娘的少给我贫嘴,快下来,脚不要乱踩,千万不要碰到那藤蔓。”

胖子说,哪条,是不是这条?说着还用脚尖去指,三叔大叫:“不要!”话还没落,那原本看上去非常普通的藤蔓突然像蛇一样昂了起来,末段间像花一样卷开,咋一看就像是一只鬼手一样,这个东西昂在那里,似乎在感觉胖子的方位。胖子只要一有动作,它也跟着移动,一左一右的,就像印度人在逗蛇一样。我心里恍然大悟,原来潘子看到的和我看到的那只五指一样长的鬼手,就是这些东西来着。

那胖子,也真不简单,竟然把脚在那里划圆圈,逗那藤蔓,我心说这家伙这么不靠谱,难怪他只能一个人来倒斗,如果他一直跟着我们,肯定有一天得给他害死。正想着,三叔果然就火了,骂道:“我说你这个人有完没完,你知道这是什么东西?快给我下来!”刚说完,胖子就遭殃了,那藤蔓一把缠住了他的脚,然后整个一卷,就几乎把他从崖壁上拽了下去,在石室的时候,我和胖子两个人都拉不过一根藤蔓,这下子,那悬崖上又没有地方可以借力,眼看胖子就不行了,我一急之下,想找块石头,扔下去砸那东西,可这悬崖他妈的光秃秃,一点渣都扣不下来,正胡乱摸着,突然就觉得脚上一紧,我低头一看,糟糕!一只鬼手藤不知道从哪里冒出来,把我的脚也缠住了,我马上想找个地方抓一下,已经来不及了,一股巨大的力量把我扯了出去。我还没反应过来,整个人已经在空中了。

那刹那间的感觉,就好像失重,手脚什么东西都抓不到,然后就重重被甩在悬崖壁上,那一子比自己撞上还惨,根本就是拍过去的!我撞得七荤八素,几乎就要吐血,就觉得那藤蔓又吃上劲道,使劲把我向下扯,我两只手都抓出血来了,也没抓到什么东西,接着就是自由落体,下面就是十五米的悬崖,我眼睛一闭,完蛋了!这下子死定了。

这个时候,突然又有三四根藤蔓被我吸引,从悬崖上卷过来,其中有一根特别粗,一下字就缠在我的腰上,我在空中像个麻花一样被裹了好几圈,然后被那特别粗的鬼手藤一带,后脑狠狠在石壁上刮了一下,脑子嗡一声,一下子就晕乎了,就觉得被那些个藤蔓拖着,一路上不是撞到树枝就是撞到石头,浑身上下没一处幸免的,直被撞得眼冒金星,几乎就失去了知觉。

等我朦胧着发现自己静止不动的时候,突然觉得极度的恶心和头晕,想要睁开眼睛,却发现眼前好像有一层沙一样,我做了几个深呼吸,逐渐缓过神来,眼前也逐渐清晰了起来,这个时候我发现,我被倒挂在那棵巨树的一根枝桠上。我的头下面,就是那放置着一具神秘尸体的石台。我仔细一看,不由大吃一惊,原来那石台上,并不是只躺着一具尸体,在我看到的那具尸体的边上,还躺着一具年轻女尸,那尸体身上披着白纱,双眼紧闭,面容安详,看上去竟然有几分的俊俏,而且身上一点也没有腐败的迹象,如果不仔细看,还觉得她是在睡觉一样。而躺在一边的那具男尸,带着一只狐狸脸的青铜面具,浑身上下披着紧身的盔甲,双手放在胸前,手中拿着一只紫金的盒子。

我扫视了这具盔甲尸好几遍,总觉得哪里有个地方让我觉得不舒服,仔细一看,才发现透过青铜面具的眼洞看,里面的尸体的眼睛竟然是睁开的,那两只青色的眼珠子正冷冷地盯着我。

\chapter{女尸}

那眼神真的让人寒毛直竖,我也直勾勾盯着他,一时间不知道怎么反应好,我现在像是腊肠一样被挂在这里,要跑也没有办法,只能一边祈祷,一边尽量想办法挣脱。不过挂了十五分钟,那盔甲尸也没有什么动作,连眼珠子也没有动一下,我不由怀疑是不是我的错觉。但是那诡异的眼神就这样盯着你,就算是神仙也会觉得不舒服。我不去看他,心想得快点想个办法下去。老是这样头倒挂着,脑子也快充血充爆了。

我用尽我全身的力气抬头,发现身上那个惨啊,几乎全部都是淤伤,我的脚被一跟藤蔓缠住,再转头一看,不由倒吸了一口冷气,只见只要是我目力能及的地方,挂满了各种各样的尸体,根本看不到头,那绝对不是说几十具几百具尸体可以形成的情景,我估计总有上万的数目,这些尸体随风摇曳,看上去像很多骨头做成的风铃,这种感觉十分的不舒服。

我仔细看了一下,发现里面有人的也有动物的,大部分已经完全干化,还有少数的一些也腐败的非常厉害,空气中不时传来一股恶臭。而大大小小的尸蹩像苍蝇一样密密麻麻的挤在这些尸体上啃食。我不由庆幸,之前特地从潘子那里弄来了一些闷油瓶子的血涂在身上,看样子还真的管用。虽然这样做有点缺德,不过缺德总比缺胳臂少腿好。

我这时候想起胖子和我一样,也被那鬼手藤抓住了腿,不由替他担心,但是往外看又都是藤蔓,什么都看不到。身上摸来摸去,只摸到一只数码相机,又没有什么东西好用,正懊恼着,突然脚上的藤蔓一松,我整个人往下一沉,几乎以为要掉下去了,忙双手向下,护住头部,没想到它只松了一下,又停住了,我睁眼一看,我的脸几乎就贴在那女尸的脸上,再往下一点就要嘴对嘴了,吓得我忙缩起嘴巴,尽量缩起脖子,就在这个时候,我眼睛一瞄,突然看到她边上盔甲尸腰部有一把小佩刀,不由大喜,心说:“这位仙女,我现在形势所逼,问你朋友借把小刀,他总不会介意吧?”想着,我扭动腰部,竭力朝那佩刀伸出手去,荡了有两三下,我突然发力,一下子抓住了刀柄,用力一抽,没想到那刀这么紧,我不单没抽出来,反而把那盔甲尸的腰带整个扯了下来。

我一看,糟了,怎么把人家裤腰带扯了,这样还不和我翻脸?忙用双腿夹住刀鞘,用力一拔,把刀拔了出来,这刀刀口寒光一闪,我就知道是把好刀,心说天助我也,然后使出我全身的力气翻了上去,只一刀就把那藤蔓切断了,我那时候只顾想着切断那藤蔓,也没想过下面是什么东西,等藤蔓一断我掉下去的时候,后悔已经晚了,才几分之一秒的工夫,我已经整个人趴在那具女尸身上了。

说实话,幸好我着地的一刹那收住力气,没有实打实地压下去,不然这尸体肯定连屎都能被我压出来,但是惯性太大,我想和女尸保持距离已经不可能了,我的脸整个就贴到她的脸上去了,只觉得冰凉冰凉的,冷得我汗毛直竖。我当时就呆了,心想,会不会有一条舌头从她嘴巴里伸出来,直接插到我喉咙里去,把我的五脏六肺都吸出来,想到这里还庆幸了一下,幸亏是个女鬼,长的还不错,要是个男鬼就恶心死了。

可呆了有半晌,也不见有舌头伸出来,心说总算运气还不错,碰到了个通情达理的主,就慢慢抬起头,想溜,头才抬了一半,突然一阵香风,那女尸的两条胳臂突然搭到了我的肩膀上,我一愣,整个人都吓得僵硬了。这个时候边上的那具尸体也发出了咯噔一声,我一听不妙,心里直叫:“老兄,现在是你老婆不让我走,不是我轻薄她,你不要搞错啊!”

转头一看,原来是我刚才扯下了他的腰带,相连处的一块甲片掉了下来,不由松了口气,现在唯一可以庆幸的是,搭着我是这具女尸而不是隔壁这个怪物,要不然我肯定已经尿裤子了。

就这样僵持十几秒,看她没进一步的动作,我不由想偷偷地从她胳臂下面把头钻出去。可是刚一动弹,她的手也跟着我的脖子移动,我往前她也往前,我往后她也往后,我心一横,猛一抬脖子,心说,我干脆就挣脱你,然后一个打滚开溜,结果没想到她的手拉得这么紧,我一个抬头,竟然把她拉得坐了起来。而且一震动,那女尸的嘴张了开来,露出了她含在嘴里的一个东西。

\chapter{钥匙}

我低头一看,那应该是一把镶嵌着珠子的铜制钥匙,那颗珠子墨绿墨绿的,应该不一般,我也看不太出来是什么成分,只知道古人有时候把珠子放到人嘴里防腐,若是我把这钥匙拿出来,说不定眼前这具千年美尸,就会瞬间变成一个木乃伊,那种恐怖的事情,我绝对不会冒险去做的。然而,现在这个情况也太尴尬了,我总不能背着这具尸体跑路。

正在犹豫不决,突然听到人的叫喊声由远而近,我一抬头,看到一个人狂叫着连撞了七八根树枝,被一条藤蔓拉到我的头顶上挂了起来,不是别人,正是那死胖子,看样子他也终于支持不住,重蹈了我的覆辙,而且伤的好像比我还厉害。幸好他没撞到头,挂在那里还直骂:“妈的,想不到这鸡巴粗的树杈杈力气还真大!”然后他就看到我了,一看我就一呆,“小同志,在花姑娘的干活?”

我真是又想哭又想笑,也不敢大声说话,一边做了个手势道:“这个是死的!你快帮我想想办法!”胖子啊了一声,在半空中扭动了一下屁股,说:“那也得把我放下来啊!”我把手里那佩刀往上一扔,他一把接住,马上就翻身上去割那藤蔓,一开始我还没有意识到,后来突然想到的时候已经来不及了,我刚想叫胖子等一下,胖子已经怪叫了一声落了下来,正趴在那盔甲尸身上,竟然把那盔甲尸的面具撞掉了,我刚想探头过去看,胖子一把转过身,对我大叫:“千万别看,这是只青眼狐狸!”

可惜他叫得太晚了,我一闪间已经看到那面具下的脸,只一眼,就让我头嗡的一声,吓得皮都乍了起来,结巴道:“这哪里是人啊!!”

那面具下面,是一张白惨惨的脸,如果你仔细去看,还能依稀分辨出人的五官,整颗人头上都没有毛发,没有眉毛和胡子,脸孔非常削尖的,已经有点畸形的程度,他的眼睛几乎只是一条长长的缝,两只青色的眼珠在两条缝里发着寒光,其他的五官几乎都无法分辨了,我可以这么说,如果只乍一看,这张脸非常像一只正在狞笑的人面狐狸,特别是他的两个青色的眼珠子,看上去更加的诡异,说实话,一般的尸体我真都还能撑,可是这一具我真的不敢用正眼去看他,太吓人了。如果在没有任何心理准备下看到,恐怕会把人吓死。胖子也吓得够戗,一个翻身翻下玉台,骇然道:“真想不到!鲁殇王竟然长的这个德行。”

“这真的是鲁殇王吗?”我问,“怎么看上去像……像只狐狸?”

胖子的眼睛在这盔甲尸体上瞄来瞄去,说:“我一个朋友和我说过,这叫青眼狐尸,很久以前,有一个人倒了一个不知道什么朝代的古墓,打开棺椁后发现里面的尸身上竟然躺着只青眼狐狸,狐狸是有妖性的东西,尸体上躺着狐狸,十分的不妙,本来应该把东西原封不动地放回去,可是那个摸金的道行未够,心有不甘,竟然偷偷留了一只玉乌龟下来。若干年后,他金盆洗手回乡娶了老婆,后来他老婆十月怀胎,那稳婆给老婆接生的时候,突然大叫一声晕了过去,那人冲进去一看,原来他老婆生的孩儿,长着一对青色的眼睛。那摸金的一开始并未察觉到是那只狐狸在作祟,只以为孩子得了怪病,四处求医,谁知道那孩子的病不仅没好,反而毛发都逐渐掉光了,脸也长的越来越像狐狸。这个时候那摸金的才发觉梗概,于是长途跋涉,回到了那个古墓里,将那只玉乌龟放了回去,自此以后那孩子的病才不再恶化,但是那狐狸样的怪脸,却怎么也变不回去了。”

他咂咂嘴,又说,“不过这青眼狐尸十分的邪门,听说看一眼,就会给他传染,脸就会慢慢变得和他一样。你刚才看了没有?”

我虽然不是十分相信,但听到会变成这个怪物,不由也打了个寒战,骂道:“别胡说,变不变是以后的事情,你先帮我弄出来再说!”

胖子一想也是,现在这个情况,再唧唧歪歪就真不是个东西了,忙过来帮我掰那个女尸的手,他憋住了力气使了好几次劲,可那手就像铁做的一样,根本纹丝不动。他狠命扯了两下,累得直喘气,看我紧张的眼神,安慰我说:“别担心,你胖爷有的是手段,实在不行我就把她手给砍下来。”

我急忙大叫:“不行,万一这尸体里有尸毒怎么办,万万不可。而且我和人家又没什么仇恨,一上来先断别人一只手,太不厚道了。”

胖子挠了挠头,也没辙了,他对我说:“一般来说尸体死而不僵,肯定是有心愿未了,你替她了了心愿,她自然就会放你走了。你不如想一下,刚才她钩你的时候,有没有什么特别的事情发生?”

我稍微一回忆就想起来了,刚才我起身的时候,她嘴巴突然张开,里面好像有一个东西,看形状好像是一把钥匙,难道就是这个?想到这里,就小心翼翼地把女尸的头扶正,轻声说了句:“得罪了。”然后一压她的两腮,那女尸杏口微张,我马上看到她舌头下面那把镶嵌着碧绿珠子的钥匙。

胖子惊奇地叫道:“靠,这可是个好东西啊。她肯定是想你把那钥匙拿出来,你想她嘴巴这么小,含了把钥匙多难受。”

我紧张道:“万一她一口咬下来怎么办?”

胖子不耐烦了,说道:“你看看你,现在混身上下都是破绽,她咬你哪里不好,非要咬你的手?”

我一想也对,于是心一横,心说大不了少两根手指,深深吸了一口气,叉起两根手指就颤抖着往她嘴里伸去,就在几乎碰到她嘴唇的时候,我突然听到有个声音在我耳朵边说道:“住手。”

\chapter{青眼狐尸}

我一呆,心说,好熟悉啊,这声音不是三叔的吗?他不是还在悬崖上嘛,怎么这个声音好像就在附近,忙转头去找他,却发现四周除了胖子并没有其他人,不由纳闷,突然又听那三叔说道:“你手上有血气,一入尸嘴马上就会起尸,千万不要乱来。”

我四处想找那声音的来源,最后发现那声音竟然来自这玉台的底下,可这玉台颜色浓郁,根本看不到下面是什么,慌忙问道:“三叔,你在这玉台下面?”

三叔说:“我以后再和你解释,你按照我的方法,将那女尸的头低下,用大拇指顶住她的喉咙,然后拍她的后脑一下。记住,一定要顶住她的喉咙,不然那钥匙会被她吞进去!”我答应了一声,照着他的话,一顶那女尸体的喉咙,然后轻轻一拍,一把钥匙就从她嘴里掉了出来。那钥匙刚掉到玉台上,我就觉得肩膀一松,那女尸双臂就垂了下来,尸身躺倒在玉石台上。

我长出一口气,心说终于解放了,就听三叔又在下面说:“大侄子,你身边是不是还有个胖子?”

我抬头看了眼胖子,他已经拿起掉下的钥匙,正在仔细地研究,点头说:“是的。”

三叔突然用杭州话问了一句:“你看看他有没有影子。”

我一听不由一愣,也没领会他是什么意思,只是条件反射的瞟了胖子的脚下一眼,只见他的影子被玉石床的影子遮住了,如果不探出头去,也看不出到底有没有。不由有点疑惑,说:“我现在看不清楚。”

三叔听上去非常紧张,对我说:“你听着,我告诉你一件事情,你不要怕,我刚刚来这个地方的路上,看到了那个胖子的尸体,你千万要小心,你眼前的这个胖子,恐怕不是人。”

我看一眼胖子,见他脸颊红润,那神态和动作怎么看怎么不像一个鬼,不由纳闷:“三叔,你会不会看错了?”

三叔说道:“不会,那肯定是他,我不会看错的,估计也是上一批盗墓者里的一个,他刚才肯定在怂恿你把手伸到那女尸的嘴里吧?那就是在害你!”

我顿时觉得害怕,问:“照你这么说,我眼前的这个胖子,是只鬼?”

三叔说道:“是,无论他说什么,你也不要相信,你现在快找找身边有什么避邪的东西。”

这个时候胖子抬起头看了我一眼,我突然觉得他的眼神非常诡异,好像非常的怨毒一样,不由马上相信了一半,忙东摸西摸,摸到那盔甲尸体的腰带,上面还连了那佩刀的刀鞘,我想古人一搬都会在自己饰带上刻下镇鬼的文字,忙拿起来。

虽然那腰带上的字已经很淡了,但我还是一眼就辨认出了这的确是鲁国的文字,难道这个人真的就是鲁殇王吗?那边上这具女尸又是谁呢?难道是他的夫人?我刹那间想过,眼睛也没有闲着,已经把腰带扫了一遍,这些文字虽然大部分我都不能看懂,但上面有用金粉描的“阴西宝帝”,还十分好认,的确是镇鬼的咒文。我心中一喜。

这个时候,我想了一件事情,问三叔道:“奇怪,这玉床又不通透,你怎么能看到我们?”

三叔说:“我也不知道,我从下面看上来,都看得很清楚,好像是块透明的玻璃一样。我走过来的时候,正看到你要从那女尸嘴里取那钥匙,才叫住你,幸亏你能听见我说话,不然你把手放进去,就糟糕了。”

我愈加纳闷了,总觉得有问题,心说:这玉床又不宽,上面两具尸体并排放得非常紧,而这里的月光又不是特别的明亮,想要在这种光线程度下,透过两具并列的尸体,看得这么清楚,似乎有点不可能。

我又望了一眼胖子,看见他还在研究那个钥匙,突然觉得有点不对劲。

以胖子的性格,就算他听不懂我讲的杭州话,必然也会插嘴,绝对不可能在那里呆呆地看一把钥匙,看这么久的时间。

我翻下玉床,一拍胖子的肩膀,刚想试探他一句,没想到,那普通的一拍,胖子的反应竟然这么大,他突然怒目圆睁,大叫一声:“你小子他妈的原来一直在骗我!”说完举起他手里的佩刀就捅了过来。我大吃一惊,往后连退了好几步,大叫:“你干什么!”

他两只眼睛通红,根本不听我说,冲过来又是一刀,那胖子动作颇犀利,我一看如果不跑肯定得给他刺伤,忙转头就跑下那石阶,胖子大叫一声:“我叫你跑!”拼了命地追过来,那咬牙切齿的样子,好像我杀了他老爸一样。

我顺着那石道拼命地跑,那胖子看上去体形臃肿,却跑得飞快,我一看那石廊又短,再跑一下子就到尽头的那个石祭台了,再后面就是满地的藤蔓,要是踩进去估计又得给挂腊肠,心里着急。心说难道他真的是个恶鬼,想拉我垫背,可是哪有恶鬼拿刀捅人的。

想到这里,前面几乎已经没路了,我一个刹车,然后就把手里的腰带当鞭子抽了过去,那胖子一个闪身,我冲上一口就咬住他的手,心说这世界敢咬鬼的我还是第一个,他痛得大叫,刀掉落到地上,我飞起一脚将那刀踢到石廊外面。

这样一下,我已经露出了破绽,胖子一把我按在地上,说道:“妈的老子掐死你!”就猛地卡住我的脖子。

我情急之下,一把用腰带勒住他的脖子,心说你狠我也不善,妈的和你拼了!

我勒着他,他掐着我。那互掐的关键就是要在自己窒息前把对方掐死,我一看胖子根本没留手,掐得我几乎舌头都吐了出来,忙也使上老劲,手上用上吃奶的力气,没想到,这腰带看上去保存得还可以,结果质量差成这样,一用力气,啪一声,竟然断了。

那腰带是牛皮做的,上面有小鳞片一样的铜甲,那牛皮一断,那些铜甲天女散花一样掉在我脸上,那块刻着“阴西宝帝”的甲片就掉进我张开的嘴巴里,我突然觉得一股苦涩的液体瞬间流进了我的喉咙里,我想起那甲片是尸体上的,恶心得猛然一呛,突然就觉得眼前一阵迷蒙,好像掉到一团黑色雾气里一样。

我十分迷惑,心想难道这么快我已经被胖子掐死了?只觉得嘴巴里的苦味越来越浓,眼前的东西越来越清晰,然后猛然一惊醒,突然发现自己被胖子压在那玉床上,他眼睛发青,死死地掐住我的脖子,而那女尸嘴巴里的钥匙也没有掉出来,双手还是紧紧钩着我肩膀,场面极端混乱!

我这才醒悟,刚才的一切都是幻觉!!

我转头看边上那具青眼狐尸,他面具还在地上,两只细缝里的眼珠,已经转到我们这一边,直直地盯着我们看。

我心说不好,难怪刚才胖子叫我不要看,这青眼狐尸的眼睛竟然这么邪门,那胖子力气这么大,这一下我就算清醒了,也要被他掐死,忙一摸嘴巴,发现嘴巴里那快甲片已经全部都融化了。正心急呢,眼角突然瞟到那狐尸手上的那只紫金盒子,也没想那么多,拼命伸过手去,拿起来朝那胖子的脑袋上就是一下。

那胖子非常的悍,大骂一声,双手又是一紧,我心说你他妈的哪里是想把我掐死,你整个儿就想把我的脖子掐断啊!心一横,竟然有了杀心,这人非常可怕,我杀心一起,手上的劲道就完全不一样了,就听梆的一声,那胖子一翻白眼,整个人被我敲得几乎一震,一下趴到我身上,我脖子一松,猛地咳出一口血来。

这个时候,我突然看到那青眼狐尸的眼睛好像突然间睁大了一样,一股奇怪的力量引得我不由去看他,突然脑子又开始混沌起来,情急之下,也顾不了胖子,一把就把他推到那尸体上,那胖子非常魁梧,正好把尸体压了个结实。这一压,那种奇怪的感觉就马上消失了。

我揉着脖子,老大几个手指头印,几乎都掐变形了,浑身上下疼得要命,这青眼狐尸的眼睛这么厉害,要不是碰巧我吞了他腰带上的那块甲片,我和胖子必然要死一个。我看了一眼刚才被我当做凶器的紫金盒子,突然发现,上面有一只个小小的钥匙孔,不由咦了一声,再看看那女尸的嘴里,心说,难道那把钥匙就是用来开这个盒子的?

\chapter{八重宝函}

那紫金盒子,手感很沉,看上去有点像缩小的八重宝函(放舍利子的八个盒子)里的银棱盝顶,只不过小了很多,那个时候佛教还没传入中国,这里面放的肯定不是舍利子。我摇了摇,没有声音,心说:难道里面装的就是胖子说的那只鬼玺?

钥匙在女尸的嘴里,我定了定神,深吸一口气,双指探入她的舌下,夹住那把钥匙,然后小心翼翼地夹了出来,那钥匙还没出她嘴巴呢,我就看到一条极细的丝线绑在那钥匙柄上,一直通到这女尸体的喉咙里去,我突然意识到有点不妙,这条线的那头好像还绑着什么东西。

爷爷和我说过,商朝的时候,中国的工匠已经可以巧妙地把一些弩机装到人的尸体里面,用金丝击发,只要盗墓贼一取出尸体嘴巴或者肛门里的玉塞或者宝珠,机关马上启动,弩箭破体而出,因为那时候人和尸体的距离往往很近,根本无法避闪,不知道有多少盗墓贼死在这种机关之下。

我按了按女尸体的肚子,果然摸到了几块坚硬的东西,心说:幸亏我手慢,如果是胖子或是潘子,恐怕已经中招了!想到这一切的安排,好像就是专门为了盗墓贼设计的,我不禁感觉到一阵寒意。

那钥匙后面的丝线是金丝,能拉不能折,我用指甲一掐就断了。我拿出钥匙,和那紫金盒子上的钥匙孔对了一下,果然可以对上。但是我不知道这个盒子里是什么蹊跷,说不定还有机关。我想了一下,暂时还是不开为妙。

就在这个时候,我突然发现,钩着我的那具女尸,突然间变得狰狞起来,我大为惊骇,只见她的脸像变质的橘子一样,瞬间瘪了下去,嗓子里发出没办法形容的声音,几秒的工夫,就在我面前,从活生生的一个美人迅速变成一具干尸,我只一抖,她那枯朽的手臂就断了,干枯的身体摔到玉台上,还在不停地收缩。

我吓得够戗,看样子这把钥匙上的宝石真的有防腐的作用,我不敢再胡思乱想,把这些东西全部塞进包里,心说此地不宜久留,然后就去背那胖子。

胖子被我砸得够戗,拉了好几下也没动静,我心说不至于吧,难不成给我打死了。这个时候也管不了这么多了,我先拿住他一只手,大吼一声:“起!”然后腰板一挺,把他过到我的背上。那胖子很重,几乎把我压得吐血。我暗暗摇头,一边走一边问候胖子的祖宗。

所幸那石走廊本来就不长,我很快就走到了中段,一出那个藤蔓缭绕的区域,我就看到了悬崖,三叔和潘子都不在上面,看样子应该回去找出路去了。我走到石廊尽头的祭祀台处,把胖子放到祭祀台上,想好好休息一下,突然看见三叔已经从最靠近地面的那个洞里钻了出来。

他对这些奇门遁甲之类的东西很熟悉,有他在,那个迷宫根本就不算什么,我怕他没看到我,一边招手一边大叫:“三叔,我在这里!!”

三叔看到我,本来想笑,可是一下子脸色就变了,一指我身后,我回头一看,胖子不知道什么时候坐了起来,而那具青眼狐尸,竟然正趴在他的背上,冷冷看着我。

\chapter{棺椁}

我的眼神一下子就被定住了,怎么也移不开,不过不知道是不是吃了腰带上的甲片关系,虽然我头转不过来,但是竟然没有出现幻觉。我的眼前一阵恍惚一阵恍惚的,但是思维却很清醒。

这个时候,我突然听到三叔他们冲过来的声音,心里大叫不好,他们没尝过这狐尸的妖术,不知道厉害,贸贸然过来肯定要出事情。我想大叫提醒他们,可是我的喉咙好像被什么东西卡住了一样,张大了嘴巴却什么话也说不出来,急得我几乎要爆血管了。

突然间我灵光一闪,发现我的手还能稍微动一下,马上两只手都做了个手枪的手势,枪头指着那狐狸尸的头,不停地点,心里直叫:潘子,你这次怎么样也要机灵点,这个动作你还看不懂你真的可以去吃屎了!

才点了几下,后面就一声枪响,青眼狐尸的头在我眼前被整个儿打爆了。我那时候正张着嘴,那尸水几乎爆了我一脸一嘴,我立即就呕吐了出来,这玩意比吃屎还恶心,我几乎把肚子里的东西都吐光了,才回过头,看到远处潘子一手捂着伤口,一手正对我做了OK的手势。我暗骂一声,用袖子把脸上的尸水擦掉。

从三叔那里到这祭祀台有一段距离,一路上都是藤蔓,十分危险,不过三叔很有办法,用石头先把那些藤吸引开,然后再自己过去,不一会儿他们就爬上了这个祭祀台。他很怕我出事,马上过来看我有没有事,一闻到我身上的味道他就一皱眉头,几乎要吐出来,我本来就不太爽,看他这样,扑上去就给他一个拥抱,把他恶心得差掉摔下去。

我见他们都安然无恙,想起一件事情,责问道:“三叔,在主墓里你们怎么丢下我跑掉了,他娘的把我吓死了,那鬼地方我一个人怎么待得下去啊?”

三叔听了,甩手就给了大奎一个头磕:“我他妈的让这个小子不要乱碰东西,他就是不听。”接着他就把他遇到的事情说了一遍,原来他们在那个墓室另一个耳室里,看到了一道墓墙,一般古墓里有墓墙,那后面肯定有个隐蔽的房间,他们自然也没有想到,这个古墓里,任何的暗门都是向下开的,三叔是何等的精明人,一眼就找到了机关,可惜那大奎手快,三叔还没弄清楚呢,那机关已经被他按下去,然后就和我们一样,掉到下一层的西周墓里去了,之后情节似乎非常的曲折离奇,三叔越说越离谱,我看他几乎都说到不着边的地方去了,忙让他打住。

三叔说:“你还真别不信,你看看我这些家伙。”他从他背后拿出一只黑色的盒子,喀嚓一弄,那盒子魔术般的变成了一把枪。我对枪有点研究,而且这枪也很有名气,一看便吓了一跳。

这是把阿雷斯折叠冲锋枪,九毫米口径,打的是手枪子弹,就像一条中华香烟那么大小,才六斤不到,很容易上手,当然因为体积太小,这枪也很不稳定。

三叔说,他们在墓道里,也发现了好几具尸体,这把枪还有一些炸药,都是从那尸体上弄下来,不仅如此,那地方全是弹孔,看样子是打了一场恶战。

我仔细检查这把枪,非常疑惑,看来,前一批进来的盗墓贼,装备非常精良,至少比我们精良得多,不知道是什么来头?这些人进来后都没出去,难道已经全部死在这里了?如果没死,他们现在又在什么地方?

我一边想一边靠到那祭祀台,没想到这貌似非常结实的石台竟然会撑不住我,我还没压上全部的重量,这祭祀台就突然一沉,矮下去半截。我们吓了一大跳,还以为触动了什么陷阱,赶紧蹲下身子。只听到一连串机关启动的声音,从我们脚下开始,一路发出,最后远处石台上传来一声巨响,我们探头一看,只见石台后的那棵巨树身上,竟然已经裂开了一个大口子,在裂口里,出现了一只用铁链固定的巨大青铜棺椁。那些铁链已经和树身合在一起,而且还绕了好几圈在青铜棺材的上面。

那三叔看得呆了,啊哦一声,说:“原来真正的棺椁在这里。”

大奎高兴地大叫:“好家伙,这么大的棺材肯定值老钱吧?这下子总算没白来!”

三叔拍了一下他的头,说:“值钱值钱,你别他娘的老惦记着钱,这东西就算值钱你也搬不走,和你说了多少便了,这叫棺椁,不是棺材!别他娘的老是丢我的脸!”

大奎摸摸头,不敢再说话,我仔细看了几眼,感觉到有点不对劲,对三叔说:“奇怪,别人的棺材都是钉上了就没预备再打开,你看这架势,这个石台的机关好像本来就为了让别人找到这只棺椁的,难道这墓主原本就打算有朝一日让别人开自己的棺?而且你看,这几根铁链子,绑得这么结实,不像是用来固定的,反而好像是不让里面的东西出来才绑上去的。”

三叔仔细一看,果然是这个情况,不由面面相觑,我们一路过来,碰到不可思议的事情数不胜数,难道这里面又是什么怪物?那到底是开好还是不开好呢?

三叔一咬牙,说:“估计这墓里值点钱的宝贝都在里面了,不过去,岂不是白来?他娘的里面有粽子又怎么样?我们现在有枪有炮,实在不行,就抄家伙和它拼了。”

我点点头,三叔又说:“况且我们现在就算原路回去也不太可能,这悬崖上每一个洞,几乎都是通到那石道迷宫里去,要从那里出去,不知道要花多少时间,最好的办法,还是从上面爬出去。”

我们抬头一看,看到了洞顶上的裂缝,月光从那洞顶上照射下来,显得非常凄凉,三叔一指那棵巨树:“你们看,这棵巨树的顶端离洞顶非常近了,而且还有很多的藤蔓从树上衍生到洞顶外面去,这简直是一座天然的梯子,而且那整棵树上这么多枝桠,非常好爬,正好有利于我们出去。”

潘子说:“三爷,你怎么在这里说胡话,那棵可是食人树,爬那棵树不是去找死?”

三叔大笑:“这棵叫九头蛇柏,我早就想到了,你没看到那些个藤蔓怎么样都不敢碰这里的石头吗?这石头叫天心岩,专克九头蛇柏,我们弄点石头灰涂在身上,保准顺顺利利的。”

大奎担心道:“能管用吗?”

三叔瞪了他一眼,我知道他又要开骂,忙说:“行了,我们去试试不就知道了?”

我们二话不说马上行动,大奎背起胖子,三叔扶起潘子,我收拾了一下装备,回头看了一眼岩洞,心想我们现在都平安,不知道那闷油瓶怎么样了,三叔叔看出了我的忧虑,说道:“他的身手,肯定能保护自己,你就放心吧。”

我点点头,凭心而论,我实在没有资格去担心闷油瓶,他的身手不知道在我之上多少,而且似乎拥有奇术,要担心也应该是他担心我。

我端着枪走在前面,他们跟在我后面,慢慢走上那高阶石台,刚才匆匆跑下来,没仔细看,原来这石台都是大块大块的天心岩垒起来的,体积这么大,不知道是怎么运进来的,那台阶上还刻了一些鹿头鹤,这种浮雕很罕见,我不由纳闷,这鲁殇王到底是什么级别的诸侯,怎么墓葬的规格这么离奇。

这个时候我们已经走到了那个树洞前面,这才看清楚,那个洞原来不是自己裂开的,而是被里面的十几根铁链扯开的,那只巨大的青铜棺椁就在面前,最起码有两米五长,我看到上面密密麻麻地刻满了铭文。

战国时期的文字比较复杂,而齐、鲁的文字是当时普遍为学者使用的文字。楚国在兼并了鲁国之后,也大量吸收了鲁国的文化,文字上也与鲁国比较相近。现在我手头上出手的战国时期的拓本,有大部分都是那个时期的东西,所以我对于这些铭文还是能看个大概。

这个时候,不知道为什么所有人都不说话,好像怕吵醒这墓主人一样。三叔拿出撬杆,敲了敲,里面发出沉闷的回音,绝对是装满了东西,三叔知道我好这些东西,轻声问我:“你能不能看懂上面写的什么?”

我摇摇头,说:“具体的我看不懂,不过可以肯定这具棺椁的主人,就是我们要找的鲁殇王,这上面的文字,应该就是他的生平,他似乎不到五十岁就死了,无子无女,而他死的时候的情景,和我以前了解到的一样,是在鲁公面前突然坐化。其他的应该都是一些他的生平。”

我对那个时候鲁国的人文不感兴趣,所以只扫了几眼就不看了。

“那这几个字是什么意思?”大奎问我,我看了一下,在棺材的中间,写着一个“启”,然后下面是一长串子丑寅卯,这几个字特别大一点,显得比较醒目,我知道这几个数字是一个日子,但是春秋战国时期,周室衰微,诸侯各行其是,历法乱得不得了,所以我也不知道这是哪一天。说:“这个应该是标明下棺的日期。不过我也不知道这是什么日子。”

我在研究铭文的时候,三叔在研究怎么开这个棺椁,他摇摇那几根铁链,这些链子每一根都有大拇指粗细,那时候中国刚刚进入铁器时代,这东西应该是属于奢侈品。经过了这么多年,大部分已经老化得不成样子,基本上只能做个摆设的用途。我让他们让开,拉开枪闩,来了几个点射,那铁链就悉数断掉,只剩下几根用来固定位置的留在那里。

三叔让我后退,说:“你也别研究了,把它搞开来再说!”

话音刚落,那个棺椁突然自己抖动了一下,从里面发出一声闷响。我刚开始还以为自己听错了,正想问别人,突然又是一震,这一下子我听得真切,不由全身一凉,心说坏了!他娘的这里面果然有问题!

\chapter{活尸}

我们全都吓得后退了好几步,虽然早就想到这棺材肯定会出一点问题,但是实际碰到,还是不由倒吸了一口凉气。这动静,分明表示里面肯定有位能动的主,棺材里的东西能动,肯定不是好事情。

大奎脸色发白,发抖说:“好像里面有个什么活的东西?三爷,这棺材,我看我们还是别开了。”

三叔仔细看了棺椁的接缝处,摇头道:“不可能,这个棺椁密封得很好,空气根本不能流通,不管里面有什么活物,就算他寿命有三千年,也早被闷死了。况且这只是个棺椁,里面还有好几层棺材呢,我们先撬掉一两层再听个清楚。”

我大概估计了这东西的重量,在我记忆里,最重的青铜椁应该是擂鼓墩曾侯乙墓的那只巨型棺椁,大概有九吨,这一只体形差不多,但是曾侯乙墓的那只是青铜镶嵌木板的,这一只全青铜,恐怕重量远远不止九吨,具体多少,我根本估计不出来。

大奎和三叔用刀先刮掉接缝处的火漆,然后把撬杆卡了进去,喊了一声,往下一压劲,只听嘎嘣一声,那青铜椁板就翘了起来,我忙上去帮忙,把那青铜板往外推,这一块板最起码有八百多斤重,推了老半天才挪出去半个边,我们累得上气不接下气,最后我们几个人同时用肩膀一顶,把板翻到一边,终于露出了里面的棺材。

那是一具精致的镶玉漆棺,上面镶满了玉石,这些玉石排列得十分工整,分菱形和圆形两种方式排列,概括了天圆地方这么个说法,那玉嵌套棺里,是一只彩绘漆木棺,因为外面被玉石贴住了,我也看不出上面画的是什么,潘子看到那棺材,眼睛都快掉下来了,捂着伤口一半脸哭,一半脸笑的:“妈的,这么多玉,这下子横着走都行了!”说着咬着牙就要下手,三叔忙叫:“不行!这是新疆玛纳斯玉,你要把玉拆开来卖,只能卖个十几万,我们这么多人还不够分的,你得把玉嵌套整个拿下来才值钱!”

潘子已经闯过祸,三叔眼睛一瞪,他就不敢造次,挠挠头退到一边去了。

三叔敲了敲那彩绘漆木棺,说:“一般战国诸侯王都是二重椁,三层棺,如果把那树算第一层椁的话,现在我们已经去掉二椁二棺了,那下面那一层,应该是最贵重的。”说完,三叔小心翼翼地用小刀将所有的金线从那漆棺上拨下来,为了不弄坏那玉嵌套棺,他拨很小心,花了半个小时,终于把整套的套棺取了出来。

玉嵌套棺一除去,我看到了那木棺上的彩绘,这些东西比铭文容易懂,我打亮一只矿灯仔细地看,上面画的是几幅叙事性的画,棺材板上的那幅可能是棺材刚刚入殓时候的情景,我看到了一棵巨大的树,中间裂了一个洞,青铜棺椁被很多骷髅抬着,还没有盖上盖子,然后边上有很多人,正恭敬地跪在那里。

三叔小心翼翼地把玉嵌套棺叠好,放到自己背包里,我试背了一下,那东西死沉死沉的,看样子背起来够戗。

有了这个东西鼓舞,大奎一下子就来劲了,二话不说,继续开那里面的彩绘漆木棺,三叔一把把他拉住,骂道:“你他妈的看见鬼就晕,看到钱就不要命,这下面只有一层了,别毛手毛脚的,悠着点。”说着蹲下去,耳朵贴在棺材板上,做了一个让我们不要说话的手势。

我们屏住呼吸,生怕干扰了他,他听了很久,转过身来,脸色惨白地说:“他娘的里面好像有呼吸声。”

我们全部都一愣,要是听见里面有鬼叫,我们兴许还能接受,但是现在里面有东西在喘气,这也太离奇了,大奎吓得结巴了,说:“该不是个活死人吧!”

三叔说:“放……屁!别他妈的在这里给我胡扯,都已经到这个地步了,难道把那棺材板给他盖回去?”说着摸出黑驴蹄子夹到掖窝里,对我做了个手势,我端起枪,大奎轮起手里的撬杆,守在那棺材边上,准备不管什么东西跳出来,先给它来一黑的。

三叔呸呸往手里吐了两口口水,先活动活动膀子给自己壮壮胆,然后就要把撬杆往里面插,就在这个时候,身后有一个声音喊道:“住手!”

我们回头一看,原来是那胖子不知道什么时候醒了,正摸着头,一边对我们摆手:“不行不行,这样开会出事情的。你们他妈的就这点阅历还想来倒他的斗。真他妈的是茅坑里打电筒,找屎(死)。”

三叔哼了一声,“那你说这么开?”

胖子甩甩手让三叔走开,自己把手伸进那漆棺和青铜棺椁的缝隙里,闭上眼睛摸索了很久,突然他手一发力,我们听到啪一声,棺材从中间整齐地裂了开来。那一刹那,我们都似乎听到了一声极端凄惨的叫声,从棺材里传了出来,我吓得手一软,枪差点脱手。

胖子马上跳了回来,双手展开,说道:“退后!”

我不自觉地端起枪,对准棺材,迅速退后了好几步,那漆棺像一朵莲花一样从棺椁中升起,然后左右裂开的棺盖翻了下来,这种巧夺天工的设计真是叹为观止,我们不禁看呆了。

同时,我们看到一个浑身黑色盔甲的人,从棺材里坐了起来,我肩膀一抬,几乎就要开枪了,那胖子一把抓住我的手,说:“别动,他身上穿的是宝贝,别弄坏了!”

我这时候终于看到,那神秘的鲁殇王是什么模样,那是一具罕见的湿尸,全身的皮肤已经白到有透明的感觉,两只眼睛闭着,看样子似乎死的时候非常痛苦,五官几乎都扭曲了,我非常奇怪,他既然有办法可以让那具少女的尸体千年不腐,为什么反而不能保存好自己的尸体。

三叔走到旁边一看,说:“我他妈的还以为又是个粽子,你看,后面有根木头撑着他。难怪他能坐起来。”

我们都走过去,果然,那是一个十分精巧的机关,只要棺材一开,里面的尸体就会被一根木棍撑着坐起来,要是普通的盗墓贼,恐怕会吓死。

这下子我们都松了口气,心说这鲁殇王真是处心积虑,可惜他也应该想到,怕鬼的不倒斗,倒斗的不怕鬼,敢在这晚上开别人棺材的,都是些亡命之徒,这样吓唬人的伎俩,也未免太小看我们了。

我们都围过去,我已经看到他身上穿的那件盔甲,其实就是最后一只棺材,学名叫金缕玉甲,可是不知道为什么上面的玉片都变成黑色的了,我走近一看,不禁一呆,只见那尸体的胸口竟然还在不停地起伏,好像还有呼吸一样。那呼吸声现在听来非常明显,我几乎能看到有湿气从他鼻子里喷出来。

大奎惊讶地张大了嘴:“这……这……这东西他妈好像是活的!”

\chapter{玉俑}

我非常震惊,往后退了好几步,全身的肌肉绷紧,生怕这尸体会突然间站起来扑过来,轻声问:“这尸体怎么会喘气?你们以前碰到过这种事没?”

大奎发抖着说:“当然没有,要是经常碰到这种事情,我宁愿去扫厕所也不来倒斗。”

我看了看潘子,他捂着他的伤口,一头是汗,说:“别管是什么,快给他一梭子,不死也死了!等一下他要站起来就麻烦了。”我一听有道理,在这地下,多想不如多做,什么事情你快一步肯定没错,忙端起枪,三叔和那胖子忙挥手,同时大叫:“等……等等!”

说着,三叔已经凑到那尸体跟前去了,他一边向我摆手,一边看尸体身上的盔甲,惊讶得嘴巴都合不拢,指着那黑色的盔甲说:“这……这不是玉俑吗?我的天,原来这个东西真的存在!”

我一头雾水,忙问那是什么,三叔激动得几乎眼泪都要流出来,结巴道:“造……造化啊,我吴老三倒了这久的斗,终于……终于让我找到了一件神器,那是玉俑啊。”他抓住我的肩膀,“只要穿了这个东西,人就会返老还童,你看到了没有,这是真的!这具尸体就是证据!”

那个时代,四五十岁已经算很老的年纪了,这一具虽然肌肉瘪了下去,但是这个人的面貌真的非常年轻。我不由暗暗吃惊,心说难道这个世界上真的有返老还童这种事情?

那胖子也看得眼睛都直了,说:“真没想到,秦始皇都找不到这东西,原来在他身上。那个什么三爷,你知道这东西怎么脱吗?”

三叔摇头,“听说这东西从外面是脱不掉的,这也是个麻烦,难道我们要把尸体整个背出去?”

他们两个检查来检查去,我看见那尸体给他们扯胳臂扯腿的,一点脾气也没有,好像也没什么危险,不由心情也逐渐缓和了下来,问道:“如果把这玉俑脱下来,那里面的人会怎么样?”

胖子倒也没想到这一点,说:“那胖爷我倒真不知道,大不了就灰飞烟灭呗。”

我说:“那他本来活的好好的,我们这样不是变谋杀了吗?”

胖子听了几乎要笑趴下了,说道:“小同志,倒斗的要有你这思想觉悟,那啥都不用干了,这古代的王公贵族,哪个不是满手血腥,就算揪出来也得枪毙。你还担心这个,吃饱撑的你。”

我一想也对,看他们忙上忙下的,也不好闲着,就去检查棺材,看看陪葬品里还有没有什么好东西,棺底上是厚厚的一层鳞片状的东西,里面一层一层都是些叫不出名字的明器,我抓了一把这些鳞片,问:“这些是什么东西?”

三叔心不在焉,闻了一下就说:“这是他脱落下来的人皮。”我一阵恶心,马上把东西扔掉,骂了句:“娘的,这鲁殇王是不是得了皮肤病,掉这么多皮。”

三叔说:“你别瞎扯,那是他脱下来的老皮,每脱一次就年轻一点,看这皮量,总脱了有五六层皮了。”

我看这些东西太恶心,像蛇皮一样,也没有兴致,这个时候,那胖子叫了一声:“有门!”

我们围过去一看,只见玉俑掖窝里有一块玉上的金丝多了个头,我纳闷:“我说,死胖子,你他娘的眼睛也太尖了,这里多个线头也能看得出来。”

胖子白了我一眼,在那里嘀咕:“你们这些南派的同志,杀心太重,倒什么墓都是连锅端,这倒斗是细致的手艺,看到没,今天要没你们家胖爷我,你们得把这尸体溶了才能把这玉俑脱出来。”

三叔面子上下不来,骂道:“去你的,还不知道是不是呢,说不定本来这里就多了条线头。”

胖子哈哈一笑,说:“你他娘的还别不信邪。”说着就去扯那线头,手才伸到一半,就听“呼”一声,我就觉得眼前什么东西闪过,那是电光火石一般,三叔反应超快,一脚把胖子踢了出去,胖子刚让开,一把黑刀就“梆”一声钉到树上,没进去大半截。我吓了一大跳,要不是三叔那一脚,胖子的脑袋已经被插穿了。

我们回头一看,只见闷油瓶站在台阶下面,浑身是血,身上不知道时候出现一只青色的麒麟文身,他的左手还保持着甩出刀后的动作,右手提着一个奇怪的东西,等我们看清楚,全部都倒吸了一口冷气。

他右手上提的,竟然是那具血尸的头颅。

闷油瓶看着我们,有点蹒跚地走上台阶,他呼吸非常沉重,看样子情况很不妙,从他满身的伤痕来看,应该是一场恶战,他先看看了那只棺材,然后对我们摆了摆手,轻声说:“让开。”

胖子脑门上青筋都爆了出来,怎么可能买他的账,跳起来就大骂道:“你他娘的刚才干什么!”

闷油瓶转过头,冷冷地瞪了他一眼,说:“杀你。”

胖子大怒,挽起袖子就要冲上去,大奎忙一把把他抱住,三叔一看气氛不对,这胖子也不是个善类,忙打圆场说:“别慌,小哥做事情肯定有理由在的,咱们先听个清楚,他这一路也没少救你命对吧,悠着点先。”

胖子一想,也对,也不好再动手,挣脱大奎,愤然地坐到地上,说道:“你们娘的人多,胖爷我一拳难敌四手,没办法,你们怎么说怎么是。”

闷油瓶把手里的血尸头放到玉床上,咳嗽了一声,说:“这具血尸就是这玉俑的上一个主人,鲁殇王倒斗的时候发现他,把玉俑脱了下来,他才变成现在这个样子。进这个玉俑,每五百年脱一次皮,脱皮的时候才能够将玉俑脱下,不然,就会变成血尸。现在你们面前这具活尸已经三千多年了,你刚才只要一拉线头,里面的马上起尸,我们全部要死在这里。”

他说完又咳嗽了几声,我看到他的嘴角开始有血渗出来,心说不好,可能已经伤到内脏了。

潘子本来已经难受地靠在一边,一直没说话,这个时候突然说道:“小哥,我潘子嘴巴直,你不要见怪,你知道的也太多了,如果方便,不妨说个明白,您到底是哪路神仙,你救了我一命,如果我有命出去,也好登门去拜个谢。”

潘子这话说的很巧,我想闷油瓶他怎么也敷衍不掉了,但是没想到他还是一声不吭,好像根本没想过要去理我们,他走到鲁殇王的尸体面前,厌恶地打量了他一眼,眼里突然寒光一闪,我还没看见他的动作,他的手已经卡住那尸体的脖子,将他提出了棺材,那尸体的喉咙里发出一声尖叫,竟然不停地抖动起来。这一切发生的太快了,我根本无法反应,闷油瓶对着那尸体冷冷地说了一句:“你活的够久了,可以死了。”手上青筋一爆,一声骨头的爆裂,那尸体四肢不停地颤抖,最后一蹬腿,皮肤迅速变成了黑色。

我们全部目瞪口呆地看着他,一时间谁也不知道该说什么,只见他将尸体往地上一扔,好像那玉俑根本是个垃圾,不值一提,我一把抓住他,“你到底是什么人!你和这鲁殇王有什么深仇大恨?”

闷油瓶看着我,看了好一会儿,说:“知道了又能怎么样?”

胖子不服气地说道:“这是什么道理,我们辛辛苦苦下到这个墓里来,好不容易开了这个棺材,你二话不说就把尸体掐死,你他妈的至少也应该给我们交代一声!”

闷油瓶子转过头,看着放在玉床上的血尸头颅,表情非常悲凉,他指了指那彩绘漆棺后部的一只紫玉匣子,说:“你们要知道的一切,都在那匣子里。”

\chapter{紫玉匣子}

紫玉就是紫水晶,一般用来做附身符和辟邪之物,很少有人用来做匣子,这个匣子,看样子是用整块的紫玉挖出来,十分罕见,紫玉不善琢磨,所以这盒子上面什么图案都没有,只在合盖处镶了一道金边,看它放的位置,应该是当这尸体的枕头用的。一般玉枕已经很珍贵了,紫玉的更是价值连成,恐怕当时的皇帝都没有这种待遇。

我们小心翼翼地捧出了这个盒子,放到地上,那盒子没有锁,我们打开一看,里面是一卷镶金黄丝帛,这东西的纤维里镶嵌着金丝,保存得非常好,我们展开一看,左起一行写了“冥公殇王地书”,然后边上密密麻麻都是小字。

胖子比起这帛书来,对那玉俑比较感兴趣,看着看不懂,就嘟囔了几声跑去研究那玉俑去了,闷油瓶拔出树上的刀,躺到一边的玉石床边上,默默地盯着那具鲁殇王的尸体,眼神迷离了起来。

我和三叔坐到他边上,仔细地翻看帛书上文字,以我的水平,只能看懂一些片段,但是把这些片段连起来,就可以看出一个大概,这份冥公殇王地书记载的东西,简直是匪夷所思,如果不是因为已经经历了这么多诡异的事情,我真的不敢相信世界还有这样的事情。

在冥公殇王地书这行字的边上,有一行小字,是他自己写的序,才寥寥几行字,后面便是他从出生到死亡的所有重大事件,如果全部都翻译出来,恐怕十天半个月都搞不定,所幸其中最主要的两件事情我看得懂。

第一件事情是鲁殇王得到鬼玺的经过,那帛书里写的比较简略,我先大概理了一下,念了出来。

他二十五继承了父亲的官位,为鲁国的军队盗掘古墓,出黄金以凑军饷,有一次,他进入了一个不知道年代的墓穴,那棺材里躺的竟然是条巨蛇,躺着一动也不动,鲁殇王胆子非常大,他心说巨蛇卧棺,肯定是妖孽,一刀就把这蛇给剁了,强行下令把这蛇给开膛破肚,结果,从那蛇肚子里剖出来一只紫金盒子。

我看到这里,不由一愣,难道我放在包里的那只盒子,就是蛇肚子里剖出来的?三叔看我不讲了,不耐烦道:“别停,继续说!”我没办法细想,只好回了回神,继续念。

那鲁殇王对这盒子也没放在心上,只当是被蛇吞进去的,后来晚上睡觉的时候,他就梦到一个白胡子老头,问他:“问什么要杀我?”

鲁殇王平时非常暴戾,没少杀人,杀了就忘,也不知道这个老头是谁,说:“想杀就杀!”

那老头突然就变成一条巨蛇来咬他,谁知道那鲁殇王凶得要命,在梦里又一刀把那蛇给砍伤了,然后一脚踩上去,就要砍那蛇头,那蛇突然就开口求饶,说自己的肉身已经被他杀了,如果魂魄再被他杀了,就永不超生了,如果他放它一马,就传他两件宝物。可以使他位极人臣,当时盗墓的军官,虽然隶属于皇帝直接管理,但是地位很低,而鲁殇王自视非常之高,这个条件对他非常有吸引力。就答应了。

那蛇就把怎么开它肚子里那只紫金盒子的办法告诉了他,还传授给他里面宝物使用的方法,那鲁殇王听完之后,“深得其中之妙”,心理觉得此事只应天知,不可传于天下,一刀就把那蛇头剁了下来。

我看到这里,不由咋舌头,这鲁殇王也太狠了。

胖子这个时候跑过来问:“那一个宝物肯定是鬼玺,那另一个是什么?古籍里从来没提到过,会不会就是这个玉俑?”

我示意他不要急,自己继续往下看去。

那鲁殇王醒了之后,用梦里的办法一试,果然开了那个盒子,但是他这里始终没写里面是什么宝物,就说他用了一下之后“颇为顺手”,他觉得这件事情不能让别人知道,就将他带去的随从,连同他们的家属一一残杀,连刚满月的小孩子都不放过。

我看到这里又倒吸了一口凉气,心说这鲁殇王肯定有点心里问题,不然怎么可能凶残到这种地步。

胖子说:“他一个人怎么可能杀掉这么多人,肯定是用了那宝物,真是急死了,你快看看下面有没有写是什么东西?”

我骂道:“你他娘的怎么这么多废话,去收拾你的玉俑去!”

他咧咧嘴,“行行,我不插嘴不就行了,你他妈的念快点,肠子都痒了!”

我不去理他,继续往下看。

接下来的几十年,他凭借那两件宝物,无往不胜,无论是打仗还是朝政,战无不克,风光一时,但是到了晚年,因为多年接触尸气,身体出现了很多顽疾,非常不方便,结果皇帝嫌他年纪太大,就去了他的兵权,让他只需要倒斗,不需要理军务,这其实就是把他贬了下来。

随着他身体一天不如一天,他开始有点怕死起来,有一天,他梦到了几十年的那条巨蛇,那巨蛇和他说,他死期已经到了,我们都在地府里等你,他一看,几乎都是他以前妄杀的人!他醒来后,想起梦里的内容,十分害怕,就去向他的军师求教。

他的军事是一个铁面先生,精通命里风水,他微微一想,对鲁殇王说,上古有一种玉俑,穿在身上可以使人返老还童,长生不老,可惜早已经绝迹,要找,只能去古墓里找,鲁殇王那个时候已经穷途末路了,这铁面先生的话不管是不是真的,都给了他一线希望,而且倒斗是他的强项。于是他彻夜研究古籍,那个时候的文献资料还是比较丰富,很多东西都没有失传,终于他在一处简书中发现了一个可能有玉俑的大墓。

接着,他动用三千多人,花了半年时间,开凿山体,在他估计的区域找到了一个规模巨大的西周皇陵,那个时候各国的国力都不怎么样,所以这个皇陵的规模在当时已经算是叹为观止了。它开山而建,利用天然的洞穴,里面的墓道利用周易八卦的原理,极端复杂,如果不是鲁殇王精通奇门遁甲,根本没有办法走进去,最奇特的是,在作为主墓的那个岩洞里,还有一棵被他称为九头蛇楠的巨树,而一具几乎皮包骨头的青年男尸,穿着一件黑色的金缕玉衣,打坐在那巨树之下的玉床上。

铁面先生看后,断然道,这就是玉俑,这青年男尸似死非死,每隔一段时间,他身上的死皮就会脱落,从里面长出新皮来,他估计这个青年男子,死的时候必然是一个枯朽的老人。

这个铁面先生,十分的了得,竟然知道如何克制血尸,他用特殊的方法,将人俑里的男尸取出,封入副墓室的石棺中,鲁殇王按照铁面先生定下的全部计划,吃了假死药,在皇帝面前假死,皇帝以为他真的可以在阴阳两界来去自如,非常害怕,为了安抚他,皇帝给了他高出一般诸侯王的墓葬待遇,他的亲信就以开凿坟墓为理由,暗地里在这座西周皇陵之上,修了一个扇子一样的古墓,因为他熟知盗墓的各种技巧,所以他四处布下疑阵,留下七个假棺,而把自己藏在西周墓的千年古树里。

在他自己进棺材之前,他将参与工程的所有人全部都杀死,推入河中,然后又毒死他的所有随从,只留下一男一女两个忠心的亲信,将他入殓,那两人也在完成全部事情之后,服毒而死。我估计尸洞里的那多数古尸,应该就是这个时候积下来的。

这个时候,我就有了一个疑问,对三叔说:“那个铁面先生最后到底是什么结局,这里好像并没有提到,难道他也殉葬死了?”

三叔摇摇头,说:“这种人非常聪明,应该早就料到鲁殇王会杀人灭口,应该不会愚忠地为他陪葬。”

闷油瓶淡淡道:“他当然不会,因为到最后,躺在玉俑里的,早就不是鲁殇王,而是他自己。”

\chapter{谎言}

这句话一出,我脑子里灵光一闪,好像有了个眉目,惊讶道:“难道最后关头,两个人竟然掉包了?”

闷油瓶点了点头,看着那具尸体:“这个人处心积虑,只不过是想借鲁殇王的势力,实现自己长生不老的目的而已。”

“这些你是怎么知道的?好像亲身经历过一样。”

“我不是经历过,”闷油瓶摇摇头,“我前几年倒斗的时候,在一个宋墓里,找到一套完整的战国帛书,这份东西,其实就是那铁面先生的自传,他在教授鲁殇王所有计划之后,就放火烧死了自己一家老小,将一具乞丐的尸体丢入火中,冒充他自己,然后自己装成乞丐,逃过了一死,那鲁殇王虽然知道有蹊跷,但也没有办法。最后,他等鲁殇王入葬后,轻易地潜入了墓穴,将已经毫无抵抗能力的鲁殇王拖出玉俑,自己躺了进去,这鲁殇王苦心经营,结果却为他们做嫁衣裳,恐怕他自己怎么也料不到。”

我奇怪道:“那具鲁殇王的尸体被拖出来,岂不又是一具血尸?那这里岂不是有两具?”

“这个他书里也没有写,可能是因为鲁殇王入俑的时间太短,还不能变成血尸。”他的眼神有点不自在,“一本自传,这些他只是略微提了一下,不可能会有详细的记载。”

我看着闷油瓶子,不知道为什么,突然觉得他这句话有点假,我看看三叔,果然他也不信,不过既然人家不想说,谎话都编出来了,你再去拆穿他,也没多大意思了。那闷油瓶说完这句话后,就好像完成任务了一样,又恢复了面无表情,站了起来说:“天快亮了,我们差不多该出去了。”

“不行,我们还没找到鬼玺呢。”胖子说道,“你看这里好东西这么多,现在走不是白来?”

闷油瓶冷冷地看了他一眼,似乎对胖子有点敌视。胖子自讨没趣,耸耸肩膀,说:“行行,不过怎么样也要把这玉俑带走吧?这东西天下可能只有这么一件了,胖爷我可是为了大家着想。”

这话倒是不错,三叔拍他的屁股说:“那你还磨蹭什么,速战速决,离开这鬼地方。”

我突然间对这些都没了兴致,也不想去帮他们,闷上眼睛准备休息一下,这个时候,突然有几滴水滴到我的脸上,我以为下雨了,抬头一看,那张血尸的怪脸,已经探出了玉床,两只没有瞳孔的眼睛,几乎就贴在我的眉毛上。

我吓得跳了起来,只见那血尸的头颅,竟然还在玉床上滚动,这个时候竟然滚落到了地上,好像有什么东西在里面一样,胖子想过去看一下,闷油瓶拉住他,说:“别动,先看看。”

胖子点点头,这个时候,一只非常小的红色尸蹩咬破了血尸的头皮,爬了出来,大奎一看,骂道:“靠!这么小一只也敢在爷爷这里露脸。”举起手里的撬杆就想去敲它。

三叔一把把他抱住,说:“笨蛋,这只他娘的是蹩王,你弄死了它,就闯祸了。”

大奎一愣,不相信道:“这么小一只就是蹩王?那些大个的岂不是要郁闷死了?”

闷油瓶也非常吃惊,一拍我的肩膀,说:“我们快点离开,蹩王在这里,我克制不住这些尸蹩,非常棘手!”

这个时候,那只红色的小尸蹩突然发出了吱吱两声,抖了抖翅膀,好像看到了我们,突然展翅向我们飞了过来。闷油瓶大叫:“有毒的!碰一下就死,快让开!”

三叔一个转身翻到我们这边,他身后的大奎本来已经有点浑浑噩噩,一时间没反应过来,竟然条件反射的一把就捏住了那虫子,他呆了一呆,突然一声惨叫,那只手瞬间就变成了血红色,不仅如此,那血红的部分非常迅速地从他胳臂蔓延了上去。

胖子大叫:“中毒了,快点断他的手!”说着就来抢闷油瓶的刀,那闷油瓶本来已经非常虚弱,被胖子一撞,黑刀就脱了手,胖子凌空一接,突然整个人往下一沉,骂道:“妈的,怎么这么重!”他几次想把刀提起来,竟然都失败了。

这个时候已经来不及,大奎痛苦得整个人都扭曲起来,几秒的工夫,他全身几乎都变成了血红色,好像所有的皮肤突然融化了一样。

他看着自己的手,非常恐惧,想大叫却叫不出声来,闷油瓶看到我想上去帮大奎,拉住我咬着牙说:“不能碰他,碰到就死!”

大奎看到我们都像看到怪物一样退开,非常惊恐,他向我冲了过来,张大着嘴巴,好像在喊:“救救我!”我看到这副情景,吓得一步都走不动,三叔冲过来,一把把我拉开,那大奎扑了个空,像疯了一样,又扑向潘子,潘子情况本来已经很不妙,根本反应不及,胖子大叫不好,一下子抢过我的枪,我大惊,知道他要开枪,忙和他夺起来,混乱间,枪突然走火,一声枪响,大奎头部中弹,整个人一震,翻倒在地上。

我脑子嗡的一声,一下子跪倒在地上,这一切发生的太快了,刚才还好好的一个人,突然就变成了这个样子,我脑子里一片空白,不知道该怎么办。

那只红色的小尸蹩吱了一声,从大奎的手里爬了出来,抖抖翅膀,那胖子骂了一声,闷油瓶大叫:“不要!”已经来不及了,胖子跑过去操起紫玉匣子,一下把那只虫子打烂。

一时间那洞穴死一般的寂静,一点声音也听不到。闷油瓶猛地抓了一把地上的石尘撒在自己身上,大叫:“快走,不然就来不及了!”

胖子看了看四周,什么都没有发生,奇怪道:“为什么要走?”

他话音刚落,原本比较寂静的洞穴,突然就嘈杂起来,无数的吱吱声从四面八方响了起来,然后,我们就看到,那岩洞上大大小小的洞穴里,一只,两只,三只,十只,一百只……无数青色的尸蹩潮水一样冲了出来,那规模,根本不能用人的语言来形容。只见一浪接一浪,前面的踩后面的,铺天盖地地爬过来。

我一看就呆了,三叔一拍我的后脑,大叫:“跑!”

他一把背起潘子,胖子还想去捡那紫玉的盒子,三叔大叫:“你他娘的不要命了!”那胖子一看搬不动,一把抓住那镶金丝帛就塞进兜里。

我们全部上树,这树上乱七八糟的藤蔓和突起很多,非常好攀爬,像我这样身手的人,也一下子就跑上了十几米,那个时候那些尸蹩已经全部涌到了树下,我往下一看,靠,我的天,整棵树下面全是青色的,要掉下去,一点骨头都剩不下来。

那些尸蹩有意识地集结了一下,突然就开始跳上来。它们爬树比我们快多了,一下子就到了我们脚根处。

那胖子爬在我上面,问:“你不是说你们这小哥的血比驱蚊水还厉害吗?怎么没用啊?”

我脑子还全是刚才大奎倒下的画面,根本不想理他,他讨了个没趣,暗骂了一声,突然我就脚下一痛,一只尸蹩已经咬住了我的小腿,我一脚踢掉,往下一看,下面像开了锅一样,尸蹩争先恐后地爬上来,这个时候,三叔在上面叫:“炸药,玉床边上那包里还有炸药!”

我问:“在哪边啊!”

三叔大骂:“你他娘的坐在边上都不知道,在左边那个口袋里!!”我往下一看,那炸药包没在那尸鳖海里,根本看不到,忙开了几枪,只打飞了几只虫子。这个时候,闷油瓶突然从他口袋里掏出几只火折子,点着往玉床上一扔,那虫子虽然已经不怕他的血,但是仍旧怕火,一看到有火下来,“刷”一声,让开了一个大圈子,一下子就露出了那只背包,胖子屁股上已经挂了好几只虫子,大叫:“娘的,快点点个炮仗,我要顶不住了!”

潘子在上面喊:“操!不行,那里面炸药太多了,炸了我们也没命!”我看到越来越多的尸蹩爬上来,知道现在犹豫肯定就是死路一条,大叫:“管不了这么多了,死就死了!”一咬牙对着那背包就是一个点射。

那爆炸太快了,就听一声巨响,我就忽悠一下,觉得我的下巴、屁股、大腿同时被打桩机打了一下,整个人被气浪冲了起来,然后重重撞在什么上面,那一下真的七浑八素,我嗓子一甜,一口血就吐了出来,眼前一片漆黑,脑子嗡嗡直叫,耳朵什么都听不到。

我好久才缓过来,一看,下面的尸蹩已经被气浪冲飞掉不少,我转头也看不到其他人,忙手脚并用,往上爬去。因为身上涂了下面石台的粉末,那些鬼手藤看到我纷纷让开,这个时候,下面又传来了一片嘈杂的叫声,我低头一看,那些的尸蹩又像潮水一样聚拢过来,它们爬得极快,我一看不行,浑身再痛也得继续爬,忙闭上眼睛,拼命地爬起来。

眼看我就要爬到裂缝口子上了,突然背上一痛,回头一看,一只尸蹩已经跳了上来,死命咬着我的背。我转身一枪,就把它打烂。可同时,另一只更大的,一下子就咬住了我的大腿,我一咬牙,拿枪一砸,把它砸了下去,可是它马上就抓住树枝又想跳上来,我回手一枪,把它也打烂掉。可是第三只第四只马上就又跳了上来。

我看到离出路才几步了,心说咬吧,反正你短时间也咬不死我,我上了地面有你们好看,想着继续往上爬,就在这个时候,我抓住树枝的手突然一阵巨痛,我转过头一看,只见一张血脸突然从树干后面探了出来,两只几乎要爆出来的眼珠子直直地盯着我。

\chapter{火}

这张脸一片血肉模糊,不知道是皮肤熔化了露出了里面的肌肉,还是血从他体内渗出来,覆盖在他脸上。刹那间我觉得这张脸非常熟悉,仔细一看竟然是大奎,心中大骇:好好的一个人,竟然成了这个样子。

他左边脑袋上被子弹削去一块皮,都看到骨头了,可是没有伤到里面的大脑,我看他受伤虽然重却不至于死,心里不由大喜,忙说:“快上去,说不定还有救!”

可是他却纹丝不动,我看他的眼神,竟然十分的怨毒,好像不甘心我们舍他而去,我大惊失色,但我的手已经被他的手握住,他身上那种恐怖的血红色,已经迅速蔓延到我的手上来了,我就觉得手上一阵火辣的奇痒,心里大叫:“完了!”

大奎嘴巴里发出含糊不清的声音,突然把我向下面拉,我想到大奎的那种全身皮肤熔化的惨状,不由一阵抓狂,狠命把他的手甩掉,可是他又一把抓住我的脚,张大嘴巴好像一定要我给他陪葬。

我大叫:“大奎,你就放我走吧,这些是命,如果你还想活下去,就跟我上去,说不定还能治好,不然你拉着我陪葬也没用啊!”

他听了这话,不知道受了什么刺激,发了疯一样扑上来,两只眼睛全是凶光,好像完全丧失了理智一样。突然他就一把掐住我的脖子,想把我掐死。

我一看不是你死就是我亡了,突然起了杀心,狠狠踢了他一脚,趁他手一松,贴着他的胸口就扣了扳机,那子弹全是磨平了头的手枪弹,力道很大,把他打得血花四溅飞了出去,他的双手在空中四处乱抓,可是什么都没抓到,重重地摔进尸蹩堆里。

这个时候,我被他抓住的那只手,已经麻得完全没有知觉了,我根本感觉不到自己手还有没有抓着那树枝,就觉得身子直往下掉,忙伸出另一手去抓边上的鬼手藤,可是那手上有天心岩粉,藤蔓一下子就缩了进去,我暗骂一声,整个人滑了下去,撞在一根大树枝上。

树枝上爬满了尸蹩,被我一撞,掉下去不少,我勉强有力气用双腿夹住,停止了继续下滑,可是四周大群的尸蹩又围了上来。我不由苦笑,现在我竟然有这么多死法可以选择,要不就摔死,要不就被虫子咬死,要不就毒死。老天真对我不薄。

正郁闷着,突然胖子从下面爬上来,踢开几只尸蹩,原来这小子爬的比我还慢,他看到我,骂了一声:“你他妈的还有心思在这里趴着,你看看老子屁股上被咬的都是窟窿了!”说着就要来扶我,我大叫:“别碰我,我中了毒,你自己先走,我没救了!”

胖子二话不说,一把背起我:“你拿个镜子照照,你他妈的面色比我还好,简直是面色红润有光泽,怎么可能中毒?”

我一奇,低头一看,只见满手都是红色的疹子,看上去好像被几千只蚊子咬了一样。可是那红色到肩膀就停住了,现在反而在慢慢地消退,不由纳闷,怎么那毒对我没用。

胖子背着我,咬着牙向上爬去,我在背后,成了他的肉盾,那些尸蹩全部都跳到我的屁股上来,张嘴就咬,疼得我大骂:“死胖子,我还以为你是好心,你他妈的原来是想把我当挡箭牌啊!”

胖子大骂:“你啰嗦什么,不服气你来背我!没看见老子屁股已经没好肉了嘛!”

我不想和他扯蛋,这九头蛇柏靠近树干的一圈挂的全是尸体,非常密集,胖子不时就会撞到一堆骨头上,幸好那些尸蹩也有同样的麻烦,太多的东西它们分辨不清,不少就跳到那些被我们撞得乱转的干尸上面大咬。

胖子一看,觉得这是个好办法,就叫我去撞那些尸体,能拨的都给它拨一下,让它们都动起来。我虽然一百个不愿意,但是也没有办法,小命要紧。

这一路上我见一个就踢一脚,一下子我们经过的地方全是打转的尸体,这虫子的智商不能和人比,就见他们乱做一团,也不知道是来追我们好,还是去咬那些打转的尸体好,竟然停在那里原地转起圈来,胖子乘机加快速度,一下子就拉开了距离,我们终于可以喘一口气。

我的手脚经过刚才的运动,已经基本恢复了知觉,我心里暗想,我中毒时候的感觉和笔记里爷爷中毒时候的感觉一样,最后爷爷也没有死,莫不是因为这样,我身上就有了免疫力了?

想着也想不明白,我看手脚已经可以动了,就让胖子给我放下来,见胖子满脸是汗,喘着粗气,心说在石台上的时候我背你背的吐血,现在算扯平了。这个时候,我突然看见,有一个人坐在胖子后面的一根树枝上,对我招了招头。

我一哆嗦,忙揉了揉眼睛,那人已经不见了。我以为他躲到那树后面去了,忙探头过去看,胖子大叫:“别磨蹭了,快走吧!”

“等一下!”我一把拉住他,“往左往左!我刚刚看到个人在对我招手。”

他叹了口气,跟着我爬过去,一看根本没人,只有一个刚能勉强容纳下一个人的树洞,里面黑漆漆的,不知道有什么东西。

胖子用手电一照,吓了一跳,只见那洞中有一堆卷起来的藤蔓,里面缠了一具已经高度腐烂的尸体,两只蓝色眼睛已经浑浊得看不到瞳孔,嘴巴张得大大的,不知道想对我说什么,胖子看着我:“怎么是个死人,你该不会是看到鬼了吧!”

这一路过来碰到的怪事情太多,有鬼也由不得我不信了,我想着,他既然招手让我们过来,肯定是有什么目的,想到这里,便习惯性地去看他的嘴巴。但是他下巴已经烂穿了,有东西也掉了,我继续找,发现他手里好像抓着什么,掰开一看,原来是一块吊坠。

下面的尸蹩又开始吱吱叫着爬上来,我也没心思再去翻他身上的东西,看他穿着迷彩服就给他敬了个礼,然后继续往上爬。胖子爬得飞快,我们离顶部的裂缝本来就不远,三下五除二就爬了上去。

我们一爬出裂口,同时往下一望,只见那些尸蹩好像一点也没有停下来的意思,几乎都涌到了裂口边上,胖子大叫:“还没到休息的时候,快跑!”

我在那地下待了这么久,已经搞不清楚方向了,就见前面草丛突然跑出一个人,扛着什么东西跑过来,我认出是三叔,不由大喜,三叔看到我大叫:“,快去后面把那些汽油都搬过来!”

我跑过去一看,原来这条裂缝和我们下盗洞的地方只隔了一个矮悬崖,才十米都不到,我们的装备都还在,我看到了那几桶汽油,心头火起,心说:“好,这下子有你们好看的。”

和胖子一人扛起一桶跑回去,三叔已经把第一桶全部都浇了下去,这时候那些尸蹩几乎已经爬到地面上了,三叔一个打火机扔下去,就见火光一冲,马上就是一阵扑鼻的焦臭,那如潮水一般的虫子瞬间就退了下去,汽油在那裂缝处形成了一道火墙,看着那些虫子在里面被烧得嗷嗷直叫,真是大快人心,我们火上浇油,把第二桶第三桶也倒下去,一下子那裂缝里喷出来的火就几乎比两个人还高了。热浪逼过来把我的眉毛都烧了。

我退后了几步,看了看手里的吊坠,上面是一块名牌,那具尸体的名字应该叫James,我擦了擦放进我的上衣口袋,心说有机会就还给你的家里人,现在你就安息吧。胖子被火热得全是汗,问三叔:“那两个人呢?”

三叔指了指后面:“潘子有点不妙,好像发烧了,那小哥,我就没见到了,还以为和你们在一起。”

我看了看胖子,胖子叹了口气:“爆炸后我根本就没看见他,那恐怕是凶多吉少了。”

三叔摇摇头,说:“不会,这人神出鬼没的,而且刚才他一直是在我们上面,就算被气浪冲散,估计也是冲到上面来了。”

我看三叔的表情,就知道他也没什么把握,那闷油瓶子虽然厉害,在炸药面前还是和我们一样,如果他被气浪摔到树外面去,真的是十死无生。

我们在附近找了一圈,没有什么收获,不见有人离开的痕迹,三叔叹了口气,对着我苦笑了一声。

我们回到营地里收拾东西,点起篝火,把包裹里的罐头热着来吃,我已经饿得够戗了,不管是什么东西都能吃下去,三叔边吃边指后面的矮悬崖:“你们看,这营地就在这裂缝的边上,看样子那老头子看到的树妖就是这棵蛇柏了,肯定是他们晚上庆祝的时候动静太大,把这蛇柏从裂缝里吸引了出来。幸亏我们没过夜,直接下到盗洞里去了,不然恐怕早就被这蛇柏拖走了。”

胖子说:“不知道那火能烧到什么时候,如果火灭了,那些虫子再出来就麻烦了,现在天快亮了,我们快点出了这个森林再说!”

我匆匆扒了几口,点点头,胖子和三叔轮流背起潘子,就往树林里出发。

一路上很平静,来的时候我们是说说唱唱,回去的时候我们是闷头赶路,几乎是逃命一样。

我已经是一个晚上没有休息,精神又高度紧张,现在体力已经到达极限了,走到最后,几乎是凭借精神的力量在支持,如果前面突然出现一张床,我躺上不要两秒就能睡着。我们走了将近半天加一个早晨的时间,走出了那片树林,然后翻过那泥石流形成的石头小坡,终于看到了那亲切的小村庄。

我们不敢松懈,先把潘子送到了村里的卫生所,那个赤脚医生过来一看,眉头大皱,忙招呼护士过来,我往那凳子上一躺,才听他们说了两句话就睡着了。

那是真的累到极点的睡眠,一个梦都没做,也不知道睡了多久,醒过来的时候,就听见外面乱作一团,不知道出了什么事情。

\chapter{紫金匣}

我迷糊着,不知道外面出了什么事情,想问三叔,却发现他也在我边上的凳子上打瞌睡,睡得比我还死。我跑到卫生所外,看见村子里的人拉的板车,拉骡子的,都急急往山里面赶去,一个山娃子边跑边叫:“不好咧,不好咧,山上起山火咧。”

我大吃了一惊,心说难道刚才我们那一把火,把林子给烧着了,回想一下刚才烧那洞的时候,确实没在边上做什么措施,如果那火蔓延开来,把森林烧起来,那真的太不该了。

我心里有点发慌,这山火一旦烧大,不是死一个两个人的问题,我们这些城市里的人,一点森林防火的意识都没有,这下子祸闯大了。

我跑进去忙叫醒三叔,两个人在那医院里搬出两只接尿用的便器,实在找不到东西也凑合了,跟着大部队向山里跑去,这个时候胖子坐在一只驴拉板车上跑过来,手里举着个脸盆大叫:“闯祸了,快上来!快去救火!”我们一齐跳了上去,那驴车歪歪扭扭的就出了村口,只见远处的山上一大片黑烟,看样子烧得很大。三叔傻了,轻声说:“看方向,还真是我们放的那一把火。”

我忙捂住他的嘴,前面有个村干部模样的人在往回跑,大叫,“快打电话给部队,前面山塌下去了!”

我一听就知道,可能是那岩洞被火烧塌掉了,心里担心,要是那些尸蹩从洞里冲出来就麻烦了,我们快驴加鞭的跑到那泥石流冲出的土堆旁,那胖子手真黑,把那驴抽的屁股都肿了。

那些村民平时都经历过防森林火灾的训练,他们一部分人在树林里开路,另有人开始用脸盆打水,往里面运去,我一看这盆盆罐罐的,来回到火场最起码要两个小时,根本是远水解不了近渴,忙叫道:“老乡们别打水了,这点水根本灭不了火,不要做无谓的牺牲,还是等部队过来吧!”

那些人像神经病一样看着我,一个年纪比较大的人说:“小伙子,这些水是用来喝的,火场里面没水喝很快就会干死的,我们要在边上砍出一片防火带,火烧到那里没东西烧了,就会自己灭了。你们不懂就不要在这里瞎掺合。”说着看了看我们手里的便器,摇了摇头。

我被他们看得脸通红,心说这下子面子丢大了,以后怎么也不敢胡乱发表意见了,忙低下头,跟着那些大部队急急进了树林,路上的树全部被砍掉了,走起来快了很多,大概一个小时以后,我们已经感觉到了温度明显升高了。前面漫天都是黑烟。

那些村民都拿出口罩往水里一浸,带到脸上,我看看胖子,他的衣服上本来就已经没多少布了,看他好像下定了决心,拿出那块镶金丝帛就浸到水来,绑到自己脸上,拿起把铲子学着那些村民挖防火沟渠。

山火蔓延极快,危害性极大,大型的山火必须出动飞机才能控制,所谓控制就是让它自行熄灭,想要像城市火灾一样浇灭是不可能的。这一棵树长成材要二十几年,但是山火十分钟就能全部烧光,破坏力极大。而且山火范围非常广,如果你只在一个点上灭火,它很快就会从你看不到的边缘绕到你后面去,等你醒悟过来,你已经在火区中央,只有等死的份了。

我记得有一部美国的片子,就是讲一群消防员被火包围以后,求救无门,在生命最后时候的故事。当然这样的情况肯定不可能发生在我们身上,现在火灾的区域还不是很大,而防火渠挖得很快。

我们一直在那里干到下午两点多,天上出现了护林队的直升飞机,不一会儿很多部队在树林里集结,替下了我们,我特别担心有人会因为这场火牺牲,幸好最后清点人数的时候只有几个人受了轻伤。

我们回到村里,几乎都要休克了,我肚子饿得要命,叫一个娃给我弄了两个烧饼,两口一个,从来没吃过这么香的,眼泪都下来了。那村支书模样的人还表扬我们,说我们城里来的人这么高的觉悟,真的非常少见。

我心说,你千万别夸了,再夸我心里真过意不去,你要知道我就是那纵火犯,非掐死我不可。

护士给潘子换了绷带,洗了伤口,他的呼吸已经明显缓和了,但是还没有醒,那医生说叫我放心,现在暂时还没有危险,等一下如果有伤员,就把潘子一齐送到市里的大医院去。我一听稍微有点心安。

我和三叔回到招待所,好好地洗了个澡,不脱光还不知道,我从上到下一看,几乎没有一处地方是好的,不是淤青就是破了皮,逃命的时候没感觉,现在它们都来提醒我了,我从浴室里出来的时候,几乎腿都迈不开。

我回到床上,一下子就睡着了。这一觉是真的非常香,一直睡到了第二天中午,起来的时候看见胖子和三叔也躺在他们床上,呼噜打得像雷一样。

我下去吃了早饭,问了服务员,火已经灭了,按这规模只能算是个小山火,军队已经撤了回去。我听了心里踏实了一点,和那卫生所的人打听了一下,潘子已经被接到济南的千佛山医院去了。我道了谢,觉得在这个地方还是不能久待,就预备着回去。

闲话也不多讲,几天后我们回到济南,我和三叔先到收容潘子的医院办理了住院手续,他现在还没有脱离危险,仍旧昏迷中,我和三叔决定在这里住几天,胖子一出山就急急和我们分了手,只留一下一个电话以后联系,他把那镶金的帛书交给我三叔处理,这一天我给医院打了电话,潘子还没有醒,不由叹了口气,这个时候三叔一脸阴沉地走了进来,骂道:“气死我了,竟然被人摆了一道!”

我大奇,以为他在古玩市场被人骗了,说:“三叔,以你的资历还被骗了,说明那东西仿的很好,你再转手出去肯定也没问题啊。”

三叔掏出了那块镶金丝帛,对我说:“转手,转个屁啊,我说的不是古董,是这个东西!!”

我几乎从床上掉下来,大叫:“什么!不可能啊!”

三叔说:“千真万确,这东西里的黄金含量,我送去检验,纯度太高,那个时代根本无法炼出来,这是一份几乎完美的赝品!”

我不敢相信,三叔叹了口气,“我老早就在怀疑了,那年轻人明明可以击败血尸,为什么一开始一味地逃跑,到最后才将那血尸除掉,他必然是想由此甩开我们,自己一个人去做一些事情。”

我惊讶道:“难道他和我们走散的那段时间里,已经去过那个洞穴,打开过鲁殇王的棺材?将这块假的镶金丝帛放进去?这怎么可能啊,一个人怎么可能做得到?而且那树洞被那些铁链扯开的,只要被人打开过,我们一定能看出痕迹的。”

三叔说:“你有没有看过那棺材的背面,他是倒斗的,他很可能在树的背后挖了个盗洞,直接从棺材的背面将那镶金丝帛掉包掉了!”说完叹了口气,“可怜我十几年的江湖经验,也没看出来,这个人,真的深不可测啊,我本来还以为只是发丘中郎将的后人,看样子的,他的来历,恐怕远不止这么简单。”

我非常不理解,说:“难道上面记录的那些东西都是假的?”

三叔点点头,气道:“这些山海经一样的故事,本来听起来就不太可信,只不过当时我们被那个古墓神秘的气氛感染,竟然相信了,现在回忆起来,破绽太多了,而且你想想就你那水平,为什么只能看懂最重要的那两段?其他那些都看不太懂,说明这两段他特别做了工夫。”

我张大嘴巴,三叔大大地叹了口气:“看样子这个鲁王宫的秘密,只有他知道了,现在那个墓都塌了,要想在进去看也不可能了。”

我这个时候灵光一闪,说:“对了,对了,我差点忘了,还不是完全没戏,我从那洞里带了东西出来!”说着就去狂翻我的背包,一边祈祷千万别丢了,好在那紫金盒子还在,我拿出来说:“就是这个,是从那狐狸尸手上拿下来的。”

三叔一看,说,“这个是只迷宫盒子啊,里面主要的空间用来装锁了,装不了多少东西,这盒子很难开的,你看。”他把那盒子的顶盖子一拧,盒子的底部四个角一齐展开,露出了一个转盘子,上面有八个孔,每个孔上都有一个数字,很像老式电话的拨号盘。“这种盒子是最古老的密码盒,你要知道密码才能开,你等一下,去那修车铺子里借个气割过来,把它割开来看看。”

三叔急急地跑了出来,我叫都来不及,心说,八个字的密码,难道是那个02200059?怎么可能啊,这个号码可是印在一个美国人的皮带钢印上的,想着我尝试性地拨了一下,0-2-2-0-0-0-5-9,咔一声,我一愣,那盒子发出一阵类似于发条的声音,盒子盖自动翻了起来。

(《七星鲁王篇》完)

◆ 第二卷 怒海潜沙 ◆

\chapter{蛇眉铜鱼}

那盒盖缓缓地自动打开,里面只有小拇指大的一个空间,放了一个小小的铜鱼,我拿出来一看,那鱼的样子很普通,但是做工很精细,特别鱼的眼睛上面眉毛的地方,是一条蛇的样子,栩栩如生,我非常惊讶,这个东西有什么贵重的,为什么要放得这么好。

这个时候三叔已经拖着个气割钢瓶走进来,看到那盒子已经开了,惊奇道:“怎么开了,你怎么打开的?”

我把那数字的事情和他一说,他也大皱眉头,道:“越来越乱了,看来这帮美国人也不是来单纯倒斗的这么简单。”他拿起那条铜鱼,突然脸色一变,咦了一声:“这不是蛇眉铜鱼吗?”

我一看他好像知道,忙问,他从贴身口袋里拿出一个东西,递给我,我一看,也是一条很精致的铜鱼,大概只有我小拇指这么大小,铜鱼的眉毛也是两条海蛇,做工很上乘,每个鳞片都非常细腻,应该和盒子里的这一只出自同一个来源,美中不足的是,他这条在鳞片的凹槽里,有很多细小的白色石灰状污垢,粘得非常牢,我一看就知道了,说:“这是海货?”

三叔点点头,我挺吃惊的,海货就是海里捞上来的古董,一般都是些青花瓷器,在海里淘古董比在陆地上方便,因为很多东西都是露出在海底地表上的,但是海里微生物太多,从海里带上的东西,大部分都有白色的灰状污垢,是很难洗掉的,所以价值上就打了折扣。

我很迷惑,记忆里三叔不会对这种低价货感兴趣,问三叔道:“你难道去倒过海斗?”

三叔点点头,说:“只有一次,我真是后悔,要是那次我能忍住不去趟那把混水,我现在肯定已经孩子都一大把了。”

那三叔的故事,我知道一点,三叔以前有个女人,也是个女中豪杰,听说他们还是在斗里认识的,那女的叫文锦,听说是个挺文静的女的,看不出是个摸金的北派,三叔和她好了有五年,女的寻龙点穴,男的探穴定位,号称是倒斗界里的神雕侠侣,后来突然就听说那女的失踪了。我只道是进斗的时候失了手,女孩子干这个本来就不合适,家里人都挺惋惜的,不过那时候我才几岁,也不懂这么多,只看到三叔一个星期像个木头一样的,老伤心老伤心,后来也就渐渐好了,这小时候的事情,我也记不清楚,现在一听到三叔好像想讲出来的样子,心里虽然很想知道,也不能表现得太八卦,问:“那时候出事情的,难道是个海斗?”

三叔叹了口气,说:“那个时候我和她都还年轻,她有几个同学都是考古队的,他们隐约知道我是个手艺人,我也没想过要瞒他们,大家都很要好,后来他们去西沙做沉船考古,我也跟着去了,只是没想到,”他顿了顿,好像不太想想起那个事情,“没想到,那水底下沉的东西,竟然会这么大。”

算起来那应该是十几年前的事情,三叔其实对海斗没什么经验,也算是被爱情冲昏了头,竟然在文锦面前夸下了海口,说自己如何如何了得,于是就跟着那考古队出了海。他们包了当地一艘渔民的船,花了两天的时间,到了西礁的碗礁附近,那地方是古代海上丝绸之路上最凶险的一段之一,沉船很多,三叔下去一看,几乎呆了,只见海底到处都是破烂的青花瓷器,那规模真的是叹为观止。

文锦告诉他,这些东西是沉船上倒下来的,被海水冲得到处都是,以前渔民一网下去,就能拉上来四五个瓷器,不过他们认为这入了水的就是海龙王的,一般都会扔回去了。

可惜的是,这些东西几乎都是烂的,很少能找到完好的,就算是找到了,上面也大多数都寄生了海生物,很难清洗干净,文锦的同学是以考古价值来看这些东西,所以都很兴奋,三叔看出去就是一片荒凉,心疼得要命,心说他妈的沉船的时候我怎么就没生出来,他也没想那时候的青花瓷器还根本不是古董。

他们在水里转了有两三天,弄上来一筐一筐的瓷器,三叔好这个,对于瓷器他是了如指掌,随便拿起一只就能讲个半天,一下子他就变成考古队的精神领袖,他姓吴,叫三省,他们那些小年轻就叫他三省哥,三叔就飘飘然了,还真把自己当他们的头了。

第四天的时候,出了个事情,有一个考古队的,坐着皮艇出去,到了黄昏还没回来,其他人急了,就让大船起锚去找,后来在碗礁两公里外的一处礁石山上找到那只搁浅的皮艇,但是上面的人不见了。

三叔一想糟糕了,可能人下水去摸东西,出了事情,忙连夜打上装备潜下去,摸了有半宿,终于找到那人的尸体,脚卡在珊瑚礁里了,已经得的涨了起来,他们把尸体拖上来,三叔看见他左手死死抓着什么,掰开一看,就是那只蛇眉铜鱼。虽然死了个人,大家很悲痛,但是三叔已经意识到这水下面可能有什么东西,不然这个人不会连夜来这里打捞。

三叔猜测,可能是白天在拖寻(用船拖着人搜索)的时候,这个人看到什么东西,没说出来,晚上想在没人的时候再回去看看,结果出了事情。当然三叔没把他的想法说出来,因为现在人已经死了,说这些也没意义了。不过,他手里抓的蛇眉铜鱼,肯定是个提示。

第二天,三叔把这个事情和那些人提了一提,当然他是这么说的:某某同志为了考古事业,加班加点的工作,不幸出了意外,不过从他手里的劳动成果来看,这位同志显然已经在海底发现了什么,他用他自己的生命换来了这个蛇眉铜鱼,所以我们不能辜负他云云。调动一番,众人情绪有点恢复,于是回到了出事的海域,下水进行了地毯式的搜索,那个时候就有了眉目。

他们在附近的水下面找到了四十多个巨大的石碇(古船锚上的配件),大小规格都一样的,上面的刻字,已经基本上看不清楚了,三叔估计,这四十多个石碇,要不就是四十艘规格一样的船上遗失的,要不就是来自于同一条船上的。一想就很明白,怎么有可能四十艘船同时在一个地方沉没,这底下,肯定有一艘十分巨大的船。甚至大到,需要用四十只锚才能固定住。

三叔对历史非常熟悉,看到这里,心里已经有了一个十分大胆的猜想,他浮上水面的时候,对文锦说:“这下面,好像是个沉船葬海底墓。”

\chapter{双层墓墙}

文锦和三叔的背景完全不同,三叔是土溜子,要不是生在倒斗世家,肯定是个土匪,凡事先考虑个利字,看人也是从利字出发。文锦就不一样,她是留洋回来的,思想比较开明,对于倒斗主要是还是兴趣,而且边倒边考,所以听三叔这么一说,她首先考虑到的是这个古墓的考古价值,当时她就想把这个设想告诉她那些同学。

沉船葬海底墓非常稀少,传说里用这种葬法的好像只有沈万三的儿子,所以文锦的想法应该是非常有良知的,但是三叔却有点为难,因为他一想到那些东西捞上来要充公就很不自在,但是文锦很有办法,一个微笑然后一个吻就把三叔从一个绿林好汉变成一个共和国的考古研究者,而且还是义务工作。

当天晚上三叔琢磨了一宿,他从来没倒过海斗,又在别人面前夸下了海口,明天不表现还不行。他想了想,这海里也不能下铲,一来力气使不上,钉不下去,二来,就算挖出来了,海泥和陆地上的完全不同,他那点破经验完全没作用。他回忆了我爷爷笔记本里的记录,我爷爷的确是倒过几次海斗,但是也没有什么特别的方法,主要还是看地形。

沉船葬海底墓,就是把陵墓修在一艘船上,然后在海里找一处谷地或者是海沟,把船砸穿,将墓沉下去,然后再在上面封上土,其实和陆地上一样,只是换到海里而已。三叔估计,他们待的地方,原来肯定是个小海谷,后来被填平了,在沉船的时候,四周必然需要很多锚来固定,如此说来,那锚落点的中心,或者偏一点的地方,肯定就是葬点。

三叔越想越有道理,顿时信心十足,第二天气也顺了,他把那些人带下水,把那些个碇石全部用绳子连了起来,然后在中间标了个点,他在那区域内的几个地方都下了铲,果然,在中心偏东边的地方,他们发现下面有木头。

接下来,他们用传统的定位法,竟然定出了一个“土”字形的巨大地宫,由两个耳室、两个配室、一个甬道和一个后殿组成,建筑面积大约有一千多平方米,其中后殿最大,长三十多米,宽十米多,看来是放棺材的地方。

三叔呆了,心说乖乖,这个斗里是什么人物,看样子真不简单。这样的规模都比得上皇陵了。

当天晚上,所有人都兴奋得睡不着觉,他们围在一起,一边吃鱼头锅海鲜一边讨论怎么进去,三叔给他们分析了沉船葬的结构,墓葬最怕水,现在不知道下面的冥殿里有没有进水,如果已经进了水,只要打个洞进去就可以了,他们都有潜水服,应该没有问题,如果下面还是一个密封的墓室,那就比较难办,因为一旦被凿穿,那水冲进去可能会造成灾难性后果,从探铲打上来的木片来看,下面应该还是有空气存在的。整个墓很大,很容易造成毛细结构,可能里面有几个房间里还存有大量的空气。

三叔的这套理论是他多年盗墓的经验,说得那些书呆子一愣一愣的,最后,他把所有的难题都集中到怎么打盗洞上去了,这个水底都是沙子,定不住型,很容易就塌,那可不是闹着玩的,在水里被压住基本就是死路一条。最后他们讨论来讨论去,决定用土办法,那渔船上有炸鱼的炸弹,先用炸弹在一边炸出一个土坑,把上面容易坍塌的沙子炸掉,然后在下面比较结实的海泥里挖一个斜向下的洞,这个工程浩大,但是这些人全部都斗志满满。三叔估计了一下,大概要一个星期时间,可是那尸体还在船上,再不送回去就要发臭了。

他们想了一个折中的方法,让大船先把尸体送回去,他们在小船上作业,因为那几天天气非常好,所以大家一点也不担心,他们把三只皮划艇绑在一起,然后把所有需要用的装备都搬到一块礁石上。

第二天大船就开走了,三叔觉得有点不安,大船一走,在海上就一点保障都没有了,但是他们当时被那大墓冲昏了头,只想了一下就又投入到工作中去,那盗洞打得很顺利,比三叔估计的快多了。可是四天后,等到他们打到墓壁了,那船还没有回来,这些人开始担心起来。三叔知道现在只有继续工作才能维持一个良好的秩序,不然可能会出现恐慌,就一直安慰他们,并不时说一些鼓舞的话来转移他们的注意力。

他们在洞里清理出一块墓墙,三叔敲了敲,这些砖头是空心的,大概是为了减少整个墓穴的重量,不然就算船再大,船底也支撑不住,他看到每隔五米,就有一个钢笔直径的小孔打在墙上,看样子这个墓设计的时候,就是以水来封墓的,里面应该充满了水。他们入下心来,开始拆砖头。

在进墓前其实三叔已经想过,在这水里,什么机关暗器都没用,因为海水阻力太大,如果有暗弩,就算没烂,那发出来的箭也是慢动作,陷坑也不可能,不要说根本掉不下去,就算掉下去了也能游上来,其他各种落石机关,要用水银击发的,在水里就完全不灵光,水银在水里流得很慢而且很容易扩散。其实这水就是一个致命的机关,古时候没有氧气设备,完全没可能去倒海斗,所以这个斗里有机关的可能性非常小。

他们卸下墓墙,里面就是空洞洞的一片黑,三叔知道现在这些人都靠不住,让他们不要动,自己打起探灯钻进去,发现只往前一米,又是一道墙,这面墙壁的用砖比外面的那层大了很多,并且墙缝里封上了白膏土。三叔夹在两道墙壁之见,前后左右照了一下,发现他头顶上的内墙上,有一个半米长宽的正方形的墓道口,三叔一看就明白了个大概,看来要进这个墓,靠挖是不行的了。

回到水面以后,他们爬上一个礁石开会,三叔说:“这个墓有两层墓墙的,外层墙和内层之间灌满了海水,然后在内墙上做一个通道往里面盘旋进水,这样的设计,里面肯定有一个空间是无水的,利用气压的原理将一部分空气留在了墓室里。现在不知道那个墓道有多长,明天我们下去三个人,每人带四个氧气筒,看看能不能撑到那里。”

他们在那里讨论来讨论去,三叔肯定是要下去,其他两个名额需要筛选,因为如果里面没水,那情况就比较复杂了,可能会有危险,这个时候,文锦突然惊叫起来,他们吓了一跳,原来不知道什么时候,他们坐的礁石竟然升高了,三叔往下看,本来离海面只有半米都不到,现在竟然有五米多。

他意识到有点不妙,抬头一看天,只见远远的海平线上,一条黑线正在逼近。他们中有一个叫李四地的男学生,父母是渔民,他一看到这个情景,吓得嘴唇发白,说:“大风暴要来了!”

\chapter{大风暴}

这个李四地水性很好,他们水里的工作都是他负责的,他说:“一个小时之内这里肯定有一场巨大的风暴,这海水退下去这么多,就是一个证据,等一下这些被低气压吸过去的海水一齐冲过来,就是一场小型的海啸,我们这里只有三只小皮艇,恐怕不是很乐观。”

他说的已经十分委婉,但是三叔看他的表情,分明是觉得他们已经死定了,这些人没见过大世面,一个个都吓得面色发白,有几个女生都哭了起来。

三叔拉着文锦的手,发现她手心理都是汗,知道她也很害怕,那个时候三叔也没有处理过这种事情,但是他到底个职业倒斗的,心里素质非常之好,当时他就提醒自己,不要乱,如果一乱那就真的没戏了!

他清点了一下人数,他们来的时候一共是十个人,现在有一个人死了,另一个人因为需要去向上头汇报事故和海下的发现,跟着大船回去了。现在加起来,只有八个人,三叔问李四地:“这风暴要持续多少时间?”

李四地说:“这种夏季风暴时间很短,大概几十分钟之后就过去了,可是那个时候海水最起码要升上去五六米,到时候这些礁石全部都得淹掉。”他摇了摇头,“这几十分钟可不是闹着玩的,被这浪一冲,要不就是撞到礁石上撞死,要不就是被卷到深海去,不是我吓唬你,这下子真的麻烦大了。”

三叔脑子转得很快,脑子里好几个方案已经瞬间提出然后否决掉了,坐皮艇划回去,找死,划得再快也跑不过风暴,用呼吸器躲到水里,这碗礁附近的海底最深也只有七米多,根本不管用。

三叔看到那几乎已经可以用肉眼看到的海底,犹如黑夜里一道闪电,突然间一个十分冒险的计划在他脑子里浮现出来。那个时候根本不容许他再去讨论可行性,他对那些人说:“我们也不要想这么多了,大家集中一下氧气瓶,看看还够多少空气,我们下古墓里去避一避!”

三叔下古墓是轻车熟路,所以没觉得有什么关系,但是其他人都是书呆子,这个提议太大胆了,这句话一出,众人哗然。三叔一看意见不统一,忙给他们分析利害关系。

他指了指海平线,说:“大家看这风暴,现在我们还没有感觉,但是大家都看过关于海啸的记录电影吧,这东西不是闹着玩的,如果在这里等风暴过来,十死无生,肯定是连尸体都找不到,而这海下面,有一个现成的避难场所,我们已经知道,这个古墓里肯定有空气,这海斗里的空气其实是活的,因为它连着活水,所以里面的空气质量应该还过得去,我们人不多,在里面待一个小时再出来,是唯一的生存机会了!”

三叔有这么一点鼓动人的天赋,不然他以后生意也做不到那么大,众人听他说的头头是道,不由心里也出现了一线希望,他们集中起所有的潜水设备,将三只皮艇都放气叠好。一切准备就绪,三叔先和他们规定了一些在水下活动的手语,然后带他们潜入水下,他自己打起一个防水的探灯,第一个爬进墓道里。

那个时候的潜水设备,头上是一个大头盔,看上去十分笨重,但是这个东西非常结实,如果前头有什么大型的海生物,有这个头盔,一下子也吞不掉他。三叔尽量使自己放松,一边游一边看,这个墓道竟然是越来越窄,按照这个趋势,最后能不能容他们通过也是个问题,好在他全套工具都在身上,实在不行,还能破出一条路来。

墓道的壁上有很多的人脸浮雕,现在上面都是厚厚的一层附着物,无法认清是哪个朝代的,这些人没见过大世面,都忘了现在的处境,围上去研究这些脸,三叔头痛不已,不得不经常停下来催促他们。

他们往前游了十五分钟,转了好几弯,已经摸不清楚方向了。三叔觉得这些人太乱,应该整顿一下,于是做了个手势让后来的人停下来,他让文锦去数数人数,看看有没有人掉队,在这狭窄的墓道里游泳都耗费体力,那些人都累得不行了,一看这手势如获大赦,都东倒西歪地坐下来。

三叔无可奈何的看着他们,心说这个老大还真不好当,他用探灯照着,想先到前面去看看,这个时候,文锦拍了拍他的脚,三叔转过头,看见她表情非常惊慌,心里一紧,心说难道真的有人掉队了?

文锦手忙脚乱的不知道怎么表达自己的意思,她伸出一个手指,不停在三叔面前晃,三叔不知道她是什么意思,问她:是不是少了一个?文锦看着三叔的嘴形,摇摇头,一只手掌全部展开,另一手伸出四个手指,把两只手放到一起,三叔非常纳闷,他仔细看着文锦的嘴形,突然发现她其实想说的是:“多了一个人!”

\chapter{海鬼}

三叔吃了一惊,如果后面少了一个或者两个人,他都可以理解,甚至所有的人都消失了,他也可以理解。但是多出一个人,太匪夷所思了,他以为文锦数错了,回头自己也数一遍,自己是第一个,文锦第二,然后依次下去,三,四,五,六,七,第八是李四地,第……

他突然倒吸了一口凉气,因为他已经看到了那多出来的第九个人,那人躲在长长的队伍后面,模模糊糊的,连个人形都看不清楚,肯定有问题。

三叔不由开始冒冷汗。他也不是害怕什么妖魔鬼怪,只是在水下面他一点经验也没有,也不知道这后面是什么东西,这粽子应该不会游泳,话又说回来,这海斗里的粽子不知道应该怎么称呼,难道叫海粽子?要不饺子?

他摇摇头,心说这李四地他娘的也真迟钝,这么个东西跟在后面都没有发现,现在这个情况,谁都指望不了,唯有冲过去看看。他暗中掏出一把匕首,藏在手后面,就往回游去。

那第九人一动也不动地站在那里,李四地看见三叔径直向他游过来,也意识到自己背后有什么不对劲,忙一回头,他一动,那个人也突然一动,好像在模仿他一样。李四地吓了一大跳,往后退了好几步,那人突然也往后退了好几步,看它好像完全是学着李四地的动作,三叔发现这人动作不仅奇怪,还有些滑稽,拿头上的探灯一照,那东西被光一刺激,慌忙向后逃去,三叔刹那间看到一张狰狞的长满鳞片的巨脸一闪而过,吓得手上的匕首都差点脱手。

李四地吓得面无土色,就要往前游,不敢再待在这里休息,三叔忙一把拉住他,他对三叔大叫,看嘴巴的形状好像是“好孩子,好孩子”。

他本来就有口音,平时说话已经很吃力,嘴形更是看不懂,三叔看他几乎歇斯底里了,竟然想摘掉自己的头盔,忙把他按到墙上。这一按不要紧,那面墙的四条缝里同时一松动,竟然就缩了进去,突然间四周的水就往那墙里猛灌进去,三叔心说不好,已经来不及了,他们就像抽水马桶里的蟑螂一样,被卷进那个墙洞里。

三叔不知道自己转了多少个圈,只觉得五脏六腑全部都被甩到一边了,突然一头就撞到什么坚硬的东西上,幸好头盔结实,他蹬了几下,猛一抬头,竟然发现自己出了水。

其他人都和他差不多时间,有几个女生都在头盔里吐了,那恶心就别提了(那歌怎么唱的,最恶心的不是看到腐尸,而是淹死在自己的呕吐物里……),几个身体素质比较好的忙扶住他们,不让他们沉下去。

三叔也一把抱住文锦,他用探照灯一照,发现似乎已经达到了冥殿,拿出防风打火机打火,火能烧起来,有氧气。于是做了个OK的手势示意空气没问题,他们几个把沉重的头盔一掀,刚吸了第一口气,几个人同时叫道:“好香啊!”

墓室里一股非常好闻的香气,很淡但是很提神,也不知道是什么东西发出的香味,三叔遇到过奇臭无比的墓室无数,这有香气的还是第一次,不由纳闷,他用探灯一扫,发现这个墓室并不是主墓,可能是个耳室,因为里面没有棺材,只有一排排的瓷器陪葬品,这些东西应该是墓主人生前用过的,而他们现在,就在那耳室中间的一个圆形喷泉眼里,三叔又看了看这里的装饰,越看越疑惑。墙壁上都是壁画,因为有水汽,被腐蚀得很厉害,他只能隐约看到,壁画上画的,好像全是人的影子。

这些类似影子的图案什么姿势的都有,长的,矮的,胖的,走路的,跳舞的,每一个都非常逼真,好像是真人影印上去的一样,但是所有的影子都很怪,它们的肚子都非常大,好像孕妇一样,文锦在壁画研究方面造诣很高,但是她也看不出一个所以然。

倒是那个李四地,看到这些壁画,吓得脸都发青了,大叫起来:“海鬼!这里有海鬼!这个是个海鬼墓。”

三叔想起刚才看到的那个怪物,心说难道那个东西就是海鬼?他自己不敢肯定,现在贸贸然把这个提出来,可能会引起恐慌,他决定暂时保密。

那李四地一直在那里大叫,因为他口音很重,他们都听成海龟,一群人哄堂大笑,弄得李四地哭也不是,跟着笑也不是,三叔看了看表,让他们都出了水,有几个胆子大的已经往耳室边上那门走去。那门不高,应该是通到甬道里去的,三叔一把他们拉回来,说:“现在我们一没有考古的设备,二没有救护的准备,你们给我老实地待在这里,哪里也不准去。这里面的墓道里不知道有没有机关,这一个小时我们是来这里避难的,大家要怀感激之心,懂不懂?”

这帮小子虽然不甘心也没有办法,只好猫在耳室里研究那些瓷器,三叔一看,就知道这些个是明初的东西,他诧异难道这里真的是沈万三那一个宗族的墓穴?

不过他古董看得太多,没什么兴致,眼下倒是比较担心这空间的空气够不够用。他又核对了一下人数,这下子对了,他松了口气,这几天他实在是累得够戗,也没好好休息,现在正好打个盹。

他靠墙坐了下来,文锦靠在他肩膀上,亲了他一下,算是奖励他这次的出色表现,三叔一下魂都飞了,本来还被这帮小子搞得一肚子怨气,现在看到文锦甜甜的一笑,觉得值,真他妈值。让他再倒着来一遍他都肯。

他们休息了一会儿,潜过水人都知道,如果没有长时间的水下操作经验,一次潜水的是非常消耗体力的,三叔虽然体力不错,但是和那些人比起来身体还没有适应,现在身体放松下来,竟然开始打哈欠,又加上那香气好像有让人宁神的效果,一下子他就觉得非常非常的困,他迷迷糊糊对文锦说:“我睡一会儿,如果时间到了就叫我一下。”

那种困乏似乎不正常,但三叔已经来不及去思考,只朦胧地看到文锦温顺地点了点头,他鼻子里都是淡淡的香气,不知道是文锦头发上的体香还是古墓特有的那种味道,总之他几乎就在瞬间,马上就进入了睡眠。

\chapter{老照片}

思绪回到现在,我已经完全被他的故事吸引过去,只觉得自己就在古墓里,怀里就是文锦的温香暖玉。三叔咳嗽了一声,我一楞,突然发现自己抱着个枕头,心中大窘,心说怎么可以对三叔的女人产生幻想,忙脸通红地问:“你怎么不说了,最后怎么样了?”

三叔苦笑一声:“没有什么可说的了,故事到这里已经结束了,我到现在还想不明白,我睡着的这段时间里,古墓里到底发生了什么事情。”他嘴唇颤抖着,“我不知道睡了多久,等我醒过来的时候,我发现那耳室里,只剩下了我一个人,其他人都不知道到哪里去了。我以为他们趁我不在跑到主墓室里去了,心里很火,因为文锦一向很听我的话,这次却和他们一起胡闹,我就想追过去。”

他掏出一只烟含在嘴巴里,脸色有点难看:“这个时候,我看到那墙上的门,竟然不见了!我转头一看,马上就发现,这里并不是我睡着的时候待的耳室,而是另一个陌生的地方,在我身后,竟然放着一只金丝楠木棺。”

我笑道:“以三叔您老人家的魄力,肯定是毫不犹豫,直接把这棺材板给掀了,把里面的好东西全部都倒出来。”

三叔骂了一声:“屁,我告诉你,我那个时候吓得屁滚尿流,棺材我是见多了,但是那只棺材里不停地有水冒出来,一拨一拨的,他娘的好像有东西在里面洗澡,我想起那李四地说的海鬼墓,你知道粽子我不怕,但是海鬼还是头一糟,吓得我几乎要尿裤子,我又担心着文锦,大喊了几声,没人应我,这个时候那棺材板就突然翻了一下。”

三叔说到这里表情很古怪,他接着说:“我那时候想也没想,看那头盔还在手上,直接一套就跳到那泉眼里去了。然后我就逃出来了。”

我一听忙说:“不对啊,那房间不是变了吗?怎么那泉眼还在?”

三叔脸一绿,结结巴巴说:“在,当然在,就那泉眼在。你他娘的别打岔!我还没说完呢,”他定了定神,继续说,“我也不管什么海啸不海啸了,找到了那个盗洞我就游了出去,我一看,老大的太阳挂在天上,也不知道是什么时候,我钻出水面,瞅见不远处好几艘大船,看样子是来捞我们的,我游回到船上去,一问时间,他娘的竟然已经是第二天的中午了。你说我在这墓里就打了一会盹,怎么就过了一天了呢?”

我盯着三叔看,太假太假,最后他肯定还碰到什么决定性的事情,不知道他为什么不肯告诉我,这老家伙最后到底在那墓里干了什么?他妈的,又不能逼他,看他说话闪闪缩缩的样子,搞得我又心痒痒。

我看他不说话了,心里很担心文锦,问他:“其他人了?他们都没出来吗?”

三叔懊恼地拍了一下大腿,“我上了船,不知道为什么没说几句就晕了过去,后来送到海南的医院,昏迷了整整一个星期,等我想回去找他们的时候,我已经找不到那个当初带我们去那个地方的船老大了,在海上,如果你不知道那个地方确切的坐标,你根本找不到,海面上看起来全是一样的。”他停了停,“我后来去问海事管理局,还有他们的研究所,发现这些小青年都失踪了,文锦也和他们一起,快二十年了,到现在一点消息都没有,我真的是一点也搞不懂那个墓是怎么回事,怎么可能平白无故人就不见了呢?”他用力一敲桌子,眼圈一红,“我他娘的后悔,那个时候逞什么能啊,如果我不去倒那个海斗,这一群人现在说不定都孙子都有了!还有文锦,我真是对不起她。”

我看三叔一把鼻涕一把泪,从没见过他这样,也不知道怎么办好,他拿起那条蛇眉铜鱼,说:“我最后想了很久,想为什么只有我能出来,其他人出不来,我和他们唯一的不同,就是我身上有这个东西。”

我看了看那鱼,心想:“如果鲁殇王也倒过海斗,他手上也有这么一条蛇眉铜鱼,是不是可以说鲁王宫和海里的那个沉船墓有关系呢?”可是一想,不对啊,两个墓差了这么长时间,一个是战国,一个是明初,打死都搭不上关系。这之间的奥秘,我怎么想也没有头绪。

三叔说完这些后,思绪有点混乱,他躺了一下,我想他刚刚又经历了一次痛苦的回忆,应该让他平静一下,没想到他突然坐了起来,转过头,对我说:“大侄子,我刚才突然想起一件事情。”

我看他脸色发白,心说你又想起了什么可怕的事情来了,他挠挠头,说:“一起和我进海斗的那几个小子里,有一个人,好像长的和那闷声不响的小哥很像!”

我一听,头皮麻了一下,说:“你不会记错吧,他那个时候肯定还只有丁点大!”

三叔仔细地回忆,眉头越皱越紧,最后说:“时间这么长了,我不能百分之百肯定,但是我还有那个时候的合照,是我们出海前拍的,我让家里给我扫描过来就行了。”

说的不如做的快,三叔一个电话吩咐下去,五分钟后,一封email就发到,三叔刚打开,我就浑身一凉,照片是黑白的,他们十个人,前面是蹲着,第二排是站着,我看到蹲在第一排中间的就是年轻时候的三叔,而他后面站的,赫然就是那个闷油瓶子!

我一身的白毛汗,还以为自己看错了,又看了一遍,果然是他,那眼神,那表情完全一样,顿时手都有点发抖,三叔看了看我,十分的疑惑,他一句话卡在喉咙里卡了半天,终于问了出来:“为……什……什么他二十年来一点都没老?”他刚说完这句话,突然好像醒悟的样子,大叫,“我明白了!我明白了!”

我看他好像疯了一样,一时间不知所措,只见他拿起他的行李就往外走,我想拉他,却被他一把甩开,回头说:“你在这里守着潘子,我要马上再去一次西沙!”说着头也不回地跑了出去。

\chapter{海南}

三叔十几岁出来跑江湖,破事情见多了,一般做事情都要打算来打算去的,像上次倒个斗都准备了很多东西,我有时候还觉得他过于谨慎,像上次那一大堆装备,百分之八十都没用上,没想这次这样毛躁,就随便拎了箱子就跑了,我看拦也拦不住他,就喊了一嗓子:“你自己当心点!”他嗷了一声算回答,就跑进电梯了。

正巧一宾馆洗脚中心的服务员上来和我结账单,看到这情景,笑着说:“你这叔叔怎么比你这侄子还毛躁,都倒过来了,还得你着紧他。”我也没办法解释,只好笑笑接过账单,一看,脸不由一黑,竟然要四千多,不由暗骂:娘的,这老小子昨天又他妈的下去搞那些弄不清楚的事情了。

看这账单,我有点发愁,这几天没少花钱,本来三叔那老小子口袋还是很充实,不过这一路逃出来,钱花得像流水一样,又给那烧了林子的村子里捐了点,身边的现金都用得差不多了,他出门习惯都不带卡,说是老派作风,这几天厚着脸皮在用我的钱,还说让他公司再给他转点,转了再还我,现在他抖抖屁股跑掉了,我就想起这个事情来了,心说该不是知道我也快没钱了,跑路了吧。

我心里很不痛快,拿出钱包一看,心就一凉,我已经习惯用快钱,也没太留心,钱包里竟然只有几张票子了,潘子现在是深度昏迷,不知道什么时候能醒,虽然那医生说没什么大的隐患,主要看他身体的恢复状况,我盘算着十天半个月我也别指望走,这潘子又孤苦伶仃的一个人,找人替我是不可能了,这么点钱肯定不够花销的。

最麻烦是现在一张四位数的账单已经横在面前,这一关都有点难过。我不好意思地笑笑,说我现金不够,要不等一下取了给他送过去。他见我这几天付钱也爽快,笑笑:“没事,明天也没事情,那您先忙您的。”

他一走我就毛了,想到的事情更多,他娘的潘子在医院里医药费每天都得四位数,这老小子这么一走我到哪里找钱给他垫去,又不能给老头子打电话,打了估计得给他骂死,这几年生意搞的这么惨淡他已经对我很有意见了,现在还学最不争气的三叔去倒斗,算了算了。

我回到房间,正烦着呢,突然看到那金缕玉棺套还躺在包里,三叔对这东西是爱护得不得了,还用油纸包了四五层,我看着突然产生一个比较冲动的念头,心说这十几天看来要好好打算一下,天天在这里吃了睡长膘然后打白条也不是办法。要不就找个古玩市场把这东西卖了,然后整点钱整个济南都兜一圈,也不算浪费时间。

想到这里觉得非常有道理,我本来就是抱着出来旅游的态度来这里的,现在搞得就像在查X档案一样,何必呢,而且现在,这事情还不能晃悠着办,不然我被人赶出来事小,潘子给人断了药可就麻烦了,看现在天还没黑,得赶紧办掉。

我想着下到大堂去问服务员,这儿哪里有倒腾古玩的地方,那服务员非常热心,直接陪我下楼,还帮我叫了个的士。上了车后我就和师傅说哪里古玩多去哪里,那师傅答应了一声就把我送到英雄山市场,我一看,这地方还有点花头在里面。

我一路上听那的哥狂侃,他说这里是比较大的古玩和书法制品的集中地,人很多,比较嘈杂,不过假货居多,没事情在这里可以和那些老板聊聊,吹吹牛皮,他们也乐意。

我背着那死沉的玉棺套就下了车,寻思着找一个大点儿的门面,这东西不是一般人能买得起的,那些大店必然和一些比较大的客人有联系,可以托他介绍,给他抽个百分之二的佣就行了,这一套我也是老行家,没人能蒙我。我在回来的路上和三叔讨论过这东西的价值,三叔说也就是百来万,这个东西是有价无市,一是很难有人肯买这么贵的东西,除非是老外,可这个东西又太大了,大件的东西本来就比小东西难一点,他估计着,如果真有人想买,他八十来万也肯松手。

有他这些话在这里我也心里有底,就在那里东张西望,没走几步,突然就瞄见一个铺子里,放着一只青铜的香炉,上面有一些铭刻的人物造型,我一看就一个激灵,那上面的人,一个个都大着个肚子,和三叔提到的海斗壁画很像,我俯下头想看仔细点,这个时候那老板就出来了,说:“哟嘿,您挺识货,我这铺子就这东西值钱。”

我一听他的口音,还是个京片子,就问他:“这上面刻的是什么啊?怎么这么怪,看这样子该不是海南来的吧?”

那人一听,表情一变,忙把我往他铺子里让,还说:“今天真碰到行家了,这东西放在这儿有年头了,您还是第一个看出苗头来的,不错,这的确是海南的。”

做古玩生意的,嘴巴甜是肯定的,我看他的表情,倒不知道他现在说的是不是真心话,还是单纯想把这东西卖给我,我手头上的资料不多,装老手肯定会露馅,就说:“不是行家不是行家,我是在海南看到过这东西,心里觉得奇怪,这东西叫什么我都不知道。”

那人请我坐下,端出一杯茶,说:“那您是谦虚了,不过您要真不知道也不要紧,我告诉您,这香炉上雕的,是种鬼,他们都叫这东西‘禁婆’,这东西的来历就说来话长,你要真有兴趣,我就给你说说?”

我一看有戏,忙装做很想买的样子,点点头,他给我做了等等的手势,把那香炉从橱窗里拿出来,放到茶几上,我一下子就闻到一股奇特的香味传了过来,不由惊讶,他嘿嘿一笑:“这个香气很特别吧?”

我问:“什么香料在里面?”

他把香炉盖子一打开,我看见有一块小小的黑色石头,我一愣,他得意地一笑:“这块就是禁婆的骨头,这香味,叫做骨香。可是个好东西,你睡觉的时候放在边上,包你睡得舒坦。”

我突然就觉得有点恶心,问:“这禁婆到底是什么东西?闻她的骨头来睡觉,太缺德了吧。”

他笑笑说:“禁婆是一个很大的概念,就相当于一个不好的东西的总称呼,那里的人,生了病或是受了伤,都说是禁婆害的,你要说她是什么东西还真不好形容,实在要说的话,可以说她是一个恶鬼。”

“哦,那这就是她的骨头?”我皱了眉头问,“这东西哪里来的?看这盖子上的海屎,好像是个海货啊。”

那人呵呵一笑:“您还说你不是行家,不错,这东西是一个渔民一个网撒下去捞上来的,不过物以稀为贵,虽然有点海屎在上面,这价钱也可是不便宜。”

我身上钱根本不够,于是叹了口气说:“可惜,我这个人好全品,这海货我是不要的,你要真想卖,不如把里面这块骨香卖给我?”

那人脸色一变,赔笑道:“那怎么成,你把这骨香买走了,我找谁买这香炉去啊?”

我看看这东西上面略有灰,知道肯定放了很久没卖出去,这种东西太冷门了,买下来不好转手,一般买来投资的人都不喜欢,乱世黄金,盛世古董,卖不出去的东西,店主自然也不会再花心思打理,我摇摇头,反正这东西我买了也没什么用,等一下我把那棺套拿出来给他一看,他要是能联系到个买主,这东西让他送给我也成,想着一笑说:“那行,咱先不谈这个,我给你看样东西。”

说着就把玉棺套拿上来,露出一个角给他看,这是不是行家,看表现就知道了,他一看脸色就变了,二话不说又把那玉棺套塞回去,然后起身把铺子的卷帘门给拉了下来,把我那杯茶倒了,给我换了另一杯上来,我一闻,操,上等的铁观音啊,看来我算是上了一个档次了。

他擦了擦头上的汗,说:“不知道这位手艺人怎么称呼啊?”

我一看,这人果然不是单纯的古董贩子,反应这样快,一眼就看出这东西是倒出来的,也不由要表示一下,客气地一笑:“敝姓吴,老板怎么称呼?”那人说:“您叫我老海就行了,那吴师傅,你这东西,打算出手,还是让我看看?”

我说:“当然是出手,这东西,放在身边有点烫手。”

他在房间里来回走了几下,问:“全不全?”

我点点头:“一片都不会少你的,刚出锅,还热火着呢。”

他坐来下,轻声说:“那吴师傅,我是个爽快人,我敢说你这东西,这整个英雄山,就我敢收,不过这东西我再正儿八经的和你抬杠也没必要,宝贝是讲不来价格的,你就和我说个心里话,多少肯放,我给你打个电话问问我朋友去。”

我想了一下,心说怎么样也要来个一百万,大奎家里得给个三十万,潘子住院最起码也得二十万,那胖子早就留了话,东西卖了钱给他汇过去,这样一个人也就分个十万多点,想起自己用命搏回来的,不由又觉得太少。不过三叔说了,倒斗就是这样的事情,不然为什么倒了一个又一个,你倒一个斗带出来的东西再珍贵,这没人买还是垃圾,所以太好的东西他都不拿,拿了也卖不掉。

我估计着一百万差不多了,对那老海做了个一的手势,他不由一喜,我一看有点郁闷,难道报低了?他拿起电话,躲到角落里轻声打了个电话,打完后开心得脸都红了,说:“成了!成了!吴师傅你运气好,这东西还真有人等着要,这一百万不高,二百万不低,我给你报了个一百二十万,你看怎么样?”

我一听,心说鬼知道你报了多少,说不定翻了一倍给人家报了过去,不过已经比我预计的多了二十万出来,心里还是很舒服,笑道:“那您那份,还是老规矩?”他笑了笑,说:“不瞒您说,那边已经多预备了点给我,这一百二十万您就收好,看你这一头伤的,这东西倒出来不容易,你要记得我的好,下次有这种东西,就别往别人家问了,直接送我这儿来,你要多少价,我都给你往上抬个百分之二十,要知道,我背后的主顾,可是大大的有钱。别人不敢收的东西,他都敢收。”他看我有点着急的样子,忙说,“您坐一会儿,我给你预备钱去,这一百二十万,别看我这铺子小,账上还不缺,我先垫给您。”

我一听,这口气还真大,俗话说的好,三十六行,古董为王,还真不假,看来这家伙手头上还是有点门道的,忙说:“等等,那这禁婆炉?您要不给我也折个价格?我一并就拿了去。”

那人嘿了一声,甩手道:“这个您喜欢就拿去,算我送您的,不瞒您说,这东西我收来就五块钱,刚才扯那么多那是套您呢。”

三个小时后,我怀揣巨款,心情好到天上去了,回酒店的时候都不想正眼看那门卫,后面还有人议论,这小子是不是中五百万了,你看那眼睛笑得睁不开了。我整理一下钱后,把所有的账先结了,又到医院交了潘子一个月的代护费用,给胖子打了钱,然后郑重地把自己那一份,连同三叔欠我的,全部转到了自己的卡里。心里总算舒坦了。

这接下来的几天我找了个当地的漂亮导游,到我济南各个地方都去兜了一圈,不过我从杭州过来,看人文景看多了,越看兴致越低,后来干脆就去找了个钓厂钓鱼去了,这几天是我活的最安逸的时候,不过人有点贱,这安逸了,竟然开始怀念倒斗时的那种刺激了。

废话不多讲,这样糜烂的生活大概过了有个把星期,我从钓厂回来,刚一进门,就听见电话在响,我在这个旅馆的电话只有三叔知道,以为他的事情弄出眉目了,忙接起来一听,对方竟然是一个陌生的男人,他第一句话就是:“你认识不认识一个叫吴三省的人?”

我听他的语气比较急,忙回道:“认识,怎么说?”

那人说道:“他失踪了。”

我一听就呆了,忙问:“那个,什么叫确认失踪?”

那人说道:“他所在的船只与陆地失去联系已经十天了,你和他是什么关系?”

我说:“我是他侄子。”

他说道:“那你能不能尽快赶到海南?”

\chapter{女人}

对方是一家规模很大的国际海洋资源开发公司,所谓海洋资源开发,其实就是根据对现存的各种航线信息和史料记载进行分析,来推断某些沉船的位置,并打捞沉船物资。

这种行为很像职业的海洋盗墓者,但是其行为又是合法的,因为在公海中发现的失事船只的资源,有相当比例可以为寻得者合法继承。当然其资源是否来自公海,根本无法考证。

这样的企业分两种,一种是打捞现代沉船,将尚未完全腐烂的船身解体拍卖,或者将获得的资源出售;二是打捞古代的沉船,将上面的古董出售给收藏家或博物馆。

这家企业属于后者,即以古代沉船为主要目标,它有很多考古顾问,每一个工程都需要大量考古和海洋方面的专家花两年或三年的时间来完成,而他们的所得也非常丰厚,所以拥有大量的先进仪器和船只。

而三叔为了尽快找到那个海底墓穴,以担保的形式,向这个公司借用了设备与人员,并以这个公司的名义,派出了一支五人的临时考察队。这本来是一桩很合算的买卖。没想到船开出去才五天前,他们后勤部门与考察队船只的联系就中断了。

他们一直等待了四十八小时,最后只有派人到失踪的海域搜索,结果一无所获。而失踪前三个小时最后确定的信息是,三叔和其他两个考察员,已经进入了海底古墓。

他们来找我的原因,是三叔在临出发前,和他们说过,如果出现意外,可以打这个电话找我帮忙。

那个人在电话里说:“现在我们还无法确认古墓里面的情况,不知道这三个人生死,所以我们准备再组织一支队伍,进去看看,因为我们这里大部分都是纸上谈兵做理论的,我们希望有一个经验丰富的向导。最低限度,必须帮他们找到墓穴的确切位置。”

我听到他把向导这两个字说得非常重,似乎是在暗示我他知道我的真正身份,不由有点保留,但是这件事情事关重大,我必然要亲自去一次,只好行缓兵之计道:“你们那边具体什么情况我也不清楚,要不等我过来再说。”

对方说:“好的,请你越快越好。”

我挂掉电话,决定马上就出发,匆忙收拾了一下东西,便让酒店给我预定最早去海口的班机票。我去过一次西沙,知道如果要到真正西沙群岛的范围,至少要飞机、车、船三种交通工具一起交替上。

接下来的十几个小时,我马不停蹄地赶路,也没时间胡思乱想,只是不停地祈祷,事情不要向最坏的地方发展。第二天中午,我的飞机抵达海口,他们公司已经派了一辆车过来接我。

来接我的人姓刘,他对我说,这次他们公司高层非常重视这件事情,因为与三叔一起失踪的一个人,是一个高层的公子,而这次的项目又是在南中国海实施的,不能张扬,所以要寻找民间人士。

我一开始还没明白民间人士是怎么一回事,后来才想明白,不觉得好笑,不过这个刘师傅只是个普通司机,也不知道更多的细节,我和他聊了一会儿,却发现车竟然开到码头上了。

我莫名其妙,这个时候一个中年人走过来,问:“是不是吴先生?”

我点点头,他打开车门,说:“请跟我来,船马上就要开了。”

我十分迷惑,说道:“船,开什么船?不是送我去宾馆吗?”

他摇摇头说:“时间太紧急了,我们必须在七个小时内赶到那个地方,在十个小时内完成这个行动,不然那里就会进入半个月的风季,到时候没有海上支援,情况更麻烦。”

我一听他们自作主张,就觉得有点不舒服,不过事关三叔的老命,我也没别的选择,只好嘟囔了一声,背起行李跟他走,到了码头,他指了指一只非常老旧的七吨铁皮渔船说:“就是这里,我们这次的配船。”

我以为他在开玩笑,他无奈地解释道:“没有办法,我们在那一带的大规模搜索已经引起边防的注意了,不得不做一下伪装,你放心,船上的设备已经是最先进的了,航行绝对没有问题。”

说着船上就有人把我的行李接了过去,他用本地话和船上的渔民说了几句,然后和我握了握手说:“船上的一些事物由宁小姐负责,她就在你后面,祝你好运!”

他们做事情的效率太高,我还没有跟上节奏,他已经快步的离开了,我转过头,正看见一个穿紧身潜水服的年轻短发女人打量着我,她看我好像很无辜地站在那里,不由失笑,招了招手说道:“跟我来。”

\chapter{变天了}

我跟她进了船仓,里面放满了一堆一堆的东西,几乎连放脚的地方也没有,看来他们准备地十分急促,所有的物资还没有来得及搬进货仓,就胡乱的扔在入口处。我边走边观察,发现主要是潜水设备、大型仪器、食物、绳子,其中氧气瓶又占了大多数。

我们穿过这些货物,到了连通着机械室的后仓,这里横七竖八的摆着几张板床,上面铺着已经油得发黑的毯子。其中一张床上坐着一个有点发福和秃顶的中年人,满脸油光发亮的,看见我进来,很神经质地站起来和我握手,说道:“幸会,幸会,鄙姓张。”

我对这人第一印象不好,不过出于礼貌,我还是和他握了一下,他那一双手倒是非常有力,看样子以前也从事过体力劳动。

宁小姐向我介绍说:“张先生是我们公司特别请来的顾问,是专门研究明朝地宫的专家,这次主要负责这个海底地宫的分析。”

我对正统的考古界并无太多兴趣,也没有听说过他的名字,不过看他面露得意之色,只好说道:“久仰。”

那秃头很夸张地摆了摆手,说道:“专家不敢当,大家研究研究而已,只不过我运气比较好,碰巧发表了几篇论文,小小成就,不提也罢。”

我从来没见人这样说话的,都不知道怎么接他的话,只好说道:“您过谦了。”

他很吃这一套,又用力地握了握我的手,问我:“不知道吴先生这次是作为什么身份被请来的?恕我直言,似乎吴先生研究的学科比较冷门,或者是我孤陋寡闻了,我还从来没在考古杂志上见到过吴先生的大号。”

这几句话分明是想贬低我,也不知道是有心还是无心的,我这个人脾气不好,听到这些几乎要发作,可一想到自己才上船不久,环境还不熟悉,只好压住火气,没好气地说道:“我专攻挖土的。”

我的语气已经很不善了,可他竟然没听出来,哦了一声:“您是建筑师?难怪,原来不是我们一个圈子内的,不过我们也算是半个同行,你盖活人的房子,我研究死人的房子,我们还是有交集的嘛。”

我一听哭笑不得了,看来这人说话虽然不靠谱,但是也不算那种阳奉阴违的人,拍了拍他说道:“我不是建筑师,我是挖掘工人,你研究的死人房子,要我先挖出来才行。”

说了这话我就有点后悔,我本来还没答应他们要亲自下斗去,现在那边的情况不明,凡事还要等我实际看了再说,想着又补充道:“不过到时候挖不挖,还要看情况,如果情况不允许,想挖都挖不了。”

他没听出我的弦外之音,还一个劲地给我递名片,说什么多一个朋友多一条路,以后去北方有什么事情可以找他帮忙,我看他和我见面不到两分钟就搞得十几年交情一样,估计再聊下去就要去结拜了,忙岔开话题,向那女人打听出事海域的情况。

那个女的相当干练,她把几个事情一列,我就知道了个大概。

原来三叔当时也无法确定那个海底墓穴具体方位,他只找出了四个有可能的区域,一个一个去找,后来肯定是给他们找到了,但是失踪船的最后一次报告比较简短,并没有提到他们最后确认的海域是哪一个,所以现在我们也得一个一个找过来。

他们的计划是从离得最近的一个仙女礁开始找起,然后到永兴岛补给一些物资,再到七连屿附近的其他三个海域去,中途停留不超过半个小时。至于寻找的办法,西沙的海水非常清澈,光线好的情况下目视入水可达三十多米深,而且海地水流活动平凡,没有流动性很强的海沙,所以几天前的盗洞,应该不会被掩盖住。

这艘船的渔老大本身对于那几片海域也非常熟悉,我们这些外行人在水面上看的水底都是一个样子的,但是在他们眼里每片水域的水底都有自己的特色,只要海底发生一些地势的变化,他就能看出来。

我从那个女人的谈话中发现,她对于水底的这三个人仍旧生存有着很大的信心,不知道这种盲目的信心是从哪里来的,当然,我也希望承她贵言,三叔在海斗中一切平安。

那张秃头看我和那女人谈得投机,把他一个人撂在一边,大概有点不爽,自顾自睡觉去了,我看这个人年纪已经到中年,脾气还像小孩子一样,不由好笑,真是一百年不死都有新闻,不知道相处下去会不会融洽。

想着,船一震,后面的渔老大起锚开船了,船的晃动开始剧烈起来,因为是老旧的船,不仅仅是左右的摇晃,还有一种不规则的前后摇摆,好像置身在摇篮里一样。我十几个小时的舟车劳顿,被这么一晃,倦意袭来,就打起了哈欠。那女人十分知趣,就让我自己好好休息,我也老实不客气,的确是累了,躺下就睡着了。

我醒过来的时候,船已经行驶到海中央,我透过窗向外面望去,发现才一个囫囵觉的工夫,已经变天了,整个大海好像一下子变成墨绿色一样,太阳消失在大片的乌云里,光线透过那些云块的缝隙照射下来,在天上形成了一幅巨大的金丝版画,同时也在海面上撒下一片金鳞,上下交相辉映,十分壮观。

不过好景不长,乌云很快便连成一体,挡住了所有的阳光,大海一下子变成了骇人的黑色,海浪翻滚起来,船随浪摆,当我们在浪谷的时候,海水是在船舷的上面,就像即将被巨浪吞食一样,非常恐怖。

我看到船夫们紧张地跑来跑去,加固着固定物资的网绳,虽然非常急促,但是船老大的脸上并没有畏惧的神情。

我在城市里待惯了,看到这情景只觉得兴奋异常,想去甲板帮忙,上去之后才知道根本不是想的那样,在现在这种情况要在甲板上站稳脚跟,不是反应快就可以,你必须对海浪和船非常熟悉,知道这次倾斜之后下次倾斜是在什么时候,事先做好准备。我显然没有这么高的水平,走了几步后,不得不抱住一块突出的铁环。

这个时候,有几个船员好像看见了什么东西,开始叫起来,我听不懂闽南话,顺着他们的手指看去,隐隐约约看到船的左侧,高起的海浪后面,好像有什么东西。

因为距离比较远,看不太清楚,只觉有可能是一艘船,这个时候那个女人从我身后走过,我就问她这些人在叫唤什么?

她身上头发湿湿的,被风吹得乱甩,仔细听了一下说:“他们好像看到一艘船。”

船老大走到我们身边,用半生不熟的普通话说:“那边好像有艘船出了事故,按照规定,我们必须要过去看看。”

这样做当然无可厚非,那女人点点头,船老大对他那些伙计用本地话很快发布了一系列指令,马上船就一个满舵转了方向,向左边开去。

风浪中的海就像丘陵,每一个浪头都是一座山,而我们的船迎着浪头冲了过去,尔后破浪而过,每破一次船上的人就洗一次海水浴,全身湿了不知道多少次,我从来没有感觉这么亢奋过,忍不住都想号叫起来。

我们一连翻过十几个浪头,终于可以看清楚那东西的大概轮廓了。

这个时候,我就听到船老大惊恐地大叫了一声,随即好几个船员都惊慌了起来,我忙问那女人又出了什么事情,她一听之下突然脸色大变,一把抓住我的手说:“千万别回头看,那是条鬼船!”

\chapter{鬼船}

我看到所有的人都慌张地把头转过去,不去看那只破船,也不知道发生了什么事情,这种形势不明了的情况下,我也不敢自作主张,忙学他们的样子背过身子,那女的发抖着对我说:“不管发生什么情况,都不要转头过去。就算有什么东西碰你,你也要当不知道。”

我一听,冷汗就下来了,问:“你别吓唬我,这里会有什么东西碰我?”

她白了我一眼,轻声说:“你不信都没用,等一下就知道了,现在快把头转过去!”

我看她说得这么邪,又看到其他船员那种惶恐的样子,好像不是在吓唬我,轻声问:“你总得告诉我,那到底是什么东西?”

那女做了个不要说话的手势,说:“闭嘴,这是冤死鬼来索命来了。”

她越这样说,我越觉得害怕,脖子不由自主的就想转过去看看,忙捏了自己大腿一把,把那脖子上的肌肉绷得就像打了石膏一样。

那船在风浪里摇得很厉害,船上的甲板还在吱吱作响,听上去快散架了,我手抓住船舷上的两个铁环,屁股死死顶住,但是上半身还是不停的在晃动,偏偏脖子又不能动,我就像一个不倒翁一样晃来晃去,有几次几乎被甩得要脱手了。

这个时候,我已经可以听到那所谓鬼船上传来的声音,咯吱咯吱的,好像是有人在甲板上走。我身上已经全被海水打湿了,加上自己的冷汗,非常不舒服,忍不住轻声问那个女人:“怎么好像有人在甲板上走,你刚才有没有看错。”

那女人很害怕,努了努嘴巴,我顺她嘴巴看去,原来船仓的玻璃上,清晰地映出了身后的情况。一艘和我们规模差不多的渔船,在我们的身后摇曳,离我们越来越近,我看得也越来越清楚,很快我就看见,那船上面有一层白色的棉花一样的海锈,看厚度,肯定在海里泡过几十年以上了。真想不通这样的船怎么还可能浮在海面上,而且上面还亮着个灯。

那些小说里出现的幽灵船,都是那种非常破败的,但是基本上还是能航行的船,但是这一艘肯定已经完全报废了,看样子就像从海底开上来的一样。我脑子转得很快,回忆了一下我看过的关于幽灵船的报道,好像都没有提到这种样子的船。

那船越开越近,我隐约觉得苗头不对,轻声说:“小姐,好像不是办法,那鬼船看样子打算撞过来啊。你要不叫渔老大开足马力溜吧?”

那女的也有点害怕,头发全贴在脸上,也没想到去拨一下,她说:“要逃的时候渔老大自己会逃,我们两艘吨位差不多,它撞过来也不怕。你拉紧了可别掉下去。”

她那说话的语气,我也听不出是在提醒我还是在讽刺我,说:“就怕他等一下跳船跑了,你可拿他没办法。”

“你少在这里挑拨离间,这渔船就是他们渔民的命,他死也不会离开船的。”那女的有点火起来,“你要是废话再多我就把你推下去!”

我一听这么凶,也不好再说话,集中注意力看着那玻璃上的鬼船,我估计以它的速度,撞击的时候也不会造成多大的震动(后来知道这是sb想法),心里渐渐平静下来。

那船越来越近,我可以清晰地看到,船上什么东西都没有,我本来还以为会看到一些恐怖的景象,不由松了口气,那船靠得很快,几乎就要贴到我们的船了,我眼睛一闭,一咬牙,准备硬顶一下那撞击。

这是一个瞬间,突然,后面的声音就消失了,我等了有十几秒,估计就算它想撞十次都撞完了,可是还没有什么动静传过来,不由奇怪,这个时候,我又听到那甲板咯吱咯吱的声音从背后传出来,我心里有点发慌,偷偷眯开一只眼睛,去看船仓的玻璃,那只鬼船已经并排和我们靠在了一起,我的背后什么都没有。

我松了口气,看了看边上,只见我旁边那个女的和我一样看着那船仓的玻璃,已经吓得呆住了,我觉得好像有什么不对劲,仔细一看,只见她的肩膀上,搭着两只干枯的手。

\chapter{枯手}

那两只干枯的手,显然是人的手,已经收缩成枯柴状,贴在那的女的身上,这样的情景,就算看着,也觉得毛骨悚然,我不知道那女的现在是什么感觉,只觉得我的背上不停地冒冷汗。

那两只手也没有进一步的行动,只是无力地垂在那里,好像是她衣服上的装饰一样,我想看看那手是从哪里伸出来的,顺着手臂望上去,但是她的头发太乱了,蓬松开来,看不清楚。

女的显然已经非常恐惧,浑身抖得厉害,如果是普通的女人,恐怕早就已经晕过去了,我看她的身子发软,估计也已经到了极限。

那船老大背对着我们跪着,一边磕头一边不知道念了什么,我听不懂他们当地的方言,但是也可以猜出来,他可能在进行某种仪式,估计是在求妈祖保佑。他念了几声,就拿出两个奇怪的半圆木片,往甲板上扔,好像是在求签一样,他扔了一次,看了看结果,又叩了几个头,拿起来再投。我看到他浑身开始发起抖了,大概问出来的结果不太理想。

我对这种一向是不相信的,但是看到船老大这么虔诚的样子,心里有点担心,这些人非常讲究这一套东西,如果那些求签的结果说我是一个恶鬼,估计他们会毫不犹豫地把我扔到海里去。

就在这个时候,突然那个女的惊呼一声,整个人突然往后一缩,也不知道是没抓稳还是被那鬼手拖了一下,竟然一下子就翻进鬼船里,这下子不得了,那鬼船一下子就漂了开去。我一看不好,也不管什么回头不回头了,转身就想跳过去救她,那船老大从后面冲过来一把把我抱住,说:“没办法了!掉到鬼船里已经救不回来了,不要去送死!”

那船老大力气很大,我甩不开他,其他那些人不知道着了什么魔一样,竟然还是不敢转过头去,我心里只骂,这个时候,那个张秃头不知道从哪里跑出来,扯起船上的锚,用力一甩,把锚甩到鬼船上,钩住了船舷。那鬼船游得飞快,一下子就把锚缆拉成直线,我们的船一震,硬生生被扯了过去。

那个船老大吓得魂不附体,抽出把刀就去砍那缆绳,被那张秃一拳打翻在地上,其他船员毛了,一个个扑了上来,那张秃竟然拔出一把手枪,一把把船老大架住,大叫:“别动,不然我杀了他!”

那几个船员没见过这种场面,这一嗓子竟然没人敢上来了,那张秃又对我说道:“小吴,我已经把他们控制住了,你快去救人!”

我张大嘴巴,怀疑自己有没有听错,这么惊涛骇浪,难道要我游过去?他还想当然地瞪了我一眼,指了指那根缆绳,吼道:“快去!年轻人要勇敢点!”

我摇摇头,这太可笑了,我体育本来是就弱项,游泳过去基本上就是送死,如果爬那根缆绳,估计就算我爬得到也是剩下一口气,还怎么救人。

这个时候,我听到那个女的在鬼船上尖叫起来,她拼命想爬到那根缆绳上面,但是好像被什么东西拖住了一样,没办法前进,只好用两只手死死抓住船舷,朝我大叫:“吴先生!救救我!”

我听得心头一晃,猛拍了自己一个巴掌,大骂:“吴邪啊吴邪,你他娘的还是不是男人!”

这一巴掌我也不知道是把自己拍醒了还是拍懵了,突然就血气上涌,一咬牙大叫:“死就死了!”

我深吸一口气,先拿起一边的游泳眼镜带上,然后脱掉鞋子,走到船舷边上,笨拙地抓住那根绷得很紧的缆绳,只见前面是惊涛骇浪,那绳子还不时淹到水里去。

那根缆绳大概就十二米长,结实程度绝对够,如果手脚快一点,也不是很危险,主要的麻烦还是在绳子上被浪打下去,想到这里,我的心里也稍微活动了一下。

我从小到大从来没遇到这么要下决心的事情,在那船舷上屁股拱来拱去好久,才慢慢爬出去第一步,我按照记忆里电视上那些特种兵的方法,倒挂在绳上,四肢并用,一边爬一边祈祷,可还没等我张开嘴巴,突然一个浪头过来,直接把我淹到水里去了,等我探头出来的时候脸都憋绿了,不过这样一下子,我也对这海浪的力气有了一个了解,心里豁然,估计爬到那边应该没问题。

就这样我在有浪打过来的时候就不动,等出水就爬几步,也不知道过了多久,我已经很靠近那艘鬼船了,这个时候,一个巨大的浪打过来,我整个沉到了水下,这一下子,我几乎被压到了一米多深,人都有点被拍蒙掉了,我憋住呼吸睁开眼睛一看,突然看到了一幅奇特的景象,只见那艘鬼船的船底,有一根长满海锈的链条,很长,离奇的是,那链条末端,有一块奇怪的东西,在很深的水底,看不清楚。

我吐出口气正准备仔细看一看,突然那缆绳一抬,我就出了水了,这一下我在浪尖上,往下一看,看见那个女的面朝上,正在用一个奇怪的姿势往鬼船的船仓里爬,我一看就吓呆了,拖着她前进的,不是她自己的手,而是那两只干枯的鬼手。

我看她一动也不动,好像失去了知觉,别无选择,只好手脚一发力,爬了过去,然后一个翻身摔进鬼船里,倒在甲板上。

\chapter{甲板}

那甲板经过多年海水腐蚀,已经不勘重负,我八十公斤的体重压上,马上发出一声咯吱,似乎就要断裂,但是我也顾不了这么多,忙去看那个女人怎么样了。

她半个身子已经被拖入黑洞洞的船仓,我一看急了,自己身上一没有任何的照明设备,二没有利器在手,一旦被拖进去,生死真的很难料。

我一个打滚翻过去,抓住她的腿,使出吃奶的力气扯了几下,发现那女的纹丝不动,而且她身上穿的是紧身的潜水衣服,不仅没有可以拉的地方,沾上海水还滑得要命,力气只能用上百分之八十。

我一看这样下去,这个女的肯定完蛋,一时间也想不到好的办法,急起来,就扑到那个女的身上,一把抱住她的腰,这样我们两个人的重量加起来最起码有一百三十多公斤,我看这两只蜡杆一样的手怎么拉。

没想到这甲板已经到了临界,我刚压上去,嘎嘣一声整个就塌了,几秒的工夫,我就随着大量腐朽潮湿的木片一起掉进了船仓里,幸好那船底还结实,不然我们就直接掉海里去了。

这一下摔得够戗,我晃晃悠悠坐起来,心里不由苦笑,刚才是拼命不想进船仓,现在反倒进来的这么爽快,这个时候,就听见那女的在下面叫:“快走开,你压死我了!”

我发现自己正坐在她屁股上,忙让开,心说这可好,以前看偶像剧,都是女的坐男人身上,现在反而倒过来了。那女的吃力地撑着腰坐起来,肩膀上的手已经不见了,我一惊,忙问:“那两只鬼手跑哪里去了?”

她一摸肩膀,惊讶道:“我也不知道,一掉到这船上来我就迷迷糊糊的,也不知道什么时候没了。你没看见吗?”

我摇摇头:“刚才掉下来的时候情况太混乱,我也没注意,不过那两只手能拖着个大活人走,肯定不是幻觉,是实在的东西,不可能凭空消失掉,肯定是刚才掉下来的时候撞掉了,你看看你身下有没有。”

这话一出,那女的吓得脸色一白,忙抬起屁股看。可惜身子下面除了木片什么都没有,我说:“可能掉下来的时候被扯掉了,它还抓着那仓口的台阶,你这么突然往下一掉,它来不及撒手,可能还留在上面。”

她点点头,觉得有道理,说:“也不知道它把我拉过来是什么企图,我看我还是要多加小心。”

我们两个人各自查看了一下四周,因为那甲板上破了一个大洞,所以照得还算通透,这船仓的内壁上也有厚厚的白色海锈,几乎把所有的东西都盖在里面,我们剥开一些,可以看到一些一般航行用的物品,不过基本上都已经腐烂得只剩下个形状。

看这船仓的规模结构,应该是七八十年代比较中型的渔船,铁皮的船身,仓室空间很大,中间由木板隔着,应该分成了船员的休息室,船老大的房间,货物仓,我们现在的位置应该在货物仓里,不过看剥出来的东西,这艘船肯定不是在载货途中沉没的。

这船的龙骨应该还没有被完全腐蚀,所以还有一定的续航能力,不然在这么大的浪中,早就被冲得解体了。

那女的看得直摇头,说:“我其实也算是个很了解船的,但是这船的情况太不符合情理了——这么厚的海锈,照理说在海底最起码也该有个十几年了。”

我问:“有没有可能是大的风暴把它从海底卷上来了。”

她回答道:“这样的可能性很少,几十年的沉船,早就应该深深埋在海沙里,就算你用起重机去吊,也很难吊起来,而且它的船身很脆,一不小心就可能被扯散架掉。”

她说的我也想到,但是我还有一点想不通,这船既然当初沉了下去,现在怎么还能浮在水上?就算有人把它捞了上来,它的船体上肯定还有当时遇险的时候留下的破洞,难道这洞还能自己补上不成?

我看这里也瞧不出什么名堂,那两只手也不见了踪影,稍微放下心来,拍了拍身上的木片站起来,招呼那个女的往仓里面走走。那两个仓之间都是用木板隔着,现在基本上已经烂的千疮百孔,我想直接把板子踢掉,那女的阻止我说:“这木板上面贴着甲板,你再用力,恐怕整个甲板都要掉下来。”

我心说,要是整个甲板都掉下来就好了,那光线照进来,心里也不会发慌。

有了鲁王宫的经验,我对于很多事情都有了比较深刻的了解,特别是几次徘徊在生死边缘之后,应变能力加强了不少,所以我在这鬼船上,虽然神经还是绷得很紧,但是没有那种脑子被吓得无法思考的情况。

那木隔板子上还正儿八经地装了一扇门,我不知道是推还是拉的,先试着一拉,那把手就和半块门板一起被拔了出来。我看看那个女的,说:“这不是和拆整块板子差不多?”

她不理我,往那黑漆漆的大门洞里看了一眼,这女的胆子应该算大了,不过我想经历了刚才那种事情,估计她也不敢贸然进去,对她说:“里面光线不够,如果要进去,还是直接再在甲板上开个天窗借点光好,免得进去了,又被什么东西搭上。”

我知道这句话说了肯定有用,果然她一犹豫,我暗笑一声,上前掰了几下,就几乎把整块板子掰了下来,这里面的仓室有一块大的床板架子,是铁做的,所以还在,床板已经烂光了,看陈设应该是那些船工待的地方,我看到这个仓室的角落里,放着一只铁橱,还关得很好,上前去拉了一下,比较松动。

在这种船上面很难找到文字记录,现在的船老大还必须天天写航行日志,那个时候识字的人都不多,所以我也没指望找到什么有用的东西,等我打开那个铁橱,不由吃了一惊,里面竟然有一只老旧的防水袋,我打开袋子,里面掉出一本已经几乎要散架的笔记,我一看,封面上写了几个字:西沙碗礁考古记录。

我翻开扉页,上面很娟秀的几个字——1984年7月,吴三省赠陈文锦

\chapter{三叔的谎言}

看到这几个字,我几乎惊讶得要晕厥过去,吴三省和陈文锦,这不是三叔和文锦的全名吗?难道这笔记本,是他们当年留下来的?但是这种东西怎么会在鬼船上出现呢?

如果说这鬼船沉没之前,船碰巧也有两个人,一个人叫吴三省,一个叫陈文锦,这两个人又碰巧也是做考古工作的,又碰巧也到西沙碗礁来考古,这样的巧合发生的几率,恐怕够我中好几个五百万了。

我想了一下,似乎不用太多考虑,这本笔记本没有别的解释,毫无疑问应该是三叔他们留下的东西,而且,看上面的署名,这本笔记本应该是三叔当年送给文锦,而文锦则用它来作为记录碗礁考古日常进度的日志,笔记本的主人,应该就是文锦。

那这艘鬼船,又和三叔他们当年的考古活动有关,甚至可能就是当时没有按时回来的那只中型渔船。

我稍微思考片刻,不由就觉得无数问号涌现到我的大脑里,开始觉得头痛欲裂起来。

这些事情,其中真正的奥妙,恐怕只有当事人才会知道。我现在所知道的皮毛,全部都是这些事情最表面的东西,似乎还缺少一个把这些都连起来的核心。如果三叔那老狐狸能老老实实地把所有事情告诉我,恐怕我现在已经可以大概知道整件事情的关键所在了。

或者这本笔记里的内容,能给我什么提示,我本来想先把这个东西藏起来,等到没人的时候再看,但是心里强烈的好奇心实在无法忍受,想着反正她迟早会知道这件事情,没有必要搞得这么神秘,也不避忌她,直接就翻看起来。

文锦是个做事情很认真的人,每一天的记录她都用相同的格式,列得清清楚楚,我看到第一页就是他们出发的第一天,7月15日,上面列出了一个名单,我看到领队果然是吴三省,那个闷油瓶叫什么,我想起三叔提过他好像姓张,一找,果然有一个人叫张起灵,难道就是他?

再一翻,前面主要的内容都是找到并确定海斗具体位置的经过,只是比三叔说的更加详细,连绳子的种类,还有推理的过程都写了出来,真的和三叔这个大老粗完全不同。真想不通他们两个人怎么能走到一起。不过这些内容我没有必要再看一遍,直接翻到最后,我一看就傻了。

其实不用看最后的内容,只看最后几条标题的记录,就够我惊讶的了,同时也将三叔那个王八蛋骂了一百遍。

只见她上面写着,7月21日,第一次进海底墓穴。

人员:吴三省。

进度:清理左右耳室和甬道,准备清理后室。

工作:使用气泵对墓室进行换气,准备长时间清理。

出水文物:金丝木双凤雕子棺(婴儿棺)。

备注:出现紧急事件,详细记录待补。

然后下面就只有一条记录,7月23日,第二次进海底墓穴。

人员:全部成员。

进度:无

工作:躲避夏季风暴

出水文物:无

备注:无

原来,三叔在带他们进去之前,自己已经进去过一次,以他的土匪秉性,肯定顺了很多东西出来,他在这里写的只进行了左右耳室和甬道的清理,谁知道他有没有开后室!说不定棺材里的东西他都已经摸过一遍了。这只老狐狸到底第一次进去时做了什么!我不由恨得牙痒。

我粗粗看了一遍,里面应该还有很多有用但不关键的记录,现在没有必要看得这么仔细,我将它收好放回防水袋里,回头看那个女人的反应。谁知道她好像根本没注意我,只是拼命地在剥船长室那块隔板上的海锈。

她动作很迅速,简直不像是在剥而是在砸,那半块板子已经被她清理了出来,我已经看到那些海锈里面包的竟然是钢。她一路剥下去,一直到船身和隔板的连接处,我发现这块隔板四周是和船身焊在一起的,似乎非常结实,而且那上面的门也是钢的,上面有一个汽车方向盘一样的旋转密封锁。

那个女的一边剥还一边在那里自言自语,好像是在说:“不要怕,不要怕,我马上放你出来。”

我听到这话有点不对劲,才发现她有点不正常,只见她利索地把那钢门边上的海锈都清理掉,我一看,那门与框之间,还有一层橡胶。这里面的仓,似乎是密封的。那女人清理完这些以后,就凭命地去转那个旋转密封锁,可是她力气远远不够,这个锁本身就非常沉重,加上里面全是海锈,不是那些力气极大的水手,根本打不开。她用力转了几下,一点反应也没有。

我心里觉得有点不妥,对她说:“里面的东西可能没浸过水,我们还是不把他打开为好,万一里面有个什么怪物,我们身上什么武器都没有,肯定得交代在这里。”

她根本不理我,还是拼命的去转,我摇摇头,这个女人真的是不可理喻,我对她完全失去了好感。

接下来的几分钟,我双手叉腰,看她在那里白费力气,心里觉得也比较解气。这时,她转过身子看着我,我以为她开窍了,谁知道她突然发出一声怪叫,人往后仰去,头发里闪电般伸出两只枯手,抓住那旋转密封锁就开始发力,那怪手力气极大,我马上听到了里面海锈碎裂的声音。

我吓得头皮发麻,几乎就要坐倒在地上。这种景象简直匪夷所思到了极点,难怪那怪手不见了,原来藏到她头发里去了,那刚才和我说话到底是鬼还是人。

这时候那旋转密封锁就已经松动了,那女的连转几圈,正准备把那钢门拉开,里面突然一声巨响,从门里冲出大量的水,那门就直接被水撞了开来,一下撞在那女人后背上,竟然把她撞得飞起来,一下把我扑倒在地上,我知道大事不妙,刚想把她推开逃命,那海水就扑头冲了过来,直接把我们两个冲出去五六丈。我勉力抬起头,正看见一张长满鳞片的巨脸,从那门后面探出来,直直盯着我看。

\chapter{海猴子}

这张狰狞的巨脸几乎比我的脑袋大了四五圈,身体还躲在那铁门后面,不知道到底是个多大的东西,从甲板的破洞里照过来的光线并不十分明亮,我无法看清楚它的五官,也不知道是鬼还是什么动物。只觉得这张脸鬼气森森,说不出的诡异。

我就这样呆呆地看着它,浑身从头皮麻到后脚跟,吓得几乎连呼吸也不会了,他娘的两条腿又开始不争气,竟然软得像面条一样。我往后艰难地退了几步,随即想到那个女人还躺在地上,这女人虽然不是什么好东西,但是见死不救总也不是办法。

我把她翻过来,发现那两只枯手又不见了,不过现在也管不了这么多,如果水再涨上来,她的头浸在水里就会淹死,我把手插在她的腋下,慢慢往后挪去,在船仓的另一头肯定通到甲板上的楼梯,只要我把这女人拖上甲板,要么就跳海,要么求救,选择就多了。

我一边迈着发抖的腿,一边在心里默念:“冷静,冷静,越是遇到这种情况越要冷静。”一点点地向后挪去,眼睛一直不敢离开那张脸。

那怪物幽幽地看着我,动也不动,一时间只听到哗哗的水声,如果它做出点什么动作,比如转转脑袋,张张嘴巴,我可能还觉得轻松点,可是它两只眼睛就直勾勾盯着我,看的我越来越发悚。心说这也太不正常了,不过你既然现在不动,就一直不动下去好了,可不要等到我快到楼梯口的时候再扑上来。

我想着,干脆不去看它,低头就加快了速度,几下就拖到楼梯口,一看,傻了,那楼梯已经烂得只剩下个架子,我一个人也不知道能不能爬得上去,更不要说这里还有个半死不活的婆娘。我看到那楼梯还有几根铁架子横在那里,拉起那婆娘的一只手,试着爬了一下,结果一踩就断,已经烂得像泥巴一样。

这下很棘手,我回头望望,幸好这怪物非常有耐心,还在那里呆着,现在我在阴暗处,和它之间有个光源(甲板的破洞)。所以我只朦胧地看到一个轮廓。这下我心安了不少,先把那女人靠在墙上,然后咬了咬牙,用力一跃,想自己先爬上去再说。

可怜我两手虽然修长,但是一点力气也没有,失败了两次,不仅没爬上去,嘴巴还磕了一下,疼得眼泪都下来了,心里非常懊恼,在那里想了半天也想不出办法来,我习惯性地转头,想看那怪物还在不在,这不转头还好,一转头,就突然看到一只巨大的东西不知道什么时候已经站在我的身后,我几乎和它脸对脸就碰上了,吓得我无法控制地大吼起来。

如果你突然回头,看到一个人无声息地站在你背后已经更够恐怖的了,现在看到这么一张狰狞的脸孔,那种恐惧真的无法表达出来,我大叫的同时,人已经不由自主地往后退去,一下子贴到舱壁上。

这个时候我已经看清楚这东西的样子,脑子里闪电般想起一件事情,我小时候听一个沿海的同学说过,他们村里有一个渔户有一次打到一只奇怪的东西,长的像个人,但是满身都是鳞片,拉回到村里一看,没人知道是什么。后来他们叫来村里一个上了岁数的老头子,这老头一看,吓得几乎没背过去,大叫:“快把它放了,这是只海猴子,等一下其他海猴子找上来,要出大事情!”

可那渔户一听这东西这么珍贵,就动了歪脑筋,想把它养起来卖给城里,就表面上对村里人说放了它,其实把它藏到自己家里去了。结果第二天,那渔户全家都失踪了,村里人觉得不妙,找了整整两天,终于在海边一个悬崖底下,发现那渔户老婆的尸体,肚子都给剖了开来,内脏都吃空了。

那老头看到了就说是其他海猴子上来报仇了,就叫了一个风水先生,在海边上摆了个供台,放了很多猪头羊头,做了好几天的法事才罢休。

我那同学还把那海猴子的样子画给我看,他平时就很会画这些东西,画得极其逼真,当时就给我幼小的心灵造成了很大的冲击,几天没睡好觉,我对这东西的印象很深,现在看到马上就想了起来。只是没想到这所谓的海猴子个头这么大。

记忆一闪而过,那怪物并不做出任何动作,只很有兴趣地盯着那靠舱壁上的女人,嘴巴里竟然流下口水来。幸好这婆娘没醒,不然真的恐怕要吓得失禁掉。

我稍微冷静下来,按了按背后的舱壁,也是那种很脆的已经腐朽的木板子,这个时候我已经有了一个计划,只要我用力往后一靠,就肯定能把舱壁靠出个洞来,那样如果那海猴子扑过来,我也有地方能退一下,只是那舱壁里头已经是船尾了,里面应该有很多机械设备,不知道能不能找到什么东西可以当武器。

我正在胡思乱想,突然听到甲板发出几声咯吱,似乎又有一个人上了这艘船,正疑惑着,就看见张秃从甲板的裂口里跳下来。这阿呆刚着地就举起手枪,先警惕地看看了那铁门,然后转过来,顿时吓得大叫:“我的妈呀!”

那怪物听到叫声,一转头就看见了他,突然发出一声极其凄凉的大叫,一矮身就扑了过来。那张秃的应变倒是非常了得,马上反应过来,往地上一趴躲过了第一击,喀嚓一下拉起枪栓,就是一枪,那怪物发出一身闷哼,肩膀上已经被打开了花,疼得一下子跳到船壁上,那张秃子又胡乱开了几枪,子弹几乎全打在我脑袋边上,吓得我一缩脖子。

海猴子非常机灵,一看这枪似乎很厉害,不敢再扑上去,佯装扑了一下,然后突然几个闪电般的连蹦,越过张秃子,直接窜回到那个铁门里。

张秃子枪跟着它扫,把舱壁上扫出一排子弹孔,马上水就涌了进来。这下子水位上升的更快了,他杀心很重,两枪将那铁门两个门轴打烂掉,然后上去一脚把门踢开,我跟他后面跑过去一看,只见一个船底有窟窿正在不停地往里面进水,那怪物正用力想钻进去,我一看这洞就知道这必然是当年出事时的破口,就是这个口子导致了这艘船沉没,不过现在已经被大量的海锈堵得只有碗口大了,那怪物力气极大,张秃子刚端起枪,它已经一头撞破一个可以容他通过的口子,然后一个猛子就扎了下去。

张秃子还是不甘心,对着水里又扫了几枪,这个时候船整个身体已经发出要断裂的呻吟声,我一看,水已经没到膝盖了。这个地方再也不能久留,要马上离开。那张秃子跑回去摇了摇那婆娘,叫了几声:“宁,宁!”看她没反应。他背起那个女的,一脚踩在我背上,利索地翻了上去。他那一脚,几乎把我踩得吐血,我一下子腰就折了一样,那张秃子在上面蹲下,对我伸出手,把我拉了上来。

\chapter{永兴岛}

我刚翻到甲板上,这鬼船就发出一声凄凉的扭曲声,好像某个巨大的部分变形了,我看到这船前后变得不在同一个水平面上了,心说不好,忙看了一眼船仓。果然是龙骨断了。

龙骨一断,船身必然回开裂,这么一艘船,一个裂口就已经非常致命了,那水几乎就是飞一样进来,估计不要五分钟这船就彻底没顶了。

那张秃子紧张的脸色发白,说道:“我们的船来了,我们快点离开这里再说。”

我回头一看,我们坐的那只渔船已经跟得很近,但是还没有靠上来,船上船老大挥着手,大叫:“你们怎么样?”

张秃子背起那个女人,对着那渔船招了招手,那渔船上欢呼了起来,然后发动机器向我们靠了过来,上面几个渔夫在那里兴奋地大叫,真想不明白他们刚才还吓得像团泥一样,这些单纯的渔民果然和我们不一样。

那鬼船因为进水,速度已经慢了下来,我们的船靠过来之后,有几个渔民跳了过来,看表情还是害怕,他们手忙脚乱地把那女人抱回到船上去,然后赶紧把锚搬回来。那个船老大大叫:“开船开船,我们快离开这个鬼地方!”

船老大让我们把那个女的放到地上,示意我扶住她,然后将她的头发撩了起来。

我已经做好了心理准备,但是看到那东西的时候,还是吸了口冷气,只见她那头发里面,蜷曲着两只枯手。现在看来,这两只手也并不是很长,皮肤都已经有点石化掉了,末端长在一团肉瘤的下面,最恶心的是,肉瘤上竟然还隐约长了一张小的人脸,那脸不知道通过什么原理,紧紧吸在那女的后脑上。

船老大看到这些表情凝重起来,先是给那个东西磕了几个头,然后从他口袋里掏出一把什么东西,就撒在那小脸上,那小脸突然尖声一叫,扭曲起来,他马上抽出一把刀,小心但迅速地插到肉瘤和头皮的中间,把那肉瘤挑了起来,然后用力一扯,扯了下来。

那东西掉到地上,扭来扭去,吓得边上看的人都往后退了好几步,几下子工夫,就融化成一团糨糊一样的东西,顺着甲板上的缝流下去。我从来没见过这东西,问:“这是——”

他把刀放到海水里洗了一下,轻声说:“这是人面臁,是那艘鬼船上的冤魂,要用牛毛撒在上面就行了。”

我看船老大的表情,就知道他已经对自己当初接下这个生意感到后悔了,嘴巴里一直嘟囔着什么,检查完那女的头发里再没其他东西,就招呼手下往后舱里走。不一会儿,船就开动了。

这个时候海面上已经平静了下来,天上的黑云虽然还在,但是已经分割成一小块一小块,阳光从那云和云的缝隙里照射下来,天空显得非常魔幻,看样子这他娘的风暴,总算是敖了过去。

我们把那女人安顿好,船老大就爬到船的顶棚上,我知道他要去看着四周的海面,那海猴子报复性极其强,不知道会不会跟着我们找机会报复。不过西沙的水很清,光线好的时候能见度有四十多米,如果有东西跟着我们,肯定一眼就能看见,所以我也并不是很担心会有这种事情发生。

这些人忙碌起来,都不理我开始跑来跑去,我因为刚才那一下子体力消耗得非常厉害,现在人一静下来,就觉得昏昏欲睡了,我找了块比较软的地方靠下来睡了一会儿,醒过来的时候,发现太阳已经西下,我们的船正贴着一个岛的海岸行驶,我看到非常漂亮的白色沙滩,就是看上去那些沙子比较粗,可能踩上去并不舒服,而我们前面就是一个码头,看样子像是要靠岸。

我一直以为会直接到下一个探点去,没想到还有靠岸的机会,随口问了一句话:“我们现在要去什么地方?”

旁边一个人回答说:“我们去永兴岛,接几个人。”

我转过头,看见那女人就坐在我边上,脸色已经恢复了过来,似乎也是刚刚醒过来的样子,我对女人没什么抵抗力,看她病怏怏的样子觉得还真是有点味道,笑了笑问她:“去接谁?”

她指了指远处码头上,隐隐约约一群背着旅行包的人,说:“就是他们,几个潜水员,还有一个和你一样的顾问,我想你肯定认识的。”

我使劲看了几眼,也觉得其中一个胖子的体形有点熟悉,但是想不起来是谁,这个时候,一个船夫已经站在船头,叫起来:“哦累累!做好准备,我们在这里!”

那胖子转过头来,大骂:“哦你个头啊,让胖爷我在这里吹了半个小时的西北风,你们他娘的有没有时间观念?”

\chapter{胖子}

我心里虽然有几丝惊讶,但是已然猜到了这个可能性,从鲁王宫里出来的人,大奎死了,三叔失踪,潘子昏迷,闷油瓶生死不明,只剩下我和这个胖子,这个组织肯定是两手准备,我估计他们的第一人选可能是胖子,我可能还是个替补。

船到码头,并没有减速,那胖子几天不见又肥了一圈,不过身手照样可以,跟着那群人同时一个纵身跳上船,往前跑了几下才定住,看到我,开心地大笑:“小同志,你也在这里啊,看来我们的阿宁小姐面子还是很大的嘛。”

那女人勉强对他一笑,看样子他们还很熟络,我对这个胖子的评价一向是毁誉参半,他的到来,我也不知道是该高兴还是悲哀,不过想起他在鲁王宫中的举动,几次都差点把我害死,不由有点头痛起来。

他把行李往甲板上一扔,就坐到我们对面,敲着背说:“这一路把我赶的,你们他妈的也催得太急了,对了,那地方找到没有?”

那个叫阿宁的女人摇摇头:“还剩下最后一个点,不出意外应该就是那个地方了。”

那胖子说:“我可和你们说过了啊,胖爷我什么寻龙点穴,探穴定位通通不会,你们地方找到了再通知我下去,要是找不到可不能怪我,钱我可照收啊,江湖规矩,你们南蛮子得入境问俗。”

阿宁头痛得叹了口气,说:“我知道你不会,已经安排好了,具体定位的事情,就由吴先生负责。”

我本来心情比较放松,一听就蒙了,我负责,我拿什么负责?我连一铲都没下过呢,忙说:“我负责?你们不是知道那海斗在什么地方吗?”

她说道:“只能估计出一个大概的方位,如果能找到盗洞最好,找不到的话,实际的定位和判断地宫的形状,还得靠你,我们手上只有一些故纸堆的资料,不可能代替土夫子的经验,你三叔很精明,这些资料一点也没有留给我们。”

我背上全是虚汗,看来今天晚上也不用睡觉了,得好好回忆回忆爷爷当年教的那些东西,不然,一旦到了那个地方,马上就要出洋相了。

下铲子我是一点问题也没有,在海底有什么不利索或者失误,都可以说是因为海水的关系,到底是土夫子又不是海夫子,这一块应该不算我的专业范畴,但是要我规划地宫,这难度也太大了,幸好我虽然没实践过,但是理论经验还在。

我想了一下,刚才紧张的心情已经平复了很多,心说船到桥头自然直,到时候真的不行,就瞎掰几句说这地宫有些古怪好了。

那胖子看看我,说道:“那就好,一切具备——不过难得来次西沙,咱们今天晚上得好好吃一顿,养足力气,这倒斗可是体力劳动。”说着就跑去找那个船老大,提溜着他,问他船上有什么海鲜没有。

阿宁似乎没什么胃口,靠到一边也不说话了,我倒是饿了,一听有海鲜,口水就多起来,也跑过去看。

西沙马鲛鱼、马鞭鱼和石斑很多,有人说,西沙的海里一半是水,一半是鱼,所以渔船出去,很少会没收获。旅游季节,在西沙钓鱼也是一件十分有意思的事情。胖子连逼带喝,那船老大十分不情愿,还是从渔箱里提出来一条大马鲛鱼,交给一个伙计,说:“拿个鱼头锅出来。”

胖子不知道刚才发生的事情,看船老大哭丧个脸,十分不爽,骂道:“他娘的老子又不是不给钱,又不是抢你的。”

不过不爽归不爽,那鱼锅子端上来的时候,那个香啊,就别提了,我一下子所有的欲望都变成食欲,以前在城市里,从来没想过会这么想吃一个东西,那胖子馋得眼睛都直了,锅子还没放稳,就直接一筷子下去夹了块鱼皮吃,烫得他眼泪都下来了。

这一锅子东西威力实在太大,不知道都饿了还是怎么了,那些个新人全部都围过来,连在仓底下睡觉的张秃都跑了上来,凑过来一闻,直说:“西沙就是好,随便烧个鱼我们那里一辈子都吃不到。”

胖子一把把他拉远,大骂:“拍马屁归拍马屁,你他娘的别口水喷进去,恶心不恶心。”

张秃一看胖子没见过,忙去和他握手,说道:“哎,生面孔啊,怎么称呼啊?”

胖子为人很直,看他一眼,问阿宁:“这秃子是谁啊?”

张秃一听脸就黑了,用力说道:“请称呼我张先生,或者张教授好吗?”

胖子也不理他,阿宁看气氛不对,接过来说道:“忘记给你们介绍了,这位是张教授,也是我们这次的顾问之一。”

胖子一听真的是教授,也不敢太放肆了,忙和张秃子握了一下手,说道:“哦,真对不住了,我还真没看出来您是个文化人,我就是一直肠子,姓王,粗人一个,你别往心上去。”

那张秃一听才勉强笑了一下,说:“这个文化人和粗人,都是人嘛,文化人还不都是粗人变的,分工不同,分工不同。”

胖子也听不懂在他讲什么,只好赔笑,那张秃不识好歹,又问:“那王先生是从事什么工作的啊?”

胖子一愣,直觉得别扭,但是也不能在文化人面前表现得太粗,说道:“这个,通俗地讲,我其实是个地下工作者。”

那张秃一听,不由肃然起敬,说道:“原来是公安战士,失敬失敬。”

我一听,忙憋住不让自己笑出来,他娘的张秃子也太啰嗦了,胖子看我笑起来,狠狠瞪了我一眼,对张秃说:“先别顾着说话,来,尝两口先。”说着就招呼其他人动筷子。

我不去管他们,夹起一筷就吃,那口感,真他娘的绝了,第一口还没咽下去呢,我第二筷子又下去了。

那胖子吃了几口,大呼过瘾,又叫着要酒喝,阿宁说道:“这出来打渔的,怎么可能带酒出来。”胖子不相信,跑到船仓里一阵折腾,大笑着抱着坛酒出来,那船老大一看,大惊失色,说这是给龙王爷喝的,说着就过来抢。

胖子大怒:“你怎么这么多废话,就你这着破酒,龙王爷喝了肯定得把你这船给收了。”说着从自己包里掏出一瓶二锅头来,一把塞给那船老大,“拿着,给龙王爷换换口味!这叫南北酒文化交流,看到没,红星二锅头,好东西,你他娘的别不知道好歹。”

那船老大呆在那里,也不知道怎么办好,那胖子就当他答应了,一把撕开封口,就给我们倒上,那酒的确不错,是黎苗乡镇有名的椰子酒,我们大吃大喝,好一通风卷残云,一直到月亮到头顶上才罢休。

那胖子最后一口酒喝掉,打了饱嗝,一拍大腿坐了坐直,说:“各位,咱吃饱了,也该谈谈正经事情了。”

\chapter{开会}

我看他胖子脸色一变,也不由振了振精神,这胖子虽然不太靠谱,但是在古墓里的表现还是可圈可点的,至少在经验方面不知道要好多少倍,我从来没独立倒过斗,也不知道是不是都要在下斗前开个动员什么的,就暂且当一回学生,听听他要怎么说。

那胖子吃的很多,肚子都鼓了起来,拍了拍说:“这海斗,我从来未倒过,事先肯定要部署一下,免得进去的时候手忙脚乱,里面肯定不比旱斗,我也先看看你们给我准备的装备怎么样。”

阿宁说道:“王先生,那你对这次有几成把握,我们不如先计划一下,心里也有个底。”

那胖子摇摇头:“不好说,根据我的经验,这海斗,一是定位困难,二是盗洞难挖,三是里面的情况不明。其中这第一第二,我们暂且不去想它,主要是这个第三,这海斗里,不知道有没有粽子,若是有,就麻烦了。若是没有,那这海斗也不过是在水里的一个旱斗而已,轻易就可拿下。”

说起粽子,我突然想起三叔和我讲的,那在墓道里碰到的怪物,越想越觉得可能就是今天在鬼船上碰到的海猴子,心里不由有点发悚,说:“这有没有粽子我不知道,但是可能有更麻烦的东西。”说着就把在鬼船上看到的那东西和这些人说了,其他人早就听张秃添油加醋地说过了,不过那张秃说的重点应该是他如何如何把我和阿宁救下来,我说的就平实得多了,等我说完,那胖子就大皱眉头,问:“操,他娘的真的还有这种东西?”

我点点头,说:“很多地方都有这东西的传说,应该不会错。”

阿宁点点头,说道:“我小时候也听过,我还以为大人吓唬我不要到河边去玩。”

这个时候,那船老大插嘴了,他说道:“不对不对,这你们就不懂了,这里打渔的渔船,都见过这东西,我告诉你们啊,这东西不是什么海猴子,这是夜叉鬼!那都是龙王爷的亲戚,你们现在得罪了他,他肯定要回来报仇的,我看我们还是快点回到岸上去,买头猪回来,请个道士作作法事,兴许他大人有大量,还能放过我们。”

张秃一听,就笑了:“我说,船大爷,我都一枪把龙王爷亲戚的肩膀给打烂了,那我岂不是孙悟空?”

船老大气得脸都绿的,说道:“你那个样子哪里像孙悟空,你就是个猪八戒!”

我们听得都乐了,张秃捏捏脸上的肥肉,大概觉得自己真的有点像,不由郁闷起来。

那胖子笑了一会儿,说道:“既然海底有这种东西,我们肯定得有武器才行,万一那海斗里就是他们的老巢,那我们岂不是跑去送死?我说阿宁小姐,你有没有准备渔叉什么的?”

那阿宁说道:“我们是考虑过这个情况,准备了一些潜水用枪,但是这些枪体积很大,而且一次只能打一发,如果有紧急情况,恐怕也没有什么大作用。”

我知道这种枪,是用压缩气体击发的,有效距离大概才四米不到,幸好还可以当长矛用。不过这枪的长度确实太长,在狭窄的墓道里可能施展不开。

胖子不理会这些,大叫:“甭管有没有用,枪这东西不嫌多,能带的都带上,明天下去,我就打头阵,小吴同志就跟在我后面,你和那个秃子就在最后,如果我一看到不对劲的东西,就摆摆手,你们就马上停下来,如果我摆摆拳头,你们就什么都别管,逃就是了。”

我们觉得安排比较合理,点了点头,又讨论了其他一些东西,我想想三叔和我提过的经历,列了一些清单出来,让他们连夜先准备好,什么探灯,匕首,火折子,密封袋子,尼龙绳子,登山扣,还有吃的,急救用品,放毒面具,百宝盒,他们准备的比较全,竟然连黑驴蹄子都准备了。

吩咐好之后,天都快亮了,那胖子说我们不能再谈了,再谈水都下不了,得休息,于是几个人各自找了个地方躺下,那椰子酒后劲很大,被海风一吹,我头就重得不行,几下子睡了过去,一直到下午才醒了过来。

其他几个人比我早醒,已经都在准备了,我用海水洗了一把脸,这时候,几个蛙人已经从水里浮了上来,一个摘下呼吸器就说:“找到了,肯定就是这个地方,盗洞也找到了。”

那阿宁一听,忙问:“有没有进去看看?”

那人摇了摇头,说:“有,但是那盗洞很长,我潜进去一段,没看到底,不敢再进去了,就出来了。”

阿宁点点头,又问了那个蛙人几个问题,转头对我们说:“行了,我们准备一下,他们清理完洞口就会叫我们,那洞口里有塌方的迹象,他们会用支架固定一下。”

我们各自去穿潜水衣,我和其他几个都很合身,就胖子,肚子包不进去,露了肚脐出来,虽然不太雅观,但是好歹是穿上去了,我们检查完装备,把该带的都带上,就一个接一个倒摔进水里。

\chapter{头发}

那盗洞离船不远,我看到海底给炸出一个大坑,洞就在坑的底部,心说果然是三叔的手段,我们在盗洞四周先搜寻了一下,没有任何坍塌的迹象,看样子三叔的技术并没有退步。

我还看到几个石头锚碇,和三叔描述的很像,但也不能肯定就是三叔所说的那些。

三叔规划出的地宫痕迹还在,我和张秃都用心记了一下,看这个盗洞的位置,应该是往耳室挖下去的,那个地方的砖应该比较薄。

我们大概找了5分钟,似乎没有再找下去的必要,那胖子对我摆了摆手,意思现在要不要进去了。阿宁看了看潜水表,点了点头。

我们现在的装备不比20年前,都是轻装上阵,我们最后在洞口核对了一下装备和约定好的暗语,确定一切没问题了,胖子才定了定神,第一个猫了进去,我们几个打开探灯跟着,一下子潜进去五六米。

这盗洞很不规则,时宽时窄,我一边游一边看这洞壁,越看越奇怪,怎么看上去不是人挖的,如果是三叔打的洞,他肯定是一个铲子一个铲子打的很工整,可是现在这那上面的痕迹,乱七八遭,坑坑挖挖,倒像是动物打的洞。

我们艰难地游了二十多米,洞口进来的光线已经照不到了,这个时候盗洞方向突然一变,竟然垂直挖了下去,我不由有些奇怪。既然还没挖到墓,何必改变方向呢。

苦于没办法说话,我也没办法表达自己的疑问,我们在这垂直的洞口休息了一下。胖子对我们做了一个小心的手势,然后自己先游了下去,我看他的灯光一直下去一直下去,直到变成一个小点,不由咋舌,心说怎么这么深。

这个时候他在下面晃了晃探灯,说明下面安全。我们马上一个接一个也潜了下去,我看着潜水表,已经有十几米深了,我从来没有潜到这么深过,不知道自己的身体能不能撑的住。

那下面已经被挖开一个很大的空间,我们马上看到了古墓的墓墙,上面破了一个大洞,我一看更加疑惑了,这洞竟然破的这么不规则,不像是一般倒斗的一块一块小心地卸下来的,有几块砖头竟然还被撞裂了。那胖子看看我,我也看看他,两个人一起吐了几个泡泡,他指指那几块破砖头,又做了个猴子的样子,我知道他是想说:这洞可能是海猴子挖出来的,不是盗洞。

我点点头表示同意,指了指他背上的水下气枪,他拿了下来,拉开保险,就往洞里游去。

这是我第二次进古墓,虽然有点兴奋,但是想起上一次的经历,还是觉得浑身不自在,特别是在水下,手脚的阻力很大,如果遇到危险,恐怕也没办法像陆地上一样快速的逃命。

墓道比我想的要大的多,我调高探灯的亮度,又转开手里的防水手电,跟在胖子屁股后面,我们几盏灯光非常的亮,一下就照出去老远,顿时整个幕道都亮了起来。我看到那墓壁的墙上,果然有三叔说的人脸浮雕,不仅如此,这些人脸浮雕的额头上面还都刻着一些奇怪的动物,雕的非常精致。我一边游一边看,越看越觉得奇怪,这些动物,大部分都是墓镇兽,但是它们都没有刻上眼睛,看上去有点诡异。

这个时候,我突然看见有一张人脸的额头上,刻的好像是三条蛇眉铜鱼,不由心里一紧,忙拉拉胖子让他停下来,然后去研究那块浮雕。

胖子正急着往里面走,很不耐烦,也不知道我发现了什么,他转过来看了几眼,没看出什么名堂来,就急得直招手,我让他等等,趴过去仔细看,只见上面有三条蛇眉铜鱼首尾相连的,形成一个环状,每条造形都不一样,我能看出其中两条就在我的包里,还有一条三只眼睛的,我从来没见过,不知道这个是提示什么的。那鱼下面的那张脸和其它的不一样,是一张明显有女性特征的脸,可是因为上面附着了很多东西的缘故,这张脸看上去有点破相。让人不太舒服。

我还想仔细研究一下,这个时候后面的阿宁也催我,我没办法,只好继续向前游去,幸好那雕刻每隔一段距离又会出现,我还能再看上几眼,看来看去,并没有发现更多的东西,只是隐约觉的有个地方有点不对劲。

看着看着,等到我数到那脸孔浮雕第五次出现的时候,才发现了问题所在,我记得第一块石头板上的人脸,眼睛是闭着的,第二块石头板,似乎有点睁开的趋势,到了第三第四块石头板子,那眼睛睁的越来越大了,现在这第五块,就已经睁的几乎全开了。

我感觉有点不妙起来,拉住胖子,让他不要往前走了,然后拿出水下画板,在上面写道:“墓墙上的人脸,眼睛在逐渐睁开来,我怕有问题!”写完指了指墙壁。

胖子摸了摸那脸,摇摇头,写道:“我没有注意,只是些石头浮雕,里面肯定是整块石头,你想的太多了。”

我很坚决地摇头,让他把枪端起来,他看我表情严肃,只好照办,不一会儿,我就看到那块相同的浮雕出现在前面,胖子被我说的也有点怕,停了下来,先用灯光照了一下。那张石脸的眼睛已经完全睁开了,整张脸面对着前方,眼神正视,看上去有点呆滞,胖子照来照去,也没什么变化,就壮起胆子走过去,摸了一下,然后对我做了个没事情的手势。

我游过去一看,果然仍旧是整块的石头,并没有什么特别的,用手指插了插它的两只眼睛,也没有反应,不由自嘲地摇摇头,看来这只是墓穴的设计者玩的一个噱头,用来吓唬可能进来的盗墓贼,没有什么特殊的寓意,我竟然在这里自己把自己吓唬了一回,真是没什么面子。那胖子拍了拍我,示意我别想这么多,快点赶路。

我们又继续往前游去,我想起三叔和我说过,他是撞到一个机关,才被吸进那个泉眼里去的,可是这些墓壁都是一个样子的,怎么可能找的到他当时撞的那块?

我脑子转的飞快,这样一直往前游也不是办法,不知道这个墓道是通到什么地方去的,说不定又是个循环,如果在里面迷路就完蛋了,我心里盘算,三叔能一眼望到最后一个人,应该是一条很长的回廊,刚才我们转了好几个弯,这样的回廊只有两个,这样说起来,找找倒也不是很困难,就是要花点时间。

这个时候,前面的胖子停了下来,我一个刹车不住,撞到了他的屁股上,以为前面出了什么状况,忙蹦紧神经,凑上去一看,原来这墓道到头了,前面被一块石头板当住了去路。

这石头板光秃秃的,上面没文字也没有浮雕,我摸了好久,找不到什么机关,不由挠了挠头,那阿宁写着问我:“怎么会是死路?”

我回写道:“有巧石机关在这附近,我们找一下,看看有没有松动的墓墙。”

他们都点点头,那胖子开始东敲敲,西敲敲,仔细检查了这些人面浮雕。我心里回忆所有笔记上提过的线索,连边上的每条缝隙都用匕首划过,可是没有任何进展,那石板还是挡在那里,纹丝不动。

我不由有些郁闷,回头想看看胖子搞得如何,发现胖子竟然在那里发呆,我拍了拍他,写着问他:“有没有什么发现?”

他表情古怪地看着我,写着问我:“海猴子长头发吗?”

我不知道他突然问这个什么意思,不由失笑,海猴子长没长头发我倒是真没注意,记忆似乎整个脑袋都是光秃秃,全是鳞片。

我如实告诉他,又问他问这个干什么,他指了指墙缝,我顺着他的手指一看,马上看到,那石板的与墓道的缝隙里,竟然飘出来一缕黑色的头发。

我惊讶地呆住了,这怎么可能,难道在石头板的那一头,靠着个人?

胖子胆子大,想伸手过去想拉一下,没想到那头发突然一缩,竟然缩回到缝隙里面去了。胖子看了我一眼,写道:“石板后面有鬼。”

\chapter{大量头发}

水底古墓里发现一缕头发,而且还能动,一般人都会马上想到有鬼,幸亏中间隔着一块石板,就算有,他也冲不过来。

没有抓住那缕头发,胖子似乎不甘心,拿灯去照那缝隙,想看看后面到底有什么。我胆子没他那么大,恐怖片里关于头发的故事还少吗?就离那个石板远远的,看胖子会有什么反应。

他凑上去看了几眼,好像真的给他看到什么东西,疑惑的定了定神,又贴过去再看,这一次他反应很大,突然就猛的一退,像逃命一样拼命游出去好几米,转身对我们拼命的摇拳头,我一开始以为他要打我,随即一想,靠!这不是让我们逃命的手势嘛。

可刚才好像没什么事情发生啊,我条件反射一样地回过头,只看见那挡路的石头板突然向上升了起来,一团黑色墨汁一样的东西从底下逐渐增大的缝隙里渗了出来,我急退几步,以为是毒水,仔细一看,吓得我下巴都僵掉了,那些黑色的东西,竟然都是人的头发!

那胖子看我们反应这么慢,忙游回来拉我们,我们这才醒悟过来,慌忙逃命,这在水下面,越紧张越消耗体力,游的就越慢,我看慌乱中没办法把握好节奏,索性学胖子一样蹬着墙走,虽然不雅观,但是速度飞快,特别是脚塌实地那种感觉非常好。

我们连蹬了大概二十几步,先到一个转弯处,那胖子一把把我们拉住,让我们躲在拐弯后面,示意先看看情况再说。

我们大口吸着氧气,匆匆往后一看,好家伙,后面的墓道里全是头发,黑漆漆一大团一大团,我看着就觉得喉咙发紧,这要多少年没理才能长的这么长啊!胖子骂了一声,拿起汽枪,对准那一团黑色的中央就射,他大概以为这枪能一下穿透过去,所以当他看到那梭镖快速飞了六七米后突然就变成慢动作,然后一下被裹进头发里的时候,脸都白了。

不过那梭镖还是起了点作用,那头发竟然好像还有意识,往后缩了一下,竟然翻滚起来,说那翻滚更像是头发里面有什么东西要出来,我们不由警惕起来,那胖子又搭上一只梭镖,准备走近点再给他来一下,这个时候,那头发猛然一缩,又猛然一放,这一下子,我马上看见从头发的最深处,吐出来一个死人。

那人穿着和我们一样款式的潜水服,有可能那三个中的一个。我只看了一眼,就看到他的鼻子嘴巴里都是头发,连两只眼珠子里都有头发生出来,一看就是窒息死的,现在已经给水泡的肿了起来,非常地难看。

我一看头皮就开始发起麻来,这头发非常邪门,还是快点走,就想拉胖子,可抬头一看,那胖子竟然不见了,我吓了一跳,忙转头,只见他已经跑去出老远,在那里给我们挥拳头。

我心里大骂,敢情你是自己先跑到安全的地方再来警告我们,忙招呼张秃和阿宁跟上去,我看到那胖子还在那里抱怨我们反应慢,立马就给他屁股上来了一脚。胖子被我踢得疼了,还不服气,想冲上来揍我,那阿宁忙栏住我们,指指后面,我一看逃命要紧,这帐出去了再她娘的和他算。

这个时候。手上的氧气记震动起来,我低头一看,糟糕,这一路过来,已经过去将近半个小时,我们又呼吸得这么急促,氧气的消耗是平时的几倍,有点过量了。我算一了还剩下的时间,情况可以说非常糟糕,如果还没有进展,我们就必须要原路赶回去,不然氧气就不够用了。可是这么出去,连三叔说的耳室都没有找到,我又有点不甘心。

这个时候,一直游在最后的张秃突然像只螃蟹一样,拉住我们身上的背带,手忙脚乱的窜到了最前面,一把纠住胖子,让他停下来,我看到他直鼓眼睛,心说,这人对古墓的构造比我了解,难道竟然给他先找到了什么线索?

果然,他让我们跟着他过去,胖子急的直跳,但他刚才表现太差,我们都不去理他,他也没有办法,只好气鼓鼓的跟在我们屁股后面。

张秃笨拙地往回游了几米,指着一块已经有点凹陷进去的墓墙让我们看,原来刚才胖子蹬着走的时候,一脚就把这块墓壁给蹬到墙壁里去。

我一看就心中大喜,往前后比画,这里果然是一处长回廊的末端,那三叔所说的机关十有八九就是这里了,不过这机关一开,水就会狂涌进去,三叔当年是带着头盔,所以没事情,我们现在头上只有个潜水镜,一但被卷进急流,难保不会撞得头破血流。

我往后看看,那头发还没有追过来,就想先提醒他们一声,这个时候,那个张秃不知道好歹,突然一把就按了上去。我还没反应过来,一下子大量的水泡就冒了出来。

我一看就知道遭了,实在没有想到,这张秃闯祸的能力和胖子比起来有过之而无不及,我一口他妈的没说出去,就觉得一股巨大推力直接从我背后冲过来,把我狠狠推进了墙上的洞里。那水流是旋转着的,我马上体会到三叔说的,什么是内脏都被甩到一边了,就感觉自己被塞进了滚筒洗衣机里,那一阵搅,几下子就晕得什么都看不清楚了。

也不知道过了多久,等我晃晃悠悠清醒过来,感觉浑身都散了架,特别是脖子,疼得不得了,幸亏没折掉,还好呼吸嘴还咬在嘴里。我定睛一看,上下左右都是黑漆漆的,胖子他们在我的下面,看样子也晕的不行了,特别是胖子,到现在还在转圈子,好像在跳芭蕾舞一样。

我看了看这井壁,是上等的汗白玉,这里用上这么好的材料,应该已经到这个墓的地宫内部了,看样子可能已经到了三叔说的那个耳室中的泉眼里,我脚一蹬向上浮去,突然头一暖,脑袋就此升出水面。

四周是一片漆黑,探灯的光集束性太强,只能照出一个点,我关掉探灯换成手电,把这个墓室仔仔细细看了一遍。墓室是见棱见角的长方形,除了宝顶上面描着五十星图之外,其他地方并没有太多的檐楣雕饰,显得朴实无华。

里面没有棺床和棺椁,所以这里应该耳室之一,我找了一下,似乎没有其他出口,只有左边一道石门连着甬道。

墓室的墙也是用非常廉价的白膏土封起来,上面本来有一些斑斑斓斓壁画,可惜已经被水汽腐蚀得一塌糊涂,我已经无法知道上面画的是不是禁婆的图案。

墓室的地上放了几遛陪葬的瓷器,只有百来个,其中还有几个非常值钱的青花云龙大瓷缸,我同时在地板上发现了一些脚印,都是湿的脚踩在地上的尘土上留下来的,看样子非常地新,估计是三叔的杰作。

我测定了空气质量,让他们陆续出水,阿宁爬了来后,首先担心起这些脚印,问道:“这是盗墓贼留下来的吗?”

我皱了皱眉头,也不敢肯定,因为我看见,在这些脚印中,有一个非常刺眼的赤脚印子,最离奇的是,这脚印很小,看样子是个小孩子的,绝对不会超过三岁。

我从来没听说过倒斗会带上小孩子,便招呼胖子过来看,他的阅历比较丰富,也许知道是怎么一回事情。

胖子看着也有点发懵,说道:“你先别管他是大是小,这脚印本身就不正常,你再仔细看看。”

我再次端详,发现脚印上有黄黄的一层蜡一样的东西,用刀刮下来一闻,不由咋舌:“这是尸蜡……!”

\chapter{大瓷罐}

尸蜡一般都是浸在水中或埋在水分充足、潮湿的泥土里的尸体,所谓的蜡就是它体内的脂肪和矿物质凝结而成的。

我顺着这脚印一路看过去,发现它一直延到房间的角落里面,一个青花云龙大瓷缸的后面。心里“咯噔”了一下。

人说阎王好送,小鬼难缠,难不成这里有一只未成年的粽子?我对胖子说道:“你看这脚印只有过去没回来的,会不会……”

话才说了一半,胖子一摆手,叫我不要说话,我转头一看,只见那是大瓷罐,突然自己晃动了一下。

胖子轻声说:“那东西,还在后面躲着呢。”

张秃装备脱了一半,腰上的带子没脱下来,现在索性不脱了,提了氧气瓶凑过来,问胖子道:“什么东西?”

胖子见他最烦,骂道:“粽子!”

他一楞:“粽子?加兴五芳斋粽子?”

胖子摇摇头,不理他了,我问胖子道:“你能不能确定,我从来没见过这么小的粽子?”

胖子说:“我也确定不了,不过不管是不是粽子,我们还是得过去看看,不然就是祸害,这斗倒了心里也不舒服。”说着就端起手里的气枪,向我招手,我心说我才不去呢,摇了摇头。

胖子叹了口气,只好招呼张秃过来,张秃第一次进斗,兴奋异常,马上就学着胖子的样子走了过去,两个人成包抄之势,向那青花大瓷罐走过去。

我虽然害怕,但是也不能在女人面前表现的太过窝囊,也装着样子,跟在张秃子后面,小心地探头看着。

我们走的很慢,生怕有什么东西突然跳出来,胖子先用电筒照了照,那大瓷罐太大了,什么都看不到,又用气枪捅了捅,他这样子很像是我小时候抓黄鼠狼的动作,我不由觉得好笑。他捅了五六下,听着似乎后面没什么东西,才壮起胆子走过去,一看就啧了一声,骂到:“他妈的,只有一个空的木头箱子,害我瞎紧张。”

我们也跟了过去,我一看,是一只只有小提琴盒子大小的双凤雕婴儿棺,那棺材盖已经被打开放在一边了,里面的白色棺底还保存得很好,但是尸体已经不见,难怪胖子会以为这只是口箱子,我说道:“这不是箱子,这是一种棺材。”

胖子一下子还不相信,但是他马上领悟,问道:“你是说,这就是那只小粽子的棺材?”

我点点头,又仔细看了看,发现棺身上被打了几个洞,有一道黑色的痕迹从洞里一直延伸到地上,看样子曾经有什么液体从这洞里流出来过,这情景,好像爷爷的笔记上也曾经提到过。

胖子用手电仔细地里里外外瞧了一遍,可惜地叹了口气:“看这棺材的规格,就知道这小孩子身上肯定有不少好东西,可惜不知道尸体到哪里去了,不然压几下,说不定还能压出几颗珠子来。”

我点点头,夭折或者陪葬的孩子,棺材里的东西总是很多,而且大多数都带在身上,特别是陪葬童子的肚子里,经常有防腐珠,都是些价值连城的东西。

我们几个人四处找了一下,想看看尸体到什么地方去了,可是前前后后都翻了一遍,连块渣都没有,看样子可能被那些盗墓者连着一块盗出去了。

胖子不甘心,还想去翻那棺材,我觉得不妥当,拉住他说:“这棺材和别的棺材不一样,绝对不是单纯放死人,还是不要碰了。”

胖子笑道:“尸体都没了,怕他个熊,你还怕这棺材跳起来咬我?”

阿宁说道:“我们来这里的目的不是倒冥器出去,还是快点到主墓室去,别浪费时间,速战速决。”

胖子自知理亏,也没办法。我们回去把潜水器械先整理好,胖子背起背包,突然看了看我,嘴巴动了动,欲言又止,好像有话想说,又有点不好意思说,我叹了口气,骂道:“你他妈的有话就说行不,什么事?”

胖子说道:“你们说,那小粽子,会不会爬到边上的大瓷罐里了?”

我看了看那大瓷罐,心中一动,还真的有这个可能。

胖子有点脸红,说:“我刚才听到那罐子方向发出的声音,好像是从罐子里面发出来的。我想粽子又不是老鼠,怎么可能自己往罐子里钻,以为听错了,现在只是随便提提,我可没别的意思啊。”

我知道他是还惦记着尸体身上的宝贝,就想讽刺他几句,这个时候,那只大瓷罐,突然咯哒一声翻倒在地上,我一呆,心说不会被他说中了吧。

四个人全部都不说话,紧张的看着那罐子,那罐子先是在原地转了几个圈,然后竟然“咕噜咕噜”地向我们滚了过来。

\chapter{甬道}

我刚才的注意力全部集中在那棺材上,没仔细看这瓷罐,忙急急退了几步,那罐子晃晃悠悠滚了几下,就改变方向朝甬道的石门滚去,最后“铛”地一声撞在门框上,停了下来。

我们几个你看看我,我看看你,都觉得莫名奇妙,难道真的给胖子说中,里面有只粽子?

我们楞了半刻,不敢轻易上前,胖子压低声音,说道:“各位,这罐子果然有点邪门啊。要不咱们先下手为强,给他来几梭镖?”

我当然不赞成,轻声说道:“千万别,先搞清楚到底是什么再说!”

我这样说,一来是我已经看出,这元明时候的青花大瓷罐,绝对是个珍品,这样的大小,世界上已经不多见了,恐怕是砸一个少一个。二来,不知道里面到底有什么古怪,如果真的是胖子说的粽子,那免不了又要开打,我刚才在水下消耗太多力气,绝对跑不动了。

但是我们现在是在十几米深的水下古墓里,这空气不知道能维持多少时间,如果僵持下去对我们没有任何好处。这一进一退很难抉择,我这人一向没什么主意,急得满头是汗。

这时胖子看我犹豫不决,说道:“咱们也不能肯定里面就是只粽子,这地方通着海,说不定是什么螃蟹龙虾爬进去了,犯不着在这里自己吓自己,还是过去看看再说。”

那女的摇摇头:“我们的主要目的还是进主墓室,不要在路上浪费阿宁时间,我看我们还是能避则避,看看其他地方还有没有什么出路。”

我一听这也是个办法,马上又将这耳室左左右右仔细检查了一遍,可惜这地方一目了然的,再没有第二道门,也没有可以供我们钻的洞。

胖子有点熬不下去,说道:“事到如今,要不就是把这罐子搬开,要不就是回去,没别的路走,不过我和你们说在前面,既然已经到了这里了,被个罐子吓回去,我王胖子肯定不干!”

我看了看阿宁的表情,也十分坚决,张秃子则一句话都不说,不知道葫芦里卖的什么药,他们三个人一齐看着我,好像在征求我的意见。

我心里还是没个清晰的决定,心说如果是贸然冲过去,当然不太妥当,但是胖子说的也有道理,这古墓里的东西,很多就是自己吓自己,我看着阿宁那种眼神,不由心里一软,说:“那行,走一步算一步,如果相安无事就算了,要是它老三老四的,我们四只梭子枪在手上,也不怕它!”

胖子拍拍我,表示鼓励,我拿出了气枪,打开保险对准那罐子,胖子打头阵,我们四个人小心翼翼的贴着那门的边往里走去。

我并不是非常的害怕,只是觉得有点紧张,身上又穿着潜水服,汗都透不出去,难受的要命。

就在我们胖子几乎能看到罐子里是什么东西的时候,突然一声响,那罐子竟然骨碌打了个转,我一下血全部冲上大脑,几乎要开枪了。

那胖子机灵地连退两步,做了个让我们不要动的手势,只看那瓷罐突然又滚动起来,这次它是直接一个弧线,咕噜咕噜滚进了黑漆漆的甬道。我们一直听着这个声音滚进去很远,才又“铛”地一声,撞到什么东西上消失了。

我们马上跟进去,里面是一片漆黑,我用电筒一照,只见这是一条汉白玉砖修的直甬,非常的简洁,里面什么东西都没有,只有在地上的两边有两条灯沟,里面是每隔1米的灯座,在甬道的另一头,有一扇玉门,而左右两面也各有一扇略小的门,一共是三个门,都敞开着,看样子已经有人进去,而那罐子,已经停在了左边那个小门中间,不动了。

我这次是真的觉得有点诡异了,这罐子的举动,好像是在给我们带路一样,就差没说一句“follow me”了。这肯定是一种有意识的行为,难不成,这罐子里的东西,不是粽子,是个鬼?

我看了看闷油瓶,也看不出他是紧张还是害怕,我只好把我的想法说出来,胖子一听有点道理,不由吸了口凉气,说道:“你这一说还真有点这个意思,我刚才也觉得,这东西这么滚着溜,简直就和一保龄球一样。”

我心里苦笑,胖子看我有点摸不着方向,又说道:“既然都到了这个地步,我们也别在这里犹豫来犹豫去,就一路跟过去,看他是什么目的,反正伸头是一刀,缩头也是一刀。”

众人点头,胖子拍拍我说:“这光秃秃的石板子路一般都有陷阱,吴老弟你看看,这地方有没有什么问题?”

我自知责无旁贷,点点头,就用手电照了照地面,这甬道底上都是小块的石头板,很可能装了强驽机关,我想既然三叔到过这个地方,如果有机关,也有可能已经被破坏或者引发掉了。但是万一没有,就比较麻烦,我提醒了他们一下,然后整了整背包就第一个往前走去。

要躲避机关,最好就是趴着贴着墙壁,但是这甬道的两边是两条灯渠,里面黑忽忽的不知道有什么东西,我们只能沿着渠边走。

我让他们要非常的注意脚下的感觉,但其实连我自己也没什么头绪,这一脚放下去,放多重,放多快,都是有讲究的,这些东西确实是经验,我是不可能有的,所以我越走就越觉得慌起来。

就这样忐忑不安的走了十几步,我身上都是个冷汗,后面那几个看我这么紧张,也慌起来,那胖子说道:“看样子这趟雷的工作还真不是这么好做,小同志,你要是太累,咱们就歇歇?”

我也没工夫和他抬杠,说:“别吵,我要是一分心,大家都得死。”话还没说完,突然脚下一振,我回头一看,只见阿宁脚下一块石板已经陷了下去,正一脸惊慌地看着我。

我哀叹一声,心说完了,怎么就怎么倒霉,这下子要被射成刺猬了,就听到一声呼啸,一支弩箭已经贴着她的耳朵就飞了过去。我还没反应过来,第二支箭也到了,直射她的胸口。

这真是电光火石,那阿宁眼神一变,闪电般地转身甩手,凌空一把就把那箭给握住了,那动作几乎就是在几分之一秒内,我甚至连她的影子都没看清楚。

我看她的身手,大吃了一惊,可是情况不容细想,只感觉到脚下一连串振动,忙大叫:“猫下去,还有暗弩!”

话音刚落,又是十几道白光射来,我忙低头躲过一支,这个时候,我突然看见那远处的罐子里,爬出来一只满身白毛的东西,迅速地钻进了左边的石门里。我刚想叫,突然胸口一痛,低头一看,靠!胸口不知道什么时候已经中了两箭,看样子还插进去了二三寸。

\chapter{箭}

我看到那箭头几乎全部没进了我的体内,顿时胸口一阵巨痛,心里慌得一塌糊涂,还不肯相信,我还这么年轻,连女人的手也没摸过,难道就这样死在一座不知名的坟墓里了?如果死在这个地方,恐怕几百年后都没人给我收尸。这样的下场,未免也太惨了一点。

箭像下雨一样射来,不知道到底是用什么东西发射的,速度太快了,根本没办法躲,胖子用他的背包当盾牌,一下子冲到我们面前,帮我们挡着了几箭,我看到他的背,不由倒吸了一口冷气,只见他背上密密麻麻已经插了十几只,就像一只插满了香的香炉一样,看样子也肯定挂定了,不过不知道为什么,他好像一点也不疼的样子。

我想起以前经常看到小说里描述人被箭射成刺猬,都没实际见到,现在总算是看到了,还是在这种情况下,不由心里暗骂,这个时候,突然就有人抓住了我的衣服,硬拽着我往那前走,我大惊失色,回头一看,竟然是那个阿宁,我看她眼神冷得可怕,心里觉得不妙,忙用力一甩,她见我想逃,毫不留情的一膝盖顶在我后腰上,这一下比胸口那两箭还疼,我全身一软,一时间疼得用不上力气,人就软了下来。她拎着我二话不说就往那中间的大玉门走去。我被当成挡箭牌,一下子肩膀、肚子、胸口又各中了一箭,疼得我几乎晕了过去。

人说最毒妇人心,我还真没信过,没想到女人真的这么狠毒,刚才还是那种害怕小女人样子,谁知道一转眼就可以拿我当人肉盾牌,去挡箭雨。

我当然不会这么伟大,用尽全身力气一扭,那女人力气并不大,我一下就挣脱了她,身子一歪倒在那灯渠里。那女人看失去掩护,马上一个翻身,一下子躲过十几箭,回头狠狠瞪了我一眼。我心说他娘的你还有脸来瞪我!大叫一声扑过去拉她,她朝我冷笑一声,一个就地打滚翻到墙边,然后高高跳起,在墙上一蹬,闪电般翻到了安全的区域,整个动作在电光火石之见完成,十分地干净利索。

我看她一箭都没中,气得拍了一下地,她转过头看了看我,突然轻藐对我做了个飞吻,然后打起手电,扭着屁股走进了中间那个玉门。

我气得差点吐血,也无可奈何,只好翻到那条灯渠里,只听着头上的箭嗖嗖地飞过去,撞在甬道墙上发出金属撞击声,这阵箭雨足足射了5分多种才停了下来,我回头看胖子,已经被射成了一个箭球,正摇摇晃晃似乎要倒下去,忙爬起来扶他,没想到他摆了摆手,示意自己没事情,问我道:“小吴,我看这些个箭有点不对劲,怎么插进去这么深都不觉的很疼啊,你给我拔几根下来看看。”

我也觉得有点不对劲,怎么这箭伤没想象的重,我呼吸还是很顺畅,不过我也没死过,不到被箭射死是什么感觉。

胖子叫着要我拔几根,我还真没这个胆子,在他面前迟疑了个半天也下不去手。这个时候张秃咬着牙站了起来,他刚才站在胖子后面,被胖子护住,也一箭都没中,见胖子被射成这样,突然说了一声:“放心,没事的。”

我和胖子同时一愣,这张秃子的声音怎么变了,而且还这么熟悉,只见他突然把身子一挺,就听咯哒一声,他的身高竟然长起来好几公分。接着,他又向前伸出手,同样一发力,又是哒一声,那手也突然长出去几寸。

我看的下巴几乎都要掉下来了,心说这不是缩骨吗?我只从我爷爷的笔记上看到过,这是古时候倒斗的基本功之一,在通过一些非常狭小的缝隙,比如说冥殿的梁孔,或者地下的虚位,都要用到这工夫。我一直没想通他的原理,所以一直当是个笑话,现在如果不是亲眼见到,真不会相信会有这么神奇的工夫存在。

(最近几年还听说洛阳盗墓村里有一些人还在用这功夫,他们把盗洞打的非常小,缩骨进去,警察路过看到,都以为是黄鼠狼洞。后来知道了这个是盗洞,也没办法下去抓人,因为等挖通了,里面的人早挖了另外一条跑掉了。可惜这功夫非常难练,就算从小练奇,如果不是全身的骨骼配合,也很难有成。)

他长出了一口气,抓住自己的耳后一拉,又撕下来一张人皮面具,露出了他原来的脸孔。我,我一看,几乎傻了,那人皮面具里面,竟然是闷油瓶!我呆了一下,突然就起了无名业火,这下子也太能装了,简直都能当影帝了,我还真的一点都没发现。

那闷油瓶甩了甩胳臂,似乎很久没活动了一样,那胖子也看的说不出话来,好久才一把拉住,说:“小哥,你这是啥意思啊?你这不存心消遣我们吗?”

闷油瓶不说话,拍了拍他,让他坐下,抓住他背上一根箭的箭头部分,用力一拧,就轻松拔了下来,我凑过去一看,那胖子身上只有一个浅浅的红印子,并没有受伤。

我惊讶的同时,心中也大喜,隐约感觉自己可能不用死了,忙学着闷油瓶的样子,去拔身上的箭,这东西一点也不难,我一子就自己拔出来一只,一看就明白了,原来这箭的箭头做的很巧妙,只要一撞上东西,锐利的头部就会缩进去,然后从箭头部翻出几只抓子一样的铁钩子,死死的咬住你的肉。

闷油瓶看了看满地的箭簇,轻声说:“刚才那一脚,那个女人是故意踩的,看来她不仅对自己的身手很自信,还想把我们全部干掉。”

我想起她刚才的飞吻,摆明了是在嘲笑我,气的都咬出牙血来了,果然是漂亮的女人都不可信,这亏我以后绝对不会再吃了!

胖子的背上几乎都是破皮,他咧着嘴巴,说:“幸好他妈的这里的箭都是莲花头,要不然还真给她得逞了,想胖爷我一世英名,如果死的时候被射成个刺猬,还不给人笑死。”

我看了看这奇怪的箭,问他们道:“为什么这里的箭都用是这个箭头的?这有什么用意吗?”

闷油瓶说:“我也不知道,但是一看你中箭就发现这是莲花箭,我想不起其他理由,或许是这墓室的主人想放我们一马,让我们知难而退。”

我觉得奇怪,这有点说不通,不过现在也不是讨论这个时候,那女的已经进了主墓室,不能让这个三八这么轻易拿了东西逃走,想着就想冲进去,闷油瓶子抓住我哦,摇了摇头,说:“刚才那只罐子鬼要我们先进左边这个墓室,肯定是有原因,我们还是按照步骤来。现在在人家的地盘上,不要乱跑。”

我一急,要是那女人等一下出来,直接跑了,也不知道去哪里追她。那胖子说道:“不怕,我们先回去把潜水的东西都藏起来,他娘的,看她能不能一口气憋到外面去!”

关键时刻还是胖子脑子活,我心说自己怎么没想到呢,马上点头,三个人快步跑回那个耳室,我用手电一照刚才放东西的地方,一看就傻了,那地方什么都没有——我们的氧气瓶竟然都不见了!

\chapter{第一次解迷}

我们三个人都呆住了,我们这一来一回也就是五分钟左右,任凭谁也不可能在这么短的时间将我们的装备统统搬走,而且从耳室到俑道,只有一条路,这些东西能搬到哪里去?

三个人对视一眼,脸色都不好看,这真是一波未平,一波又起,胖子这个时候也害怕起来,说:“难道这里还不只一只粽子?”

我摆摆手,现在不是讨论粽子的时候,这粽子我们尚且可以拼命,没有潜水设备,我们怎么通过那几十米长的海底墓道,这问题非常的严重,弄不好我们几个都要困死在这水底的墓穴里。

我问胖子:“刚最后一个脱下装备的是你,你过来放的时候有没有挪过地方?”

胖子说道:“当然没有!这8个钢瓶份量这么重,我吃饱了撑的搬来搬去。”

我心想也是,那个时候我们都在场,要是谁把这些东西挪了地方,肯定能知道,而且这东西的确很重,要想一口气全部搬掉几乎是不现实的。

我们在那里发了一会呆,胖子见干想也不是办法,就提议四处去找找,说就算是有鬼来搬东西,也必然会留下什么线索。我心想也是,就跑去把一只只瓷罐搬开,看看是不是给藏在后面了,这其实有点自欺欺人,这么丁点大的地方,如果有什么东西,一眼就能看到,但是那个时候只能死马当活马医。

我们找的非常细致,足找了五六分钟,我越找觉得越不对劲,又不知道问题出在什么地方,只觉得这里所有的东西都有一种说不出的古怪。最后还是胖子发现了,他突然大骂了一声:“娘的!这里根本不是刚才我们呆的地方麻!”

我转头过去一看,只见他的手电照在角落里,我记忆里那里本来是什么都没有,现在竟然有一根石柱,一边嵌在墙壁里,另一边露在外面,上面雕了很多的珍禽异兽,这是与刚才完全不同的一种墓室结构。我们马上再看其他三个角落,果然,四个角落都有一样的变化,我脑门上开始冒汗,这不仅仅不符合常理,简直是匪夷所思啊。

我看向闷油瓶,他点了点头说:“他说的对,这里似乎是另一个房间,那边角落里的那只婴儿棺材也不见了,陪葬品的摆设也非常不同,而且,你看顶上——”

我抬头一看,吓了一大跳,只见宝顶浮雕上的阴阳星图竟然变成了两条互相缠绕的巨蛇,盘绕在整个圆梁上,刻的栩栩如生,好像就要扑下来咬我一样,我看的心里发悚,忙低下头说道:“这是怎么一回事情,难道我们进错门了?”

胖子说道:“怎么可能,这里明摆着是自古华山一条路,这地方又大,我们从这里去了那破道,在破道里被射成刺猬又跑回到这里来,没错啊!他娘的这样都能错我王字倒过来写?”

我这个时候已经意识到,有可能我们也碰上了三叔二十年前遇到的事情,不过眼下的情景又和他叙述的有点不同,不知道这里面生了什么变故。当时三叔并未脱下身上的潜水设备,才能够侥幸从这泉眼里逃出去,而我进来的时候,明明知道可能会发生这种事情,竟然一点都没有做防备,我想到这里,不由有点自责。

胖子已经被搞的有点懵了,问我道:“你们南派不是对古墓里的机关很熟悉吗?这样的事情你以前见过没?”

我当然是没见到过,叹了口气:“这里也没外人,我就实话和你们说了吧,我这还是第二次进斗,不要说什么巧石机关了,我连这些瓶瓶罐罐的名字都叫不利索,你们也别指望我。”

胖子听了还不信,说道:“小同志你可别吓唬我啊,我还真指望你能看出个门道来呢。”

我苦笑了一声,也不知道怎么回答他,对他说:“现在这情况这么离奇,就算我真的是精于此道,估计也没有办法,你看这几分钟的工夫,什么机关能把一个房间里的陈设全部都变掉,连房子的结构都改了?这是不可能做到的,肯定有别的原因。”

闷油瓶淡淡的点点头,表示同意,胖子挠挠头说:“那不是机关是什么?难道是法术?”

我听他一提到这个,倒也想起来,说:“怎么说呢,也有这个可能,我以前听过一个故事,说是一个倒斗的进了一个古墓,发现里面富丽堂皇,像一个宫殿一样,里面竟然还有一个人在喝酒,那人看他过来,不仅请他喝酒,还送了条腰带给他。他和那人喝了好几杯,就醉倒在古墓里了,醒过来一看,自己倒在一个破败的棺材边上,那腰带是一条蛇。不是和我们现在的情况有点像?”

胖子说道:“像个屁,那他他娘的至少还有酒喝,我们只有水,怎么和人家比。”

我一听也是,这个时候,我有点犹豫要不要把三叔的事情告诉他们,主要是这事情没头没尾的,说出来有可能会牵涉到闷油瓶,我现在还不知道他的立场是什么,万一一句说的不对,麻烦更大,想来想去,我打定注意,说一半瞒一半。

那胖子还在那里唉声叹气,我让他们坐下来,把一些关于三叔的事情,挑了一些说了出来,胖子不停的插嘴,我实在说不下去,只好越说越简短,最后胖子竟然大骂:“臭小子,你他妈的知道这么多都不说,简直可恶,你看现在可好,弄了个半死不活的境地!”

闷油瓶听的入神,这个时候一把抓住我,问:“三叔昏迷的时候说了什么?你再说一遍!”

我看他表情这么严肃,结巴道:“他,他说的是‘电梯’。”

闷油瓶哦了一声,突然一笑,说:“原来是这么一回事情——”

\chapter{继续解迷}

他起身走到俑道石门处,摸了摸门框,说:“这的确是一个机关,而且还十分的简单,只能骗骗小孩子,所以你三叔二十年前看不出来,二十年后就能发现。”

胖子看他似乎知道了什么,说道:“小哥,你知道了就快说,别卖关子了,我他娘的急死了!”

闷油瓶说道:“我举一个例子,你一听就明白,如果有两层楼房,每层有一个房间,你从二楼的房间走出来,这个时候,我在这一楼的底下再盖一层,等你回来的时候,二楼的房间已经在三楼了,而一楼的房间变成了二楼。”

这个例子其实举的不好,胖子听的莫名奇妙,伸出两个手指,在那里琢磨:“一二,二一,一二一,他娘的什么一二三的,越说我越糊涂!”

我是一下子就听明白了,三叔所说的电梯竟然是这个意思,看样子他刚发现这个秘密的时候,脑子里第一个想到的就是这个词语,我感叹的同时心里不由一震,这真是既在意料之外,又在情理之中,而且这么一个结构并不复杂,确实只能算是骗骗小孩子的把戏。

我看胖子实在没办法听懂,又和他解释了一遍,他这才明白,突然兴致索然,说道:“原来如此,他娘的还真是简单,我还以为有更大的玄机在里面,原来不过如此。”

我心里暗说惭愧,我本来就是学建筑的,这个机关完全是建筑学的范畴,我竟然一点也没有发觉,真的应该检讨一下。看来凡事还得往简单处想才是道理。

闷油瓶的表情并没有轻松起来,他仔细检查了门框后,又走过去看泉眼里的水,看他的举动,似乎还有什么没有想通,我问他道:“怎么,还有问题?”

他点点头,说道:“三叔说的经过,和我们的经历,有一个很大的矛盾,不知道你有没有发觉。”

我疑惑的看着他,其实我也觉得他刚才提出的说法,有点不妥当的地方,但是我又想不到是哪里,闷油瓶说:“三叔是躺在这个房间里,并没有走出俑道,无论房间再怎么升降,他看到仍旧应该是这个房间,怎么可能会变化呢~”

我心里一亮,的确是这样,他又说道:“而且,古墓中的耳室,从来是左右对称的,不可能只有一间。按道理,我们的对面,应该还有一个房间才对。”

我们走进俑道,拿起手电照了一下,对面只有一面汗白玉的砖墙,并没有什么门,闷油瓶耳朵贴在墙上,两只手指按住砖缝,一点一点的摸过去,摸了有十几分钟,走过来摇了摇头,看样子是块货真价实的砖头墙。

胖子等的不耐烦了,打了个哈欠说道:“也别管什么耳室了,他娘的出去的路还没有找到呢,就算知道了是怎么一回事情,还不是照样死?”

胖子说的很有道理,我叹了口气,心想着三叔怎么两次都能逃出来,他到底是用什么方法的,他第二次出来的时候身上也没有潜水设备,难道他是硬生生从古墓里闭气游出来的?

他所经历的事情当中,必然还有一些什么我不知道的,可这老油条就是不说,三叔啊三叔,你可知道你几句轻描淡写的扯蛋,可能就要把你的侄子给害死在这十几米深的海底了。

他们两个都不出声,似乎是在思考这整个事情,我心里也盘算了一下,其实要从古墓里出去,无非是几条道路,一是原路返回,这当然是不可能的,除非我们的肺活量能和海豚一样,这第二就是找到当初工匠们留下的秘密通道,这在旱斗是事备功半的办法,但是在海斗里,恐怕也不现实,因为沉船葬海底墓是整个在船上修好之后再沉入海底的,就算用通道,也必然是通到海里,这水就成了隔绝阴阳最便利的媒介。

第三,就是最笨的招数——直接挖出去。我抬头看看宝顶,只看见累累砖头,不由长叹一口气,看样子就算能挖的动,也是个巨大的工程。

我试着自己来设计这个海底墓,看看如果按照最简单的建筑原理,这宝顶之上会好是什么东西。

现在可以肯定的是,光是砖头肯定是无法形成气密结构,在砖缝里必然有密封有的白膏土,上面应该还有木板子上多道火漆做隔水密封层,然后最上面再上膏土。

想到这里,我突然灵光一闪,已经有了一个很大胆的计划,我兴奋的对他们说道:“其实我们也不用怕,我估计我们离海面也就十几米,这个墓室为了容纳这个电梯的机关,必然要造的非常的高,墓顶离海底也不会太远,实在不行,可以直接挖上去,这海斗上面的水并不是很深,如果在退潮的时候做,我估计只要上面的沙子不塌下来,还是有机会出去的。”

胖子挥挥手,懒洋洋的说道:“我们进来的时候也没带什么工具,上面都是整块的石砖,用什么挖,用手吗?”

我说道:“这你就不懂了,沉船葬海底墓,大部份的砖头都是空心的,能压不能砸,我们只要能找几个金属的东西,用力敲几下,肯定能搞出个洞来。”

胖子一听,整个人一振,说道:“哎——这办法听上去兴许能行,我们也别他娘的倒什么斗了,直接翻点工具出来,这墓这么大,那主墓室里肯定有赔葬的铜器。”

这人就是这样,如果自己死定了,就什么事情都不想去做,但一知道还有一线希望,全身的智慧都会调动起来。我脑子转的飞快,一下子心里就有了一个这个洞打法的腹稿,我在大学里是学建筑的,这东西我太熟悉了,仔细一推敲,所有的方面都符合条件,只要这洞能在退潮的这几十分钟内完成,逃出去的可能行很大!

这时候闷油瓶说道:“离退潮还有很长时间,这里的空气不知道能不能撑到那个时候,一切还要看天意。”

胖子跳起来,说道:“他娘的蛋的,那就管他退潮还是不退潮的,先找家伙凿开来再说,这么闷死太憋屈了,我宁可找只粽子痛痛快块的被咬死!”

我本来想告诉他,如果在没有退潮的时候挖穿,头顶上的水起码有2米深,这水一下子冲进来,不要说爬出去了,这墓室这么大点空前全部灌满大概也只要几分钟。不过我看他兴致这么高昂,不想打击他。

我们三个振作精神,整理一下东西,就往俑道走去,刚出那甬道的石门,三个人同时一愣,胖子骂到:“这地方他娘的也太邪门了。”

在我们面前,本来还是那一堵砖墙的地方,竟然出现了一个门。我用手电一照,就照到那门里面,有一只巨大的金丝楠木棺。

\chapter{开棺}

鉴于对于这个墓室上下双层结构的推断,这里出现一个门我已经不觉得奇怪了,必然是刚才我们谈话的时候,这一边的房间也发生了上下偏移,虽然还不知道这墓主人这样设计的用意,但是我也不会再次慌张。

倒是里面这是棺材,吓了我一大跳,这金丝楠木是上上等的棺材料子,几千年以来,棺材的大小都是起决于木料的大小,这棺材的个头巨大,看样子实际用来做棺材的楠木原木,恐怕和明长陵里那32根用来做巨柱的金丝楠木差不多粗细。这东西可能比等身的一块白银还要值钱。

可是这样贵重的棺材,怎么会放在耳室里这么古怪,如果这样贵重的棺材都只能放在耳室里,那主墓室里最起码是只金棺才行,我感觉到越来越莫名奇妙,这墓室的主人,毫无规矩可言,不仅把这里的风水位置全部打乱,而且到处设下极其机巧的陷阱,却又不取人性命。不知道到底想干什么。

倒斗的看到棺材免不了会手痒,特别是这么气势磅礴的一只,里面必然会有不少好东西,我看到胖子看的眼睛都直了,笑道:“怎么,看到棺材就连命也不要了,要不先进去捞几件出来?”

我这是讽刺他,谁知道他没听出来,一本正经的说道:“你胖爷我觉悟高,现在我们的主要任务是找工具来把这狗日的墓顶搞穿掉,你别给我开小差,等我们弄来了家伙,再回来捞几样也不迟!”

我一听他吹鼻子上眼了,也觉得好笑,说到:“等你回来,鬼知道这门还在不在。说不定又翻下去了。”

胖子还是想这明器的,一听觉得有道理,不由为难起来,这个时候,闷油瓶突然对我们摆了摆手,轻声说:“别说话。”

我们看他表情严肃,忙捂住嘴巴,不知道发生了什么事情,他拔出气枪,轻声说道:“这不是一般的棺材,这是养尸棺。”

我一听没听明白,疑惑的望向他,可他根本不想多解释,一猫腰就走进了放棺材的耳室,胖子本来还想维持自己觉悟高的形象,一看闷油瓶老实不客气就奔那棺材去了,马上恢复自己觉悟低的本性,急忙跟了进去。

我一看甬道里一片漆黑,自己一个人呆在外面太恐怖了,不敢怠慢,也跟着跑了进去。

这斗室和我们来的那间一摸一样,宝顶上是两条巨蟒浮雕,中间一个泉眼,只是没有那些瓷器陪葬,只有一只巨大的棺材离墙三尺放着。

闷油瓶抽出军刀,直接插进棺材缝里,慢慢的划起来,似乎在找什么机关,胖子以为他要开棺材了,大叫:“慢点慢点,看你这小哥平时这么老实,怎么看见棺材就像不要命一样。”说着就拿出个蜡烛跑到角落里想点。

我一看,气的大骂:“他妈的我们就这么点空气了,你还点蜡烛,你不要命了。”

胖子没好气地说道:“一只蜡烛能烧你多少空气,大不了你胖爷我少呼吸几口。”说着就打起来手里的防风打火机,那火光一亮,突然就照出角落里的一个东西,胖子平时胆子够大,也被吓得一屁股坐到地上,我看他倒地,忙打上手电一照,不由也吓得缩了一下。

那角落里竟然蹲着一只干瘪的死猫,个头奇大,但是已经成干尸的状态,两只眼洞直勾勾看着胖子,身上大部分的皮都掉了,下巴张开着,露出一排獠牙,看上非常不舒服。

我从小最怕死猫,小时候家里人经常把抓住偷鱼的野猫掉死在树上,任其腐烂,我那个时候小,不懂那是什么,结果有一天在树下玩的时候,上面一具猫尸脖子腐烂的断裂,猫头一下子就掉在我手里,我一看到那獠牙和眼洞就吓得尿了裤子,几天魂都没回来。

胖子看到的眼前不过是具猫尸,不由骂了一声,一脚把它踢开,然后点上蜡烛。就往棺材走去,我感到有点不对,墓室里竟然有猫尸,难道不怕起尸吗?

不过这地方不和情理的地方太多了,我隐约感觉到,似乎这墓室的主人故意在反着规矩做事情,什么都按规矩的反面来,墓室不能有什么,他就放什么。这样下去,到了主墓里不知道还会碰到什么事情。

这个时候闷油瓶已经找到了那棺材的八宝玲珑锁,拿出百宝盒,用里面的两个钩子在棺材缝里一勾,喀嚓一声,机关破解,同时整个棺材盖子往上一弹,一股黑水就瞬间涌了出来。胖子也顾不得恶心,一下子推开棺材盖子,往里一看,吓得大叫:“狗日的,这么多粽子!”

\chapter{一个人}

这棺材盖子一开,我就觉得一股腥臭的味道扑面而来,凑上去一看,只见棺材里全是黑水,上面水雾缭绕,湿气腾腾,下面隐约可以看到肢横交错,也不知道有多少尸体在里面,都已经蜡化并粘在了一起,成一个巨大的尸块,我光手就能数出12只,这情景别提有多恶心了。

闷油瓶看到这个情景,皱了皱眉头,但是表情已经一松,手里的枪也垂了下来,看他的变化,我估计这东西应该并没什么危险,不知道他刚才紧张的是什么。

棺材里面有几溜暗金色的圆钉,每隔几公分就从上往下钉上一排,在水里也看不清楚是纯金的是还是镏金的,那尸块的下面有一块奇怪的东西,胖子用手电从下往上一寸一寸的照,看着似乎是一块刻着字的石板。尸体之间以及手上,都有玉器和象牙器,这种东西价值连成又好携带。

胖子看着心痒,但是那尸体太恶心,任他再莽,也不敢把手伸进这飘着一层人油的棺材里捞东西,他琢磨了半天,也没想出办法,只好放弃,转去研究里面的尸体,一边看一边摇头:“这他娘的也太惨了,还说这个墓主人是修道之人,这么阴邪的东西都摆了出来,怎么可能得道,活该被我们来倒斗。”

我一直不明就里,只是看到这里面的情景,觉得神经有点受不刺激,不敢再看第二眼,问道:“这合葬棺怎么这么恶心。”

胖子失笑:“小同志,你傻了吧,你看到谁合葬葬的像麻花一样?这东西明显是活葬葬下去的,这些人堆在一起,被下了药灌水闷死在里面,这叫养气藏尸。”

我听他说到麻花,就觉得喉咙直发痒,我这个时候肚子已经很饿,这个尸块和大麻花重叠在一起,感觉简直胆汁都要冲出来,不过听他的话,好像也知道这东西的背景,我定了定神,就问其详细。

胖子看我不懂,有心买弄,说道:“你连这也不知道?那这可就是小孩没娘,说来话长了,话说我当年还在长白山的崇山峻岭——”

我听他又开始胡扯,说到:“你少他娘的给我扯这些,也不看看是什么时候,这养尸关长白山什么事情,不知道就别扯鸡吧蛋!”

胖子这种人就怕别人激他,脖子一硬说:“谁说我不知道的,我只不过想从大处说起,你不想听就算了,这东西叫做养尸棺,是风水上的学问——,一般啊用在什么山陵里,如果有这个棺材,说明这个古墓里有两个风水极好的棺位,如果不在棺材位上都放上棺材,那个空出来的棺位因为聚着海川的灵气,就会招惹来那些带妖性的东西,所以在这放一个养尸棺,里面葬上墓主人的一个有血缘关系的人,算是合葬,这个棺材必须和主墓室里的一模一样,这在风水上叫做养气,懂不?”

胖子背书一样一口气说完,我听的半懂,不由咋舌:“那这里面的这么多人,都是——”

胖子一拍大腿:“所以说嘛,这人他娘的可能把他的全家全部都给塞进去的,太惨了!”

我大叫:“怎么可能会有这种事情,这选好的风水,本来就是为了后代着想,现在把全家一齐葬了,风水好还有个屁用!”

胖子看我还当真了,说道:“说什么你还信什么,那些有钱人哪有这么笨,肯定是找了几个外戚的穷侄子来陪葬,这东西,明墓里最多,我见过不少,不过没见过这么大的。”

我看着这尸块,想着下葬时候的情景,心里也不免动容,还是爷爷那句话,人心是最不可测的,为了一点点根本没有事实依据的事情,这些人的命就如果草芥一样被夺去了。

不过既然棺材盖已经开了,胖子想必也不会这么轻易罢手,他挠了挠头,说道:“看这些人这么可怜,我看要不我们去隔壁拿几个罐子来把这些水都舀出去,棺中积水是最不吉利的。”

我知道他想干什么,说道:“看你这贼样,就知道你还在打这些冥器的主意的,你就不能给我安稳点,呆会冥殿里有的是东西给你拿。”

胖子脸一红,骂道:“他娘的你胖爷我是这种人吗?”

我也懒的和他扯蛋,说道:“现在也不是管这个闲事情的时候,等一下我们出不去,闷死在这里,恐怕连个棺材都没有,到时候可没人来可怜我们。”

提起这个事情,我们马上又紧张起来,胖子二话不说,先在这耳室里找了一圈,可惜除了一只猫尸之外,其他可以利用的东西都没有。

闷油瓶一直在呆呆的看着那堆尸块,他看了很久,突然好像看出什么,吸了口凉气。

这个人平时非常镇静,一但紧张必然有大事情发生,所以他这一个动作,我被吓了一跳,忙猫腰举枪。

他还是眉头紧皱的站在那里,死死的盯着棺材,足足沉默了有五分钟,才转头对我们说道:“这里面,其实只有一个人——”

\chapter{瓷画}

我刚刚才明白胖子说的养气藏尸是怎么一回事情,闷油瓶又冒出来这么一句,还说的没头没尾,我一时理解不了,就问他怎么回事情。

闷油瓶一指棺材,说道:“你仔细看他们的头,有什么区别。”

我顺着他的手指看去,只看见6个脑袋有大有小,像一串葡萄一样挂在躯干上,除了恶心之外并没有任何特殊的地方。我摇摇头表示我看不出来,他又让我再看仔细点,这次我眯起眼睛来看,终于发现了一个问题。

原来这堆尸骸,除了最上面的那个头之外,其他几个似乎都没有五官。不仅如此,连基本的头部骨廓都没有,看样子像一些巨大的肉瘤长在上面。

看到这里我已经明白了他的意思,马上顺着他的思路找了下去,又发现每只手的关节,似乎真的都连在一跟躯干上,只不过这躯干已经扭曲的非常厉害,好像是放在洗衣机里脱水过一样,加上这黑水浑浑浊浊的阻碍人的视线,所以看上去就像很多的尸体拧在一起。

我越看心越发寒,但是对于结论还有一些保留,如果这棺材里躺的是一个长着12只手脚的罕见畸形,那他的来历和身份到底是什么?在那个年代,这样一个怪物,为什么会被养育到这么大。

胖子也看出了门道,吐了涂舌头对我们说道:“我的姥姥,这东西是人吗?简直就是一只虫子!”

他这话形容的贴切,就是比较缺德,我说道:“我们隔着水看不清楚,下结论还为时太早。按道理上来讲,这么严重的畸形,简直就是一个妖孽,刚生下来的时候必然会被父母弄死,绝对没有机会养的这么大。”

闷油瓶淡淡说道:“凡事无绝对。”

我摇摇头,还是不能全信,胖子说道:“要知道其实也很简单,不如按我说的,去隔壁拿几个盆子来把这水舀了,好看的清楚点,而且你看这尸块下面还有块石头板,我们一并弄出来瞧瞧,说不定还有什么意外发现。”

我一听来了兴趣,进到这个海斗以来,我连一个文字都没有看到过,对于墓主人的认识还是一片空白,如果这块石板上的文字我能看懂,至少我也能推测出个一二来,对我们的行动说不定也有帮助。

我和胖子一拍即和,二话不说就转身回到俑道对面,挑了三只有柄的瓷碗,这些东西在外面都是百万珍品,在我手里算是还了本原,该是什么是什么。

出于职业习惯,我拿起这碗的时候,不自觉的就看起上面清花釉来,这一看我就一呆,没想到这上面的花纹,竟然都是一些叙事的图案。

大概是进来的时候一心想着三叔的事情,也没仔细去研究这些陪葬品,现在看到,我马上就想起一个很不起眼的事情:三叔在进了这个斗以后,也和我一样,只是粗略的看一下这些东西就去休息了,但是其他那些人不同,那些人第一次进斗,非常的兴奋,必然仔细的研究了这些瓷器,难道这上面还有什么关键性的东西!

我想到这里,忙拿起几只碗仔细去看,发现这些画都是在讲一群人在修建一个土木工程,有修石头的,有运原木的,还有搭木梁的,这瓷器摆放的顺序就是工程的进展顺序,我越看越有震惊,头上汗都出来了,胖子看我在那里一个一个的琢磨瓷器,奇怪道:“挑个罐子有这么难吗?别挑了,随便找个称手的就行了。”

我根本没听进去,趴下来边爬边看,一直看到最后一个八角瓶子,上面的图案是一个巨门打腰子的情景,再往后就没了,看样子应该还有更多的东西记录在别的瓷器上。

我看的简直是惊心动魄,连呼吸都喘不过来,虽然只凭这些简单的画还看不出来他们到底是在修筑什么东西,不过看里面的描述,这个工程浩大的程度,几乎已经和故宫差不多了,然而上面的结构完全不是中原的风格,他娘的我实在想不出那个时候中国哪里还有这么大的建筑。

我回了回神,就想把这个惊人的发现告诉胖子,转头一看,只见背后一片漆黑,胖子早就不知所踪。

我一愣,心里直骂,这死胖子也真是的,走了也不和我先说一声,知道我一个人不敢呆在这个地方,我随手拿了个盆子,站起来就急急往对面的耳室跑去,刚进俑道,我就呆了。

只见对面耳室的那扇门竟然没了,又变回了那汉白玉的砖墙!

我只到是机关的原因,但是没想到这机关竟然如此迅速,连一点声音也没有,不由慌起来,一个人呆在漆黑一片的古墓里,这种事情我可再也不想经历了。

我冷静了一下,自我安慰说,这墓室的活动非常频繁,只要我能够耐心的等待,估计几分钟之后,那门必然又会出现。

可没了胖子在边上,这古墓里安静的吓人,连心跳都像打雷一样,四周又暗的离谱,在这种地方,一分钟就像一个小时一样,实在没法子耐心的等待。

我深吸了一口气,用手电照了照前面三个黑洞洞的门洞,也看不到里面有什么东西,这世界上最恐怖的东西,永远是在自己的心里,我只要一静下心来,总觉得那门里有什么东西正看着我,悚的要命。

我拍了自己一个巴掌让自己平复下来,低头就往耳室里走,想着再去看看那些瓷器,免的看漏了,就在这个时候,我突然听到了一声让人毛骨悚然的叫声从耳室里传了出来,拿手电一照,只看见一只巨大海猴子正从泉眼里钻出来,半个身子已经爬上了岸,那张张满鳞片的狰狞巨脸,我这一辈子都不会忘记。

我叫了一声我的姥姥,撒腿就往甬道里跑,也不管有没有其他机关,闭着眼睛,一路冲了过去,眼看就要成功到达安全地带了,突然脚下一绊就一个狗吃屎倒在那只罐子边上,还好我反应还可以,就势一滚就坐了起来,回头一看,只看见两只闪着绿光的眼睛急速冲出耳室,径直向我冲了过来。

我一咬牙,一把抱起那罐子,就想砸过去,那海猴子反应非常快,看我有了武器,也不硬冲,马上就转向跳到甬道顶上,我趁这个机会,直溜一声就转到左边的玉门里,一下子就把那玉门重新推上。

那玉门下面是有一个自动的石栓,门一合上那石栓就自动弹了上来,海猴子在外面撕叫了几身,狠狠撞了几下门,看样子非常的不甘心,我知道这种门材质非常坚固,血肉之躯是绝对撞不破的,忙定了定神,那海猴子见撞了半天没反应,竟然想从门缝里钻进来,我看着它那大脑袋直往里蹭,心头火起,举起气枪,直接对着门缝就是一梭镖,也不知道打在它什么地方,只听那海猴子惨叫了一声,一下子就跳开老远。

我不知道隔壁的那门和这个墓室是不是相通的,忙有搭上一只梭镖,然后把手点和矿灯都打开,一下子就几乎把这个墓室整个儿照了出来,我一看,吓了一跳,只见这是一个巨大的圆形墓室里,中间竟然有一个巨大的水池,我的脚就站在水池的边缘上,再一往后一步,就必然要掉下去。

水池的中间,浮着一只巨大的洗脚盆一样的东西,静静的停在池的中间,我看到他上面的描画和浮雕,就知道,这必然是一只棺椁。我不由想笑,这个墓主人还真会想,把自己的棺材修成一个澡盆的样子,看样子他身前必然很喜欢泡澡。

我又往水里照去,只见这水简直深不见低,不知道有多深,说不定一直就通到这个墓的底部,正在寻思这到底是一个什么样意图的设计,突然就觉得脖子痒了起来。

\chapter{无题}

我回后摸了一下,才知道是刚才莲花箭中箭的地方,那四只铁钩子嵌进我的肉里,虽然没有取我的性命,但仍旧刮去了几块皮肤,现在汗水滴下来,竟然刺激的痒起来。

不仅如此,身上其他几个中箭的地方,也开始隐约有点发痒,不过这痒尚且可以忍受,我无暇顾及这些身体上细微的感觉变化,使尽揉几下后,就继续去研究那个这个奇怪的墓室。

我并不了解明代普通的墓葬地宫结构,只知道一点贵族墓葬的知识,不知道这两者之间的区别大不大,只好勉强将眼前看到的和知道的对号入座。

按照我的想法,我现在呆的是左配殿,对面与我相望的是右配殿,左右配殿应该互相对称,里面按道理应该各有一个用汉白玉垒起的棺床,棺床平面用金砖(澄浆砖)平铺,中央会有一长方形孔穴,内填黄土,称为“金井”。现在这些全部没有,只有一个大水池。

这只是其中一个奇怪的地方,另一个就是在两个配殿中间的那个门,应该是通往后殿,那才是放棺椁的地方,何以现在配室里有棺材,而且还是脸盆的形状,要知道这种盆棺是战国时期的东西,明朝是完完不会有的。

说起战国,我又想起了鲁王宫里的拿出来的蛇眉铜鱼,这两个地方都发现了这个东西,而这里又有一个战国时期才会有的棺材,难道是巧合吗?

一时间想的心乱如麻,再也想不进去。

这个时候我已经围着那水池走了一圈,有回到了门口,那只被我用来当成凶器的大瓷罐倒在那里,我心里一动,就随手拿起来看上面的瓷画。

这应该是另一个耳室里的东西,但是单幅的图案并不能表达什么信息,我只看到一个穿着明朝服侍的人,站在一座山上,看下面的一个工地,旁边还有几个穿着官服的人,看样子,是一幅视察工地的情形。

我通过这些瓷器上的图案,大概能猜到这个墓主人必然不是什么皇公贵族,很有可能是一个工匠或者建筑师,只有这种人才会有能力和知识,在古墓中使用如此古怪的设计,其他的人就算有这个想法,也没能力建造。

而明初的能人巧匠并不多,看这个墓的规模,必然是一个地位显赫,能派的上号的。这个人不仅要有这个资格修建一个像明皇宫一样浩大的工程,又必须懂风水和奇淫巧术,这样的人其实也不难猜测。

我只想了几秒,一个名字就跳进了我的大脑里——汪藏海。

这个人可以说是一个奇人,他在风水上的造诣可以说是登峰造极,就因为如此,他被任命直接参与设计了整个明皇宫,还附带设计了好几个中国的大城市,那个时候,他的一句话,甚至使得几个城市在中国彻底消失。我在古籍中还了解到他有一本关于风水的著作,里面的内容深奥到了极点,简直可以说窥见天机,可惜他的后人只抄录了几本,都已经失传。

而且,相传沈万三在周庄银子浜底下的水底墓,也是这个人设计的。这样一个人,为自己建造这样的一个墓穴,简直是绰绰有余。

我觉得自己的猜测很有道理,现在只要能找到一点点的文字资料,就可以知道我想的到底是对还是错,可惜这个墓主人好像是个文盲一样,一点铭文也没有留下。

这个时候,突然咕咚的几声从水池里传了过来,我一下子思绪被吓得一断,忙用手电去照,只看见那水池里的一个角落里,竟然开始有水泡冒上来,还时大时小,一阵一阵的,并没有规律,似乎这深不见底的水里,有什么东西正在活动。

我一下子慌了,马上端起枪,紧张的盯着那个气泡,突然一下子,一个白花花的东西一下子冲上了岸,一个打滚翻到墙边上,大口的喘着气,我一看大喜,竟然是胖子,上衣已经脱了,露出个大肚子在那里直鼓,他一边喘一边看到我,甩了甩手,说道:“他——娘的,我——差点就——憋——憋死了。”

我刚想问他是怎么回事情,突然脚边上又是一个人出水,我一看原来是闷油瓶也翻了上来,也裸着上身,可是身上的黑色麒麟不知道哪里去了,他明显没有胖子这么吃力,只是仰起头大大的吸了一口,看见我,说道:“这里是左边还是右边?”

我说左边,他松了口气,一下子也坐了下来,捂住自己的手腕,我看到他手腕上,有一个黑色抓印,突然有一股不好的预感。

胖子喘了半天才缓过来,捂着肚子直叹气,我问他们怎么过来的,他吐了几口口水,说道:“别提了,幸好你没看见,吓死我了。娘的,幸亏那棺材底下的石板子下面有一个洞通到这里来,不然我们就死在那地方了。”

我纳闷,问:“什么东西这么可怕?”

胖子对我说道:“我操,我连形容都形容不出来,就一句话,那六体连环尸肚子里,他娘的还有一只东西。”

\chapter{继续无题}

胖子说完又咳嗽几声,连吐了好几口口水,我听的着急,忙让他接着说,胖子挠了挠后背,说:“上吊也得喘口气,这事情发生的太快,我一下子也说不上来,你得等我组织组织语言。”

我看他那样子,真的是脸色发白,讲话的声音都阴阳怪气的,看来气管里还有水,忙用力帮他拍了几下背,他被我拍的人都缩起来,狂咳嗽出很多粘糊糊的东西,说道:“行了行了,再拍他娘的就被你拍死了!”

我催道:“行了就快说,你们到底遇到什么了?”

他醒了醒鼻子,就把他们遇到的事情和我简单叙述了一遍,事情发生的非常快,所以他的叙述也比较乱,但是我还是大概的知道了来龙去脉。

原来当时他看我在那里傻呆呆的看着瓷画,又催了我几声,可是我当时专心的要命,根本没有听见,他见我没反应也不来催我,大概是心里惦记着那些值钱的玉配饰,就先自己跑了回去干起来,他那个时候心里想的是,我挑完之后自然也会走过来,两个耳室不过五,六步路,必然不可能会什么意外。

可是,他接下来看到的东西,把他的精力完全吸引了过去,以至于他完全忘记了我的存在,也根本没有注意到那石门是什么时候消失的了。

他回到棺材边上,两个人一起舀水,很快那尸块就浮出了水面,胖子仔细一看,不由骇然,原来那几个他本来以为是头的肉瘤,其实都是女人肥硕的乳房,肥的都的挂了下来,拖在扭曲的躯干上,胖子当时就傻了,他可真没想到这竟然还是一具女尸。

可是,应该按道理既然有12只手,应该有12只乳房才对,可是正面才只有5只,难道背上还有?他们想着就琢磨怎么把这尸体从棺材里抬出来。

胖子先试着用枪当钩子把尸体勾出来,可是尸体太软,身体几乎全部蜡化,滑腻腻的根本没地方着力,带上手套用手更加不行,就像捏肥皂一样,一捏就下来一层油,恶心的要命。最后闷油瓶想了个办法,他们把衣服脱了下来,一个人包头,一个人包脚,用枪一穿做了个扁担,两个人一抬就把她抬了出来放到地上。

在探灯强光的照射下,尸体迅速的干化变黑,这下子他们看的透彻,另外的几个乳房已经被割掉了,留下了几个碗口大的疤在身体两侧,她的身体也并不是扭曲,而是由于身上肥肉横身,堆起来一像山一样。

当时他们也没想为什么这个女尸的肚子这么大,只道是太肥,根本就没有看出她其实是在生育期间死去的,肚子里面另有乾坤。

尸体抬出之后,就露出了下面的石碑,闷油瓶说,这是压棺石,是为了一旦这个海底的墓的气闭结构被破坏,棺材不至于浮起来。那快压棺石非常的粗糙,只刻了一列斗大的字。

胖子看了几眼看不懂,才想起我。直到这个时候,两个人才发现,那墙壁上的门已经不见了。胖子一看就慌了,倒不是担心我,而是担心自己出不去,闷油瓶让他别怕,说这门到时候自然会出现,急也没有用。这个时候最重要的是把手头上的事情做掉,胖子见他这么镇静,也松下心来。

两个人想把石碑从棺材里拿出来,却发现石棺材非常的重,而且四周浇了松汁,牢牢的粘在棺材底上。胖子一看,这不和情理啊,他用力敲了敲这石碑,突然发现下面竟然竟然是空心的。

他们点起火折子把松汁全部烧融,然后将石头搬开,下面露出一个大洞,胖子虽然人比较粗,但是他的阅历非常丰富,一看就惊讶的嘴巴都合不拢,这个洞还不是古墓的设计者特意做在这里的,这是一个盗洞!

这可是一个爆炸性的发现,其他方面下不说,光是这个盗洞在定位上可以说是天下无双,竟然直接挖到了棺材下面,如果不是有这一块压棺石档着,估计里面的尸体早就被拖入洞中,最离奇的是,这个墓位于海底,这个盗洞是用什么方式怎么打的?

而且,如果这个墓室上下电梯结构的,那棺材下面应该就是另一个墓室才对,怎么会空间可以容纳这么深的一个洞。当下胖子就肯定,我们关于墓室机关想法,可能是错误的。

这一下子整个事情又堕入了迷雾里,两个人同时沉默了,胖子心里很清楚,因为这个洞,这里养气藏尸的局已经被破坏了,这具尸体虽然已经蜡化,无法再尸变,但是这个地方的势已经不在,必然对整个墓穴的风水造成了影响,虽然现在不知道总体的变化如何,但是难保不会突然从一个灵穴变成一个败穴,胖子在风水上造诣虽然不高,但是到底是北派的人,他知道这种转变非常不妙。

可是他到底不是这方面的专才,一往细处想,脑子就不够用了,他认为这几个石碑上的字可能是关键,就描了下来,这个时候,他听蹲在女尸边上闷油瓶突然大叫,“糟糕!”

他一转头,竟然看见闷油瓶的左手被从女尸体内伸出的一只长满白毛的小手死死抓住。胖子没想到那女尸肚子里还有一个死婴,吓了一跳,不过他反应到底是快,回过神之后马上拿枪,上去对着那女尸的肚子就是一梭镖,这一下子似乎打的正是地方,闷油瓶一下子就挣脱了,胖子还想再射,闷油瓶大叫:“射不死的!快走!”说完就拉他去钻那棺材里的盗洞。

胖子一看那里面有剩余的棺液流下去,恶心的要命,一下子还下不去脚。可是回头一照,只看见女尸肚子上凸出一张脸的形状,好像拼命想钻出来,那女尸体肚子上的皮已经被拉的透明了,连里面那东西的五官都看的清楚了,他不由后背发寒,心说君子不吃眼前亏,一咬牙跟着钻了进去。

盗洞是开砖头挖出来的,做的非常的巧妙,把所有的砖头只敲掉一半,这样就能自然在盗洞的顶部做成一个拱形的砖梁。上面的东西不会压下来,这种手艺真的是考工夫,估计没个几天时间还完不成。

闷油瓶已经爬进去有几步了,胖子在后面拼命的追,他也不知道这盗洞到底通向哪里,爬了没几步,突然就发现盗洞往下倾斜,下面开始竟然有水,不过有水的一段似乎并不长,他看到有灯光衍射进来,预计到可能是我,就往水里一潜,才游了没几步前面果然就一宽,变成了一个大池,那个时候两个人都没气了,拼命浮上来,一出水马上就看到我拿梭镖对着他们。

我听到这里忍不住说:“敢情你也只看见一只手啊。”

胖子说道:“你胖爷我倒是根本不怕那东西,不过这小哥这么厉害,看到那东西都逃,你说我逞什么能耐,不过话说回来,我还真是不明白咱干嘛要跑吗,小哥,那东西是什么玩意,真有这么厉害?我看着就那点分量,给它来几梭镖,估计也能搞定啊。”

闷油瓶摸了摸自己的手腕,说道:“那只是一只白毛旱魃,砍掉它的头就能杀死,不过他一死大量尸毒蒸发出来,我们就这么点空气了,并不合算。”

我听了吃惊,旱魃一说是传说中能引起旱灾的鬼,又说是僵尸在养尸地里呆久了,就可能变魃,诗经上就说过,旱魃为虐,如惔如焚,总之关于这个东西的说法多之又多,没想到竟然是这么个样子。不过这些都并不是重要的,我进到古墓里来,早就预备见到稀奇古怪的东西,倒是那个盗洞,非比寻藏,竟然是通到这个水池来的,这不太可能,我估计这水池下面的盗洞口子必然只是一个出口,可能是这人打盗洞的时候,并不能肯定主棺的位置,就向几个可能方位都打了盗洞,这个只是其中之一。想到这里,就问他们有没有发现叉路?

胖子摇头说没有,这盗洞并不长,很明显是一路到底,我听了也并不沮丧,因为砖头洞嘛,要用砖头藏起个洞口,太方便了。

想着这个盗洞既然没有破块气密结构,其进口也必然是在这个古墓内,找到了也没有什么用处,我估计他肯定从泉眼进耳室后,耳室还处在无门的状态,他没有办法,只好影挖出了一个道来,不过这人也真倒霉,往耳室挖,挖到压棺石,往配室挖,挖到个水池,不知道主墓室有没有被他挖通。

想着,胖子突然说道:“你们说旱魃会不会游泳的?”

我一愣,不知道他什么意思,他指了指水里,我回头一看,只见那水池的中心,突然冒出了大量的气泡。

\chapter{石碑}

那些水泡均匀的冒上来,频率很快,同时还有向外扩张的趋势,似乎那水池底下有一只大家伙,正在不停的喘气。我们三个人都戒备起来,端起枪,后背紧紧贴着墙壁。我已经紧张的有点力不从心,手心里全是汗,不知道等待我的将是什么结果。那些水泡冒了大约有五分钟,突然水池底下传来一声令人费解的闷响。

与此同时,水池的水位竟然开始下降,水面上逐渐出现了十几个旋涡,只见水花飞溅,好像十几个抽水马桶同时在抽水,那只盆棺就随着水流拼命的转起来,就像一只陀螺一样。在一瞬间,水平面就下去了二三米,我看得莫名其妙,忙拿手电往水池里一照,竟然看见水池的内壁上出现了一道石阶,这石阶顺石壁盘旋而下,似乎是直通池底。

水下的非常快,我还没来得及仔细的观察就已经消失在漆黑一片的池底,只有旋涡的轰鸣还在不停的传来。我用手电略微扫了一下,发现这个水池是一个碗状,上面宽下面窄,足有十几米深,用手电的穿透力不够,加上下面水雾缭绕,池底隐藏在迷蒙的黑暗之中,什么都看不清楚。

我想起我们还有那种穿透力极强的深水谈灯,不知道它对水雾有没有作用,忙招呼他们打起来,并将光线调到最大,从三个不同的方向同时向下面照去。

这下子虽然没有照的通透,不过下面的样子算是勉强勾勒了出来,池底是一个10米直径的圆形平面,上面雕着浮雕,具体是什么图案看不清楚,不过倒是能肯定上面有好几个大洞,看来就好似下水的口子。

池底的中央囤着那团水气,里面黑影搓搓,不知有什么东西,胖子眼睛很毒,琢磨了半天,说道:“你有没有看出来,那池底上中间,好像有一块石碑?”

我顺着他指的方向看下去,只能看到一个轮廓,胖子说道:“这石头阶梯这样下去,不知道通到何处,说不定下面还有其他的通道,我们下去看看!”说着一跳就跳到了那个石头台阶上。

这古墓诡异异常,我并不赞成贸然下去,叫道:“你别急,这样下去太危险了,至少也要等到下面水雾都散了。”

胖子已经往下走了好几步,说道:“没事,我就下去看看,如果不好走自然会回来。”

我知道他的脾气,也不不拦他,看着他往下走了大概有两圈,似乎碰到什么,蹲下来去看,才看几了几秒就抬头对我们大叫:“狗日的,这里竟然有洋文!”

我听到这句话一愣,怎么可能,明朝古墓里出洋文,这是唱的那出和哪出,大声说道:“你他娘的胡说什么,古墓里怎么可能有洋文,你别是把花纹看叉了?”

胖子气的大骂:“你胖爷虽然洋文不好,他娘的ABCD总知道,你也把我看的太扁了!你要不信自己下来看!”

我说道:“那上面刻的是什么你给我念念。”

胖子简直出离愤怒,骂道:“我要他娘的看的懂,还用叫你下来!”

我本不打算下去,可这样一搞,不下去也不不行了,叹了口气,学着胖子一跳,跳到那石阶上,那石阶只有半米长,似乎是用整块的青刚岩架空而成的。一端插进池壁里,我用力跺了几脚,非常的稳固,没人坍塌的危险。这个时候闷油瓶子也跳了下来,我们一前一后,向胖子走去。

胖子站在那台阶上,就像一堵墙一样,他指着在池壁说:“快看这里,这他娘的要不是洋文,我把王字倒过来写!”

我一看,上面真的被人用凿子敲了几个字母出来,看痕迹不新不旧的,就想到有可能是20年三叔他们那批人刻出来,不由暗暗吃惊,难道三叔在睡觉的时候,这批人到过这个地方?那他们的失踪会不会就和这个奇怪池有关?

胖子看我发起呆来,用力拍了我一下,“到底是不是,快说啊!”

我忙点头说:“是是,我向你道歉,这还真他娘的是英文。”

胖子得意起来,一拍大腿,“我说怎么这么奇怪呢,这破斗找了这么久,连一点稍微好的点东西都没有,敢情是洋人兄弟捷足先登了,想当年八国联军来的时候,可没给我们剩下什么东西,这次不用说,估计啥也没了。”

我想了一下,说道:“也不能说是洋人,中国人也可以写洋文字,说到雕刻,雕洋文比雕中文所花的时间要少的多了,这几个字母都是缩写,我觉得可能是个标志,你看他的刻的非常的匆忙,恐怕是当时他往下走的时候,发生了什么紧急的事情,或是有人在催他,他为了给后面来的人留个记号,才刻了几个字母在这里。”

胖子说道:“听你这么一说,倒也对,你说他们到这下面去干什么?难不成有什么宝贝?”

我知道他又想到别处去了,不去理他,胖子追着我说道:“反正咱还有的是时间,不如下去看看,说不定能找到点青铜器当工具,岂不是一举两得?”

我看着下面,宝贝我是不稀罕,有命赚没命花的钱我才不要,不过如果在下面能够知道文锦他们的下落,倒是值得去看一下。正在犹豫还要不要继续走,突然听到边上的闷油瓶说道:“这地方我好像来过!”

\chapter{池底}

闷油瓶说完这句话,也不理我的追问,快步向下跑去,我看到似乎有一丝真相的曙光,自然不肯放过,忙追了下去。

水池底下的雾气在不断的上升,我才走了十个台阶,就进入到浓密的雾气中,能见度急剧下降,我刚开始还能看见胖子的背影,几步之后,前面能看到的只剩下一个手电的光点。加上那胖子胆子大,三步并成一步的跑,结果一下子就把我甩去好远,结果才下去一圈还不到,我连胖子手电的光点都看不到了。

这下子我有点慌起来,我现在是在一片云雾缭绕之中,往前往后往右都只能看出去半米不到,这种能看见有又不清楚的感觉,比在绝对黑暗里还难受。

池面与池底的垂直距离并不长,走了有一只烟的工夫,胖子就在下面叫道:“我这里已经到底了!”

我听到他脚踩到积水的声音,忙几步并作一步跑下去,突然脚一凉,踩进了水来。原来池底的水并没有全部抽走,还有大概到小腿深的积水,难怪我在上面向下看的时候,怎么也看不清楚。

我观察了一下这个地方,发现这里已经几乎是雾气的中心了,能见度更低,我摸着池壁走了几步,就听胖子在左边叫到:“你注意水下面,这里都是进水的洞,千万踩进去。”

我用脚探了探,果然,前后都有碗口大小的坑,看来在这里走路要极度小心才行。这个时候胖子晃着手电从雾里钻出来,让我跟着他走。

我点点头,尾随他趟水进去,走了几步,突然看到前面出现几个黑色的轮廓,不知道是什么东西,胖子显然已经看过了,一点也不怕,招呼我别磨蹭,我跟他走过去一看,原来是四只半人多高的石猴,蹲在石座上,面朝四方,不知道在祈祷什么,我知道这个叫定海石猴,一般沉在池塘底下,辟邪用的,在这里出现也算正常。

我放下心来,又往里走了几步,只见那四只石猴的中间还树着一块二米多高的大青冈石碑,闷油瓶正打手电照着石碑仔细的看。

我走过去问他:“怎么样,你看到这些有没有想起来什么?”

他指了指碑前面的基石,我一看,上面刻了几行小楷,胖子看不懂问我上面写的是什么意思,我说:“这几句话就是告诉我们,墓的主人修建了一个天宫,通往天宫的门就在这石碑的里面,如果和你有缘,这门就会打开,你走这门啊,就可以上天了。”

胖子看了看这石碑,说道:“有个屁的门啊。”

我对他说道:“这句话有点像禅话,这种话每个人都有不同的理解,他的本来意思,不是说这石碑中真的有一扇门,可能是是指碑上的内容可能隐藏了什么东西。”

胖子对我说道:“他娘的,这碑上有‘内容’吗,我可一字也看不到!”

我抬头一看,看到石碑正面光秃秃的,打磨的异常光亮,几乎就像一块玉一样,然而上面竟然一个字也没有。我也觉得纳闷,说道:“这里写了有缘才会打开,你和天宫没缘,当然没有。”

胖子呸了一声,叹了口气就俯下身子在水里摸起来,一边摸还一边嘀咕:“我和天宫没缘分不要紧,我和明器有缘分就行了。”

我转头去看闷油瓶,他的脸色很差,我问了他几句他也不理我,只是仔细的盯着这块石碑,好像在找什么东西,我觉得奇怪,一块光板而已,不知道他聚精会神的在看什么。这个时候胖子拍了拍手,我转过头,看见他从水里捞起来一只潜水镜,说:“看来这里来过不少人。”

我走过去对他说:“我三叔出去的时候,身上没有潜水器械,这些东西可能是他的。你看看有没有氧气瓶。”

话刚说完,胖子已经从水里摸出一个被撞扁掉的氧气瓶来,他试着用了一下,似乎不行,扔回到水里去,说道:“这下面尽是些破烂,难为我还这么高跑下来,真是空欢喜一场,我看我们还是快点上去,难保什么时候这水又要满上来,到时候飞都来不及。”

我看看了水位,觉得胖子说的有道理,就走回去找闷游瓶。一看,他竟然不在那里了,我叫了几声,没人答应,心里突然咯噔一下。

这小子就像鬼魅一样,经常突然出现又突然失踪,这下子千万不要又消失。

我想到这里,忙招呼胖子四处去找,虽然雾气很浓,但是这个地方不大,我们兜了两圈,终于发现他坐在池壁的角落里,正在呆呆的看着前方,我一看他的眼神就觉得不对劲了,眼睛里已经没有了他经常有的那种淡定,换成了一种几乎死灰一样几近绝望的眼神,整个人看上去就像一个死人一样。

我忙问怎么回事情,他的抬头看着我,用几乎听不到的声音说:“二十年前的事情,我想起来了——”

\chapter{二十年前}

闷油瓶,不,应该说是张起灵,他的语气平缓,丝毫不带一丝感情色彩,从他的叙述中,我渐渐看到了这个巨大迷团的一角。然而我没有办法从他的叙述中了解,他在整个事件中所想所听,也无法了解他真正的身世背景,我们暂时把他想象成一个沉默睿智的青年。

在深深的海底,无法听到海面上的狂风怒号,但是还是能够感觉到风暴来临前的那种窒息。

张起灵他静静的坐在耳室的角落里,看着他的同伴们争先恐后的去研究地上的青花瓷器。这些瓷器对与他来说,毫无吸引力,而这几个看上去比他年长一些的学长,却已经被这些东西完全吸引了过去。

他们互相传阅,有的想把上面的花纹描录下来,有的在讨论上面图案的意思,这个时候,突然有个人叫道:“你们快来看!这些瓷器底下有蹊跷!”

说这句话的人名字叫霍玲,是考队三个女生中的年纪最小的一个,父母是一高干,平时娇生惯养的,特别喜欢大惊小怪的来吸引别人的注意,张起灵听到她的声音就觉得头痛起来,不过她这样的女生这个小团队中还是比较受欢迎的,这一声娇滴滴的声音,马上把其他几个人勾引了过去。

这些男生都争相恐后,希望能够在霍铃面前显示自己的学问,纷纷叫道:“能什么蹊跷?拿给我看看。”霍玲翻过手里的一个瓷器,让他们看,一个看了一眼,说道:“这个啊,我知道,这个叫窑号,代表这只瓷器的产地。”

另一个马上反驳,说道:“不对,明窑的窑号不是这个样的,这可能是代表这个墓主人身份的府号铭文!”

第一个就有点面子上挂不住,说道:“府号铭文一般都是四个字的,这里只有一个字,还非常的生僻,你说的更加不可能。”

两个人承文革的遗风,说着说着就文斗起来,而且有演变成武斗的倾向,见惯这种场面的霍玲叹了口气,突然看到张起灵冷冷的靠在角落里,根本没有理会她,心中哼了一声,径直走到了过去,把青花瓷长颈瓶递到他面前,很俏皮的说:“小张,你帮我看看,这是什么?”

张起灵根本不想理她,淡淡的瞄了一眼,什么也没看清楚,就转过头说道:“不知道。”

霍玲脸色一变,她很少在男人面前吃闭门羹,不由心中不舒服,说道:“小张,不准你敷衍我,仔细看看再回答!”说着一下子把那瓶子塞到张起灵手里。

张起灵叹了口起,无可奈何,只好拿起来,霍玲得意的指给他看,原来那只被碰倒的青花瓷长颈瓶的底部,有一个特殊的刻文。

这个刻纹张起灵从来没有见过,不由心中一楞。一般的瓷器底部都是从哪里出窑的窑号,然而这个刻文,有凹凸的手感,却不是任何窑号的名称,更像一个编号。

他随手拿起另一只,翻过来一看,果然也有,却和他刚才看到不同,这一下子他突然隐约感到,这些瓷器似乎并不是单纯的陪葬品这么简单。

霍玲看他神色变化,以为这块木头终于开窍了,问道:“小张,怎么样,这到底是什么东西?”

张起灵根本把她当成透明的,他拿起这些瓷器,一连看了十几只,发现每只的底部都有不用的符号,而且这些符号有规律的变化着,似乎是一种有固定排列顺序的编号。

为什么要给这些瓷器编号呢,难道他们的排列顺序是这么严格的吗?还是,如果不按这些编号排列,就达不到某种目的呢?张起灵心中无数的思绪闪过,不由仔细的端详起这些瓷器来。

他一看之下,又觉的愕然,因为瓷器的花纹所描绘的内容,不是春耕,不是庭院,却是一幅工匠在雕琢巨型石像的画面,这种画面在古代是不登大雅之堂的,何以会将起描绘在瓷器上?

他一路看下去,渐渐发现了一些端倪,这些瓷画,在单独看起来时候并无什么特别之处,但是只要按照排列的顺序,你就会发现,这些画面都是连续的,似乎是在描绘一个巨大工程的进展情况。

这个时候所有的人都被他奇特的举动吸引住了,几个男生不知他卖的是什么关子,都莫名其妙的盯着他。

张起灵并未理会这些人,他没有像我一样一路看下去,而是直接走到了最后一个小巧的瓷花双耳壶边上,拿起来仔细一看,心中已然一动,只见这最后一只双耳壶上,已经勾勒出了整个工程完工时的情景。

那是一座无法用语言来描述的,漂浮在天上的宫殿,宫殿下方云雾缭绕,宫殿的建造者们,站在地面上,仰望着天空中,而边上的一座山上有一个道者,正怡然自得微笑。

这小小的双耳壶无法表达出这个工程的任何雄伟壮观之处,但是张起灵还是感到了一阵无法抑制的激动,因为他知道他找到了什么东西。

他几乎可以断定,上面描绘的内容,就是明初的鬼手神匠汪藏海,所设计建造的云顶天宫!

这传说中可以飘在天上的宫殿,老早出现在了明间传说之中,然而那时候的解释是,汪藏海是利用一只巨大的风筝配合大量的金丝线,来造成美仑美幻空中宫殿的假象,来取悦朱元璋。

可是如果传说是正确的话,那这里所描绘的情景,又是什么呢?如果传说不正确的话,那么,这些瓷画是不是说明,汪藏海真的造了一坐飘在天上的宫殿?传说与事实,事实与传说,哪个真哪个假,张起灵开始迷茫起来。

他思索了一会儿,毫无头绪,就把这些事情告诉了还不明就里同伴,这些人当然不信,忙按照他的方法,一个瓷器一个瓷器的看下去,不由一个个看的目瞪口呆,这不仅是中国历史上绝无仅有的,也是最匪夷所思的发现。那个霍玲一看到自己的发现竟然引出了这么重大的发现,不由欣喜若狂,就在张起灵脸上亲了一小口,这一下另的几个男的马上吃起醋来。

偏偏张起灵没有察觉到这一点,他可能根本不知道是谁亲了他,也不想知道,直接走到文锦边上,提议马上进后殿搜索,他认为,更多的线索,必然可以在棺椁中找到。

文锦到底是个负责人,她一想,认为这样做太危险了,忙说道:“不行,绝对不行,没有领队的带领,我们不能自己进去古墓!”

张起灵看她不同意,也不多废话,自顾自收拾自己的装备,就往甬道走去,文锦到底是一个女中豪杰,看他如此不把自己放在眼里,不由也心中不快,就想出手教训他一下,反正她在研究所里也经常耍几招功夫,教训一下那些不服她的毛小活子。

想着,她突然上前发力,想一把抓住张起灵单薄的手腕的关节,这叫做扣脉门,脉门一但扣住,就可以四量拨千斤,她一个女人力气自然不大,但是只要率先发难,也足以让张起灵这个大男人疼的求饶。

另几个男的都中过文锦这一招,不由暗自发笑,想看张起灵的笑话。

这一招她百试百灵,一般没武功底子的人根本防不胜防,然而她这一下却没有扣着,不由大吃了一惊,这时候,张起灵已经回过头来,淡淡说道:“你放心,我自己能照顾自己!”

文锦冷笑一声,说道:“你拿什么来照顾你自己?小张,你在所里就是出了名的无组织无纪律,可这里是古墓,请你不考虑自己,也要考虑考虑大家的安全。”

张起灵点点头,竟然说道:“我会考虑的,我很快就回来。”

文锦小脸都气红了,心说怎么摊上这么个刺头,看他那不温不火的语气,自己又没办法发火,上去一把拉住他,说道:“不行,说什么你也不准备去,我们已经少了一个人了,你叫我回去怎么向所里交代?”

张起灵似乎有点不耐烦,转过头,眼神一冷,说道:“放手。”

文锦非常坚决的看着他,我想任何男人看到她这么可爱的一个女人,用那种眼神看着自己,都会妥协,可是张起灵突然睁大双眼,眼神瞬间就变的犹如恶鬼一样,文锦被一下子吓的手都软了,被他一下子甩开。

等她再看,那张起灵的眼神又变回那种淡淡的什么无法看出的样子,向她点了点头,说道:“谢谢!”

其他人看到这一幕,以为文锦竟然同意了他的要求,都不服气起来,人就是这样,只要有一个人破了规矩,其他人都会蜂拥而上,其他几个人看张起灵走进了甬道,一方面怕他占了所有的功劳,一方面也燃起了已经压制下去的好奇,纷纷吵着要跟上去。

文锦到底是个女人,她知道她刚才的手一放,自己已经失去对这些人的控制,事到如今,除非手里有把枪,不然已经没有任何办法可以阻止这些年轻人了。

三叔的脾气又不好,如果这个时候摇醒吴三省,以他的脾气,必然会为了自己的面子和张起灵发生剧烈的冲突,事情可能会一发而不可收拾,最后衡量利弊,她决定自己带他们进入后殿看看,并尽快回来。以她多年倒斗的经验,如果这只是一个普通的墓穴,必然没有问题。

之后的过程,与我们经历的基本相同,至于他们如何通过机关重重的甬道,发现了池内的阶梯,然后下到池底,虽然也十分的曲折离奇,但是并不是需要叙述的重点,张起灵讲述的时候也是一句话就带了过去,最关键的事情,还是他们下到了水雾缭绕的池底,看到那块无字石碑以后。

这池底的情景简直是诡异莫名,那些浓雾在手电的照耀下,不时变化成各种各样的脸谱,让人不由自主的产生畏惧的心里,走下最后一阶石梯的时候,一行人突然就变的团结起来,大气都不敢出,在雾气中互相拉扯,战战兢兢,生怕有什么东西突然冲出来。

霍玲见张起灵,毫无畏惧,而边上其他几个人平日里威风八面的所谓所里的学长,如今都闪闪缩缩躲在他的身后,不由对他生出一点好感,对那些男生说道:“你看看你们几个,都比小张打了好几岁,连他的渣都比不上,丢人不丢人!”

他们那个年纪的人,正是出身牛犊不怕虎的时候,被霍玲这么一说,血气上涌,也不要命了,都抢着冲张起灵前面去,池地空间不大,他们跑了几步,看没什么事情发生;胆子又大起来,径直走进雾气的中央,才走了几步。突然领头的那个大叫:“里面有只怪物!”边叫边逃回来。

这一嗓子几乎把所有人都吓的屁滚尿流,后面几个也不管自己有没有看到,头皮一麻,也跟着后退,张起灵不理他们,领着其他几个人自顾自走了进去,就看到了那只所谓的怪物,就是那只定海石猴。

随即,他们就看到了另外的几只定海石猴和那块神秘的无字石碑。

瞬时间,所有人都被深深的震撼了,虽然眼前的这些东西并不壮观,但是在这些人眼里,意义非凡,这古墓里的一切的一切,都打翻了教科书一样的千年不变的中国墓葬观念。有着不可估量的考古价值。

连文锦都被惊讶的说不出话来,喃喃道:“我的天,这些东西太让人难以置信了,这里说不定会成为中国考古界的又一里程碑。”

震惊过后,就是狂喜,那个年代,一个重大的发现意味着巨大的机会,一但把这个发现公布出去,他们的名字马上就会家喻户晓,想到这里,有几个笨点的已经傻笑起来,还有一个兴奋异常,竟然控制不住开始跳起舞来。

这个时候,惹起这场祸头的张起灵却深深的皱起了眉头,他看的比任何人都仔细,早已看到石碑基石上的篆刻古文。

“此碑于有缘者,即现天宫门,入之,可得仙境也。”

这一句话给他的震撼,远远在于其他这些发现,他没有半点被边上人的癫狂所感染,陷入了深深的沉思之中。

按照他的想法,这样的文字,不可能无缘无故的写在这个地方,所谓有物则必有其用,墓主人把这些东西摆在这里,必然有不得不这样做的理由。

那这石碑中通往天宫的门,到底在什么地方呢?如何才算有缘呢?他站到石碑前面,一寸一寸的找起来,可是石碑就是石碑,没有任何机关或者暗文的痕迹。

其他的人闹了一会儿,也逐渐冷静下来,文锦觉得时间已经差不多,再在这里耽搁并不妥当,就招呼他们回去。那几个人开心也开心够了,见识也见识到了,也收起心来,说说笑笑的就往阶梯走去,文锦一个一个的数过来,数到最后,发现张起灵还没过来。

张起灵刚开始不服从领队,坚持要来后殿,现在又不肯归队,想到这里,文锦非常的生气,但是职责所在,总不能扔下他不管,她语气很差吩咐了其他人一声,一队人又快步走回到雾气中。

他们走了几部,看见张起灵还蹲石碑前面在研究什么,文锦不由心头火起,叫道:“你还不走!到底要别扭到——?”话才说了一半,霍玲一把拉住她的手,拼命叫她不要说话,文锦纳闷,看了看其他人,发现他们都有点惊慌的神色,非常不解。

霍玲看她还没反应过来,忙指了指雾气之中,文锦顺她的手看过去,只见张起灵的边上不到两米的距离雾气深处,出现了一个巨大的人影。

\chapter{奇门遁甲}

那个巨大人影几乎于石碑同样的高度,依稀看到有头有脖子,于人无异,只是他站在那里的姿势,伛偻着腰,说不出的怪异,让人看着不汗而栗。

文锦冷汗直冒,他们一行人站在石阶与池底衔接处,与那个巨人只有五步的距离,说长不长说短不短,非常的尴尬。池底雾气翻腾,所有的照明只赖几盏功率不大的手电,一时间也也无法看清这个东西的到底是人是鬼。而刚才这里这么多人,里里外外都搜索过了,这10米开外的池底,除了中央四只定海石猴和一块无字的石碑之外,并无其他东西,这个巨大的“人”,到底是什么时候冒出来的?谁都不知道。

而这个该死的张起灵好象一点都没有察觉,仍旧入神的看着石碑,不知道到底在研究些什么。文锦简直对他恨的咬牙,无奈自己是负责人,不能丢下他不管,现在一时间她也没有对策,只好嘱咐身后的不人要乱动。

过了有五六分钟,这个巨“人”仍旧躲在雾气之后,好象没有任何行动的打算。

这个时候霍玲已经忍不住了,轻声叫道:“小张,你还傻蹲在这里干什么啊~快点到我们这里来。”

文锦吓的忙阻止她,张起灵离这个东西太近,一但情况发生变化,两步的距离很难全身而退,最好的办法,就是暂时维持现状。

文锦迅速分析了一下形式,在古墓中凶险的事情虽然不少,但是只要你知道你碰到的是什么东西,自然就有办法对付,就怕你身处险境,却还没有摸到头绪,往往就死的不明不白。

文锦稍微一分析,觉得这个地方不可能有粽子,因为这个古墓所选的位置非常之好,西沙群岛几百年受到人为的骚扰很少,几坐环形岛礁在海面上星星点点,在海下却是连成一片,形成一条连绵不段的海地山川,山川藏在海底,聚风养气,东有龙头,西有龙尾,是一条非常少见的海底龙脉。而龙先属水而后飞天,所以水龙在风水学上,还略高于山龙。

这样一个地方,如果有棺材必然真的是有官有财,特别是如果这个古墓真的葬的是汪藏海的话,此人看名字必然是五行缺水,这样一来在海墓之中更加相得益彰,简直可以说把风水上所谓的天地人和都占尽了。

所以除非风水书都是瞎掰,不然这里绝对不会有粽子。文锦想到这里,心中已经释然,既然不是僵尸,那必然是人或者动物,只要是活的东西,这里这么多人,不要说你身高两米,就算你身高三米我们也能把你拿下。

这个时候,其中一个男生说道:“文锦,我看不对劲啊,我记得在那个位置上面,应该是那只石头猴子,该不会是有什么东西站到石猴上面去了吧。”

文锦心中一动,她突然想到,该不会是三叔醒了过来,发现他们不在,进了这里找他们,这个人行事比较不正经,可能是怪他们不服从他的命令,就躲到雾气后面,然后爬上石猴来吓唬他们。

如果真是这样,那简直太可恶,文锦想到这里,已经觉得这是最有可能的解释,想着她就对那影子叫道:“吴三省!你别玩了!快给我下来!”

如果对方真的是三叔,这么一吼必然就知道自己已经漏底了,那继续硬撑下去也没有必要,三叔是豁达之人,这种小事情,他大笑两声也就算了,绝对不会介意。

谁知道他话音未落,那个影子突然伸出一只手,对他们一摆,好象是让他们不要说话!

文锦一看他那身形,手的长度和他的身高不成比例,果然是有人站到了石猴之上,她想也没想,断定就是三叔,气的一跺脚,快步跑了上去,一个箭步跳上石猴,就去扯他的耳朵。

这是她对付三叔最后的一招,因为他们约定过永不吵架,只要文锦气到极点,就可以去拉三叔的耳朵,让他知道,自己已经非常的生气了。一般遇到这种情况,三叔就算有豹子胆,也不敢再放肆了。

说是迟那是快,她刚跳上石猴,还没来的及动手,石猴上那人就一把把她抱住,一手捂上她的嘴巴,轻声说:“我是小张!别说话!自己看下面!”

文锦本来已经怒不可遏,可一听着声音,不由一愣,这真的是张起灵的声音!他怎么会站在石猴上?

她转念一想,突然出了一声冷汗,不对啊!!如果这样说的话,那石碑前面蹲着的又是谁啊!

她马上回忆刚才的情景,那个时候她只看到石碑前面蹲着个人,手上又有手电,而这么多人中惟独缺了张起灵一个,所以她才会马上下了判断,难道这是一个先入为主的错误!

想到这里,她马上探出头去看,一看她就一楞,只见那碑的前面蹲着的人,穿着和他们一样的潜水服,看体形,不是别人,竟然是三叔!

而且三叔还有点不对劲,文锦一开始还不明白他在做什么,仔细一看,才发现他竟然在对着那快光滑的犹如镜子的石头碑,梳头发,让人觉得毛骨悚然的是,他那种扭捏的动作,分明是女人才会做的出来。

三叔梳了一会儿头,又转了转脸,仔细的看着石镜里的自己,就像一个未出闺阁的少女放梳妆打扮完毕,在最后看一下效果。

石镜里的三叔的脸,似笑非笑,看上去鬼气森森的,说不出的诡异。这样的画面,如果是平时,肯定是很好笑,但是现在,文锦只觉得手脚发凉,连大气都不敢出。

下面那些人看石猴上的两个抱在一起,一动不动,以为真的是三叔装鬼吓人,不由松了口气。那霍玲担心张起灵,突然就跑到那石碑前的那人背后,一拍他的肩膀,说道:“小张,你到底在这里发什么楞啊?”

这一下子真是出呼所有的人的意料,张起灵暗叫一声糟糕,想阻止已经来不及了,只见石碑前的那人猛的站了起来,吓的霍铃一声尖叫,不过她马上发现站在面前的是三叔,由吓转怒,大骂:“吴三省,是你!你不去睡你的觉,蹲在这里发什么神经!”

三叔看到霍玲,突然用手遮住脸,怪叫一声,用力推了她一把,把她推倒在地上,然后转头就跑,张起灵一看不对,马上跳下石猴追了过去。他的速度非常快,但是经过霍玲的时候,他稍停了一下,看她有没有受伤,就是这一下,却坏了大事,那霍玲一看张起灵看她倒地,就冲了过来,以为是关切她,不由心中一热,竟然就去去抱他。

张起灵心中不由一叹,这几秒的耽搁,足以让他失去所有的先机,他一个打滚就从她胳臂下面翻了过去,再一看三叔,他已经跑进浓雾,看影子,几乎已经跑到池壁边上了。

张起灵大叫一声:“看住石阶!不要让他上去!”说着就直追了过去,这个时候,他隐约就看见,前面的三叔突然一个侧身,一瞬间,似乎是穿进了墙壁里。然而雾气实在太浓,到底是怎么个过程,他一点都没有看到。

张起灵追到池壁边上,无人可追,不得不刹车停下。他并不相信三叔钻进了墙里,虽然他不是那种什么都讲唯物论的老八股,但是这样的情景,过于匪夷所思,必然有蹊跷在里面。

他呆了片刻,马上用手去摸这块石墙,然而这块石墙却是实实在在的,张起灵不相信这个世界真的有穿墙术这种东西,他伸出两只奇长的手指,往那石墙上一贴,一瞬间,他那极度敏感的手指,马上就感觉到,这面石墙,竟然是在非常缓慢的转动的!

他马上脑子就嗡的一声,完了!刚才竟然一点都没有发觉,这个池,竟然本身就是一个巨大的机关!

他突然觉得,非常的感慨,这简直是可是说是一个古工程上的奇迹,自己的所谓的经验,在这个墓主人面前,就像一个小孩子一样幼稚。

但是,这个机关的目的是什么?他们下来的这几分钟里,似乎整个池底并没有什么变化啊!这个该死的汪藏海,总不会只是想在自己的墓里搞一个旋转餐厅吧。

对于机关的原理,张起灵并不陌生,用他自己的话说,他对于中国古墓的陷阱机关的了解,超过世界上任何人(原话),他对于机关的工作原理,起源,缺点,甚至发明者的名字,都非常了解。

按照他的经验,这个机关,必然是用最简单的原理来运作的。因为他知道一般所谓的巧簧机关,木弓暗弩,无论是多少好的材料,经过少则几百年,多则上千年的岁月,其用来击发的引信,都已经腐烂无法使用,能够阻挡盗墓贼的,往往是最简单的墓墙外的防盗沙层(盗墓贼掘到沙层后,上面大量流沙陷落,会将盗墓者活活的闷死,但这也是非常被动的手段,现在盗墓者反而会根据洛阳铲中带沙,而确定古墓的实际位置,并直接从墓顶硬穿十二层青砖而过)。

要一个机关能够几百年几千年的运做下去,必要使用几百年几千年都不会腐烂的材料,比如说石头和不会干枯的活水。这些东西,这里都有,而且这里的水还会根据潮汐的变化,提供一种动力,使得利用起来,更加的方便。

如果墓主人是汪藏海,那么这个人,从他对奇淫巧术的痴迷程度和运用能力,已经达到化境,恐怕世界上再没人可以超的过他。

张起灵一边想一边去摸其他地方的石墙,他的心里,已经有了一个模糊的设想,这个墙上肯定有一个入口,刚才他一个迟疑的功夫这个入口已经转移了位置,他一路感觉过去,才往前走了几步,果然就发现了一个暗门。

不可能这么容易就被他找到的,他摇摇头,不敢入内,继续往前一路走下去,这一下他越走越疑惑,最后一数,这里小小的地方,竟然被他摸到了八个暗门,这下子他心里一盘算,似乎已经知道,这他娘的不是奇门遁甲吗?

\chapter{生门}

奇门遁甲起源於四千六百多年前,几乎和中国有文字记载的历史一样长,世界上最早使用奇门遁甲的第一人就是老祖宗黄帝,然后一路传下来,你可以看到世界上几乎每一个军事家或者军事都会一点,但是事实上到了汉代以后,奇门遁甲已经不是全本,因为黄石老人传给张良之后,这个鸟人把他归纳简化,以至于后来人的基本上都看不懂他到底在说些什么。

我对奇门遁甲的了解主要来自于家里的二叔(非三叔也),虽然所知也不多,张起灵提到这个的时候,我还不至于像胖子一样好像在听天书。奇门遁甲起先有四千三百二十局,到黄帝手上的时候,他只看懂一千零八十局,到张良那会儿七十二局,现在到我二叔手上整理出来的只有四十二局,已经非常难得,世界流传只有十八局,其他各局都是三叔偶然从一个汉墓中找到。

奇门遁甲虽然玄妙,但是他其实是兵法和命数理论,用他来摆阵属于发挥余热,奇门遁甲阵又叫八阵,分八个门开门休门生门死门惊门伤门杜门景门,生门为生,死门为死,入其他各门,则又见八门,周而复始。

张起灵找到的这八个暗门,自然而然就想到奇门遁甲一说,这些暗门其实非常的窄,只能容纳一个人侧身而过,这里雾气弥漫,外面又有一面可能转动的砖门,只要一推,就能打开,进去之后活门自动关闭,不去摸根本看不出这里还有如此的蹊跷。

张起灵有点对自己的大意耿耿于怀,他不是一个莽撞的人,但是刚才过于急功近利了,天底下的奇淫巧术都是以小以精为荣,这个却是反其道而行,即大而全,反而让他防不胜防。

他走回石碑处,把发现和众人一说,众人哗然,这门学问非常深奥,他们刚经过文化大革命的洗礼,怎么可能懂得这些,文锦沉思片刻,突然说道:“刚才三省的行为这么诡异,好像被一个女鬼附身了一样,会不会这鬼就是这个墓穴的主人,他刚才钻进的那个暗门,会不会就是生门呢?”

张起灵看她眼睛里神采熠熠,似乎已经想到了什么,问她道:“你是不是想到什么?”

文锦让他门跟着她,自己转身走到那块石碑前面,也学着三叔的样子,半跪了下来,开始梳起头发,她的身段非常之好,这样一个姿势,非常的有魅力,一下子几个男的都看的呆了,她梳了几下,又很矜持的转了转头,这一转,她突然就一抖,马上叫起来:“找到了!”

众人一听马上围了过去,对着石碑东看西看,搞了半天却什么都看不到,文锦说:“不对,你们一定要像我这样,跪在这里,才能看的到!”张起灵似乎有点醒悟,忙跪下来,文锦在他肩膀上一压,说:“你太高了,再低一点,目光不能直视,要侧视,盯住自己的鬓角。”

张起灵觉得好笑,也学着她的样子,梳了梳头发,然后非常的女性化的一瞥,突然他就看见自己在石碑的倒影里,鬓角的地方有浅浅的三条首尾相连的鱼,非常模糊,他又动了一下头,发现只要角度稍微一偏,就马上消失看不到了。

他哦了一声,终于知道所谓的有缘是什么意思了,心理不由暗骂,看来,只有爱美的女人,碰巧跪在这一块石碑前面整理头发,才有可能会看到这个标记,而且太高太矮都不行,幸亏文锦观察的仔细,不然自己这个大男人,无论怎么想也找不到这个秘密。(我听了也恍然大悟,不过话又说回来,这个墓主人难道是个色狼吗?)

他仔细盯着这条鱼,发现这个印记也在缓缓的移动,看样子,这块石碑里面,应该是有一个和池壁转速一样的机关,这个印记对着的位置,永远是所谓的天门。他想到这里,忙让文锦看着,自己打起一只手电,跑到池边,一个一个暗门的定位,到了第三个暗门的时候,文锦看到印记和手电的光点重合了,大叫一声:“就是这个!”

所有人一声欢呼,连张起灵都忍不住用力握了一下拳头,他用力推开暗门,第一个侧身走了进去,里面是非常窄的一条走道,一直往里面通去,这次张起灵非常的细心,他先摸了摸四周的墙壁,确定再没有其他的机关,才叫他们进来。

这条走道也是用青冈石板堆砌而成,只有一个人宽,两个稍微胖点的人就走不过去,张起灵打着手电走在最前面,一眼看过去,发现前面的那种黑暗,和青岗石的颜色参合在一起,变成了一种青幽幽的感觉,似乎是幽冥里的颜色。他收敛全部的精神,走的非常小心,只要有一点奇怪的声响就要停下等个半天,不过这个时候他已经完全成为了这群人的精神领袖,人人对他言听计从,没人敢说半句废话。

他们走了有半只烟的时间,前后都已经一片漆黑,张起灵觉得似乎整个宇宙只剩下他们几个,他心里也开始不舒服起来,这个时候,走道开始向上倾斜起来,他顺着这个势往上一看,发现非常远的地方前面竟然出现了亮光,昏黄昏黄的,好像夕阳的光,不是很亮,但是很温暖,张起灵知道那里就是已经到头了,招呼了一声,几步并作一步冲过去,只看着那个光点越来越近,突然脚下一平,整个世界好像突然被金光笼罩起来,他忙眯起眼睛一看,不由惊叫了一声,差点跪了下来。

在他们前面,出现了一个巨大的四方形的房间,这绝对不是单纯的大,那是一种极端的霸气,整个建筑的氛围只能用磅礴来形容,简直给人一种不得不下跪的冲动。

房间的每一边,都有十根整根的金丝楠木柱子,三人围抱不住,好似天涯海角的撑天柱一样。整个房间由黄浆砖砌成,左右十丈,上面粱雕檐画,光五爪金龙就有十条,极端的金碧辉煌。而几乎有十米高的宝顶上,镶嵌了一幅五十星图,每一颗星星,都是一颗璀璨的夜明珠,估计都有鹅蛋大小,正在发出幽幽的黄光,房间的四个角落里,各有一面大镜子,光线互相反射,虽然不是很亮,但是足以照亮整个空间。让他们最吃惊的,却是房间的中间,放着一个巨大的石盘,张起灵一看就知道了,石盘上面,是一个规模宏大的宫殿模型。虽然只是一个模型,但是其龙楼宝殿,假石流水,一应俱全,非常的壮观。

张起灵跑过去,兴奋地围着转了好几圈,马上就明白了,这就是云顶天宫的模型,他本来就不相信这个古墓里会有一个宫殿这么离谱,所以也没有觉得失望,但是心中的迷团更浓了,看样子,汪藏海真的造了一个天宫,那这个天宫在什么地方?难道真的在天上。

这个发现太惊人了,所有人都兴奋又喊又叫,几个男生还起哄的把霍玲抬上了石盘,霍玲傻笑着刚站稳,突然尖叫了一声,跳了下来,叫道:“上面有个死人!”

张起灵一惊,忙飞身跳上去一看,只见整个模型的中间,是一个圆形的玉石花园,花园里面,一个石头座上,打坐着一具已经完全收缩的干尸,身上的衣服已经破烂光了,露出来的躯干呈现黑色,这是一具非常难得的坐化金身,自然风干的非常好,只要往金粉里蘸蘸,就可以放到寺院里供起来了。这具尸体一手指天,一手指地,头发和指甲和其他的金身一样,死后都在不断的生长,特别是指甲,几乎和手指差不多长了,看上去有点不妥。

他一个飞跃跳到这具干尸前面,毫不顾及,就先去看他的嘴巴,发现嘴巴里并无东西,然后叉住他的掖下,一路按下去,文锦也跳了上来,看的清清楚楚,忙一个纵身跳到他的背后,轻声质问道:“张起灵,你到底是什么来路,怎么会倒斗的这一套!”

张起灵看了她一眼,并没有回答,文锦火了,一把抓住他的手,说道:“你分明就是个倒斗的,不然不会在古墓中如此的镇定,你跟着我们,到底什么目的?”

张起灵做了不要说话的手势,指了指这具干尸,说道:“这些不重要,你看!”说着,他将干尸的衣服脱下,只见这具尸体肚子上,有一条非常长的伤疤,从左边最后一根肋骨一直到丹田,他自己先按了一下干尸的肚子,然后抓住文锦的手也按了上去,文锦一哆嗦,果然,尸体的肚子里明显藏了什么东西。

张起灵抬起头,他现在还不敢肯定要不要把东西拿出来,如果这个人临死都要把一个东西藏在自己的肚子里,说明这件东西对他来说非常重要,或者这也是死者考验他们的一个方法,他的原则是绝对不会为了古墓里的东西而破坏尸体,张起灵心理斗争了很久,又看了一眼文锦,文锦是北派,自然讲究道义,她摇了摇头,说道:“取之不仁,必遭天谴。”

张起灵叹了口气,也决定放弃,他退后一步,给那尸体磕了一个头,等他抬起来的时候,突然发现尸体好像哪里不对了。他左看右看,突然倒吸一口凉气,原来这具干尸,竟然露出了一个诡异的微笑。

\chapter{连环}

这真是前无古人后无来者,就算是粽子,他也只见过能蹦能跳的,从来没见过会笑的,张起灵觉得心中一紧,急忙后退一步,全身戒备,准备应对它的下一步动作,没有想到的是,那具干尸原本指着天的手,突然一动,变成了水平指向东边,同时,整个房间突然一暗,宝顶上的夜明珠不知道什么原因,瞬间熄灭了。

他们进来的时候,为了节约电池,已经关掉了手电,这一下子其他几个人都吓了叫了起来,张起灵发现虽然房间变暗,但是并没有变成一片漆黑,忙抬头一看,发现最靠近四面墙的四颗夜明珠并没有熄灭,就像漆黑街道上的昏暗路灯一样,只照亮了一小块区域,这个时候,边上传来了李四地发抖的声音:“墙上有~有~脸!”

张起灵一个激灵,忙转头一看,只见这东边那颗夜明珠所照亮的黄浆砖墙,都出现了光影的变化,平白无故显现出一张巨大的惨白人脸来。

张起灵知道必然又是一个把戏,有点厌烦地跳下石台,走到东边的砖墙前一看,发现墙上的其实是一幅影画,这种画是当光线从一个固定角度射过来时候,由墙上沟壑的影子所形成的,如果光线的角度不对,画就不会出现,但是因为这些线条太诡异了,在高度紧张的情况下,很容易被人想象成可怕的人脸。

他仔细看了看,不由心中一动,眼前的这一幅似乎是叙事画,而且看内容,应该是在展示云顶天宫刚完工时候的情形,他看到所谓的天宫,其实是建筑在一座非常陡峭的山脉上,山顶云雾缭绕,把整个宫殿都包了起来,才给人一种浮在云上的感觉。张起灵看着那座山峰的情景,似乎白雪皑皑,海拔应该非常的高,不知道是在哪座山上。

他转了转头,发现四面墙上都有影画出现,忙转到南面的砖墙继续看,只见这一幅,天宫下面的悬崖上,被修凿很多的有栈道相连石窟,一行工人,正在用一个“桔槔(吊车)”将一具巨大的棺材,顺着悬崖一个石窟一个石窟的向上拉升,而送葬的队伍,则排成一排,顺着栈道艰难的往上攀。张起灵啊了一声,这个天宫,难道竟然是一个陵墓,那这棺材里装的,是谁呢?

他继续走下去,西边的那幅影画,更加的奇怪,只见悬崖上的栈道,竟然燃起了熊熊裂火,这应该是守陵的士兵在入殓仪式结束之后,为了保证陵墓的安全,而把进入天宫的唯一的道路烧毁。这样一来,基本上可以杜绝所有的小规模盗墓行为,无论南派北派,均没有人有能力到一个海拔如此高的地方,爬上百米悬崖,去倒一个斗,不可能也没有必要。

他记忆里并没有遇到过这样的墓葬,不由觉得惊讶,忙跑到最后一幅影画之前,一看就呆了,因为这幅画却出奇的简单:山顶上的天宫突然消失了,只见一片皑皑的白雪,不仅如此,连悬崖都被一片白色盖住。虽然并不是很生动,但是张起灵已经知道了这应该是一场雪崩。

他猜测,可能是大火使得温度上升,天宫上方的积雪松动,造成了大规模的雪崩,不仅把整个天宫掩埋在了白雪之下,还覆盖了整个山头,把这座宫殿变成了一个货真价实的坟墓。

他看到这里,不由长出了一口气,真没想到这个云顶天宫,最后的命运竟然是这样的。看来汪藏海对此也是耿耿于怀,自己的杰出作品在建成后没多少时间就直接被雪崩压毁,够他到死都郁闷的了,也难怪他要把这件事情通过这种隐秘的方式记录下来,这应该是一个地位显赫人物的陵墓,他肯定不能把这件作品公诸于世,但是以他这么喜欢炫耀的性格,他肯定会以某种方式让后人知道,自己的作品里,还有一座这么壮观的云顶天宫。

现在唯一不知道的,就是这座坟墓里埋的是什么人了,张起灵深吸了一口气,这个时候,他突然看到文锦和其他两个人正在试图搬动东南角的那面大镜子。他觉得很奇怪,忙问她在干什么,文锦焦急的说:“我刚才看到三省躲在这面镜子的后面,一闪又不见了。”

张起灵这才想起三叔的事情,忙上去帮了一把,这面2米高的镏金福字纹铜镜非常的重,他们用尽全身的力气,才挪开了半米,众人探头一看,只见镜子后面的墙角壁上,竟然有一个半人多高的方洞,张起灵照了照里面,只见一片黑漆漆,不知道通到哪里去。

吴三省前几天规划地宫的时候,并没有发现这里还有这么大一个房间,但是张起灵早就知道,地宫并非他规划的这么简单,因为沉船葬和陆葬不同,有一个沉船的过程,这个过程中船必须保持绝对的平衡,所以对陵墓的对称性要求非常高,吴三省规划出来的地宫虽然没有原则上的错误,但是明显的头重脚轻,如果以这样的结构来沉,估计整个墓会倒栽进海里。

他那个时候也懒的去出这个风头,就没和吴三省说,先在想起来,这里有一个用来平衡的通道,也不足为奇。

他和众人解释了一下,打起手电第一个走了进去,因为手电在进盗洞的时候一直开着,基本上都有点电力不足,文锦就让他们前后各开一只,其他人全部关掉。这个石道里面相当的宽,几乎可以四个人并排走,霍玲看到张起灵和文锦走的如此的近,不由有点不舒服,就硬挤上去,这个时候,张起灵已经觉得事情有点不对了,他隐约看到前面的黑暗中,有什么东西正在蠕动。

同时,空气中那股越来越浓的香味,也引起了他的注意,这种感觉,好像是他们正在走近香味的源头一样,再往里走了几步后,这些味道已经香的让他无法集中自己的精神,他回头想问文锦,突然发现,身后的几个人已经倒在了地上,文锦摸着自己的额头,迷糊的看了他一眼,一下子倒在了他的怀里。

张起灵心叫不好,马上闭住呼吸,然而已经来不及了,他只觉得一股无法抗拒的困意袭来,开始向墙壁上靠去,然后逐渐失去了意识,朦胧中,他看到三叔蹲了下来,面无表情的看着他。

闷油瓶说到这里,深吸了一口气,沉默了下来,说道:“我醒过来的时候,自己躺在医院的病床上面,什么都不记得,什么都不知道,直到几个月后,才一点一点的开始想起一些零碎的片段,后来又过了几年,我开始发现,我自己的身体出了点问题。”

我忍不住想插嘴问他,是不是发现自己不会老,但是他没给我这个机会,就接着说道:“我现在还不能告诉你是什么问题,不过我在三个月前,碰到了你的三叔,我发觉他非常的眼熟,为了想起更多的事情,就跟着你们去了鲁王宫。”他讲到这里,突然转向我,说道:“我在鲁王宫里,发现你的三叔很有问题!”

我一楞,不知道他是什么意思,他继续说道:“你们从青铜棺里拿出来的那块金丝帛书,其实是假的,早就被你三叔调包了。”

我大吃了一惊,叫道:“胡说!他娘的那不是被你掉包的吗?”

闷油瓶淡淡的看了我一眼,说道:“不是,是你三叔自己,他和大奎两个人,从树的后面打洞,直接挖到棺材底上,这大概也是为什么,大奎必须要死的原因。”

我听的浑身发冷,比任何时候都要紧张,虽然仍旧想站在三叔这一边,但是脑子里已经犹如一到闪电划过,无数的景象跳了出来,我想起大奎是怎么中毒的,想起潘子为什么在上树之前还很清醒,等我们在地面上看到他的时候却已经深度昏迷,想起我和胖子还没有爬出那条缝隙的时候,他已经扛着汽油筒跑了过来。

我无法再想下去了,只觉得世界上的一切都颠倒了,不知道谁说的是真话,谁是骗子,我到底应该相信谁。我觉得脑子一片混乱,无法控制的自言自语道:“不对不对,事情没有这么简单,没有动机,三叔他到底为什么要这么做?”

闷油瓶淡淡的说道:“如果这个人真的是你三叔的话,的确是没有动机。但是——”他说到这里叹了口气。

我没有明白他的意思,不过心里似乎已经相信了他,不由苦笑,我原来一直在想三叔到底有多少东西在骗我,现在,我必须要想的是到底他有多少东西没有在骗我了。

事情发生这样的变化,我真的没有想到,不过转念头一想,现在想这些也没有什么用,无论谁真谁假,都要等到我们逃出去后才有意思,不然死在这里,知道了真相有能怎么样。

想到这里,我忙定了定神,让自己放松了一下,这个时候,我发现胖子已经走到了石碑前面,笨浊的蹲着,翘起个兰花指头,在那里晃晃悠悠的梳起头来,我皱了皱眉头,叫道:“死胖子,你他娘的又在搞什么鸡吧事情,你就不能给我消停点?”

他转了一下头,装成女人的声音,说道:“哀家他娘的正在梳头~,梳个头又要不了你的命,你罗嗦什么?”我简直无可奈何,问他道:“梳头?你难道也想去那个天门里看看?”

胖子说道:“当然,这么壮观的情景,胖爷我怎么可能错过,况且,你看我们下来一次也不容易,那女人又跑了,看来我们的佣金也指望了,再怎么样,也得挖几颗夜明珠过来,所谓有钱就不倒斗,倒斗就不空手嘛。”

我骂道:“敢情刚才你听了这么久,就听到个夜明珠啊?”

他听了不服气了,说道:“哎,你还真不能这么说我,你胖爷我要进这个天门,还有另外一个非常重要的原因,你们可知道是什么吗?”

\chapter{血字}

胖子听了不怒反笑,似乎早就准备好了应对的方法,说道:“当然不是,胖爷我要进这个天门,还有另外一个非常重要的原因,你们可知道是什么吗?”

我对他说道:“谁知道你葫芦里卖的是什么药,你爱说不说,别忘了我们现在还是在落难,要是那些不着边际的事情,还是免了。”

胖子对我说道:“你别着急,我要说的这个事情,和我们现在的处境大大的有关系,你刚才没听这小哥说嘛,这个入天门的走道,是个上坡,而那个放着天宫模型的大房间,又非常之高,这高上加高,至少有个十几米,你想想这古墓总共才多深啊,我估计那房间的宝顶,应该整个古墓的最顶端,我们要出去,就应该从那里动脑筋!”

我一听心里一亮,忙估算了一下,我刚下到水底墓道的时候,看过水压计,那个时候已经是水下十三米,我们现在所在的这个池底,又在这个基础上下去了十几米,就是说我们应该是在水下二十米到三十米之间。这样算来,放着云顶天宫模型的那个房间,顶部离海底,最多也只有十米不到,的确正如胖子所说的。

刚才只顾着听故事,真没注意到这些细节,我不由对胖子刮目相看,这家伙看似莽撞,其实心里通明的很,看样子以后有事情也不能瞒着他,想到这里,我就对他们说道:“胖子这次倒是说到点子上了,不过现今知道了这些也不顶事,我们赤手空拳,不要说爬不上十米高的宝顶,就算爬上去了,手里没家伙,上面几层砖顶,如何下的去手。我看我们还是得先去找几件象样的金属冥器来,尽快实施反打盗洞的计划,再磨蹭下去,恐怕就要错过退潮的时间了。”

我说虽这么说,其实心里没底,因为这这一路过来,看到的赔葬品除了瓷器就是石器,连一件金属的都没有,有点不符合常理,我隐约觉得说不定也是这墓主人特意安排,现在只能去后殿里找找,要那也没有,那真是天要亡我也。

胖子听了我的话,哈哈一笑,说道:“这我也想好了,那大房间四面不是有镏金的福字纹铜镜吗?你也是倒腾古玩的,总该知道这镜子是啥样子的吧?我们把那镜子腿给拆下来,那东西老沉老沉的,绝对能当锤子使唤。”

我刚才听这名字就觉得很熟悉,听他说起,才想起我的确经手过这种东西,不过具体是什么样子的,我也记不清楚了,看胖子说的信誓旦旦,不像是瞎掰,不由也放下心来,对他说道:“那行,这这事情我们就怎么定了,事不易迟,我们马上就行动,不过到了那个地方之后,你可什么都别碰,千万千万,这地方到处是机关,我们以后的年月还长着呢,范不着为了几件死人的东西,把自己也交代在这儿!”

胖子听了点了点头,表示除了砖头,其他坚决不碰,我怕他还在动那些夜明珠的注意,又强调了几遍,只说到他烦。我又把那地方的具体结构问了个清楚,把可能遇到的情况,要采取的必要措施,和他们一一说了,然后三个人依计形式,先找到了天门,然后胖子打头,闷油瓶在后,我就夹在中间,径直走进了那条狭窄的天道里去。

我在闷油瓶的叙述中已经听过天道里的情景,但是自己进去,又是别有一番滋味,刚开始并没有感觉,只觉得是晚上走在嘉兴西塘的石皮弄里,窄了点而已,可是走了一段时间后,前后都已经没了边际,才开始慌起来,我走在中间,黑倒是不怕,只是四周太安静了,我们都穿着脚蹼,脚步声噼里啪啦的,在狭长的走道里听起来十分的怪异,似乎后面跟着个什么怪物跟着似地,胖子神经大条,对这些没感觉,就是这道太窄,他走起来很不舒服,也直埋怨:“这石道他娘的也不知道是谁造的,摆明了歧视我们胖子,你说这通往天门的天道,怎么寒碜成这个样子,要天上的道都这个样子,弥勒佛都不用出门了。”

我对他说道:“话不能这么说,他这样设计肯定有他的道理,这是船葬,船再大也有个限度,估计他为了突出表现自己的天宫,其他地方只好竟然节约空间了,而且历来倒斗的都是又矮又瘦,谁会想到胖子也能做这一行。”

胖子听了颇得意,说道:“那是,说到摸金一派,古往今来,别的不说,论身板你胖爷怎么样也是第一,不过胖归胖,一点也不影响我的身手是不,这叫——哎哟!”

胖子说着突然人一定,走不动了,我一看,原来他两个肩膀顶住了两边的石壁,卡在了走道里,大笑:“叫你胡吹,自己打自己脸了吧。”

胖子往前动了动,怎么样都过不去,纳闷道:“小吴,你先别笑,不对啊,我刚才还走的挺顺,怎么就卡住了。”

我看了看四周,说道:“看来这石道并不是一样长宽的,可能刚进来那段略微宽一点,现在逐渐变窄了,你后退了几步,看看能不能抽出身来。”

胖子扭着大屁股,使劲往后挪了几步,却还是老样子,说道:“不对不对,不是这个原因,这道明显比刚才窄,我看是这墙有蹊跷,小吴,我看这事情恐怕不妙。”

刚才一路过来一直蒙头就往前走,也没有注意这些墙壁,听他这么一说,我也觉得好像是变窄了一点,于是左右手各撑住一面墙壁,一下一股奇怪的感觉传来,我呀了一声:“不好,这两面墙好像正在合拢!”

闷油瓶也摸了摸墙,点点头,说道:“看样子有变故,没时间了,我们退出去再做打算!”

我一听,心说这可不是闹着完的,被这两块墙板压一下,估计就成三个烙饼了,于是一回头撒腿就跑,胖子看我们跑的如此快,忙用力一转侧过身子,急的大叫:“等我等我,别他娘的光顾自己。”

我从来没跑过这么快,几乎是连滚带爬,几乎全身的力量都用上了,等我跑到出口的地方,那两面墙壁明显又合拢了很多,连我都要侧起身子才能通过,胖子更是不行,几乎是像螃蟹一样只能横着走。闷油瓶伸手就去开那个暗门,弄了两下,突然骂了一声,转过头来对我说道:“有人在外面把门轴卡死了!”

胖子一听,脸都绿了,大骂:“这狗日的天门,这下子完了,你们快想想办法,不然哥几个今天就归位了!”

我急火攻心,看着这石墙一点一点压过来,真他娘的比死还难受,可一时间能有什么办法,这种情况除非有奇遇,否则大罗神仙也没辙啊,说到:“能有什么办法,往前跑吧,跑的快说不定还有一先生机!”

闷油瓶一把拉住我,摇头说:“过去起码要十分钟,来不及了,我们往上看看!”说着双脚蹬住两边的墙壁,就往爬去,我抬头一看,只见上面同样黑漆漆一片,也不见任何变宽的迹象,不知道爬上去有什么用,不过事到如今,总比在这里等死好,想着招呼胖子一起开爬。

这走道变窄,爬起来简直和走路一样方便,我们一路向上,几分钟之内就直爬了十几米,胖子不由咋舌,说道:“还是这位小哥脑子快,这下好了,我们可以在被压成饼之前先跳楼自杀!免的受那皮肉之苦。”

我也没听出来胖子是不是真新的,不过想起要被压成肉饼就一阵恶心,这可不是爽快的死法,说不定你还能听到自己头骨被压爆的声音,我真的是宁愿摔死也不想被活活夹死,这时候闷油瓶在上面叫:“先别胡思乱想,我们还有时间,你们还记得不记得,棺材下面的那个盗洞?”

胖子说道:“当然记得,但是和我们有什么关系?”话一出口他就哦了一声:“我懂了,你是说,我们要学习他的精神,不到最后关头永不放弃是不是?”

闷油瓶说到:“不是,这个世界上没有一个倒斗的会放着地宫不走,反而在地宫的墙壁里打洞钻来钻去的,如果是这样,那只有一个原因,他遇到了什么困境必须在地宫的墙上开洞逃命。”

我一听就明白了,不由心一动,说到:“你是说打这个洞的人,和我们一样,也是在这种情况下才被迫去开这个盗洞的?”

我不得不佩服闷油瓶的思维敏捷,也知道他为什么要往上爬了,这地板和两面墙壁都是青冈石,除非有炸药,否则怎么样也没办法打出个洞来,唯一可能下手的地方,必然只有看不到天花板。

说话间我们已经爬到了顶部,再上去就是一层青砖,我敲了一下,不由大喜,我们料的不错,果然是空心的,这种砖头能压不能凿,有合适的工具,开个孔应该非常方便。不过我望了一下四周,只见一片漆黑,看不到盗洞,胖子说到:“糟糕了,小哥,你说这石道这么长,要是他把入口打在走道那头怎么办?”

闷油瓶说道:“任何人遇到这种情况,肯定先是往出口跑,发现出口的门被卡住了,才会用反打盗洞这种迫不得已的办法,所以这盗洞口必然是在这里附近,如果他打在另一面,我们也只有认栽。”这话说的非常有说服力,我和胖子点了一下头,打起精神开始向边上搜索,这个时候我和闷油瓶的情况还好,侧着身子身前身后都还有一个拳头可以放,胖子已经几乎到了极限了,要缩着肚子才能在这夹缝里移动了,我看的出这给他的压力颇大。就安慰他,说脂肪的压缩比还是很大的,只要墙壁不顶到你的骨头,就不算有事情,他听了脸都青了,摆摆手叫我别废话。

我们从最外面开始,一直往里爬了十几米,但是什么都没发现,其实横着爬比爬高更消耗体力,我的脚已经开始发软起来,几次都差点滑下去,我知道如果两面墙再合拢一点,我的膝盖就要没办法弯曲了,那时候移动起来更困难,前面又黑漆漆一片,不知道那个盗洞究竟开在什么地方,如果真如闷油瓶说的,万一在那走道在另一端,那我还真不知道该怎么面对这种死亡。

早知道这样,也许还是被海猴子咬咬死的痛快多了,人多说粽子鬼怪有多么多么可怕,现在我倒是宁可遇到十几只粽子,也不想一点办法也没有的在这里活活给压扁掉。

这个时候,前面的闷油瓶突然用手电照了我一下,示意我们过去,我和胖子以为终于找到了,大喜过望,忙拼了命的挤到他身边,抬头一看,不由一楞,只见头顶上的青砖上,写了一行血字:“吴三省害我,走投无路,含冤而死,天地为鉴——解连环。”

我看的心惊肉跳,心说这又不是武侠小说,问道:“这~这是什么意思?这个人又是谁?为什么说三叔害他?”

闷油瓶说道:“这个解连环也是考古队的人,就是手里捏着蛇眉铜鱼,死在珊瑚礁里的那个。”

我啊了一声,脑子又是一乱,闷油瓶推了我一把,说:“他既然在这里留了字,又没有被夹死在这里,说明盗洞肯定在附近,现在没时间想他的话是什么意思,我们快往前走。”

我跟着爬了几步,突然想起来,解连环,这个名字怎么这么熟悉啊,好像听我爷爷提到过。

\chapter{脱困}

我稍微一回忆,就想起解连环是谁了,说起来解家和我们吴家还是有点渊源的,可能要扯到表亲的表亲那一份关系上了,俗话说一表三千里,到了我这一代,和他们也并不是很熟络了,但是他们也是一个历史很悠久的倒斗世家,解连环,似乎和三叔走的比较近的一个二世祖,我最多见过几眼,不过爷爷责备三叔的时候,经常提到解家的事情,就说因为三叔,我们吴家这一辈子都没办法在解家面前抬起头来,可惜了解连环这孩子,跟着你还出了事情!

现在想来,原来解连环是这么死的,难怪我老头子不让我跟着三叔混,原来三叔以前有前科在。

胖子在后面推我,我也没办法再细想,咬紧牙关又往前挪了几步,砖顶上出现了一个黑漆漆的洞口,胖子开心的大叫,他其实大限已经到了,前后都被青冈石蹭的血红血红的,好像刚洗了土耳其浴一样。我也比他好不了多少,脚都有点用不上力气。不过现在也不急这一时半刻,闷油瓶先往上一探,钻了进去,踢了踢盗洞的两壁,确定够结实,才把我也拉了进去,胖子就有点麻烦,我一个人还扯不动他,就看他发起狠来,大叫着用力就往上拱,背上的皮的都磨掉一大块才脱身。

我们站稳之后再看下面,不由后怕,两面墙之间已经夹的只剩下一条窄缝,我不敢去想如果我还没脱身现在是什么样子,这一次真是天无绝人之路,再迟几分钟,就算发现了盗洞,我们也爬不进去了。

我又抬头往上看了看,只见这盗洞垂直向上打了大概只有一人多高好,马上变了个角度,倾斜着往东边打去,估计应该是和上面的那个盗洞相连,我的脚直发软,已经坚持不了多少时间了,催着闷油瓶快点向上,三个人爬到倾斜的那一段,吃不消力气,往洞壁上一靠就直喘大气。

这时候下面传来了石墙完全闭合的声音,我长出了一口气,揉着腿,敲着蹦紧的小腿肌肉,尽力放松下来。刚才实在太紧张了,现在人一松就觉得有点发懵,直打哈欠。胖子靠在那里面如死灰,身上都是破皮,一边喘一边说:“这次算是长了记性了,回去之后怎么样我也得减几斤下来,要不然我王字倒过来写。”

这砖头盗洞刚才听他们说过了,打的非常的好,看样子这个解连环也不是等闲之辈,我往上照了照,看着整个盗洞是之字性向上的,在建筑学上说,这样打法,就算发生小规模的坍塌,也不会照成很大的危险,如果为了节约力气一个直井上去,上面的砖头整个儿塌下来,结局和被一只打桩机打了一下没区别。

胖子歇了一会儿,就问闷油瓶:“我说小哥,这到底是怎么一回事情,怎么二十年前走这条道还是好好的,这次就差点被夹死,你是不是带错路了!”

闷油瓶在闭门养神,想了一下说:“这个可能性不大,除非那石碑里指示生门的记号被人调过了,你看刚才情况这么险恶,估计我们是进了死门了。”

胖子就纳闷了,问道:“会不会是那个女人发现我们没死,又来暗算我们?”

我摇摇头,要说她狠毒那我是承认,但我不认为她这个能力去改动几百年前的古墓机关,这实在离谱,但是这里又没有第五个人了,我想了一下,不由有点怀疑,难道是三叔?(前面情节修改后,三叔是在这个古墓里失踪的。)

闷油瓶看出了我的忧虑,拍了拍我,说:“其实我对于这个事情也有一个假设,你如果这么介怀的话,不妨听我分析一下。”

他是这件事情的参与者,而且可以说亲身经历了最主要的部分,他能提供点意见给我,我当然不会拒绝,于是点点头,请他说下去,闷油瓶说道:“先假设,二十年前,三叔和谢连环是认识的,甚至关系非常好,但是他们没有表现出来,在我们第一次拖寻的时候,解连环可能已经发现了海底墓的存在,但是他没有对任何人说,只告诉了吴三省。”

他们两个都是倒斗出身,这个时候自然不会放过这个机会,于是他们趁别人不注意,找了一个时间,偷偷潜入了这个古墓,他们两个人都是高手,这应该一点也不难。然而他们进入了古墓之后,发生了什么意想不到的变故,导致三叔起了杀心,想设计杀掉解连环。

具体过程我们无法知道,但是可以确定解连环在走投无路的时候,在这走道的砖顶上留下了血书,却突然发现这面砖顶是空心的,他随身必然还有一些工具,就极快的打了一个盗洞,保住了性命。

我点点头,分析到这里可以说是天衣无缝,他继续说道。

谢连环脱身之后,想借这个盗洞脱身,他凭借自己的经验,在几次失败后,终于出了逃出了这个古墓,之后他当然马上想去找吴三省算帐,没想到碰到吴三省后,去被反被他杀死。将他的尸体伪装成被珊瑚礁卡住意外死亡的样子。

我听到他这样分析,心里有点不舒服,可是我找不出理由来反驳他,而且他也说了是假设,我定了定,继续听下去。

之后,吴三省为了某一个目的,或者真的是为了躲避风暴,将我们全部带进了海底墓穴,然后自己假装睡觉,这个时候,我发现了瓷器的秘密,将所有人都带到那个水池的底下,这可能是他没有想到的,他没有办法,只好装成被女鬼附身,将我们引进了放置模型的房间,然后在那个镜子后面的通道里,把我们全部迷倒。

他在我们昏迷之后,应该对我们做了一些事情,之后我是出么出来的,其他人怎么样了,我都无法判断,但是我肯定其他人也应该像我一样,失去了记忆,在过去的二十年里,就算见到对方,也只会觉得眼熟而已。我听到这里,反问他道:“为什么三叔当时不干脆杀了你们,这样不是一了白了?”

闷油瓶说道:“我也想不通,不过,也许他当时认为没有杀我们的必要,因为毕竟我们什么都不知道。”

他这样的假设,几乎是把三叔想象成一个处心积虑,早有预谋的大魔头,我实在无法接受,在我的印象里,三叔不会也绝对不是这样的人。

胖子听到这里,好像有所顿悟,对我说道:“小吴,我倒想起个事情,可能能解释这个事情,不过我说了你们可别笑我。”

我一听,现在真是集思广益的时候,胖子脑子直,说不定能想到啥我想不到的事情,忙叫他快说,他故做神秘,轻声说道:“我看,这事情其实很简单,你三叔到了这个地方以后,也许碰到什么……不干净的东西,就中了招了,小哥刚才不是说你三叔学女人梳头吗?你想啊,他这不是提示你们找天门的办法吗?这事情谁知道的最清楚?那就是这墓里的老鬼啊,我看,你三叔肯定给这墓主人的冤魂给控制住了,要是找到你三叔,你直接一盆狗血浇上去,把那鬼逼出来就没事情了。”

我看他说的越说越悬乎,说道:“你这解释他娘的都赶上聊斋了,我和我三叔生活了二十几年,从来没觉得他像个女人过,你这个不算。”

胖子说道:“我可没说这鬼也一定是女人啊,这神经病还分发作和不发作的时候呢,说不定你三叔人前的时候很正常,人后就涂着个胭脂在做刺绣呢。”胖子说了就敲起个兰花指头,我看着好笑,说道:“你以为是东方不败啊,还刺绣,你这个说不通。”

闷油瓶听着胖子说话,说道:“不,他说这个,我看的确有可能,在古墓里,的确有过这种事情发生。”

胖子见有人还同意他的意见,马上牛起来,说道:“你看,我胖子绝对不会瞎掰,我估计着,这和这墓在海底很有关系,风水风水,所谓风声水起,遇水而止,你知道为啥水鬼要找替身吗,因为他的魂魄出不去,这古墓建在水里,风水虽然好,但是对墓主人就大大的不利。”

我听他说的一套一套,也不由的不相信,说道:“要不,咱们先记着,要真能找到三叔,我搞个开个光的佛印往脑门上一印,看看有没有效果。”

我们又各自提了一些想法,这时候我们都缓过劲来了,胖子看了看表,说道:“咱们也别在这里开代表大会了,要真像我说的这样,我们要是在这里饿死,魂魄也肯定出不去,到时连胎都投不了,那就亏大了。”

胖子说到这里,挠了挠后背,又问我:“小吴,你有没有觉得,进了这个古墓之后,不知道什么时候开始,身上痒的厉害?”

\chapter{盗洞}

我正准备开爬,听到他问,不由也缩了缩脖子,刚才实在太紧张了,也没有注意,其实在甬道的时候,我已经感觉被莲花箭割破的伤口,有点发炎的迹象,但是痒着痒着,又似乎好了点起来,我撩开衣服,看了一下伤口,发现伤口上的红肿已经消退了下去,也没有什么异样的感觉。说道:“有感觉,不过现在已经不痒了,这里湿气这么重,可能是过敏吧。”

胖子痒的厉害,说道:“那这过敏有什么办法可以暂时治一下,我刚才出了一声冷汗,现在痒起来没完了。”说着还不停地往墙上蹭,我看他后面都有血条给他蹭出来,觉得有点不对劲,忙让他给我看看,他一边扭动着身子一边转过来,手还不停的挠,我拍开他的手,用手电一照,看见他背部的被莲花箭刮破的伤口上竟然长出了很多白毛,恶心的要命,随口就说道:“胖子,你多久没洗澡了?”

胖子啊了一声:“洗澡?问这个干嘛,这属于个人隐私,我不方便回答。”

我说道:“你他娘的有日子没洗了吧,我告诉你,你也别害怕,你背上好像发霉了,白霉,天下奇观啊,估计你再坚持个几个月还能种个灵芝出来。”

胖子听的云里雾里的,说道:“什么,白煤?煤还有白的?你说话别这么费劲,到底怎么回事情?”

我看着闷油瓶皱了皱眉头,似乎情况不妙,也不敢再开玩笑下去,闷油瓶挤过来用手按了一下,一按就一包黑血,轻声对我说道:“麻烦了,刚才那莲花箭里有蹊跷。”

我觉得奇怪,但是我刚才也中箭了,按道理应该和他一样才对,难道我爷爷遗传给我的体质真的这么特别,我忙把自己的伤口露出来,表示我的疑问。

闷油瓶看了看我的伤口,啧了一声,也搞不清是怎么回事,这时候胖子怕起来,转头问我道:“什么毛!他娘的别没头没尾的,哪长毛了?”说着又用手去摸,我赶紧抓住他,说道:“别动,你好像得啥皮肤病了,让我们再给你仔细看看,你可千万别抓,再抓可就留下疤了。”

他痒的厉害,哪里忍的住,我对闷油瓶说道:“这样下去不行,得想个办法,我听人说过,有些人收不住皮肤病的痒,自杀的都有!”

胖子叫道:“我他娘的现在就想自杀!可痒死我了,要不你就学学关公刮骨疗伤,把那两块肉给我剜了得了。”

我小时候也得过皮肤病,土办法是有一点,就是有点恶心,对他说道:“挖肉是不用,你真以为你肉多啊,我也不是华佗,不过我身上还有点爽皮水,给你先涂上,可能有点疼,你可忍着。”

闷油瓶楞了一下,胖子也啊了一声,说道:“所以说你们城里人就是娇贵,他娘的倒斗还带着爽肤水,下回你干脆带副扑克牌下来,我们被困住的时候还能锄会大D。”

我当然不可能带着这种东西,呸呸两口唾液就涂在胖子背上,带上手套就给他涂开了,没成想胖子这么碍不住疼,口水一涂开他惨叫了一声,人直往前逃去,大骂:“你他娘的涂的什么东西!我的姥姥,你还不如剜了我呢,这下子胖子我真的要归位了。”

我一看,这疼就是管用了,说道:“看你那点出息,疼比痒好熬啊,你现在还痒不痒?”

胖子在哪里手舞足蹈了一阵子,算是缓了过来,奇道:“诶,小吴,行啊,你那什么东西这么灵,还真舒坦多了,那爽皮水什么牌子的。”

我看他要知道我是口水涂上去的,非宰了我不可,忙说道:“别跟个娘们似的,我们快走。”

闷油瓶看着好笑,也直摇头,我还是第一次看见他不是苦笑,不由也觉得他变的似乎有点人情味起来,看样子人之间还是要多交流的嘛。

不过他笑了一之后,又变成一张扑克脸,招呼我们跟上,三个人顺着盗洞迂回着向上,爬了大概有半根烟的时间,闷油瓶在前面说道:“分叉口。”

我挤上去,果然,左右各打了两条通道,我往左边那条照了一下,看到只往里面一点,就有砖头垒了起来,是条死路,看来砖头外面就是闷油瓶他们从右耳室到左配室的那条道。不知道为什么被他给封了起来。难道怕什么东西从那棺材那里过来?

不过他既然封起来了,那最后脱身的盗洞口必然是在右边,闷油瓶和我想法一致,对我指了指,三个人二话不说,继续开爬。

说实话我长这么大还没有爬过这么长的时间,已经汗流浃背,一般的土洞爬起来还没这么累,主要是膝盖没东西顶着,不会疼,现在下面都是砖头渣子,爬在上面像受刑一样,直觉得两条膝盖滚烫滚烫,看样子做人还是有好处的,下辈子还得争取做人。

我胡思乱想着,闷油瓶已经停了下来,做了个叫我不要出声手势,胖子看不到前面,轻声问我:“又怎么了?”

我让他别说话,这个时候闷油瓶已经关掉了手电,我和胖子很知趣,也马上关掉,一下子我们陷入到了绝对的黑暗之中,我这个时候非常的冷静,心跳都没有加速(事后想起来,刚才差点被墙壁夹死的经历对我的影响很大,我在心理上已经克服了对古墓的恐惧)我还不知道他是什么用意,不过在古墓里,听他的总是没错的。

我们安静了一会儿,呼吸平缓下来,身上的汗也干了,这个时候,我听到上面的砖顶之上,有什么东西走了过去,似乎是个人,我心中一惊,看样子我们上面应该已经是后殿或者是甬道了,这人是谁,会不会是阿宁?或者是三叔?

正在猜测,我突然感觉到后背脖子上痒痒的,心里一个激灵,心说难道我也长出毛来了?忙回手摸了一下,正摸到一团湿搭瘩的东西,贴在我脖子上,我以为胖子挤过来了,暗骂了一声,用力一推,把那东西推了回去,手伸回来的时候,突然发现指甲里粘呼呼的,还有股淡淡的香味。

我恶心的把这些东西搽到边上的砖头上,心说胖子的刺猬头上肯定喷了不少发油,呆会儿要是找到水源肯定得好好洗洗,这胖子头上的头油还指不定是几个月前的呢。

正想着,脖子上又痒了起来,这死胖子不知道又在搞什么稀奇的名堂,我不由无名火起,一把拎住那团东西,把他按到墙壁上去,这个时候,我突然发现有点不对劲,怎么这胖子的脸这么小起来。我小心的支起身体,摸了一下,心里咯噔一下,那些湿瘩瘩的东西怎么好像都是头发,我又摸了两把,发现这些头发全部都缠在一起了,手伸进去就被绞住,我咽了口吐沫,开始冒白毛汗,胖子肯定没这么多头发,这些头发是谁的!

我想起水墓道里那团吃人的头发,呼吸开始困难起来,不敢打开手电,那东西好像就离我几个公分,我一开肯定给他对上眼,这种刺激我可顶不住,正想着,我就感觉到一只纤细的湿手一下子摸到了我的脸上,冰凉冰凉的,手指甲非常的锋利,我头皮开始麻起来,脸上的肉不由自主的发起抖。

那手的手指甲刮着我的脖子,然后收了回去,不一会儿,我就感觉到那东西的头凑了过来,那团湿漉漉的头发,一下子贴到我的脸上,我恶心的只咬紧牙关,已经准备爆起了,突然这个时候,那团头发里突然有一个女声,非常的轻,她在我耳朵边说到:“你是谁?”

那声音真的非常的轻,但是我却听的很清楚,不由大吃了一惊,同时这个女人的身体就靠了过来,硬是挤进了我的怀里,纤细的手搭上了我的肩膀,然后搂住了我的脖子,我本能的发起抖来,只觉得这个女人非常的娇小,她的嘴巴贴上我的耳朵,呵出的气都是冰凉的,我彻底懵了,只听她又说道:“请抱住我。”

我听到这句话,就像是着了魔一样,虽然手还在不停的抗拒,但是却根本不听我大脑的命令,一下子搂住了她的腰,这一下更不得了,我一下子感觉到,这女人竟然什么都没穿,皮肤冰凉但是出奇的光滑,我不由心里一乱,脸就红了起来,这个时候,那女人的嘴巴已经移到了我的下巴上,一碰一碰的,好像在暗示我去吻她,我完全失去控制,刚想一头吻下去,突然闷油瓶的手电就亮了,我一下子就看到了我搂在怀里的“东西”,不由头皮一炸,浑身的寒毛都竖了起来。

\chapter{禁婆}

我听到这句话,就像是着了魔一样,虽然手还在不停的抗拒,但是却根本不听我大脑的命令,一下子搂住了她的腰,这一下更不得了,我一下子感觉到,这女人竟然什么都没穿,皮肤冰凉但是出奇的光滑,我不由心里一乱,脸就红了起来,这个时候,那女人的嘴巴已经移到了我的下巴上,一碰一碰的,好像在暗示我去吻她,我完全失去控制,刚想一头吻下去,突然闷油瓶的手电就亮了,我一下子就看到了我搂在怀里的“东西”,不由头皮一炸,浑身的寒毛都竖了起来。

我的眼前一个手掌不到的地方,赫然是一张惨白的巨大人脸,上面的皮肤不知道在海里泡了多少年了,全部都肿成透明的颜色,最让人毛骨悚然的是,它的两只妖眼竟然没有眼白,黑色的眼珠几乎占满了整个眼框,咋一看像极了一具被剜去双目的狰狞的腐尸。

这一下子把我吓得几乎要疯了,我歇斯底里的大吼一声,一把把它推开,拼命往前爬去,脑子里只有一个字:逃。可是那走道很难通过两个人,我和闷油瓶卡在了一起,动弹不得,我看挤不过去,一把抓住他,大叫:“鬼!有水鬼!”他一把捂住我的嘴巴,轻声问我:“别叫!水鬼在哪里?”

我转过身子狂指后面:“就在后面,就……”

话说了一半我就一呆了,心里啊了一下,只见我身后竟然什么都没有,没有人脸,没有头发,连一点水渍都没有,我的手指几乎戳到了胖子的脸上,把他弄的莫名其妙,说到:“去你妈的,你才是水鬼。”

我这下子懵了,忙探头去找,东看西看,真的不见了,但是不对啊,刚才的感觉这么真实,不可能是幻觉啊,难道我真的给这古墓逼出心理问题来了?我心脏还在狂跳,脑子里又一团迷雾,都不知道该有什么反应好了。

胖子看我脸都绿了,就安慰我道:“怎么回事情,你别急,慢慢说。”

我结巴道:“刚才我看到很多头发,裸体女人,还有水鬼!还想亲我!”

我思维很混乱,说了半天也不知道自己在说些什么,胖子最后不耐烦了,说道:“小吴,你该不会是做梦了吧,要真有水鬼,那也得先从我身上爬过去啊?”他拍了拍我的肩膀,又说道:“不过你二十好几了,梦见个裸体女人正常,你胖爷年轻那会儿,也梦见过不少,没事。”我骂道:“你他娘的别寒碜我,我刚才那肯定不是做梦!你看我脖子还湿着呢,就是给它蹭的!”说着我就把脖子露给他们看,闷油瓶和胖子用手摸了一下,都皱了一下眉头,胖子还抬头看了看盗洞的砖顶,以为上面水漏了下来,我和他说这是不可能的,砖头缝里都抹了白膏土,水密性非常的好。

胖子奇怪道:“这就怪了,这里就一条道,按道理要是有什么东西爬到你身上,我不可能不知道啊。”

我说道:“该不会是你睡着了吧?被人从你身上爬过去都不知道。”

胖子没好气道:“去你的,胖子我就算是睡着了,别人从我身上睬过去还能不知道?况且在这里地方,你能睡的着吗?你要是不信,看看我背上有没有脚印!”说着他就一转身,让我们看他的背。

我当时已经缓过劲道来了,也没想到那东西竟然会趴在他背上,胖子一转身,那东西就转过头来,嘴巴直碰在我鼻尖上,我吓得喉咙都抽筋了,吱了一声拼命就往后退去。可是才爬了两步,突然脚上一紧,低头一看,发现小腿上不知道什么时候缠满了头发。我用力想将脚扯出来,但是根本挣脱不开,同时大量的头发开始往我身上缠绕过来,直往我嘴巴里钻,我平生最怕就是嘴巴里有毛,忙用手乱挡。慌乱间,闷油瓶一把扯住我领子,将我向他那里拉去。

他才拉了没几步,自己的手也被搅在了头发里,再也拉不动,我回头一看,胖子已经被裹成个蛹一样,在里面直扭,可那东西却又不见了,整个墓道里面都是头发,就像进了黑色的盘丝洞一样。

闷油瓶用力把自己的手抽出来,连忙问我道:“身上有没有火源?这东西怕火!”

我一摸自己腰里的腰包,摸出一只防风打火机来,不由大喜,这东西是在船上吃鱼头火锅的时候,问船老大拿来点煤油炉的,点完后直接就给我揣兜里了,想不到还真成了救命的家伙,想着忙打起来就去烧身上的头发,那些头发虽然很湿,但是火一烧就能烧断一大把,我几下子就挣脱了出来。忙冲到胖子身上,刚想拉他,突然就从边上的头发堆里探出一张巨脸,几乎一下子就趴到了我背上。

我一看完了,根本没时间躲,头一低,竟然一拳就打了过去,那完全是人到了极端恐惧的时候的条件反射,这一拳我也不知道用了多少力气,只听啪一声,把它的鼻子都打的凹了进去,打出一团的黑水。还亏了我手里的是防风的打火机,这一下子竟然还没熄灭,我咬紧牙关想给它再来一下,却发现那东西一个哆嗦,竟然往后缩了一下。

我一看,突然心中一亮,有门啊,他娘的,果然是神鬼怕恶人,这鬼还怕拳头,我想着脑子也糊涂掉了,竟然兴奋起来,抬脚就朝它面门一踹,把它的脸都踢歪了,直踢回到头发里去。我怕再一脚就要被他缠住了,忙回退了几步,把打火机举起来,和它对峙起来。

那脸藏在头发里,露出一个非常怨毒的表情,但是它忌讳着火,不敢贸然上前,这个时候闷油瓶不知道从那里掏出来几只湿的火折子,往我的打火机蹭了几下就烧了起来,这火大多了,那怪物尖叫了一声,竟然开始往后逃。我看它几下子就缩的很远,把胖子给让了出来,忙趁这个机会把缠在胖子头部的头发烧掉。

闷油瓶一直把那怪物逼到消失在黑暗里,才把手放了下来,这个时候火折子都快烧到他的手了,我低头去看胖子,只间他的鼻子和嘴巴里全是断发,脸都憋的青了,忙用力槌他的胸口,直把他打的突然一口气上来,鼻孔里喷出一大团黑色的东西。

我长出一口气,幸亏胖子肺活量大,一下子自己就把气管通了,不然我就算是死也不会牺牲自己去给他做人工呼吸。

胖子喘了一会儿,把气管里的剩下的东西都咳嗽了出来,才半死不活的问我们:“我的姥姥,那东西到底啥玩意啊?”

我把一直捏在手里不肯放手的打火机按灭,只觉得那打火机已经滚烫滚烫,手上的皮都烫掉了,闷油瓶也比我好不了多少,他甩着手,对胖子说道:“这是应该是禁婆。”

我听英雄山的老海说过这东西,不太相信,啊了一声,问道:“真的有禁婆这东西?”

闷油瓶点点头,说道:“我也不知道这东西是怎么产生的,不过这一代传说很多。应该不会错。”

我觉得奇怪,就问他详细的情况,但是他也只摇头,只说:“禁婆是水里孕育出来的,我知道它肯定怕火,其他我真的不清楚,就像粽子一样,从古至今我们只知道粽子怕黑驴蹄子,但是他为什么怕谁都不清楚,我只是没想到这东西还有思想,我们一定要小心,它肯定还躲在我们后头。”

胖子心有余悸,往我们这里靠了靠,问道:“奇怪了,这墓的风水这么好,怎么里面有这么多稀奇古怪的东西?”

这个墓风水好不好,我现在还真不敢肯定,不过对于禁婆,我倒是查过一些资料,这禁婆在山区的少数民族里其实代表的是巫师和灵媒,可是在海边的老传说里就是天下间最恶的鬼,不知道为什么会出现这样的差异,不过禁婆的下场一般比人惨,要是被人抓住,一般都是直接切断手脚,然后活埋,一说禁婆的起源,一般都是和孕妇有关,放养尸棺的那个耳室恐怕和这东西脱不了干系。还有三叔说过的大肚子壁画,禁婆在这里应该并不是偶然,说不定还是墓主人故意安置的。

我想着,闷油瓶担心等一下那东西又跟过来,招手让我们继续前进,我听了听盗洞顶上,已经没了声音,不知道刚才走过去的到底是谁,我们在下面折腾动静这么大,说不定已经被他听见了,此地不益久留,还是快点开路。

我看了看胖子,他表示没问题,我看他也不想呆在这里,就让他手电打起来,挂在自己腰带上,这样后面我们也能随时注意。我把打火机纂在手里,就继续前进。

我们再往前爬了一段,盗洞突然又开始之字形的向上,我看了看边上,原来他一路打过来到了这里,再往前就是墓墙,估计外面就是海水,他只能改变方向,向上找出路,可能这个解连环的思路也和我们一样,想从墓的最顶端出去。

我们从进这个盗洞开始,一直到这里,大概也就半个小时时间,看样子这个海底墓穴并不大,一路过来,我有了一个大概的感觉,其实这个墓室的长度和宽度并不长,主要的问题还是在它的高度上,现在我能估计到的高度就有将近三十米,那如果按照现在的标准,三米一层楼房的话,这座墓深入海底应该有10层楼这么高,虽然雄伟,但是也不算奇迹。

我们现在没有办法走回头路,只好继续往上爬,又爬了有一只烟的工夫,突然闷油瓶不动了,我推了他一下,他回头,轻声说:“没路了。”

我一楞,不可能啊,忙挤上去看,只见上面果然到了尽头,被几块很大的青岗岩板档住了,我用手推了一下,这些石板非常的重,但是也并不是推不开,我和闷油瓶两个人试着用力往上一抬,抬起来一小条缝,马上,我们就发现上面的那个墓室里竟然有光漏下来,正在纳闷,手上一松,我们头顶上的那块石板突然消失了。

\chapter{混战}

我稍微错愕了一下,马上意识到头顶上的石板肯定是被什么人抬了上去,那一刹那我还以为是三叔或者阿宁,因为古墓里除了他们再没有其他人了,可是我一抬头,却看见一只魁梧的长满鳞片的海猴子,躬起个背,居高临下地俯视着我。我用眼角的余光瞄到它的肩膀上血肉模糊,还插着一只梭镖,心里一叹,真是他娘的冤家路窄,这东西还真贴上我了。

我没想到还会有这么戏剧性的事情发生,一下子不知所措,这时候突然有人拉我的裤子,我低头一看,原来是闷油瓶。他正示意我快下去,我看到这海猴子身躯庞大,马上知道了他的用意,也忙往下爬去。我下面的盗洞是一个斜坡,本来我就是和闷油瓶挤在一起,行动非常的不便了,这下子手忙脚乱更是慢了半拍,才下去几步,海猴子“咕噜”了一声,猛的就探头下来。我看到那张狰狞的猴脸直逼着我就来了,吓得脚下一滑,一屁股撞在盗洞壁上。

这下子虽然屁股巨痛,但是我乘机顺势滑了下去,心说天祝我也,这样就能迅速回到盗洞里面,那海猴子体积这么大,打死都钻不进来,这下子至少可以缓一下心跳。我那时候想的很美,可是天不从人愿,才滑下去半米,突然就发现胖子堵在下面,正一个劲的往上钻,大叫:“上去上去,那鸡婆又爬上来了!”我一听大吃一惊,忙往他身后看去,只间一大团头发已经爬上了最后一个“之”字的转弯处,心里骂了一句,真是福无双致,货不单行,怕什么来什么。我忙把打火机扔给胖子,让他先挡一下,自己抬头去看上面的情况,才刚动脖子,突然肩膀就一阵巨痛,我转头一看,原来那海猴子的肩膀虽然太宽,但是脖子还是非常的灵活,我一个不注意,已经被它一口咬住右肩。

这下子麻烦了,它这一口咬的恰倒好处,獠牙深深的刺进了我的皮肉,疼的我几乎要晕厥过去,缺没有伤到筋骨。我刚想挣扎,它用力一扯,把我整个儿拖出了盗洞。

海猴子将我叼在半空,似乎没有想要马上杀我,但是我知道,只要它用力一甩,就能把我从肩膀处撕成两段,这个时候就算是再怕也必须要反抗了,我突然看到它的肩膀上有我打进去的那一支梭镖,情急之下就是一脚,这一下子正踢到地方,梭镖竟然被我又踢进去四五分。它“熬”了一声,一下子把我甩了出去。

我使尽全身的力气,在地上滚了七八圈,总算缓冲了落地时候的撞击,可是再想站起来,整只右手已经完全使不上力气了。那海猴子疼的脑羞成怒,狂吼了几声又扑了上来,这一次是直奔我的脖子,看样子想直接把我的喉咙咬断。

它来势极快,我避无可避,只好用手去挡。这无疑是螳臂挡车,但是如果不这样,我恐怕连脑袋都保不住。这个时候,胖子突然从后面扑了过来,一下子抱住了海猴子的脚,把它绊了个狗吃屎,两个人同时倒地,滚成一团。胖子非常敏捷,还想学武松打虎爬到它背上去,可那海猴子的力气极大,胖子根本压不住它,被它一脚踢的飞了出去。

我一看胖子也制不住它,心叫不妙,果然那海猴子朝胖子呲了呲牙,转头又向我扑过来,我一看你他妈的是针对我啊!忙去摸腰里挂着的气枪,一摸就想了起来,刚才爬石壁的时候,为了顺利脱身,早就把那长矛一样的枪扔了,如今可能已经被压成一团麻花了。

现在后悔也来不及,海猴子瞬间就到了我面前,我以为它肯定会一口咬住我的脖子,把我的脑袋扯下来,索性把眼睛一闭就在那里等死,没想到它似乎还有气没消,一脚狠狠踩在我的肚子上,这一脚差点没把我的脊椎给踩折掉,我一口血吐出来,疼的几乎失去了意识。它还不罢休,又抬脚想踩我的胸口,可是脚刚抬起来,突然“帮”的一声巨响,我也不知道是怎么一回事情,只见它敖一声就被敲的飞了出去,摔了好几个跟头。

我转头一看,只见胖子天神一样走了过来,手里举着面大铜镜,现在还在不停的震动,我看了咋舌,看来造成刚才巨响的凶器就是这个了,这胖子的手真黑,那一下要是人,就铁定给拍死了,我暗自提醒自己,以后千万不能得罪他。

胖子此时正在气头上,不等那海猴子爬起来,冲上去又是反手一下,同样“棒”一声巨响,那海猴子脸都被敲的变形,又滚出去好几米。可惜这海猴子体格非常的健壮,这几下子没对它造成重创,不过它也知道了胖子的厉害,再也不敢冲过来,几个飞窜爬上了一根柱子,在上面对着胖子直吼。这个时候我已经发现了,这里就是闷油瓶说的放置天宫模型的房间,最直接的证据,就是房间四面墙上,有四幅巨大的影画,我现在没办法仔细去看这些画的内容是否和他描述的符合,但是可以肯定,这里的情景在他们离开二十年后,一点也没有变化。不过让我诧异的是,这个房间并没有他说的那么大,这里能让我感觉他所说的壮观的,只有边上金丝楠木柱,的确是三人环抱,货真价实,其他的东西,顶多只能算是豪华而已。

胖子一击得胜,嚣张起来,骂了一声:“操你妈的,老子粽子都敲死不知道多少个了,你一只破猴子在我面人五人六的,简直不把你胖爷爷当回事情。”说着就想把镜子甩上去,可是这铜镜分量也实在够重,胖子刚才那两下牟足了力气,这一次却举都举不起来,在原地晃了好几个圈。

这海猴子非常狡诈,看他发力不成,突然就从柱子上跳了下来,猛的把胖子扑倒在地上,胖子反应不及被压在了下面,一时间也推不开,结果结结实实挨了那海猴子一爪子,这一巴掌就直接甩掉胖子一块皮,胖子什么时候吃过这种亏,一下子眼睛都红了,狂吼一声,一口就咬住他的脸,那海猴子疼大吼一声,跳起来远远的逃出去好几步。

我看到海猴子脸上的鳞片被撕下来一大块,鲜血淋漓,看上去更加的狰狞,不过它也被胖子搞懵了,变的谨慎起来,开始远远的站着观察我们,似乎想找出胖子的破绽。胖子这个时候也是硬撑着,我看他气都接不上来,体力消耗的很厉害。

双方对峙了几分钟,这海猴子毕竟是动物,没办法和人一样,开始精神不集中起来,它打了个哈欠,转了转头,开始左顾右盼,马上,它就看到闷油瓶正在咬牙把盗洞口的石板盖回去,那石板非常的重,一个人实在很难抬动,他只能一寸一寸的拖着,这海猴子看到闷油瓶一个人落单,杀心又起,大吼了一声就冲了过去。

我心里一惊,没想到这东西也颇有人性,知道吃软怕硬,忙大叫:“当心!!”

闷油瓶已然察觉后面劲风突起,没有办法,只好放下石板,一个打滚先逃过一击,那海猴子一爪落空,马上又是一扑。我知道闷油瓶必然有能力对付这东西,也不是很担心,只见他往前跑了几步,把海猴子引到一根楠木柱边上,突然一跃,第一脚踩到柱子上,然后一蹬,凌空跳舞一样的一个转身,两只膝盖就狠狠压在了那海猴子肩膀上,只把那海猴子压的身子一矮,查点跪了下去。我不知道这是什么功夫,只看的眼睛一亮,不过那海猴子非常的强壮,这一下子几乎没对它造成影响,不过闷油瓶还不罢休,不仅没有立即跳下来,反而双腿一夹,用膝盖夹住了它的脑袋,然后腰部用力一拧,就听一声清脆的喀啦,那海猴子的脑袋不自然的被拧成了180度,整块颈骨都被绞断了。

这一系列动作几乎在一秒内全部完成,简直是秒杀,我和胖子看的下巴都掉了下来,都觉得自己脖子一疼,好像抽了筋一样,我想起那血尸的头,心说肯定也是这样被他拧下来的,不由直吸凉气,这一招太狠了,我都替那海猴子觉的不值。

闷油瓶跳下来后,忙冲回去搬那块石板,我看到一团头发已经从盗洞口里冒了上来,忙叫胖子去帮忙,胖子还是老办法,先用打火机把那团头发逼下去,然后和闷油瓶一起把青岗石盖回了原位。那禁婆很不甘心,在下面撞了好几下,想把石板撞开,胖子怕它把石板撞裂了,索性一屁股坐了上去,把洞口牢牢的压死。

撞击的声音一直持续了十分钟,无奈胖子加上石板,不是一般人能抬的动的,胖子被震的力竭,下面的东西才平息下来。他骂了声娘,累的一下子躺到地板上不动了。

我看危险过去了,长出了一口气,这个时候右手已经恢复了知觉,可以做一些稍微的活动了。我看到闷油瓶走到了东南边的角落里,忙跟了过去,那里的镜子已经被移开了,墙上果然有一个黑漆漆的洞口,只有半人高,里面看上去非常的深邃,不知道通到哪里。

\chapter{墙洞}

这个洞口应该是整个事件中比较关键的一点,闷油瓶的回忆到这里就中断了,以后的事情就是一个迷,洞中有什么,他是怎么出来的,其他人是否像他一样失去了记忆,现在还都是一点根据都没有的推测。

我仔细的打量着这个洞,单从外表上来看,这只能说是一个位置不太合理的人工门洞(除了地道战里,我还没有见过谁会把门开在这个地方),门里面能看到的地方,都是用和外面一样的黄浆砖,在结构上非常的普通,这样的洞我在山西烧炭的工厂里里见过不知道多少,都是用来做砖窑的天井,但是开在这里,在墓穴的格局上就显的非常的突兀,不知道是干什么用处的。

在我的记忆里,几乎所有的墓室都是对称结构的,很少会在一个地方莫名其妙的开个通道或者多一个房间,除非这个墓的主人本身就有这种癖好。如果不是这个原因的话,那么只有两种可能性:

我第一想到的是,里面可能放置了什么隐秘的陪葬品,这倒也并不奇怪,在爷爷的笔记上面,在自己的墓中设计暗室的人比比皆是,但是这些暗室一般都伪装的非常好,这个洞,即没有活门,也没有伪装,单单就是在外面放了面镜子,似乎也太儿戏了。

第二种可能性就是和风水有关系,我推断的理由是,镜子是风水里面很重要的道具,放在这里应该有一种讲法,一般来说,要在一个房间里开一扇门,是风水里“通”的表现,就是说要把什么东西引进来,或者放出去。

这是小风水,和古时候的大风水又有很大不同,就像佛法里的大乘和小乘一样,小风水讲究的是改,就是通过一定的手段,将小范围内不好的改成好的,对于这一块知识,因为比较有趣,我知道要比大风水多一些。

我顺着这面镜子的对角线,走开去查看其他地方,希望能给我找到一些提示。这里整个房间的布置,和闷油瓶说的一模一样,但是因为它还维持着二十年前的样子,所以只有四个方向上是有夜明珠照明的,中间的天宫模型隐藏在黑暗中,只能打着手电看几个局部,我在扫视了几圈后,目光被墙上的影画吸引了过去。

这四幅影画的内容,我之前已经描述过了,但是当时我也是听闷油瓶形容出来的,十分的模糊,现在自己来看,就发现这些画其实非常的写实,只要你够细心,还可以看出很多具体东西来。

首先,我一眼就发现,画中白雪皑皑的山脉,很有可能是吉林的长白山的北坡。这并不是我的记忆力惊人,只是长白山的几坐主峰非常有特点,凡是所有去过那里的人,应该都能分辨出来的。

第二是我注意到了第二幅画里,送葬的队伍,穿的都是元服,这也就是说,这个棺材里的人,应该是一个地位显赫的元朝权贵,那这云顶天宫的修建时间,很有可能是元末朝代交替的时候,在这样的乱世中还有能力修建这样一座巨大的陵墓,这个墓主人肯定不简单。

第三是最让我吃惊的,所有送葬的队伍,都是女人,这实在是非常的不合情理,我不知道蒙古族的墓葬仪式如何,但是全部由女人送葬,真是闻所未闻。

其他诸如此类的小细节非常之多,不知道是雕刻师有意留下的线索,还是他们本身的行事作风就是如此。

我看到这里,心中已经非常清楚,凭借这些线索,只要在当地找一个熟悉地形的山民,绝对就有可能找到这座宫殿的位置,只不过,它埋在几百年的雪层下面,冻土非常的松软,一但挖掘的不小心,一次小小的雪崩就足以让你永远长绵在雪层里。

但是这些提示应该和墙角的洞没有关系,我又去检查其他几个角落里的镜子和后面的墙壁,发现并没有什么特别,看样子所有的问题,只有进了那个洞,才有机会找到答案。我回到洞口,看到闷油瓶仍旧看着,眼睛里出现了少有的犹豫,似乎在考虑什么问题。他看到我走过来,突然对我说道:“我可能还得进去一次。”

“不行。”我听了大吃一惊,“这你不是去送死吗?如果你再失忆二十年,一切都没意义了。”

他淡淡道:“我和你们不同,对于你们来说,这里的事情只是一段离奇的经历而已,而对于我,是一个巨大的心结,如果不解开,就算我什么都记得,这一辈子也不会好过。”

我听了心里急起来,连说不行,其实我并不是不能理解他,但是现在我们所处的环境不容许节外生枝,尽快出去才是我们应该考虑的事情。不然就算我知道世界上所有的秘密有怎么样,空气耗尽,所有的人都会窒息而死,这些秘密也会随之马上失去价值。

我把我的顾虑和他一说,他也表现的有点矛盾,问我:“那你有多少把握,我们能够出去?”

听他这样一问,我才想起我还没有仔细看过这里的宝顶部,忙抬头细瞧。

在我看过的所有笔记里,明墓的顶部都被描述的非常牢固,所谓七横八纵,按照我的想法,这个宝顶为了对抗压力,应该是用拱形的结构,中心高,两边低,但是现在看来,它好像沿用了陆地地宫方法,做成了一个平顶。那么在任何一个地方开洞,都关系不大。

宝顶离我们有十米多高,这里没有可以垫脚的东西,只能先从边上的柱子做文章,用镜腿在上面敲出几个坑出来,然后爬上去,敲裂表面的白膏土,然后开始处理青砖,我们也不需要太小心,只要算好时间,破坏上面的承压结构,上面自然就会塌下一个洞来,我们等到海水把这个墓灌满,就能轻易的逃出去。

这个计划,最关键的就是把握好时间,如果不是在退潮的时候,承压结构一破坏,说不定整个宝顶都会被狂涌进来的海水冲垮,把我们压死在里面。

我把这些和闷油瓶说了一遍,我和他强调,其实我们出去的机会非常大,只不过一出去,这个墓就要彻底完蛋了,但是这个墓并不会消失,里面该有的东西都还是会有,他大可以过几天备好装备再回来,并不急于这一时。

他点点头,终于被我说服了,胖子实在敖不住,说道:“既然这样说,那还等什么,我们干脆现在就动手,先把这柱子搞定。勉的呆会儿手忙脚乱。”

我看了一眼手表,离退潮还有六个小时,时间还很充分,摇头道:“我们刚才体力消耗的非常厉害,又一点也没有进食,人的状态非常的低,这个时候应该好好的休息,等一下我们出去了之后,不知道会遇到什么情况,说不定上面的船已经开走了,如果没体力,出去了又淹死,那太亏了。”

胖子本来积极性很高,听我说的有道理,郁闷的挠了挠头,说道:“他娘的还要等?那行,我先睡会儿,什么时候开工了什么时候叫我。”

我也找了个地方靠着,但是脑子并没有停下来,我算着如果海水开始灌进来,大概是怎么一个走法,现在往池底石碑的通道已经封闭了,虽然不是密封,但是入水肯定比进水要慢,大量水肯定会先涌进那个奇怪的墙洞里,只是不知道这个矮洞通到什么地方去,如果他和其他房间连通,就非常的麻烦,这里会形成一个旋涡,把我们整个儿圈进去。

想到这里,我不由自主的看了一眼洞的深处,盘算着,有什么办法,可能把这个洞堵住,随即我想到,可以把那些模型堆在一起,我估计着这洞口的高度和宽度,想着怎么样来堵合适。

然而在我集中注意力的那一刹间,我的心中,陡地升起了一股极其异样的感觉。

在门洞里的黑暗中,有一股力量,正在强烈的吸引着我的视线。这种力量不仅强烈,还有一定的强迫性,我想转过头去,却发现脖子怎么动也动不了,就连眼珠都没有办法转动。

同时,我立即就感觉到焦躁,这种焦躁,很难形容,就好像一个饥饿到了极点的人,拿到一包食物,却怎么也撕不开包装一样。这种焦躁,很快又在我心里,产生了一股强烈的冲动,想要进这个门里去看看。

这一切几乎都是在一瞬间发生的,一点也没有预兆,他们感觉到不妥时,已经晚了,我一下子推开前面的闷油瓶,向洞里冲去。因为我离那洞口非常的近,所以几步便冲进了黑暗里面,他想拉也来不及。那个时候,我完全没有想过自己到底在干什么,一心只想跑到这个洞的最深处去看看,我连手电都没有打,就在黑暗里向前狂奔,根本不管自己的脚下。也没有注意身后有没有追上来。

可是才跑了几步,突然身后一阵劲风,随即左脚的膝关节一阵巨痛,整只脚使不上力气,扑倒在地上。

这一跤摔的非常厉害,我的额头撞到了地板,疼的我脑子嗡嗡直叫,鼻子都磕出了血来。但是这样跌出了一步之后,我心里的焦躁,突然就消失了,一切都恢复了正常。

我心里咯噔了一下,只觉得有一股说不出来的奇异之感,这个洞穴太厉害了,单单看到一团黑色,就可以让人丧失心智,我刚才入神的一看,便中了招数了。

我回头一看,看见闷油瓶和胖子已经追了进来,有一只手电就躺在一边,看样子就是这个东西,打中了我的膝关节。

他们两个走到我的边上,二话不说,架着我就往外拖,但是我一只膝盖受了伤,站也站不起来,他们拖了几下竟然没能抬动,加上这里光线又暗,场面混乱之极。

胖子看一只手太不方便,就把手电夹在掖窝里,用两只手来拖我,他的动作非常暴力,我被他拉的几乎要休克过去。

就在这个时候,他的手电光扫过一个地方,我的眼睛一闪,好像看到那黑暗里,蹲着一个人。

那光的速度太快了,我没看清楚,但是我很确定,那肯定是一个人,我马上想到了三叔,忙大叫:“等一下,前面有人!”

胖子听了,回头一照,扫到一个背影,但是他已经站了起来,正在快速的向洞里跑去。

这一下子三个都看的很清楚,我们一楞,但是都没有看清楚那是谁,闷油瓶反应最快,立即大叫:“快追!”说完飞也似的追了上去,胖子大骂一声,只好跟上。

我使了几下劲道,只能勉强站起来,一瘸一拐的跟上去,这个时候,闷油瓶已经和那个人扭在了一起,随即胖子也扑了上去,这两个人一前一后,直接把那人按在了地上。胖子拿电筒一照,“啊”了一声,叫道:“是阿宁!”

我跟上去一看,大吃了一惊,只见她蓬头污面,身上的潜水服都被勾破了,身上散发着一股难闻的味道。鼻子和嘴角都有血迹。真不知道她遇到了什么事情,竟然搞成这个样子。不过随即我就发现,其实我们三个也好不到那里去,特别是胖子,简直是浑身是伤口,惨不忍睹。

胖子看到这个女人就有火,指着她的鼻子就开骂,可才骂了几句,闷油瓶突然阻止了他,说道:“等一下,她有点不对劲!”

\chapter{珊瑚树}

闷油瓶话一出,我才发现这阿宁的表情,非常的木然,甚至可以说是呆滞,和以前那种神采飞扬的样子大相径庭。现在被闷油瓶按在地上,也不挣扎,也不说话,甚至看都不看我们,好像这事情和她无关一样。

胖子看着觉得奇怪,说道:“是他娘的有点怪了,我骂的这么难听她都没反应,要在平时,我挤兑她几句,她早一脚踢过来了。”

我知道他手黑,问他:“刚才你有没有下重手,你看她话都说不出来,我看十有八九是你下手太狠,把她给打懵了。”

胖子大怒,说道:“你少他娘的胡扯,我能这么对待一个女士吗?刚才我就按她的脚,还是轻轻的,连个印子都没留下,你要不信就问小哥。”

闷油瓶让我们别吵,说道:“你们放心,她身上没什么大碍,只是神智不太清楚。可能受了什么刺激。”说着他又用手在她面前挥了挥,还打了一个响指,可是阿宁一点反应也没有。

胖子挠了挠头,想不明白,说道:“会不会是这娘们看到什么东西,给吓傻了?”

我说道:“这女人狠的要命,身手又好,她怎么对我你也看见了,这种人怎么可能会给吓傻,你可千万别被给她骗了,说不定她这样子是装出来的。”

胖子一听,也怀疑起来,说道:“你说的是不错,最毒妇人心,我们还是小心点好,要不,我们一人甩几个巴掌给她,看她有什么反应?这女人很要强,我们几巴掌下去,任她是什么贞洁烈女,铜头铁臂,也——”

我看他扯到哪里都不知道了,骂道:“打住,他妈的你革命片子看多了,想学国民党特务?你看她这样子,你下的去手吗?”

胖子举起他那大巴掌,对着阿宁的小脸象征性的甩了两下,发现还真下不去手,泄气道:“只可惜你胖爷从没打过女人,那他娘的你说怎么办吧?”

我和她相处的时间不多,要通过她的动作来判断她是不是假装的,根本不可能,说道:“这东西一时半会儿也判断不了,我看我们把她绑起来,先带出去再说。到时候直接报警,让警察去处理这事情。”

胖子大怒:“你他妈是真傻还是假傻,我们爷三是倒斗的,你知道啥叫倒斗不?交给警察,你脑袋撞猪上了吧?”

我还真想懵了,被胖子一说,真想打自己一巴掌,心说他娘的怎么心态还没调整过来,又把自己当古董摊子的小老板了,忙对胖子说道:“我前几次下盗洞都是赶鸭子上架,心里一直还当自己是个好市民,习惯了有困难找民警,嘴巴一快就说出来了,你就当我放屁。我重新说。”

胖子摆摆手,说道:“得了,我看你也没什么好办法,咱们还是看这位小哥的,指望你,黄花菜都凉了。”

我被他说的没脾气,只好去看闷油瓶,他正在用手电照她的眼睛,看我们转过头来,说道:“不用争了,她的瞳孔呆滞,反应很慢,比‘吓傻了’要严重的多。不可能是装出来的。”

我看他似乎很有把握,没理由怀疑他,问他:“那能不能看出来是怎么造成的?”

闷油瓶摇摇头,说道:“这方面我只懂点皮毛,也是自己做检查的时候听到的,要再进一步判断,我就无能为力了。得去专业的医院。”

我叹了口气,想起这个女人以前那种神采飞扬的样子,不由感叹,说道:“那行,我看一时半会儿,也搞不清楚到底哪里出了问题,我们也别在这里胡思乱想了,先把她带出去再说。”

这提议他们都没意见,一至通过,胖子对闷油瓶说:“那这就这么招,也别磨蹭了,这地方这么邪呼,我们四处看看,如果没什么东西就赶快出去吧。”

我本来已经忘了自己在什么地方,他一说起来,马上觉得一阵寒意,直想马上就走,不过看他们两个人各有各的目的,也不好说出来,只好硬着头皮点了点头。

胖子马上转身,用手电照了照洞的深处,我顺着他的手电光看过去,只见这洞并不很长,在几十步外,已经可以看见底部的东西,但是手电的穿透力不够,只照出个轮廓。

我的视力没胖子好,也不知道里面有什么,现在只指望他什么都看不到,快点死了这条心,这个地方我是一分钟都呆不下去。

胖子仔细照了一下,突然皱了皱眉头,好像看到了什么。我顺着他的视线看过去,却什么都看不到,只听他轻声问我们道:“两位,你们看这最里面,是不是一颗树?”

我“啊”了一声,说道:“古墓里怎么可能有棵树,这里又没阳光,有没人给他浇水,要真有树,也早烂了。”

胖子看了很久,可能也不敢肯定,于是非要指给我看,我没办法,只好顺他的意思,不过我实在是看不清楚,眼睛都瞪的掉下来,也只模模糊糊地看到一棵枝桠一样的东西,轮廓挺熟悉的,但想不起来是什么,对他说道:“我看不清楚,不过那肯定不是树。”

胖子又照了照,顽固道:“我看像是棵树,你看还闪着金光,你要不信我们过去看看。”

我看他醉翁之意不在酒,怒道:“你别以为我不知道你打的什么算盘,就算里面是颗金树,你扛的走吗。”

胖子看被我拆穿了,也不以为然,说道:“能不能扛的走,要去看看才知道,说不定边上还有些小件的,你说我们要是没进来,也就算了,现在进来了,看到有好东西,怎么样也要观光一下!况且我们进来到了这里,不深不浅,要出事情早出事情,没什么可怕的,对不对?”

我心中懊恼,却没有办法,胖子的逻辑我很清楚,观光观光,观察之后就拿光,这家伙简直是一恶魔转世,谁碰到谁倒霉。

我刚想讽刺他一句,就看到闷油瓶做了个别吵的手势,轻声说道:“全部跟着我,别掉队。”说着自己头也不回,径直就向黑暗里走去。

胖子看了大喜,背起阿宁就跟了上去,我只觉的奇怪,但是闷油瓶走的很急,我来不及仔细考虑,只好也先一瘸一拐地跟上去再说。

闷油瓶快步走在前面,这个砖洞从里到外都是一样的宽度,从我们的位置到洞底根本没多少路,我们很快就来到了那棵所谓的树的面前,这里已经是整个砖洞的最里面了,闷油瓶举起手电一照,我们就看到它的真面目。

那是一枝白色的巨大珊瑚,有一人多高,分成十二个枝叉,呈发散状,造型的确十分像一颗树,整个珊瑚雕琢的很好,但是质地非常的普通,并不是非常名贵的东西。

珊瑚种在一个巨大的瓷盆里,用卵石压着,它的枝桠上,还挂着很多金色的小铃铛,胖子看到的那种金光,应该就是这些铃铛反射出来的。但是这些铃铛绝对不是黄金做的,因为它们的缝隙里,已经出现了铜绿,里面的材料,估计是黄铜。外表经过镏金,才能保持现在的光泽度。

胖子没看到金树,大为失望,但是他还没死心,把其他地方照了遍,问我道:“小吴,你说这珊瑚,值不值钱?”

我对这个倒还有所研究,想起刚才他那德性,就有心挤兑他,说道:“不是我打击你,这品质,市场价格16块一斤,已经算不错了。”

胖子听了半信半疑,又去问闷油瓶,闷油瓶点点头,他一下子就郁闷了,骂道:“操,我还以为这次发达了,他娘的结果还是一场空。”

我呵呵一笑,说道:“胖子,你也别泄气,我告诉你,珊瑚虽然不值钱,但是你看这上面的铃铛,这些可是好东西。”

胖子不相信我,说道:“我看你一脸坏笑,你可别胡诌啊,这破铃铛我也倒过不少,也就千来块,你说值钱在什么地方?”

我说道:“就你那点生意头脑,当然看不出来,实话和你说,具体价值我估计不出来,但是肯定比等体积的黄金值钱。你看这些铃铛上的花纹,年代比明代还要早,在那个时候也算是件古董,懂我意思不?”

胖子被我侃的一楞一楞的,也不知道我说的是不是真的,我心里看着痛快,也不说破。其实我也看不出这些铃铛是什么来历,铃铛这东西,在古董里也算是冷门,一般倒的最多的还是瓷器和陶器,金属的东西会生锈,需要特殊的保存方法,这些技术只有大的博物馆能用,百姓家里,就算再有钱,也经不起这样的折腾。何况铃铛又是金属器里比较复杂的,有很多细小的零件,保存的很全的,就非常的珍贵。

胖子琢磨了一会儿,还是不相信我的话,就想摘一个下来看看,闷油瓶一把抓住他,说道:“别动。”

胖子一只脚已经踩上了那是放满卵石的盆,硬是被他拉了下来,觉的奇怪,问他怎么回事情,闷油瓶子不理他,反而问我道:“你还记得不记得这种铃铛,在哪里看到过?”

\chapter{困境}

他这一说,我马上就想起了几个星期前的事情。

那时候我们正准备去倒鲁王宫,经过尸洞的时候,逮到过一只大尸蹩,那虫子的尾巴上,就挂着一只这样的铃铛,里面有一只青色的大蜈蚣,爬动催响铃铛的时候,会发出犹如人窃窃私语的声音,声如鬼魅,似乎有着神秘的力量,我们当时几乎都被这声音迷住,幸亏闷油瓶机灵,一脚把我们踢到水里,才算清醒。

三叔后来看过这东西,说它的年月还在战国以前,具体是哪个朝代他也不知道,不过那时候事情危急,我也没放在心上,后面在鲁王宫里的经历简直像恶梦一样,没疯已经不错,哪里还记得这些。

不过现在要我去辨认,我也不敢肯定,因为当时尸洞里也和现在一样,也就几盏矿灯照明,那铃铛弄下来没多久就给潘子一脚给踩烂了,要两相对比已经不可能,我只能看个大概。

如果这真是尸洞里看到的那种铃铛,那胖子刚才如果一碰,还真不得了,那时候一只已经把我们全部迷的无法自控,这里最起码有四十只,只要一个小小的抖动,真不知道会发生什么情况。

闷油瓶看我想了起来,说道:“那尸洞里肯定还有古怪,那积尸地本来就是一个巨大的墓室,只不过不知道怎么会和汪葬海扯上关系。”

胖子听我们说起过这事情,知道铃铛的来历,纳闷:“你们有没有看错,这战国前的东西,怎么又在这儿出现,这未免也太巧了一点。难不成,这汪藏海,也是个盗墓的?”

他这话一出,我和闷油瓶都楞了一下。

“这样说来,倒也有这个可能。”闷油瓶想了想,说道:“他早年是干什么的,谁也不知道,而且又精通风水,他要是盗墓,应该游刃有余。不过,我记得他家世比较显赫,他们家几代都是风水大家,衣食不愁,总不会做这种下贱的工作。”

闷油瓶说起下贱来,面不改色,似乎没意识到把我们也骂了进去,我说道:“我觉得不太可能,倒斗的,肯定会在自己墓里留下个什么标志,好让后世的近来的时候,有所避忌,你在这里看到这种东西没?”

闷油瓶摇摇头,“我刚才也有留意,确实一点迹象都没有。”

他在这方面的造诣深不可测,他说没有,我知道必然是真的没有,说道:“那这样何以解释这里会有这么个东西,会不会他本身就好古董,把自己心爱的藏品也拿来陪葬?”

“我们一路过来,也没看到其他的古董,你说的也不对,我看,可能是另一种情况。”胖子似乎想到什么,面露得意之色:“其实除了倒斗的,还有另外一种人也经常会碰到古墓,你们知道是什么吗?”

我听了马上就醒悟了:“你是说,他是在做工程的时候,在工地上挖到这些东西?”

胖子点头:“这人可说是当时最大一包工头,很可能会碰到这种情况,只要回去查一下资料,就能知道那个时候,他有没有去过山东的瓜子庙。”

胖子的说法合情合理,我不由又有些佩服他,不过这东西决计是不能碰了,我猜想可能阿宁就是碰了这颗珊瑚树,这么多铃铛一起响起,才会变的精神失常,只是不知道这些铃铛在她大脑产生什么景象,会有这么厉害的效果。

本来人就很容易受到暗示,现在又是在这么一座古墓里,气氛神秘,神经稍微脆弱一点,自己就会疯掉,我觉得,甚至闷油瓶的失忆,也可能是这些东西造成的,因为我发现这些铃铛的挂绳都用铜丝很精确的绑在珊瑚树上,珊瑚本来里面就有空洞,传音极佳,这东西摆在这里,就像一件乐器,发出的声音可以有千万种,难保里面有一种就能让人忘掉一切。

不过我这些想法有点天马行空,也不好意思说出来,三个人呆立了片刻,胖子就说道:“看来这洞底也就这么点花头,这蹊跷还在这些铃铛上面,要不扯呼?”

我看这洞也没什么妖魔鬼怪,心里也放松不少,现在走不走倒也无所谓了,不过看表,退潮的时间也快到了,在这里呆着也没意思,四个人就向后退去。

我边走边想,心里还有两个疑问,第一是闷油瓶二十年前进这个墙洞的时候,是被三叔引进去的,和他一起晕倒的那些人,现在在什么地方?是不是三叔把他们运了出去?

第二是闷油瓶当年进去的时候,闻到了一股非常奇特的香味,现在却没有了,难道这表示,二十年前,这洞可能还有什么其他东西在?

这些答案,必须要找到三叔的时候才能知道。

而三叔又不见了,要找到他,不知道猴年马月,说不定他就此不在出现,这些疑问就要变成千古之迷了。

如果真如胖子说的,三叔是被这墓里的冤魂给缠住了,那他会到什么地方去呢,他看到闷油瓶的照片时,说的“我明白了”,到底是明白了什么呢?

想着我就觉得整个事情还缺一点东西,只要再给我一点线索,我就能把所有的事情连起来。而我的直觉告诉我,这东西应该和鲁王宫有关。

我想着,四个人已经走出了那个矮洞,胖子把阿宁放到地上,就说道:“现在时间应该差不多了,我们怎么样也该动手了。”

我想到现在出逃的事情还是头一等,就收回心神,开始交代事情,因为我从来没真正开过明墓的宝顶,所以心里也没有什么把握,只有走一步算一步。

说完之后,三人依计行事,胖子老早憋了一肚子劲,抄起家伙就在一根柱子上凿开了,可他小看了金丝楠木的质地,几下子下来,已经喘的不行,可柱子上就被他劈掉一点。

他看了不对劲,说道:“小吴,这柱子也太结实了,要照这样弄法,一个礼拜这梯子也搭不起来。”

我说道:“你先别急,只要你劈掉最外面那层,里面就好对付。”

胖子半信半疑,拿着家伙使上十二分的力气,才勉强有了点起色,几下过后,胖子已经拨开外面铁一样的木质层,掏出一个可以容纳一脚的空间。

我现在知道了那洞是死路,海水进来,只能透着砖缝往下漏,不用担心会产生旋涡,就抄起家伙就去帮胖子,才砸了两下,才发现这活还真得他干,他力气大不说,耐力还好,这一路折腾下来,只见他生龙活虎,没有一点疲态。我自己在他边上一点,同样的时间,已经累的几乎手都抬不起来。

我们干的昏天黑地,三个小时后,终于在一根柱子上码好脚洞,这底下的尚且好弄,一直到上面,要踩着已经码好的爬上去,悬在半空中,力气都使不上,最后只有浅浅的弄出一个可以放进一个前脚掌的印子,不过不管怎么样,还是给我们搞定了。

我们把自己的裤子衣服都脱下来,因为都是潜水的衣服,很有弹性,索性割成一条一条的绑成一根绳子,像墨西哥爬树人一样做了一个绳套,围着这根柱子圈起来,三个人三个方向,将绳子绷直了,就向上爬去。

这一路也不知道是怎么爬上来的,每上去一点都像死一次样,胖子累的直叫唤:“你们两个跟上来干啥,我上去凿了就行了,反正水下来,你们能浮起来,现在这皮绳都快把我扣成东坡肉了,小吴,你他娘的还是给我下去,不然我顶不住了。”

我说道:“你以为我想上去,我是没看到实际情况,不想你送死,这上面不知道有没有夹层,如果有的话,你一家伙下去,流沙下来就直接把这房间整个儿埋了。”

我那是实话,墓墙里的流沙层是最常见的反盗墓措施,前面也说过了,是比较有效的,一个有流沙层的大墓,如果要顺利进去,就要在下盗洞的时候开一个下沙井,把流沙先放出来,有时候放空一面墙就要几天几夜,说明这流沙量的惊人。我们现在没这个条件,如果真碰到这种墓穴,就只好另想办法了。如果上面不是流沙,而是强酸或者火油,那就更糟糕了。

胖子倒的斗多了,自然知道我说的不假,挥了挥手示意那就爬吧。

我们咬紧牙关,又花了半个小时,才到了最上面,胖子站稳之后,几乎力竭,抱着那柱子一动也不动,说道:“他娘的,要再这样折腾我,我可就归位了。”

我让他喘口气先,等一下凿砖还得靠他,自己小心的试探着敲了敲宝顶,闷油瓶示意我不要停,自己把手指按到顶上,感觉了一下,说道:“实心的。”

胖子听了,他也实在不敢休息,二话不说,就开始凿顶上的白膏土,他不敢太用力气,因为到底这绳子不结实,万一断了,全部都得摔成重伤。

我们都伸着手,搭在他肩膀上,万一这绳子一断,还能拉他一下,不至于直接从十米高的地方摔下去。不过他一声的油汗,估计真要掉下去,要抓也抓不住。

白膏土很脆,他凿了几下,就剥下来一大块,露出了里面的青砖,胖子看了一眼,突然叫不好,忙叫我摸,我用力探过手去,一摸,傻了。

这些砖头之间,竟然浇了铁浆。

\chapter{炸弹}

我们三个人互相看了一眼,脸色都很不好看。

砖头用铁浆浇死后,就和现在钢筋混凝土一样,你就算在平地上给你只大锤子,也无济于事,不要说现在这种情况。

在这上面,最起码还有七层这样的结构,而且互相错落,要凿开这里,没有现代化的设备,已经绝无可能。

我心里懊恼,只怪自己怎么没想到这一层,平顶的抗压性大不如拱顶,那上面的砖头肯定要用东西加固,明墓里对这一套东西没什么办法,都是千篇一律的用铁水浇死,自己凭借一本笔记加上三脚猫的建筑常识,纸上谈兵,满口喷粪,现在总算吃到苦头了。

胖子看着我,问道:“建筑师同志,现在怎么办?你给拿个主意。”

“那能怎么办,死马当活马,先抄家伙上,”我还想碰碰运气,说道:“二百多年了,我就不信它还这么结实。”

胖子看我也没怎么慌张,以为问题不太严重,就去敲那些砖头,空心的砖头很好敲碎,但是砖头碎掉之后,边上铁浆凝固成的铁条还在,胖子十分力气打上去,也只是在上面敲出几个印子,他一看就知道不对劲,说:“不行,这铁浆条子往上都有一个巴掌厚,你开量解放卡车来都不一定能撞的穿。”

我也敲了几下,整的虎口发麻,知道这的确不是蛮力可以搞定的东西,不由泄气,“看样子我们小看古时候的建筑工艺了,这铁条子纯度很高,根本砸不动。”

胖子说:“要不磨磨看,古人不是说嘛,只要工夫深,铁杵磨成针。”

“拉倒吧,你么厚的铁浆条子,你磨到猴年马月去,”我说道:“还有二十分钟就是退潮了,等你磨完了,我们早圆满了。”

胖子火了,“那你说怎么办?你没听那女的说过吗,这一带不久就是风季,起码要持续一个礼拜,我们现在出不去,就只能在这下面呆上七天。”他着重强调,“七天,他娘的我们不闷死也饿死了。”

我知道问题的严重性,对他们说道:“你们这方面的经验要比我丰富的多,就这种墓墙,如果是在平时碰到,你们会用什么办法?”

胖子和闷油瓶想都没想,一齐说道:“炸药!”

胖子看我楞了一下,解释道:“你不用奇怪,这种墙的结实程度超乎你的想象,老子以前倒这种斗,洞一般都开在底上,如果非要在这种墙上硬打进去,只有用炸药。”

我听了心里悲凉,他说的情况我也知道,但是在这几百年的古墓里,叫我到那里去弄炸药?想来又不由后悔,我记得在下水前,那阿宁还问过我要不要带一些下去,当时我因为给鲁王宫里那一炮给炸懵了,对这东西十分的抗拒,直接就给扔回仓里了,如果三叔当时在场的话,肯定就会带上。

现在想起来,那时候的想法太幼稚了,如果还有下次,绝对不能这么儿戏。

我看从上面出去是没希望了,只好放弃:“那看样子这法子行不通了,我们还得从长计议。”

“他娘的还从长?我们只有二十分钟都不到了?”胖子说道:“我看,要实在不行,我们还从原路摸回去,说不不定那放着我们潜水设备那墓室已经回来了。”

我点点头,虽然我很不想再进那盗洞,但是现在也没有别的办法。如此一来,又要面对盗洞里那怪物,实在是头疼的事情。

这个时候,闷油瓶突然说道:“等等!你们先呆在这里别动!我想到有一个地方可能有炸药!”

没等我们反应过来,他就突然一松绳子,滑下了柱子。

胖子看了看我,一脸的迷惑,我朝他摇摇头,表示我也不明白。

闷油瓶性格一本正经的,不可能开玩笑,但是又实在想不这里什么地方会有炸药,他现在和我们一样已经脱成光条了,就剩一条内裤,也不可能藏在什么地方。我盯着他,只见他一个飞身就跳上了房间中间的天宫石盘上,顺着他的手电光,我就看到他蹲到石盘中心的一具打坐的干尸前面,不停的摸着什么。

这具尸体应该就是他说的坐化金身,只是不知道他到那里去找什么,我想着,突然间,我心里就啊了一下,原来是这样!

这个时候闷油瓶已经把整具干尸小心翼翼的抬了起来,干化的尸体几乎就只剩下骨头的重量,并没有废多大力气,胖子问我:“他到底在干什么?”

我说道:“我也只是猜测,那干尸体内,可能有一个机关,由八宝转子击发,里面可能有炸药。如果对尸体不敬,想取尸身内的宝物,可能就会直接引爆。”

胖子听了咋舌:“他怎么会知道这种事情?”

“二十年前,他摸过这具尸体的时候,那个时候可能已经知道了,你看他刚才只是说‘可能’,就是说他也不确定。”我说道:“只是不知道,这几百年的炸药,还管不管用。”

我说着,闷油瓶已经把干尸搬到了柱子底下,对我们说道:“下来一个帮忙。”

我看胖子下去实在太麻烦了,就让他呆着,自己爬下去,闷油瓶把那干尸过到我背上,用绳子捆住,说道:“千万别撞到,如果里面的机关还管用,一触即发。”

我近距离看到这具坐化金身,只觉得闷油瓶刚才的描述不及这真实的万一,这尸体全身发黑,黑到发亮的感觉,好像不是肉身,而是用什么光滑的材质雕刻成的,肌肉都已经凹陷,特别是嘴角,似笑非笑,看了直出鸡皮疙瘩,总之一句话,这尸体,根本不像在寺院里看到的那些高僧,反尔让人感觉十分的不祥。

我看着实在不敢碰,问他:“你确定这尸体没问题吗?我总觉得,他好像有什么诡计,你看他的表情,怎么这么的——这么的——”

“妖异。”闷油瓶接着我的话说道:“我也不明白,这具尸体的确给人不舒服的感觉,但是他已经干化了,无法尸变。”

我点点头,冷汗都冒了出来,问他:“那就好,你确定这里面的炸药还能用?”

他说道:“只要八宝转子能用,炸药肯定能用,现在就怕这机关老化了。”

背了具干尸在身上,我浑身不自在,特别是看到他的指甲这么长,横在我的面前,鬼森森的,脚都有点软,我想起湘西的赶尸匠,就是像我这个样子把尸体背在背上,但是人家是里三层外三层的包起来的,我倒好,干尸裸体,我也裸体,肉贴肉,那种干巴巴的感觉真他娘的别提多寒人了。

不过现在也没办法,还好光线还可以,我还能看的清楚,不至于胡思乱想,我咬紧牙关,就当着身上背着个麻袋,开始一步一步向上爬,闷油瓶爬在我后面,防备着我如果脚滑,失足掉下来。

我爬了有五六步,突然觉得那干尸体有点不对劲,因为我的后背就贴着它的尸皮,所以感觉的非常清楚,那尸体好像突然变大了一点。我停下来仔细感觉了一下,又感觉不出什么特别的异样来。

我回头看了看闷油瓶,他在我下面,如果尸体有什么异化,他应该能马上看见,但是他好像什么都没发觉,难道是我自己多心了?

也难怪,背着具这么妖异的尸体,很难不多心。

想着,听到胖子在上面催我,我只好继续向上,因为过于紧张了,脚都有点抖,我想早点结束这种情况,三步并两步,好不容易爬到顶端。

胖子可以说阅尸无数,不过看到这具尸体后也露出了不太舒服的表情,毕竟,你用绑尸绳挂着尸体的时候,还有两三拳的距离在,现在就像跳贴面舞一样,感觉肯定难受。

我硬着头皮,对他说到:“你把这个固定到宝顶上去,然后马上下来,我们在下面引爆,如果里面的机关还能运作,应该没有问题。”

胖子看了看宝顶,说道:“你唬我呢?我他妈的怎么固定?你想让老子学董存瑞吗?”

我一抬头,宝顶上面没什么可以钩挂的地方,如果要把爆炸的力度全部发挥出来,必须把整个尸身紧紧贴着宝顶,这的确是个问题。

我想了想,说道:“实在不行,就把它头朝下绑在这柱子上,快一点,时间快到了。”

胖子把尸体小心翼翼的接了过去,摆了摆,问我道:“哎,真奇怪,这尸体怎么还有条尾巴?”

\chapter{脱皮}

“哪里来的尾巴,我刚才怎么没看见?”我以为他在拿我开涮,说道:“你可别拿我开心。”

“这不就是?”胖子一本正经指给我看:“你眼神也太‘神’了,这么突兀一根东西,都看不见?”

我顺着胖子的手指看过去,看见坐化金身的尾骨上,真的有一根突起,三寸长,两根手指粗细,黝黑黝黑的,看上去与尸体本身的干化程度一样,看上去有点像硬化了的牛尾巴,向上弯曲着。

我觉得奇怪了,刚才搬动的时候,好像没见过这东西,难道是刚才长出来的?

回忆了一下,也没个头绪,刚才人高度紧张,到底有没有看到,自己也记不清楚了,我心里陡然升起一股寒意,突然有一种十分不吉祥的感觉。

随即我提醒自己,现在不是怪力乱神的时候,而且就这么一根干巴巴的东西,也不能肯定这是尾巴,于是对胖子说道:“你结论也别下的太早,人身上怎么会长尾巴,别是人的鸡巴,你仔细再看看。”

“去你妈的!”胖子大笑:“鸡吧能长在屁股上?再说了,谁死了还这么——这么——”

我知道他想说这么,马上打断他的话:“得了得了,你管他是什么,反正呆会儿炸完后连渣都不会剩下。你再研究,过几年就该别人研究我们了。”

胖子被我一句话提醒,当下反应过来,也不去管那根奇怪的东西了,忙下手干活。

我帮着他把尸体倒了个转,把本来用来辅助爬柱子的绳子取下来,艰难的把干尸固定到柱子上去,现在还没办法估计爆炸会有多剧烈,不过我记得听三侠五义的时候,那里面的九子连环炮已经可以把十层的金刚岩崩裂,这玩意照道理也不会差到那里去。

绑好之后,我用力扯了一下,慌慌张张的,弄的也不甚结实,但是应付一段时间应该够了。

当下我也不想再呆在上面,检查一遍,见一切妥当,就准备下去。

一想到爆破的时间就要到了,我心里就禁不住的紧张,现在行不行就看这一招了,只求上帝保佑,这其他的事情,出去了再说,我也不奢求什么都顺利,至少给我小命保住。

正胡思乱想着,胖子拉住我,说:“等一下,我还缺一点没弄好。”

我刚才全部检查过一次了,听了一楞,“缺什么,这不都齐了?”

胖子让我先别下去,然后转过头去,对那干尸体说:“这位尾巴前辈,不管你是人是猴子,你都已经归西了,这臭皮囊对你也没什么用处了。虽然我们拿来当炸药包是过分了一点,但是实在是形势所逼迫,你大人有大量,千万别和我们计较,等一下你就当蒸个桑拿,与世无争,百无禁忌。”说完给那金身象征性的拜了拜。

我大怒,扯着他的内裤就往下拽,骂道:“他娘的,什么时候了,你还有心思玩这一套!”

他直溜一身就猾到我边上,说道:“你不懂,这东西看着就邪,难保不会找我们晦气,而且人家在这里坐的好好的,我们把他拿来当炸药包,本身是我们不对,怎么样过过场子的话还是要说的。”

我边爬边骂:“少来,你搬十二手尸的时候干嘛去了?也没见你给人家磕头?现在他只不过长条尾巴,有什么大惊小怪的。”

这南北两派的矛盾就是这样产生的,可以说是意识形态的不同,胖子听的不爽,闷哼了一声,转头去不理我了。

我们下到地上,闷油瓶背起阿宁,招呼我们到墓室的角落,我们把其他几面铜镜搬到自己面前,当成盾牌一样,万一等一下炸弹威力太大,不至于被碎石误伤。一切就绪,就等时间一到,靠闷油瓶精准的技术,将一根镜腿,甩过去引爆金身肚子里的机关。他在鲁王宫里飞刀几乎就把胖子定死了,这一下子应该不成问题,而且这个时候考虑其他方法也没有用,我一边祈祷,一边集中精力看表。

海水涨落潮规律是:每天涨潮有两次,相隔12小时。高潮时间一般能维持一个多小时才开始退潮,最低潮时间在两次高潮中间的时间。这个时候海平面最低,有的时候甚至会露出海底。

不过这里的海底应该不会这么浅,不然这里搁浅的船,会比现在多的多。我估计,如果能将到二米以下,那是非常理想的。

我不知道低潮能维持多久,在我记忆里,应该是非常短的时间,我们需要等水把上面的破口冲大,会耽搁一段时间,所以刚开始一分钟都不能耽搁。

这还是比较乐观的估计,其他可能还会有突发情况,到时候只能随机应变,我想着越来越没底起来,到底是自己胡乱说出来的,如果等一下情况没我想的那样发展,而是整个顶整个儿塌下来,那可真对不起他们几个了,我想着,人也不由感觉到紧张起来。胖子看我表情,大概知道我有点心虚,不安的问道:“两位,实话告诉我,你们是不是也没啥把握?”

我不知道怎么回他,敷衍道:“现在这情况,都不好说,反正箭在弦上,你等一下看着就是了。”

胖子叹了口气:“真是,你越我越觉得慌,你说等一下要是这东西不爆?你们还有没有其他对策?先说出来,也让我心里安一点。”

我说道:“办法倒是有,就你刚才说的是一条,原路回去,看看我们进来那墓室,有没有重新出现。要不然,还有个不是办法的办法,就是在这里呆着,等第三拨人进来救咱们。”

胖子说道:“那哪能等的到,他们要不进来,我们怎么办?等一辈子?那不变成西沙海底活死人墓,摸金校尉绝迹江湖。”

我安慰胖子道:“我的意思,这里虽然险恶异常,我们一时走不了,也不会马上死,只要有时间,我们再从长计议,总能想出办法来,你看这里的空间大,空气还够好几天的,我想一个星期问题不大,我们多睡觉,少运动,尽量节约着用。”

胖子不吃这一套,说道:“空气够,你也得吃东西啊,这里又不是深山老林子,啥也没有,连西北风都没的喝,我宁可闷死也不想饿死。”

我笑了起来,说道:“办法是人想出来的,你看这身膘,饿个个把星期也饿不死。你要真饿的不行,还有只海猴子呢,吃了海猴子,要还不顶饿,那就把下面那禁婆也逮来剥了。”

胖子听了也乐了,这家伙只要有人跟他抬杠他就起劲,拍我的肩膀道:“行,你这句话说的颇有胖子我的风格,干革命就要有天不怕地不怕的精神,看样子这一次的确长进了不少。”

我话出口也挺吃惊的,怎么我也开始说起这种不着边的话起来了,看样子是给胖子影响了,不成,绝对不能变成胖子那样。当下我就不在扯皮,继续注意我的手表,还有五分钟,这个时候如果要引爆,应该也没多大的区别了,我对闷油瓶说,让他好准备一下,别等一下失手了,那金身绑的本来就不牢固,呆会儿掉下柱子,在下面爆了,可不是好完的事情。

闷油瓶掂了掂手里的家伙,点头同意,这个时候,突然胖子叫了起来:“吓?那干尸呢?”我们一听坏了,猛抬头,发现柱子上的那尸体竟然没了,我第一反应就是刚才没绑结实,掉下来了,往下一看,地上也没有,不由大骂,这下子真邪了门了。

这节骨眼上出这种事情我可真没想到,刚才预备着随机应变,都是自己安慰自己的,没想到这么快就应验了。

“你看你看,我说吧,他娘的有尾巴的东西肯定邪门。”胖子叫起来:“快找找在什么地方。”

我们一齐冲了出去,一眼就看到,我们要找的那东西正扒在柱子后面的宝顶上,用指甲紧紧抓着上面的浮雕,身上的黑色硬皮已经尽数龟裂,正一片一片的掉下来,里面血淋淋的,不知道是什么。

我看到绳子还绑在它的腰上,因为那是几股潜水服的材料做起来的,绑一个人还是非常的牢靠的,所以它也一下子没挣脱开,不过看这情况,也支持不了多久了。

胖子看了叫起来:“快,趁他还没逃了,先引爆了再说!!”

闷油瓶哪用他提醒,胖子话才起了个头,我就听一声破风,同时一到青光已经飞了过去,直插那干尸的肚子。

\chapter{脱出}

我大叫不秒,这闷油瓶也动作太快了,我们都还冲在外面,这样一下子,万一爆炸,我们肯定得遭殃。

可等我想到已经来不及了,就见眼前突然白光一闪,胖子已经一把我把扑倒在地上,然后就是一声巨响,整个墓室猛然巨震,一股滚烫的气浪直接把我们掀了起来,我足足在空中打了六七个转,被炸到三丈外,一头撞在墙上。

这一下真是实实在在挨了,好在胖子把我扑倒,不然脖子肯定就断了。我撞上墙的一瞬间失去意识,什么都看不到,就听到耳朵嗡嗡直响,还以为自己死了,不过过了一会儿,眼前突然就有光了,我试着睁开,马上就看到天旋地转,满眼的黄灰,头晕的直想呕吐。

我艰难的爬起来,已经听到很多乱七八遭的声音,但是我没办法去分辨他们,只觉得吵的厉害,头痛欲裂,混乱间闷油瓶咳嗽着从烟雾里跑了出来,问道:“有没有事情!”

我说话都咬到自己的舌头,对他摆手,表示还行,我们两个捂着嘴巴去找胖子,我跑了两步,一下子就看到胖子坐在那里,肩膀被一块碎砖削去一块皮,看到闷油瓶,破口大骂:“我操,你他娘的动作也太快了,至少等我们先退几步,老子再往边上挪两公分,一只手就要报废了。”

闷油瓶一摊手,让我们看他手里的镜腿:“你弄错了,刚才不是我!”

“啊!不是你!”我们两个同时大吃了一惊。

刚才那劲道,那准头,绝对是极其厉害的人,不是他会是谁?胖子刚才就在我身边,而且看他那样子,准头绝对没这么好,我就更不可能,要说其他人,只有一个——我心里灵光一闪,忙回头去找阿宁。

胖子和我想的一样,我们两个跑到角落里一看,哪里还有她的影子,胖子骂了一声:“是那婆娘!他娘的她果然是装的!”

闷油瓶露出了不敢相信的表情的,看样子他对自己刚才的判断很有信心,没想到会出错误。我对这个女的又要重新估计,说道:“这女的真是个高人,我看像江湖上的老油子了,我从来没见过一个装傻,能装的这么像。”

胖子说道:“我看哪止是老油子,简直就是他妈的奥什么卡的影后,下次逮到她,她装什么我都不信。”说完抄起家伙就要去找,闷油瓶忙拉住他,说道:“没时间了,算了。”

我也劝他:“不要节外生枝,我们现在最重要的是去看看有没有把宝顶炸开!你要咽不下这口气,也等出去再说吧。”话音未落,突然从顶上传来一声十分悠长凄凉的声音,似乎又是一根什么东西正在缓缓断裂。这声音不大,却让我一下子把心吊到嗓子眼上去了,心说不会吧,就这样一个炸,你就要塌,你也太给我面子了。

胖子本来还很不甘心,一听这声音脸也白了,问我:“这他娘的什么声音?小吴,看这情形,好像比你说的炸出个洞要严重的多啊?”

我抬头去看那炸出来的洞,不有咋舌,那干尸肚子里的炸弹威力颇大,超呼我的想象,那上面的铁浆条子已经全部都炸断,足炸出一个直径半米不到的洞,砖顶上方的防水层被炸裂,海水涌进来,形成了一个小瀑布,我刚才听到的奇怪声音,就是瀑布不断变大的水声,估计再过不久,洞口就会被会完全被冲垮。

而边上的金丝楠木柱子已经被炸断,一条巨大的裂缝一直从上裂到底部,并且有倾倒的迹象,这根价值不菲的柱子,算是彻底报废了。

看来就是因为断了根柱子的原因,上面有一条横粱受到了影响,可能真的会塌下来,听这声音,这横粱必然已经出现了裂缝,就算现在不塌,过一段时间肯定劫数难逃。

我安慰胖子,说道:“没事,你放心,这墓比一般的墓要结实多了,只要不现在不地震,肯定塌不下来。”

话还没说完,脚下的地面突然开始震动起来,我早就预料到这个海底古墓的气密结构被破坏,下面的海水肯定也在不停的涌上来,只是没想到动静竟然着么大,不由紧张的有点晕眩。

那震动越来越剧烈,非常的恐怖,而且这恐怖实实在在,更加的真切,如果再按这样的速度发展下去,恐怕这宝顶还没塌下来,我们站的地板倒要塌了。胖子被吓得不行,叫道:“我的怪怪,怎么这会二又地动山摇的,该不会真是地震了吧,我说小吴,你刚才炸的到底是什么部位?”

我解释给他听,然后对他说道:“没事,正常现象,我们做好准备,说不定等一下这里所有的缝里都会有水冲出来,小心被水喷到,这压力不得了,就像拳头一样,碰到能冲你个跟头。”话音刚落,突然一声怪响,那块盖着盗洞口的青纲岩板被一股急流冲飞了起来,海水就像喷泉一样直冲到七八米高。我还没反应过来,紧接着,又见一个东西从那盗洞里喷了出来,直撞上宝顶,然后摔到中间的石盘上。速度太快,我也看不清楚是什么,不过这盗洞里也没其他的,估计是那禁婆。

这东西被冲出来,又是个不大不小的麻烦,说不定还会是个很大的麻烦,在水里也没办法点火,要是被他缠住,那更不堪设想。

可惜现在我没功夫考虑它,那盗洞口边上的整个地面拱了起来,就像火山喷发一样,汹涌澎湃,而且水位上的非常之快。几乎就是瞬间,我们已经漂到离地面五六米的高度。

我四处去找阿宁,这时候爆炸产生的烟雾已经消失的差不多了,但是仍旧没看到她,估计可能在某根柱子后面,胖子水性不太好,游的非常吃力,无力再去理会她,不过这里就一个出口,等一下无论如何我们也会碰到一起,胖子朝我直使眼色,大概是想等一下找找她的晦气,我对女人还是下不去手,就不去理他。

我们又漂了几分钟,脑袋已经顶在宝顶上了,突然胖子就向边上游去,我不知道他想干什么,大叫:“就一分不到这里就要全没了,你搞什么,不想活了?”

他径直游到一颗夜明珠边上,用手里的家伙敲下来一颗,塞进自己内裤里,然后游回来,说道:“顺点东西回去赔偿我的精神损失,图个彩头。”

我几乎想掐死他,不够这个时候我没话来骂他,也没时间骂了,一下子水已经没到我的眼睛下面,我把鼻子翘上去,贪婪的呼吸这最后几口空气,几秒后,耳朵一凉,整个人已经浸入了水中。

我给胖子做了个手势,他水性最差,我让他第一个上去,他摇摇头,示意他自己太胖了,万一卡在洞里,大家一起死,我点点头,先第一个游进了那个破洞,那洞下面大,上面窄,我一探头,上面就是大概十七八个巴掌厚的海沙,最顶上松散的那些不停的塌下来,一片白雾,我眼睛都睁不开,只好几个大力的蹬踏,一下子漂了上去。

时间算的非常好,那个时候海水非常的浅,不过我也已经到了憋气的极限,几乎是手忙脚乱的游了上去,一出水就几乎晕厥了,马上大力的吸了一口气,狂喘起来。

过了几秒,胖子和闷油瓶几乎同时也探出了水,胖子一出水就呛了鼻子,边咳嗽边大笑:“我操!真没想过真的成功了,我王胖子终于出来了!哈哈!”

我定了定神,看了一下四周,这个时候已经是夕阳晚照,海平线上的火烧云倒影在海水里,分外的妖娆,太阳是深红色,发出昏黄的光芒,把一切裹在一团柔和里,形成一幅非常瑰丽安详的景象。

我一路过来,也看过几个日落,但是从来没觉得像这个这么美过,不由感慨万千。不过马上我的脚就感觉到有点抽筋的迹象,我忙转头去找我们的船,发现就靠不远出的一处礁石上,心里又是一安,有船在,马上就能脱离这苦海,好好睡一觉了。

胖子回过神后,想起了什么,突然又潜下水去,我跟着他一潜,只见阿宁正卡在那个洞里,拼命的挣扎,就是出不来。

真是怪了,这女人比胖子苗条不知道多少倍,胖子都出来的这么顺利,这女的没道理会被卡住。

阿宁气已经到极限了,突然看她喉咙一紧,从嘴巴里吐出大一串气泡,开始翻白眼,我和胖子潜下去,一人拉住她一只手,就往外拽。

这一下我就发现,里面还有一股力气在把她拉下去,不过我们有两个人,力气占了上风,只一个回合,就把阿宁从那洞里拉了出来,我看到一大团头发缠在她上,马上知道刚才是怎么一回事了。

那洞里现在已经裹满了黑色的头发,看样子等一下禁婆很可能会爬出来,最好不要呆在水里了,我们浮上水面,胖子探了探她的呼吸,发现她全身软绵绵的,好像脱力了一样,但是呼吸倒是还有,我们三个游回到船边上,把那女人拉了上去,看她不停的在吐水,眼睛直翻白,好像情况比较不妙。

我对溺水没什么了解,忙大叫:“船老大!有人呛着水了!快出来救人!”

喊了两声,竟然一点反映都没有,我奇怪起来,先让胖子看着,自己走进船仓找了一圈,不由纳闷,竟然一个人都没有。我心理陡然出现一股异样的感觉,不可能啊,这里是远海,怎么可能整船人都没了,如果去游泳,至少应该留几个看船啊。

我又大叫了几声,还是没反应,倒是胖子应我了,他跑进来,问干什么,我指给他一看,说道:“有情况,船上没人!”

胖子一楞,也找了一圈,挠了挠头,说道:“真没人,可鱼仓里的鱼还是活的,说明他们半个小时前还在打渔,就这么点时间,人到哪里去了?”

\chapter{总结}

我检查了一下方向舵边上的仪器,看上去都很正常,说道:“这船挺正常的,不像是出了什么事故……你说,可不可能是给海防的逮到了,一船人都给办回去了?”胖子摇头说不对:“人走了,船肯定也得拖走,丢在这里算什么事?绝对不会是海防的关系。这一带乱,有很多乱七把遭的船,我们去货仓看看,要是东西都没了,那就是遇上海盗了。”

我知道海盗的事情,来的时候船老大和我说过不少,心里总感觉这东西不太真实,胖子说起来,我还有点惊讶,问它道:“这地方说是近海不近,但是说是远海也不远啊,海盗能猖獗到这份上?”

胖子笑我幼稚:“多新鲜啊,你真当人民解放军是万能的?老虎也有打瞌睡的时候,我告诉你,这片海,越南人也有,日本人也有,马来西亚的也有,表面上看不出来,其实暗潮汹涌啊,私底下你知道多少毒品,走私,偷度,海盗的船,而且他们一个个手里都有枪,这里出现一艘无人船,不稀奇。”

我们走进货仓,一进去就闻到一股茶叶的味道,胖子前我后,里里外外看了一遍,物资都在,摆的和我下水前一样,甚至在我们躺过的那床板上,还放了一杯茶,我一摸,说道:“真他娘的奇怪了,还是温的。”

胖子无奈的笑笑说:“这怪事天天有,今天特别多,难不成这整船的人都给鬼叼去了?”

我说:“你看这茶才喝了几口,但是茶杯盖却盖着,说明他们走的很匆忙,但是不慌乱,在什么情况下你会走的很匆忙,但是不慌乱?”

胖子耸耸肩膀说不知道,我想了一下,也想象不出这里发生了什么事情,想着我们又走回驾驶室,胖子扯起无线电喊了几声救命,没人理他,这个时候我看到放在一边的收音机,就打开来,正听到台湾渔业电台的台风警报。

我们上来的时候已经能感觉到风大了起来,不过是黄昏的时候,看不到太远的地方,广播里说着一些术语我也听不懂,不过最后一句:“请海上船只进港避难”倒是强调了好几遍。

胖子和我的脸色都有点黑,本来这个时候,我们啥也不用管,躺着船老大自然会想办法,现在给我们把一船人都给变没了,这老天爷也真会给我们开玩笑。

胖子看了看表,说道:“看样子我们在这里呆着也不是办法,就这小破船,等一下我们都得飞到天上去。我先把船开出去,在深海碰到台风还能颠簸一下,这里都是暗礁,一起浪就肯定触礁,你去把那锚给起了。”

说着他点上个烟,啪啪开了几个仪器,动作还像摸像样的,我觉得奇怪,“你他娘的会不会开船?这事情可不是开玩笑,我们四个好不容易出来,等一下给你整个儿撞礁石上去,一起喂鱼。”

胖子朝我嘿嘿一笑,说他这叫天赋,不要说船,就飞机,给他捣鼓几下也能开到天上。

我听了不知道他是不是认真的,还是很不放心,胖子老练的拉响引擎,对我说他以前上山下乡的时候,当过什么渔队的生产组长,这一套基本的东西他还是会的,加上来的时候看那驾驶的操作过,这些高新科技的东西他都看了个大概,相信如果不遇上什么大风浪,开回去绝对没什么问题。

其实他所谓的生产组长,就是撑着个竹筏在山溪里摸鱼,不过当时我看他说的信誓旦旦,不像是在晃点我,竟然就信了,还屁颠屁颠地跑去起锚。

船开动之后,胖子让我别去烦他,说现在还在暗礁区,他得集中精力,我看他一脑门子汗,表情严肃,知道他是在说正经的,就走回甲板去。

闷油瓶正给阿宁揉手,促进她的血液循环,她看起来比刚上来的时候好了一点,但是脸色还是难看,呼吸长出短近,很不稳定。我问闷油瓶怎么样,他点点头,估计问题应该不大。

我拿出点干粮,给几个人都吃了点,经历了这么多事情,虽然现在还没有脱离险境,但是总算是回到自己熟悉的地方了,我放松下来,人就开始犯困,于是换上自己的便服,裹着个毛毯就靠驾驶室外面打起瞌睡来。

本来我只想睡个几个小时,然后就去看看胖子要不要替班,可是人不争气,醒过来的时候已经是第二天,不知道是上午还是下午。

我看了看边上的海。浪很大,零散能看到几只海鸟,都飞的很低,天是阴的,云一片一片压在一起,好像要下雨的样子,海上没什么高楼大厦挡着,乌云充实你所有的视野,人在这种景象下面,会觉得自己特别渺小,那种压迫感和城市里不能比。

我瞄了一眼驾驶室,胖子缩在一边睡觉,呼噜打的雷一样,闷油瓶正在掌舵,我刚睡醒,虽然觉得这情景不太对劲,但是也没有太在意,又转过去睡个了回笼觉,一直到中午才给胖子拍醒了。

“天真无邪同志,吃饭了,自己拿筷子。”

我睁开眼睛,看着胖子煮起个鱼头火锅,正在用筷子扳着,汤已经泛白,火候正好,我看这鱼还挺面熟,好像是船老大的那条石斑,心里一笑,这条鱼胖子垂涎了很久,不过船老大死活不让吃,说是要卖给酒店,没想到还是没逃脱胖子的黑手。

胖子忙着掰葱,放辣椒,拍鱼,看样子也是个老手,我笑道:“胖子,行啊,有两下子,这招哪里学来的?”

胖子说道:“老子上山下乡的时候,没娘没老婆,什么都得自己来,那时候在老山区里打猎捞鱼掏蜂窝,什么事情没干过,这区区一鱼汤,小意思。”

我朝他竖起大拇指,“胖哥,胖爷,我很少真心佩服人,你他娘的太厉害的,我得向你学习。”

他不吃这一套,骂说:“他娘的马屁少拍,要吃就快吃,不吃滚一边去,口水别喷进去!”

我当然不会放弃美食,马上下筷子抢肉,二十分钟不到,一条3斤石斑就被我们下肚,直吃的我直翻酸水。

吃饱了胖子就去换闷油瓶子的班,这船上有自己导航的装置,我们不会用,不然这船自己就会开。胖子吃饱了喝足了,一手扶着轮舵,一手就掏出他夜明珠直看,嘴里还哼着小曲:“竹楼里的好姑娘,光彩夺目像夜明珠啊。”

哼着哼着,他看我呆坐在那里,就把那珠子递给我,说道:“你闲着也是闲着,帮我估计个价格,看看大概能搞个多少钱?”

我接过来一掂量,说道:“假的,这不是夜明珠。”

胖子几乎没背过气去,瞪着眼睛看着我,我忙安慰他:“别激动,假的也值钱,这是鱼眼石,你知道啥叫鱼目混珠吗?就是指这个,这东西也极少见,就看有没有买主,我刚才看见的时候就知道了,你想,一个宝顶上安这么多夜明珠,你以为他汪藏海是什么人,可能吗?整个中国皇室,几百年积累下来,也就能搞这么十来颗。”

胖子听了心理舒服点,骂道:“他妈的你以后说话能不能不要只说一半,气短的能给你吓伤掉。那你给估计一下,这玩意能值多少钱?”

我还真没经手过这东西,只能推测一下我手里那几个主顾大概能出多少,我报了几个价格,胖子都不满意,说这是命拼回来,要是没好价,宁可放家里当台灯,我叹了口气,说:“那行,我上次在济南认识了一个大客,我回头给你问问,我估计换幢别墅应该问题不大,你就别想了。”胖子说:“那你可得费心,我这别墅可就指望你了,话说回来,他娘的早知道再憋几分钟再敲一颗下来,那就能换艘小飞机开开了,咱也学学美国富豪,对吧。”

我看他白日梦做到天上去了,不去理他,他把珠子放进自己兜里,问我:“这次没找着你三叔,你有啥打算?我看这事情还没完,你还得受累。”

我原本打算回去,把他那屋子翻个底朝天,看看他到底他娘的在搞什么鬼,胖子问起来,我又不能如实说,无奈的笑笑:“我还能有什么打算,回去继续开我的铺子。这斗我是绝对不敢再下来,这赚的是钱,亏的是命,不合算。”

胖子大笑,也没继续说什么。

几个小时后,我们抵达了永兴岛,岛上正在做防灾准备,避难的渔船很多,我们整理好自己的行李,趁着乱就逃了上去,船也不要了,胖子背着阿宁就先送到了岛上的军医卫生院,然后我们找了个招待所住下来,渔民一般都呆在自己的船上,有什么事情好照应,台风来了又没几个游客,这招待所基本上都空着。

我们在岛上一直呆到航班恢复,大概呆了有七天的时间,期间不止一次讨论一下这个海底墓穴,得出了不少共识。

首先我们都承认这个是汪藏海的墓穴,但是打坐在石盘上的金身是不是他,都不能肯定。因为那具干尸明显给人动过手脚,汪藏海虽然古怪,但是也不至于这么丧心病狂。

第二,云顶天宫就在长白山上,至于里面葬的是谁,也不得而知道,只能推断,里面因该是一个蒙古人,而且大有可能是一个身份地位十分特殊的女人。

第三,蛇眉铜鱼出现在鲁王宫和海底墓里,六角铜铃也出现这两个地方,说明,六角铃铛和蛇眉铜鱼,可能有某种联系。鲁殇王是盗墓的,汪藏海是做工程的,他们两个的唯一的共通点就是经常要挖土,他们是不是都在某一个地方挖到什么,也是未知数。

第四,是闷油瓶提出的,他画了一张草图给我们,把我们在古墓里的行动路线画了出来,大概勾画了一个古墓的结构,然后他指着几个地方,这些区域是夹在顶室(我们破口的地方)和底下的墓室之间的,这里应该还有几个房间,闷油瓶估计,这个墓室的结构,和战国皇陵有点像,那这几个悬空的房间,其中一个应该是珍禽异兽坑,那些稀奇古怪的东西,说不顶就是这里来的。

我听了冒白毛汗,问他:“你是说这汪藏海逮着旱魃和禁婆当宠物?这也他娘的太牛皮了吧?”

闷油瓶子点点头,说:“他不是第一个,商周几个皇陵,始皇陵里都有。特别是汪藏海好这个,他这样做,无可厚非。”

我闲暇的时候,不时拿出手提电脑,拨号着上网,想查查汪藏海的资料,可是网上少的可怜,只知道澳门是他设计的,还是copy另外一座城的样子。接下来几天无聊到死,风大的根本出不了门,第四天的时候电话线都断了,我们只好跟胖子锄大D,闷油瓶不好这个,整天就靠在床看天花板,一看就是一天,我也拿他没办法。

胖子背上的那些白毛,后来没去管他,竟然莫名其妙的好了,我怀疑还真是我的口水管用,感觉滋味怪怪的,但是这些事情我也不想深究,后来也就忘了,其实这个时候,我应该感觉到不对劲,无奈性格生死在这里,得过且过,活该我要经历这一劫数。

这几天我也试探着问了闷油瓶的身世,但是他都好像没听见,这人装傻的本领,可能比起阿宁来还要略胜一畴。

第五天的时候,电话线又通了,我又继续上网,这个时候我脑子想着张起灵的身世,突然有了个灵感,既然张起灵可以恢复记忆,那其他的人如果和他的经历一样,说不定也有人恢复了记忆,想着我就鬼使神差的把他的名字打进去搜索,一搜索不得了,全是同名同姓的记录,我随便点了几个,发现都不是用有的信息。

这样找不是办法,我又把三叔的名字也加了进去,这一下子,就只剩下了一条信息,看标题,是一则寻人启示。

这个发现在我的意料之外,我一下子感觉到有点窒息起来,点开一看,竟然就是那张他们出发前在码头拍的合影,被人扫描了上去,下面还列出了所有人的名字,我一路看下去,发现最后还写了一句话。

这句话才短短的几个字,却把我的思绪全部都吸引了过去。

“鱼在我这里。”

(《怒海潜沙篇》完)

◆ 第三卷 秦岭神树 ◆

\chapter{老痒出狱}

这句话才短短的几个字,却把我的思绪全部都吸引了过去。

“鱼在我这里……”

什么鱼?难道是蛇眉铜鱼?

从古墓石刻上图案来看,这种奇怪的铜鱼应该是三条首尾衔接在一起,现在我手里有两条,确实应该还有一条和我手里的配成一套。这句莫名其妙的话的意思,会不会是想暗示,那最后一条鱼在他手里?

这条信息的发布者,他既然有这张照片,又知道鱼的事情,会不会当年失踪人里的其中之一?

我仔细翻了一遍这张网页,看发布的时间,应该是在两年以前,亏的这个网站没有倒闭,不然这条信息肯定早就消失在互连网上。信息除了这一句话外,没有任何署名和联系方式。

我感觉到一种不和谐,既然是寻人,又不留下自己的联系方式,这不白搭吗?

我变着花样在google里搜索,希望能找到更多的信息,但是搜来搜去就这么一条是和这个有关系的。

我不由沮丧,不过这已经是很大的发现了,至少可以说明,在两年前,还有人在关注二十年的事情,那么这个人到底是谁呢?

不久,这该死的风暴终于过去了,风暴过去后第二天,就有琼沙轮从文昌的清澜港过来,我们见这里待无可待,就收拾行李准备回去。

临走的时候我们去军医卫生所找阿宁,她却已经不见了,问那医生,他说几天前有一群外国人顶着风暴突然过来,将她接走了,他以为是我们一起的,而且大风刮了电话线,他们那一区的一直没修好,所以一直没通知我们。

我心里明了,必然是阿宁在岛上的接应将她带走了,这几天风暴封闭小岛,我们就是有心阻止没有办法。

胖子大骂,说便宜了她,我却不由的松了口气,本来我就不知道应该怎么处置她,不可能杀了她,又不会严刑逼供,这样的情况正中我的下怀,走就走吧,反正她也没拿我们怎么样。

只是,他们的公司进到海斗里,实在不像是去救人这么简单,他们到底有什么目的,三叔和他们之间到底发生了什么,他人现在到底在哪里?这些隐藏的秘密,不知道何时才能浮出西沙蔚蓝宁静的海面。

长话短说,我们乘坐琼沙轮回到大陆,两天之后,在海口机场,我和闷油瓶以及胖子告别,上了飞往杭州的飞机,现实中的生活总是出奇的顺利,四个小时之后,我就回到了杭州的家中。

长时间的高强度活动使我筋疲力尽,接下来的时间我蒙头睡觉,每天只起来一次,都是饿醒的,随便从冰箱里拿了点东西吃下去又躺下。不知不觉的,过去了两个星期时间。有朋友以为我死在家里了,过来找我,我才醒悟过来,自己已经休息够了。

睡的太多,浑身难受,我先给王盟打了电话,问了问铺子里的情况,除了没什么生意之外,一切正常,其实没生意也是正常的一部份,老板不在,要是有生意就怪了,然后又打电话给三姑六婆、七姨丈,凡是和三叔有来往的亲戚,我全部问了一遍,知道不知道三叔的下落,但是都没有什么结果,我最后打到三叔铺子里,他一个伙计接了电话,我问他:“吴三爷回来过吗?”

伙计迟疑了一下,说:“三爷是没回来过,不过有一个怪人说是你的兄弟,非要我们告诉他你在什么地方,我不知道他什么来路,不过看他滑头滑脑的,不像是个好东西,就给你打发了,他临走的时候留了个电话号码,你要不打过去看看?”

我呆了一下,心里觉得奇怪的,我各方面的点头朋友很多,但是能想到去三叔那边找我的,倒也数不出几个来,想了一下,问他:“那人多大年纪?”

“这我可说不准,大概和你差不多,比你老成点,板寸头,三角眼,鼻梁挺高的,架着副眼镜,戴着个耳环,看上去不中不洋,不伦不类的。”

“不伦不类?”我重复着这几句话,心说到底是谁啊,想着忽然心里一跳,问那伙计道:“那人说话是不是不太利索?”

“对,对,对……,那家伙一句话要结巴个十几次才讲完。”

我心里一乐,已经知道对方是什么人了,忙把电话号码要了过来,随即打了过去。不一会儿电话便接通了,里面传来了一个既熟悉又陌生的声音,“谁——谁——谁啊?(结巴)”

我呵呵一笑,说道:“我操你的蛋,连我的声音都听不出来啦?”

他愣了一下,发出几声兴奋的声音,大叫:“三——三——三年没听你说话了,当然听——听不出来了,你看你那嗓子,还真发育了。”

我不由心里发酸,直想掉眼泪,骂道:“你还有脸说我,几年一点音信也不给我,我还以为你死了呢!”

电话对面那个就是老痒,他真名叫什么我已经忘记了,我和他从小穿同一条裤子长大,什么事情都一起干,有段时间好得几乎像一个人,他家里比较穷,大学毕业后找不到工作,就到我铺子里来打工,别看他这人嘴巴不利索,特别会呼悠人,两人臭味相投,胡乱经营,日子过的倒也逍遥自在。

不料三年前,这小子不学好,跟着一江西老表去秦岭那边倒斗玩儿,结果被逮住了,那老表就被直接判无期,他靠一张嘴呼悠来呼悠去,把自己呼悠成一个受到社会不良势力蒙骗的大好青年,结果就捞了三年有期徒刑。刚开始一段时间,我还想去见他,可是这小子死要面子,就是不肯见我。后来我搬了家,就这么断了联系,没想到他现在竟然出狱了。

说起来他会去倒斗,我也有很大的关系,我自小就在他面前吹嘘着爷爷如何如何厉害,还拿着爷爷的宝贝在他面前炫耀,估计那时他就动了倒斗的歪脑筋了,这小子胆子贼大,小时候我出主意他闯祸,只是没想到,这掉脑袋的事情,他竟然也敢付诛行动了。

我和他有三年的话要讲,一打开话匣子就关不住了!直说到嘴巴抽筋,手机发烫还不过瘾,我说的兴起,对他说道:“你他娘的晚上没事吧,哥们我为你接风,咱们去搓一顿,喝个痛快。”

老痒也正说得兴起,回道:“那——那敢情好,老子三年没吃过大块肉,这次要吃个爽!”

这事就这样拍板了,我也兴奋得睡不着觉,胡乱洗了个澡,把家里收拾了一番,就去约定的酒店等那小子,把菜单上所有大块肉的菜都点了一份,傍晚时分不到,那小子就来了,我一看,哟呵,这小子不正常,蹲了三年生牢大狱,竟然还肥了。

我们二个老友见面,二话不说,先干掉了半瓶五粮液,回忆以前的生活,看看现在的情况,都不由唏嘘,直喝到酒足饭饱,桌面上盘子底朝天,才发现已经说得无话可说了。

我那时候酒也喝多了,脑子犯混,就说起了他当年犯事的事儿,打着饱嗝问他:“你实话告诉我,你当年到底他娘的倒到什么东西?你那江西老表竟然还被判了个无期?”

话一出我就后悔了,心说我提这事情干什么,等一下勾起他的伤心事情,我还不好圆场子。

没想到他一听我问,竟然面露得意之色,扣着牙,说:“我倒出来的东西,嘿嘿,邪门的很,不是——是我不告诉你,就算我告诉你了,你也不知道。”

我看他看不起我,大怒:“你拉倒吧,老子可不是三年前的毛头小子了,唐宋元明清,只要你能说出形状来,我就能知道是啥东西。”

老痒看我一本正经的,笑道:“就——就你那熊样,你还唐宋元明清!”说着他就要用筷子蘸着酒,在桌子上画了个奇怪的形状,“你——你见过这东西没?”

我醉眼朦胧,看了几眼也看不清楚,只觉得像一棵树,又像一根柱子,骂道:“你个驴蛋,蹲了三年窑子,画画一点也没长进,你画的这个叫啥?整个一棒槌!”

老痒说道:“你——你——你就凑和着看吧!就你那——那眼神,也就只配看这种画!”

我仔细看了一下,实在是画的不知所云,对他说:“鬼知道你画的是什么,你看这几个分叉,你的意思是花纹吧,画得和树叉似的,这画太次,我看不出来!”

老痒得意的一笑,压低着声音,很神秘的对我说:“你还别——别说,这就是树叉,手腕粗细的青铜树叉!?”

我一听“哟喝”,这家伙原来还倒了个青铜器出来,这真是不要命了,给他判了个三年还真是算已经赚了,对他道:“这东西得多重呀,你小件的东西不倒,倒个宠然大物,这不找逮吗?”

他拍了拍我的肩膀,剥了一个葱爆芋艿,丢到嘴里说道:“你不了解当时的情况,那地方和你想的不同,说起来就话长了。”

我对青铜器略有研究,琢磨着他画的那个东西,想起前不久在三星堆挖出来的那几棵青铜森神树,还真有点像。

三星堆是古蜀的遗迹,严格说来已经不算是我们古董买卖能涉及的范畴了,年代太远,过于珍贵,价格开多少都不算高,要是老痒去的地方有这东西,那也不知道该说他是走运还是倒霉。

我一下子对这东西发生了兴趣的,我就问他当时经过是怎么样的,他喝多了,也没想过隐瞒,一五一十就说了出来。

他们那时候,进秦岭已经走了十几天,除了满眼的原始森林,什么也没找到,几乎进入了弹尽粮绝的境地。

老痒和他老表其实都没有盗墓的基本常识,只是怀着满腔的热情,此时他老表已经心灰意冷,打了退堂鼓,老痒一直坚持着,才没有马上折反回去。

这一天,他们跋涉到了一个隐藏在崇山峻岭之中的山谷,这样的山谷这几天他们不知道见过多少了,不过这一次,老痒却发现这里有点不同。

这里的地理环境非常奇特,海拔很低,温度很高,在山谷的中心,有一片地域广阔老榕树林海,哇,那林子,也不知道里面有多少棵十人无法环抱的榕树,遮天避日,榕树根爬满了地面,几乎没空隙可走。

老痒的老表一看这情景,就觉得不太对劲,榕树林能长成这样的规模,不像是自然形成的。

地仙里有句老话,叫“咸地不长篙,日上九八桥,秃山不冒林,必有沙泥淘”,就是说,草和树生长的不正常的地方,地底下或者四周就可能有问题,也许会有古墓。

榕树根系如蛇,互相缠绕,林子比一般的树林要密集很多,进入恐怕会吃点苦头,但是想想这一次来吃了这么苦头什么也没捞着,他老表心里也不舒服,心一横,就带着老痒走了进去。

他们一直往里走,直走到夕阳西下,才慢慢靠近林海了的腹地,这里四周夜枭的叫声此起彼伏,光线极度的昏暗,他们打起手电,放慢前进的速度,以免迷路。

就在这个时候,他老表给什么东西拌了一下,差点摔倒。老痒忙扶住他,转过身一看,原来是脚下的榕树根包里,裹着什么东西,高出了地面一块。

他们用短斧砍掉那榕树根包的几根根须,把里面的东西暴露出来,用手电一照,原来是一个的长满青苔的石头人,看服饰似乎是两汉以前的风格,浮雕着十分精美的图腾图案。

这个石头人的出现,让老痒他们马上意识道,这个林子确实存在着什么东西。老古话说的果然没错……

他们在石头人的四周四处查看,很快,他们便发现这里的榕树林地表的落叶泥下面,埋着很多大型的石板,似乎是一条古道的遗迹,那石人就位于在古石道遗迹的一边,似乎是这条石道的守护俑。

这样的格局,会不会是皇陵的神道?老痒想:还在外面几十里外那小村子的时候,有老人说这里的山里埋了好几个西晋候,难不成辛苦了这么多天,真给他们碰上了?

要是真的,那这几天受的苦可真值得了。

他和他老表两个人商量一下,决定先顺着古道找找看,如果附近有古墓,必然还有什么痕迹。

他们顺着古道跋涉,又走了好几个小时,进入了林海的中心地带,在石道的两边,他们又发现了不少石人的遗迹,有的横倒在石道上,有个给裹进了树的内部,都长满了青苔,神道的痕迹,越来越明显。

老痒他们暗自兴奋,加快了脚步,可奇怪的是,越往顺着古道前进,四周气生根却越走越密集。到了最后,老痒他们不得不将根须砍断,才能勉强通过,似乎这里的树木,不希望有陌生人走这一条道路。

这样一直走到了后半夜,筋疲力尽之下,前面的树缝中才出现了月光,老痒感觉可能石道的尽头到了,他们翻过大堆的乱石头,砍断了最后一根气生根,从榕树林里钻了出来。

一下子,月光下,一个巨大的向下凹陷的倒金字塔形的石坑出现在了他们的视野里,足有一个足球场这么大,形状就像一个巨大的斗,扣在森林的中间,坑四边的坡面给修成了阶梯,足有一百来阶,通向坑的底部。

老痒当时看的几乎傻了,他从来没想到石道的尽头,竟然是这么壮观的古建筑遗迹,只觉得心跳加速度,几乎双腿发软想跪下来。给这个坑磕头。

但很显然这里并不是古墓,那这里是什么地方,又是哪一个朝代遗留下来的?

老痒的老表颇有一些道行,看到这情形,也是十分的震惊,对老痒说道:“这里肯定是和一种祭祀仪式有关,看上去是个祭坛,我们快下去看看,祭祀坑有没有什么冥器。”

这时候天上已经起了白霉月,光线非常晦涩,他们打起手电以免给蛇一样的根须绊倒,忐忑不安的顺着石阶向下,来到坑底。

这整个坑四周都给四周榕树的气生根掩藏住了,如果不是跟着古道,就算在边上走过也找不到这里。而坑里面的石板也几乎都裂成拼图玩具,大量的根须从石头里挤出来,又插进边上的缝隙里去,整个遗迹已经给破坏的面目全非。

坑底也覆盖上了厚厚的一层杂草,只有少数地方,才有露出下面青色石板的痕迹。

杂草都有半人高,他们用砍刀一边砍着一边前进,不久便来到了祭坛的中心。

祭坛的中心有一个被一圈石头围起来的土井,土井大概有十多米深,手电照下去,底下也全是草。他们用绳索下到井底,先是四处找了找,见没有什么东西,就直接打下洛阳铲子。

第一铲打到了十五米,没有见底,老痒拔了出来,拍碎泥块,发现带出的泥里面混着碳灰,好象焚烧过大量的东西,而碳灰里面,他们还发现了几粒陶器和玉片的碎片。

腐泥里的碳土是焚烧祭品时候的遗迹,而这些烧剩下的陶器和玉片,都是当时的祭品。看来这个土井是当年祭祀死者的时候焚烧祭品的地方,而且还不止一次的使用过。

老痒这时候已经按奈不住自己的兴奋了,在历史上,在祭祀的时候,往往会焚烧大量的精美青铜器和玉器,如果能挖出来一两个,他们真是发财了。

他们开始用铲子挖掘起来,轮流开工,不知疲倦,不一会儿,就在坑底挖下去大概七米,大量的玉器和陶器的碎片给挖了出来,连数都数不清楚,什么玉片,玉饼,陶罐子,陶壶,几乎什么都有,很快,一边就堆了一堆这种东西。

可惜的是,大部分的玉器和陶器都是破损的,这在市面上价值不大,这让老痒他们很失望,而最失望的,是没有他们想要的青铜器。

他们不死心,继续挖着,很快挖到了十米的深度,还是没挖出什么好东西,而直土坑挖到十米以上一点就已经是极限了,再挖,就得考虑到盗洞的坍塌问题,他们不得不停了下来。

他老表还是比较谨慎,说挖了这么久都没东西,恐怕这祭坛祭祀的时候没有用青铜的祭器,别挖了,拣连破烂回去也能回本了,算我们倒霉。

可是老痒不甘心,不管他老表怎么说,他还是要继续开挖,他让他老表上去,自己一个人又挖了大概两个小时,一直挖到十四米多,忽然当的一声,他的铲碰到一块金属的东西。

老痒和他的老表互相对视了一眼,俯下身去一看,土坑的中心部分,出现了一个暗绿色的突起。

果然有青铜器,老痒心里咯噔了一声,手都颤抖了起来。他老表欢呼了一声,仍掉铲子就跳进坑里,两个人开始用手去挖这个突起。

很快,一个奇怪的东西便出现在了他们眼前,那是一根青铜的棍子,但是具体是什么感觉不出来。他们拨掉表面的碳土的时候,一根精致的青铜铸造的树枝出现在了他们面前。

他们两个大喜过望,从来没见过这东西啊,那肯定值老钱了,忙撒开膀子想把这东西挖出来,他们用手向下挖了几公尺,没有见到底,拔了拔不出来,就用铲子挖,一路挖下去,只挖到又是六七米,那青铜树枝还是没有见到底的样子。

老痒开始觉得奇怪起来,做古董的经历告诉他,很少有超过三米高的青铜器,但是眼前的这东西,按照保守估计,最起码也得有二十米高,这太不寻常了,这泥下面,到底还埋了多少。

盗洞已经将近二十米深,再挖肯定得塌。但是空手回去实在是让人不爽,两个人一头雾水,呆在那里,不知道怎么办好。

最后,还是他老表有办法,他在青铜枝桠的底部,大概一米外的地方,对着青铜枝桠的方向斜着敲进了一只洛阳铲头,然后一直加上罗纹钢管斜着打下去,一直敲下去到十米左右,钢管的敲打声一下子变的沉闷,再也敲不下去了。

老痒说到这里,表情都有点不自然,点上一烟狠狠吸了口气,说道:“那就是说,最起码那青铜枝桠在泥下面的部分还有十米左右的长度,那就是总长最起码是三十米,这么大的东西,就算挖出来也带不回去了。”

我听了咋舌,觉得他说的有点夸张,河南安阳侯家庄武官村出土的司母戊鼎,是我国现存最大的青铜器,也只有一米多高,当时要铸造这样大的东西,已经需要将近两三百人同时协作了,要铸造三十多米高的青铜树,启不是要上万人才行?

但是看他说的这么多,也不好去反驳他,问道:“那后来怎么样?有没有继续挖下去?”

老痒道:“没有,我是想挖的,我那老表却突然说,这东西可能是神物,说不定真的是从地里长出来,不能挖了,后来我一想,再挖也太不保险了,就放弃了——你说怪不怪?我估计这树叉还是一大青铜器的一部分,下面的东西,可能更大,要全刨出来,恐怕得震惊世界。”

我奇怪道:“那就是说你没把那青铜树搬出来啊,你是怎么被逮到的?”

他说:“这事情我说起来就觉得怪,我们当时候不甘心,又在其他地方刨了几个坑,总算挖出来点完整的锅碗瓢盆,出了秦岭之后,想找个地方销脏,但是我那老表,自从见了那东西后就神经兮兮的,一到城里,他见人就说那铜树枝桠的事情,秦岭那地方自古对盗墓就生恶痛绝,风声一直很紧,我们上一古玩店去出货的时候,有几个人听我老表乱说,看出了我们的身份,就把我们给举报了!幸亏逮我那公安和咱们是老乡,一看我还年轻,就让我咬着说‘被人骗了’才勉强判了三年,我那老表本来也就四五年,没想到他疯了一样,把以前倒斗的事全部抖了出来,就给判了个无期,差点就毙了。”

我“哦”了一声:“那你真是背到家了,忙活这么久啥也没捞着,我告诉你多少次了,不要就地销脏,你干的是外八行的买卖,跟当地人犯冲,这叫现世报应。”

老痒神秘的一笑,说:“我——我也不算是啥也没捞——捞着,你看这东西——丁?”说着就指了指他的耳环!

\chapter{六角铃铛}

我凑过去一看,眼睛就再也移不开了,一把楸住他的耳朵,把他拎到面前仔细来瞧,一看之下不由倒吸一口冷气,那耳环四四方方,只有小拇指尖的大小,别人看了兴许还以为是路边摊上一块钱两对的便宜货,但是我仔细一看就发现,这其实是一只六角铃铛。

无论外形,颜色,除了小一点以外,与我在尸洞和海底墓中见到的那种,很有几分相似,只是上面的花纹,似乎有一点略微的不同。

我立即酒醒了大半,问他:“这玩意你从哪里弄来的?”

他被我楸的咧起嘴巴,大怒:“你——你——你他娘的喝多了,你知道我——最讨厌别人楸我耳朵,你再——再楸我就和你急!”

我一看,我喝了点酒劲还真没少使,忙放开他的耳朵。

他揉着被我楸红的耳朵,咧着嘴巴:“我靠,还真是下的去手啊你,见到好东西也不用这样嘛,哎呀我的耳朵哎。”

我已经没心思跟他扯皮了,问道:“快说,这东西是怎么回事情,哪里搞来的?”

他嘿嘿一笑,得意的说:“没见过吧,说出来嫉妒死你,这东西是我在那祭祀坑,一只粽子身上顺下来的,怎么样?你看,青中带黑,上等的青铜古器,也不同于你卖的那些西贝货。”

我越听越糊涂:“什么粽子?你不是说只挖出点锅碗瓢盆吗?怎么又多了只粽子?”

老痒以为我是嫉妒他,越发得意,说道:“那粽子给藤绳裹成个蛹一样,是我在那土坑的其他位置挖的时候挖出来的,大概是一身份比较高的人牲,这东西就戴——戴在那粽子耳朵上,我看不错就顺下来了,怎么,你这么紧张?这东——东西还有来历?值钱不值钱?”

我脑子里乱成一团,各种思绪都冒了上来,直皱眉头,心说那到底是什么地方?这种铃铛出现在这里,难道他说的那个石头坑,和我以前经历的那些事情还有关系?

老痒这时候发觉有点不对劲了,奇怪道:“干什么,脸都拧一起了,看到我倒了个好东西,也不用这样啊,你要真喜欢,我这个送给你。”

我说道:“不是,他娘的不瞒你说,你这耳环不是普通的东西,虽然它的来历我不知道,但是我却在其他地方见到过,这是这么回事情——”

我把鲁王宫和海底墓里的事和他迅速讲了一遍,着重说了那铃铛的事情,只听得他脸色一会儿白一会儿青,一脸的茫然。

半晌,他才感叹到:“我的姥姥,本来我还以为我的三年牢也够我吹一辈子了,和你一比,就啥都不是了。你干的这事逮住就得枪毙呀。”

我看他的表情竟然是无比羡慕,说道:“这有什么好比的,要是早知道倒斗是这样的事情,打死我我也不会去那几个地方。”指着他的耳朵道:“倒是你的铃铛奇怪,这种铃铛诡异的紧。只要一发声,就能蛊惑人心,怎么你戴在耳朵上却一点事都没?”

“没你说的这么邪吧,我拿下来让你瞅瞅!”说着他便把耳环摘了下来。

我拿着耳环对着灯一照,又闻了闻味道,就知道了怎么回事情,里面灌了松香,响不起来了,又翻着两面仔细的看,越看越觉得和古墓里看的那只相象。

老痒看我翻来覆去的看,以为我喜欢这东西,把耳环又戴了回去,说道:“你要真喜欢,那地方里还有不少,都是未经开发的处女粽子,地方我做了记号了,我们可以再去看看,说不准还有其他宝贝。”说着看了看四周,压低了声音,神秘道:“说实话,你兄弟我的环境实在不怎么样,这几天正打算再去干一票呢。”

我以为他是在开玩笑,回道:“拉倒,我可不想陪你去吃牢饭。你也最好别动这心。这年头,还是安稳点过日子好啦!”

老痒凑近了我一点,一本正经的轻声道:“话——话不是这么讲的,你想想,你有家里给你撑——撑着,干嘛都可以,我已经浪费三年时间了,一无所有,我不动——动歪脑筋不行呀!”

我看他表情认真起来,不像是在开玩笑,骂道:“你做梦吧,他娘的,三年窑子白蹲了,我可告诉你,出来再犯再进去可是二进宫,可是从重罚,你要是一不小心,说不定就直接被毙了。”

“要真这么倒霉,那也是没办法的事情。”老痒道,“我也是没得选择了,火烧眉毛了,才想到再走这一步,我已经想好了,先在杭州待一段时间,接着还得去秦岭,怎么样也得先倒个十几万回来,这次我来找你,也是主要为了这事情,希望兄弟你和我一起去,出货的时候提点提点我。”

我看他面有愁色,没好气道:“什么叫没得选择,你不就是缺钱,缺多少说个话,兄弟这里拿,利息按中国银行的固定打95折算给你。”

老痒推了我一把,鄙视道,“拉倒吧你,你有多少家当我还不知道,要你掏个十万,八万你还能掏出来,再多你有吗?真是,装什么阔?”

我骂道:“十万八万你还瞧不上眼,你他娘的想干啥啊?看上明星了?你小子吃饱了撑的,刚出来就这么花头,拜托你成熟一点。”

老痒不爱听这话,骂了一声,摆了摆手道:“我想干什么和你没关系,你没钱就没钱,别来教训我——算了,咱们兄弟重逢,帮不帮也无所谓,别谈这扫兴的事情。”说着就给我倒酒。

我看他看不起我,酒气上脑子,大怒:“我说老痒,你他妈的别小瞧人,这几年我也有点闲钱,你实话告诉我,你到底需要多少钱?老子立马拿来给你!”

他看了看我,酒也上来了,认真了,站起来,举起四个手指在我面前晃了晃:“这个数,你要有我给你当牛骑。”

“四十万?”我问道,倒也不多,现在四十万要说是巨款,倒也真不算什么钱,“没问题,马上去拿,我家里就有!”

没想到他摇了摇头,“再加一个零!”

“四百万?”我张大嘴巴,一下子人就凉了,“我的姥姥,我真服了你,你他娘的拿这么多想干啥去啊?”

老痒哎了一声,说道:“你别问这么多,总之我就缺这么多钱,你说你拿不拿的出吧。”

四百万不是个小数目,虽然说现在拍卖会上,随便一破瓷器就能拍到上千万,但是那是炒作居多,整个市场购买力有限度啊,从斗里挖上来的东西是整个文物倒卖的第一环节,利润本来就不高,有个十万就可以偷笑了,这四百万,我真没有。

老痒看我表情松动,知道我在给吓到了,给我满了一杯酒道:“我说你拿不出来吧?要是只四十万兄弟我还需要来找你?”

我道:“那也别下定论,我帮你去借借看,做这一行的暴富的挺多,说不定能筹到,不过你得告诉我你要这么多钱干什么?”

老痒把头转到一边,啧了一口道:“筹什么钱,你问谁去筹,你的朋友我哪个不认识的,谁能有这么多钱,而且这事情我还不能告诉你,反正有了这四百万,可以解决我一个性命悠关大问题。”

我一想倒也是,我的很多朋友都是老痒介绍给我的,真没几个能借的出钱来,问我老爷子要,那吝啬鬼说不定会杀了我,这事情还真不好办。

老痒拍了拍我,用一种很作做的语气道:“老吴,所以说咱们别谈借钱,说其他办法,最好的办法,就是你辛苦一次,陪你兄弟我过过场子,反正你也不是第一次了,你就别别扭了,这又不是啥大事情,说到底其实这不叫倒斗,咱们就去那殉葬坑里,你给我挑挑,哪些值钱,哪些不值钱,这叫做捡洋落,不犯法,你就当旅游好了,那边好山好水,山里的姑娘那身段和那啥似的,你还没搞对象吧,去那里看看,说不定还能娶个傣家族姑娘回来。”

我没心思听他胡说,摇头:“你说的容易,你那破地方,能有四百万的东西吗?你要是想一次搞这么多,你得找个两汉的,这种墓早给人挖光了,你肯定白忙一场。”

老痒耐着性子道:“哎呀,你以为我傻的,这事情都想不到,我告诉你,我这次回去,不是冲那个祭祀坑去的。上次我和我老表去那地方的时候,我老表就和我说了,有祭祀坑的附近,肯定有大型的皇族陵墓,我这一次,就是以那个为目标,你不是会风水,去看看,我觉得肯定能找到!”

我不想理他,“你找别人去,古墓我更不想去。”

老痒推了一下:“老吴,你不够兄弟啊,你想想这事情多好,一来你能帮我,二来,另一边你三叔的事情你也得要查下去啊,我这事情又和你三叔有关系,就算不为了我,为了你自己,为什么不去看看呢?”

他一提到耳环的事情,我心里又感觉不舒服起来。他这话倒是说的没错,三叔那事情,扑朔迷离,线索少的可怜,而这种铃铛,瓜子庙的尸洞和海底墓里都出现过,关系重大,要是没抓住这个机会,恐怕这事情查起来就更加的困难。

可是想前两次的经历,我的脚就开始有点发软,心里还有后怕,加上爬山的种种辛苦,实在是不想尝试。

我犹豫了几分钟,转念一想,觉得就算我不去,以我的性格,恐怕以后的日子也不太会好过,这一次老痒这样来求我,也算难得,再拒绝下去,以后不太好见面了,不如先答应下来,过去看看形势,实在不行,临时变卦也行。

但凡是我们这种人,命里有太极,对于不知道的事情,有一种极强的好奇心,给自己找到台阶下,我的心里马上塌实了。

想着我就打定了注意,对老痒说道:“那行,既然你都说成这样了,兄弟我就陪你走一趟,不过你得把这耳环先给我,我去看看,这东西到底是什么朝代的东西,到底值钱不值钱,要不值钱,说明那地方不值得去,你还得另做打算。”

老痒一听我肯帮他,马上大喜过望,忙不喋的点头,“行,你说什么是什么,送给你都行啊!”

我说道:“不过我丑话说在前头,下去之后任何事都得听我的,放屁也得先通知我一声,听到不?”

这小子早已什么都听不进了,心早已飞到秦岭去了,一边给我添酒,一边拍马屁道:“那是那是,只要能倒到四百万,你就是我的再生父母。不要说不放屁,你让我吃屁都没问题!”

我俩趁着酒劲,就把这事给拍板了。接下来又扯了一会儿女人,胡天海地,喝到半夜,都到桌子底下躺着去了。

接下来一个月,我们各自都有事情要处理,上次我们去山东买的那些东西在那边就地掩埋了,装备要重新买过,我根据这两次的经验写了张条子给他,让他去办齐了。

随后我通过关系弄了点军药过来,去山东的时候,水壶的重量实在太重,消耗了太多无谓的体力,秦岭之中山溪众多,不需要带太多的水,但是很有必要准备一些治疗腹泻的药品。我们这些城市里的肠胃,肯定适应不了大山里的天然溪水。

嘱咐完我就先飞到济南,到英雄山找老海,把胖子那颗鱼眼石给老海看。

老海看了之后乐得嘴巴都合不拢,笑道:“这位爷,我这是卖古董的,你这东西应该拿到珠宝店去,让他们给你估价。”

我说:“这鱼眼石也是古董呀,少说也有四五百年了。”

他笑笑:“我也知道,您拿出来的东西肯定是好货,这珠子要是镶在钗上,或者镶在帽子上那就是宝贝了。可你就这么光溜溜一颗,让我怎么整?你说是古董人家也不大相信呀。要不这样吧,我去给你搞支玉钗来,咱们把这球子给镶上去,看看能不能卖?我先给你点订金,你把东西放我这,识货的人自然会出好价钱。”

他说的诚恳,我也没时间去和他折腾这事情,只好依他,拿了他二十五万订金,灰溜溜的回到杭州。接下来拿着老痒给我的那耳环,去找我爷爷的一个朋友,请教他这铃铛耳环到底是什么来路的,到底值得不值得我长途跋涉去陕西受罪。

那老爷子姓齐,是杭州第一代古董商人,现在算是一个国学大师,在好几个大学都有客坐的头衔,特别是对少数民族,有相当的研究,我将那铃铛呈现过去的时候,我明显发现他的眼神直了,接那铃铛的手都抖了。

齐老爷子把铃铛拿过去后,整整看了那铃铛三个小时,翻了六七本砖头一样的书,才抬起头来,我在边上都等要的要睡着了,他看了看我,叹了口气道:“惭愧惭愧,老头子我搞少数民族这么久时间,还从来没有见过这样的东西,小邪,你告诉你阿公,这东西是哪里弄来的?”

长辈面前,我也不敢敷衍,就调着重点胡乱编了个故事说了,看他听的两眼放光,我感觉事情似乎不简单,问他道:“阿公,怎么,这东西有啥问题吗?”

老爷子又叹了口气,说按照他的分析,这铃铛的工艺,可以追述到夏朝到西周之间,上面的纹路,叫做双身人面纹蛇,极有可能来自是古时候陕西到湖北之间生活的一个叫做“厍国”古国,这个国家在二千年前,突然间消失了。

这个国家的历史时断时显,零星出现于不少古简之中,西周早期似乎有过一段时间的突然繁盛,然后西周中期,就突然消声灭迹了,似乎是在十年到二十年的时间里,迅速的消失在原始丛林里了。

在很多神话传说中都有他们的存在,山海经里也有大量的篇幅记载,其中提到的川外“蛇国”,应该就是这个国家,厍是蛇的偕音,这个民族把一种人面两个身体的蛇当作神灵,所以很多装饰上,都有双身蛇的纹路。

现在研究这个国家历史的人,大部分认为,这“厍国”是神秘“华胥古国”分裂出来的后裔,其前身要推到母系社会的时候,这个国家以双身人面蛇为图腾,主要是因为“华胥古国”有“伏羲人面蛇身”的传说。

因为这些资料都是来自古籍和出土的文书,所以这个国家是不是真的存在,学界一直都有争议。这是铃铛,放到古玩市场可能没人识货,但是对于一些专门研究这门学问的人,是无价之宝。

我一听到这东西这么冷门,心里就咯噔了一声,如果是这样,即使我们能找到古墓把东西带出来,恐怕价格也卖不高,那这一次恐怕还是白去。

齐老爷子看我的表情,就问我有什么问题,我知道他是老商人了,就把我的处境和他一说。

老头子想了想,先是说了我一通不是,然后又拍了拍我的肩膀,表示如果我想卖这东西,他可以帮我找到很好的买家,四百万绝对不是问题。但是,找到这件事情绝对不能说出去。

听了老爷子的话,我心里已经明白了个大概,妈的这老家伙看来也是暗潮汹涌,私底下估计还在做那些解放前的勾当,不过有他牵线搭桥,我是非常放心,忙点头道谢。

从老爷子那里出来,临走还拿了不少厍国的资料,我在出租车上翻了翻,看到了有很多壁画的照片,其中有一些画很奇怪,花的是大量人跪拜在一棵树前面祈祷的画面,傍边有几个注释,好象是说,厍国最重要的祭祀活动,是祭祀一种“蛇神树”,传说这种树只要奉献鲜血,就能够满足的任何要求,是一种愿望树。

这棵树的形状,于老痒给我画的很像,难道他挖出的那棵青铜树,就是这种蛇神树的图腾?

很多壁画里都有人面蛇的花纹,显然是厍国最主要的特色,瓜子庙尸洞和海底墓穴里发现的那种铃铛,当时上面有没有双身人面蛇的花纹我已经记不得了,但是看外形,这三个地方的铃铛肯定出自同一个来源,那这神秘的厍国可能是关键所在。

两天后,开往西安的长途卧铺汽车上,我和老痒并排两张床,一边嗑瓜子,一边聊天。

本来我打算直接坐飞机飞到西安再说,可我没三叔那么大的面子,一大包违禁品卡在安检口子上,只好换坐汽车,而且只能坐私人承包的大巴。

为了省过境费,这车一会儿上高速,一会儿下高速,在山沟沟里转来转去,无聊的紧,我就和老痒瞎侃,说那地方可能有个汉墓,这地方可能有个唐陵,说的老痒恨不得中途下车去挖。

聊着,老痒问我除了去他三年前到的那个坑里看看,还要不要去其他地方,到底进山不容易,要能带多点出来,就别浪费,要是能找到附近可能存在的其他陵墓,那是更好不过。

我其实早有这个打算,那一带附近可能是古代蛇国的范围,除了那个殉葬坑和附近的古墓,应该还有其他的遗迹,如果能找到一二,拿点东西出来,对于我要查的事情是很有帮助的。我心里这么打算,但是嘴上没说出来,对他开玩笑道:“别贪心,你他娘的回去的路记得不记得都不知道呢,要是找不到那殉葬坑,我看你怎么办。”

老痒朝我贼笑,说他早就留下了记号,我大笑:“三年了,在那种深山老岭里,什么记号能保存三年?”

他哈哈大笑起来,说:“你就瞧好吧,我那记号别说三年,三十年都还管用。”

我不知道他搞什么花样,懒得理他,又聊了一会儿,晕晕沉沉的,就睡了过去。

到了西安后,我们找了个小招待所过了一夜,吃了当地的酸菜炒米和芙蓉汤,顺便逛了逛夜市,直逛到十二点多,老痒惦记着炒米的味道又嚷着要去吃夜排档,我们就在路边随便找了家排档坐了下来,点了二瓶啤酒,边喝边吃,这时候也没忌讳,心说我们这一口南方话这边的人也听不懂,就聊起明天倒斗的事情。聊着聊着,就听边上一老头说道:“两位,想去啊答做土货买卖勒?”

\chapter{跟踪}

我们正聊得起劲,他这句话没头没尾,口音又重,我们根本听不懂,老痒“啊”了一声,问道:“啊答是什么地方?”

那老头子看我们听不懂,便换了口音很重的普通话问我们:“俺的意思是两位想去啥地方做买卖?是不是来挖土货的?”

我不知道什么叫土货,而且在南方人情冷漠,除了推销的,很少有人会在路边摊和人随便搭腔。一时不知道怎么反应,幸好老痒反应快,学着那老头子的腔调说道:“俺——俺们是来旅游的,对土特产不感兴趣。你——你老爷子是卖土货的?”

那老头子哈哈一笑,对我们摆摆手就走回到自己的坐位上去,我们两人莫名其妙,就听老头子对他几个同桌轻声说道:“没事没事,俩个刚上冈冈的青头,哈也不懂,不用搭理。”

老痒一听,脸色略微一变,就轻声招呼我走,我觉得奇怪,但看他神情紧张,就丢下十块钱,和他离开这个路边摊子。直走到一个转弯处,我就问老痒:“干啥要走?酒才喝到一半呢?”

老痒鬼鬼祟祟的往后看了一眼,说道:“那——那老头子,刚才他对同桌说我俩是上冈冈的青——青头,我在牢里听那几个走江湖的人说过,上冈冈就是这里盗墓的黑话,这青头就是指我们不是道上的人,这一班人一身子土腥子味,恐怕也是来跑地仙的,刚才听到我们说倒斗的事情,才过来打探。”

我笑说:“那也不至于要走呀,兵来将挡,水来土淹,这大庭广众之下,他们能拿我们怎么样?”

老痒拍拍我,说我不懂,这黑道上的事情说不清楚,刚才我们说的那些话估计已经全部被听过去了,也不知道哪些人能听懂多少,现在好墓可遇不可求,要是给他们盯上了,夜长梦多。

我知道他在牢里恐怕听些狱友添油加醋的说了不少事情,也不去和他强辩。点点头就回招待所去了。

第二天,我们不到七点就起来了,每人负重十五公斤的装备和干粮,向中国最大的龙脉进发。

我之前来过秦岭几次,每次来都是给导游提溜着转,从来不知道这路该怎么走。所以这次还得跟着老痒,他三年前过来地时候也是跟在旅行团里,旅行团怎么走他这次也得怎么走,不然就认不到路了。

我们经西宝高速大约三个小时的车程到达陕西宝鸡的常羊山。然后又转向嘉陵江的源头。

我平时走逛了直来直去的路,这盘山公路五秒一小转,十秒一大转,我脑袋顶在前面的坐位上,只觉得五脏六腑翻腾,老痒更是不济,他三年没坐过车了,这一路上已经晕得够呛了,这一次更是了不得,胆汗都要吐出来了,直说:“老了,老了,人老了不中用了,三——三年前走这条路的时候还能跟边上的娘们扯皮,没想到这次连眼皮都睁——睁不开了。”

我骂道:“你他娘的废话别这么多,放着高速路不走,你非要走羊肠盘山道,现在后悔有个屁用。”

老痒朝我摆摆手,叫我别和他说话,他难受着呢。

这个时候,突然间听到一声爆炸声从远处传来,震得车窗玻璃翁翁作响,全车一阵骚动,我往窗外一看,只见对面山上漫起满天的尘烟,老痒吓了一大跳,问我:“咋——咋回事?地——地震啦!”

前面一个当地人样子的中年人回过头来,笑道:“两位外地来的,这都不知道?那是有人在炸墓,这季节,一天里总有两三炮。”

我奇道:“这光天化日之下,这盗墓的胆子这么大?”他咧开嘴笑露出满口黄牙,“对面那山和这山可不一样,他别看中间只隔着一条嘉陵江,我们这边还有盘山道,那边可是连走路的地方都没。你就算现在报警,警察赶到那边最起码要一天一夜,除非你能长翅膀飞过去,不然就只能干瞪眼。”

我点点头,咋舌道:“还有这种事情?”

那人看了看爆炸的地方,笑道,“这也算咱们这地方的特色,特别是现在这个季节,前两天还逮住一拨呢,现在古墓也越来越少了,没几年好折腾了,深山里头可能还有点,不过路太难走了,政府也只能听之任之。不过看刚才这一动静,怕是炸药放太多了。”

我“哦”了一声,转头看向窗外,这里应该是秦岭无数支脉中的一支,只见一片莽莽森林,成片的茂密树冠之下所发生的情景根本无法窥得。

出来之前,我查过资料,陕西境内的秦岭呈峰腰状分布,东、西两翼各分出数支山脉。山岭与盆地相间排列,有许多深切山岭的河流。八百里秦川自古以来就是有名的文物古迹荟萃之地,特别是北坡有着许多帝王陵墓群,其他达官贵人、富豪巨绅的墓葬就更加不计其数,所以这里永远是盗墓贼蜂拥而至的地方,只是想不到还没进秦岭深处,就有盗墓贼在这里明目张胆的炸墓,看样子现在要找到一两个值得倒的墓绝对不是这么容易的事了。

那本地人挺热情,话题一打开,就不想收,递过来一根烟问我道:“你们两个娃娃是来旅游的吧?想到哪个地方去啊?”

我说道:“想到太白山里去看看。”他点点头,说道:“你们不跟着旅行团可走不远,这山里面七拐八拐的,弄不好就会迷路,要不要俺给我们带一段路?俺就住在保护区边上的一个村里面,翻过两个山头就到,你看这出来玩的,找个导游也是必要的嘛。”

我一听,敢情这家伙还是个黑导游,这大山里面民风彪悍,可别把我带到山沟里捅了,忙摇头道:“不用不用,我们自己有安排。”

那人道:“你先别摇头,这里不比其他地方,这里山多林子密,你们要自己贸贸然进深山里面去,很容易出危险,您可得好好考虑考虑,这一带做向导的,我也算小有名气,绝对不会吓唬你们的。”

我看他说的也算诚恳,也不好马上推辞,就告诉他这次来主要是想去山里的少数民族村子里,计划先在山下呆几天,所以也不急着需要向导,等过几天真要动身进山了,再去找他。

那人马上道:“那赶早不如赶巧,这条线我带的最多,您要到最近的一个瑶族村子,也得翻过这座山。”说着他指了指远处一条连绵不段的山脉,“这叫做蛇头山,最高的地方有海拔一千多米,整个山像蟒蛇的头,所以叫蛇头山,所有十里八乡的路客,要想去正宗的少数民族寨子里去看看,全得一步一步翻过去,这山里死的人多了,去年还有几个艺术学院的学生进去写生,到现在还没出来呢,你说要没人带行吗?”

我顺着他手指的地方看去,只见蛇头山横亘在视野尽头,山呈碧绿,山顶高耸入云,因为气候的关系,正条山脉都在云雾缭绕之中,不见真面目,只有对着嘉陵江的一面勉强可以看到,可惜临江的都是悬崖,山势非常陡峭,我看连猴子也不一定爬的上去。

这真是“云横秦岭家何在?雪拥蓝关马不前”。我看着不由暗暗咋舌头,心说要爬过这山还有命在?

车又开了个把小时,总算到了太白山脚下,我和老痒跌跌撞撞的下了车,那黑导游非得介绍旅馆给我们,我看着既然到了他的地盘,也不能老是敬酒不吹吃罚酒,就跟着他去了,他把我们带到一农家乐的小旅馆里,我一看,价钱还不贵,看样子这人倒还是真的热心。

把我们安顿好,他就拱手告辞,临走给我们留了个电话,就说什么时候进山了,就打他电话给他,他给我带进去。

农家乐的老板娘挺热情,给我们做了晚饭,我们不好意思和他们一家在客厅里吃,就和老痒回到自己房间,靠在窗台上,一边吃一边看这里的地图。

那黑导游说的没错,从这边进去,要进到秦岭原始森林的内部,需要翻过一座海拔一千多米的大山,这是我完全没有想到的,以我们现在的阅历,要自己进山,实在是等于送死一样,但是如果找那个导游带我们进去,那他势必要带我们出来,这让他等上一天两天还行,我们这一进去可能就是个把星期在山里跑,他难免不会起疑心。

老痒上次来的时候,他老表是找了一个同行的老手带路,现在他老表进去吃牢饭了,那老手自然也是无从找起,他也没想过要再来一次,对山路没什么记忆,这一次靠他也是没门。问了老板娘,也说没有其他办法,一般村寨里的人也就是有集市的时候出来一下,都是翻着山过来的,从来没听他们说过那里还有捷径。看样子要过这座山,还真有点困难。

正琢磨着怎么办,老痒拍了拍我,轻声道:“老吴,快——快看,下面那人是谁?”

我瞄了一眼窗外,只看到窗下农家院的天井里,来了五个人,我仔细一看,其中一个竟然是我们在西安路边摊上遇到的那个老头子。

我心里嘀咕,怎么这帮人也来了这里,该不成真给老痒说中了,他们也是来踩盘子的?

老痒把窗帘拉上,只留出一条缝隙,轻声对我道:“这几个家伙也是大包小包的,和我们贼像啊,该不会在西安那会儿听到了我们说话,想跟在我们后面,找机会截胡?”

我摇了摇头,看着老板娘走出来,笑着把他们迎了进去,说道:“不像,你看着这亲热程度,估计这些人经常来这里投宿,是熟客。这里客栈也不多,应该是碰巧和他们住到一会了。”要是老痒说的没错,他们也是来倒斗的,那这里应该是他们固定的落脚点,他们每次来做活,恐怕都是住在这里。

老痒担心道:“那不妙啊,他们在西安已经听过我们讲话,要是让他们在这里看到我们,难保不会打我们注意,要不连夜就撤吧?”

我想了想,觉得这非但不是麻烦,而且还是一个好机会,摇了摇头道:“不,这些人是苍蝇,无缝的蛋不落,来这里肯定有目标,我们两个啥经验也没有,与其乱闯,不如跟着他们,一来可以看看有没有洋落好捡,二来,也可以跟着他们过山。”

老痒道:“这些人都是亡命徒,杀个人不当回事儿的,跟着他们,要给他们发现了,说不定会给做掉,这样会不会太冒险了?”

我嘿嘿一笑,嘲笑道:“你小子什么时候变这么婆妈了,这里是深山老林子,那有这么容易被发现。而且我们又不是傻子,给发现了不会跑吗?你要真担心,怎么就先跟着看看,看他们警觉性怎么样,要是跟不下去了,咱们不跟就是了,也没什么损失,对吧?”

老痒听我这么说,一时间也没话反驳我,只好点头,我们马上把东西准备好,免的明天慌乱,我心里盘算着以后几天可能很不轻松,就后把闹钟调早,让老痒别搞其他事情,各自睡觉休息。

这一路过来实在是太过疲劳,一睡就睡到了中午,闹钟根本没听到,我睁眼一看太阳老大,猛的惊醒过来,赶紧跳起来把老痒叫醒,下去一问老板娘,那几个人已经走了,往蛇头山下去了,走了也不长时间。

我们两人匆匆忙忙的买了几个烧饼当干粮,一路急赶,只往山里追去。跑了大约十五分钟,总算在山脚下的景点入口追上了他们。

\chapter{继续跟踪}

那群人买了票后,直接进了景区,我们谨慎的跟了上去,远远的跟在后面。

这景区没什么人,我们怕给他们发现,只能往灌木丛里钻,皮肉遭了点委屈,被锋利的杂草和灌木刮的东一道西一道的,又疼又痒。跟了一会儿,我们已经感觉有点吃不消。

往蛇头山的山脚下,其实已经进入蛇头山的范围,这里的几个旅游点,都用石头铺了山路,走起来并不困难,山路顺着山势蜿蜒曲折,两边有山溪和很多名人的磨崖石刻,风景很美,但是这一拨人一路直奔,中途也不停留观赏,好像对秦岭的景色一点都不感兴趣。

我的体力最近不错,一路走着没什么大感觉,而老痒因为在牢里劳改,没时间做运动,心肺功能已经完全不如我,不一会儿,已经明显体力不支,开始喘大气。

山里越走越静,我们也不敢说话,闷声跟在他们后面,一直跟到天黑,月亮上到上半夜,那帮人才停了下来。

我们远远的找个灌木从蹲下,监视着他们,这时候老痒拉了拉我,我回头看他,见他脸色惨白,满头大汗,知道他坚持不住了,忙给了他口水,让他休息一下。

老痒一边喘气,一边对我说:“老——老吴,我看就这么算——算了吧,他们倒他们的,我们倒我们的,再跟下去我就要歇菜了。”

我自己也差不多了,听到他这么说,心里老大不痛快,轻声骂道:“我说他妈的,你就只蹲了三年窑子,怎么没用成这样子?现在才不跟……刚才那些罪不是都白受了?给我咬咬牙挺着。”

老痒道:“那你估计还得跟多少时间……他们停下来是不是到地方了?”

我看了看他们,说道:“不是,这里还是太浅,离过山还有很长一段距离呢,估计是走累了休息了。你看他们生了火,晚上要待在这了,我们也别浪费时间,先填饱肚子睡觉再说。”

我们也窝了下来,找了个草从,可惜这半夜里我们也不能生火,一生火就被人发现,身上衣服鞋子汗湿了也不能哄干,本来还能把干粮烤了再吃,现在只能冷冰冰的干嚼,老痒叹气,只埋怨我出的馊主意。

我也后悔,自己心里难受,但老痒那话我就不爱听,心说我来帮你还这么废话,骂他道:要是这点苦头都受不了,咱们就回去,不然再往山里头去,估计也得逃回来。

老痒郁闷了半天,突然说:“不对,老吴,我们这样被动的跟——跟踪也不是办法,也不知道他们是不是要过山,要是他们顺着山头子直接往林子里走,我们不完蛋了。”

我一听,心里咯噔了一声,心说对啊,自己想当然以为进这山的人就是要翻山过去,要是这些人真不过山,而是在附近转悠,不是给他们弄死了?

这可真难办了,又不能去问他们,我看了看前面的火光,一下子呆了。

老痒看我没主意,直叹气,想了想,说指望我算完蛋,还是靠他,他过去偷听一下那几个人说话,他们现在进山,总不会一句话也不提自己要干的事情。

我给他说的没脾气,只好同意,不过他一个人我不放心,我也跟着他摸过去。

一路走得蹑手蹑脚,不过这山里静的厉害,我们走不了多远就能听到他们说话的声音,老痒拉着我,示意躲在这里就行了,不需要再往前摸了。

我点点头,两个蹲了下来,屏住呼吸,听到他们正在那里大笑,出乎我们意料的是,里面竟然有两个人说话的声音带着浓浓的广东腔。

这真是怪了,从来没有听说过广东人也好这个。

他们在那里说说笑笑,只听有一个年轻的声音道:“泰叔,你给俺们估计估计,这还得走多少时间才能到?老子今天腿都快断了。”

一个沙哑的声音回道:“叫你平日里修生养性,你奶奶的只知道吃喝嫖赌,泡在女人堆里,这趟有你受的。俺告诉你,要过这蛇头山,这有路的还得走上两天,没路的那俺可就说不准了。你要受不住,现在就下山去吧,别再拖老子的后腿。”

那年轻人显然对泰叔有点忌讳,说道:“最近我是虚了点,您放心,这趟买卖做成了,俺们再也不用到这山沟沟里来了,俺们跟着王老板和李老板到香港去见识见识,也过过上等人的生活,对不?”

有一个广东口音的人就说了:“嗨啊嗨啊,没问题啊,我们说好的嘛,你们把东西搞定,有多少我们要多少啊,这次是一辈子的买卖,做好了大家都可以退休了。到时候香港的花花绿绿的大世界,有的是地方大把大把的花钱,这么点辛苦还是值得的嘛。”

那泰叔就说道:“李老板,你话别先说得这么满,可这斗在不在那地方,可就你一张嘴巴说的,可别给我们假消息,扑空了。”

李老板回道:“哎呀,我说你这个老泰嘛,就是心眼太多了,大家合作了这么久,我哪一次失手过嘛,实话和你们说,只要去过我们这一次要去的地方,秦始皇帝的坟墓你们也不会想去挖了。”

泰叔显然不喜欢听这种套话,冷笑道:“这话我就不太信了,您也别放马前炮,话说回来,俺们的确合作很年了,不过俺还从来不知道你到底是哪里得来的这些消息。这也是最后一次了,你要是没啥忌讳,就和俺们兄弟们说说,让我的兄弟也长长见识。”

“是啊,说说!”那年轻人马上附和道,“我以后也好跟我那些娘们吹吹牛!”

李老板笑了笑,回道:“哎呀,你们两个……,真是……你们要是真想知道我告诉你也可以,但是说出来恐怕你还不信。”

\chapter{偷听}

那班人安静了好一会儿,才听李老板说道:“本来嘛,这种事情我是不会告诉别人的嘛,不过大家跟着我这么久了,我当你们是自己人了,你们既然想知道,我就说一下好了嘛。”

那年轻人马上兴奋道:“那敢情好,不瞒您说我们还一直猜呢,您是不是有什么绝活儿,一找就能找到古墓的位置。”

李老板又顿了顿,听上去也是不太愿意讲的,说道:“那有这么神,其实也不是什么秘密,这事情和我祖上有关,我的族谱上有这么一件事情,我说出来你们听听。”

李老板说着,就讲了一件很有趣的事情:

那是北魏时候的事情,兵慌马乱的,一天不知道打多少次仗,成年人都死光了,他的先祖,不到六岁,就得出去放牛,维持家计。

那一年,他们的村子附近发生暴乱,官兵来镇压,村里人都逃难去了,他们家里没来的及走,给堵在屋子里面,外面杀的天昏地暗,一直到第三天才平息掉。

他先祖战战兢兢,偷偷爬出去看,发现满地的尸体,还有很多人没有断气,他吓的发呆,忙跑去找他的牛,结果进牛栏一看,牛已经不见了,稻草里,却躺着个伤兵。

那兵是个哑巴,不会说话,伤的已经很重了。我老祖宗当时年纪太小,也无法分辨这到底是官兵还是造反的,只看他可怜,就取了点水给他喝,还给他用布止了血。但是那哑巴伤的实在太重,坚持了没多少时间就不行了。

临死的时候,他拿出一卷写满字的麻布,交给了他祖宗,还做着手势,让我的祖宗好好保管。

可惜,他老祖宗家里全是文盲,根本不知道上面写的什么,后来那年大寒,冻死了很多人,家里人就把这块布,当成布料做了棉衣。

成年后,他祖宗就给征当了兵,在南北朝的征战中,屡建功勋,后来给提到了校尉,但是当时因为流年积弱,朝代更新太快,到了他先祖晚年,家势又逐渐衰落,结果死的时候,陪葬的东西,只剩下那条棉衣。

之后他们的家族经过几次兴衰的更替,到了晚清的时候,已经是一方地主,一次迁祖坟的时候,几个长公不当心,把棺木倾斜,里面的尸骨倾斜而出,倒了一地。在清理骸骨的时候,他的爷爷发现,里面所有的东西都烂光了,但是那陪葬的棉衣里的那块布,却依然保存的完好。

他爷爷感觉很奇怪,将这块布,交给他家里一个做古董生意的人,一看之下,便发现,那块布名堂不小,上面的字,叫做哑文,是传说哑巴才能看懂的字。

李老板说到这里,问他们道:“你们可知道这块布用来做什么吗?”

众人沉默了一下,一个刚才没听过的声音说道:“这个在下倒是略有耳闻,当时候,北魏有一只军队,都是哑巴,这东西,是他们传机密消息的东西,上面的字都是‘哑文’,一般人还看不懂,在下还是听自己的大爷说的。”

李老板点头,道:“师爷到底是师爷,那你可知道,这只军队又是干什么的吗?”

那师爷笑道:“那我就不甚清楚了,不过,听说,这只北魏的军队,是沿袭曹操的摸金校尉,明里是皇帝的护卫,暗地里,也做着倒斗的买卖……,因为是哑巴,又用只有他们知道的哑文,所以他们所倒的古墓,都只有他们和皇帝知道,他们的行迹,也一直非常的神秘。”

说到这里,那师爷顿了顿,似乎想到了什么,问道:“李老板,莫非你说的那块麻布,竟是‘河木集’?”

李老板一下子哈哈大笑,得意的点了点头,说道:“厉害厉害,有师爷你在,老子想卖个关子都卖不到,不错,就是这东西。”

师爷吸了口凉气,回道:“那可真了不得啊,同人不同命,有这东西,该是李家发财啊。”

那年轻人听不懂,问师爷道:“河木集是什么东西?和古墓又有什么关系啊?”

师爷道:“传说这哑巴军找到古墓之后,通常并不是急于开挖,而是记录了下来,用马踏平,灌上铁浆子,等到需要的时候再根据记录重新找回,这记录古墓位置的东西就叫《河木集》,取何处有墓之意。”

那年轻人吃惊道:“我考,那这么说,上几次我们去倒的那几个斗,都是这上面得来的消息?哇,李老板,那你可太不实在了,有这么个宝贝,也该分我们多点嘛。”

李老板笑道:“也不尽是,祖上的东西又不是用不完的,我家祖宗棺材里那块白布,记载了二十四个古墓的位置,现在要去的这个,已经是最后一个,不过这一个,应该是所有古墓里面,最好的。”

那年轻人问道:“那上面有没有说,里面都有些什么东西啊?”

李老板皱了皱眉头道:“那倒没有详细记载,不过那白布上说,这一个斗中的宝贝,凡人无法消受,是极品中的极品,比秦始皇帝还要好上三分,绝对不会有错的,你们就相信我吧。”

我和老痒听到这里,已经知道他们来到这里,的确是有一个目标,但是我们没想到,这几个人,竟然来头这么大。老痒问我:“你——你说这个姓李的说的是不是真——真的?世上还能有比秦始皇陵还好的斗?”

我摇摇头回道:“这我可说不准,不过你看他说得这么信誓旦旦,没一万也有五千,他们明天肯定过山,我们跟着就是了。”

老痒说道,“那我——们干脆跟到底算了,他们这一次的目标应该不小,就算捡他们吃剩下的,也能混个半饱。那破殉葬坑,咱们就别去了?”

他这话因为紧张结巴得特别厉害,有几个字就说得特别的响,我一听糟了,忙捂住他的嘴巴,让他别激动,同时竖起耳朵听那边的反应,但是已经晚了。那边突然间就静了下来,显然已经发觉了附近有异样。

我和老痒忙屏住呼吸,竭力不发出一点声音,心跳得像打鼓一样,他们也都不说话,似乎在努力听周围的声音。双方都不出声,就这样僵持了好几分钟,那老泰熬不住了,轻声说道:“二麻子(那年轻人),好像后面有动静,去看看是什么东西。”

听完这句话,我就听到两声清晰的手枪上膛声,一下子就一身冷汗。看样子果然是悍匪,这下子怕是要给老痒害死了。

我转头看了看四周的环境,如果现在马上逃跑,我有八成的把握能逃的掉,但是以后跟踪他们就麻烦了,如果现在不跑,我实在没把握能在他们眼皮底下躲过去。

正在犹豫不决时,突然从远处传来一阵嘈杂的声音,我向那发出声音的地方望去,只见一排四五只手电正向我们这边靠拢,是巡山队过来了。这时候就听到泰叔轻声叫了一声:“妈的,咱们扯呼。”说完几个人匆匆忙忙地把火踩灭,背起装备就往森林深处跑去。

老痒刚才还吓得半死,现在一看人跑了,又急起来,忙问我:“怎——怎么办?追——追不追?”

我小心翼翼的探头一看,发现他们一群人都没有打手电,森林里面一片漆黑,早已看不到人影,说道:“不成,你看这黑灯瞎火的,我们这么个追法说不定能追到他们前面去,我们先歇着,明天跟着他们的脚印走,相信他们也不会走太远,还得停下来休息。”

老痒心里干着急,也每办法,这时候那几个巡山队的人已经离我们很近了,我们再不走,估计要被逮个正着了,我让喋喋不休的老痒闭嘴,拉着他匆匆忙忙的往另一个方向的森林深处钻去。

我们不敢走的太远,怕明天回去找不到地方,两个人躲在一个灌木丛地后面,看着远处手电逐渐远去,才松下心来。

我想了想,对老痒说道:“这一路过来,当地人都说现在这季节是盗墓最猖獗的时候,恐怕这晚上巡逻的人不会少,我琢磨着我们也别想好好睡了,找个地方窝一个晚上,明天得赶紧再往里头走走,不然两个外地人在这里,给逮住了没办法交代。”

老痒点头称是,我摇了他一下,他竟然已经在半睡半醒之中了,我暗叹了一声,把衣服裹了裹,心说看样子上半夜得我来守了,可我往树上一靠,迷迷糊糊着,不知不觉也睡了过去。

第二天,我们一大早就醒了过来,由于睡在树下,一头的鸟屎,臭得我都想吐了,老痒也不管这些,拿手捞了几下,就嚷着要赶紧去找那班人,我实在无法忍受顶着鸟屎在森林里到处跑,只好牺牲了半壶水冲了一下。

我跟着老痒急急跑回昨天待的地方,心里祈祷地上能留下些线索,但是兜了好几个圈子,我们连昨天那堆篝火的残骸都没有找到。老痒对我很有意见,一直在我耳边唠叨:“所以说——说,昨天让你跟——跟上去嘛,你看——看,现在倒好,煮——煮熟的鸭——鸭子都飞了。”

我大怒:“他娘的,哪来这么多意见,你看这里就一条山路,他们能走到什么地方去,我们一直往前,我就不信找不到。”

我们延着山路快步追赶,走了整整一个上午,路都已经走完了,还是没有发现他们的踪影,再往前去就是一片极其茂密的森林,树木攀天,灌木丛生,完全没有路标,我看着心里有点发悚,这说明这后面的路连巡山队都不会去走,那算是真正进入到蛇头山内,深山老林之中了,至此往上,才算是真正的山路,不知道有多少峭壁等着我们去爬。

这一路过来,再没有看见任何篝火的痕迹,我心里已经沉了下来,这几个人可能昨天晚上给巡山队吓跑之后,就没有休息,直接赶夜路前进了,要真这样,我们赶上他们的机会就几乎是零。

我站在山路的尽头犹豫了一下,马上做了决定,人的精力是限度的,这些人如果赶了一夜路,那他们今天白天无论如何也得休息了,而且晚上赶路远比白天要慢的多,他们肯定还在我们前面不远的地方。我们跟上去还有希望,只是走起路来要小心点,不能给他们发现了。

我们从背包里掏出军用匕首挂在腰间,两个人各折了一根大树枝当拐杖,这秦岭之中多有野兽,说大了去就老虎和熊,往小的说有狼和野猪,要不是不走运碰上一两只,我和老痒够他们吃好几顿了。

老痒问我,如果我料错了,追不上他们怎么办,我心里琢磨了一下,对他说根据来之前查过的资料,这山里面有不少采药人搭的临时窝棚,里面有炊具,柴木和风干的肉类,我们如果能找到一个,那今天晚上就可以好好的休息下,然后再作打算。

老痒道:“你可得确定,咱们现在要回头还有机会,再往里走——走?你——你看这四周连——连个鬼影都没,等迷在林子里面就晚了,蜀道难,难于上青天,自古长安入蜀,一千年来这连绵几百里的大山里面不知道死过多少人,还不知道晚上闹不闹鬼呢。”

我嘲笑他道:“刚来时那股雄心壮志哪里去了,我说你他娘的就是一个纸上谈兵的。这还没到山里头呢就给我蜀道难了,你要不敢进去,那咱就回去。”

老痒笑道:“我是提出困难在先,看你的决心会不会动摇,现在看来咱们的小吴同学果然已经屏弃了书生气,向我们这样的流氓靠拢了,你放心,你兄弟我绝对不是纸上谈兵的人,不要说蜀道难,狗道难都不怕。”

我们一边拿树枝敲着前方的灌木,一边进入丛林,以远处一座山峰为方向,闷头走,没有道路的“山路”非常难走,地上几乎都是草藤,顶上又是茂密的树冠,阳光极难照下来,走了不知道多久,只觉得天昏地暗,哪里都好象是看到过的,就在我开始怀疑我们是不是在原地兜圈子的时候,山势转陡向上,前面出现了一面峭壁,一排不知道什么时候修建的栈道修在上面。

栈道年久失修,已经呈现出一种暗绿的潮湿的颜色,上面缠绕着大量的春花腾和猪草,似乎很久没人走过,我们正想爬上去,忽然听到一边树林里有人叫道:“喂!你们是干什么的?”

我和老痒吓了一跳,转头过去一看,一队人马正从远处走来,都是当地人摸样的人,有男有女,似乎也是和我们一样要到山对面的村落去的。

我不知道是该高兴还是害怕,忙打了眼色让老痒把腰里的匕首藏起来,然后迎上前去,装作很诚恳的样子问他们道:“大兄弟大妹子,我是外地来的游客,想到山对面的村子去,打听一下,再往前的村子还有多少山路?”

一个穿红大褂的妇女打量了一下我,说道:“你是说俺们村吗?你大老远跑来到俺们破村里来干嘛?”

我一看,这里的妇女警惕性挺强,瞎掰道:“我来找个人,你们那村我前两年来过,那时候有个老大爷招待过我,这次我回来看看他,不过两年没来了,路已经不会走了。”

那中年妇女瞪了我一眼,骂道:“我呸,就你那贼摸贼样,谁知道你安的什么心?你们这样的人俺见多了,不是去挖坟墓的就是偷猎的,想骗老娘,你还不够火候。”

我被她骂得瞠目结舌,不知道怎么回话好,老痒一把把我推到一边,啪一张一百块递到这中年妇女面前,说道:“哪——哪那么多废话,你哪只眼睛看到我们挖坟墓了,客气点回答问题,这——这一——一百块就是你的,他娘的,再敢罗——罗嗦半句,老子给你一耳光。”

这队伍里还有好几个壮汉,我听老痒这一说,心说要遭,山民彪悍,你还敢说这个,当下往后退了一步,准备开溜。谁知道这中年妇女后面一个男人看到这钱,马上笑眯眯的接过去,说道:“别生气,别生气,俺媳妇和你们开玩笑呢,你们想去俺们村,得往左边走,绕过这个山头,有一个瀑布,顺着这个瀑布的水一直往前走,那是最快过山的捷径了,只要跟着山溪走,就一定能到俺村了。”

老痒咧咧嘴,问道:“你骗人吧,要绕过去,上这个栈道不是更快吗?”

那男人道:“这个栈道,不知道什么年月修的了,从来没加固过,现在已经没人敢走了。”

我听了心里咋舌头,心说幸亏遇到他们,刚才走的蒙了,差点就上去,要困在上面真不知道怎么办好。

那男人看了看天色,说道:“哎呀,我看你们今天晚上也赶不到了,得在这山里过夜了。那山溪有几条支流,你要是没走熟悉,肯定会走叉掉,要不这样吧,我们是去那边打猪草,你们要不等等我们,我们明天就回村里去,跟我们一起走,就没事情了。”说着便来帮我拿装备。

我一看他还挺热心的,看样子不像是坏人,心里迅速盘算了一下,我们要去的地方是在这蛇头山另一面的峡谷,那翻这座就已经花了我们将近三天时间,人的负重有限度,不可能带超过十天的干粮,我们翻过这山之后肯定还得进他们村子买点东西,走在我们前面的五个人现在也没影子,说不定和我们走了岔路了,如今难得碰到人,就不用冒迷路的危险了。

我和老痒交换了一下眼色,忙点头道:“那大兄弟,谢谢你了,来来来!”说着掏出香烟,给几个男的都分了一根。

那中年妇女还想罗嗦,那男人瞪了他一眼,她白了我们一眼也不敢说什么了。

山里的风气,一般男人是家主,女人都没什么说话的地位,只要搞好和几个男人的关系,这些个村姑子应该拿我们没办法,我看着那中年妇女的表情,心里暗笑。

我们加入他们的队伍,那男人年纪最大,似乎不用干太多活,老痒就集中火力和他套近呼,那男的告诉我们,他是村里的书记,这村子太落后,虽然通了电线,但是交通不方便发展不起来,现在年青人都往外跑了,农活没人做了,他们这些干部都的赶几十里山路出来打猪草。不过他腰有毛病,做不了多少时间就得歇息。

我一边应着,心里也感慨,这些人也不容易。

我们跟他们走了一段,到了一处地方,他们开始干活,我们就在一边查看地形,不过这里山势偏低,山那头的景象,并无法看的很全,只觉得山连着山,一片的郁郁葱葱,老痒所说的那个殉葬坑,也不知道在广翱山脉中什么地方?

打完猪草已经是晚上,我们帮忙背着几乎有我本人体积这么大的一大包草,背着夕阳往回走了大概一个小时,天已经渐渐黑下来了,走着走着,我突然发现老痒的表情变了,眼睛只看着四周,不停的瞄来瞄去。

我问他干什么?他低声说道:“这地方我上次来过,如果我记的没错,再往前走肯定有个落脚点。”

果然走了不久,前面出现了一个采药人的木头窝棚,老痒表情兴奋起来,给我打眼色,意思是我没说错吧?那男人推开门,转回头对我说道:“咱们今天就在这里过夜,这里还有灶台,你们要愿意可以自己煮东西。”

我跟着他们进去,发现这是个两层的窝棚,由一只梯子相连,上面是个阁楼,里面没家具,但是铺着几块大木板,房间的中央有一个土坑,里面都是炭灰,相信是用来生火取暖的。我们放下装备,在外面胡乱捡了点柴火,赶紧生火取暖。然后从包里掏出干粮,直接烘烤着吃,等我们吃完,外面已经黑压压一片了,四周传来野兽的叫声。

老痒点了一支烟,问村支书那是什么,后者也说不清楚,这里打猎的人早就死没了,要找村里的老人才知道。又说道:“晚上我们男人每人只能睡半宿,得有个人看着这火不让它灭掉,不然恐怕外面的野兽要进来的。”

我不置可否,这一天的路累得够呛,想到以后可能连续几个星期都得这样过,不由有点悔当初答应老痒,对老痒说:“我守最后一班好了,我先打个盹,你半夜里叫醒我换班。”刚说完他就大声抗议,但是我糊里糊涂的已经不知道他在说什么,不一会儿就进入了梦乡。

这一觉睡得不太安隐,我翻来覆去的到了后半夜的时候,突然有人摇我,睁开眼睛一看,其他人都睡觉了,老痒一边四处看着,一边轻轻推我,轻声叫道:“起来,快起来!”

\chapter{挖掘}

我睡得很不踏实,几乎是在半梦半醒坐了起来,心里一股起床火,刚想骂他,他捂住我的嘴巴,轻声道:“别说话,跟我来。”

我莫名奇妙,见他表情不善,又不知道发生了什么事情,批上外衣坐了起来,问道:“干什么?出了什么事情?”

老痒轻声说道:“跟我来,我带你去看点东西。”

我盯了他好一会儿,心里觉得奇怪,不过看他的表情,不像是在玩我,于是披上外衣,就跟他偷偷走出屋外。

窝棚外面就是森林,老痒拿出指北针,确定了一下方位,从我们装备里拆出折叠铲子,招呼我跟着他。

我们打着手电,走在下风口,足足走了十分钟,他才停了下来,用铲子插了插脚下的地,说道:“就是这里了?”

我心里疑惑到了极点,看他的样子,难不成半夜三更他想来这里种树?

他看我表情不善,忙解释道:“我和我老表上次从山里出来的时候,也是在这里过的夜,那天晚上我发现他半夜偷偷溜了出来,不知道去干什么,所以我就跟着他,结果发现他在这里埋什么东西?不过那时候我们情况很槽糕,我没力气去管这闲事情,只想快点出山去,所以也没去计较这事情,现在想起来,那时候的情景有点不正常。”

“你确定就是这里?”我问道。

他点点头,“我老表从那洞里出来就神经兮兮,不知道中了什么邪,我肯定他有事情瞒着我们,这一次正巧回到这里,我准备挖开来看看,他到底埋了什么?你帮我望望风。”

我点点头,老痒开始下铲。

这里的土似乎不硬,但是那些村民还睡在不远的地方,不知道会不会吵醒,所以老痒每挖三下,都要停下来听听周围的动静。

他挖了足有半个小时,我开始怀疑他是不是弄错地方了,突然,他的铲子似乎插到了什么金属的东西,发出一声清脆的声音。

他停止了挖掘,俯下身去,从坑里拿出了一根棍状的物体。

棍状的物体上都是泥,我无法判断那是什么,但是我直觉上,感觉似乎是一根骨头,老痒略微擦拭了一下,脸色已经一变,对我道:“我操,竟然是这个东西。”

我凑过去看,那是一根长着绿色铜锈的青铜铸器,底上有很明显的断口,是给人从另一件青铜器上锯下来的,接着手电的光,我能看到上面有着类似于单头双身蛇的抽象图案。应该是老爷子说的“厍族”的东西。

老痒对我道:“这就是我和你说的那青铜的枝桠,没想到我老表竟然偷偷把这东西锯下来了。”

我皱了皱眉头,他们这些人,可以说是整个盗墓阶级中最没有素质的一群,也是数量最多的一群,为了几千块钱,破坏一件绝世珍品,是再平常不过的事情。

老痒继续挖掘,看还能挖出什么来,但是挖了半天没有任何东西再出现,他开始将土回填回去。

我们将这枝桠用布包好,蹑手蹑脚的走了回去,其他人一天劳作,都还在熟睡,我们却再也睡不着了,他在我对面坐了下来,开始往篝火里加柴。

我看到老痒脸色凝重,忧心之态又现,忍不住问道:“这几天看你忽喜忽忧的,你是不是有什么难言之隐啊?长痔疮了?”

老痒点上只烟,说道:“哎,要是这么简单就好了,我是觉得有点不对劲,有点事情想不通啊。”

我不说话,听他说下去。

老痒道:“主要是我老表的事情,我和他进山的时候,他还很正常,但是自从他看到这根青铜枝桠之后,我就感觉他开始变了,刚开始我老表只是突然变得有点神经质,逐渐的,我就发现,他整个人好像越来越失常起来……”

我问道:“你的意思是,你老表疯掉,和这玩意有关系?”

老痒点点头,“你看,他偷偷的把这东西锯下来带出来,又埋了起来,是为了什么呢?”

我看着老痒摆弄那根青铜的枝桠,忽然感觉上这东西哪里见过,忙掏出王教授给我的资料,翻到一张图片出一比对,果然不错,那是1845年一个英国传教士汤马士在湘西一个山洞岩石壁画上临摹下来的东西,是一棵类似于树的图腾,汤马士在画下面注释说,这是当地土民的“神树”。后来这份笔记流落到王教授手里,王教授根据其中的描述,认为这种神树是蛇国的文化的图腾之一,代表着大地与生育的神性。

我将青铜的枝桠对比上去,发现这一段只是树枝的末梢,如果按照这个比例来说,那整棵青铜树应该有七八十米高,如果整体发掘出来,足以震惊世界了。

我拍了拍老痒,让他别多想,如果真是这枝桠的问题,那他也早就和他老表一样了。

\chapter{夹子沟}

经过了五个小时的跋涉,第二天下午,我们终于翻过蛇头山,来到山下第一个小村寨里,我们百般谢过带我们过来的书记,然后在村口分别,老痒来过这里,带我进去找他上次寄宿的村户。

这个山村依着陡峭的山势而建,夹杂着石头搭建的足有百年历史的明清样式的民房,村中道路是一个完全的青石板坡路,道路最上面的人家的地基足足比最下面的人家高了百来米,山溪从路边的沟渠中穿过,到处是绿色的青苔。我一路观赏,不少民居的围墙,都有不同年代的墓砖搀杂其中,古时候掘墓取砖的风气由此可见一斑。

我们在老痒上次住过的人家买了干粮,在他们家里用溪水洗了个澡,然后将衣服洗了晒出去,自己穿着短裤坐在溪水边上,商量下一步怎么办。

要赶上前面那五个人已经不可能也没必要了,反正我们已经顺利的过山了,现在就要靠老痒所谓的记号,找到他三年前来过的那个地方。

我问他到底做了什么记号,他这么有信心现在还能找到?老痒告诉我,他上次去过的那个殉葬坑,要通过一段十分奇特的地貌,叫做“夹子沟”,这里的人都知道那个地方,而过了那一段地貌,离他说的那地方就不远了,不过的是,夹子沟离这个村庄有四十多公里远,几乎是在原始丛林的腹地。

因为有了没有向导进山的惨痛经历,我们请教了那书记,想找一个向导,带带下面更加艰难的旅程。

书记让自己的小孩子带我们去找一个老猎人,我们跟那光屁股小孩子在村子里四处转悠了几圈,来到了一户两层的瓦房子前面,小孩子指了指在那里晒太阳的一个白胡子老头,说:“就是他,老刘头。”

刘老头是外地人,年轻时候逃壮丁来到这里,一直定居下来,是这里的老猎户了,他八十多岁,身体还很好,几乎所有进老林子的考察队啊考古队啊盗墓的啊,刚开始都要他带上几次,他也乐的吃这碗饭,一来来钱快,二来地位高,我们说明来意,他也不奇怪,只对我们摇头,说:“不中,这个时间不能去夹子沟。”

我听了纳闷,问他:“怎么不能进山啊,现在秋高气爽,正是好打猎的好时节,这个时候不进,那什么时候能进啊?”

他叫他儿子给我上了茶水,说道:“这个季节,山里头特别邪呼,闹鬼闹的很凶。我八十多了,不会骗你们,夹子沟那个地方,其实是条阴兵的栈道,你要是碰上他们借道,那就得给顺便捎上,被勾了魂魄,邪门的很呢。”

我没有去过那个地方,不知道那里是个什么样的地理环境,心里觉得好笑,不过老一代人有他们自己的世界观,我们也不好勉强,央求了一下没结果,就只好问他进山路线的情况。

老人告诉我们,从这个村子进到秦川崇山峻岭之中,往西走七天,会有一座天门山,两边都是峭壁,无法攀爬,但是山中有一道奇特的裂缝,只能并排两人通过,就是我们常说的“一线天”,也就是老痒说的“夹子沟”,相传南北朝末期,当地有人看到,有一只北魏的军队经过栈道入秦川,这只军队很奇怪,行军中没有一个人说话,直入山中。军队经过这一山缝时,突然地动山摇,巨大的缝隙突然闭合,将部队夹入大山内部,从此失去了踪迹,再没有出来。

到了清朝的时候,这里来过几个风水先生替一有钱人找坟地,进山十几天,出来的时候几乎不成人形,都说这天门山内有一道黄泉瀑布,连着地府,他们差点进去就出不来。

一开始,山里人也都不信,不过后来很多人都说在沟里,听见山里有战马奔腾的声音传出来,这些事情才越传越厉害。有人还串起来说,说是地府的阴兵便是由黄泉瀑布进出阴阳两界,那南北朝末期的北魏军队,就是自阳间返回地府的鬼兵。

老爷子说,到天门山的那一段路,我们可以走上一走,但是天门山后,那是世代人所能达到的极限,再往后的丛林里有什么,谁也不知道了,从古到今,凡是进去里面的人,无论是清朝的鞑子军,还是国民党的败兵,没有一个出来过,他年纪大了,不能带我去,村里其他人又都没有去过,要是我们真想去,他可以给我们指个方向,只要按他说的走,七八天工夫肯定能到,但是进去后发生什么事情,他一概不负责。

爷爷的笔记里说过,寻找陵墓,凡是有很详尽的民间传说的地方,都要特别注意,所以我特别留意的听了老爷子的这一段话,心里已然有了几分把握,我们要去的那个地方,确实应该是在那一带附近。

我们谢过老爷子就想离开,老人家大概很少有客人,所以热情的很,一定要我们留下来吃饭,我们执意要走。他也没有办法,就让给我们包了几个腌制的荤菜,我本来嫌麻烦,不想要,但是一看里面有烧肉,想起自己这几天吃的都是干粮,肚子实在不争气,就收了下来。

休息了一天,我们再次赶路,这一次目标明确,我们顺着指北针的方向,咬紧牙关,翻山过河,一头扎进了中国腹地最神秘的茫茫原始丛林之中。

沿途无话,期间个中辛苦我都不想用文字记录下来,只知道七天之后,老痒叫着看到树冠之上显现出的天门山顶之后,我们停下整顿,发现自己已经和野人无样了。

老痒观察四周的地方,告诉我就是这里!通过这个夹子沟,那边就是一个小峡谷,他们发现的那个殉葬坑,就是在那个里面。

我爬上一棵巨大的老杉,拿起已经只有一边能用望远镜看去,天门山的山形挺拔,山势奇伟,上面鬼岭妖松,景色十分奇特,但是山也并不见得像是一道门的样子,不知道天门山的名字由何得来,而那中间的一线天,从我这里看去,只是一道黑色的细线。

我们爬上了矮山脊继续像天门山靠拢,顺着山势向前走去,边走边查看前面的地形,将近正午,来到了天门山的山脚下,夹子沟的起始段的一片乱石岭就在我们眼前。

秦岭实在是一个很奇妙的地方,特别是那些没有经过旅游开发的地段,有很多奇妙的景色,在天门山的峭壁下直接抬头,会发现地势极端的壮观,形容的普通一点,就一座巨大的山岩被一把利剑劈了一下,中间形成了一条细小的裂缝,这条裂缝的底部,就是夹子沟,因为山岩的地势极高,所以这里产生的一线天景观不同于那些矮山,抬放眼看去,只能看到一条极细的光线,在遥远的天顶,真的犹如整个天空浓缩成一线一样,如果不是亲身经历,无法领略到这其中的万一。

夹子沟内,底部乱石叠嶂,两边不时有清泉撒下,石头上到处是绿色青苔,非常难走,不过这里却并没有远看的时候那么狭窄,而且光线很好,因为起始处的山势并不高,所以天上并不是一线天,而是“一根天”。

老痒回忆,通过这个夹子沟最起码要一个下午时间,而且里面过堂风极大,地面潮湿,生火很不方便,于是我们就在入口处不远停了下来,点上篝火,开始吃午饭,我们将老爷子带给我们的腌菜放到吃剩下的罐头食品里,然后用火加热,象吃火锅一样的吃,山民们烧菜都重口感,所以味道并不怎么样,但是比起我们的干粮,已经好上不知道多少倍了,所以前几天我们都节省着吃,现在靠近目的地了,可以放开怀抱,我和老痒几乎是狼吞虎咽,很快就把腌肉吃了个干净。

我并没有吃饱,想起那有一些腌山鸡炒笋,就想索性吃光算了,不料回手一摸,发现那只放食物的袋子,已经不见了。

我四处找了一遍,却没有发现,觉得很纳闷,就问老痒,就听老痒在那里骂:“我操,谁把骨头吐到我领子里!”

我一看不对,我刚才吃的时候,几乎把骨头都吞了下去,哪里还会扔出去这么浪费。

正在奇怪的时候,又有一块骨头从悬崖上面掉了下来,我抬头一看,只见十几只金毛大猴子,不知道什么时候爬到了我们的头顶的山壁上,其中一只,正拿着我装山鸡炒笋的袋子,吃里面的鸡肉,看它吃的样子,应该是从来没吃过这么好吃的东西,几乎连袋子都吃了进去。

很快,它就将所有的东西都吃了干净,然后爬了下来,眼睛死死盯住我们的背包。

我心说不好,这些猴子可能以为我们包里全部都是吃的,想来抢了,这可麻烦了,正想着,那只猴子已经发出一声尖叫,一刹那所有的猴子开始向我们逼近。

\chapter{猴子}

大号的猴王看着我,不停的裂开嘴巴,露出自己的白森森的獠牙,同时发出一种带有威胁性的声音,好象是在警告我们。

我和老痒各自拿起一根顶端燃烧着的柴火,拼命舞动,将冲上来的猴子逼退,有几只动作慢了一点,屁股就被我狠狠的烧了一下,疼的它尖叫着逃到很远的地方。

但是同时,有几只特别机灵的猴子,正在偷偷的靠近我们的行李,等我看出苗头的时候,为时已晚,老痒还没有放入背包的几个防水袋被一只小猴子一把抓了过去,我一看暗叫糟糕,忙上去抢,可等我一走开,我的身后也窜出了一只猴子,想要来抢我的行李。

幸运的是,我的行李十分沉重,它拖了几下,发现没有办法很顺利的拖走,只好作罢,转而把手伸进行李包中,想将里面的小件东西拿出来。

我心里吃惊不已:这些猴子的行动非常熟练,这样子围攻人类,肯定不是第一次了,我一直认为猴子就算再聪明也有个限度,现在看来,如果只算抢劫这一个职业,我们还不一定能比的过他们。

我这里一分神,那只猴子已经从我的包里掏出一只盒子,我一看不得了,那是一包压缩饼干,也不管正在追的那只,冲回去,飞起一脚将那只猴子踢飞,然后捡起盒子,赶忙塞进包里。

这个时候,突然眼前黄光一闪,那猴王已经跳将起来,一爪抓向我的脸,我看过猴子捕杀兔子,它们的爪子非常锋利,要是给抓到,我非破相不可。

情急之下,我来不及侧身,只好抡起柴火棍去挡,那猴子一下子就在我手上抓出了一道长长的血痕,我疼的一龇牙,柴火棍脱手掉了出去。

猴王落地之后马上反扑过来,我来不及去捡柴火棍,只好匆忙间一脚踢了过去,谁知道它竟然一下子抱住我的腿,顺势就狠狠咬了我一口。

这一下实在是厉害,我疼的几乎抓狂,一巴掌就拍了过去,它反应很快,一个翻身立即跳了开去。我胡乱一抓,鬼使神差,给我一把抓住了它的尾巴。

猴子的尾巴非常重要,打斗中被抓住尾巴,等于被判了死刑,它一下子也慌了,发出一声嘶吼,不顾一切的朝我面门扑来。

我心里杀心已起,一个侧身躲过它的最后一击,抡起它的尾巴就用力往地上一摔,我估计着,这只猴子最起码也有40多斤重,这一下虽然不致命,也足已经把它摔的蒙了过去。

可是那猴子却强壮的出奇,这一下虽然我自己感觉用了杀手,它却一点事情都没有,反而惨叫着还想再扑过来。我一下子有点不知所措,忙又用力一甩,将它狠狠的拍到一棵树上,这一次用力过大,手吃不住力气,它被我甩出去好几米,翻滚几下,一下子跳了起来,爬到一棵树上。

老痒惦记着被抢去的那几个袋子,还在追那几只刚才抢我们东西的饿猴子,那些猴子看猴王刚才吃了亏,哪会和他硬拼,一下子逃散,但是它们并不逃远,而是继续做着威胁的动作,他去追其中一只,另几只就跟在他后面,向他丢石头,搞得他非常郁闷,就这样东一下西一下,猴子一只没打着,他自己倒已经气喘吁吁了。

我隐约看了觉得不妙,这几只野生猴子个子巨大。行动灵活,最麻烦的是他们一点也不怕人,我对付一只猴王已经非常吃力,要是有两只猴子同时攻击我。恐怕今天就有可能在这里吃大亏,而且猴子的记忆力很强,我们这一次莫名其妙的惹上这些猢狲,若不能彻底解决,恐怕以后不得安宁。

老痒追了半天,筋疲力尽,喘着气跑回来说:“不——不行,这些猴子跑得太快了,我们别和它们一般见识了,还是走吧,那些丢了的东西,就当送给山神爷的见面礼好了。”

我一想也实在没有办法。在老林里和猴子抢东西,我们实在没有胜算,万一时间耗下去,说不定还会有别的损失。而且,虽然丢了一些东西,但是都不是很关键,象冷光棒,我们用火把代替就可以了。

于是我点点头对老痒说道:“说的对,这里面很深,一旦天黑下来,我们的路就更难走,不过,你小子他娘的得把东西看好点,别在着了猢狲的道儿。”

老痒想起刚才那事情,气就不打一处来,对我摆摆手说:“行了,你就别提了,这梁子算是结下了。”

我们两个绑紧背包,大声呼喝着赶开猴群,继续往窄路里走去,那些猴子看我们走了,以为我们逃了,纷纷跳上两边的山壁撵了过来,一边撵还一边向我们发出嘲讽的声音,老痒听了火大,回头大骂:“你们这帮猢狲别得意,老子要是还有机会回来。他你们全逮回去吃了!”

那群猴子看到他大叫,撵得更起劲了,特别是那只猴王,摆出胜利者的姿态,一路跟的很近,想趁我不注意再扑上来,老痒看着就火了,捡起地上的时候扔在那只猴王鼻梁上,这一下打的颇重,直把那只猴王打的几乎从峭壁上摔下来。

没想到的是,那些猴子恼羞成怒,纷纷捡起地上的东西丢过来,很快我脑袋上连中几下石头和泥块,幸好没别人看到,不然我只能一头撞死挽回颜面。

我们一路狂奔跑,跑了足有半只烟的工夫才停下来,我一看,我们已经完全进入到这条夹子沟里,上面的“一根天”已经变成“一线天”,因为两块山壁之间的距离更窄了,两边崖顶就有一种要压下来的感觉,让人看着背脊发寒,恨不得马上走出这里。

看来那刘老头所言非虚,我心里暗道,搞不好这条山隙真是通向黄泉路的。

再往前走,这种感觉更甚,以这种趋势,如果不是事先打听过,我必然以为这最里面,两座山是合在一起的。

我回忆着那老向导说过的话,想着他说的那个传说。

阴兵的传说我听过不少,也有不少无聊的人给过推测,比较有名就是云南的惊马槽,传说是南蛮王孟获找人挖的,这地方现在还在。一到雷雨季节,就会传出兵器交击的撕杀声,另一个就是唐山大地震的时候,更加玄乎,听说是有很多看到一长列马车队,载着十万头颅从唐山出来。正遇上进城救灾的解放军运输队,而后云云我也不记得了。

老痒还说了一些其他的事情,说这条沟自从形成以来应该几乎没人走过,却一棵杂草也不长,好象天天被马匹践踏一样,前几年还有人想在这里建一个景点,但是只要施工队一来,这里就开始下大雨,每次都是这样。搞的那几个领导一点办法也没有,加上离村庄实在太远,只好作罢。

我们继续深入,逐渐走的有点麻木,这山缝也不知道多长,越往里面光线就越暗,温度也降了下来,感觉阴森森的,有种非常莫名的被窥视的感觉。而且不知道什么时候,后面的猴子也没有跟着我们了,一下子整个山缝里就安静的有点可怕,只剩下风吹过的呼啸声和另外一些说不出名堂的古怪声音。这种感觉,让我们都非常的不舒服。

我和老痒一个人说一个脑筋急转弯,转移自己的注意力,不被这山缝里诡异的气氛所影响,虽然如此,我的心里还是感觉到非常的不安,而且随着我们的越来越深入,这种不安就越来越明显,我甚至有几次都感觉到,我们头上的那一线天,随时可能消失,我们会被永远困在漆黑一片的大山内部。

我胡思乱想着,也不知道走了多久,忽然,走在前面的老痒停了下来,我一时反应不及,撞在了他的背上,这一下撞的很厉害,我有点窝火,问他:“怎么回事情?说停就停,也不言语一声。”

他转过头来,脸色惨白,嘴巴抖了半天,结巴着说道:“老吴,前——前面——有个人——”

我楞了一楞,心说什么“人”,这种地方离最近的村庄最起码有四十多公里,怎么可能会有人在,忙探头过去看。只是一眼,我便头皮一麻,脑子嗡的一声,几乎咬到自己的舌头,脚后跟一磕,坐倒在地上。

原来前面的山缝阴影中,真的站着一个“人”形状的东西,脸隐没在黑色影子里,木然的看着我们。

\chapter{石人}

一路在一种木然的状态下,突然发现前面出现了这个东西,很少有人能马上反应过来。

我和老痒不由自主的后退,想和它保持距离,但是一时间我们都挪动不了自己的腿,只觉得心脏狂跳,浑身僵硬无比。

老痒比我胆子大一点,深吸了一口气后,对着那人喊道:“你……什么人?”

那人一点反应都没有,一动不动,似乎是一块石头一样。

老痒压低声音问我道:“你看他怎么不理我们?老吴,该不是给那刘老头说中了,遇到阴兵了?”

一阵冷风吹过,我略微清醒一点,说道:“别慌,是人就不用怕他,咱们看清楚再说!”说着掏出了手电,向它照去。

那个“人”穿着一身奇怪的古代衣服,裸露的手臂呈现灰白的颜色,木然的立在夹沟的中间。在昏暗的山缝阴影里,显得极其的诡异。手电照到它的身上,他一点反应也没有。

这个时候,我却发现了不对劲的地方。

原来,这个人的身上,竟然长着绿色的青苔。

无论是什么东西,除了乌龟,他怎么样也无法容许自己的身上长出青苔吧?我仔细看去,发现这“人”不是“肉”的,而似乎是用石头雕刻而成,只不过他的雕刻手法过于写实,在光线不足的情况下,才会被误会成真的。

虽然如此,我却笑不出来,这个石人简直是鬼斧神工,雕刻的太逼真了,就算我们近距离去看,也觉得场面骇人,头上直冒冷汗。

我们心有余悸的走过去,发现这“石人”的下半身被压在碎石头堆里,大概是随着上面的石头坍塌一齐掉下来的,脑袋部分已经没了,只剩下一个脖子,我抬头看去,果然看到峭壁的上方有一个地方岩石松散,只不过整个山势倾斜,形成了一个死角,我看不到实际的情况。

石人双臂裸露,不是汉文化的风格,在他身上刻的衣饰上,我发现了双身蛇的纹路,衣服的风格我从来没有见过,色彩已经有点退色,石人的头部缺失,大概是摔下来的时候砸碎了。

看到这些,我已经肯定,这东西,应该是一个陪葬的石人俑。

我看了看头顶,石人俑从上面坍塌下来,看样子这上面有东西。

老痒性子急,不等我看清楚,已经毛手毛脚的爬了上去,我跟着他趴在峭壁上,顺着坡度一点一点的移动,很快,就爬到了发生坍塌的地方。

上面似乎是一个依山壁开凿的浅坑,不少相似的石头人俑拜访在洞里,奇怪的是,这几个石头人的脑袋都不见了,脖子上放着人的骷髅,结合处用泥合了起来。

我知道这叫人头俑,是古时候打仗,携带整具尸体回来邀功太重,就砍下人头,这些人头给放在石身上,充当活人来殉葬。

西周原先还有壁画,但是已经给雨水冲刷成无法辨认的色块,洞的底部有一座依着山势雕刻的半身人像,胸口到脑袋已经被翻数炸掉了,只剩下一只手和半只肩膀还能分辨出来。

在塌口的中间,被炸出一个蓝球大小的黑幽幽洞口,我按耐心中的狂喜,拿电筒往里面照了照,发现里面空间极大。

我的直觉告诉我,这巨大石人像后面有可能是个古墓,而且很可能是老痒所说的那个巨大的殉葬坑所服务的主墓穴,只不过不知道是哪里的高人,已经走进过一趟了。

一般来说,能想到把墓修在这种地方的,墓主的身份肯定显赫,但是能把这种地方的斗都倒掉的,更是高手中的高手,普通的盗墓贼,就算他在这夹子沟里来回走上几百趟,也绝对想不到头顶上另有乾坤。

我和老痒合计了一下,决定先进去看看,反正目的地就在附近了,如果里面没东西,再出来也不会。做我们这一行的,有洞不钻,那是要难受死的。

他比较瘦,打头钻进洞里,这洞在里面的位置偏高,他脚踩不到底,只好贴在壁上,我把手电递给他,他接过一照,说道:“我操,里面有积水。”

我探头进去,看到里面是一个很大的拱顶的石室,是开凿出来的,顶上有一些壁画的痕迹,积水水位很高,几乎到了拱顶的边缘处,透过水面可以看到,浸在水里的四边的石墙上都凿着浅坑,里面全是长满青苔的无头石俑,这些积水,不知道是下雨的时候,雨水从这个洞口流进来积起来的,还是另有原因。

老痒和我说,他上次来的时候,那石头人俑还没有坍塌下来,如此算来,这被炸出的口子,应该还是这三年里做的。这里面的水不可能是雨水。

我让他小心为妙,老痒仗着自己水性好,一松手就跳了下去,一下子水就没到了他的胸口,他吓了一跳,差点滑倒。

我看着咋舌头,这水深得过头了,问他:“你踩踩水底,怎么样,下面是泥还是石头?”

老痒说道:“踩不到水底。他娘的,这水真他妈的凉。”

我将两个背包里的防水布都拿出来,把背包包起来,一个仍给他,另一个自己背上,然后小心的滑进水里,马上,一股凉气就从我的脚底板冒了上来,把我冷得打了个哆嗦。

脚下空空如也,果然很深,我心里道,因为事先我没有想到会在水里作业,没准备什么应对的装备,我们只有打着手电向里面游去。

才游了几下,就看到一个石门开在最里面的石头壁上。

石门因为水位的关系,显的很矮,矮门里是一条大概两辆解放汽车宽的石道,一片漆黑,我们手电扫过的地方,都是青灰色石壁,有粗略修凿过的迹象,有几段地方上面的也有壁画,但是这里的壁画已经是腐蚀的根本看不出来了。

一直往里面游了十几米,突然石道就一拐弯,呈90度的直角,我用手电照了照,发现里面深得吓人,不由停下脚步,不敢贸然进去。

事实上,现在的情况,再往里面走就不太明智了,这水深成这个样子,又看不到水里的情景,实在有点让人发慌,要是等一下水里冒出个什么东西来,就算是块木头,也能把我吓个半死。

老痒看了看四周的石壁,问我:“你有没有发现,这个墓虽然挺大,但是修得很粗糙,人看这些石头茬子?一块比一块难看,根本没修过,说这墓老板会不会也不太有钱,开了山就没钱装修了。”

我说道:“这可能只是整个陵区最外沿的地方,你看这里摆了这么多未完工的石俑,可能是陵墓工匠采石雕刻的地方,再往里去看看,应该会更清楚。”

我们继续往前,有游了几分钟,在通过那个转弯口的时候,听到前面黑暗里,传来了几声沉闷的水声,似乎有个什么东西正在水里潜行。

我抓住老痒的手,将他手里的手电,强行转向水声传来的方向,马上,我就看见,同时水面上出现了一道三角的水痕,瞬间沉入水中。

我还没反应过来那是什么,老痒已经一把拍开我的手,转头大叫了一声:“跑!”

\chapter{哲罗鲑}

老痒说是这样说,但是我们弓在齐脖深的积水里,如何逃得快,我扑腾了几下,回头一看,那三角的水痕已经闪电般向我冲了过来,经过的水面翻起一阵浑浊。

我赶紧将手电绑在自己的手腕上,拔出横插在皮带里的匕首,将背包背到前面当成盾牌,同时招呼老痒帮忙,却发现这小子已经屁颠屁颠的游出去十几米了。

我心里将他十代祖宗骂了遍,这个时候再不容我多想,那怪物闪电般冲过来,转眼便到了眼前。

我矮下身子,就准备硬吃这怪物的一击。那三角的水痕来的飞快,到了我面前三尺左右,突然水面出现一个扭曲的波纹,水痕却消失不见了。

说是迟,还是快,还没等我纳闷,突然我的眼前就炸开了一团水花,同时一股巨大的力量撞在了我的胸口,这一下子实在太快了。我根本不知道是怎么一回事情,鼻子里呛进一口臭水,酸的我睁不开眼睛。

我被这股力量压进了水里,顶着我向前游去,一下子我就被推出去十几米,我入水的时候根本没时间换气,气非常短,已经差不多到了极限。要是一直给它顶下去,非窒息了不可,于是咬紧牙关,操起匕首胡乱一桶,就觉得手里一震,也不知道桶在了什么地方,那家伙吃痛,猛地在水里一扭。将我甩的整个人倒了转,我脑袋拍在了墙上,一下子就蒙了。

不过好歹这一刀算是起了作用,我觉得胸口一松,那股力量消失了。

我知它松了口。挣扎着探出头来,贪婪的呼吸了一口空气,同时一摸背包,他娘的已经整个儿被撕走了一半,里面的东西都掉的差不多了,幸亏我把背包挡在胸口,不然这一下我已经挂了,这东西的咬力也太厉害了。

这时候四周光线非常差,只看见老痒的手电在后面直晃。但是这些微弱的光根本照不出什么来,反而把水片照的反光,影响我的视野。

我喘了几口气,脑子清醒了不少,这时候就发现手里的匕首没了,也不知道是刚才撞墙的时候掉进水里了,还是压根没拔出来,心里长叹一声,现在赤手空拳。又没了背包的保护,要是给它再来一口,估计掉出来的就是俺的内脏了。

我贴到石壁上,这里地方狭窄,这样贴着一边。它想要一口咬住我的身体也没有这么容易。

刚才搏斗的时候,我依稀感觉是条大鱼,可是这密封的矿洞里怎么可能会有鱼,而且还是这么大一条,这太不符合情理了。就算有,它吃什么,吃石头吗?

老痒从后面追了上来,看见我就大叫:“你没事情吧,没缺胳臂少腿吧?”

我忙拦住他,让他贴住墙,说道:“别过来,那玩意还在附近!”

他没听到我说什么,还问:“没事情吧,刚才我是想弄出点声音,吸引他的注意力,没想到他不吃这一——”话说到一半,突然他整个人一歪,一下被扯进了水里,水花四溅,同时水里拍出一条大鱼尾巴,绿水扑了我一脸。

我心里暗叫不好,老痒不知道是什么地方被咬到了,要是咬在身上,那真的不得了,不死也得残废。

我摸遍身上,再没有别的武器,只从口袋里掏出一把开军用罐头的刀来,这刀却是好钢口,但是太短,桶一百刀也不一定能把人桶死,现如今也没得挑剔,我大叫一声,飞身就扑进水里,向老痒那个方向游了过去。

那个地方正在混战,在水里我什么都看不见,只能用摸的,才摸了两把,正赶上鱼尾甩过来,面门被狠狠拍了一下,我被拍的七荤八素,身子在水里打了好几个转,脖子几乎折了。

巴掌把我拍的有点火起,咬紧钢牙再次冲了过去,慌乱间我一把抱住一个东西,只觉得滑腻腻,一摸全是鳞片。心说就是你了,也不是鱼的哪个部位,操起罐头刀就捅。

虽然这罐头刀短,但是横切的刃口非常的锋利,那怪物中刀后,身体狂扭,我再也抱不住,被甩的撞出水面,但是有了上次的教训,我的手死死拽住罐头刀不放,刀的倒钩卡在他身体里,它一用力气往前,整个儿在它身上拉了一条大口子。

等我再探出头来的时候,绿色的水面上已经全是红色的鲜血,两种颜色混合在一起,非常的恶心,我将手抬出水面,发现罐头刀已经卷了起来,卷起的刃口翻上来,切进了我被水泡的发白的手指,只是刚才太过投入,一点也没有察觉。

现在也管不了这么多了,我定了定神,刚向前一步,突然一只巨大的鱼头冲出了水面,我只看到一口密集的獠牙向我的脑袋扑来。情急之下一个后仰,那鱼就扑在了我的身上。一下把我压到了水下。

我在水里拼命的挣扎,想抓住什么东西,这个时候,一个人抓住了我的手,猛的将我拉出了水,我抬头一看,正是满身是血的老痒,在那里大喘粗气。

“怎么样?”我忙问:“你刚才给咬到什么地方了?”

他从水里拿出半只背包,苦笑了一声,我松了口气,看样子这里的地方太过狭窄,这条鱼只能攻击我们胸口的位置,这真是不幸中的大幸。

水里一片浑浊,那条大鱼显然吃痛,不停的在水里翻腾,不时还撞到一边的石壁,我们戒备着,可是不久,它却在不远处肚皮朝天的浮了上来,两只鳍还在不停的抖动,但看来已经不行了。

我等了一段时间,看它确实僵硬了,才大着胆子向它游了过去。

这鱼起码有两米半长,脑袋很大,长着一张脸盆一样大的嘴巴,里面全是细小有倒钩的牙齿,最奇怪的,这鱼的脑门上还有着很奇怪的花纹,一把匕首没柄插在那里,不知道是老痒插的还是我插的。

我这个时候已经看出,这是条哲罗鲑,淡水鱼算它最狠,如果说起这种品种,那这条鱼还算是小的,只不过这种只在冰冷水系里的鱼,怎么会钻到这个地方来,如何钻进来的?

正疑惑着,就听老痒叫道:“快看,那里有台阶。”

刚才一团混战,已经不知道自己给那鱼带到了什么地方,看样子已经进入了这个石道的深处,我转头看去,一边的水下,有几道简陋的台阶一直延生出水面,上面有一片高地。手电扫过,可以看到一些壁画。

我们浑身又冷又痒,急需休整,两个人商量了一下,决定先到没水的地方,把伤口处理一下。

老痒冻的厉害,也不和我多说,拎住这鱼的腮片,就往里面拖去。我看了奇怪,问他还要这鱼干什么?他说道:“我们包里那些装备给它吞下去,那可了不得,我们还指望这些东西发财呢,怎么样也要弄出来。”

我听了只摇头,拿他没办法,只好帮着将鱼向前推去,这种几乎笔直的台阶,我先爬了上去,上面是一个用木头撑起来的石室,一边还有一条通往其他地方的石道,里面一片漆黑,不过这个地方倒是比较宽敞,应该是暂时堆放采出来的石料和废石用,那些支持的木头已经稀疏烂光,四周的壁画非常简单,倾向于抽象的风格,我浑身难受,没心思去仔细看。

我们将衣服全部脱光,用角落里的烂木头堆起一个火堆,开始烘烤衣服,老痒着急他的装备,光着身子就去刨那鱼腹,边切还边对我说:“这鱼这么大,就这么扔了浪费,等一下我们割点肉出去,吃吃看怎么样?”

我从老痒的半只包里翻出一些药品来,先给自己的手指消了毒,然后用创口贴包好,说道:“你自己吃吧,这水太脏,也不知道这鱼是从哪里来的,吃什么长大的,想想就不保险。”

老痒这个时候已经将大鱼的胃刨了出来,一刀划破胃囊,顿时一股恶臭扑面而来,简直能把我熏死过去,我的脑袋不由自主的转过去一看,只见一团稀烂的东西从它的胃里淌了出来,其中一个圆圆的东西滚了几下,到了我的面前。

我一看,“阿哦”了一声。

那竟然是一个人头。

\chapter{人头}

我们进山以来,除了那向导大爷给的几个野味,吃的都是干巴巴的干粮,那几个野味又没吃上几口,就给猴子给搅和了,现在谗劲还没过去,老痒说鱼肉的时候,我嘴上说不吃,其实心里已经有点心动,脑子还幻想出在海上吃鱼头火锅的情景。

可这该死的一刀,就把我的美梦破灭了,我看着那血淋淋粘满胃酸的人头,和鱼头火锅的情景重叠在一起,一股反胃直翻上喉咙,几乎就现喷了出来。

老痒平时胆子颇大,说起死人,没一千也见过八百,但看到这副情景,却也脸色发白,半天没有缓过气来。

强忍住恶心,我用匕首将人头反转过来,发现他脸上的皮肤略微有点溃烂,但是整个头还是比较完整,应该是刚吃下去不久,这鱼在吞吃人头的时候,大概咀嚼了几下,使的头骨下鄂的形状有点变形,面貌已经无法用语言来形容了,无法判断到底是什么人。

这人进这鱼胃并没有多少时间,就是说他是刚死不久。

我一手捂住鼻子,一手用匕首将从鱼胃里淌出来的东西一样一样拨开,想看看这人的其他部分在什么地方,很快,我找到了手和一些肉块,都已经有一定程度的腐蚀,没有可以看出这人身份的地方。

我继续翻了几下,找到了被它吞下去的我们的背包,里面的东西已经和胃里事物残渣混合在了一起,除了那些实在无法放弃的,其他的我全部都拨到一边。那些干粮虽然都用塑料纸包的好好的,但是我实在无法说服自己去吃他们。

忽然,我看到在一团糊状物中,有一块黑色的东西,没等我把它全部拨出来,老痒已经叫了起来:“操,是把‘拍子撩’。”

我不知道什么是拍子撩,猜测肯定又是他从牢里学的什么歪话,拨出来一看,是一把土制的手枪,这种枪真的非常土。就是把小口径双管猎枪的长枪管给锯了,然后把枪托修成手枪的样子。有两个枪管,能打两次,但是不能自己退弹壳,得象装子弹一样,将空弹壳拿出来,所以用来打那些没有攻击力的小野兽还行,要是碰上大型野兽,一枪没打死的话,等你上完子弹开第二枪,脖子早就被咬断了。另外,这枪近距离威力惊人,但是如果超过二十米就连狗都打不死,其实用性和正式手枪根本不能比。

我将枪拨出来,在地上把上面的东西蹭没了,才拿出来,拨开枪管子一看,里面有两发猎枪子弹,在手枪枪管下面还一个装子弹的铁匣子,里面大概有八发子弹,四蓝四红,什么类型的不知道。

这人可能是来山里偷猎的,偶然发现了这洞,想进来看看,结果喂了鱼了。这枪可能是鱼丝咬人肉的时候一起吞下去的,人倒霉就是这样,谁能想到这地方会有条这么大的食肉鱼。

枪是好东西,紧急时候可以用来保命,只是子弹太少了。老痒把我们那些装备掏出来后,又在鱼胃里捣鼓了几下,但是却没有更多的发现,我看了看鱼的身上,只见除了我们造成的那几个伤口外,另外还有一些细小的弹孔,这鱼在袭击我们前,已经受了伤,只不过它中的是铁沙弹,杀伤力太小,并没有致命。

老痒看这鱼觉得奇怪,问我道:“老吴,你说这地方怎么会有这种杀人鱼,会不会是有人养在这里的?”

我对他道:“不是,我看是这石道的水面下面,还有其他的水道,连到附近的地下河,而这里的地下河通常又连着嘉陵江,这鱼肯定是从江里游过来的。”

老痒道:“不对啊,几千年没潜水设备,他们怎么去挖这些水下的水道啊?”

我看老痒挺感兴趣,解释道:“那不是挖的,我估计是因为事故形成的。”

学建筑的时候,有一门自然力学讲地质结构。里面提过岩石山里经常有太古时代造山运动时候形成的中空地带,叫做岩脉,如果岩脉和山溪想连,就有可能形成山内部的水系。打矿的一但打到这里。就有可能出现巨大的事故。小则冲毁几个矿道。大则淹掉整个工作面。

这里是采石洞,一般不会设排水的坑道,这里给淹成这样。可能就是因为发生了这样的事故。

不过,由此我们也可以推断出,采石洞的规模可能比我们看到的要大的多,不过因为淹在水下,所以看不出来,用了这么多的石料,我们要去的古墓必然规模也不会小到哪里去。

我们把鱼的尸体和人头都推回水里,但是这味道闻着实在太难受,我们也休息了没多久,看衣服差不多干了,我们重新穿带整齐,将所有必须的东西装进口袋里,就匆忙动身。

老痒打起手电,在前面开路,两人一前一后,径直走进后面的石道中。

里面同样一片漆黑,石俑和动物俑横倒在石道上,两边的洞墙上坑坑洼洼,裂缝横生,有时候还能看到浮雕石刻的半成品。

这些东西个头都很大,我不禁在想,这里采出的石料,是如何运到古墓中去的。

按照齐老爷子给我的资料,蛇国的疆域并不大,大多数都是山区,狩猎是主要生活方式,生产力比较落后,应该不具备长途运送石料这样的实力。为了方便运送,古墓应该是在比较靠近的地方才对。

刚才我们进来的那洞,是盗墓贼炸出来的,那就是说,这采石洞的出口应该在另一边,难不成一路过去,这样就能到达地宫的入口?

不过也有不少人为了隐藏自己墓地的位置,故意在很远准备材料,那就是我们不能控制的了。

我们往里走了有半个小时工夫,前后都已经一片漆黑,老痒的手电电池耗尽,开始闪烁,我感觉累了,就招呼停下来换电池,顺便抽个烟提提神。

我们坐到地上,把手电放在地上,照着那些逼真的石人。老痒就问我道:“这些个石像,一个个雕的这么逼真,实在悚的慌,你说这是什么朝代的东西,我怎么就一点头绪都没有?”

我和他一样,也是一头雾水,中国的泥石雕刻历史渊远流长,和古印度,藏文化有过长时间的融合过程,但是以写实为主要表现手段的雕刻手法,在我记忆里只出现过一次,那就是秦始皇的兵马俑,可是这里的石像和兵马俑又是完全不同,实在是一个异类。

不过,石俑身上都有双身蛇纹的显著特征,肯定是属于古厍族文化范畴,不管这个矿洞是不是属于我们要去的那个古墓的,我们现在已经进入古蛇国的领域,是绝没有错的了。

老痒话很多,一边抽烟一边问这问那,我给问疲了,就让他别什么事情都问我,我又不是考古的,咱们拿了东西就走,研究这些事情,让他们那些老教授去做。

换好电池没走几步,前面出现了手电光线的反射,似乎是到底了,我们跑上前去,果然,前面是一面石壁,石道的尽头是一个不大的石室,里面倒着不少破碎的无头石人俑,四周有石灯,石室的中间,放着一只石棺。

石棺很大,棺盖上面的雕着一条双身蛇,两条蛇身分别缠绕住棺材的两边,雕刻的非常精制,但是蛇尾巴的地方明显还没有完成,只雕出了一个大概。

手电照上去,棺材的石料显现出凝脂一样半透明的白色,棺盖没有合上,露出了一条手臂粗细的缝。整个棺材放在棺床上,四周再没有任何的东西。

看来是一个陪葬棺,可能是入殓的时候多余出来的,或者雕刻来备用的,给废弃在这里。

怎么这条石道这么长,只通到这地方,我纳闷起来,不可能啊,这里明显是一个堆次品的地方,没有出口,那这石道两头都是封闭的,难道运输石料的道路,是在刚才通过的水道水位以下?或者说是这个石室里有秘道?

如果入口在水下,那可就糟糕了,我心里暗道。

这个石室里没有什么没有什么奇怪的东西后,我和老痒四处看了看,最后围到了那石棺的一边。

老痒第一次见棺材,很希奇,围着转了两圈,问我:“里面会不会有粽子?”

我想也没想,道:“不会,没听说过先入殓再雕棺材的,这应该是空棺。”

老痒把眼睛凑到棺材盖的缝隙处,用手电照了照,道:“但是里面好象装了是什么东西?不信你过来看。”

我走到他一边,远远的一看,果然,从棺材的缝隙里看下去,有一个黑色的影子躺在里面。可是是什么,还真看出来。

老痒吹开棺材盖上的灰尘,敲了敲,想把手电伸进棺材的缝隙里去照,但是我们买的那手电头太大了,试了半天插不进去,他问道:“要不要打开看一下?”

我心里感觉有点异样,以前开棺材的时候边上总有几个老手,这一次就我一个人,没什么自信,摇头:“这事情不对劲,我感觉不好,别贸然打开。”

话还没说完,老痒忽然往后一缩,退了好几步,一屁股坐到了地方,手电都脱手滚了开去。

我给他吓了一跳,刚想问他干什么,忽然手上一凉,低头一看,一只干枯惨白的手不知道什么时候从棺盖的缝隙里伸了出来,正抓在我的手腕上。

\chapter{地下河}

我顿时头皮发乍,起了一身的筛子,发了疯一样想把自己的手抽回来。可是那枯手力气极大,不仅没办法脱手,还直把我往棺材里拉去。

我吓得几乎失去理智,混乱中掏出了拍子撩,想用它来把那只尸手打断。可没等我瞄准,后面突然一阵混乱,把我拿枪的手猛地给扭住了。

我当时不知道扭住我手的是什么东西,一边大吼一边挣扎,不知道怎么的,竟然把那只尸手甩掉了,然后一脚蹬在石棺上,连着我后面的东西全部摔了个人仰马翻。

在地上打了两个滚,我已经知道袭击我的是人,一下子胆子大起来,一个翻身跳了起来,甩手就准备放一枪。

可没等我看清楚面前到底是什么人,就听嘣的一声,不知道哪里刮来一道劲风,我的后脑给人狠狠敲了一下,我眼一黑,直接给打蒙了过去。

我被砸得扑倒在地,这时至少有两个人上来架住我的手,将我提了起来,押到棺材边上。回头一看,老痒也给制住了,已经五花大绑,按在地上。

我身后那人用我的皮带将我的手绑住,把我也推倒在地上,然后用枪顶了顶我的头,这时候我才看到他们的样子,这几个人,竟然是我们在西安路边摊子上碰到的那几个家伙。

这些人怎么会也在这里?我心里惊讶到了极点。难不成,他们真和老痒说的,一直在留意我们,跟到了这里?

这下糟糕了,这几个是亡命之徒,落入他们的手里恐怕凶多吉少,这种地方简直是杀人的最佳地点,尸体恐怕几百年都不会被发现。

那几个人把我们绑好后,丢到一边,也不来打也不来杀,而是去推我们刚才看的那石棺盖。我和老痒一看,看到那干枯的手臂还挂在棺材外面呢,不由得面如土色,吓得大叫:“你们干什么,里面那是只粽子!放出来我们都要倒霉!”

那几个人一听,一愣,马上哄堂大笑,一个年轻人说道:“什么粽子?你好好看里面是什么!”

说着用力一推棺盖,在我和老痒的大叫中,棺材盖子轰隆一声给推到了一边,随即,一个干瘦农民模样的老头从棺材里坐了起来。

我一看,我靠,这不是那个泰叔吗?他怎么会坐在棺材里面?随即我马上就明白了,心里真想抽自己的一嘴巴,我操,竟然给人耍了!

泰叔站起来,将他那只白得犹如死人一样、布满干枯皱纹的鬼手收进衣服里,然后翻出棺材,来到我们面前。

我看着他的手,指甲是黄色的,又长又尖,忽然我想起小时候爷爷的一个朋友,这人的脚给粽子抓过一下,流了十几天脓才好,但是脚从此就萎缩,形容枯槁,和那泰叔的手看上去一模一样。

我心里暗道,难不成这泰叔手这个样子,也是给粽子抓伤所致?后悔刚才自己怎么就没想到,要是刚才没给吓成这样,我们就没这么容易给逮住了。

泰叔打量了我们几眼,也不说话,只是点起一支烟,用他们那里的方言和边上几个人说了几句话,那几个人看了看我们,都点了点头。

我以为他们要对我们不利了,不由全身戒备,没想到他们却不来理我们,而是围到了棺材的边上。那泰叔改用普通话,对一个人道:“王老板,根据李老板当时说的八卦方位,这个地方就是当年陵墓地下水道的入口,但是这里啥也没有,这是怎么回事?”

一个有点胖的中年人,吃力地蹲下来,拿出一本簿子看了看,说道:“不会错嘛,就是这个地方啦,肯定是封墓的时候,把入口藏起来了,暗门应该就在这个房间里。”

泰叔看了看四周,又问其中另一个人:“凉师爷,你对这有研究,你怎么看?”

那个人躲在黑暗里,我看不到他的样子,只听一个颇年轻的声音说道:“李老板的地图我看过,应该是不会错的,刚才我也随便看了看,如果要说有暗门,那其他地方是不会有了,肯定是在这棺材下面的棺床。”

他们低下头来,看着石棺下的突起部分,老泰拿枪柄敲了敲,说道:“那怎么打开?”

凉师爷想了想,摇了摇头:“不晓得,推开来看看。”

泰叔站了起来,走到那年轻人边上。两个人肩膀抵着棺材,用力一推,喀喇一声,棺材挪了一点位置,下面的棺床上,露出了一个黑色的缝隙。

其他人也上去帮忙,几个人用力推了几下,空的棺材滑下一半,一个一米见宽的入口呈现在我们面前。

我伸长脖子一看,里边黑幽幽一片,似乎有一道十分陡峭的石阶一直通到下面。我闻到一股古怪的气味从下面弥漫了上来,有点熟悉,但是想不起是什么。

那年轻人用手电照了照,就想探头下去,被泰叔拦住了,他用下巴指了指我,用他们当地的语言说了句话。那年轻人点了点,过来把我拉到洞边,将我的双手双脚解开,然后一把把我推到洞里,用枪指了指我的头,让我下去。

我一看,知道他们刚才没杀我们,原来是有这一层估计,这里的暗道他们没走过,怕有机关,想拿我们去趟雷。想起老痒当时求我的时候,说这一路就当旅游,心里顿时后悔得不得了,心说我怎么就听了他了,这下子好了,下面的楼梯上十有八九会有机关,死定了。

我活动活动了手,想着要不就和他们拼了,反正横竖是死,就算下到暗道里没机关,以后趟雷的机会还多着呢,总不会次次这么走运,和他们拼了说不定还有一线生机。这时候老痒却朝我打了个眼色,轻声说:“没事情,尽管下去。”

我心里纳闷,他又没走过,怎么知道没事情,不过看他那神情,好像是胸有成足,一下子也摸不着他有什么打算,于是把手电绑到手上,双手撑住一边,小心翼翼地先用脚探了下去。

我深呼吸了一口,先用手电住下一照,发现这是个几乎笔直的走道,深得看不到底,四周泛绿的石壁上不知道为什么非常的潮湿,手按上去有点打滑。可是下面又没水,不知道这湿气是从哪里来的。

我想下去,那泰叔拍了拍我的头,递给我一只哨子,说道:“到了底,就吹一下,半个小时要是听不到声音,俺就宰了你哥们。”

我知道他是怕我自己跑了,心里冷笑一声,把哨子接了过来,就缩头下了地道里。

这种几乎笔直的石阶爬起来十分吃力,他们开凿的时候并不仔细,有些浅有些深,大部分只能踩住小半只脚,我下去了十几步,已经开始喘气,脚尖开始痛起来。抬头望去,上面的石门已经变成一个小小的方形光点,四周的黑暗像墨汁一样挤过来,我看到几个隐约的影子在上面闪动着,显然他们不停地在往我这边看。

一开始我还担心这些石阶会设有机关,所以走得特别小心,但是越往下,我发现这石道修得越粗糙,石头都是整块整块的,这样的做工,肯定不会有机关。

走着走着,矿道走势一改,逐渐开始出现角度,阶梯也好走起来,我看到这一段的岩石明显变成了红褐色,照上去还有很多细小的反射。

这种石头大概是花岗石,里面有一些云母,非常的坚硬,他们将矿道改向,大概是想避过这一条花岗石带。那这里应该已经是大山的内部了。

不知道什么时候起,矿道的下面传来水声,经过几个弯后,那水声大了起来,听上去如万马奔腾一样,水流十分的湍急。

我看了看表,自己已经走了快二十分钟,感觉再往里去,哨子的声音可能就传不到上面了,于是拿出哨子先吹了几声。

声音一路盘旋上去,很快,上面也传来一声哨音回音。

我继续往下,前面地矿道边宽阔起来,出口很快出现在视野里,前面吹来了一股强风,几乎把我吹得跌倒。我向下跑了几步,忽然耳边一声轰鸣,人已经走出暗道,来到了一处河滩之上,同时,一条奔腾的地下河出现在了我的眼前。

这条地下河大概有一个篮球场那么宽,洞顶有大概十米多高,左右两边无限延伸开去,不知道通到什么地方。山洞的顶上没有钟乳,但是四周的石头经过多年的冲刷,变得很圆滑,我看着这洞的规模,知道不是人工开凿出来的。

水流非常湍急,刚才我在上面听到的巨大水声,就是因为这里的洞穴结构好像一个扩音器,将流水的声音扩大,我往中间走了走,发现水温颇高,有点下不去脚,而且越往前走水越深,几步就没到我的膝盖了,于是赶紧退了回去。

这里应该是一条岩脉,就像人体内的血管一样,是大山的血管。我往两边看了一下,发现两边地下河道似乎呈现出收缩的趋势,宽度逐渐变小,在左边的那条河道两边的岩壁上,还拉着很多铁链。

正在奇怪的时候,那年轻人已经怪叫着从暗道里走了出来,一脚踩在水里,大叫:“我操,这么烫!”

我回头看去,看到另一个年轻人跟着他后边走出来,这人带着副眼镜,看上去文绉绉的,应该就是那个凉师爷,他走近的时候,我才发现其实这人也上了点年纪了,并没有远看那么年轻。第三个出来的是老痒,后面跟着一个有点发福的中年人,然后就是泰叔,我以为后边应该还有一个人,却发现没人跟着了,心里纳闷,进山的时候,他们不是五个人的吗?

他们几个全部都打起手电,几条光柱在岩脉里来回扫荡,那凉师爷低叫了一声:“真是鬼斧神工,通往陵墓的神道竟然是条地下河,要不是亲眼看到,我还真不信。”

那年轻人往水里走了几步,皱了皱眉头退了回来,对那几个人说道:“他娘的还挺深,泰叔,这里难走,不好趟。”

泰叔看了一眼王老板,问道:“王老板,现在该怎么走,你那宝贝地图上有没有写?”

王老板翻着他的本子,说道:“地图上说,他们上次来探陵,曾在水下设下两条铁锁,一直摸着那铁锁,就能到达地宫的入口!”

手电都照向水里,果然,一条大概手腕粗的乌黑铁链横在水底,泰叔将它拉出了水,掂量了一下,叫道:“他妈的,还真的有。”

年轻人走过去拉了几下,拉不动,有点不安地看了一眼前面,说道:“泰叔,这样走水路,恐怕不太妥当吧,刚才李老板死得那么惨,要是再碰到那种鱼,我们全部都得交待了啊。”

凉师爷摸了摸水,说道:“没事,这里水这么热,底下肯定有温泉口,绝对不会有鱼,有也焖熟了,二麻子你想太多了。”

二麻子咧了咧嘴巴,似乎不太相信,问道:“真的?”

凉师爷拍了拍他的肩膀,刚想说什么,突然二麻子背后的水里炸起了一个巨大的浪花,几乎是一瞬间,我们就被冲得摔进水里,浑身湿透。我慌乱间把手电转回去一看,只见一道水柱冲出水面,碰到洞顶,滚烫的水变成雨一样地洒落下来。

凉师爷吓得脸色惨白,坐在水里直发抖,不知道有没有尿裤子。那泰叔到底是见过风浪的人,站起的时候一手已经将枪拔了出来,对着凉师爷大叫:“他妈的这是啥玩意!”

\chapter{黄泉的瀑布}

地下河水水流湍急,水温极高,原来以为里面肯定没有生物,没想到话还没凉,水里突然冲出一股黄色的水柱,直腾上洞顶,将所有人全部冲倒在浅滩上。

混乱之下我也没看清直接给水柱冲到的二麻子情况如何,只听到泰叔大声地问凉师爷水里是什么东西,后者给吓得屁滚尿流,连话也说不出来,根本无法回答他。我转头去看,也只看到一大片水花,水底下到底有什么东西,连个形状也分辨不出来。

那水柱子冲上洞顶片刻也不见衰落,反而有越来越凶猛的势头,让我想起海里的鲸鱼,可这山沟沟里怎么可能会有鲸鱼,要真能碰上这么离谱的事情我也不想活了,可除了鲸鱼,什么东西还能扑腾出这么大的动静?我转念一想,难不成就是传说中的那种有二十多米长、头如解放卡车的成年哲罗鲑?心里直叫命苦,这年头菩萨闭眼,什么妖魔鬼怪都出来溜达,这斗恐怕是倒不成了。

这时候二麻子突然扑腾了几下从水里钻了出来,不知道为何浑身通红,才走了几步就跌倒在水里,一动也不动。泰叔不知道发生了什么事,狠狠地踢了我一脚,让我去把他拉回来。

我心中暗骂这老家伙不是东西,可是后脊梁有枪顶着也没有办法,只好硬着头皮冲进水花里,水柱喷上洞顶的水正下雨一样淋下来,我一给淋就发现不对,这水烫得离谱,沾到身上就是一个水疱,慌忙间只有拉起衣服遮挡,另一只手去拉那二麻子。

可是当手一碰到二麻子的身体,我就给烫得一缩,心中骇然,他娘的这孙子已经熟了,没救了。

这时候,忽然又是一声巨响,水柱子那里又喷出一道黄气,我一看不对,这他娘的绝对不是鱼,任何生物在这么高温度的水里活动,早熬成老汤了。

老痒冲我大叫:“你他娘的发什么愣呢,快潜到水里去,这是间歇性的热喷泉,烫死人不偿命的。”

这水柱越来越大,滚烫的水开始像瓢泼大雨一样洒下来,我忙猫着腰钻进地下河里,其余的人被越来越大的沸水雨烫得跟杀猪似的,一看我往水里逃,也纷纷扎猛子跟了过来。

喷泉水和地下河水混合在一起,河水的温度也高了很多,一猛子扎下去,简直就是游进了砂锅里,全身都烧了起来。我游出几米探出头来,回头一看,泉眼四周的水已经沸腾了起来,热流迅速蔓延,我能看到几乎整个河面都开始冒出水气,再不找个地方出水,就要和那二麻子一样的下场了。

这时候再返回进来时的矿道已经不可能了,那边的水是温度最高的,几乎已经沸腾了起来,只有硬着头皮顺着地下水道去了。我看着水流的方向,心里后悔,刚才下水的时候应该选择逆流的方向,这样水流会把热水带到相反的方向,现在我们和热水一起顺势而下,在水中和水比快,简直是开玩笑。

不过事到如今,也没有其他方法,难道就在这里等死吗?我对老痒打了个招呼,一马当先游在最前面,后面几个全部跟着我游了过去。

借着水流的速度,我一下子就冲进去好几百米,感觉上水温已经不再上升,当下松了一口气,回头仰泳同时拿电筒一照,看见老痒正在对我拼命地招手,对着我大叫:“停下!停!前面——”

他话没说完,突然就给什么东西撞了一下,嘴巴给压进了水里,后面几个字没听到,这个时候我已经听到身后传来了轰鸣的水声,转头一照,只见前面不远处水花翻腾,赫然是一个大的断崖,黄色的水流从断崖处倾斜而下,悬崖的下方是打雷一样的轰鸣,这肯定是一个巨大的瀑布。

我一下子就麻瓜了,这下子不得了,给冲下去那是死无全尸啊。老痒这个时候又探出头来,大叫:“靠边!靠边!”我这才反应过来,赶紧游向水道边缘,用力扒住洞壁,一连给水流带出去三四米才将自己停了下来,刚想松一口气,突然那个凉师爷就一边叫着救命一边从后面撞了上来,一下子把我撞了出去,两个人在水里滚成一团。

我再探出头来的时候已经给冲到瀑布边上了,当下再没有可以应变的时间和办法,我下意识地伸手乱抓,突然就给我抓到一根铁链,我一咬牙扑过去死抱住铁链,终于在瀑布的边缘停住了身体,向下望去,双脚已经荡在悬崖下面,下面水声隆隆,漆黑一片,不知道有多高。

正庆幸自己命大,谁知道下面有人推开我的脚,我用手电一照,原来凉师爷正挂在另一根铁链上,我的脚正踩在他头上。我用力踹了他两脚,把他踹到一边,往边上一摸,发现四周的水下横着大量的铁链条,交错在一起,好像一条栏杆一样将从上游冲下来的东西拦住,只不过现在有些铁链已经断了,从瀑布上挂了下去,出现了不少缺口。

老痒漂到我一边,我一把抓住他的手,将他拉到我身边,同时泰叔和那个胖老板也全部在另一边抓住铁链停了下来。二麻子的尸体从我们身边漂过,在铁链上打了个转,卡在了两条铁链之间,老痒伸过手去,将他腰里的拍字撩和手枪全拿了过来。

我看他拿到枪来,努力伸出水面就想去打泰叔,忙一把拉住他,骂道:“你他娘的想什么呢,枪管里有水,你想爆膛吗!”

老痒大叫:“现在不干掉他们,就没机会了。”

我将他扯回来,大叫道:“你现在还有心思想这个,快看前面!”

他转头一看,前面一片蒸汽腾腾,沸水已经到了,经过几百米的冷却,这水丝毫不见降温,我在几十米外已经能感觉到热浪冲了过来。老痒看着那水,哭道:“他妈的,没想到我吃了这么久涮羊肉,今天自己也要给涮一回了。”

我不想就这么送命,急得直咬牙,心说怎么办?现在唯一生存的机会,就是顺着瀑布冲下去,但是下面什么环境根本不知道,要是太高,和跳楼没区别啊。

挂在我下面的那个凉师爷突然朝我叫道:“我有办法!”

我问道:“什么办法?快说!”

“你先把我拉上去!”那凉师爷大叫,“拉上去我再告诉你,不然我们一起死!”

我赶紧探手下去,将他拉上来,一把揪住他的领子:“快说!”

他紧紧抱着铁链,看了一眼汹涌而来的沸水,不由咽了一口唾沫:“烫水是漂在冷水上头的,我们潜水下去,等上头的烫水漂过去了,如果能闭气熬得过那段时间就还有一线生机!”

我一听,也没工夫去想可不可行了,一把将他又推回到下面,然后自己一个猛子就扎进了水里,拉着铁链条一直往下。

这地下河非常深,我一直潜到二米左右,感觉四周的温度低了很多,当下屏气宁神,准备等上面的热流通过。

这个时候,我的手突然碰到一团东西,好像有什么挂在铁链上面,我拿手电一照,突然看见一张极度狰狞的脸出现在铁链后面,吓得我一口气没憋住差点把水吸进肺里去。

水下的铁链上缠着一具腐烂的尸体,身上的肉已经泡烂了,两只眼洞直勾勾地瞪着我,看上去分外的狰狞。我仔细一看,发现他穿的是一件冬天的登山服,身后还背着一只背包。

看样子是个登山者,怎么会给冲到这里了?我用嘴巴咬住手电(登山战术手电后部有专门供身体其他部位使用的零件),在他的身上找了一下,发现了几支写生用的笔,又打开掉在铁链边上的背包,里面有画板和很多颜料,我心里明白了,这家伙应该是车上那黑导游说的,前几年在山里失踪的那几个写生的学生。

尸体应该是上游冲下来,卡在这里的,那这条地下河的上游应该是地上,这人也真是时运不济,死在了这里。

我翻了翻里面的东西,虽然没什么特别有用的东西,反正自己的背包也没了,有胜过无,便将这包背到自己身上。

这时候,四周水温一热,滚水已经到了,我马上就觉得浑身刺痛,咬紧牙齿,继续向下潜去。

滚烫的水一下子将我包围了,只是几秒钟的工夫,我马上就意识到凉师爷这方法行不通,这沸水的水量太大了,潜下去只不过是烫全熟和烫七成熟的区别,边上和我一起潜水下来的老痒给烫得抓了狂,用力踢了我一脚,指了指瀑布那边,意思是潜水没用,要烫死了,不如跳下去痛快!

我看了一眼尸体,心说哥们,老子马上就下去陪你了,突然一股更热的沸水涌来,我一咬牙,一松手,就顺着水流滚下了断崖。

\chapter{深潭}

我朦朦胧胧地睁开眼睛,发现自己躺在地上,四周一片漆黑,我摸了摸手腕,绑在上面的手电已经不知去向。

身下是一块冰冷的平板,边上好像还有流水的声音,这是什么地方?

我深深地呼吸了一口,记忆开始一点一点地出现在脑子里,瀑布,滚烫的泉水,铁链上的尸体,忽然一道白光闪过,刚才的情形浮现在我的脑子里。

我刚才好像是顺着水流直坠下断崖,然后就掉进了下面的水池里,那水冰凉冰凉的,和滚烫的泉水有着天壤之别,入水的那一刹那,我觉得耳朵突然一静,然后就什么都不记得了。估计是因为落水的时候冲撞到了什么东西,把自己磕晕过去了,从几十米高空摔到水里,如果姿势不对,和摔在水泥板子上是没有区别的。

我摸了摸身子,还是湿的,难道我掉下瀑布之后,给下面的水流继续冲到了这里?还是干脆我已经死了,来到了阴曹地府?

我试着站起来,才微抬起头来,突然咚的一声,脑袋撞在了什么东西上,疼得我眼冒金星,忙用手一摸,上面好像是一块平板,心里奇怪,怎么这里这么矮?难道我给冲到了什么岩石的缝隙里或者石头下面了?

我四处摸了一下,发现并不是这样,自己的四周围一尺内都是粗糙的木板,敲了敲,后面是空心的。这样小的空间,我只能躺着转身,连抬个头或者伸个懒腰都不行。

我撑了撑上面,想看看这些木板的厚度,却发现上面的木板可以活动,用手一撑,嘣的一声,黑暗中突然出现了一道光。我顶起膝盖,轻轻地将上面的木板移开,坐起身子来,一看外面,不由一愣。

这里是一个汉白玉的石室,四个角落里都点着火把,将周围照得通亮,我看了看头上的宝顶,是两条互相缠绕的蟒蛇,而我竟然是坐在一只棺材里面,棺材的盖子被我翻在一边。

靠!这是什么地方?谁把我放到棺材里去了?

我走出棺材,观察四周,心里越来越奇怪,汉白玉的材质,雕刻着蟒蛇的宝顶,非常熟悉,想了想,马上会意,这里和海底墓的墓室几乎一样。

不会吧?

四处走动了一圈,发现古怪的事情还不止这么点。我身上的衣服不知道什么时候给人换了,换成了一件类似于潜水服的橡胶衣服,就是那种八十年代潜水员穿的衣服。心里更加奇怪了,这么老款式的衣服他娘的是哪里搞过来的?

我拔起墙角的火把,从这个墓室的门口走了出去。外面是一条甬道,我只是一看,就“啊”了一声,我的天啊,汉白玉的直甬道,一直通到尽头的三道玉门,真的和海底墓一模一样!

这是怎么回事?我怎么回来了?我的头皮炸了起来,思维开始混乱起来,这里到底是一个很像海底墓穴的墓室,还是我根本就没有从海底墓出来过?我的天啊,到底是怎么回事?

我用力揉了揉自己的脸,把火把抬高,仔细地看了看这里的环境,想找出什么破绽来,如果是一个相似的墓室,肯定有什么东西会有区别的。

甬道之上架着一个木头架子,就像脚手架一样,上面铺着木板,成为通过甬道的一道简陋的天桥,可以防止触发机关,不知道是谁架在上面的。我小心翼翼地爬了上去,走到了甬道的对面,中间后殿的玉门里亮着火把的光芒,左右两个配殿一片漆黑。

这时,我想起了老痒,他在瀑布之上和我一起跳了下去,我掉落潭中,昏迷了那么久,到了这个莫名其妙的地方,他的处境怎么样了?

我一面想,一面向有火光传出来的门走过去。火光相当明亮,从玉门下面的门缝下透出来。来到门口,我听到门内有声响传出来,当我将耳朵贴在门上时,听到了一下咳嗽声。

接着,便是一个人的声音道:“怎么办?开不开棺材?”

另一个声音,听来十分为难:“三省说暂时不要动这里的东西,我们还是听他的吧。”

一听到这两个人的声音,我便怔了一怔,第一个讲话的人竟然是闷油瓶,第二个讲话的却听不出来。而且他们还提到了三叔,怎么,难道三叔在这个地方?

而令我惊讶的还在后面,我立时又听到了第三个人的声音,那人道:“吴三省现在还在睡觉呢,我们只是打开看一下,又有什么关系,我站在小张这一边。”

我不是十分听得懂他们的对话,但那第三个人,毫无疑问是个女人。

他们这几句话,是什么意思呢?听起来,好像是闷油瓶想开一个棺材,而另一个人因为三叔的警告而犹豫不决,这个时候有一个女人站出来支持了闷油瓶,我当下觉得一头雾水,怎么,闷油瓶已经找到了三叔了吗?

我一面想着,一面趴到门缝里,想看看里面说话的是谁,可惜门缝里所能看到的范围有限,我只看到一个女人的背面,穿着和我一样颜色的潜水服,身材很娇小,梳着一条大辫子。

这时,我听到了第四个声音说道:“齐羽怎么办?这小子也真能耍,不知道跑到什么地方去了,难道我们就将他丢在这里吗?”

我听得他这样说,不禁陡地一呆,齐羽,好像也是三叔的笔记里面,写在前面的名单里的人之一,难怪有点熟悉,等等,不对。

我忽然感觉到非常的不自在。齐羽。这个名字不是熟悉这么简单,好像经常听到,我心里有一种很特别的感觉。

这个时候,门缝里的那个女人移了一步,让出了一个空间,我看到闷油瓶子正站在一只黑色的棺材边上,手里拿着撬杆子,犹豫着什么,然后另一个女人走进了我的视野。我一看到她的脸,惊讶得几乎将手里的火把掉落到了地上。

这人,不是文锦吗?老天,怎么回事?我虽然没见过她的真人,但是三叔有很多她的照片,过年看老照片的时候,我经常能看到,所以一眼就认了出来,绝对没错。

我心里的疑惑到了极点,几乎就想推门进去,就在这个时候,又有一个陌生的男声说:“这座海底墓这么大,我们想要找到他谈何容易,我看还是算了,我们沿路刻下记号,他看到了自然会跟过来。小张,你不如动手吧。”

闷油瓶点点头,举起撬杠,就要下手,这个时候,突然从左边的配室里,传来了一阵轰鸣的水声,把我吓了一跳。

后殿里的人全部都转过头,那个男人问道:“什么声音,好像是从隔壁传来的!”

“走!去看看!”闷油瓶放下撬杆,向门口跑来,我一看不对,忙一个转身,躲进了右边的配室里,将火把放在地上踩灭,几乎是同时我就看到一行人跑出了后殿,冲进一边的玉门,接着就有一个女人惊叫道:“快看,这里有个水池!”

我躲在门后,心里极度诧异,刚才的情形,不就是张起灵为我描述的,他们在三叔睡着之后发生的事情,可是我怎么好像亲身经历一样,难道这是幻觉吗?还是干脆已经疯了?

四周重新归于黑暗,我深呼吸了几口,想去重新点燃火把,这个时候,又有一个人举着火折子出现在了视野里。那人从甬道上的天桥处走了下来,偷偷地躲到了左配室玉门的后面,往里面看了看,我稍微一看,就发现那是年轻时候的三叔,他好像非常懊恼,眉头皱得很紧。

过了一会儿,张起灵他们的声音逐渐变得远去,应该正在走入池里的盘旋楼梯。三叔吹熄了火折子,闪进了玉门内,我看得心惊肉跳,当下不管自己在幻觉还是做梦了,忙跟了上去,才贴上左配室的门,想偷偷往里看一看,忽然眼前一闪,三叔突然又从门里走了出来,一下子掐住了我的脖子,轻声说道:“原来是你跟着我!”说完突然手一紧,死死扣住了我的喉管。

情急之间,我想大叫:“三叔!我是你侄子啊!”可是怎么也叫不出口,只好拼命去掰他的手,想把他的手指掰开。

掰着掰着,我忽然听到有一个声音说道:“老吴,醒醒,你是不是做噩梦了?”

我打了一个激灵,突然眼前一黑,发现周围的东西突然都消失了,眼前朦胧中,老痒正在摇我。

原来是一个梦啊,我苦笑了一声,摸着自己的脖子坐起来,转头一看,发现自己正躺在一个石滩上,边上是一个水潭,瀑布的轰鸣声还是非常的响亮,但是我却看不到瀑布的位置,石滩上点着篝火。老痒正扶着我问我有没有事。

我摆手说没事情,然后用力捏了捏自己的鼻梁,心里非常奇怪,自己怎么会做了一个这么奇怪的梦,难道真的是日有所思,夜有所梦?

老痒把水壶递给我,我喝了一口水,看了看四周,嘶哑着问他:“这里是什么地方?我怎么了?”

老痒说道:“这里是瀑布下的水潭边缘,那瀑布就在那里,你刚才掉进水里的时候摔昏过去了,老子死死拽着你你才没给瀑布底下的乱流卷到水下去,你可真得谢谢我,我现在吃奶的力气都没了。”

我骂了一声,尝试着站起来,发现自己并没什么大碍,困难地走了几步,环顾四周。篝火的光照开去,我们待的石滩不大,呈现一个月牙形,一边的黑色水潭面积巨大,洞顶无数像腿粗的钟乳垂入水面,形成各种形状的石柱子,而水塘的四周有几个溶洞,大如象穴,小如鼠道,一个个深不见底,有的在水位上,有的在下,地下河水从里面注入流出,是个典型的喀斯特溶洞地下湖。

我知道这种地理环境,一般是在第四季冰川时期形成,要经过万年的逐渐扩张贯通才达到眼前的规模,这些岩洞的历史已经远长过人类的历史了,没想到天门山内,还有这样的地方。

浅滩上,除了我们之外,还有很多搁浅的树枝和杂物,老痒已经拖上来晾干,那堆篝火就是用这些东西烧起来的。水潭寒气逼人,如果没有这一团篝火,恐怕我已经冻毙了。

我想起泰叔他们,问老痒道:“其他几个人情况怎么样了?”

老痒道:“那几个龟儿子恐怕没我们这么走运,下水的时候就没看到他们,不知道有没有跳下来,我想要是他们跟我们一样,那不是给冲到其他地方去了,就是已经淹死了。”顿了顿又道,“不过我们的情况也不是很好,装备全没了,也不知道接下去该怎么走,你看这里分支岔路很多,这种洞又是出名的复杂,像迷宫一样,走起来非常棘手。”

我数了一下,我能看到的水面以上可以行走的溶洞大概就有七八个,黑暗中的就更多了,就说道:“刚才听那个广东胖子说,要通过这一段溶洞区域,必须找到那条古时候先民用来引路的铁链,这段铁链给隐没在水下,一端在密道的尽头,那另一段应该是在这水潭子里,如果能摸到,就能顺着它进入古墓的腹地了。”

老痒皱了皱眉想了想,说道:“说到铁链子,我想起个事。你知道,从上面掉下来那一刹那我是清醒的,一下子给插进水里最起码有六七米,那水底下他妈的全是我们刚才在石道里看到的石头人俑,那时候一晃眼的工夫,我好像真看到有一条铁链子横在水里,不过我告诉你,这铁链子不是通到这些个溶洞里去的,而是直插到瀑布下面的乱流中去的。”

我听了一愣,怎么可能,如果是这样,那通往古墓的入口,难道会是在这瀑布的后面,隐藏在急流之中?

我听着不远处瀑布的轰鸣,想起刚才我们坠落时候的情景,忽然心里灵光一闪,对老痒说道:“那就更没错了,而且要是我料想的不错,这座古墓也许不是修建在我们‘阳世’,而是隐藏在阴曹地府里……”

地狱!

老痒听我这么说,不明白我是什么意思,被我森然的口气所感染,他低声问道:“你胡说什么,怎么可能有这种事情?”

我摇头问道:“你还记得不记得村里老刘头和我们说起的,古时候那清朝道士说的黄泉瀑布和山中阴兵万马奔腾的传说?”

老痒点头道:“当然记得,说是天门山内有一道黄泉瀑布,这条瀑布就是阴阳两界的通道,当时你不是说这是迷信吗?”

我说道:“不,现在看来这不是迷信,是我们领会错了前人的意思。你回忆一下,刚才那条我们坠落的瀑布,因为水下温泉的关系,瀑布的水流呈现一种奇异的黄色,如果我料想的不错,那就是所谓的‘黄泉’瀑布。”

老痒想了想,说道:“像是有点像,可是不可能啊,只有曾经进过山内、看到过这里的人,才能知道瀑布的事情,但是这里环境复杂,不是一般人能进来的,”说到这里,他自己都已经意识到什么,叫道:“我操,那传说中清朝的风水先生,难不成是我们的同行?”

我点头同意,表扬道:“总算还有点推理能力。”

老痒兴奋起来,说道:“那就说得通了,你想那大部分的阴兵传说,也是清朝年间流传起来,会不会就是从这几个风水先生这里故意散播出去的?”

我点头:“那是大有这个可能,不过我们现在不用去理这一层,你再来回忆,那传说中还有一个说法,就是‘黄泉瀑布是阴阳两界的通道’,你想铁链通到瀑布之后,那瀑布后面必然有通往古墓的通道,如此说来,那古墓不正是在阴曹地府里的吗?”

老痒脸色难看起来,说道:“不会吧,你可别吓我,那里面要真是阴曹地府,那我们进去不死定了?”

我骂了他一声,说道:“我靠,你还真信,你想那几个风水先生既然是我们的同行,他们说的话就不能这么值得去了解。我觉得有两种可能,第一,这可能是当时的一句暗语,意思是,这条瀑布就是古墓和现实世界之间的通道;第二,或者是他们在瀑布后面的溶洞里看到了什么景象,让他们以为,他们来到了阴曹地府之中。”

我顿了顿,又道:“如果是第二,那我们可能要做好心理准备,这里面恐怕有着什么恐怖的景象……”

老痒沉默下来,好久才道:“要不我们还是回去算了……”

我摇摇头,好不容易来到这里,不进去看看太可惜了,而且,这瀑布如此巨大磅礴,怎么可能爬得上去,四周的溶洞又是九死一生的地方,现在只有到达古墓,然后再找寻办法出去,才是明智的选择。

老痒说服不了我,只得听从我,我们一边休息,一边开始检查装备,看看还有多少东西剩下了。

武器方面,我们身上还有拍子撩和老痒从二麻子那里弄来的托加列夫手枪,火力应该不成问题。其他方面,我翻开从水底那尸体上带下来的背包,从包里找到一些不知道有没有过期的罐头食物、白酒、水壶、手套,还有大量写生用笔和油画颜料。

老痒觉得这些都没用,想把它们扔了,我告诉他,白酒应该能御寒,颜料可以沿途做一下记号,手套也是有用处的,我们身无长物,还是都留着好了。

整顿再三,我发现最头疼的是,我们没有照明的工具,老痒的手电已经彻底没电了,我的也不知道早掉哪里去了,如果要举着火把去游泳,那真的糟糕了。

老痒把手枪往前面拉了拉,看了看四周的黑暗,说道:“只有一个办法了,咱们把这些柴堆起来,把火烧大了,然后借着火光游过去,这样就算游不到,也能再对着火光游回来,你说怎么样。”

我想了想,知道这是唯一可行的了,说道:“那行,咱们就先赌一把。”

我们脱下衣服,全部塞进包里,然后又用手套和木棍做了几个短火把,先放进背包的防水层里,然后燃起大火,暖了身体之后,跳进水里,开始顺着水声向瀑布游去。

水寒气逼人,游了几把我就觉得身上所有的热量一下子给吸走了,好在我最近有点发胖,不至于一下子就冻僵。

游了大概五分钟,水声逐渐变大,我和老痒停下来,一边踩水,一边听四周的动静,想判断好方向再游。

这个时候,在我们不远处,突然有什么东西在水面上划了一下,我们赶紧回头,却因为已经离开火堆太远,而看不清是什么。

老痒掏出托加列夫手枪,将枪管里的水甩干净,举得老高,警惕地看着四周,问道:“老吴,这里该不会有那种裸体鲑鱼吧?”

我背脊发寒,想到这里水域广阔,要是真有那种杀人鱼,我们肯定早死定了,刚想说没有,不远处却又传来一声水声,非常清晰,心里顿时不安起来,说道:“我不知道,不管怎么说,咱们快游,这种鱼害怕喧闹,我们越靠近瀑布越安全。”

我们两个马上甩动双臂,向瀑布继续游去,此时身后的火光越来越微弱,变成一个小点,我们只好硬着头皮在黑暗里一边呼应一边前进。

不一会儿,水流逐渐湍急,靠近了瀑布的水流领域,我们加大力度,速度却越来越慢,游泳开始艰难起来,我咬紧牙关想扑水到前面,几次都没有成功。

体力一点一点消耗,眼看就要给水流冲回去,我心急如焚,这时候老痒大叫,这样游是绝对游不过去的,前面是瀑布落下的水流激起的乱流区域,里面全是大大小小的漩涡,要想过去,必须贴着潭底,一点一点从乱流下面潜水过去。

说着他一个猛子翻进水里,一下子便消失了,我也跟着他潜了下去,顶着急流向前拼命前进了几米,下到水潭的底部,忽然看到前面的水底,竟然有一点模糊的白色亮光。

我认得那光亮,那是我的防水手电,我心里暗叫,这一千多块的东西果然够结实,现在还亮着,忙鼓起一股力气向它游去。

水潭的底部没有任何的生物,白色光源照到的地方,我看到大量的石俑整齐地摆在下面,上面已经腐烂成白骨的人头有的已经脱落,有的还牢牢长在石俑的脖子上,水潭的中间,似乎还有一座石台,上方的水中还似乎漂着一具白布裹着的尸体。

此时候我的手电对我吸引力最大,我看了几眼,便不去管这些东西,潜入石人中间,抱着石人固定身体,一步一步向手电靠拢。

就在我马上就要够到的时候,忽然后面一道水流冲了过来,我心知不妙,马上戒备,却没有想到会有东西用力撞我,眼前一团白影闪过,撞在我的手上,我抓着石头的人一下子吃不住力气,松了开去,人马上向上浮了起来。

我大叫不好,一刹那已经冲进了上方乱流的中心,前面顶着我的力道突然一下子改变了方向,将我向边上冲去,我哎呀一声一下子乱了方寸,直给水流卷得翻了几个跟头,再也控制不了自己的姿势。

混乱中我不知道被卷了多少个弯,只感觉好几次看到眼前有一道白色的影子闪过,却都没看清楚是什么。

意识迅速地模糊,我以为自己死定了,这时候,我的后背猛撞到一条东西上面,疼得我一下子清醒过来,忙回头一抓,正是老痒说的那条铁链。

再也顾不得我的手电了,我拉着铁链,用力向铁链的尽头爬去,几下便到了瀑布的正下方,但是我的气已经到了极限,只觉得一股千钧之力由头上倾泻下来,只把我向潭底压去,爬了还不到两米就再也动弹不得了。

老痒从后面赶了上来,一把抓住我的手,将我往上扯去,我们一边拉着锁链,一边乱蹬那些石人,终于冲过了瀑布下方的区域,我忽然感觉头上的压力一松,马上就浮出了水面,大口喘气,眼前直发晕。

四周漆黑,只听见老痒的喘息声,他咳嗽了几下,问我道:“没事情吧?我们好像已经过来了。”

我也咳嗽了几声当作回应,说道:“快点火照照,这水潭子不太对劲,这水里恐怕有不干净的东西。”

老痒“喀喀”打着打火机,想看四周的环境,可是周围水花太大了,火一点上就灭掉了。

我们摸索着向里游去,忽然,我又听到了瀑布外的那种水声,这一次离得非常近,听起来就好像是两三米外的地方,有什么东西游过一样。

“小心一点!”我想起在水里撞我的白影,顿时紧张起来,对老痒说道:“附近好像有什么东西……”话还没说完,我突然感觉到一只冰凉黏滑的手,一下搭到了我的肩膀上。

我顿时吓得大叫,心说到底是什么东西,难不成是水下面的石头人活了?本能地在水里一个翻滚,一脚就踢在后面那东西的身上,将他踹了开去,然后自己猛又探出水来,对老痒大叫:“妈的,水下面有鬼!操家伙,快!”

老痒已经打起了打火机,给我吓了一跳,忙转来照我,不照还好,一照之下,我们两个全部头皮发麻,几乎吓死过去。只见我身后的水面下,浮出来一个惨白的人头,正看着我们,露出了一个狰狞的表情。

我们吓得向后蹬了好几下,老痒慌乱中想掏枪出来,可是怎么也拔不出来。

那人头翻起了白眼,嘴巴张了张,似乎说了一句什么话,然后一下子向我扑了过来,我大叫一声想要逃跑,却发现无路可逃,那人头一下子压在了我的身上。

我歇斯底里地大吼起来,用力想把他推开,却被他死死抱住,极度混乱中,我忽然听到那人头在我耳边清晰地说了一句——“救……命……”

我一愣,停止了动作,脑子里傻了,心说水鬼怎么可能会喊救命,忙扶正那人头,拨开他的头发一看,几乎没吐血。

我的天,这哪里是什么水鬼,这不就是那一帮人中的那个凉师爷嘛。

这人已经体力透支,双翻眼白,几乎要晕过去,难怪脸色白成这样。我赶紧转到他身后把他拉住,托出水面,一边招呼老痒来帮忙。

老痒走近了一看,马上也认出了他,纳闷问道:“他娘的,这人怎么会在这里?他是怎么进来的?”

我对老痒道:“这家伙可能落单了,不敢一个人行动,所以就一直在我们边上监视我们,见我们下水了,他以为我们找到了出去的路,就也下水跟着我们,不过他没想到我们下水是要去那么危险的地方。”刚才一路上听到的水声,估计就是他跟着我们时候发出来的。

凉师爷还背着背包,吸了水,拉着他直往水里去,老痒赶紧将背包从他身上扒了下来,问我道:“那我们现在拿他怎么办?这人是他们一伙的,带着会不会给我们添麻烦?”

我也觉得头疼,但是麻烦也得带着啊,总不能把人沉在这里,说道:“现在也没办法了,先找个地方出水,以后再处置他。”

我们调整姿势,向内游了几米,水下便出现了一道宽长的石阶,一直从水底拾阶而上,直到高出水面十几阶。我们缓慢地靠近,然后踩着阶梯走出水面。

我累得筋疲力尽,一下子就软倒在台阶上,大口地喘气,一边的老痒兴奋异常,掏出了准备好的火把,浇上白酒,点起来照明,一下子四周豁然明亮起来。

我转头四处看去,原来这所谓通往地府的入口,也只不过是藏在瀑布后面的一个溶洞,不大不小,似乎也是天然生成的,不过有些地方有过人为修平的痕迹。

阶梯之上是一座青纹石石台,石台的四周有四根石柱,上面刻满了鸟兽的纹路,石台中放置着一个奇怪的高大青铜容器,像一个大的葫芦瓶,高度超过我一个脑袋,锈痕斑斑,上面都是双身蛇和祭祀活动的图案。

这是一个祭坛,我心里暗想,厍族重祭祀不重葬制,出现这个东西,看样子的确已经十分靠近古墓了。

我们走上石台,将包裹和凉师爷放到地上,又走到石台的另一面观察,那里有一道十人宽的石阶,蜿蜒一直向下通向这个洞的深处,足有上百阶,火把的光线照不到底部,无法知道下面是什么。我对老痒道:“如果这是通往地府的入口,这里就是鬼门关了,这下面恐怕便是十八层地狱,你怕不怕?”

老痒指了指一边的凉师爷,说道:“怕个屁,我恨不得快点下去,可是这个家伙怎么办?”

古墓的入口如此接近,我和老痒都按捺不住想要马上下去看看,可是碍于多了凉师爷这个拖油瓶,又不能扔下他不管,只好先把他弄醒再说。

我们把他的衣服扒了,然后给他灌了两口白酒,他的脸色迅速缓和了起来。老痒翻开他的眼睛看了看,问道:“喂,能不能说话?”

凉师爷已经逐渐恢复了意识,知道落到了我们手里,无奈地点了点头,咳嗽了一声。

老痒说道:“你别怕,我们和你们那伙人不一样,我们不会拿你怎么样。不过我们也要保证自己的安全,你给我老实一点,我们就带着你继续进去,不然我就把你直接崩了,你明白了吗?”

凉师爷又点了点头,张开嘴巴想说什么,却又说不出口。

老痒又灌了几口酒到他嘴巴里,把他灌得剧烈咳嗽,又抽出皮带,把他的手捆了个结实,对我说:“我还是不放心,这些人个个都是亡命徒,还是先把他绑上再说。”

凉师爷也实在没气力反抗,由得老痒把自己绑上。我们看他应该没什么问题了,将他架起来,让他打头,三个人来到石台的另一边,踩着石阶向下走去。

一般来说,蛇国并不擅长机关和巧术,但是出于谨慎,这百来阶的石阶,我们还是走了很长时间,终于,前面出现了平坦的地面,我们来到了阶梯的底部。

阶梯的底部,是一块秃出的黑色石梁,再过去,就是一个断崖。

这种地貌,可能是地下水道所在的岩脉是一个阶梯形向下的结构,有些地方发生过山体运动,造成一系列的断层而形成。

断崖下面一片漆黑,多高、有什么都看不清楚。

这下子我们发愁了,如果有手电倒还好,现在一个小火把,如何照得到下面有什么东西?老痒问我怎么办,要不要把火把扔下去。我说这怎么行,火把下去了,我们怎么下去?

这时候,凉师爷有气无力道:“两位,在……在下的包里有信号枪……”

老痒忙往他的包里一摸,果然摸出一把信号枪来,看了看凉师爷,惊讶道:“哎,你这人不错,还真合作啊。”

我检查了一下,信号枪没什么问题,拉开保险,然后对着悬崖的上方“砰”一声打出一发信号弹。

曳光闪过,照亮了一大片区域,一刹那,整个山洞清晰地呈现在了我的面前。

我们往下看去,一下子,三个人全部僵住了。

一开始,我还没有意识看到了什么,等我明白过来,人一下就蒙了,张大嘴巴,几乎不敢相信自己的眼睛。

前面本来就虚弱的凉师爷,看到下面的情形,早我一步软倒在地上,几乎掉下去,老痒也面色苍白,不自觉地向后退了一步。

悬崖下面十几尺的地方,是一个天然的大洞穴,里面密密麻麻堆满了枯柴一样的东西,仔细一看,你就可以知道那全是骨头,一片挨着一片,有些地方还累起来好几层,足有上万具之多。

“这……这是什么地方!”我惊叹道,“我的天啊,这不是万人坑吗?”

难怪那几个风水先生会说自己看到了阴间,这种景象太震撼人了,无论是谁看到,都肯定以为是地狱里的情形!

但是,不知为什么,我觉得眼前的这景象好像很熟悉,好像看到过?我皱了皱眉头,回忆了一下,忽然间,我的脑子里出现了一幅相同的情形,对啊!山东瓜子庙附近的那个尸洞,不是和这里非常相像吗?

我一下子思维混乱到了极点,只觉得喉咙里像是卡了什么东西,什么话也说不出来。这里果然和山东瓜子庙的尸洞有联系!那山体上的水晶棺材,还有尸堆里那长发及地的白衣女尸,这里会不会也有?

我马上四处去看,这时候,在空中的信号弹已经滑行到了弧线的尽头,在光线熄灭的一刹那,我好像看见在这些尸体的中间,有一块奇怪的地方。

\chapter{休息}

老痒重新装填了一发信号弹,朝刚才第一颗信号弹熄灭的地方开了一枪,将那里重新照亮,我看见那是一块没有堆放任何尸体的空地,位于整个洞穴的中心,大概有二三十平方米,信号弹的光线不足以让我看清这块区域是否有特别,只不过,有一点可以肯定,这块空地是向下凹陷的,应该是一个坑。

老痒这时候已经镇定了下来,指着那坑说,他三年前看到的殉葬坑和这里差不多,中间也有这么一个空地,那怎么样也挖不到底的青铜枝桠,就是位于这坑的中间。

照明弹的光线衰竭,洞穴里又恢复到一片漆黑,老痒还想再装填一发,被我拦住。现在该看的我们已经看得差不多了,不必要无谓浪费资源。

老痒问我道:“现在怎么办?闹了半天这阴间就是这么一回事,说不定这里也就是一个祭祀的地方,我们还要不要下去?”

我想了想,说道:“那李老板‘河木集’里说这斗里有好东西,应该不会错,咱们跟着铁链来到这里,路也没问题,我看他说的东西就在下面。最可疑的地方是尸体中间的那块空地,我觉得我们还是要过去看看……不过这尸体堆积的地方,历来是最邪门的地方,我们得做好准备,应付最麻烦的事情。”

我本来想说说在山东碰到的那些个事,回头一想不把这两个人吓死才怪,于是改口扯到别处去了。

老痒已经压根儿不想下去了,不过提议到这里来的人是他,他也不好打退堂鼓,只好不情愿地点了点头。

我回忆刚才看到的情形,要到达那块空地,无法避免地要下到悬崖下面,从尸体中穿过,从我们所在的石梁到那块平地大约也就是二百米左右,应该问题不大。问题是如何爬下这二十几米高的悬崖,我们没有绳子,徒手爬下去的可能性有多大,还要从长计议。

另外就是这下面有没有粽子,下面保存完好的尸体应该不多,大多数已经干涸或者成为枯骨了,但是刚才在照明弹的照耀下,我看到很多尸体的表情非常的狰狞,超出了人类表情所能表现出来的极限,不知道是怎么回事。

正琢磨着,忽然听到一声摔倒的声音。

我回头一看,只见凉师爷正蹑手蹑脚地想退回到石阶上去。

老痒马上举枪把他逼住,喝道:“再往后走一步,我就打断你的腿,然后把你从这里丢下去。”

凉师爷一听到他的声音,吓得拔腿就跑,老痒朝天开了一枪,霹雳一样的枪声顿时响彻整个山洞。

凉师爷给枪声吓得停了下来,缩着脖子转身说道:“别开枪!别开枪!我不跑还不行吗?”

老痒骂道:“鬼才信你,给我回来好好蹲着,再跑一次,我就把你料理了!”

凉师爷灰溜溜地走了回来,蹲到我们边上,哭丧着脸对我们说道:“两位小哥,你看在下只是一个知识分子,跟着老泰混口饭吃,糊弄一下那广东客人,按判起来也是个次犯,你们还是放过在下得了。你们现在要去做大买卖,在下手无缚鸡之力,跟着你们也是累赘,万一一个手脚不利索,连累你们就不好了。”

老痒见他手里竟然还拿着他那只背包,不由大怒,用枪指了指,对他说道:“你以为我们想带着你啊,你要我们放过你也行,把那包留下,你爱上哪儿快活去哪儿快活。”

凉师爷为难地看了看那包:“可这包是在下的……有道是君子——”老痒扬了扬手里的枪,说道:“我不是君子,我是畜生,甭跟我讲道理。”

我觉得这凉师爷颇有点道行,要是把他放回去,碰上泰叔他们,等于给自己增加了一个敌人,留下兴许还能起个牵制的作用,我阻止老痒说下去,转头对凉师爷说:“我们现在处境还不明朗,你一个人走掉,就算给你全套装备,没有经验也出不去,不如这样,你跟我们下去看看,如果有好东西,泰老头给你多少,我们也给你多少,三个人一起行动,生还的几率大一点。你看这里阴气冲天的,要是碰上个孤魂野鬼,谁也救不了你。”

老痒马上接着说道:“你要是不想去也行,不过把该留下的都留下,把衣服也给我脱下来……”

他听到我说也给他留一份明器,顿时就露出动摇的神色,又加上老痒一吓唬,马上说道:“别别,有话好商量,既然两位这么看得起在下,那在下也不便推辞,其实以在下的学识,能和两位的经验配合在一起,实在是珠联璧合。”

我一听敢情这小子还是棵墙头草,两边倒,变卦变得这么快,心里觉得好笑。爷爷说得对,人心险恶,这个世界上真是什么样的人都有。

我们将凉师爷包里的东西重新拿回来,倒了出来,寻找有没有可以利用的,比如说绳索和照明工具,但是他的包里主要是食物和衣服,凉师爷说他们重要的装备都是由泰叔和二麻子这两个骨干背着的,他这把信号枪也是在走散的时候用来求救的。

没有绳子,下悬崖肯定要学壁虎游墙,这里这么陡峭,也不知道适合不适合攀爬。我打起为数不多的几个冷烟火的其中一个,往悬崖下扔去,一路照下去,看到有很多地方可以落脚,如果有持久的照明工具,爬下去不会太难。

现在在外面已经是晚上十一点左右了,我们一路上都没停过,今天晚上我们就不下去了,好好休息一下,把伤口也处理一下,等到明天再下去,不然在疲劳状态下到坑里,如果里面有什么情况,肯定会出纰漏。

泰叔和那个胖胖的广东人现在是死是活,我们也不知道,他们手里到底还有两支枪,碰在一起免不了又是一番恶斗,还是要提防一点。

我本想问问凉师爷他们几个人的来历,但是转念一想,现在问不合适,我们现在的关系这么紧张,他必然不肯说,要等到人放松的时候问他,才可能听到真话。

我把我的想法和老痒一说,老痒点点头表示同意,不过他道:“这里太他娘的那什么了,下面这么多尸体,我们还是上去,到祭祀台那里去休息。”

我想想也是,于是重新爬上石阶,回到祭坛处。

老痒重新点起炭火,将一只罐头捞空,放在火上烧了点水,将一些干粮泡软分开吃掉,几个人吃饱了后,又吃了一些巧克力增加血糖。

老痒吃完后就困得不行了,我让他们先睡一会儿,我来看着火,老痒说这里也没什么野兽,不用这么上心,我偷偷告诉他,我主要还是要看着那凉师爷,这种看上去窝囊的人,往往越是深藏不露,我们两个都睡着了,说不定他就会露出本来面目来了。

老痒说道:“要你不放心,我把他敲昏得了。”

我忙摆手,心说要敲傻了就麻烦了。

老痒自顾自睡觉,我掏出藏在衣服内袋的拍子撩,打开保险插在皮带上,然后又烧了一罐水擦拭自己的伤口,在瀑布的时候,我手上的烫伤很严重,如果处理得不好,肯定会造成感染。

等这些都处理好了,我叫醒了老痒,自己才睡了下去,这一觉睡得极其不舒服,浑身酸痛,伤口又痒又疼,醒过来的时候,发现自己才睡了五个小时,身体难受得鼻子都塞住了。

老痒给我烧了烫水洗脸,我感觉好了一点,吃早饭的时候,我看凉师爷表情没昨天这么戒备了,就旁敲侧击地问了几句老泰这几个人的来历。

凉师爷已经知道了我们的名字,他看了看我,听出了我的意思,眼睛一转,对我说道:“小吴哥,既然咱们现在是一伙的了,我也不瞒着你,我们来的时候是五个人,其中只有泰叔和二麻子是专门干这个的,在下是跟着那李老板和王老板来的,一来想见识一下鲜货是怎么出土的,二来两位老板让我把墓里最值钱的东西先挑出来,所以说实在的,在下真的是一个很冤枉的角色。”

老痒听到他这样说,就问他:“奇怪,刚才看到你们是四个人,那第五个人呢?”

凉师爷说道:“你说的那个人就是李老板,刚才我们从矿道下来的时候,他去一道水坑去洗脸,结果我们发现他的时候,他的脑袋已不知道给水里的什么东西咬没了……”

我和老痒正在吃东西,忙让他别说了,下面的事情我们已经知道了,再说我们就吃不进东西去了。

我看他似乎打算全盘托出,心里说这人也算是识时务,又乘机问他那两个老板的背景。

凉师爷站了起来,说道:“说起那两个老板的背景,不说不知道,一说可要吓你们一跳,他们可不是普通的古董商人,你们且听我细细讲来……”

\chapter{爬}

凉师爷当下放下手里的食物,将这两个人背景简略地向我们叙述了一遍。

那两个广东来的老板,姓王的叫王祈,姓李的叫李琵琶,两个人都是佛山人,在当地的古董界里有很大名气,其中李琵琶的背景我们已经知道了,发家全凭记载大量古墓位置的《河木集》。

而我之所以知道这些,原因是我和老痒曾经偷听过他的说话,不过他所说的一切都是他的一面之词,其中有几分夸张,我们就不得而知了,如今听凉师爷说起来,言之确凿,可信得多。

而王祈的家世就没有李琵琶显赫,但是却更加真实,他的祖上从事的职业,叫做朝奉。

何为朝奉?朝奉就是指在当铺中干活的伙计,坐在高高在上的柜台上,在短时间内判断一件东西的价值与真伪,就是他们的工作。

其中,负责高级物品鉴定与日常行政事务的,叫做大朝奉。一个大当铺的大朝奉,可以说是世界上见识宝物最多的人,什么稀奇古怪的东西他都见过。王祈的祖上,就是一个有名的大朝奉,叫做王宪初,他在晚年的时候写了一本笔记,叫做《古毓斋奇劫余录》,这本东西堪称奇书,上面记载了他一生所遇到的他认为奇异的物品,并详细记录了物主的说明、他的判断等等,对考古工作有很强的横向参考价值。

王祈本身文化不高,但是他的记忆力非常好,这本《古毓斋奇劫余录》里的东西,他看过多次,不知不觉中全部都记了下来。正巧有一次,在一街头的交流会上,他看到一只白玉狮子,与《古毓斋奇劫余录》里记载的一种藏头盒很像,他当着众人的面,按着《古毓斋奇劫余录》里的记录,将这只白玉狮子放进茶水里,没过多久,那只狮子竟然自己张开嘴巴,从里面吐出了一枚金叶子,从此王祈便名声大噪,一发不可收拾。

至于这两个人什么时候走到一起的,凉师爷也说不清楚,他做师爷的也不好过问。

听到这里,我就问凉师爷,为什么这一次他们两个要亲自来这里?这些人养尊处优惯了,怎么受得了这种折腾?

老痒说道:“这有什么想不通,这就叫做闲钱烧脑,是钱多了给闹的,这些有钱人,钱多了就不知道自己是谁了,都要去寻找自己的人生价值,有些人家财万贯还要出去要饭,这不稀奇。”

凉师爷呵呵一笑,说道:“我刚开始也这样想,但是后来我发现不是,这一次他们两个非常坚决,按照我的估计,这里面可能有隐情,答案就在这古墓里面。”

我问他道:“对了,师爷,你既然看过《河木集》,那你知道不知道,这进入瀑布之后,以后的路该怎么走?”

凉师爷看了我一眼,说道:“这《河木集》是李琵琶的宝贝,我只是在李琵琶死后抓紧看了几眼他的笔记,其他的内容倒看到不少,不过这进古墓的那部分,倒是没有看到,那东西后来给那姓王的老板拿在手里,我也没机会去看。不过看昨天见到的情况,那古墓的入口,八九不离十就在下面的尸体堆里。”

既然凉师爷说不知道,我们也只好相信他。我们吃好早饭,背起背包,我给凉师爷松开皮带,然后将自己的衣服脱下系在腰间,系紧鞋带,三个人各自准备完毕,来到石梁,就开始尝试着向下攀爬第一步。

令人觉得讽刺的是,在三个人里面,我可能算是体力最好的,所以火把就由我拿着。想当日我在鲁王宫里,可完全是属于添头的档次,怎么这一次就担当了这么重大的责任,我自己也不知道是怎么一回事。

话虽这么说,对于现在这种状况我也没有话好说,我们一步一步,缓慢地将自己的身体放到悬崖下面,向漆黑一片的洞底爬去。

这一路爬得很艰苦,有几次我几乎从悬崖上滑落下去,但是总体来说,这里虽然陡峭,但是并不难攀爬,胆大心细,就是小丫头片子也能爬下来,只不过是多消耗点时间而已。

下到一半的时候,凉师爷的脚已经抖得不行,看样子这人不太习惯爬山,大概足足花了大半包烟的工夫,我的脚才踩到了久违的地面。

从地面上去看那些尸体,有一种无法言明的恐惧非常强烈,这些尸体应该都是殉葬的奴隶或者战俘,尸体长年累月在太阳晒不到的阴冷潮湿的洞里,骨头上呈现出一种霉变的黑色,空气中更是弥漫着很浓的霉味。很多尸体都曾经给肢解过,尸体的表情狰狞,我甚至发现很多尸体好像都长着獠牙。

我把凉师爷从悬崖上扶了下来,他一个蹒跚就踩到了一颗头骨上,将早已经腐烂的头盖踩了一个窟窿,幸亏被我拉住才没陷进去。他好不容易站稳了,擦了擦头上的汗,说道:“真是让你们见笑了,在下自小就体弱多病,见风就倒,就我这身子骨,这倒斗的买卖恐怕是没有下次了。”

我安慰了他几句,抬高火把照亮四周,看看这路该怎么走。

尸体堆积如山,尸体之间,有一条小径直直通向前面,火光有限,我们只能看到十几米外,再远就看不到了,不过我们在悬崖上面看的时候,已经看准这条路就是直通到那块平地上的,估计着只要往前就能到地方。

凉师爷体力透支得太厉害,实在走不动了,我让他在这里先喘口气,也顺便看看,这里的尸体是个什么样的情况。

我们四处转了几圈,看了半天,我发现凉师爷明显有表情的变化,问他:“看出来什么了?”

他对我说道:“这里好像有一些不是人的尸体,这些头骨的结构不对。”

我心里直起鸡皮疙瘩,心说难不成是尸变之后的僵尸骨?忙问他如果这不是人,那会是什么?

凉师爷对我说道:“现在看也看不出来,你们要想知道,我得多看几个,最好能找到没完全腐烂的,在这些尸体堆积处的内部不知道有没有,要不要看看?”

老痒倒吸了一口凉气,说道:“你说得倒是轻巧,这里面的尸体给这么重的阴气罩着,肯定有尸变的迹象,要是开出只粽子来,我们也没带黑驴蹄子,你又不能蹦不能跳的,弄不好,恐怕三个人都得交待在这里。”

我和老痒的想法一样,就对凉师爷说:“不用了,咱们又不搞研究。”

凉师爷估计早先也听过不少粽子的事情,点头对我们说:“我也就是说说,要我干我还不肯呢。”

我看火把用了很久,烧得很快,火焰坚持不了多少时间,在这种地方如果火把熄灭,那是要命的事情。想要再制作照明的东西非常困难,最差的情况,我们不得不摸着尸体走路,于是就不让多歇,蹲了几下就催着他们上路。

我们沿着小径向前走去,两边是一排又一排的尸体,在尸体的中间,让我欣慰的是看到很多石人混杂在里面,洞穴的底上是泥土,这让我觉得很惊讶,走在上面并不是很踏实,想起这些黑色东西也许都是死人腐烂而成的,我就觉得有一种脚底板发凉的感觉。

走了一会儿,火把的火焰就小了下来,光照的范围逐渐缩小,我们加快脚步,开始向前小跑,不一会儿我就开始觉得奇怪,从悬崖上面看下来,这里距离也就二百多米,脚力最差五分钟内肯定就到了,怎么我们走了将近一刻钟还是没看到那坑的影,难道是黑灯瞎火的,在什么地方走了岔口了?

我们又向前跑了一支烟的工夫,还是老样子,前后都只能看到成堆的骨头,再远的地方就是一片黑蒙蒙的,我不由暗骂,这下子失算了,没有想到下到底下来,这里的视野被黑暗所限制,不管哪里看来都是一样,现在不知道跑到哪个角落里去了。

这时候凉师爷实在不行了,一把拉住我大喘气,说道:“小吴哥,别……跑了,没……用,我们可能中招了。”

\chapter{尸阵}

我们跑了半天头昏脑涨,却怎么也见不到目的地,心里早就已经在犯嘀咕了,一听凉师爷突然这么说,老痒便停下来问他道:“师爷,什么中招,怎么个说法?”

凉师爷一边揉着胸口一边指了指地上,对我们说道:“两……位小哥,你们看这骨头,是不是很眼熟啊。”

我闻言把火把抬高,果然看到地上有一只头骨,上面有一个窟窿,好像是他爬下悬崖的时候压坏的那一具,我心中暗暗感觉不妙,回头一照,果然后面不远处,就是那块悬崖。

老痒看了看四周,埋怨道:“老吴,你怎么带的路,这不是刚才我们下来的地方吗?”

我没好气道:“我也不知道,这地方哪里都看起来一样,他娘的一直走也没有注意,不知道是不是进了岔口,给绕了回来。”

凉师爷气顺了过来,对我们摆了摆手道:“不对,你们都没注意,在下记得清清楚楚,这条小径一直都是笔直的,没有转弯或者岔路,这事情不简单,要是我没弄错,我们可能被什么东西给糊弄了。”

老痒知道苗头不对,脸色一下子变得惨白,说道:“那糟了,难不成这里这些尸体的冤魂,为了保护他们的圣地,而不让我们靠近那块空地?”

我心里苦笑,四周这么多的尸体,千尸聚气,要说没脏东西谁也不信。凉师爷却又摇了摇头:“我想不太会,我身上带着开光的东西,要迷你们会迷,但是我绝对没事。”

我知道这人的确有点学识,问他道:“凉师爷,你这方面的见识应该比我们多,你估计这是怎么一回事,咱们的火把也坚持不了多久了,等一下火灭了,就真的叫天天不应叫地地不灵了,得快点想个办法。”

凉师爷说道:“依在下看,我们之所以走了个圈子,是这里的尸体排列有问题,这几千只骨头纵横交错,其间可能运用了某些奇门易术,使得整个山洞变成一个迷宫,你知道诸葛亮的八阵图,用几堆石头就能困住十几万大军。这里的几堆骨头困住我们三个,那还不是小菜一碟?”

诸葛亮驱兵取乱石,在临山傍江的鱼腹浦沙滩上布下石阵挡住陆逊的故事,我和老痒都知道,可是小说描写毕竟是夸张,我根本不相信区区几堆石头就能有这么大作用,要是果真如此,还要造这么多坦克大炮干什么?

老痒也不信,对他说道:“师爷,你可别拿糊弄广东老板那一套来糊弄我们,您自己可也困在这儿呢,这八阵图的事情,我听评书里说过,根本不是你说的那一回事,况且了,咱们在悬崖上看的,这里的骨头排列凌乱无章,也没发现什么特别的布置啊。怎么下来之后就能把我们困得团团转,难不成这里的尸体还能自己跑路不成?”

说完这个,老痒忽然意识到什么,忙捂住嘴巴,向四周作揖,轻声说道:“大吉大利,小孩子不懂事,各位别见怪啊。”

凉师爷说道:“这可不同,你在上面看是一个大概,就这么点时间,你能把尸体之间的脉络走向全记下来?下来之后这里一片漆黑,只要每一具尸体摆放的稍微偏移一点,就可能把我们引到事先设计好的歧路上去,不知不觉就在走回头路了,两位小哥也是过来人,大道理我也不说了,古人的心智我们可不能小看啊。”

我觉得凉师爷说的有点道理,但是也不能全信,不管怎么说这里肯定是有什么蹊跷,要走到那块空地恐怕不是简单的事情,又问他有什么主意。

凉师爷叹了口气:“不是在下吹牛,这区区一个阵法我是不在话下,不出意外定能手到擒来,不过凡事都需要一定的时间,恐怕咱们的火炬坚持不到那个时候。况且,在下认为现在这个时候咱们还有更重要的事情要先决定。”

我知道他的意思,顿感头痛,眼下的主要问题还不是破这个阵,而是怎么面对我们的处境,不走不是办法,走下去也不是办法,这一次能走运回到原来的地方,再走一次就不一定了,到时候火把一熄灭,前没村后没店的,不困死才怪。

其实破阵的最简单的方法,就是从边上那些尸体上踩过去,不过这个建议谁也没提。

僵持了几分钟,火把上的火焰扑腾了几声,逐渐虚弱了下来。老痒看了看火把,突然叫道:“他娘的,我有个点子,要不我们一把火把这里的骨头全烧了,给它来个火烧连营十八里,烧光了就干净了。”

我一听这人时傻时聪明,这种点子也想得出来,大骂道:“这里的骨头都已经快石化了,绝对烧不起来,而且就算烧起来,你这不是等于自焚啊,就算不烧死也给烟熏死了,算了,我看这样吧,我先往前走走,你们看着我的火把的走向,一旦我的移动偏移了方向,你们就叫停我,我们就知道问题出在什么地方了。”

老痒说道:“不行,万一走到一半火把熄了,你一个人情况更糟糕,到时候谁去救你去,这种时候我们绝对不能走散。”

我也是急了,刚想反驳,手上的火把突然闪动了两下,终于坚持不住,扑哧一声熄灭了。

\chapter{鬼吹灯}

火把一熄灭,本来就不甚明亮的空间突然漆黑一片,我吓出了一身白毛汗,火把差点脱手掉到地上。

凉师爷胆子更小,当时就怪叫了一声,撒腿就跑,才跑没几步就听到“嘣”一声,大概是撞在了什么上,疼得嗷嗷直叫。

我掏出打火机,照了照火把,发现上面的燃头并没有烧完,不知道为什么火焰就突然熄灭了,难道是风吹的?可这里也没风啊。

老痒幸灾乐祸地说道:“老吴,你的手艺的确不行,这火把也太不经烧了,说灭就灭,真是非洲爸爸跳绳子——黑(吓he)老子一跳。”

我骂道:“你他娘的罗嗦什么,有空挤对我,不如去看看师爷怎么样了,别给摔进死人堆里去了。”说着我将火把重新点燃,抬高一看,只见凉师爷正倒在一具骸骨上,骨头架子散了一地。

我上去将他扶起来,只见他面色惨白,给吓得不轻,老痒拍了他一下,说道:“师爷,您还真是逗,就您这胆子,还想来倒斗?”

凉师爷见火把又烧了起来,松了口气,说道:“两……两位别误会,在下不是怕黑,是刚才,他娘的好像有啥东西在我脖子后面吹气,凉飕飕的,我以为粽子出来了,一下子给吓得没魂了。”

老痒大笑:“什么凉气,我看是你的凉汗滴脖子里去了,这粽子在您背后,不啄你一口,还往您脖子上吹气,他娘的您以为粽子都是小姐啊?”

我也说道:“是啊,凉师爷,镇静一点,别自己吓唬自己。”

凉师爷看我们不信,急了,咳嗽道:“两……两位小哥,千万要信我,刚才肯定有人在我后脖子上吹气,那感觉真他娘的瘆人,我看这里不止我们仨,还有别的东西在!”

我看他的表情,想起刚才火把突然就熄灭了,觉得凉师爷的话也不是完全不可信。火把不比蜡烛,上面的燃头不烧光,是很难熄灭的,刚才这一下子,肯定是出了什么问题。而且在这种地方,留个心眼总是好的。

想着,我给老痒使了个眼色,意思是还是去看看保险。老痒点点头,两个人掏枪出来,一前一后就往凉师爷刚才站的地方走去。

凉师爷刚才站的地方,身后一尺不到就是一具石人,石人的脑袋已经干枯了,绝对不会是这东西吹气,那唯一可以藏身的地方,就是石人的背后。

我和老痒小心翼翼地走过去,先用火把探一下,然后再侧头去瞄一眼,生怕有什么东西突然冲出来,然后老痒猛地跳了过去,大叫:“举起手来。”

什么都没发生,后面什么都没有。

我松了口气,心说看来凉师爷确实是吓糊涂了,不过这也不能怪他,刚才这种环境下,要是以前没来过这种地方,害怕是难免的。想当年在鲁王宫里,我还不是一样,胆子这东西,的确是要靠练出来的。

老痒白了我一眼,摇了摇头,两个人转过身子,刚想将枪收起来,突然“扑哧”一声,我手上的火把又灭了。

我一下蒙了,怎么回事,这火灭得也太突然了,就在这个时候,黑暗中的老痒忽然大叫:“我操!老吴,当心!这里真有什么东西!快把火把点起来!”

我一下子醒悟过来,忙去掏打火机,还没摸到呢,突然背后一凉,一道劲风闪电般袭了过来,我心叫糟糕,黑灯瞎火的,看不清来的是什么,忙一矮身子,那道劲风贴着我的头皮掠了过去,同时我脚下一个踉跄,扑倒在地上。

这一跤摔得倒不是很疼,只是撞到了边上几个石人,稀里哗啦的,不知道什么东西掉了我一脸,我顾不得恶心,我急忙打起打火机,以最快的速度将火把点了起来。

一照之下,只见老痒和凉师爷都面如土色趴倒在地上,凉师爷已经吓得糊涂了,直叫阿弥陀佛。

老痒心有余悸,对我说道:“快照照,他娘的刚才到底是什么东西?怎么速度这么快!”

我咬紧牙关站起来,举着火把一转,发现除了又给我们撞翻了几个石人外,四周什么变化都没有,连个脚印也不见一个,当下心里骇然,刚才那一道劲风急如闪电,可见对方靠得极近,可这里石头和尸体密布,就这么打起打火机的工夫,一片漆黑的,就算逃得再快,也不可能什么痕迹都不留下,我又转念一想,我操,难道是真碰上鬼了不成?

火把灭了两次,难道这鬼还想效仿鬼吹灯,把我这火把当蜡烛了,他娘的也太没职业道德,要吹也不能吹火把啊。

我将火把压到肩膀下,免得突然又给弄熄了,然后将凉师爷架起来,这人已经进入恍惚状态了,怎么拉都站不直,像摊烂泥一样。我提了两把,实在拉不起来,老痒没有办法,上去就啪啪两个耳光。

我怕老痒下手太狠,忙将他拦住,这时候凉师爷倒反应了过来,一看四周,号啕大哭:“哎呀我的娘啊,你说我这人真是多事,好好在家待着多好啊,干什么学人倒斗,这下子完蛋喽,客死异乡——”

老痒看他没完没了,一把捂住他的嘴巴,骂道:“有完没完,一把年纪了害臊不害臊,再吵吵我们把你扔这儿,你自己爬回去。”

凉师爷是情绪失控,被我们一吓唬,他马上抹了把脸,不敢再发出声音。老痒转头问我道:“老吴,刚才那是什么东西,你有没有看清楚?是不是粽子?”

我朝他招招手,说道:“不会,你看我们打了个照面,连对方毛都没看见,粽子没这么快。”

老痒对我说道:“你看这里这么多死尸,要说没粽子谁也不信啊,我听说粽子也有分等级的,该不会我们这次不巧,碰到了粽子里的轻功高手!”

我不想和他扯皮,走到给凉师爷撞散架的那几具尸体边上,用手枪拨了拨里面的东西,对他说道:“这里的环境这么潮湿,大部分尸体已经只剩下骨头了,上面还长着黑色的霉丝,这东西绝成不了僵尸。我敢用我的人头担保。”

凉师爷这时候总算镇定了下来,抽着鼻子说道:“两位小哥,这是不是粽子和咱们没关系,我看趁着现在还有火把,我们还是快点爬回到悬崖上面去,以后的事情再想办法。”

我知道他是经不住刺激,萌生了退意,便拍了拍他,解释说现在敌在暗我在明,如果现在去爬悬崖,指不定什么时候又来一拨,我们避无可避,就只能到阴曹地府里去哭给阎王听了,所以局势没明朗前,还是不要轻举妄动。

老痒说道:“老吴说得对,这不我们还有枪嘛,就算真是粽子,一两只我们也不怕他。”

凉师爷一把鼻涕一把泪,在那里直摇头:“小哥,您别安慰我,就我们这两把枪,碰到粽子是死定了,恐怕留个全尸都难。”

我没碰到过真正意义上的粽子,也不知道枪打不打得动,不过既然是肉做的,我就不信还能硬得过子弹。

想到这里,我的脸色算是缓和了下来,没刚才那么紧张了,想了想,觉得就等在这里也不是办法,还是得往前走,要真不行就踩尸体吧,反正现在也给我们撞翻了不少,没什么好怕,至于道义问题,自己小命不保,我也管不上了。

老痒一听,也觉得这是没有办法之中的最好办法,当下我们架起凉师爷,手枪上膛。还是老痒打头,我殿后,三个人咬紧牙关,顺着小路再一次往尸阵的深处走去。

我们上一次走过的时候留的痕迹还在,我记得有几个地方老痒还特别用力在泥地上踩出了几个脚印,我们顺着这些痕迹一路过去,果然没有发现任何的岔路,走着走着,我突然觉得有点不对劲,怎么这里的尸体腐朽得这么不均匀,有些尸体烂得连骨头都没了,可有些却还有皮肉,刚想把他们叫停仔细看看,突然“咣”一声,地上一具骨架子突然就散了架,骷髅一下子滚到了一边,我吓了一跳,刚一回头,就听“扑哧”一声,手上的火把第三次熄灭了。

我有了上次的经验,马上一蹲身子,这时候就听边上一阵混乱,老痒大叫:“我操!我逮住它了!”

\chapter{骨头的故事}

他话音未落,我就不知道给谁踢了一脚,正中脸部,差点给踢晕过去,随即我就听到稀里哗啦的一连串骨头压裂的声音,不知道出了什么事情。慌乱之中,我忙将火把点燃,定睛一看,只见老痒正和什么东西扭打在一起,已经滚进尸堆里,整一排骨头给撞得七零八落,人头骨散落一地。

我赶紧上去帮忙,却发现根本帮不上手,那东西体形不大,却猛劲十足,老痒一百多斤的体重压在它身上也压它不住,两个身体翻在一起,横冲直撞的,我根本近不了身,而且稍有不甚就会莫名其妙地被踢一脚,我几次尝试都无法进入战团,只能站在外面干看没办法。

一会儿工夫,老痒就要坚持不住了,那东西几次都几乎成功脱身,我一看再不去不行了,只好招呼凉师爷,两个一上一下,扑到老痒身上,将老痒和那东西压到身子底下,老痒也没想到我会来这一招,给压得够戗,忙大叫:“你他妈的悠着点!老子脊梁骨要断了。”

我使劲按住老痒,将三个人的体重完全压到下面那东西身上,发现没什么动静了,才问他道:“怎么样?那玩意死了没?”

老痒牙缝里挤出几个字来:“我不知道!不过你他娘的再不松开,我就死了!”

我看他脸憋得通红,赶紧撤下力道,老痒一个翻身起来,长出了一口气,对我说道:“你——你他娘的下手也太狠了,别以为是小时候叠个七八个人都没事情,幸亏老子脊梁骨硬,不然非半身瘫痪不可!”

我说你罗嗦什么,要不是你搞不定那东西,我犯得着这么大年纪还叠罗汉吗?你腰折,我他娘的也不轻松呢。

老痒听了,一边揉着自己的腰,一边大骂我没良心,我不去理他,转向凉师爷道:“话说回来,那东西到底是什么,怎么个子不大力气却惊人,要仔细看看。”

听我一说,三个人都回过神来,我们探头过去,只见那骨头堆里,有一团灰色的毛茸茸的东西,大概有一只猞猁这么大,给我们压得扁扁的,还在不停地颤抖。

老痒拾起一根人的大腿骨,将那团东西翻了身,我一看,操!闹了这么久,敢情是只大耗子。我看看老痒和凉师爷,他们也看看我,三个人都笑了,难怪刚才怎么找也找不到袭击者,原来是这么一回事。这耗子袭击完了我们之后,肯定是随便往哪个骷髅的眼洞里一钻,就踪迹全无,我们这群SB,还以为遇见鬼了,真是老母鸡管自己叫妈——自己下(吓)自己。

不过我转念一想,又觉得很不妥当,这只耗子,他娘的也太大了,也不知道是什么品种的,说不定还是吃着尸体长大的,也不知道这洞里还有多少这样的耗子,要是碰上一群,那得吃不了兜着走。

老痒和我心念相同,笑了一下后脸色也一变,说道:“不好,这老鼠皇帝给我们压死了,不知道他的鼠子鼠孙会不会找我们麻烦,我看要不还是快撤,别留在案发现场。”

我点了点头表示同意,老痒转过头去,刚走了几步,突然又说道:“哎,糟糕——我们往哪边走好呢?”

我抬头一看,原来刚才一阵混战,颠来倒去的,这前后又是一样,如今已经分不出哪里是我们来的方向,哪里是我们要去的方向了。

虽然我心里有一点点感觉,依稀能分辨正确的位置,但是这种感觉太淡,我几乎不能肯定自己是不是想的就是正确的,一犹豫,这感觉就消失得无影无踪。

老痒前后看了不下十几次,看实在没办法啊,对我说道:“算了,我们甩开膀子横着冲过去吧。”

我看了看,还是觉得有点不妥,就想问凉师爷意见,转头一看却发现他根本没有在听我们说话,而是在专心致志地收拾地上的那些骸骨。

我心下觉得奇怪,拉住老痒,两个人探过头去看他在搞什么。

这一场人鼠大战,牵连了十几具尸体,这些尸体早就已经腐朽得犹如沙土,所以一经撞击,形神俱灭,大部分都碎成了小骨片,地上一片狼藉。凉师爷不知道为什么,将剩下的没有碎裂的骨头一根一根地从地上拿起来,放到一边。

这些骨头大多数也不完整,大概是给这些大耗子当成了磨牙的工具,上面坑坑洼洼的,有些都已经无法分辨是人体上的哪一块。

老痒看凉师爷已经想得入神,心里好奇,问他道:“师爷,你这又是在捣哪门子蒜啊?”

凉师爷怔了一下,转过头来,对我说道:“了不得,给这耗子一捣乱,倒是错打错着,给在下发现了一个大秘密。”

我看他两眼放光,兴奋莫名,心里更加奇怪,这些骨头能有什么秘密?

凉师爷让我们蹲下来,拿起一根骨头给我们,问:“两位,看看,能不能看出点什么来?”

我和老痒对视一眼,不知道他在玩什么花样,老痒做了一个很怪的笑容,说道:“您这不寒碜我们吗,咱们是倒腾死人的东西,不是倒腾死人的,你还是直说吧,说完了我们赶紧赶路。”

凉师爷不好意思地笑了笑,说道:“在下是太兴奋了,话都不会说了,别介意,你们先让我想想怎么说,呃——你们看骨头这个地方,仔细看看。”

我接过骨头,自己一看,只见他指的那个地方,有一道很平滑的缺口,切口和骨头是一个颜色,年代应该也比较久远,但是凉师爷给我看这个有什么用意,我却想不出来。

凉师爷看我一脸疑惑,说道:“看不出来也没关系,我来和你们说,这根骨头是人的锁骨,就是这个位置。”他指了指自己的脖子,接着说,“这一道缺口,叫做陈旧性骨伤,是死前造成的,你看切口尖锐,一点骨头愈合的情况都没有,说明这道伤口的时间和这人死亡的时间是非常接近的。”

老痒一听,还以为是什么事情呢,当下很不耐烦,说道:“这种事情算什么秘密,骨头受伤了真可怜,不过我们还是快点走吧,火把都快烧没了。”

凉师爷忙摆手道:“再给我三分钟,马上说完了!”

我看他非常兴奋,不说清楚肯定也不会罢休,老痒罗里八嗦的反而耽误时间,忙使了个眼色让老痒别插嘴,转头对凉师爷说道:“别理他,您快说。”

他咽了口吐沫,说道:“刚才说到哪里了,哦,这伤口的时间和这人死亡的时间是非常接近的,在下大概能断定,这道伤口应该是这个人死亡的原因,之所以是在这个位置,大概是被人用刀从锁骨上方切断了颈动脉,下刀太快,所以划到了骨头上。”

我一听纳闷,问道:“按你这么说,这具骨头的主人,是给人割喉杀死的!”

凉师爷很诡异地一笑,摇了摇头:“不止这一具,这里所有的尸体,都是这样死的,你看,光这里就有七根锁骨,上面都有这样的切痕,而一般的古代祭祀人牲,都是让牺牲跪在祭祀品前,然后祭师在他身后割喉咙,但是这里的人,却是给人在面前一刀断喉,所以,我觉得,这些人大部分不是给活祭的,而是在战斗中战死的。”

凉师爷说完这话,目光如炬地看着我,我给他看得直发毛,心说这人怎么回事,战死就战死呗,用得着兴奋成这个样子嘛,忙问他道:“凉师爷,你说的大秘密,就是指这个?”

凉师爷故作神秘,说道:“不是不是,这只是大秘密的序章而已,接下来我要说的,才是正题。”

说着从尸体的碎片里又掏出一片东西,对我说道:“大秘密,就藏在这个东西里。”

我接过来一看,是一片无法形容的东西,似乎是斗笠,又像是盔甲的一部分,不过这东西既然不是骨头,那必然是明器。我拿起来对着火把仔细一看,惊讶道:“是青铜的甲片?”

凉师爷点点头:“不错。”

这时候,不知道是给神经兮兮的师爷感染了,还是我本身的直觉,我隐约觉得凉师爷说的事情可能真的有什么惊天动地的成分在里面,一时间给搞得一身冷汗。

凉师父接着说道:“这是汉代之后才出来的盔甲样式,你看这一片,没有衬里,是夏天的盔甲,这人死的时候是在夏天,还有,最奇怪的是这个东西。”他从那片盔甲的碎片里小心地剥出一片东西,“你看,这一片东西虽然不起眼,但是却是关键啊,小吴哥,你是明白人,一看就知道这是什么东西。”

我已经给搞个浑身冰凉,顺着他的意思一看,马上就明白了,那片东西,不是别的,正是一小片丝绸,大概是尸体腐烂的时候,被尸液粘到甲片上去了。

这些都是汉人的东西,怎么会出现在早在几千年前就灭绝的厍人的陪葬坑里?

凉师爷看了看这里,说道:“如果我料想的不错,这里其实不是一个殉葬坑,而是一个战场,这里的尸体有两派,一派是这古墓的守护人,一派是一股汉人的军队。”

\chapter{火龙阵}

我想起了夹子沟的传说,那消失在山里的是不说话的北魏军队,心里已经明白了一大半,不说话,其实是指那是一群哑巴组成的军队,可能也就是凉师爷说起的北魏时期的“不言骑”。这些士兵是绝对不会透露秘密的,所以皇帝让他们去执行那些不光彩的任务,比如说盗墓。

一千年前,蛇国的后裔已经消失在与汉族通婚和海外,但是这山洞里面为了某位酋长守护陵墓的一批蛇国先民却繁衍了下来,不知道为什么原因,那支北魏的军队会知道山中有这样一座陵墓。

汉人的军队杀入这里,攻破了迷宫一样的溶洞,杀入殉葬坑内,蛇国的先民誓死抵抗,可惜无论如何也不是装备先进的不言骑的对手,所有的人被屠杀殆尽。

可以肯定,这里的尸体,绝大多数都是厍人的遗体,那我们在这里走圈子,可能真的是聚集的冤魂仍旧在守护着他们祖先的陵墓,不让我们这些侵略者靠近。

那真是难办了,难道就这样回去,白走一趟?我心里是大不甘心,可是,如果真的有鬼魂作祟,我们怎么样也是没有胜算的。

火把逐渐没有光芒了,闪了几下,火苗小得犹如蜡烛一般。

老痒此时也不来催我们了,因为他知道,用普通的方法,已经不可能到达古墓的入口了,无论有没有鬼,火把的时间也不够了。

凉师爷道:“既然这里是战场,那尸体就不可能做过手脚,这里就不是什么尸阵,我估计,咱们真是给鬼迷了眼睛了,这就是鬼打墙啊,各位知道不知道有什么办法可以克制?”

老痒无奈地叹了口气,“我山西老表说,碰到这种事情,用红线绑着左脚,就能走出去,可我们身上也没红的,要不,咱们用自己的血来染?”

我对老痒道:“那千万不要,这地方冒出血气,总是感觉不太好的事情,咱们再想想办法。”

凉师爷道:“对了,我听我师傅说过,鬼打墙必须得在黑暗的环境里,咱们不是还有信号弹吗,打起一颗,然后一路跑过去,我估计比用火把要好,至少不会给迷住。”

我一听有点道理,只要我们知道我们要去的地方,无论怎么样也迷不住我们。于是给老痒打了个眼色。

老痒叹口气,掏出信号枪,说道:“太浪费了。”说着抬手对着头上就是一枪。

流星一样的信号弹射上半空,我下意识地抬头看去,等着它开始燃烧,没想到这颗流星飞着飞着,突然就啪的一声,反弹了一下,直直坠落下来。

我一看哎呀了一声,心说日你个板板,忘记这里是山洞了,笔直往上打信号弹,还没开始燃烧就会撞到洞顶。

信号弹飞快地坠落下来,直到几乎落地才噗地一声绽放开来,这种是探险队用的五氧化二磷信号弹,大概可以燃烧五十秒,初始引燃温度非常高,我一看它离地面的距离,就知道要糟糕,果然,它落地才几秒钟,那面已经燃起了火苗。

我踢了老痒一脚,骂他没脑子,幸亏都是骨头,要不然这一下子,我们还得跑回去救火。话还没说完,凉师爷拍了拍我的手,叫道:两位爷爷,这次要糟!

我回头一看,只见刚才起小火苗的地方,突然蹿起来一条火墙,不可思议的是,这道火墙正在以惊人的速度顺着尸堆之间的小径蔓延,一时间只见一条贴地而行的火龙在漆黑一片的山洞里游走,所到之处,小径两边的骨头无不发出爆裂的声音。

凉师爷看到此景,面色惨白,急忙蹲下身子抠起一把地上的泥土,闻了一下就大叫:火油!泥里浇了火油!

我一听大惊失色,蹲下一捏泥土,果然没错,忙叫老痒把火把扑熄,心里那个寒啊,没想到这尸阵里还藏了这么厉害的一招,恐怕是这里的先民为了保护古墓里的东西而设的最后一倒防线,可惜当时没来得及用,结果给我们引发了。

这一路过来没出事真是奇迹,要是刚才不小心把火把掉到地上,那爷爷我们几个已经烧成焦炭了。

远处的火龙丝毫不见懈怠,不知道何时已经分成两路,火焰蹿起一人多高,瞬时间将这个洞照得通明。我大概一看,发现终于可以看清楚这里的格局,只见整个尸阵中脉路通达,不大一个地方,其中的小径却是连成一气,这条火龙迟早会烧到我们这里来的,一定要找个地方避一下。

我焦急地四处张望,看到那凹陷的空地其实就在我们左手边十几米处,可是中间已经隔起了一道火墙,里面的泥土却没有烧起来,似乎是一个避难的好地方。此时火龙头已经在向我们冲过来,没时间考虑了,我对他们大叫:别在这里傻看了,那个坑在那里!他娘的冲过去再说!

他二人反应过来,直接踩着尸体向那片空地冲了过去,我也不知道自己还有跨栏的潜质,那些石人我竟然能够一跨而过,才几秒钟我就已经来到火墙之前,一股灼热的气浪扑面而来。

我想一鼓作气冲过去,可是刚贴近火墙,就闻到了头发烧焦的味道,脚下一犹豫,就想停下来,可惜我惯性极大,想刹车却刹不住,只好大叫一声,闭着眼睛跳了过去,幸好速度够快,只是觉得身上一烫就已经滚倒在地上。我打了一个滚将身上的火压熄,接着老痒和凉师爷也冲了过来,纷纷滚倒灭火。

这时我已经知道这里的地面为什么会下陷,原来表层的土已经给人铲掉了,我一滚之下也来不及细看,老痒已经惨叫着滚到我的身边。

我忙脱下外衣,帮着将老痒身上的火拍熄,扶起来一看,人倒是没事情,只是眉毛烧没了,转头却见凉师爷不停地翻滚,可身上的火就是不灭,我想到大概是因为他在地上摔倒过,衣服上沾上了火油,所以压不灭,便赶紧将他扑倒,用地上的泥将火压熄。

凉师爷嗷嗷直叫,浑身冒出白烟,我和老痒将他的衣服剥开,只见背上有几处已经焦黑,幸好冷汗出了不少,起了点保护作用,总体来说不算严重。我打开水壶,将半壶水浇在他背上,给他降温,然后抬头去看四周的形势。

我们所处的空地已经给火墙阻隔,外面乱成一团,热浪袭来,身上所有的毛都发出卷曲的声音,不少骨头大概是因为里面气体蒸腾的关系,不停地爆裂,骨碎子飞起半空高。我一看大势已去,尸洞必然被完全焚毁,这里地处低洼,等一下氧气说不定会给烧光,不焖死也给烫死了。

正在抓狂的时候,老痒一把拉住我,大叫:大事不妙,抄——抄家伙,阎王爷点名来了!

我不知道他是什么意思,转头一看,忽然见六七只大耗子给火烧疯了,竟然蹿过火墙,直奔我的面门就咬了过来,我一猫腰躲了过去,老痒不等它们再次扑来,一枪将一只打飞,我举起熄灭了的火把,当成武器也将扑过来的几只敲飞,可是同时,另十几只耗子闪电一样窜了出来,这一次我离得太近,背上给抓了几下,立即滚倒在地上。老痒又是几枪,将它们逼退,我抬头一看,乖乖,火墙外面,已经全是大大小小的耗子,给烧红的眼睛全部都直勾勾盯着我们。

我心里直叫不好,跳进来的这几只耗子被老痒的枪声震慑,暂时不敢靠近,但是在火墙之外的那些,见我们所站的这块地方似乎不会给烧着,必然会一只接一只地舍命冲进来,数量越来越多,再过几分钟,等到它们发现自己数量占了优势,必然会一拥而上,将我们吃成骷髅。

我看在这里硬拼就太不值得了,拉住老痒,让他暂时别去管这些耗子,最重要的是想办法出去,这时候凉师爷对我们大叫:“这里有个盗洞!”

我们回过头去,看见土坑的中心,有一个不起眼的小洞,不知道是谁挖的。老痒忙退出弹匣,看了看子弹,把枪塞给我,然后背起凉师爷就往坑的中心走去,我一手拿枪,一手拿拍子撩,跟在他后面。

才走了没几步,最近的几只老鼠突然尖叫一声,闪电一般扑了过来,我抬手连开了四枪,打中了三只,还有两只已经扑到了我的面前,我再无办法,甩出拍子撩,一声巨响,将两只老鼠凌空打成了肉泥。

\chapter{秦岭神树}

因为是左手开的拍子撩,加上拍子撩后坐力大得吓人,这几枪之后,我只觉得虎口发麻,手竟然举不起来了,不过好在声势惊人,就连老痒也吓得几乎一个踉跄,那些老鼠一下子退了下去,不敢再贸然攻击过来。

我一看这是个机会,忙催促老痒快点,拍子撩近距离威力巨大,但是子弹有限,就算一枪打死十只,也远远不够。下一次再开枪,就不知道有没有这么好的效果了。

思索间已经退到土坑的中央,我往下一看,地上果然有一个黑幽幽的洞口,依稀可见土表下面的砖层。老痒吃力地将凉师爷塞进那个洞里,正贴着他的脊梁骨一溜到底,他手一松,凉师爷就掉了下去,接着他也一猫腰,双手撑着地跳了下去。

我在后面殿后,听到里面老痒大声招呼我,才学着老痒,单手撑地跳入洞里。

下去还不到半个身子,双脚着了地,打起打火机一看,老痒正焦急地等我下来,凉师爷摔在一边,不知道死活。

我将打火机交给老痒,让他找点东西照明,自己捡起地上一些兵器,胡乱将下来的口子堵住,防止老鼠进来。

老痒点燃墓室四周墙上的火把,四处一照,发现这里是一个明显蛇国风格的石室,石室四周全部用条石做壁,上面全是色彩斑斓的壁画,顶上是条石镶嵌青砖,只是因为潮湿的关系,几乎目力能及的地方全部都有霉斑的痕迹。

石室很小,除了一些兵器和工具,什么陪葬品也没有,石室的中心,也没有棺椁,但是地板上倒有棺材放置过的痕迹。

此外也没有看到通往其他地方的甬道,我只是粗略的一看,就不禁奇怪,难道外面这些死人要保护的古墓,就是这么屁股大、什么都没有的地方?

热气从顶上喷下来,我们感觉到氧气不够了,壁画因为温度的关系,颜色越加艳丽起来,让人不敢正视。我们心里都知道,待在这里虽然可以暂时保命,但是也不是长久之计。

我喝了几口水,然后去看凉师爷怎么样了,一摸他的额头,发现他全身滚烫,气息微弱,是体温过高的症状,忙将剩下的半壶水给他灌下去,老痒掐了几下他的人中,总算把他掐得缓过来。

外面的老鼠已经疯了,围在盗洞口拼命地嘶叫,拼了命的想进来,无奈洞口全是青铜的利器,它们怎么钻也钻不进来。

老痒四处转了几圈,发现没有出口,便问我这里会不会也有秘道?要真没有,我们这一次就得蒸成人干了。

我看了看四周,几乎没有什么地方可以设置机关,这里太小,一目了然,刚想说不可能,忽然喀喇一声,盗洞口的东西塌下来一块,一只老鼠竟然咬碎了一块砖,直往缝隙里钻来,可惜脑袋太大卡在了两块砖头之间。

这些耗子咬不动青铜,竟然开始咬四周松散的青砖,我心里暗叫不好,这些青砖虽然也很结实,但是到底不比金属,耗子不要命地咬起来,说不定也能给咬开。

我拣起一把长矛,将那老鼠顶回去,然后大叫老痒帮忙,老痒忙把自己的外衣一脱,用兵器挑着塞进盗洞口的缝隙里。

可是他那衣服不顶用,没顶几下,就被那耗子咬破了个大洞,接着十几只耗子顺着长矛的杆子就爬了下来。

我们赶紧撒手,那几只耗子跳到地上,也不来攻击我们,反而朝一处墙角冲去。

老痒一看,忽然恍然大悟,大叫:“老吴,它们是在找路逃跑!快跟着它们!”

我们忙冲过去,发现那边墙脚竟然有一个不起眼的耗子洞,趴下身子一看,墙后面,竟然好像是空的。

老痒不由分说,扯起地上一把铜锤,轮起来就朝那墙砸去,只一下,石板子就裂了,墙上出现了人头这样大的一个洞,我们探进去一看,后面竟然还有一个石室。

“我靠,原来这里的秘道要靠砸的!”老痒叫着,又砸了几锤子将洞砸大,我们两个扛起凉师爷就爬了进去。

隔壁的石室里面没有任何的装饰,只是石室的中心有一个四方的直井通往下面更深的地方,下面没水,那些老鼠毫不停留,直接就跳入到直井里面。

后面传来墓室的砖顶开裂的声音,回头一看,用来封砖的铅水已经软化,这里的墓室很快就会坍塌下来,我和老痒心一横,死就死吧,咬着牙跟着老耗子跳进了井里。

那井有轻微的坡度,我一路滑下去,重重摔了一下,然后又是一滚,摔到了一块平地上。想到老痒和凉师爷就在我后面,忙往边上一挪,果然,老痒一屁股摔在了我刚才站的那地方,接着是凉师爷压到了他的身上,把他压得怪叫起来。

上面传来一声轰鸣,然后是剧烈的震动,墓室终于给火烧塌了,炽热的石头从我们掉下来的地方倾泻下来,直朝我们劈头盖脸地砸过来。

老痒抱着头坐起来,问我道:“这里是什么地方?”

我举起老痒从墓室中拿来的火把,转头一看,还是四方形的井道,只不过横了过来,道:“是古墓的排水井,排水系统的一部分。”

老痒看了看四周复杂的井道,问道:“那我们现在往哪里走?”

我看了看他,心说我怎么知道,这时候几只耗子从上面滑落,从老痒的肩膀上跳了下来,一下子跑进前面的通道中。

我心里一动,忙道:“跟着它们!”说完赶紧向前追去。

那几只耗子爬得极快,很快,便带我们过了好几个转弯口,我们几乎快跟不住它们了。我们连滚带爬地跟在后面,坚持了足有十多分钟,忽然,前面吹来一阵微风,那几只耗子一闪就消失了。我还没明白怎么一回事,立即脚下一空,几乎是滚着冲出了排水井。

我不知道外面是什么环境,忙一个翻身站起来,这时候老痒他们也跟着摔了出来,四周一片漆黑,我忙举起火把去照。

四周豁然开朗,这里不是墓室,而是一个巨大的圆形直井的底部,直径大概有六十多米,底上凹陷成一个深坑。石头井的四周都有火架子,我上去点燃了几个,将四周照得更亮。

边上的直井壁明显有开凿过的痕迹,显然这个圆井是人工造成,只是他们挖到这么深干什么呢?难道这里也是上面采石洞的一部分?

我隐隐约约还看见坑的中心竖着一根什么巨大的东西,可惜光线不够看不清楚。这里的温度很高,一股滚烫的劲风由上而下吹来,吹得人头昏脑涨,连站立都不稳。

我举起火炬,让老痒背着凉师爷走到坑里,在火把的照明下,坑里的情况一清二楚。

坑里东倒西歪的全是外面看到的人头石俑,几乎有百来具,人头都已经风干,坑中间竖着的,是一根直径十米左右的大青铜柱子,乍一看还以为是一道有弧度的青铜墙,直上而去,高不可攀。

青铜柱子的底部直直插入到坑底的石头里,好像是从那里长出来的一样,将四周的岩石都胀裂出许多条裂缝。

青铜柱之上还有很多细小但是粗细不一的铜棍,与老痒带着的那一根非常相似,我估计了一下,密密麻麻不下千根,再往上不知道还有多少。整个青铜柱的形状,就犹如一棵从石头中长出的大树,枝桠繁盛,直插地表。

凉师爷看得心里发凉,从老痒背上下来,说道:“建造这里的人一定是想把这青铜树挖出来,你们看这里的边上开凿的痕迹,竟然挖到了山底还没有找到尽头,那这青铜柱子,不知道插到地底下有多深。”

我看着心里也发寒,这样巨形的金属器,早就超出了当时的冶炼水平,那些厍族的先民,不可能有这样的技术,可如果不是他们铸造的,那这青铜树,又是谁立在这里的?难道是从地狱里长出来的?

这时候,凉师爷突然拍了我一下,我转头一看,发现一直没说话的老痒,正直勾勾盯着那青铜树,径直走了过去。

\chapter{继续爬}

我看到老痒的表情不对,心里闪过一丝异样,忙大叫了一声他的名字。老痒给我吓了一跳,一下子反应过来,打了个哆嗦站在了原地。

我们俩忙跑过去,问他刚才想干什么?

老痒看了看这棵树,又看了看我们,疑惑道:“我也不知道,真奇怪,刚才我一看到这树,就好像习惯一样,突然想……爬上去。”

爬上去?我怀疑地看着老痒,抬头看了看这树,心说你又不是猴子,怎么看树就爬,问他:“是不是给这东西的气势所感染了,一般人看到高的东西,都有想爬的冲动。”

老痒摇了摇头:“我也不知道。”

凉师爷看了看这青铜树,说道:“这东西这么大,有点邪,咱们看的时候小心一点,尽量别去碰它。”

老痒点点头表示同意,我举起火把,向青铜巨树的根部走去。

青铜树是比较稀少的文物,我记忆里除了三星堆里出土过之外,其他地方好像没有,我也是从纪录片中稍微了解了一下,考古界对此成因并没有定论,说法很多。

贴近去看,可以发现青铜树的表面并不光滑,上面刻满了双身蛇的图腾,象征着青铜器的神性。

凉师爷看了半天,对我说道:“这么大一家伙,估计是个祭器,商周左右的东西,具体在祭祀的时候干什么用,太古老了,超出我的见识了。”

这和来之前老头子给我说的很接近,不过商周左右,商就是六百多年,周五百二十二年,加起来就一千一百多年了,左右一下,加上个夏四百多年,几乎占了整个中国有记载历史的一半,这个判断等于没说。

我问他能不能精确点,能不能看出,到底是商周哪一段?

凉师爷摊了摊手说没办法:“这东西肉眼看不出来,在下只能给你猜。你看锈色偏黑灰,可能是锡青铜、铅锡青铜和铅青铜中的一种,西周的可能性最大,大概能有个五成。另五成我就说不出来了,你也知道我们这一行的规矩,我知道这些已经不错了,再往深里讲在下只能瞎掰。”

做古董这一行在朝代上有一条分界线,大量的古董都是宋以后出的,唐以前的东西少,商周更是干脆就几乎没有,业内对于这种东西的认识不多,凉师爷的确算是不错了,比我强多了。

我听他说了这么多,仍然没什么概念,问道:“那就按照西周,您能不能给判断一下,西周的青铜工艺水平,理论上能不能铸出这种东西来?”

凉师爷说:“这问题我更回答不了,我只知道那时候青铜器要先做陶范(陶制的模具),理论上说只要能做出陶范来,就有可能铸出成品,不过这东西太大了,恐怕用传统工艺是做不出来的。”

老痒问他道:“师爷,你说这东西会不会是什么史前文明的遗迹,我在报纸看到了,有些几亿年前的煤矿里还挖到铁钉呢,这东西这么大,那时候的‘人’估计做不出来吧?”

凉师爷摇了摇头:“两位小太爷,这我还真觉得不一定,公元前1000年到公元元年左右历史上叫奇迹时代,很多不可能的东西都是那时候建造出来的,像长城、金字塔、秦始皇陵、巴比塔,你要说这一根青铜树不可能铸出来,那也很难说,毕竟那时候咱们老祖宗已经会铸青铜器了,皇帝一声令下,下面人蒙头苦干,用个几十年,也不是没有可能。”

凉师爷说的有点道理,不过当时冶金业低下,有这么多的青铜可以利用吗?秦始皇收天下之兵才铸造了十二金人,这一棵树,恐怕能铸上百个了,这么多的青铜是哪里来的?

我想来想去,也没想出个所以然来,倒是想起了另外一件事。我们在偷听李琵琶说话的时候,听到他说过,这个古墓里的东西,比秦始皇陵还好,可我们一路下来,也没看见什么好东西,这里也到头了,要说好处就是这棵铜树,可我们又不是收破烂的。

虽说这树也够一千个收破烂的忙活一辈子了……

他的《河木集》上一定写了什么东西,能够吸引他到这里来,他这种人宝贝见多了,能让他说那种话的,这东西肯定非同小可,可这东西到底是什么,在这里的什么地方呢?

照理,这里应该是整个古墓,或者神迹的中心了,要有好东西,也应该在这附近,可是除了这棵树,这里肯定没有任何东西是李琵琶这种人看得上眼的……等等……树?

我想着,忽然灵光一闪,抬头看了看头顶,心道:会不会吸引他来的东西,是藏在了这棵青铜树的树上了?

这种巨大的青铜树,对于古时候蛇国的先民来说,无意是极其浩大的工程,可以说是神迹,难保他们的王不会把自己的陵墓设在他们认为最靠近神的地方,那如果这的确是一个古墓的话,墓主人的棺椁也应该在青铜树上,所有的明器也应该在这上面。

我把我的想法和其他两个人一说,他们都觉得有道理,我问他们,那既然这样,要不要爬上去看看?

老痒当然是同意的,说道:“都到这份上了,爬几步有啥大不了的,这上面这么多棍儿,和爬楼梯似的,不用使多大力气的。”

我也不介意爬上一段,只是凉师爷刚刚给火烤了,又体力透支,再让他上树,恐怕他这条小命就交待了,要是瘫在不上不下的地方,我们还得照顾他,实在没这个闲力气。

我转过头去,想对他说要不在下面等我们,我们两个上去就行了,却看见凉师爷用力揉了揉脸,然后一拍我:“没事,最后一关,怎么也要去看看!”

我看他眼神坚决,知道是劝不动,无须做无用的尝试,于是将背包扎紧,举起火把,对老痒说:“那咱们就继续。”

老痒带上包里的手套,当下第一个踩着铜树上的枝桠,开始攀爬,我和凉师爷也学他的样子,跟在后面,跟着他落脚的顺序一路向上。

上面的枝桠不紧不密,爬起来相当顺手,老痒一边爬,一边提醒我们注意下一步的动作,不要大意踩空了。

贴着青铜的树壁,我看得更加清楚。这些伸展出来的树枝都是与这根躯干同时铸出来的,接口处完美无瑕,没有一丝锻痕。不过,让我觉得意外的是,上面的双身蛇之间的缝隙很深,似乎一直刻到躯干的深处,我都看不到雕刻沟里面有什么。

因为太过在意动作,我们很快汗流浃背,气喘如牛。我向下望去,发现看不到底上的坑,只能看到门边上的火坛微弱的光芒,这么点高度,看上去却是无底的深渊。

爬了一会儿,凉师爷就体力不支,我招呼老痒停了下来,打了个手势让他别急,让凉师爷休息一下。

凉师爷如获大赦,一下子就蹲了下来,他累得够戗,汗都是淡的,脚颤颤悠悠,几乎都站不稳,我坐在枝桠上,双脚荡在半空也很不踏实,根本没办法很好地休息。

老痒看我们太紧张了,把干粮丢给我们,让我们嘴巴里嚼着,对我们说道:“你们这个样子可不行啊,这上面还有百来米呢,就这个体力,没准我们得在树上过一夜,要不,老吴你给咱们讲个荤段子放松一下?”

我累得都不想说话,骂道:“去,你就不累?你看你小腿哆嗦的,要说荤段子自己说,老子没这个力气。”

老痒咬了一口玉米饼子,说道:“我讲就我讲,不过你得先回答我一个问题。老吴,你说咱们发现了这东西,要是通知政府,能不能用咱们的名字命名啊?”

我对这倒真是一点都不知道了,转头看凉师爷,凉师爷喘着气摆了摆手:“这位痒爷,你有没有听过有什么东西给叫成王二麻子方鼎、赵土根三脚觚的?历来国宝的发现人都是农民和建筑工人,你要以他们的名字命名,那就有趣了,咱们也不是歧视劳动人民的意思,不过中国人的名字不像老外,直接拿来用,你不觉得寒得慌吗?”

老痒想了想,觉得有点道理,又问:“那至少也给我个命名权,对吧?那个谁发现个岛屿不都是可以由第一发现者命名的?”

凉师爷说道:“那好像是有这么个规定,不过我还真没去研究过。”

我问老痒道:“干啥问这些,你钱都没搞利落,还想名利双收啊,你也不想想一个人没事能找到这种地方来吗,你干什么的还不是一目了然。”

老痒说道:“我是觉得这玩意挺有意思的,你说这么大根铜柱子,给取什么名字好呢?你们也给想想,以后咱们吹起牛来也好统一口径。”

我这时候不想再动这些无聊的念头,对他说道:“算你第一个发现,该你取,我没你这么有心情。”

老痒看了看上面,说道:“我一看到这东西,脑子里就闪现出一个词,你看这一根柱子,叫‘我爱一条柴’怎么样?”

我没好气道:“你是不是没营养的片子看多了?你爱一条柴,起这名字,信不信出去能有雷劈你?”

老痒当下一笑,凉师爷也乐得直摇头,这一笑间,人总算是放松了下来。

我们吃完之后,力气恢复了不少,老痒就催促着继续赶路,我抬起脚刚想走,忽然发现底下好像有什么不对劲,仔细一看,咦?门边上的火坛子怎么灭了。

老痒皱了皱眉头:“该不会是给这里的风吹熄了吧?”

我摇摇头,说不会,这火坛子火头这么大,比我做的那个不知道专业多少倍,不可能给风吹熄灭了,下面该不是出了什么事情?

正想着,忽然整棵铜树轻微地震动了一下,好像给什么撞了一下,凉师爷吸了一口凉气,忙问怎么回事?

老痒对我们做个噤声的手势,然后把手做成喇叭状贴在铜壁上,一听之下,脸色大变,对我们轻声说道:“他娘的,好像有东西上来了!”

\chapter{裂痕}

我心里一紧,想到了泰叔,我们从瀑布上冲下来之后就一直没有他们的消息,难道现在已经跟过来了?一想之下又不对,外面火龙阵一时半会儿熄灭不了,墓室也塌了,他们应该过不来;第二,要爬上来,那就得有照明的工具,下面的火把熄灭了,又没手电的光点,他们没有理由摸黑上来。

那上来的到底是什么?

想到这里我就冒上冷汗了,我们现在凌空不过是十几米,活动的空间有限,不好做太大的动作,真要是遇上啥离奇的事情,不知道该怎么应对。

老痒给我使了个眼色,意思要不先下手为强,冲下去看看。我摆了摆手让他冷静,现在敌暗我明,绝对不能莽撞,要真是泰叔他们摸黑上来,下去一个照面免不了就是一番恶战,子弹不长眼睛,这么近的距离,说不定就会两败俱伤。想到这,我心里一转,有了一个计划,当下取下自己的皮带将火把绑在一根枝桠上,然后招呼老痒和凉师爷,躲进火把照不到的黑暗里。

下面人看我们,只能看到我们的火把光线,如此一来,我们也隐入到黑暗之中,反而可以反客为主,打他们一个措手不及。

三个人各自屏住呼吸,用手做成听筒,贴在铜壁上,可以感觉到一种很轻微的颤动声正在由远而近,频率又乱又快,好像有很多的人不停地在用指甲挠着铜树上的纹路。我听着越发觉得不妙,泰叔他们只有两人,恐怕无法可以发出如此密集的声音,难不成是耗子跟进来了?

我心里后悔刚才没有好好处理那个盗洞,暗骂一声,将拍子撩也交到右手上,站在我上面的老痒也子弹上膛,两个人准备随时暴起发难。

来者行动非常迅速,毫不犹豫,转眼已经来到我们身下。只是还没进入火把的照明范围,我只能隐约看到几个模糊的影子,似乎是人,又似乎不是,我紧张得手心冒汗,精神高度集中,这几秒钟,时间好像停止了一样。

突然间,最下面老痒的脸色变得极端惊恐,大叫:“我操!上上上!快上去!”不等他说完,凉师爷似乎也看到了什么,发出了一声非常凄凉的惊叫,两个人见了鬼一样地向上飞快逃去。

我不知道他们看到了什么恐怖的东西,下意识地往下一望,发现黑暗中有什么东西正在蠕动,却看不清楚。老痒看我不动,大叫一声:“老吴,你他妈的傻站在那里干什么,快跑!”

我发现他的脸色极度苍白,心里打了个寒战,也顾不得弄清楚是怎么回事,拔出火把,咬紧牙关就跟了上去。

我给老痒他们的表情感染,心里紧张得要命,又不知道爬上来的到底是什么,越爬越觉得浑身发凉,越凉就爬得越快,最后完全陷入到一种疯狂的状态中去,只觉得头皮发麻,浑身僵硬,脑子里只想着跟在他们后面,其他什么都顾不上了。

足爬了半支烟工夫,前面的凉师爷终于停了下来。我爬到他的身边,发现他不是不想爬,而是实在爬不动了,脸上毫无血色,整个人已经到了极限。

他汗如雨下,看我还要向上,竟然一把抱住我的腿,对我说道:“等……等一下!别……别丢下我,我……我只歇一下,就和你一起爬!”

我给他拉得一停,只觉得腿一软,竟然也使不上力气,不听使唤地开始发起抖来。

刚才游泳、攀悬崖都是在极度紧张的环境下做出的高强度运动,肌肉早就不堪重负,现在又是一路极其耗费体力的爬高,没意识到还好,人一停下来,肌肉马上失去控制,就算咬紧牙关也没有办法。

我心急如焚,却无处发力,往上一看,黑漆漆的不知道还有多高,不由心里发寒,心说这样爬要爬到猴年马月去,就算爬到了顶又能如何,还不是一场大战,到时候体力更差,说不定连枪都举不起来。想到这里我把心一横,顺手将火把递给凉师爷,同时甩出拍子撩对着下面,对他说道:“爬个屁!他妈的老子也爬不动了,算了,管他娘的是什么,和他拼了!”

凉师爷听我这么说,脸孔都扭曲了起来,几乎就要晕倒从青铜树上摔下去,我赶紧将他扶住,四处一望,发现老痒不知道哪里去了,忙问他:“老痒呢?刚才是在我们上面还是下面?”

凉师爷连说话的力气都几乎没有了,摆了摆手,指了指下面。

我记得刚才爬的时候,我们一路狂奔,老痒看我拿着火把,为了给我殿后,的确让我甩在了下头,急忙让凉师爷将火把探下去查看。这一照之下,却几乎没把我的魂魄吓飞,只见下面的黑暗中,有一个人像猴子一样趴在青铜树上,毫无表情地看着我们。

这人脸足有普通人的一个半大,五官犹如石头雕刻的一般,一点人气都没有。凉师爷将火把探下去的时候,它忽然向后缩了一下,似乎忌讳靠近火焰。然而同时它的脸上,却露出一种似笑非笑的表情,极端的诡异。

我看到这张脸,心里打了个哆嗦,心说老痒在我们下面,现在不见了踪影,难不成已经遭殃了?但随即想到,若是已经遇难,他有手枪在手,怎么样也要开上几枪,没有听到声音,或许是在下面躲起来了。

凉师爷看到这张脸,魂飞魄散,怪叫一声向上飞快地逃去,我想阻止已经来不及了,回头再看下面,猛然发现那张怪异的巨脸已经贴了上来,几乎就到了我的脚下。

刚才远远看还好,现在一下子离得如此近,只见整张脸在我脚边狞笑,出其不意之间如何不慌,我条件反射般地甩手就是一枪,就听“砰”一声巨响,拍子撩吐出一条火舌,正中巨脸的面门。

这一枪距离太近,铁沙弹直接将整张巨脸轰得粉碎,牵扯力将巨脸的身体扯落青铜树,跌落到了黑暗里。

我没想到手枪如此奏效,当下松了口气,正想上去拉住凉师爷,突然从巨脸跌落的地方,又探出两张惨白的大脸,我大惊失色,甩手又想开枪,可是连扣两次扳机,都没有反应,随即想到这拍子撩只能装两发子弹,打完之后必须手动退弹装弹才能继续使用。

可是现在的情形根本无法容我这么从容地装填子弹,我刚掰开弹膛,一只爪子就已经搭到了我的肩膀上,我一回头,正看见一张巨脸贴着我的鼻子凑了过来,原来有一个东西不知道何时已经绕到了我的背后。

凉师爷已经将火把带远,光线逐渐昏暗,我看不清楚这人的五官,也没办法判断这到底是什么,只好狗急跳墙,一脑袋撞了上去。

这一下我是用了十足的力气,没想到这脸就像石头一样硬,撞得我脑子嗡的一声,几乎要从树上摔落下去,这时候突然听到老痒不知道在哪里叫了一声:“躲开!!”同时砰一声枪响,一道火光呼啸而过,打在我脑袋边上的铜树上,溅起漫天的火星。

我给这一枪震得几乎蒙过去,急忙退到一边,一摸脸蛋,马上骇然不止——脸上竟然给子弹的气流划出了一道血痕。

老痒继续在下面开枪,一时间子弹乱飞,到处都是火星,可惜没有一枪打中目标,几乎全部都打到了铜树上,有几颗子弹还反弹了好几下,像弹珠一样在我眼前飞来飞去。

我再也无暇顾及那些怪物,左躲右闪,一边心里暗骂,老痒这家伙枪法太差了,再这样下去,他娘的今天搞不好会死在他手上。

不过这几枪却给我赢得了时间,那些怪人给子弹打得有些忌讳,纷纷退后,我乘机从拍子撩枪管下的铁盒子中取出两发子弹,塞进枪管子里,甩了一下上膛,对准最近的那张怪脸就是一枪,将它打得飞了出去,掉下铜树。

我眼前的威胁解除,马上低头去看老痒,却发现更多的怪物从黑暗里探出了头来,能看到的就已经有十几张巨脸,这些东西似乎看上我一样,几乎同时一动,犹如鬼魅一样向我包抄过来。

我看得心惊肉跳,实在想不出这些到底是什么东西,从它们躯体的形状来看,应该是人,可是人怎么可能用这种类似于猴子的姿势在攀爬,而且这些怪物脑袋这么大,已经超出正常人的范围了。可是,如果不是人,那又会是什么呢?

转眼间两只怪物跳到了我的边上,一只抓住了我的脚就向下拉,另一只直接趴到了我的脖子上,我知道不可能再有换子弹的机会,当下变枪为锤子,朝那贴上来的怪物脸就是狠狠的一下。

我本想将这怪物打下树去,它却只是后仰了一下,马上又贴了过来,这个时候,我突然发现那张巨脸喀嚓了一声,竟然出现了一条裂痕。

\chapter{摔死}

我愣了一下,心说这是怎么回事,怎么脸还能开裂?皮肤干成这样?可没等我仔细看,下面拉着我脚踝的怪物突然发力,把我拉了一个踉跄。这东西力气很大,我根本没办法和它硬抗,只好顺着它的力气跳了下去,紧接着一手抓住附近的青铜枝桠,另一只手贴着那怪物的喉咙就是一枪,“砰”一声将它的脑袋轰了下来。

这枪开得实在太勉强,巨大的后坐力几乎把我从枝桠上甩了下来,我咬紧牙关才确保人枪不失,这一边无头的尸体给枪的冲力掀离了青铜树,可是它的手还死死抓着我的脚,整具尸体挂在我的脚下,将我直往下拉去。

我单手无法吃住两个人的重量,咬着牙低头想找一根能够搭脚的枝桠站稳了,再想办法将那尸体甩下去,这时候才给我打裂脸的那一只怪物突然倒挂了下来,一爪子卡住了我的脖子,就将我向上提去,我的脖子像给裹了紧箍咒,连一丝空气都无法进去,脸马上就憋得通红,情急之下我抡起拍子撩朝它的脑袋乱砸。

我是用了死力气,那几下要是砸在人脸上,肯定就全烂了,那怪物也给我砸得蒙了,头不停地乱晃想要躲开,我一记重击正巧打在了那怪物脸上的裂缝上,它怪叫了一声,突然松开爪子,跳到了我头顶上方的枝桠上,发狂地抓起自己的脸来。

我失去支撑,重量全部回到我的手上,一下子没抓住,脱手直坠下去一米多,忙抱住一根突出的青铜枝桠停住身体,抬头一看,只见那怪物的脸竟然完全碎裂了开来,变成了一小片一小片的白色碎片,开始像奶皮一样脱落。

很快,所有的白色碎片全部掉了下来。我接住一片,竟然是石头的,难道这些人都是雕像吗?又抬头一看,只见石头脸脱落之后,里面竟然还有一张长满了黄毛的脸。

我仔细一看那脸,突然恍然大悟,对下面大叫道:“老痒!我知道这些狗日的是什么东西了,这些他娘的都是些猴子,大个的猴子!”

老痒在下面的黑暗里,看不清楚是什么状况,只听到他回道:“猴你爷爷!哪有猴子长人脸的,那不成精了!”

我大吼道:“那不是人脸!那是面具!这些猴子带着石头人脸面具!”

老痒已经从下面的黑暗中爬了上来,身上的衣服几乎都给撕成一条一条的了,朝我大叫:“甭管是什么了!猴子又怎么样,你打得过吗?”

我朝他身下一看,只见下面黑影幢幢,不知道有多少这种带着面具的猴子正在追上来。我又爬上几米,打开弹匣一看,红色的子弹已经用光了,只剩下几发蓝色的,大概不是铁砂弹,而是那种大钢珠子弹,这东西远距离的威力不错,但是不如火炮一样的铁砂。我一看猴子跟了上来,忙双手握住枪柄,向下连开了两枪。

钢珠子弹发散了出去,威力减少了很多,但是大范围杀伤的效果还是发挥了出来,最近的几只猴子给打得血肉模糊,远处也有不少中弹,要是能够五发连发,我甚至可以把这些东西全部都干掉。

猴子们似乎给拍子撩的威力震慑住了,全部放慢了逼近的步伐,转身跟着老痒去追凉师爷。那只给我打破面具的猴子,看到我们,竟然开始害怕,朝我们一龇牙,飞也似的向一边退去。老痒奇怪地看了看我,问道:“我靠,还真是猴子,这是怎么回事?”

我心里也觉得非常奇怪,这些猴子的面具是谁给它们带上去的?又为什么要带?面具上面既没有眼洞,也没有嘴洞,这些猴子平时怎么生存啊?

凉师爷已经拉下我们十几米,现在正趴在那里喘气,我们很快赶上了他,发现他已经神情恍惚,幸好那个地方枝桠密集起来,他整个人架在那里,不至于掉下来,火把落在他身下半截的地方,卡在三根枝桠之间。

老痒过去拿起火把,另一手抬起将那只没面具的猴子打落,手枪子弹算是完全告罄,他随手就想将手枪砸下去,可手举到一半,又有些不舍得,将它插回到皮带里,然后举起火把对着下面挥动,想用火焰把这些猴子逼退。那些猴子果然有一些畏惧,火把扫过的地方,它们全部都往后缩去,可是火把一挪开,它们又迅速地压了过来,一点也不给我们喘息的机会。

老痒在那里挥了半天,非但没有将它们赶开,反而包围圈越来越小了,我扯了扯凉师爷,像一滩烂泥一样动也动不了,老痒大叫:“别管他了,顶不住了,撤了!”

我急火攻心,真想一脚把凉师爷踢下去算了,可是这家伙也不是什么穷凶极恶的人,这时候我还真下不去手。我将他抬起来,用力向上拉了一下,但是他的屁股反而从两根枝桠之间掉了下去,情况变得更糟糕。

老痒用火把将一只猴子吓开,对我大骂道:“该死!你到底在干什么,这家伙不是我们一伙的,要是一切顺利,说不定他已经把你给宰了,你他娘的别在那里搞优待俘虏。”

我装上子弹,又是两枪,两声巨响掀飞了五只猴子,将猴群逼退了将近六米,然后甩抢换上了最后两颗子弹,刚想打完算了,突然凉师爷一把抓住了我的手,有气无力道:“这些东西怕火,信号弹……”

我一听猛然醒悟,老痒反应很快,回手已经掏出信号枪,瞄了瞄问我:“怎么打,直接打下去没用的!”

我夺过信号枪,对着对面的岩壁就是一枪,信号弹闪电般打在几十米外的岩石上,又反弹回来打在青铜树上,如此闪电般反弹了两三次,突然在猴群中炸亮,极高的温度一下子将那些猴子烧得乱窜起来。我不等第一发熄灭,又连射两发,一下子整个空间亮起了刺眼的白光。

老痒给照得眼睛发花,几乎要掉下去,我将他的头掰到一边,大叫:“别看!距离太近了,比电焊还厉害一百倍,会烧坏视网膜的!”

三个人同时闭上眼睛,但是仍旧能够感觉到那种光线几乎刺入眼皮,猴子们给强光照得发了疯,只听下面一阵混乱,同时传来一股皮肉烧焦的臭味。

也不知道过了多久,强烈的光线才暗下来。我眯开眼睛看了看下面,猴子已经不见了,我的眼睛给烧得灼痛,看东西非常的模糊,老痒更是眼泪直流,拼命地用手去揉,凉师爷这次彻底晕了过去,要不是我拎着他的领子,他早就掉下去了。

我看到猴子不见了,松了口气,也不知道它们是害怕高温,还是怕这种强光,如果它们当时对着这些强光直视,那十有八九已经全部暴盲。没有十天半个月恢复不了,我想着松了口气,把凉师爷拍醒,一把架住他的胳臂,将他的身体抬直,想拖着他往上,不过这家伙实在是太次,我只能将他扶正,要让他离开原来的位置,却一点办法也没有。

他坐稳之后,我又缩到一边去看老痒,他眯着眼睛,一边骂娘一边吐口水,不过总算是能看见了,问我道:“你他娘的做事情之前就不会知会一声,要是把我给搞瞎了,我和你拼了。”

我骂道:“他娘的你还有脸说这些,我救了你的命知道不?再说你这不没瞎吗?”

老痒看了看下面:“别说,这一招还真管用,猴子跑了还是都烧死了?”

我对他说恐怕烧死是不太可能,大概是暂时退下去了,说不定还会再上来,不过我们既然发现了对付它们的办法,也就不用再怕,信号弹还有几发,足够应付几次的。

这猴子带的面具,做工精细,雕得简直和真人一样,难道与我们在山崖上看到的那一尊写实的雕像有关系?可是它们为什么攻击我们?

我以前倒是看过一本小说,说是有古代文明训练大猩猩来守卫矿井,这些大猩猩在古代文明毁灭了之后,仍旧将自己守卫矿井时所受的杀戮训练通过教育传达给了下一代,这样一直到几千年后,大猩猩的后代们仍旧守卫着矿井的遗迹,将来探险的探险队屠杀殆尽。

可这些是猴子,显然没大猩猩这么聪明,应该做不到这么高难度的事情,我本想问问凉师爷,可看到凉师爷的面色,我知道问了也是白搭,这人完全处在崩溃边缘,要是再不休息,恐怕就此要报废了。

我们在那个地方待了有十几分钟,再没有看到猴子从下面探出头来,总算松了口气。老痒拿出一些食物,又想让我们吃,我们都拒绝了,现在不是肚子饿的问题,而是缺乏休息的问题,你就算给我直接吃葡萄糖我也走不动。

我靠在几根枝桠上,头枕着背包,不知不觉就开始打起瞌睡来,老痒和凉师爷迷迷糊糊地,也没有阻止我,就在我即将睡着的时候,突然一连串的撞击声从上面传了过来,同时整棵青铜树剧烈地震动了起来,似乎有一只巨大的怪物正在爬下来。

我心说坏了,刚搞定猴子,又惊动了什么大家伙,难不成“金刚”从上面下来了?正不知道往哪里躲好,突然一道黑色的影子闪电般落下,狠狠撞进三棵枝桠之间,一股腥臭的液体溅了我一脸。

这一下撞得非常厉害,整棵青铜树都为之震动,几乎把我震得掉下去,我们三个全部都给吓了个半死,好久才反应过来。

老痒最先冷静下来,举高火把招呼我们过去看看是什么东西掉下来了。我们走近一看,发现那竟然是一个人,给卡在了青铜树桠之间,身体非常不自然地扭曲着,眼睛瞪得老大,满脸是血,肋骨破体而出,一看就知道是高空摔下来摔死的。

老痒将火把探过去照了照他的脸,忽然叫道:“我操,是那龟儿的泰叔。这老家伙原来在我们前面,难怪一直没看到他们!”

凉师爷颤抖着靠过去,看了看上面,又按了按泰叔的胸口,一股血从尸体的嘴巴和鼻子里涌了出来。他叹了口气,说道:“高空坠死,内脏都碎了,怎么会摔下来,这么不小心?”

我看了看他的脚,骨头已经戳了出来,浑身几乎都很不自然地扭曲着,应该是摔下来的时候不停地撞到那些青铜枝桠造成的。凉师爷又按了按他的四肢,吸了口凉气道:“两位,这上面看样子不是一般的高,你看泰叔,全部的长骨头都断了,没百来米摔不成这样。”

我心里不由暗暗叫苦。我们刚才这一通狂爬,大概也就上来了五六十米,已经累成这个样子,上面要真还有这么高,怎么爬啊。就算爬到上面,估计也什么力气都没了,搞不好就会像泰叔一样摔成十八截。

想到这里,凉师爷和我都露出了痛苦的表情。

老痒并不感觉到前途渺茫,看到我们这样子,忙拍了拍我的肩膀,说什么就算有几百米,横过来跑一下,几秒钟就完了,现在不过是竖了起来,又有什么好担心的。我说滚你爷爷的,照你这么说珠穆朗玛峰也才八千八百四十八米,你骑辆脚踏车半个小时也就上去了,咱们现在不是对抗摩擦力,而是在对付地心吸引力,知道不?

老痒对我摆了摆手,表示不想和我吵,说着就去解泰叔的背包,将里面的东西翻出来,看看有什么我们能用。一看之下,大喜过望,在凉师爷那个队伍里,泰叔和那个叫二麻子的年轻人背负着主要的设备,大部分的东西都在,手枪子弹、几根雷管、信号枪、绳子,最开心的是找到了一只手电,我操,一想到刚才在千棺洞里怕火把熄灭要死要活的情况,我真想把这手电贴过来亲几下,高科技就是好啊。

老痒换了弹匣,将其他东西整理了一下,背到自己背上,对我们说道:“那群猢狲肯定还在下面,这地方不能久待,我们歇一下,马上就得上去,泰山诸位都爬过吧,一千三百米,还不是一天一个来回?没事,就当观光旅游。”

凉师爷脸色略有好转,苦笑了一声,用手指做了一个走路的手势,说道:“这位痒哥……泰山那是走上去的,用脚就行了,我们现在可是直上直下,这怎么能说到一块呢?而且那是五岳风情,有的是云海怪石,这里看什么啊。”

老痒踢了踢一边的青铜树身,说道:“老子他娘的是打个比方,这青铜树虽然比不上泰山的风景,但至少也壮观是吧,您两位就迁就一点,胜利就在眼前了,别泄气,赶紧收拾收拾,咱们咬咬牙,一鼓作气上到顶上,绝对是大好风景。”

我敲了敲自己已经开始发胀的小腿,对他说不是不想咬牙,实在已经没办法了,再咬牙根就从下巴里戳出来了。我尚且还能挤出点力气,凉师爷现在是剩下半条命了,与其急着赶这几分钟,不如歇个透效果还好一点。

凉师爷感激地看了我一眼,老痒叹了口气,说那行,不过得把这泰叔的尸体弄下去,放这里看着心里不舒服。

我看到泰叔那五官扭曲、死不瞑目的样子,心里倒没有什么特别的感觉,但是他那对暴出眼眶的眼睛,还真是有点可怕,这时候也不想婆婆妈妈的讲什么道德不道德,和老痒两个人小心翼翼地想将泰叔的尸体从枝桠上抬起来。

从这里的高空坠落,一路下来必然会撞到不少突出的青铜枝桠,没有直接掉到底下摔成烂泥巴算是运气不错了,我抬泰叔尸体的时候,发现凉师爷说得不错,尸体全身都软得离谱,似乎所有的骨头都碎了,一动之下,大量的血从他折断的身体里涌了出来,顺着枝桠流进青铜树上的纹路里,然后沿着纹路中间的沟壑向下面流去。

我和凉师爷同时看到这个现象,都愣了一下,凉师爷马上让我们停住,打起手电往沟壑里一照,又看了看那些青铜树桠,说道:“两位,在下大概知道这青铜树是干什么用的了!”

\chapter{祭祀}

我和老痒听到这么说,就一齐问他想到了什么。他挠了挠头,说道:“在下只是大概推测,这棵铜树可能并不是关键,起作用的可能是树上面这些沟壑,当时祭祀时候,这东西可能是用来收集一些液体,比如说雨水、血液或者露水之类的东西。”

老痒问他道:“是不是就像以前皇帝收集露水来泡茶叶一样?那叫什么,无根水?”

凉师爷用自己的钢笔在那些沟壑里刮出一些黑色的积垢,经过几千年的岁月,也无法分辨这些是不是先人干涸的血液还是雨水中的沉淀物。他又看了看这些枝桠,说道:“你看,这些枝桠下面也有像刺刀放血槽一样的东西,一直通到双身蛇路中,这枝桠在祭坛中必然也有功用。有可能,真是和血祭有关系。”

我不是很明白,就让凉师爷仔细说说,为什么说这些沟壑和当年的血祭有关,这种血祭又是怎么进行的。

凉师爷对我说,西周时代的祭祀虽然不如商代那么残暴,但是人牲是难免的,所谓不同的祭祀方式,只不过是把人牲杀死的方法不同而已,比如祭祀土地,就把人活埋;祭祀火神,就把人烧死;祭祀河神,就丢河里去。

这里这么一棵通天一样的青铜巨树,祭祀的可能就是扶桑若木之类的神树,也有可能是司木之神句芒,通常这一类神用的都是血祭。

刚才泰叔的血液顺着青铜枝桠,流进青铜树上的双身蛇中,一路往下,这样的一条线路,如果不是事先设计好的,根本无法运行得如此流畅。加上青铜枝桠上面的那些刺刀放血槽一样的痕迹,事情就很明白了,这里必然是用来进行血祭的祭器。

所谓血祭,大多数时候是以血入地。受祭祀的时候,必然是将牺牲钉死在这些青铜枝桠上,将尸体的血液引出,汇入到树身上的双身蛇路中。如果血液不在半途凝结,必然会一直流到这棵青铜树深深埋藏在岩石底下的根部,象征着以血来奉献给神的意思。

说得形象一点,整棵树的纹路就像医院解剖室里的引血槽,几张尸床上的血,无论多少,最后由这些沟壑汇进引血槽,然后流进下水管道。只不过这里的引血槽,被做成了看似用来装饰的纹路,这也正好可以说明,为什么这些双身蛇之间的沟壑,会深得如此离谱。

这样残忍而又大规模的祭祀,显然就算实力再强大的国家,也无法长期举行,所以古籍中也只是零星记载,至于具体仪式的过程,需要多少人牲,一切都无从得知了。

我听了凉师爷的话,一方面感叹古人的智慧,另一方面也感到一丝心寒,如此巨大的一个工程,竟然只是用来做一件杀人的工具,实在是愚蠢之极。想着无数奴隶给倒插在这些枝桠上面,血液顺着这些青铜的沟壑将整棵树变成一根血柱,我就感觉到似乎有刺骨的寒气从那些沟壑里渗透出来。

想着有点心虚,我对老痒说:“我们还是走快一点,不然等一下泰叔的血流下去,说不定那司木之神以为又有人来献祭了,老人家出来遛遛,说不准能把我们当祭品。”

老痒根本没把凉师爷的话放在心上,对我说道:“你也别尽相信他,中国那时候哪里会有这么多人给你杀着玩,我看这里叉着放血的说不定都是猪头羊头什么,咱们再爬上去点,说不定还能看见几只千年猪肉干插着。况且就算是人又如何,一个人死了之后,血很快就会凝结,你放心吧,这里这么高,血流不到底就干了,再说了,就你那血,人家也看不上啊,以前人家多天然啊,吃的是无农药的食物,喝的是无污染的水,那整个就是农夫的血——有点甜。你现在可好,你那血流出来,人家老人家喝了肯定得食物中毒,所以说这就是一糊弄人的东西。”

我听了脑门上筋都暴了出来,不由分说开口大骂:“我操,什么归什么,我的血怎么就有毒了?你他妈嘴巴能不能消停点……”

凉师爷看我真火了,忙打圆场道:“两位,这个审时度势啊,现在这情况,就别说俏皮话了,你们不觉得,这些枝桠,怎么就越来越密了,再这样下去,再往上就不好爬了?”

老痒说道:“这里本来就是有疏有密的,密了才好爬啊,难不成你还想越疏越好,最好每一根都相距两米以上,我们在这几十米高空叠罗汉?”

我对老痒说道:“你先别下结论,我看是有点不对劲,你把手电打起来。”

我们上来的时候,照明仍旧用的是火炬,因为泰叔包里的那只手电电源并不是很充足,我们不想浪费,但是我现在想要看清楚远处的东西,用火把是做不到的。

老痒打起手电,将光束集中起来,往上照去,只见我们头顶上,青铜枝桠有一个逐渐密集增多的趋势,往上七八米处,已经密集得犹如荆棘一样,要继续上去,只有倒挂出去,然后踩着这些枝桠的尖头爬上去,而这样做比起我们贴着铜树攀爬,要危险很多。

事到如今,就算前面是龙潭虎穴我们也要闯了。老痒让我们待在原地别动,自己先爬到枝桠外面,然后从上面将泰叔那里找到的绳子丢了下来,我和凉师爷一手抓着绳子,跟着爬了上去。

再往上望去,这里的情形已经不像我们在下面看到的那样子,青铜枝桠几乎密集到了无处插手的地步。我爬了一段,心说难怪泰叔会掉下来,看这趋势,再上去恐怕连踩脚的地方都很难找了,只要一个不留神,或者给上面的那种过堂风一吹,指不定就下去陪泰叔了。

老痒在这个时候却爬得很快,我已经没有力气去叫住他,只能收敛精神,一方面不让自己掉队,一方面又要时刻提醒自己小心失足。同时火把也无法在这个时候使用,因为根本没有多余的手去拿它,我只能将其熄灭,插到自己的腰间。

这一段因为过于险要,几乎没人说话,很快,在手电的照射下,我发现青铜树四周的岩壁也开始有了变化,出现了天然的钟乳石和一些溶解的岩帘,显然这里已经出了人工开凿的范围,上面这一段已经是天然形成的岩洞。

通过这一段的时候,岩壁开始收缩,我还发现两边的岩壁上,开始出现一些大小不同的岩洞,都不深,能看到底,有几个岩洞里似乎还有什么东西,给手电照射会发生一定的反应。这些现象,让我逐渐感觉到不安,但是岩壁离我们到底有几十米的距离,我就不信有什么变数,能够从对面直接影响到我们。

我给边上的岩洞吸引了注意力,没有发现前面攀爬的老痒与凉师爷已经停了下来,直到撞到凉师爷的屁股才反应过来,抬头一看,只见在上方,出现了很多那种带着面具的猴子,就和我们刚才在下面遇到的一模一样。

再仔细一看,却发现这些猴子已经死了,尸体给上面吹下来的热风吹成尸干,怪异地扭曲着,手脚卡在密集的枝桠里面,才没有掉落到下面。这样的干尸足有几十具,那种诡异的面具没有随着尸体的干瘪而脱落,仍然默默地盯着我们,似乎随时会复活一样。

我们放慢脚步,仔细地观察这些奇怪的东西。

猴子的身体似乎得了一种皮肤病,毛发大部分都脱落了,呈现灰白的颜色,看起来与人类的皮肤有几分相似,但是仔细去看,却发现有非常明显的病斑,从体形来看,这些猴子大约有一个十五六岁孩子这么高(当然不是姚明),也许还略高一点,在这种情况下,我对于身高的感觉几乎失灵。

猴子脸上的面具,看上去是石头质地,打磨得非常完美,我甚至怀疑有可能是瓷制,从面具与猴子头部的结合处来看,这面具似乎是被烙进肉里,或者用什么血腥的手段,直接和脸长在一起了。

大部分的干尸都很完整,只有少数只剩下一个肢体,大概是因为年代太过久远,尸体干化过于厉害而导致的自然碎裂。

凉师爷让我们先别爬,指着一具干尸说道:“等一下,我觉得这些猴子的姿势有点古怪,我好像在哪里看过,等我仔细看一下。”

老痒对他说道:“就你麻烦,什么都要看,小心点,等一下该下面的猴子觉得你姿势古怪了。”

凉师爷没有理会老痒,小心翼翼地爬近最近的一具干尸,拿住它的面具,干燥的脸部皮肤随即开裂,凉师爷轻松地将面具撕了下来。他凑进那干尸的脸看了看,转头对我们说:“两……位,这……好像不是猴子,这是张……人脸啊。”

\chapter{螭蛊}

干尸的眼睛已经完全干缩,只剩下两个黑洞洞的眼眶,嘴巴不可思议地张大着,露出残缺的牙齿,整个脸部因为脱水变形,呈现出相当狰狞的表情,让人不敢正视。而从他的牙齿可以看出来,这具干尸并不是猴子,而是如假包换的人!

老痒呆了一下,说道:“这是怎么回事,老吴,你刚才不是说是只猴子吗?这……这……摆明了是人啊。”

我结巴道:“我……我也不知道,刚才我打裂那面具,我看到那的确是只猴子,还是只黄毛的大猴子,这……这……真把我搞糊涂了。”我说着就想探头过去,看看是不是因为光线的关系,看走眼了。

凉师爷忽然摆了摆手,让我别碰尸体,自己小心地站直身子,将他手里的面具翻转过来,我看到面具后面嘴巴的位置,竟然有一个拳头大小犹如蜗牛壳一样的螺旋凸起,上面有一个小洞。凉师爷把面具对着自己的脸比画了一下,转头对我们道:“这面具好像得张着嘴巴才能戴。”

老痒奇道:“张着嘴巴?那不是嘴里像塞了个呼吸器一样,多难受啊。”

我看到干尸的样子,嘴巴张得很大,对凉师爷说:“难不成这块蜗牛壳里有什么蹊跷,你砸碎了看看,这些面具都是长到这些猴子的肉里的,嘴巴眼睛都遮住了,它们肯定有其他方式来进食和看东西。”

凉师爷用自己的钢笔插入那个洞里,用力一撬,“蜗牛壳”就碎裂开来,露出了里面一段类似于螃蟹脚的东西。凉师爷将这东西扯出来,发现是一条从来没见过的虫子,已经变成化石状,如果稍微一用力,就会断成几段。

“看来这面具不会是自愿戴上去的。”凉师爷皱着眉头说道,“不过这东西的确是人造的,你们看面具里面的纹路,和树上的双身蛇大致相同,肯定和铸造这棵铜树的人有关系。”

老痒将面具接过来,饶有兴趣地看了半天,说道:“这条应该就是西周时候的老虫子,说不定现在已经绝迹了,难怪我们不认识。哎,你们看,这虫子好像只有半截。”

说完他看了看我们,问道:“另半截到什么地方去了?”

这条虫子蜷缩在面具嘴巴部分的突出空腔里,按照这么说,这条虫子另一半所在的地方只有一个,我想到这一点,下意识地往干尸的嘴巴里看去,果然看见,在黑洞洞的大嘴里,另有半条虫子附在舌头的位置上,干枯的虫体一直插进尸体的喉管里,不知道进入了什么器官。因为干尸萎缩的肌肉和化石般的虫体很像,所以不仔细看,会以为这条虫子是干枯的舌头。

凉师爷看到这副情形,脸色一变,叫道:“快扔掉,快扔掉!我的老天,快扔掉!这面具可能是活的!”说完他就一掌拍了过去,将老痒手里的面具打落,面具飞速坠入黑暗之中,撞在枝桠上面,啪的一声,摔得粉碎。

老痒给他吓了一跳,差点抓不稳摔下去,忙问他发什么神经,什么叫面具是活的?

凉师爷咳了一声,似乎很懊悔的样子,又是挠头又是皱眉头,说道:“在下真是惭愧,怎么就这么笨呢,早先怎么就没想到,这……铜树,这祭祀方法,摆明了就不是咱们汉人的东西,哎,我真是蠢货,蠢到家了!”

“你他妈的瞎掰什么啊?”老痒火了,“什么蠢货,和面具有什么关系?有什么话直说好不好?”

凉师爷摆了摆手,说道:“不是,你耐心听在下说,这事情我还得从头说起,不过,怎么说好呢?那还得从刚才咱们说的血祭的事情开始……”

原来,血祭这种祭祀方式,在西周时,主要是用在少数民族的祭祀活动中,当然那个时候的少数民族和我们现在的完全不同,这些民族大部分已经消失或者融入到汉族中来了。大规模的血祭,在汉族正史中并没有记载,但是在一些少数民族遗址中有零星发现,可惜由于语言文字的失传,没有更为详细的资料。

而少数民族的祭祀圣地,都是非常神圣的,不仅有人把守,并且还会由祭师施下某种异术,以保护自己的神不受骚扰。在少数民族传说中,施法的过程非常的神秘,这种异术流传到现在,给神化成了小说里无所不能的蛊术。

凉师爷又说,蛊术自魏晋南北朝那时候起分了一分,到宋代又是一分,秦之前的蛊术非常厉害,简直和现在的超能力差不多,但是所有的蛊都是由虫而起,蛊术在那个时候就叫做皿虫术。这些戴着面具的猴子和干尸,诡秘莫名,可能就是这种远古蛊术的产物。

他曾经听说过一种蛊术,叫做螭蛊,可以将人变得非常有攻击性,而现在藏在面具背后嘴巴位置空腔里的、那种深入喉咙的虫子,可能就是古老的螭蛊原形,这种虫子也许可以影响动物或者人的神经系统,攻击外来的陌生人。所以当我将它们的面具击碎之后,那只猴子就恢复了本性,开始本能地远离我们。

螭蛊能够在宿主的体内繁殖,等到宿主死亡之后,它们会依附在某个地方,比如说这种面具的空腔里,等待着下一个宿主的靠近,然后通过某种方式寄生过去。

这具干尸,说不定就是当时在这里狩猎的猎人,不走运碰到了休眠状态的螭蛊,结果中了招,被这种古老邪术给害了。

当然,这种东西完全没有记录可寻,也不知道是不是真的,不过面具之中藏有虫子,且深入人喉,是不争的事实,这绝对不是一件平常的事情,要小心防备。

听到凉师爷这么说,我起了一身的鸡皮疙瘩,其实在来之前,老爷子给我的资料里面,也提到过相似的事情,但是当时我只是草草看了看,心说这不是和美国电影的桥段一样嘛,没想到还是真的,想不到老美的科幻片还得借鉴我们老祖宗的技术,真不知道该说光荣好还是惭愧好。

转头看去,诡异的干尸仍旧一动不动挂在那里,惨白的面具似笑非笑,似乎正在等待我们靠近。

老痒脸色有点难看,犯了嘀咕,问凉师爷:“你说得也太恐怖了,那如果给这螭蛊附上了,马上扯下来总没事吧,不会有啥隐患吧?”

凉师爷说:“我也没中过,螭蛊很难解,我想要是给附上了,绝没办法简单地扯下来了事。这种事情,咱们还是预防为主,这些干尸,我们尽量别靠近了。泰叔也是从这里掉下去的,他这样的老江湖,估计总不会是失足,要小心一点。”

老痒皱了皱眉头,想说什么,又没出口。我就问他,照着现在这样子,还要爬多长时间,如果上面全是这样密集的枝桠,估计累死也到不了顶。老痒对我说,上面还会稀疏起来,当时他爬的时候,只有一只小手电,照明很差,没有注意到这些干尸,也没猴子来袭击他,所以现在他也不知道自己爬到什么地方了,不过反正自古华山一根柱,你往上爬总不会爬到其他地方去。

我感觉此地不宜久留,就招呼他们先过了这一段再说。和凉师爷一起的还有一个胖老板,此人大有可能在我们上面,要是给他先到了顶上,就麻烦了。要是埋伏起来,我们三个说不定就会死得不明不白。

老痒说:“说得有道理,你等一下,我打一发照明弹,看看上面有什么埋伏没。”说着拿出信号枪,对着上方,笔直地开了一枪。

信号弹飞到顶端,并没有撞到头,我心里咯噔了一声,这种子弹最起码能打到二百多米的高度,难不成还有二百多米要爬,呵呵,那真是要命了。

信号弹烧了起来,向上看去,果然再往上不远的地方,枝桠又稀松了起来,想不通为什么要这么设计,而且从下面看上去,二百米的范围也不是无法目极,我还是可以看到一些东西的,虽然无法说出那是什么。

信号弹落下来,老痒注视了一段时间,说道:“看样子那胖广东老板没埋伏在上面,说不定就泰叔一个人活着进到这里来了,毕竟外面那棺材阵不是那么好……哎,那些是啥东西?”

信号弹落到离我们还有六十几米的时候,我们看到那一段的青铜树干上,有不少凸起的东西。仔细一看,我就觉得后脑一麻,冷汗直冒到了脚底,整个足有十米的一段距离,青铜树干上,附满了一张又一张的脸,不!应该说是那种诡异的面具。

\chapter{凌空}

信号弹坠落下来,划过这一段区域,这些脸动了起来,纷纷避开灼热的光球,看上去,就像一只又一只长着人脸的甲虫。

这些应该就是凉师爷口中所说的螭蛊的正身,古人将它们养在特殊的面具里,竟然繁衍了下来,刚才我还半信半疑,想不到这么快就碰上了,还是这么一大群。

脸依附在沟壑横生的青铜树上,给流动的光线一照射,呈现出不同的表情,或痛苦,或忧郁,或狰狞,或阴笑,我从来没见过如此诡异的景象,看得我寒毛直竖。

凉师爷说起来慷慨,一见到真东西也不行了,颤抖着对我说道:“两……两位小哥,这些都是活的,那些螭蛊在面具底下附着呢,怎么办,我们怎么过去?”

“别慌,”老痒说道,“你看它们对信号弹的反应,这些东西肯定怕光怕热,我们把火把点起来,慢慢走上去,们不敢碰我们。”

我摇了摇头:“别绝对化,信号弹的温度和亮度非常高,它们当然怕,火把就不一样,你别忘了刚才那些猴子,碰到信号弹都逃了,但是你用火把吓它们,它们只不过是后退一下而已,我估计你打着火把上去,不但通不过,还会给包围起来,到时候要脱身就难了。”

“那你说怎么办?”老痒问我道,“你是不是有啥主意了?”

我说道:“现成的主意我没有,只是一个初步的想法,不知道成不成。”

老痒不耐烦道:“我知道你鬼主意多,那你快说。”

我指了指几十米开外的岩壁,说道:“直接这么上去太危险了,如果真的像凉师爷说的,这些活面具肯定有什么法子能爬到我们脸上来。硬闯肯定会有牺牲,我们不如绕过去,你有没有什么办法可以让我们荡到对面的岩壁上去,上面这么多窟窿,也不难爬,我们也可以好好休息一下。”

老痒看了看我指的方向,叫道:“这……么远?荡过去?”

我点点头,比画了一下:“我脑子就这么一个想法,我们不是还有绳子吗?你拿出来看看够不够长,如果这招不行,我看只有下去,下次带只喷火器过来。”

老痒拿下盘在腰间的绳子,这是从泰叔身上扒下来的装备之一,上面有U.aa标签。世界上最好的登山绳,特种部队都用这个,看样子他们也挺舍得花钱买装备。

我早在去鲁王宫之前,曾经帮三叔采购过装备,查了大量的资料。所以我知道这种绳子,如果直径在十毫米以上,几乎可以承受三吨的冲击力(就是突然坠下)。支持我们三个人的重量,绰绰有余……

强度足够,只是不知道长度够不够,老痒将它垂下树去,目测了一下,不由叫了一声糟糕,绳子总长只有十几米,要到达对面,还差很长一截。

“怎么办?”他问我,“就算把我们的皮带接起来也不够。”

我捏了捏绳子,发现这是十六厘米的双股绳,不由灵机一动,说道:“没事。咱们把这绳子的两股拆了,连成一条,就够了。”

“小吴哥,行不行啊?这绳子这么细,不会断吧?”凉师爷问道,“你看,这简直比米面还细,您可别乱来啊。”

“国外登山杂志上是这么说的,总不会骗我们。”

我将绳子外面的单织外网层撸起来,抽出一条非常细的尼龙绳,自己也咽了口唾沫,真他娘的太细了,按照常识来说,这么细的绳子肯定没办法承受我们的重量,不过国外的资料上确实是这么说的,八毫米直径的这种加强尼龙纤维,已经可以用来做登山的副绳,只要不发生大强度的坠落,是不会轻易断的。当然,使用这种绳子有一定的危险性,所以一般都是两条一起用,我们只有一条,还要请上帝多保佑。

还是相信高科技吧,我想到,总不会这么倒霉。

我将接好的绳子递给老痒,他从背包里拿出一只水壶,用一种水手结绑好,用来当作重物体,用力甩向对面,失败了好几次后,终于绕住了对面的一根石笋,一拉,绳子绷紧,固定得非常结实。

“行了,”老痒说道,“他妈的总算搞定了,老吴,这绳子不去说它,对面这些石头靠不靠得住?”

“我不知道。”我说道,一边想着如果石头靠不住会怎么样,我大概会给荡回到青铜树这一边,运气好一点撞到树干上,撞个半死,运气不好就直接给树上的枝桠插成筛子。

绳子的这一边也给绑在一根青铜枝桠上,老痒打了个比较特殊的结,好让我们过去的时候,可以在对面将这个结解开。这个结非常复杂,看得我眼花缭乱,我问他哪里学来的这种本事,他说是牢里。

一切准备就绪,我最后扯了扯绳子,确认两边都已经结实了,就招呼他们开爬,结果他们两个人都没动,我看了他们一眼,发现他们正用一种打死也不第一个爬的眼神看着我,显然第一个上这么细的绳子,需要非常大的勇气。我又叫了两声,两个人都摇了摇头,我只好暗骂一声,硬着头皮自己先上去。

上去之前,我将身上的拍子撩和背包分别转交给老痒和凉师爷,尽量减少自己的重量,这些东西可以绑在绳子的那一头,等一下老痒隔空解绳子的时候,将它们一起荡到下头,再拉上来就行了,老痒对对面的那些山洞也不太放心,就将他的手枪塞给我,如果碰到什么突发情况,也好挡一挡。

我感叹一声,大有烈士赴死的感觉,拍了拍二人的肩膀,就转头向绳子爬去。

脚离开绳子的一刹那,我的神经几乎和这根绳子绷得一样紧,眼一闭牙一咬,就准备听绳子断掉的那一声脆响,结果这绳子竟然支持住了,只是发出了一声让人非常不舒服的“咯吱”声,那是两边的结突然收紧发出的声音。

我心里念着别往下看,可是眼睛还是不由自主地向下瞟了一眼,我的天!我呻吟了一声,马上转过头,闭上眼睛,念阿弥陀佛。

老痒叫道:“喂,老吴,你磨蹭什么?快爬啊,你待在上面更危险。”

我问候了老痒的祖宗一声,深吸了一口气,移动手脚,开始向对面爬去。这种绳子有一定的弹性,每走一步,都会发生非常剧烈的抖动,我爬得万分惊险,加上绳子实在太细,非常抠手,不一会儿,就感觉到有点力不从心。爬到后来,我的脑子一片空白,连自己都不知道怎么踩到了实地,我的脚马上一软,抱住那石笋就摊成一团,在那里大喘。

火把在我这里。我点起来插到一边,看了看老痒他们,看见凉师爷正哆哆嗦嗦地爬到绳子上去,老痒拉住他,让他先别爬。叫我先看看这边的情况如何,如果不适合攀爬,或者有别的危险,可以省点力气。

我看了看四周几个岩洞,都只有半人高,是人工开凿出来的,不过经过千年雨水渗透,上面也出现了不少刚成型的钟乳,里面很潮湿。这些岩洞开在这里,可能和当年铸造这根庞然大树的工程有关系。

往上看去,这些岩洞之间的距离只有三四尺,虽然爬起来不会太连贯,但是也不至于很困难。岩洞里面空无一物,没有什么危险,刚才在树上看到洞里有什么东西,大概是光影变化造成的错觉,在这样幽暗的地方,神经难免会有点过敏。

我一边安慰自己,一边再次确认,然后抬手给老痒打招呼。

老痒拍了拍凉师爷,让他先走,后者用手揉了揉自己的脸,爬上了绳子,向我移动过来。

看凉师爷爬绳子简直是对神经的考验,其间过程我就不说了,十分钟后,我总算把一摊烂泥一样的师爷拉到了我这一边。

最后就是老痒。他深吸了口气,将手电绑在自己手上,又把那边的结检查了一遍,才小心翼翼地爬上了绳子,他爬得很快,不一会儿就到了绳子的中段,这个时候,我这边缚绳子的石笋突然发出了一声怪声。三个人同时不动,老痒一脸惊恐地看了我一眼,我回过头一看,心里咯噔一声——石笋上面出现了一道裂痕。

要倒霉了!我转头大叫:“快爬!这里顶不住了!”

我叫了几声,老痒却一动不动,直勾勾地看着我,然后竟然开始后退,一边退还一边打手势,好像让我也回去。

干什么?我心里想,突然涌起了一股不祥的预感。

老痒拼命地指着我们头顶,一边小声叫道:“快跑……”

凉师爷和我奇怪地抬头一看,我一下就惊呆了。

刚才还空无一物的岩壁上,竟然已经爬满了那种人脸面具,相互簇动着,一边发出的声音,一边潮水一样向我们缓慢地围了过来。乍一看下去,就像无数的人贴着墙壁俯视我们。

我这时候真想抽自己一个巴掌,真他娘的笨,树上有螭蛊,怎么就没想到岩壁上也会有,这下子完蛋了,难不成我的下场就是变成像那些猴子一样的东西,在这里干死?那还不如一头跳下去痛快。

老痒看我们发呆,大叫:“别发呆了!回来!把绳子割了!”

我一听反应了过来,几步跳回到石笋边上,用力一纵,跳上绳子,冲击力将绳子猛地往下一扯,石笋发出一连串令人毛骨悚然的开裂声,没等我抓稳,凉师爷也跳了上来,绳子一下给拉长了十几公分,绷到了极限。我马上听到一种非常不吉祥的声音,然后啪的一声脆响,世界上最结实的绳子,也终于晚节不保,断成两段。

八毫米宽的绳子果然无法承受三个人的重量,随着一声脆响,铜树那一边的打结处拉断,我们像荡秋千一样划过一道大弧线,重重撞到了一边的崖壁上,给撞得七荤八素的,几乎吐血。

最下面的老痒撞得最厉害,一时抓不住绳子,向下滑去,他慌忙扒住了边上的石头缝隙,才停住身子,我和凉师爷也好不到哪里去,我的脑袋划过一道岩棱,给磨出一道口子,鲜血直流。凉师爷垂直吊在那里吃不住力气,绳子在手心里打滑,一下子就哧溜到底,幸亏下面还有一个老痒,才没掉下去。

上面石笋继续发出开裂的声音,随时有可能断裂,我赶紧伸手,抓住边上的钟乳柱,跳了过去,然后把凉师爷也拉了过来,凉师爷吓得够戗,抬头就直说谢谢,才说了一句,突然一张面具就从上面蹿了下来,一下子抓在了他的脸上。

那一瞬间,我似乎看到面具底下,几只螃蟹腿一样的爪子伸了出来,凉师爷发出“呜”的一声惨叫,想用手掩脸,但是已经晚了,面具已经盖了上去。他拼命想扯掉面具,可是那面具好像贴在他脸上一样,几次扯出来又吸了回去。我想去帮他,可是他发了狂一样地乱撞,还没靠近,就被他一下子顶翻了出去,我一手重新扯住绳子,滑到老痒边上才勉强定住。

我看了看脚下面的万丈深渊,心里暗骂,刚想再上去帮凉师爷。一抬头,一只大手一样的黑影从天而降,一下子抓在了我的脸上,我眼前一黑,什么都看不见,只觉得几只毛茸茸的东西直往我嘴巴里钻。

慌乱间,我只有一只手抓住岩石缝隙,一只手去掰那个面具,同时咬紧牙关,不让那东西进来,才掰了一下,那面具竟然自己掉了下来,我赶紧把它扔了出去,结果不巧正扔到老痒屁股上,老痒大骂一声,忙不迭地一枪柄将它砸了下去。

我舒了口气,一转头,又是四五只螭蛊跳到了我的头边,吓得我一个哆嗦,抬手就是四枪,可是根本不管用,一下子又是十几只涌了过来,我和老痒向下退去,这时候就听到“呜呜”的惨叫,抬头再看,凉师爷已经遭了殃,身上爬满了螭蛊,他大叫挣扎,想将螭蛊拍下身去,可是他拍掉一只,就有更多的蹿了上来。

我一边后退,一边开枪,一直把子弹打完,形势一点改善都没有,潮水一样的螭蛊从我们两边直围过来,转头一看,四周岩壁上面已经爬满了这种东西,互相触动,一时间满耳都是诡异莫名的声响,简直让人头疼欲裂,一个分神,就有几只蹿起来,直往人脸上扑,一个不小心就有可能中招。

我们一直向下退去,可是不可能快得过这些东西,很快就给围了个结实,几乎要绝望的时候,老痒开枪了,拍子撩一声巨响,将我们头顶上的螭蛊扫飞了一片,最近的几只面具马上给打得粉碎,碎片像下雪一样从我头顶上落下来。

可是不到一秒钟,给拍子撩轰开的一段空白岩壁马上又给后面的螭蛊覆盖了,老痒一看没用,赶紧用衣服包住自己的头,对我大叫:“老吴!我掩护你,你快把嘴巴包住,然后去拿火把!”

我抬头一看,火把还卡在当时我顺手找的一处突起上,周围一圈没有螭蛊,显然这些东西的确怕火,可是我和火把之间的这段距离,密密麻麻全是螭蛊,根本没可能爬上去,我对老痒大叫:“还是你去吧,我来掩护你!”

“我没招了!你搏一下吧!”老痒一边大叫,一边用拍子撩乱砸,“真他妈的倒霉!”

我看着这些东西,心里直发抖,这些螭蛊,并没有多大的攻击力,只是数量实在太多了,又有坚硬的面具保护,很难完全杀死,而且这些还只是几千年繁衍后幸存剩下来的,当年为了保护这棵铜树,古人到底制造了多少这种东西,就无法想象了。

老痒又一次甩开身上的螭蛊,想爬到我的身边来,可是在抬头看我的时候,他突然呆住了,叫道:“老吴,你怎么回事?”

我看他呆在那里,几只面具落在他肩膀上直往他脸上的衣服里爬去,大叫道:“什么怎么回事!小心!”

老痒才反应过来,慌忙把肩膀上的螭蛊拍掉,然后对我道:“老吴,我说你——没发现?这不对啊!”

“什么不对!”我将他拉过来,不耐烦地大叫,“什么时候了,有屁快放!”

“你看看你,身上一只面具都没有啊!它们怎么不爬你身上去!不可能啊!”

我低头一看,自己也啊了一声,又看了看凉师爷和老痒,他们身上都爬满了螭蛊,怎么甩都甩不掉,可是我身上,的确一只也没有。

我心里咯噔了一下,马上回忆起,从刚才到现在,除了飞到我脸上的那只外,身上的确也没有爬上来过。刚才一路混乱,一直没有发现,还觉得自己运气不错,现在看来,有点不对劲。我急忙往四周看去,发现那些螭蛊虽然同样也向我爬来,但是一靠近我,突然就改变方向,向其他地方爬去,似乎像忌讳火把一样忌讳着我。

“怎么回事?”我心里奇怪道,赶紧试探性地一抬手,去抓最近的一只面具,手还没碰到,那一片的螭蛊已经稀里哗啦地向后退去。

我看了看老痒,老痒也看了看我,两个人都莫名其妙,老痒叫道:“我的爷爷,这一招真酷,你是不是手上不当心沾了什么东西,快看看!”

我马上一看,手上除了我撞伤后留下的血滞和污垢之外,并没有其他的特别。

这可怪了,它们怕我什么呢?难道它们的寄生还有选择性?

我看到这些螭蛊退却的样子,想起了闷油瓶震退尸蹩的那一幕,心里冒出了个问号。

等等,难道是……血?

怎么可能,这些穷凶极恶的东西怎么可能怕我这个普通人的血呢?

我疑惑地看了看手,脑子里一团糨糊,什么都想不清楚。

这一边老痒已经抵挡不住,我反射一样,试探性地朝老痒一伸手,让我瞠目结舌的事情发生了,附在他身上的螭蛊,像蟑螂见了杀虫水样飞也似的退了开去,情形和尸蹩见了闷油瓶的血一模一样。

“不是吧!”我下巴都掉到了地上,心说不用这么给我面子吧。

老痒还不明白怎么回事,大叫着要爬上去拿火把,我拍了拍他,对他说:“等等,你看,好像有点不对劲。”

说完,我将手向上扬起,向已经在抽搐的凉师爷爬了几步,几步而已,那些地方的螭蛊潮水一样地退了出去,刚才那些整齐的面具触动声,突然间乱成一团,被一种惊恐的吱吱声压了过去。

老痒目瞪口呆地看着我,好像在看着什么怪物一样,我不去理会他,爬到上面,把手往凉师爷脸上一放,那只面具突然就拱了起来,我马上抓住它,用力一扯,将面具扯了下来,还顺带扯出了一条满是黏液的“舌头”一样的东西。凉师爷本来已经在半昏迷状态了,那“舌头”一拔出他的喉咙,立马就呕吐了出来,喷了自己一身。

手里的螭蛊剧烈地挣扎,我几乎抓不住,那舌头一样的东西又太恶心,我只好用力往石头上一砸,砸了一手的绿汁。

身边的螭蛊退了开去,但是却不走远,在我们身边形成了一个巨大的包围圈,不停地收缩,老痒赶紧把火把拔了回来,扫了一圈,将它们逼得稍微远一点。这时候凉师爷咳嗽了两声,似乎恢复了知觉,老痒又去拿了水壶,收回了剩余的绳子。可惜我们其他的装备和食物都还在树上面,不知道有没有办法能拿回来。

我把水倒在手里,给凉师爷润了润嘴唇,他总算缓了过来。看见我,竟然两行眼泪流了下来,我一看傻眼了,赶紧将他扔到一边。老痒神经崩紧太久,有点神经质,我对他说有火把在,它们肯定靠不过来,让他放松,不然会疯掉。他看螭蛊果然不再靠近,才松了一口气,将火把插到我们中间的一个地方,马上问我道:“老吴,怎么回事情,啥时候你变这么牛了?也不早点使出来,弄得我们这么狼狈。”

我看着自己的手,摇了摇头,说道:“我他妈的自己也不知道,还以为做梦呢。”

老痒看了看我手上的血,沾了点闻了闻,也不相信我这么厉害,问我道:“你刚才过来的时候,一路上有没有粘上什么特别的东西?你仔细想想……说不定给你碰上了什么这些破面具的克星,你自己不知道。”

我想了想,我碰过的东西,他们都碰过了的,要说没碰过的,只有我的血,可是这不可能,要是我的血这么强劲,在鲁王宫我就发威了,哪会那么浪费,那……难道是那时候沾上了他的血,现在还有用,不是吧——我摇了摇头,自言自语地否定了。

凉师爷听我们说了刚才的事情,就问我们是怎么一回事,他给面具遮了眼睛,什么都没有看到,老痒又存心挤对我,对他说道:“你不知道,刚才咱老吴,可是威风了一把,那是这么一回事……”

凉师爷听他一说,啧了一声,说道:“小吴哥,你有没有吃过一种东西,是黑色的,这么大——”

\chapter{麒麟竭}

我正在惊讶当中,他这样问我,脑子里没什么概念,摇了摇头道:“这么大?好像没吃过,怎么说?凉师爷,你想到啥了?”

凉师爷沾了我一点血,闻了闻,对我说道:“听你刚才说的情况,我倒想起一件事,我早先时候听一个老先生说过,有一种东西,人吃了之后,血能驱邪的,邪虫不近,是一种非常罕见的中药,你想想,有没有吃过类似的东西?”

我啊了一声,黑色的甲片状?中药?这真把我难倒了,最近事情发生得太多,吃东西的时候大部分都很仓促,也没有生过什么病,吃了什么东西,我一向也不太在意,现在突然问起来真的一点也记不起来。

老痒嘲笑我道:“老子只听说过黑狗血、公鸡血能驱邪,想不到啊,咱们家老吴也有这本事,这事情你可别说出去,不然人人都找你借血,几天就给你挤成人干了。”说完大笑起来。

我骂道:“你他妈的能不能积点口德?什么狗鸡!我告诉你,人血自古都是最能驱邪的东西,特别是死囚的血,刑场上面还有人托法医蘸白布挂在门梁上呢,不懂别乱说。”

老痒看我急了,得意地大笑起来,笑了两声突然哎哟起来,摸着后背,咧了咧嘴巴,大概是早先那里受了伤,现在给笑得牵疼起来了。

我心说活该,不去理他,对凉师爷道:“你要不再给我形容得具体一点,光黑色的,甲片,满足条件的东西太多了,这东西有啥明显特征没有?”

凉师爷想了想,不好意思道:“我自己没亲眼见过,只听过别人形容,时间也挺久了,特意去想,真想不起来。”

我听了不由失望,叹了口气。

凉师爷一笑,说道:“小哥,你也别太在意,这也不是什么坏事情,刚才要不是你,我们就完蛋了。我看着,这是命数,冥冥中自有注定,你想啊,以后您倒斗的时候,有了这资本,什么斗都不在话下啊。”

我听了心里挺不是滋味,这一路走成这样,说明我这人命寒,以后还倒斗,估计是找死。我抬头看了看上面,对他们说:“话说回来,现在没经过化验,也不知道是不是真是我的血在起作用,要不是倒也麻烦,趁着这个机会,咱们最好快点上去,过了这一段再说。”

凉师爷本想再休息,可看到潜伏在四周蠢蠢欲动的蛊虫,还是同意了我的想法。我们再次动身爬了几步,老痒突然抓住我的手,让我停下来,哑声道:“等……等一下!”

我回头一看,发现他脸色惨白,一头冷汗,表情大大的不妥当,心里咯噔了一下,问他怎么回事?

老痒一手抓着岩石,一手摸着后背,龇着牙道:“我也不知道怎么回事,刚才一笑,背上就疼得要命,可能是刚才绳子断的时候给撞得有点伤筋了,你给我看看,怎么疼得这么厉害,力气都用不上。”

刚才绳子断裂之后的那一下撞击着实不轻,我早就感觉到浑身疼痛,不过刚才情况危急,没时间考虑这些,现在气氛一缓和下来,这些伤口就开始发作,老痒在绳子的最下端,撞得比我们厉害得多,该不会是什么地方骨折了?

我让他别动,撩开他的衣服,只见后背第三条肋骨的地方一片淤青,竟然有一点凹陷,我顺手按了一下,他突然就像杀猪一样地叫了起来,背一弓,几乎没把我撞下去。

我心说不好,这伤看样子不简单,碰一下就疼成这样,难道真的骨折了?

老痒脸都扭了起来,艰难地回过头,问我怎么样?我皱着眉头,也不知道怎么对他说才好,只好说道:“光这样看也看不出来,不过你疼成这样,我们不能爬了,搞不好骨头已经断了,再做剧烈运动,可不是开玩笑的,要找个平坦的地方仔细检查一下。”

老痒一心想早点上去,此时已经挣扎着起来,咬着牙说:“仔细检查就免了,咱们的火把和手电都没办法坚持太长时间,不能停在这个地方,到了上面再说吧。”

凉师爷看了看他的背后,摇了摇头说道:“不,痒哥,小吴哥说得对,你这背上都变形了,一定得看看,要是真骨折了,得马上处理才行,不然骨头很容易刺进胸腔里去,那时候就完蛋了,这方面我还懂点,咱们现在也离顶上不远了,没什么不好耽搁的。”

老痒还想和他犟两句,可能实在太疼了,话到嘴边变成了呻吟,我看到边上那些矮小的岩洞,里面似乎比较平坦,给凉师爷打了个脸色,两个人不由分说,将其架起来,扶进边上一个相对最好的岩洞里。我拿回火把,插在洞口,防止蛊虫进来。

这个洞大概有七八米深,一米高不到,因为长年照不到阳光,空气又非常潮湿,岩壁上有一层给霉菌腐蚀的斑点,似乎有一些人类活动过的迹象,不过并不明显。进到五六米的地方,就可以看到洞穴的底部,是一块粗糙的岩面。其他再无东西。

我查看了一下,看没有什么危险,才把枪收起来。凉师爷用拍子撩做了一个固定器,用绳子绑在老痒的背上,老痒脸色稍微缓和了一点。我心说这做师爷的就是不一样,什么都会,看来要是下次倒斗,咱们也要找个这样的人才。

凉师爷弄妥之后,我问他情况怎么样,他压低声音对我说道:“骨头应该没断,不过肯定开裂了,我给他暂时固定了一下,应该不会那么疼了,不过小吴哥,你最好劝劝你这位朋友,他这样子,绝对不能再往上爬了。”

我看了凉师爷一眼,知道他是话中有话,意思大概是劝我下去。一路上他暗示我也不是一次两次了,话说回来,这样的冒险对于他来说真的非常的勉强,我看得出他早就萌生了退意,可惜碍于老痒的坚持,没办法提出来,现在给他找到一个借口,自然会借题发挥。

不过这样一来,关于老痒的伤势,我就不知道该不该信他的话了。

凉师爷看我怀疑,马上又说:“小吴哥,虽然我不是跟你们一路的,不过大家都是江湖上混的,有些事情我不会打马虎眼,你自己有个数,说实在话,你看看我们现在的样子,如果坚持上去,恐怕这一次真的会死在这里。”

我看了一眼老痒,他正忍受着疼痛,并没有注意我们说话,于是拍了拍凉师爷的肩膀,轻声对他说:“这事还要看看情况,你也去休息,现在讲这个不是时候,就算要下去,也得休息够了才行。”

凉师爷嘟囔了一声,靠到一边,揉起自己的大腿,不吱声了。我检查了一下剩下的东西,也坐下来,揉了揉太阳穴,开始考虑凉师爷说的话。

本来我对李琵琶所说的事没有多少兴趣,早先要我放弃,我不会有什么意见,但是现在既然已经千辛万苦爬到这里,到这个时候才放弃,心里倒也有点不舍,有点临阵退缩的感觉,但是我心里知道,凉师爷说的话是有道理的,现在我们一个人骨折,一个人身体状况非常不稳定,而我自己也到了体力的极限,如果还要莽撞地爬上去,实在是不明智的行为。

不过这样一来,老痒那一关就很难过,毕竟我和他才是一路的,现在联合外人来对付他,这朋友可能就做不下去了,而且凉师爷这人看上去挺窝囊的,可是到底是老江湖,这说不定就是他分化我们的一招,要是顺着他的思路走,可能会进到他的圈套里,这真是个两难的决定。

想来想去,想不出个所以然,干脆不想了,走一步是一步。

我转头去看他们时,凉师爷已经睡着了,他累得够戗,现在呼噜都打了起来,老痒也眯了过去,不过睡得不深,大概是背上伤口的问题。这个小洞虽然潮湿阴冷,但是比起吊在外面要舒适很多,我一看他们睡得这么香,无尽的倦意袭来,虽然心里逼着自己不能睡,但是还是不知不觉地睡了过去。

这一觉睡得极其香甜,醒来的时候,浑身酥软,一种舒适的刺痛传遍全身,这时候火把已经非常微弱,显然我睡了比较久的时间,探出头去一看,外面的蛊虫已经不见了,只有零星几只还趴在那里。

我松了口气,打起手电向上照了照,从这里看上去,我们离铜树的顶部大概只有三到四个小时的路程,上面的东西,几乎说是垂手可得,现在下去,真的有点可惜。

老痒还没有醒过来,不过神态安详,似乎好了很多,我转头去看凉师爷,想叫醒他,商量下一步怎么办,一看,却发现刚才他躺着的那个地方空了,他并不在那里。

“嗯?”我下意识地愣了一下,用手电往山洞深处一照,也不见他的踪影,心说人哪里去了?这个时候,我忽然看到原本给老痒固定伤口的拍子撩没了,马上起了一身冷汗,一股不祥的预感袭来,一摸自己的腰间,果然,我的手枪也没了!

“王八蛋!”我大骂一声,真是没想到,看上去这么没种的一个人,竟然会在我睡觉的时候偷走我的枪偷跑掉!可是,为什么他不把手电也一起拿走,没有照明工具,他怎么行动啊?我这时候急火攻心,也没有仔细考虑,抄起火把就想出去追他,这家伙脚程慢,如果走了不久,绝对追得上。

刚一踩出洞穴,我还没来得及分辨他是向上去了还是向下去了,眼前就突然一晃,一团黑影子从上面荡了下来,一脚踢在我的胸口,我只觉得一股气上来,结实地倒摔回了洞里。倒地之后,我咬牙想站起来,可是下巴又给打了一下,这一下打得非常的狠,我几乎给打晕过去,迷糊间,看到一个叼着香烟的胖子正猫进洞里,手里拿着一杆短步枪,凉师爷一脸铁青地跟在他的后面。

我只看了一眼,就认出那胖子就是两个广东老板中的一个,姓王的那个,他拿枪对着我,让我靠边去,转头对凉师爷道:“老凉,边(哪)个后生吃过麒麟竭嘛?”

\chapter{逼近}

凉师爷用下巴指了指我,一脸轻蔑之色,我心里暗骂,你个吃里扒外的,老子一路过来也算照顾你,想不到竟然这样对我,早知道这样,当初就把你给做掉,免留后患。

胖老板从背包里拿出了固体燃料风灯,点燃放在地上,这东西是登高海拔雪山时候用的装备,既可以照明,又可以取暖,一下子整个山洞便亮了起来。接着他又掏出几块压缩饼干丢给我,做这些事情的时候,手里的短步枪枪口始终对着我。

我接过他丢过来的饼干,觉得莫名其妙,心说这是唱的哪出啊?当下把饼干丢回给他,说道:“哥们两个撂你们手上,要杀就杀,哪这么多废话?”

凉师爷咧嘴笑了一下,转向胖老板,说道:“我说吧,青头就是青头,还搞不清楚状况。”

王老板摇了摇头,又把饼干丢给我,说道:“后生仔,出来跑江湖,脑门要放亮嘛,给你东西吃,就是没打算动你们,你这个样子,碰上脾气差的,那是讨死嘛。”

这人和那老泰比起来,气质完全不同,那老泰一眼看上去,就是那种杀人不眨眼的亡命徒,这胖老板倒是一团和气,看上去让人放松不少,只不过他刚才踹我的那一脚,很有力道,不是那种古董老板能踹出来的,到底是什么身份,我一点也摸不透。

王老板瞥了一眼,似乎是读出了我眉宇间的疑惑,狠狠吸了一口烟,继续说道:“我和老泰他们不一样的,我是个生意人。生意场上,没有永远的朋友,也没有永远的敌人。”

凉师爷说道:“王老板,你不如和他们直说了吧,这俩小子脑子都拐不过弯来,姓吴的小子还比较好说话,等那睡觉的小子醒过来,恐怕还要折腾一番。”

王老板笑了一声,又对我说道:“好吧,当着真人不说假话,我就说得直白点。我呢,是个做生意的,不喜欢动刀动枪的。现在这种情况,你们自己也看见了,就算不落在我手里,你们也很难出得去,老泰已经死了,要对付你们也没什么意思,你考虑考虑,要不要和我合作。我保管你们不吃亏,还有得赚。”

我一听这不是当初我对凉师爷说的话吗?他娘的隔几个钟头又转我这里来了,真是风水轮流转啊。

看我没任何表示,他又递了支烟过来,说道:“你就算不答应也没关系,我会给你们点装备,让你们自己下去,不过你一个人带着一个病号,这路怎么走,你自己想过没有?”

他说的倒是实在话,我竟然听得有点心动,可转念一想,他有装备有武器,干吗还要找我合作?这不等于铺好摊子让人家来赚钱吗?一定有阴谋,他们这些跑江湖的心机太深了,你看凉师爷一路跟着我们过来都是一副献媚的嘴脸,一找到机会马上就给他反客为主了,我们一点都没防备,与他们相比起来,我们真的太嫩了,他们找我合作,必然有什么针对性的目的。

我的思绪一刹那闪过,心里已经有了计划,他们的这个条件,我必须要先答应下来,就像当初凉师爷跟着我们一样,以后再想办法逃脱。况且正如他所说,要想把老痒平安地带下去,至少还需要一个人的帮助,我一个人,实在太勉强。这两个人明显轻视我,这与我当时犯的错误一样,我肯定可以找到一个机会反客为主,至少弄到一把枪。

想到这里,我的脸色缓和了下来,装出犹豫的样子,问他:“好,就算你说的有道理,我可以和你们合作,但是你必须先让我知道,你们到底需要我干什么?”

王老板松了口气,给凉师爷打了个眼色,后者拍了拍我,说道:“识时务者为俊杰,小吴哥,既然你点头了,咱们就还是自己人,在下也就不瞒你什么,自然会把知道的告诉你们,不过这可是说来话长,我们边吃边讲如何?”

我看他靠过来,真想一把掐死他,不过眼角一扫,就看到王老板手里的枪口,仍旧指着我的方向,心里压住内火,勉强一笑,说道:“请说。”

凉师爷看了看外面的铜树,说道:“说起这个东西,可是了不得,根据《河木集》上的记载,最初发现这棵铜树,还是在北魏高祖孝文皇帝十三年——”

李琵琶死了以后,在很短的时间里,凉师爷已经将《河木集》中关于这个墓穴的章节,仔细研究过一遍,《河木集》是一种便条,写得非常随意,有时候用的是哑文,有时候用汉文,还有一小部分是用一种谁也不认识的文字写的,而关于这里的这一段,大部分是用哑文所写,现在大陆,能读得懂哑文的已经不超过二十个人,凉师爷正是其中之一。

哑文记录的事情,一共有三件:

第一件事情是北魏高祖孝文皇帝十三年,大致是太白山一带一处官矿的矿监上报,有寻矿人发现一根青铜古柱,其根部似乎一直挖入山底,未见到底的迹象,不知道入地其深。

这事情在当地闹得沸沸扬扬的,一说这柱子是有灵性的,你越挖它就越往下长,永远也挖不到头,又说这是盘古开天的时候用的斧头柄子,再挖就能把斧头给挖出来。甚至有风水师傅说,那是玉皇大帝打下的钉子,用来将秦岭的龙脉钉住,不然这条地龙就要飞到天上去了。这根铜柱,入地有八百里,不能再挖,一挖全中国就要倒霉了。

不久,一骑哑巴军就接到密令,开赴太白山确认传说的真伪,可是这一队哑巴军却离奇失踪了(估计可能给守陵的厍人杀光了)。四个月后,另一营的哑巴军又接到密令,这一次他们找到了青铜树,领着三千死囚,让他们接管这个太白山,封山扎营,继续挖掘。

第二件事情是北魏高祖孝文皇帝十八年春,说这一挖就挖了四年零三个月。三千死囚向上一直挖通了我们现在所在的溶洞,向下一直挖到山底,没有挖出铜树的根部,却挖出了一只龙纹石头盒子,内是空心。藏有一物,却没有缝隙,怎么打也打不开,他们不敢妄动,将这盒子送进宫里。

第三件事情很简短,是在北魏高祖孝文皇帝十八年的年末,《河木集》上记道,皇帝赐赏,加封二等爵位,每人赏百两金,犒赏全营,众人酒醉,《河木集》的主人和几个熟络的兵卒喝得神志不清,打赌去爬那青铜古树。

(文章到了这一段,下面全部都是不知名的文字,不知道是否有特别的用意,凉师爷无法看懂,实在遗憾。)

凉师爷告诉我们,另一个老板李琵琶能够看懂这些东西,但是问他下面写的是什么,他决计不说,神秘得要命,这一点,不知道是什么缘故。

《河木集》最后,有一段汉字记录着攀爬过程,我们这个位置再往上,会有绕着岩壁的栈道,是当初他们为了最后让皇帝来看的时候准备的,可惜修到近顶的时候就修不上去,而且修栈道的时候,经常有人无端由的坠崖,后来就不了了之。

我们爬出矮洞,王老板递给我一只望远镜,自己打着强光手电给我照明,调整了焦距之后,果然看到上面不远处,似乎有几段木头的栈道卡在崖壁之上,几个盘旋一直向上。我们的手电电源微弱,照不到这么远,所以当时没有发现。

王老板的意思是,如果能到达那条栈道,沿着它攀爬可以省不少力气,只不过栈道之上必然会有蹊跷,凉师爷是文人,让他研究东西行,打仗就不行,所以这路还得我们两个去走。

我没他这么乐观,拿着望远镜看了半天,也没看清楚这些栈道到底是个什么样子,这里光线太昏暗了,加上栈道的边缘似乎给一些植物根须一样的东西裹住,与在旅游区爬过的那种钢结构栈道有很大的不同。《河木集》写于南北朝代,传到今日时隔千年,这些栈道是否完整还不清楚,更不要说结实不结实了。

王老板说,当年修这条东西,是用来给皇帝游览用的,不是采掘的临时栈道,所以在用料和做工上一定非常讲究,现在很多汉代的古建筑都非常牢固,所以他认为问题不大,实在不行,我们还有大量的绳索,有了这些栈道,爬起来自然也方便得多。

他说得非常决绝,一点也不给人商量的语气,我暗骂一声,只好不再发表意见。他和凉师爷又稍作商议,决定再让我休息十五分钟,然后胖老板带我上去,凉师爷和老痒留在这里。

刚才睡了一觉,精力恢复了很多,又吃了点东西。王老板也坐了下来,用广东话和凉师爷聊起了天,我并不是很能听懂,不过大概也知道他们聊的事情,跟那胖老板说的麒麟竭有关系。我对这事情,心里一直有个疙瘩,心想反正现在和他们的关系表面上缓和了,正好乘机问个清楚,就问凉师爷,这麒麟竭到底是什么?会不会有什么危害?

凉师爷说道:“关于这方面完全不用担心,我刚才没把事情全告诉你们,是给自己留一手,以防你们跑路的时候,给自己留下换命的资本,现在既然咱们已经正式结盟了,我也说出来,免得你心里不舒服。”

麒麟竭就是麒麟血凝结成的血块,是一味非常名贵的中药,不过它却不是真正的麒麟的血,而是一种植物的汁液,这种植物叫做麒麟血藤,又名血蛇藤,一般在比较靠南边的地方才有。

麒麟竭放置的年代越久,功效越好,初期它只有一些普通的功用,一般用来入药,但是在中医里面,还有一种罕见的用法,就是用来熏尸。古时候有些少数民族和一些山村里的习俗,会将一块麒麟竭压在尸体的肚脐之上一起入殓,可以剔除尸体的阴气,尸体虽会腐烂,但是不会招来蛆虫。

麒麟竭随着年代的逐渐长远,会逐渐由暗红变黑,年代越久黑得越沉。到了一定的时候,性质就会改变,变得入口即化,人吃了以后,邪虫不近,夏天连蚊子都不敢找你。

当然这只是传说,凉师爷也只是听别人说过,今天第一次看到这种情况,才开始相信有这么一回事,至于会不会有什么副作用,没有相关的记录。不过中药一般毒性很低,他让我不用担心:“与其想这些,我觉得最麻烦的还是那些蛊虫,《河木集》记载开凿的时候,并没有挖到任何这种面具,到底是不是古人布下的疑阵,还是杀光外面千口人命的手动的手脚,我还不能肯定。你们上去的时候,还是要多加小心,不可大意。”

我们休息了片刻,老痒还是没有清醒,胖老板取下装备给我,我带上战术头灯,背上绳子,继续向上方栈道的边缘进发。

按常理到达那条栈道并不远,但是现实中总有一丝无奈,目测的距离总是要比实际距离近很多,我们预计一个小时就要登顶,结果半个小时后才勉强爬到栈道下方。

我这才发现,胖老板的说法是对的,栈道保存得非常好,倒不是因为皇帝要走的栈道所以修得坚固点,而是栈道一直在修葺当中,所以外面还有一层油竹竿搭成的脚架,这种东西非常防潮,经过几百年的腐蚀,仍然非常结实。走上去还能听到韧性的嘎吱声。

这里应该十分贴近地表,从边上的绝壁上垂下很多树木的根系,犹如缠绕植物一样缠绕着边上的扶栏。有些根须非常粗大,简直就像章鱼的触手一样挡在栈道上,越往上这些东西就越多,非常难以行走。有几段整个被根系包在里面,几乎找不到立足的地方,只好用砍刀开路,或者干脆爬过去。

因为树木根系的侵袭,这里的岩石开裂,不时还有石头掉下来,我们一边抱着头,一边还要小心脚下,走得竟然感觉比爬的时候还累。

我们只顾着走,也不知道上去了几圈,前面的栈道出现了一道非常大的缺口,有将近十米的距离,因为边上的岩石迸裂,塌了下去。我比画了一下距离,对王老板说:“没办法,跳不过去,要上绳子了。”

此时离我们出发已经快一个小时,但是从上往下看去,仿佛并没有上来多远,看来想在一个小时内到达树顶已经不可能了。我们之前爬得太急,体力消耗得非常厉害,只好暂时先休息一下。这个垂直的溶洞里非常阴冷,又非常潮湿,我走了这一段,身上的衣服全部都是汗水,粘在身上非常的难受,一时半会又干不透彻,很容易生病,一定要想办法取个暖才行。

我们找了一个树根和栈道包在一起的树根洞里,王老板将固体风灯拿出来,用匕首挂在一棵树根上。我脱掉衣服先将内衣烘干,然后胡乱吃了一点东西,王老板表情非常严肃,一边和我说着话,一边用强光战术手电去照对面的铜树,照了一会儿,他对我道:“你来看,这里已经能看到顶上,上面是什么东西?”

我拿起望远镜观察,上面大约只有十几米的地方,已经是铜树的顶部,从洞的上面垂落下很多树根,将那一片区域全部挡住,勉强可以看到,那里被裹在一大团根系里,大量根须一直顺着铜树缠绕下来,里面有什么东西,实在是看不清楚。

环绕洞壁向上的栈道,还要比这铜树的顶部高出很多,这个和《河木集》记载的不同,有可能经过长年累月的挖掘,沉重的铜树有再次沉入岩层中的趋势,几百年下来,高度已经下降到栈道之下了。

这些从洞顶上垂下的根须,可能就是我们来的时候,从金鱼山顶上看到的那几棵十几人才能环抱的大榕树,现在看来,它们的根系比它们的枝叶还要壮观,这些犹如苍白的鬼爪一样的东西,犹如麻花一样拧在一起,就像一只巨手,抓住这一根铜柱,想将其从地狱里拉出来,又好像一根缠满了化石巨蟒的巨大图腾,让人浑身起鸡皮疙瘩。

我正看得入神,却听胖老板对我说道:“你看树根长得如此茂密,说明这里的岩壳上面应该就是表土层,这里是一个天然的溶洞,古人来祭祀不可能是穿山进来的,上面一定有一个洞系可以通到外面,弄不好,我们不用原路回去。”

我听他话里有话,心里一喜,如果不用原路回去,那真是一件美事,可这天然的溶洞,必然也不是什么平和之地,到时候能不能走得出去,还要另外合计。王老板推了推我,说道:“这铜树顶上是这么个情况,不过你看那几个根堆里,好像有一座铜像,这里太远,看也看不清楚,咱们换个地方去看个仔细。”

我顺着他手指的方向看去,看到柱顶的下方,根堆缠绕中似乎有两只青铜雕刻的手臂,与我们在夹子沟看到的那一座有一丝妖冶的雕像遗迹非常类似,只是当时它的脸被盗墓贼炸烂了,我当时有一种很奇特的第六感觉,总感觉到这张脸会有什么不妥当,如今正好看上一看,这家伙到底长什么样子。

\chapter{老套路}

按道理,要看到那雕像的脸不难,可是我们是由下往上仰望,无论走到哪里,因为角度的关系,仍旧看不清楚。我心中懊恼,对于雕像的不吉的感觉也越来越浓了。

王老板大概也和我有同样的感觉,越是想看到,越看不清楚,急得他脸色铁青。我们换了几处地方,皆不满意,最后还是决定先爬过坍塌的栈道再说,这里的岩壁上全是树根,爬起来也不会有多大困难,加之下面还有几层栈道,如果失足也不会摔死,没什么好担心的。

我们再次回到那一段坍塌的栈道边上,王老板检查了一下那些垂下的根须的结实程度,用多功能镐挂住,敏捷地爬到峭壁上。我一边给他打着手电照明,一边诅咒他掉下去,可惜这王老板的身手和他的体形非常不相配,三下五除二,已经攀到了对岸,跳到栈道上。

他回头将多功能镐抛回给我,然后自顾自向前跑去,大概是心急想看看那上面到底有什么。我打开头上的头灯,学着他的样子爬上峭壁,一手挂着多功能镐,另一只手摸着根须前进。这些东西不知道生长了多少年,摸上去竟然犹如石头一样,坚硬异常,不似有生命。上面的纹路也很似动物的鳞片,如果眼神差点,肯定以为是什么古生物的化石。

我爬得很小心,进度很慢,才爬到一半的距离就听到王老板叫道:“快到我这里来,这里可以看得清楚点,那团树根里面好像还不止……一座雕像。不知道到底雕的是什么。”

我听到他的话,咬紧牙关,手脚并用,最后抓住一根根须荡到对岸,然后寻着他的手电追去,看到他已经绕着栈道上了三层,正举着望远镜,查看铜树那里的情况。我向他望的地方看去,因为角度变化,的确可以看到有一些东西被裹在树根里面,但是具体是什么,还是很模糊。

我气喘吁吁地跟上,接过他的望远镜之后,才看清楚,在蟒蛇一样的巨大树根团里面,露着很多生锈的青铜手臂。从数量看来,里面应该是最起码有四座雕像,立于四个方向。凭借露出的部分,也无法准确地判断雕得是不是同一个造型,其他的部分给深深裹在树根里面,目测一下,尺寸很大,大概和我们在山崖上看到的那座石头差不多大小。

老痒所说的“大好处”,不会是这些恐怖的树根,那肯定是这树根里包的东西。但这些雕像就算真的是有什么莫大的价值,我们也带不走啊,对面应该还有什么蹊跷是我们所不知道的,待在这里绝对发现不了,一定要过去才行。

我们继续顺着栈道上前,因为靠近溶洞的上段尽头,崖壁与铜树之间的距离也逐渐接近,我们看得也越来越清楚。铜树之顶原来有一个圆形的祭祀台,朝四个方向有青铜的四座雕像。本来我们以为换几个方向就能看到雕像的真面目,可是越往上越失望,它们的身体和面孔都牢牢地裹在了树根里面,想要看清楚,不砍掉这些树根恐怕是不太可能了。

我们来到栈道上与那祭祀台基本平行的地方,王老板停了下来,看了一会儿,对我说道:“这四座雕像放在四角,说明中心肯定还放着什么东西,本来如果我们的装备都在,可以再往上一段距离,用聚光灯照个清楚,可惜这些东西都掉进瀑布里了,没办法,后生仔,我们得过去再说了。”说着他已经将多功能镐有刃口的一端折了回去,折成钩子形状,绑到绳子上,做成一只飞爪,像西部牛仔一样甩了几个圈后扔了出去。

多功能镐甩了一个抛物线,钩在了对面祭祀台边上的一根树根上,绕了几圈,正好钩回到绳子上。王老板拉紧绳子,拉得树根抖动了一下,很多奇怪的灰色虫子从树根的缝隙里给惊了出来,四散而逃,速度很快。

王老板皱了皱眉头,说道:“后生仔,这次该你先上了嘛!”

我知道是他忌讳这些虫子,心里暗骂了一声,目测了一下距离,这里比我们刚才爬的时候近了很多,应该问题不大,于是点了点头,爬坡上绳子。

才爬了几步,我也不由得佩服起王老板,这绳子甩得真好,两端成一个大概六十度向下倾斜的角,只要双腿夹住绳子,自然就会滑向对面,不用花一点力气我凌空划过,一下便到了祭祀台上的树根上,立即抓牢上面的根须站稳。

王老板在对面做了个手势,让我先探察一下形势,我回头一看,那些灰色的虫子并不是螭蛊,面是一种类似蝉的幼虫的昆虫,数量颇多,但是应该不会有什么危害。我赶走它们,对对面的王老板做了个手势,他用手电照了照我的四周,确定真没虫子了,才爬上绳子。

这里的树根几乎都有我的两三根大腿粗细,纠结在一起,碰到的地方已经融成一体,没碰到一起的地方就镂空为一个个窟窿,时间长了,融到一起的地方多起来,里面镂空的窟窿就四通八达的,这在榕树林里面很常见,有大片榕树的地方,甚至整片林子都粘在一起,里面一个树洞连着一个树洞,进去就出不来,比鬼林子还邪。

我们抓着树根转了一圈,发现这里年代实在太久,包得非常彻底,看不到下面是什么。这些树根又砍不动,不知道如何是好。呆了片刻,王老板说可能从这些树根之间的镂空里看下去才能看到,咱们分头找,一个洞一个洞照过来,肯定能看到。

我心说盖得这么厚,这也不太可能,不过他没准备和我讨论,只是抬了抬手让我去做。

我隐约地感觉这人十分的暴戾,和以前我认识的那个王胖子有点像。心说他们俩该不是亲戚吧?不过我的那个王胖子可可爱得多了,而且很爽快,这个人太阴了。

这些树根盘在这里,像一个坟墩一样,用手电照到那些镂空的窟窿里,也照不到底,我们搞了半天,累得一头是汗,还是什么都看不到。我还把腰给闪了,酸得我直冒冷汗。

两个人这下没办法了,王老板看了看我,忽然骂了声:“王八蛋,难道李琵琶这衰人算计我?”

我心里也嘀咕,这里既然什么都没有,为什么老痒要这么强调。他应该不会开这种无聊的玩笑,问题还是在我们身上,到底出在哪里,哪里疏忽了?

两个人都不说话,静静地在那里想事情,我想着老痒一路过来和我说的谎话,这些谎话不管是处于什么心态,无非是想把我引到这个地方来,可到了这里之后,却什么都没有看到。而那个所谓的不能告诉我的,而且就算我知道了也是不会去做的好处,到底是什么?现在还是一点也看不出来。

正想得出神,王老板突然推了我一下,我转头刚想说话,他做了个别出声的手势。

我心说干什么,他摆了摆手,小心翼翼地拉我蹲下来,仔细去听那树根里面……

我立刻凝神静气,侧耳去听,这里没有风声,在这寂静无比的溶洞里,贴着那树根,清楚地听到树根里面传来一声一声的轻微的“的……的……的”声,好像有人被冻得磨牙。

那声音不大,不注意必然听不见,很有语音规律,和血尸的声音完全不同,也不会是那些虫子在树干里爬行发出的声音。

王老板轻声说道:“这声音每一声的间隔都一样长,好像和尚敲木鱼一样,有可能是什么机关动作的声音,这里面的确有东西在,只是不知道是活物还是死的。”

我开始冒出白毛汗,这几千年的老树根里竟然有人磨牙,难道是遇到了树妖不成?我刚想说话,王老板抿着嘴巴摇了摇头,举起短步枪,拉上枪栓,让我跟上,蹑手蹑脚地循着声音走去,我们走到一个榕树根洞边上,发现声音就是从这里传出来的,王老板打开手电往洞里一照,声音戛然而止。

他瞄了一眼我,轻声说道:“没错,应该就是这里,《河木集》说的东西就在这里面,可能得从这里进去才行。”

我皱了皱眉头,说道:“这里面的根系洞非常复杂,比那些溶洞地形的洞系要复杂得多,而且不知道这铜柱是不是空心的,贸然进去,可能会有危险。”

他点了点头,说道:“我知道,所以我们两个不能同时进去,先下去一个探路。”

我心里咯噔一声,心说你该不会想让我进去吧。

王老板看我犹豫了一下,把短步枪举了起来,轻声说:“我太胖了,你先下去,我跟在你后面,给你殿后,你放心,不会出事情的。”说着他推了我一把,将我往那个洞里推去。

我低头看了看下面,一片漆黑,回头一看,他正面目严峻地看着我,脸上透出一股子阴冷的表情。我咬了咬牙,只好又带上头灯,再次充当趟雷的角色,刚想进去,胖老板又把我叫住,递给我一只小型的对讲机,说道:“如果里面很深,就用这个,去吧,后生仔有前途。”

我心里暗骂,接过来,先熟悉了一下使用方法,然后放进兜里,说道:“王老板,咱们明人不说暗话,我这是给你去拼命,你怎么样也要给我点武器,万一我挂在里面你也就没戏了,对吧?你不给我枪,冷兵器总要给我一把吧?”

王老板戒备地看了我一眼,大概觉得我说的也有道理,不情愿地从自己的靴子里掏出一把小匕首,丢给我,同时枪口马上就指向我,笑道:“你看,我这人糊涂,就给忘了嘛。”

我接过匕首,发现是那种长柄猎刀,专门用来刨皮用的。心说有总比没有强,操了一声,头一低钻进洞里,闻到了一股霉味,我带上防毒面具,才继续向里爬去。

里面非常的潮湿,树根的表皮与外面完全不同,非常松软,还有很多不知名字的蘑菇长在里面,很多蝉的幼虫受到我的惊吓,开始逃窜。我往里爬了一段,一下呆住了,前面至少出现了几个岔口,该走哪一个?

仔细一看,其中一个岔口上有一个标记,应该是前人画上去的,不管了,我爬向那个有标记的岔口,又前进了几米,突然前面一空,上半身已经探了出去。

我上半身挂在洞口,打开头上的探灯四处一照,这里是一个矮小的空洞,里面盘根错结,全是树根。说得实在一点,这里不过是整个根包里根须比较稀疏的地方,我正觉得奇怪为什么会出现这种情况,忽然看见树根的里面,有一块石板露出一角。

仔细一看,那竟然是一只巨大的石头棺椁。棺椁下面有一个棺床,现在也给裹了个结实。从我刚才爬的距离来判断,这里应该就是祭祀台的中央没错,这就是我们要找的东西。

我手脚并用,来到露出一角的石棺椁边上,这才看清楚,这东西还不是一般的大,几乎像一只袖珍的集装箱了,椁盖的边缘和铜树上一样阴刻着一圈双身蛇。其他部分几乎和树根长在一起,上面有什么浮雕无法知晓。

王老板在外面大叫了两声,我正给看得蒙了,也没回他,他以为我下到铜树里面去了,从对讲机里问道:“后生仔,里面有什么?”

“有一只棺材!”我说道,一边尽量找一个地方至少能让我坐起来,趴着太难受了。

“棺材?能不能看出是谁的?”

我骂了一声:“我怎么知道,不过这棺椁给运到这里也不容易,如此兴师动众的,里面躺的可能就是这青铜树的修铸者。”把自己的棺材放在这里,大概想着升天的时候,离天宫近一点。不知道这到底是什么人物,竟然有这么大的手笔。

这个时候我看到棺椁的盖子和椁身并没有密合在一起,有一段树根已经顺着缝隙长进了棺椁里,将盖子抬起了一点。我感觉到很奇怪,“嗯”了一声。

王老板听了很紧张,忙问:“怎么回事?”

“这棺材……盖子没盖好。”我说道,向那缝隙爬了过去,难道入殓的时候棺椁没盖好?让树根长了进去?

我想了想,觉得也不会,可能是细小的树根须长入棺椁盖之下后,不断长粗,将盖子抬了起来。这些树根四通八达的,说不定已经撑满了这只棺椁,表质层这么硬,我们手里的这些家伙就算能砍得动,也不知道猴年马月能挖出来。

我爬到缝隙边上,用探灯往里面照了照,里面似乎是全空的,灰蒙蒙一片,光线好像给什么吸收了一样,什么都照不出来。

历来考古中,从椁中将棺材起出来是最麻烦的。正规的棺椁,都是棺壁贴着椁壁,最多给你留一公分的空隙就很不错了,这一具却反潮流,里面有着相当大的空间,十分怪异,不知道又是什么讲究。西周时期的墓葬习俗已经比较成熟,就算是王宫贵族也不会使用如此离谱的墓葬方法,看样子凉师爷说的没错,这里应该是当时少数民族的一处王墓,并且这一个国力似乎也不弱,至少应该与当时的西周王朝不相伯仲。

我拿起对讲机,说道:“这棺椁是空的,里面不知道有什么,我的探灯没你手电这么厉害,太暗,你可以进来了,这里很安全。”说着,我已经向我刚才探出来的那个洞爬去,心说只要你一探出头来,老子就卡住你,看你怎么办。

对讲机发出几声静电干扰的声音,里面传来几声声音,我听不清楚。

“什么?”我问道。

随着几声静电干扰,从对讲机里传来了一些奇怪的声音,非常嘈杂,一点也听不清楚。

“什么?”我不耐烦地又叫了一声。

\chapter{鬼雾}

在这狭窄黑暗的空间里,一只棺椁边上,突然从对讲机里传来类似鬼魅一样的呼号声,既像有人在哭泣,又像有人在发抖着念着什么东西,让我着实吓了一跳,我赶紧将声音关小,拍了拍看看是怎么一回事。

这是Moto生产的军用对讲机,使用塑胶外壳,非常适合在恶劣条件下使用,照道理不会这么容易出故障。我开关了几次,开始的那种怪声倒是没了,扬声器里却断断续续地发出呲呲的静电声,似乎是有人呼叫,又无法听到清晰的语句。我连喊了几声也不见好转,调动频率,也没有作用。

我摆弄过这些电子东西,知道这种动静并不是物理上的故障,而是电波干扰,产生的原因很多,大到太阳黑子爆发,小到家用电器运转,都会产生相同的效果。我们现在深处地下,给太阳黑子影响到的机会不大;这种深山老林里的溶洞里,也不会有什么家用电器,这种干扰到底是哪里来的?

我将对讲机四处移动,寻找干扰的源头,很快我便发现,只要将它靠近巨大的棺椁,嘈杂声就会严重,如果离得远一点,嘈杂声就会减轻,非常奇怪。难道干扰源竟然在棺椁里面?我将对讲机小心翼翼地伸进椁盖和椁身的缝隙,刹那间,那种嘈杂声音突然爆发到了离奇的响度,就好像有人突然间惨叫了起来一样。吓得我手一松,几乎把对讲机掉进棺椁里。

糟糕,我心里想,看样子没错,棺椁里面有什么东西正在发射不规则的电磁波,这太不可思议了,是自然现象,还是有什么古怪?

我知道植物是可以发射微弱的电波信号的,而且在不同的外界条件下,植物发出的电波信号也不相同,比如说你给它播放舒适的音乐的时候,或者用刀割它的时候,它发出的是两种完全相反的信号,这被称为植物的语言。可是这些信号都是极其微弱的,就算你用专门的仪器都不一定能探测到,不用说给普通的对讲机接收了。

还有一些特别的情况,也能够在自然条件下产生强烈的电磁波影响通讯,比如说地震前夕,或者火山爆发的时候,但是这种干扰是带有破坏性的,绝对不会像现在这样温和。

我看着这巨大的棺椁,想到了一个不太可能的可能,就是在大规模的屠杀或者大型的土葬墓地附近,经常会有奇怪的电磁波干扰,持续不断,一说那是尸体腐烂发出的能量产生的,一说那是大量鬼魂发出的信息。这强烈的电磁波,会不会是棺椁中的尸体发出的呢?

这里的光线极其晦暗,老榕树苍白的根部在探灯的照射下,看上去就像一根一根畸形的蛇骨,加上这让人发麻的嘈杂声,就像有什么东西正在棺椁的内部,狂叫着催促我进去。我感到出了一身鸡皮疙瘩,无比的烦杂,赶紧将对讲机拿出来关掉。

四周安静了下来,我一下子感觉到头晕,大概是这里潮湿的空气和古怪的味道让我开始缺氧,看着周围的环境,心里感觉到一阵发寒,这是我一路上都没有感觉到过的。

王老板一直在外面大叫,想必是听不到我的回答,正急得直跳,他的喊声经过树根里三层外三层的过滤,到我这里已经变得十分微弱,就像人在十几层被子里面听外面的人说话,很难听得清晰。

刚才我还考虑着把王老板骗过来,在这里制服他,现在却已经改变了主意,想着是否还是暂时先退出去好,这地方邪得慌,待得久了真让人全身不舒服。这主要还是一个人的原因,如果有两个或三个人在我身边,应该能镇定很多。

考虑再三,犹豫不决的老毛病又犯了,就是拿不定主意。外面的王老板叫了一会儿也就不叫了,我听到他在外面大声地骂了几句,就静了下来,大概也不知道怎么办好,谅他的脾气,应该不敢钻进来查看。他们这种跑江湖的人,虽然在社会上万般的强横,但是在这种诡异的地方,又听到有棺材,还是有着本能的畏惧。棺材代表着钱和权力不能控制的死亡,是非人力所能撼动的权威,这一点倒斗的人反而很难体会。

正出神地想着,忽然,我又听到了那磨牙一般的“的……的……的”的声音,不知道从什么地方响了起来,比刚才在外面的时候要清晰得多。

现在听得真切,这种声音,像是有人穿着木屐走在石头地板上的脚步声,但是这声音没有起伏,不像是在来回走动,倒像是在……不停地跳。

声音非常有规律,一下一下的,在这寂静的环境里,分外让人觉得心惊肉跳,我刚刚已经给吓了一跳,现在听起来,简直像催命符一样,我的心脏也跟着这个节奏颤抖进来。

一时间我感觉到有点奇怪,我怎么会这么害怕,我应该已经克服这种恐惧了。我镇定了一下,拿下了我的防毒面具,闻了闻四周真实的味道。一般来说,防毒面具能将一些对人体有害的异味清除掉,所以带着防毒面具,闻到的味道是加工过的。有时候一些有毒物的标志性气味会给过滤掉,但是在特殊情况下有毒物却还是能够穿过面具,反而会造成中毒。

四周的味道对鼻黏膜非常的刺激,我刚吸了一口就打了个喷嚏,浑身冒冷汗,赶紧又把面具带上。

我听了一会儿,声音并不是来自其他地方,按照方位来看,好像是从石头棺椁的内部传出来的。

我开始冒汗,一手拔出了长柄猎刀,匍匐着向那缝隙靠近,想听个清楚。可是自己的心跳反而越来越响,等爬到那棺椁的缝隙边上的时候,心跳得简直就要从我的嗓子里跳出来了。

我知道自己是给这里的环境感染了,有一段时间我以为自己已经克服了这毛病,现在看来还没有。想象力丰富是做这一行的大忌,我一边提醒自己,一边宁神静气,脑子里想象着四周的光线明亮起来,并没有这么黑暗,又深呼吸了几口,总算压下了躁动的心脏。我叹了口气,转过耳朵,想好好分辨这到底是什么声音。

可就在这个时候,那声音突然停止了,一下子就是鬼一样的寂静,我被这突然的变化吓得浑身一紧,同时,我忽然感觉到,好像有一只什么东西突然搭到了我的肩膀上!

我头皮一乍,眼前几乎一黑,人疯了一样地回手就是一刀,一下子探灯就撞到了一根树根上,立即熄灭,四周变得一团漆黑,紧接着,我的手被什么给缠住,拼命向后扭去,我吓得完全失去了思考能力,号叫了一声,用尽了全身力气想翻过身来,一挣扎,身子下面的一根还未完全角质化的树根咔嚓一下,我整个人一沉,和我身后的东西一起掉进了一个浅坑里。

我掉下去的同时,忽然听到有人骂了一声:“你个衰鬼!”然后手电就亮了,王老板一边紧紧压着我,一边用手电照着我的眼睛,照得几乎要瞎了。我刚想用手去遮,突然就给他甩了一个巴掌,完全没有留力,我鼻子马上就是一凉,开始流鼻血。

他打完我之后,又狠狠骂了我几声,说道:“你个仆街仔,给你脸你不要脸,跟我肥佬玩花样,你去死吧。”

我马上就意识到是怎么一回事,他娘的这广东来的死胖子竟然有胆子偷偷摸进来,这人大概是看我没反应,以为我在跟他玩花样,又忌讳我在里面,怕进去之后着了我的道,竟然没开手电,偷偷爬了进来,正碰上我在听那鬼跳声,结果差点就给我回手一刀给做了,现在大概是以为我想杀了他。

我想解释,但是他卡着我的脖子,我说不出话来。他好像气得够戗,又是一巴掌,打得我耳朵嗡的一声,我一下子心头火起,心说我操你奶奶的,敢这样打人,说明根本就没把我当人看,当即一头就撞了过去,将他撞了个结实,两个人又滚在一起,你一拳我一脚,一下子滚到棺椁缝隙的边上,他力气比我大,一下子又占得上风,把我压在身上,抬头就想掐我,结果这里太矮,他头一抬,撞在一根树根上,把他撞得一愣,我乘机猛地一脚顶在他的胯下,将他顶翻了出去,然后扑上去抢过他的手电,对着他的脑袋就是一下,将他砸蒙了过去。

我压在他的身上,看他暂时无法动弹,就用手电去照四周,发现这鸟人的装备和枪都没带进来,想必是觉得里面太狭窄,怕走火伤到自己。我又去摸他身上,想去拿他的匕首,突然他将我向上一顶,我也和他一样,一头撞在顶上,撞得眼冒金星,急忙翻到一边,免得再给他顶一下。我脑浆都要从鼻子里出来了。

王老板爬起来,身上全是根系的细须和被碾碎的菌类植物,脸已经气得扭曲了起来,喘着粗气,眼睛都红了,我知道他动了杀机了,像他这种混混起家、一步一步爬上来的人,杀心肯定很重,动不动就想置对方于死地。看来这一次,真的要拼个你死我活了。

王老板顺了顺气,从皮带中拔出匕首,反手握住,气势汹汹地向我逼近过来,我的短柄猎刀比他那把匕首短了整整一半,就算能捅到他也伤不到要害,此时只好拿手电做武器,追着他的眼睛照,不过这死胖子非常凶悍,根本不来看我,一边转头避过强光,一边就闪电一样冲了过来,一刀就划向我的脖子,我矮头躲过,左手抓住他的手,右手突然熄灭了手电。

他的眼睛已经习惯了强光,突然间熄灭,他下意识地就停了一下,我记住了他脑袋的方位,飞起手电,抡圆了胳臂就是一击,黑暗中我听到一声闷哼,手电竟然给砸得亮了起来。我对着他的位置一照,看到他已经给我打出一嘴巴的血,正倒在那里,似乎快没意识了。

我不知道他是装的还是真给抽晕了,用力一脚将他踹向那个缝隙,如果他没昏,肯定得反抗,不然他就要掉进棺椁里去了。我一连踹了好几脚,他的双脚先滑了进去,可惜到胸口的时候,给卡住了,我上去又补了一脚,用力将他往里面顶。

王老板像死鱼一样卡了很久,一下子滑进了缝隙,在那一刹那,我总算松了口气,心说果然是昏过去了,就在这时候,突然一只胖手从缝隙伸了出来,一下子抓住我踹他的那只脚,猛地就往下拉去。

这一下真是猝不及防,我已经全身放松了,只觉得眼前一花,已经整个儿给拖进了棺椁里。我心里直叫完蛋了,竟然掉进去了,这真是前无古人、后无来者的事情,慌乱间去抓四周的东西,一下子却什么都没抓住,直掉进无穷的黑暗里!

王老板拉着我一路下滑,我原本判断这棺椁也就一人多高,现在一进去才发现不对,这里面有一个凹陷,看样子的确是凹进了铜树的里面。我一连滑了大概三四米,才一屁股坐在什么上面,疼得我一龇牙,同时王老板也松了手,似乎想要再次扑上来。

我马上用手电照射四周,想看看王老板在不在我边上,一扫之下,只看见满眼的雾气,灰蒙蒙一片,半米外就什么都看不到了。

我站起来,用手电大力地甩了几下四周,什么都没有打到。这里雾气这么浓,王老板掉下来之后,肯定也是什么也看不清楚,大概躲藏到雾气里面去了。

我感觉到很奇怪,怎么会有这么大的雾气在这棺椁里面,要说是熏香,千年还不散也不太可能啊。我用手拨了拨,雾气之浓,简直好像是水一样,一拨之下竟然出现了肉眼看得见的气流漩涡。

棺椁中间的东西一点也看不清楚,我也不敢走进去,只能先看看我滑下来的那一边能不能爬上去,向上看去,也看不到什么,只发现树根从缝隙中生进来,似乎并没有非常肆意地生长充满里面,只是像爬山虎一样贴着棺椁的内壁和底部,树根上面张满了类似于绒毛的真菌,一摸就掉,有点像霉菌丝。

棺椁内壁没有给树根覆盖的地方,有一些浮雕,我一眼就看出,里面的一些图案,应该就是与外面立着的那四座雕像一样的风格,不过这些图案也大部分给遮住了。长柄刀的刀刃太薄了,用来切上面的树根还是有点吃力,我将一些发散的新生根须切下之后,那些已经角质化、和椁壁黏在一起的主根却毫无办法,一刀下去就像切在石头上,只能切出一条白线。

虽然如此,我还是能分辨清楚一些内容,那应该是修筑青铜古树时候的情景,上面的人穿着左衽的衣服,出乎我意料的是,我发现上面的青铜树是分节的,看来这根巨型铸器并不是一次性修铸成的,可能历经了好几代人,一节一节地铸接,最后才成为这么壮观的艺术品。

浮雕很多,但是我不敢随意走动,看完了背后这一块后,我回头看了一眼雾气,只觉得一股莫名的恐惧传来,于是踩着边上的树根,想顺原路爬回去。

可是奇怪的是,看似非常利于攀爬的树根,我上去了两次,都很快滑了下来,简直和踩在冰上一样。我一摸上面,发现这些真菌给压扁之后,非常的滑腻,像润滑油一样,要爬上去,一个人似乎挺困难的。

我定了定神,心里想着该怎么办,看样子得把上面的真菌先刮了,才能上去,或者把刀当成登山镐,也不知道行不行。

正思考的时候,“的……的……”一阵异常清晰的怪声,突然又出现了,这一次,是在我的背后,似乎十分的近。

\chapter{偷袭}

将我们引入的这诡异怪声突然出现在我的背后,虽然声音不大,但在寂静无比的棺椁内却犹如炸雷一样,无比的清晰,听得我浑身一颤,脑门上的肌肉一紧,又是一头的冷汗。

这个棺椁大概有六七米长短,说长不长,说短不短,由着声音判断,声源应该离我不超过一米,那几乎就是贴着我的后背,可以拍拍我肩膀的距离。“的……的……”有规律的一声一声,简直就是靠着门板听敲门的感觉,一股凉气由我的后脖子一溜到底,直下到我的脚后跟。

一时间我全身的肌肉都僵硬得无法动弹,考虑着要不要回头去看,还是想装作没有听见这声音,不去理会它。不过马上我就反应了过来,自己也哭笑不得,咬了咬舌头提醒自己:要镇定下来,这个时候其实根本没有选择,只有去面对,害怕和找借口根本是等死的表现。

僵持了片刻,那鬼魅一般的声音不急不缓,既没有再度靠近,也没有远去,我深吸了一口气,咬牙握紧短刀,缓缓地回头,去看后面到底是什么。

随着我回身的动作,那怪声突然停止了,我定睛一看,在我背后的灰色雾气中,却什么都没有,刚才怪声传来的方向,仍旧是一片灰蒙蒙的,只是给我的动作所扰动,出现了一些诡异的气流,很快就平复下来,变得和刚才一样均匀。

我咽了口唾沫,觉得有点意外,用手电照了照四周,没有任何的异常,那声音好像从来没出现过一样。

刚才声音离我如此之近,我听得无比清晰,绝对不是错觉,我转身的动作也就一秒钟左右,如果是由什么移动的物体发出的,它也不可能以这么快的速度消失掉,难道,声音来自别的地方?是我判断错误?

我下意识地往前跨了一步,想去寻找声音的来源。突然间,一个人影猛地从我边上的雾气中扑了过来。我眼睛很贼,正好瞄到出现状况,急忙矮身,那人影没有抓住我,但是还是将我撞倒在地。我就地一滚,回头一看,撞我的那人体形肥胖,正是将我拉进这里的王老板。

我骂了一声,亮出短柄猎刀,想与他做个了断,没想到他一闪之间又躲进了雾气里,不见了影子。

我不由鄙夷地吐了口口水,刚才搏斗中他的匕首应该掉在了外面,现在忌讳我手里的短刀,不敢和我正面冲突,而躲在雾气里,等着我靠近,然后实施突袭,和刚才的那种嚣张劲完全不一样。他娘的肯定是个小人。

不过话说回来,这里的情况这么诡异,这家伙的胆子还不是一般的大,要是我,既没有手电也没有武器,哪里还敢偷袭别人,早就缩在角落里发抖了。好在这里的雾气浓得像水一样,一有什么东西运动,就会出现非常明显的轨迹,他想偷袭我也没有这么容易得手,否则刚才那一下,我已经给他按倒了。

我想到这里,又觉得奇怪,如此说来,那怪声的主人,如果是在这棺椁中移动,必然会产生移动的轨迹,可是我刚才看去的时候,雾气平滑,不像有什么东西移动过的样子,难道它没有形体吗?是只鬼?

我一边防备着王老板再次偷袭过来,一边站起身子,这棺椁里面的空间并不大,刚才一滚,不知道滚到了哪个位置,要赶快退到边上,想办法爬上去。

这里总体不大,现在向四周一看,已经贴近了棺椁的中心。透过雾气,我看到中心部分有一些东西,看影子,似乎是从棺椁的顶上挂下了很多的绳子,一直连到棺椁的底部。我以为是贴在顶部的树枝垂下的气生根,再往前一步,用手电一照,才发现不是,那些东西,都是手腕粗细的青铜链条,上面缠满了真菌和榕树的须根,一直由顶上缠绕到底,但是铁链好像只是给固定在了棺椁顶和棺椁底之间,下方并没有拴着什么东西。

这只石头棺椁说是巨大,其实这样的尺寸,西汉和五代的几个给大掀顶的贵族墓里都有发现。这东西说起来叫棺椁,其实应该叫做椁室才比较恰当,如果按照土葬墓,正式的内棺椁应该放在这个椁室的中央,财力雄厚的,石椁室内还要紧贴着十几层木椁,一直贴到最里面的椁边上。

现在我走了几步,按照棺椁的大小,至少也应该看到内棺椁的大致形状了,可是现在却只看到几根链条,地上不见放着东西。难道这椁里面竟然是空空如也的吗?那刚才的声音又是从哪里来的呢?那诡异的无线电干扰又是来自什么地方?

我愣了半天,又往前走了一步,想走到青铜链的中间去,看看它拴着的棺椁底上是不是有什么活门。才踏出去一步,忽然脚下一空,整个人向下掉去,我赶紧拉住面前的青铜链,滑下数米才定住身子,吓得出了一身冷汗。

怎么回事情,他妈的怎么好像踩空了一样?我心有余悸,手电向下照去,也看不到地面,下面雾气特别浓重,脚向下踩去,踩进雾里,竟然踩不到任何东西,似乎有一个很深的凹陷。

果然有蹊跷,我想,这椁室内嵌入青铜树顶上的祭祀台两米,中间什么都没有,可能是像战国时期那样的多层内嵌式椁法。这只椁室中间也许还有一处凹陷,叫做棺井,下面才是真的棺位,不知道这棺井有多深,真是好险,要是刚才一脚踩空掉下去,说不定会摔死。

这里的几根青铜链条,也许是将棺材放下棺井时用的起重装置的一部分,装尸体的内棺椁应该就在我的正下面。

正想着,突然边上的雾又是一阵扰动,王老板又冲了过来,这一次他手里拿着什么兵器,猛地就扑向我。这里雾气这么浓,大概是冲着我手电光点来判断我的位置的,我一看不对,下意识地大叫了一声:“不要!停下!”

但是已经晚了,王老板“哎呀”一声,一脚踩空,一下子就掉了下去。我感觉到下面的铁链猛地一震,大概是给他抓住了,同时我的手里发出了咕唧一声,身体竟然开始向下滑去。

我转头一看,原来是上面蘑菇一样的真菌给我的手挤压,压出很多滑腻的像油蜡一样的汁液,使得青铜链条有如涂了一层油一样。我心里大叫不好,急忙将短柄刀往链条的孔里一插,结果该死的还插不进去,三下五除二,刀卡在了树根里面,我用力一绞,才把身体停下来。此时我已经滑下去不下十米,进入到了棺井的内部,青桐树的树杆里面了。

王老板一头是血,吊在我下方的青铜链上,离我大约一只脚的距离,他也拉不住链条,用他的皮带穿过了一个链条孔,才勉强停住。我用手电照他,他骂着转头避开刺眼的光线。

我看他暂时对我构不成威胁,就去看棺井的情况,青铜树的树干内部与外部一样,刻着深入沟壑的双身蛇路,树根从上面蜿蜒下来,顺着纹路一路向下。里面的雾气比上面要稀薄了很多,我环视一周,迫切想知道这只在椁室中心的棺井有多大,如果太大,我爬出去恐怕又是个大问题。

棺井是一个长方形,四米长二米宽,正好可以容纳一只棺椁宽松地放入。我用手可以摸到棺井的井壁,不知道是不是因为雾气的关系,这里的树根并没有寄生大量的真菌,可以看见树根的本色。棺井里的空气漂浮着一股异味,可能是外面雾太多,防毒面具里面的隔离介质开始受潮,效果开始下降,我可以感觉到异味越来越浓,直呛我的鼻子。由此看来,王老板一定也不好受。

向下看去,我吃了一惊,可以看到铁链一直垂到下面的黑暗中、我手电照不到的地方,非常的长,从这里看下去,整个棺井深不见底,看上去竟然好像一直通了下去,没有底一样。

不会吧?我想,心里竟然有了一种感觉,难道整棵青铜树都是空心的,我们爬上来的高度已经不下三百米,这根铜树深入地下多深还不知道,如果是空心的,那它的底部到底会是什么地方?地心吗?地狱吗?这根巨形空心的圆柱体,插在这里又有什么意义呢?

王老板也看得非常惊讶,两个人都不说话,直勾勾地看着下面,忽然,“的……的……”两声作响,那种阴森的敲击声,突然又出现在了我们四周!

我和王老板对看了一眼,目光全部投向身下的一片幽黑中,那声音,竟然是从这下面的深渊传上来的。

\chapter{和解}

从这里听上去,这声音又有点不同,带着一点的回声,似乎是从很深的地方传来的。随着声音的节奏,我还可以清晰地感觉到,青铜链正在轻微地短幅震动,好像另一头正顶在一个巨人的动脉上一样。

这种现象让我心里升出一丝无法抵抗的寒意,因为我没有感觉到一丝风从下面吹上来,而我们两个人也没有办法使得如此沉重的青铜链产生这么高频率的震动,那下面的黑暗中,牵动着这几根青铜链的又是什么呢?

王老板若有所思地静静听着,照道理他没有经历过这种事情,应该比我还害怕才对,但是看他的表情,却出奇地镇定,似乎正在判断着什么。

僵持了一会儿,那声音终于沉寂了下来,青铜锁链也停止了震动,我没来由地松了口气,人几乎要从锁链上软了下去。

王老板仍旧没有反应,他静静地想了一会儿,拿出一支香烟点上,狠狠吸了一口,然后从口袋里掏出了一只小型的荧光棒,摇了两下,将里面的荧光摇亮。

我不知道他想干什么,冷冷地看着他。等到荧光棒反应到最亮,他突然顺着青铜链往下一抛,绿色的光柱便打着圈儿坠了下去。

光圈儿越来越小,迅速地消失在了我的视野里,我以为它会一直掉下去,直到消失在黑暗里,忽然,在看到和看不到的视觉极限处,荧光棒打在了什么东西上,“嘣”的一声弹了一下,飞到了一边的青铜壁上,又坠了下去,瞬间便消失了踪影。

这青铜链下面大概五六十米处的确挂了个东西,可惜荧火棒的光线太弱了,刚才那一下,我只看到一个大概的轮廓,似乎是一只水晶棺材,带一丝黄色,也可能是比较常见的商石棺(一种半透明的黄色石料)。

王老板抬头挑衅似的看了看我,忽然松开自己手里的皮带,一边打起打火机,一边开始向下滑去,很快,他便进入到了黑暗里,只能看到一点不断缩小的火光。

我考虑片刻,不知道为何觉得不妙,王老板似乎是胸有成竹,此人熟知各种奇异物品,难不成他已经知道下面是什么东西,而要去取?我想起老痒对我说的事情,不由得,也不甘心就这样落入他的手中,忙一扯手上的短柄猎刀,跟着他滑了下去。

下落的速度开始很快,上面缠绕下来的树根到了下面就没了,到了后段,我们的速度都慢了下来,大约只用十几秒,已经下到了刚才估计的高度。我看到下面的火光停了下来,忙双腿一紧,夹住锁链也停住身势。

低头一看,王老板已经到了锁链的尽头,身下几米就是刚才荧光棒撞击的地方,他正伏下身子,用自己的打火机去照,但是因为光线太过微弱,看不到这东西整体的形状,只看到一块黄色的水晶状物体悬挂在半空。

我打亮手电的光圈,在强光手电的照射下,这东西的全貌一下子便显现了出来。

出乎我的意料,青铜锁链下面悬挂的并不是商石棺,甚至不是一只棺材,而是一块棺材形的巨大琥珀状巨石,似乎是天然的,非常的通透,在手电光芒下,反射出犹如黄金一般的琉璃之光,只要稍微转动一下手电的角度,整个空间就呈现出流光异彩、瑰丽非凡的景象。

从顶上垂下来的四根青铜锁链,一直铸入了琥珀的内部,顺着锁链向里面看去,还可以看到琥珀里面有一个人形的黑色影子,非常的模糊,能勉强分辨出头和肩膀,影子的肩膀高高地耸起,好像两个驼峰一样,整个人蜷缩着,好像胎儿在母体内的样子。

我从来没见过这东西,一刹那简直目瞪口呆,说不出话来,王老板却出奇地冷静,只是观察了一下,就滑了下去,试探着想踩到琥珀上面,我赶紧叫停:“不要!”

王老板回头,莫名其妙地看着我。

我对他说道:“我从来没有见过这么大的琥珀,说不定是松香石,你踩上去,可能会碎。”

王老板很轻蔑地一笑,说道:“你懂个屁,什么琥珀,这是尸茧。”说着已经踩了上去,那尸茧倒也真的结实,晃了一晃一点动静也没有。

我一看他没事,不甘落后,双脚一松,也滑到琥珀尸茧上,同时操起短柄的猎刀,就想插回腰上去,免得一手手电、一手匕首的,在这滑不溜秋的琥珀尸茧上,也不好行走。

没想到王老板会错了意思,看我下来,戒备地一猫腰,抽起皮带架在胸口,就准备干架,我给吓了一跳,原本要插回到腰上的短刀也架了起来。

一时间,气氛紧张到了极点,但是谁也没动,因为两个人都知道,在这个地方,稍有闪失,就不是给人踢一脚就能了事的,下面就是万丈深渊,你力气再大,脾气再凶悍,掉下去完蛋也就是一两秒时间。

王老板到底是江湖中人,拿得起放得下,僵持片刻,先是摆了摆手,对我说道:“后生仔,到这份上了,大家退一步,犯不着同归于尽。随便谁死,对谁都没好处,这地方不是一个人能上得去的。”

我看了看头顶,发现他说的没错,在这个地方,要爬上去,至少要两个人,只要还在这下面,他应该不敢动我,不然他可能死得比我还悲惨,但是这人非常的狡猾,不可太过相信。

我先是缓缓地放下了猎刀,做了个和解的手势,将刚才无线电干扰的事情简短地说了一遍,好让双方都有个台阶下,毕竟刚才我也是下了杀心的,他没可能这么容易放下戒备。

王老板拿出自己的对讲机,半信半疑地打开,里面突然炸出一连串高分贝的静电嘈杂声,声音极其刺耳,好像一个人撕破嗓子撕心裂肺大叫一样。王老板听得心惊肉跳,赶紧将对讲机关掉,骂道:“我操。”

我也给吓得半死,这里一定已经非常靠近干扰的源头,声音才会刺耳到如此地步。我真想不到世界上还有这么可怕的声音,再多听几秒,我说不定就要失去心神跳下去了。

王老板将皮带拴回到自己腰上,说道:“这次算老子错,你也知道,我们跑江湖的,不多几个心眼不成。”他指了指自己脸上给我打肿的那一块,“后生仔,你下手也不轻。我们这次扯平,私人恩怨出去再算,怎么样?”

我心里冷笑,他刚才本性已露,我已经断定他必然早就打算出去之后要将我们灭口,现在说这些不过是缓兵之计,不过这个时候,的确还是需要互相利用,于是点头,将手电抛给他,以示平衡。

我们暂时和解,但是我仍旧不敢和他靠得太近,免得突然就给他推下去。他显然也有这样的顾虑,两个人心照不宣,一边戒备着对方,一边小心地蹲下身子,仔细去看脚下的尸茧。

尸茧的表面上有很多自然形成的纹路,里面的透明度不高,要想从外面看到尸体是不太可能的,可能要通过X光扫描,或者把尸茧打破才行。最奇特的还是里面的人形影子,这应该就是裹在里面的尸体,不过,这尸体的形状太怪了,怎么看怎么不像人。

\chapter{大胆假设,小心求证}

尸茧这种东西,早几年在川南和内蒙古都挖出来过,但都是脸盆这么大,有些像玉,有些像琥珀,里面裹有干瘪的小动物或者小孩子的尸体,少有成年人的,这些东西一般都是作为陪葬品出土的,没人知道是怎么做出来的。

根据古籍记载,这东西有可能是先秦的时候方士用来炼丹的药引子,是把不足月的孕妇浸入药液里弄死,装在缸里,埋十七年再挖上来,肚子里的孩子就会变成尸茧。外面这一层东西,是孕妇的胎盘石化后的物质,你看到的琥珀色,其实是里面的羊水凝固而成。也有人说,这是一种尸体防腐技术,用特殊的混合中药的树脂将尸体裹住,让尸体不丧失水分。

我听说过这东西的存在,但是因为这东西价值太大,从来没经手过,如今看到了,也不知道门道怎么看,加上为了缓和一下我和王老板之间的气氛,我就试探着问了他几个问题。

王老板告诉我,早年他的曾祖父在香港做大朝奉的时候,见过一些因为日本战乱跑去移民的有钱人当出的宝物,其中就有琥珀尸茧。

尸茧有大有小,其中的东西也各不一样,有的就如普通的昆虫琥珀,有的里面却裹着人的尸体。

他曾祖父曾经看到过一只尸茧,里面有一个穿红衣裳的小女娃子,十六七岁,闭着眼睛就像睡着了一样,栩栩如生。

他看着这小女孩,觉得可怜,就乘老板不注意,把这东西烧了。那时候兵荒马乱的,老板也没有察觉,结果当然晚上做了个梦,梦见那红衣裳小女娃子来找他,给他磕头说谢谢。

后来王老板自己做古玩生意,也接触过这种东西,但是这么大、里面看不清楚有什么的,他倒还是第一次见。

我觉得有点意外,难不成李琵琶说的“比秦始皇陵还好的好处”就是指这个?不可能啊?虽然说这东西也是十分罕见的,但是绝对称不上“比秦始皇陵还好”这样的档次。

王老板自己也觉得奇怪,但是他相信《河木集》里的信息不会错,就蹲了下去,小心地贴上琥珀的表面,想看清楚里面是不是有什么不得了的明器,给熔在琥珀里了。

这里由青铜链条固定,我和他不能同时走到一端,不然会失去平衡,所以我待在了原地,扶住青铜链,看他有什么收获。

王老板上上下下看了好几眼,仍旧什么都没有发现,只说琥珀的里面似乎有一层液体在流动,影响透明度。里面除了那黑色的影子,再无其他的东西。

再看四周,下面是一个黑漆漆的深渊,没有任何迹象显示可以爬下去,这青铜树的顶部,神秘的棺椁里的东西,就是这么一块琥珀。

我们两个都沉默了下来,一句话也说不出来,这琥珀虽然值钱,但是这么重,靠我们两个人也抬不上去,这里的一切,对于我们来说毫无意义,我就算了,但是王老板一路过来,死了这么多人,当然非常郁闷。

沉默了一会儿,我觉得问题还是在李琵琶的话上,就问王老板,李琵琶在来的路上,或多或少有没有透露过什么?看李琵琶这个人的性格,也不是什么能保守秘密的人,应该会不当心说出点东西来。

王老板的表情变了变,说道:“你看人倒也是挺准的,李琵琶的确不是个嘴巴紧的人,不过奇怪的是,这次过来,他的口风特别地紧。我记得他只是一直对我们说,到这里来,我们要什么都有,叫我们不要担心,其他的什么都没说。他这个人喜欢玩神秘主义,经常这样搪塞我们。”

只要到这里来,想要什么都有。

我重复了一下,心里觉得奇怪,这一句话很怪,似乎有什么内在的意思。

转念一想,我忽然有了一个念头,哎呀了一声,心道:“难道,竟然是这样?”

王老板看我的表情顿时变得古怪,莫名其妙地看着我,不知道我想到了什么。

我兴奋地挠着头,脑子里飞快地转着:李琵琶说的是到这里来,这句话有歧义,也许他们都误解了他的意思,关键的是那个到字,就是说,关键不是你们能拿到什么,而是要先到那个地方去,到了那个地方,你们想要什么就有什么!!

而在齐老爷子给我的资料上,我看到过这样一张照片,上面是洞穴壁画,有一棵青铜树,很多人形状的图案在树下跪拜。很多人认为,那是古人祈求丰收的意思,但是,从照片里拍到的边上一些象形文字来看,他们却是在许愿,上面记录说,古人向这棵青铜树许愿并奉献鲜血,那愿望就会实现。

这看上去是一种迷信,但是我一想到李琵琶说的那句话,又不得不把两件事情连起来。

难道说,这李琵琶来这里的目的,是相信这棵青铜树真的有帮人达成愿望的能力?

我突然想笑,又笑不出来,如果真是这样,这的确是当之无愧的天大的好处。天下任何的利益,都没有这好处的亿万分之一值钱。可是,这是根本不可能的,这人如果真是这个目的,好像也太不可思议了,而且,他自然不能言明,不然谁会跟他来啊。

我将我的想法讲给王老板听,让我出乎意料的是,王老板听了之后,非但没有觉得好笑,而且好像突然想起了什么,说道:“不对,也不是这么说,好像真的有这个可能。”

我啊了一声,心说不会吧,问他怎么可能呢?

他道:“就是刚才,我们两个从上面掉下来的时候,我一落地,怕你偷袭我,马上就往雾气的中心跑去,那个时候,我也看到了这几条青铜链条,但是,我从青铜链条中间穿过的时候,却没掉下去,地下是实的。可是第二次我偷袭你的时候,却一脚踩空了,这下面已经有了个洞,我以为是我在雾气里看走眼了,当时也没有在意,现在想起来,好像这洞是凭空就出来了一样。”

“你是什么意思?”我问道。

他道:“我的意思是,我第一次踩过那块地方的时候,当时我在想,这下面应该有一个棺井,但是我踩的时候却没有,而当第二次我去踩的时候,那个棺井便产生了,这,算不算我的愿望实现了?”

我怀疑地看着他,心说怎么可能会有这种事情,是不是他当时被我打蒙了,糊涂了?

王老板看我不相信,道:“是真的,我一直在奇怪,《河木集》从来没有错误,如果李琵琶说的好处是这个,那他肯定有非常的自信,说不定真的有这个可能。”

我皱着眉头,还是不信,用心理学的话来说,李琵琶那句话的意思就是——只要到了这个地方,你们的潜意识就可以影响周围的环境,使得你们潜意识里的想象变成实在的物体。

如果这样的话,青铜树真的有这样的能力,那我们现在所看到的一切,都有可能是我们自己制造出来的。这青铜树原来不是这样的,这山洞原来也不是这样的,这里的尸体原来也不是这样的。

如果那《河木集》的主人,在当时攀爬,或者拷问厍国先民的时候,已经知道了这棵青铜树拥有神仙一样的“物质化”力量,那李琵琶肯定也是想得到这种力量,才煽动这帮人来这个地方的。

以这个为前提的话,李琵琶的话倒是可以解释了,但是其他就乱套了,那这里现在是一个潜在意识和真实交织的世界,实际上青铜树的原形到底是什么样子的?这里又是如何一个景象呢?

这样的事情,是不是太过于古怪了?有没有可能会发生呢?

我们爬上来的时候,很多东西,比如带着螭蛊面具的猴子,岩壁上的空洞,说不定都是我们自己实体化出来的东西。

这种力量初看上去很好,但是我仔细一想,却觉得莫名的恐怖,人的思想是不受控制的,比如说你拥有这种力量,你去看一部恐怖片,看完之后,说不定会发现恐怖片里的尸体正吊在你身后的吊扇上往下淌血。比如说你走过墓地,说不定……

也许受过心理学训练的人,能够在一定程度上控制这种力量,那岂不是可以控制世界,等等——不对,我忽然想到了什么。

老痒他们挖出的青铜枝桠,应该也是一棵这种许愿树的图腾,他老表偷偷把那青铜枝桠带出来,难不成是知道了这树有这样的力量?但是他怎么会疯了,那现在枝桠在老痒手里,会不会老痒也知道这件事情的内情?

我看着边上的树,突然想到,如果是真的话,那我现在岂不是可以对这个树许一个愿望,让我知道这一切是怎么一回事。随即我就笑了,怎么可能,我竟然还相信了,面前只不过是一块大一点的青铜而已——

想到这里,我忽然感觉一股异样,一连串的思维突然从我的大脑里穿了过去,我心里一个咯噔,猛转过头,盯着王老板看。

\chapter{失控}

来的时候,凉师爷和我们说过,王老板是一个粗人,从小在道上混的,文化水平很低,他唯一可以炫耀的,就是他祖传的那本《劫余录》。这样一个人,我刚才给他解释潜意识的时候,他竟然一下子就明白了,还能举出例子来,这说明他或多或少对心理学有一点了解。

刚才我就感觉到有一些奇怪,但是并没有太过在意,以为这只是凑巧的事情。

也许王老板有着高尚的情操,在坑蒙拐骗的同时,还一直抽出时间自修心理学,想做一个有文化的黑社会成员。但是看他那种暴戾劲,又不太可能。

一想到这些,我不由自主地看向王老板,一种很奇怪的预感笼罩着我,心里感觉到非常的异样——眼前的这个人,会不会不是王老板呢?

他正在考虑我提出的那个想法,想得出神,一时间也没有注意到我正用异样的眼神看着他。我乘机打量着他的表情,他的衣服,还有他身上的很多细节的地方。

一直以来我对王老板都没什么印象,一来他不太说话,二来他的动作也不突出,我在爬上青铜树前,只见过他一两次,此时也没有多少记忆来判断眼前的人的真伪。

但是一看之下,我还是感觉到自己好像发现了一个问题,但是我又不敢肯定。

为了验证我的想法,我突然装出看到了什么的样子,在他面前挥了挥手,轻声叫道:“王老板!”

王老板一下子转过头来,问道:“什么?”

“千万不要动!”我做了个手势,让他不要动,自己小心地一点一点走了过去。

他很紧张地看着我,以为肩膀上沾了什么东西,用眼睛直往边上瞟。我走到他身边,按了按他的胸口,心里哎呀了一声,什么都没做,就退了回来。

他给我弄得莫名其妙,也轻声问:“干什么?出了什么事?”

我此时心里已经有了几分把握,看了他一眼,说道:“我觉得你的衣服很奇怪,你哪里买的?”

王老板用一种看到神经病人的表情看着我,失笑道:“有没有搞错啊,突然问我这个问题。”

我说道:“一点也没有搞错,王老板,几个月前,我第一次去倒斗,我的叔叔让我去采购东西,那个时候我也想买你身上这个牌子的登山服,但是我后来没买,你知道为什么吗?因为这种衣服胸口的两只口袋,看上去很大,其实是假的,是用来做装饰的,我当时觉得探险用的衣服,当然是口袋越多越好,所以就买了另一个款式。”

王老板摸了摸那两只口袋,表情变了一下。

我拍了拍手,轻声说道:“所以我感觉有点奇怪,你刚才那根荧光棒,还有你的香烟,到底是从哪里掏出来的,嗯,王老板?”一道闪光在我的头脑闪过,我几乎脱口而出,“或者——还是叫你老痒比较好?”

王老板呆呆地看着我,隔了好久,才扑哧一声笑了出来,忽然间,肥胖的身体开始收缩,就好像一只泄了气的气球一样,一下子瘪了下去。

我看着王老板的脸一点一点地变化,慢慢的,变成了老痒的脸孔,就知道自己猜对了。

他最后舒展了一下身子,叹了口气,说道:“吴邪不愧是吴邪,他娘的从小就只有你骗我的份,我难得想骗你一次,还是给你拆穿了。”

我冷冷地看着他,问道:“少废话,你在玩什么花样?”

他苦笑了一下,摆了摆手,“听我解释,听我解释,哎呀!我就知道嘛,这事情没这么容易蒙混过去。”

看我不说话,他才说道:“我的目的不是骗你,但是这件事情一定要这么做才有用,等一下你听我解释完了,你就知道,我这样做是有苦衷的。”

我看到他自如地控制自己的外表,已经意识到他对这种能力的运用超出了我的想象,那他必然对所有的事情都有所了解了,那到这个地方来的目的,就肯定不是钱了。因为有了这种能力,钱根本就不是问题。

但是有着这种能力,几乎可说是无敌的,他还有什么目的达不到的,非要来这种鬼地方?难道这种能力,有什么不足的地方?

不管怎么样,我现在已经肯定,从他来找我的那一刻起,我就掉进了一个处心积虑的圈套里,也就是说他一开始就在撒谎,亏我还这么相信他,这该死的龟儿子,要是我能控制这种力量,我就把他变成一只猪。

老痒看到我的表情变化,知道我虽然表面上冷静,但是心里已经火到了极点,一时间也不知道如何来平息我的怒火,不知所措地看着我。

呆了半晌,他突然叹了口气,好像想通了什么一样,从口袋里面掏出一张照片,说道:“你看看这个,我再解释给你听。”

我接过来用手电一照,照片上是他的妈妈,头发已经斑白了,可能是太过操劳的原因。看来老痒坐牢的那几年,她受的打击很大。她妈妈年轻时很漂亮,对我们都很好,我们都叫她漂亮阿姨。我老爸和我每年都会去看她几次。

我不知道他把这照片拿出来干什么,对他道:“你什么意思?”

他叹了口气,黯然地一笑:“我不是说我需要钱吗?其实我是骗你的,我来这里的目的,是为了我妈,我妈在我坐牢的时候,已经走了。”

我啊了一声,用一种极度怀疑的眼神看着他,皱起了眉头,问道:“你妈……去世了?”

他默默地点了点头,看了看自己的手,说道:“我出狱的第二天,急不可待地回到家里,想让我妈有一个惊喜,可是等我推开房门的时候,却闻到了一股恶臭,我妈趴在缝纫机上,一动不动。我以为我妈犯心脏病了,马上去扶她,等我把她扶起来的时候,你知道他妈的我看到了什么吗?!”

老痒闭上眼睛,痛苦地呻吟起来:“她的脸,已经粘在了缝纫机上,一拉就全部撕了下来,我的天——”

我不知道他妈已经去世了,一下子也不知道该怎么反应好,呆在那里看着他。老痒这个人非常孝顺,他绝对不会用他妈妈来开这种玩笑。

他摸了摸额头,又说道:“我把我妈收殓了之后,一个人待在空房子里,一下子不知道怎么办好,我也不敢睡觉,一躺下,就看到我妈粘在缝纫机上的脸。就这样一直待了九天,我肚子饿得要命,心想要不就饿死算了,可是这个时候,突然,我就闻到了香味从厨房里飘出来,好像有人在炒菜。我过去一看,看到我妈竟然又出现了,看到我过来,还说:等一下,马上就好了。”

我听到这里,已经意识到这是怎么回事了。

老痒继续说道:“我一开始还以为我想我妈想得疯了,出现幻觉了。后来,我逐渐发觉了不对劲,这不是幻觉,不仅是我,连卖菜的都看到了我妈。我才知道我妈真的回来了,她真的和以前一模一样,连烧出的菜的味道都一样。

“如果是别人,可能会以为见鬼了,但是我没有,我开始思考这是怎么一回事。逐渐地,我开始发觉,我四周的环境有一种说不出的不对劲,但是还没有找到关键,直到有一次,我看电视看了一个通宵,结果你猜怎么的,那天晚上竟然是断电,整个小区只有我家照样有电,所有的电器,没电照样开,连插头都不用插。

“我不知道这是怎么一回事,这个时候,我的老表给我写了一封信,信里他告诉我,他也出现了这样的情况,当时我一下就明白了,这和那棵青铜树有关系。

“我看了很多的书,知道了那棵树,可能就是古人说的许愿蛇神树,我这种能力,可能就是从那青铜树上来的。一开始我很开心,以为自己发财了,可等我研究了这种能力,并且开始逐渐可以控制的时候,出了问题。

“你一旦用你的思维去控制这种能力,如果你无法屏除杂念,很多东西就会混合起来,变得非常糟糕。所以,有一天,我起来的时候,看见我妈妈背对着我在做缝纫,我一看到她坐在缝纫机上,我吓坏了,蹑手蹑脚地走过去,你知道我看到了什么,我的天,我妈她的脸……”

老痒做了好几个动作,但是实在说不下去了,在那里长叹了好几声。

我听得心里感觉到一股寒意,实在无法想象那时的情景有多可怕。

老痒凭空就从手里变出了一支香烟,放进嘴巴里,没用打火机,烟就着了,他猛吸了一口,接着说道:“自那个时候开始,我意识到了这种力量的恐怖,但是我不甘心,我很想我妈回来,所以我必须找一个人过来,找一个认识我妈、又有很干净的潜意识的人,就是你,老吴。同时,我还得把我自己的能力消除掉。”

我没有想到老痒的目的竟然是这个,说道:“但是,老痒,这事情听起来,好像是在逆天而行的感觉,人死是不能复生的。”

他说道:“老吴,我也不是很贪心,我只要三年,只要和我妈再相处三年我就满足了,你到我家里来的时候也不少,你也不舍得我妈就这样孤零零地死去吧?”

我叹了口气,想着如果他妈真的复活了,我还敢不敢到他家里去,这棵青桐树不知道到底是谁立在这里的,竟然有这么妖邪的力量,用那种力量物化出来的人,到底算不算是人呢。

想了半天,我还是摇了摇头:“这事我做不到,老痒,你妈妈已经死了,她已经归土了,你就……你就让她去吧,不要拽着她不放了。”

老痒笑了笑:“已经晚了,老吴,你不明白,这件事情和你想不想帮我是没关系的,这也是我为什么不能告诉你我的目的的原因,现在,我想我的目的已经达成了。”

我没听懂他在说什么,问道:“什么意思?”

他举了举自己的手,说:“你先实验一下,你能不能物化出什么东西来。”

我不知道他想干什么,看了看自己的手,心里想着石头的形象,试图也将我的意念实体化,但是使劲了半天,手上还是空空如也。毫无疑问,这种能力很难使用,普通人是无法控制自己的潜意识的。

老痒有点得意地对我说道:“你看,这种力量,你有意而为之的时候,肯定是没有用处的。不然我刚才肚子饿的时候,应该会有烤鸭自己飞过来。只有在特定的情况下,它才会出现,这非常难,老吴,只能引导,无法使用,就算受过训练,也非常困难,你想要在这里变台电视机出来,这么复杂的东西,是无论如何也变不出来的。”

我看着他,“你是说,这种能力是被动的?需要一个心理引导?”

他点点头,“对,比如我刚才和你说的那些话,已经可以在你大脑里引导你的思维,而使得在几百里外的我的家里,物化出一个人。”

我一下呆住了,看着他,说道:“胡扯,你他妈的以为我真的什么都信啊?”

老痒摇摇头,就在这个时候,突然青铜树连带着整个琥珀震动了一下,我们两个脚下一滑,差点都摔下去,赶紧抓住边上的青铜链条,低头一看,只见我们身下的深渊里,好像有什么东西在蠕动一样,每蠕动一次,青铜树就震动一下,一下子地动山摇,连站都站不稳。

我拉住青铜链条,一边觉得奇怪,一边想起一件事情,回头问老痒:“对了,刚才那‘的……的……的’的怪声音,是不是也是你弄出来的?”

老痒也疑惑地看了看下面,点头说道:“是啊,我用这个声音,把你引到根盘里面去,然后我把守在外面的那王老板打晕了。那个无线电干扰,只不过是不想让你听到王老板和我打斗的声音。”

我皱起眉头,叫道:“那这个震动是怎么回事!”

老痒脸色也变了,说道:“我也不知道,不过,老吴,对这棵青铜树,你的第一印象是什么?”

我一听他这么说,突然打了个哆嗦,“我想……它是通到地狱里去的……”说着看着下面,“不会吧,你该不是说,下面的东西,是……”

老痒猛踢了我一脚,大叫:“白痴,不要乱想!”

话音刚落,一只巨大的眼睛,出现在了下面的黑暗深处,紫色的瞳孔,像猫一样变成了一条诡异的窄线。

\chapter{坍塌}

下面的巨眼迅速地逼近,情况混乱,加上整棵青铜树都震得厉害,我也看不清楚它是靠什么来攀爬的,只知道按这样的速度,不出十分钟我们就要打遭遇战了。

老痒看得脸都绿了,直埋怨我:“你脑子里装的到底是些什么东西?”

我大叫冤枉:“老子对天发誓,我也是第一次见这东西,要是有半句假话天打雷劈。”

他看我说的这么决绝,愣了愣,“不可能,不是你是谁?”

此时也无法估计这么多了,我对他说别废话了,快想个办法,给这么瞪着也难受。

他说道:“也不用太担心,就是一只眼睛而已,难不成它用眼皮夹死我们?等一下它上来,老子一脚把它给踢瞎了。”

话音未落,突然有一只章鱼一样巨大的触手卷了上来,一下打到琥珀上,我们像空中飞人一样荡了一圈,撞到青铜壁上,琥珀撞了个粉碎,里面的尸体直接给分了尸,随着琥珀的碎片天女散花一样地掉了下去。

我们两个在最后关头死死抓住青铜锁链,才幸免保得不失,但是也给转得头昏脑涨,我对老痒叫道:“这下子玩笑开大了,你不是能变吗?快变门大炮出来,把这玩意儿给轰了。”

老痒大骂:“你他娘的胡说什么!有那么容易吗?快跑!”

我们二话不说就顺着青铜锁链往上爬,才爬了几步,突然手上一滑,开始使不上力气。我想起树根上面的那种滑腻的植物,心中恐惧,这下完蛋了,难道要死在这里?

这时候老痒将手一抬,我突然就感觉那种滑腻的感觉消失了,他像猴子一样几下便爬了上去,将我拉了过来,我一下子没抓稳差点脱手。埋怨道:“有这本事,直接变只梯子多好?”

他骂道:“拜托你不要这么多意见!”

我们两个咬着牙爬进棺室,上面的雾气已经消散去,我想乘着这个机会看一下其他几幅浮雕。老痒说你别看了,这都什么时候了,拉着我就往椁壁上爬,突然那只触手闪电一般从棺井中卷了上来,一下子把椁室的巨大石头盖子顶得飞上了天。这一下力量极其的霸道,连铁条一样的树根都给撞得粉碎,一时间整棵青铜树狂震,满眼是树根的根须、腐朽的树皮和灰尘。大片的树根短枝因为突然破裂,像子弹一样飞了出去,打在栈道上,扫塌了一大片。我们两个正趴在一根滑溜溜的树根上,这一下直接把我们甩出了椁室,摔倒在祭祀台上。

那只触手冲出青铜树后就不想进去了,四处乱卷,连打了两下,将四周的几座青铜雕像拍得都变了形。我和老痒狼狈地低头连躲了几下,老痒指了指栈道说快下去,在上面死定了。我想起给老痒在外面打晕的王老板,心说虽然是个王八蛋,但是这人也不是十恶不赦,也不能放着不管,忙转头去找,然而一眼却看不到,难不成刚才给那些炸开的树根带下去了?

四周的树根已经给连根拔了,只剩下衍生到祭祀台下面的那些。老痒看我在那里左顾右盼,踢了我一脚,让我看天,我抬头一看,给撞到天上去的巨大石板正打着转儿地摔下来,赶紧逃命,老痒一个打滚背起挂在一根残枝上的背包,两个人鱼跃跳上了那根用来做绳桥的登山绳。

我们刚抓住绳子,后面的石板就重重摔在了祭祀台上,给摔了个粉碎,发出震耳欲聋的声音,我们抓着的绳子也给牵连着好像钢琴的琴弦一样颤抖,几乎不堪重负。

回头一看,刚才我们登山镐钩住的树根,上端已经随着包裹着棺椁的榕树根盘给扯飞了,现在只剩下可怜的一点点,给我们的体重拉着,登山镐直往外脱,好像坚持不了多久了。

我越来越觉得不妙,回头让老痒快爬,说要不然咱们就要步老泰的后尘了!老痒一听猛打了我一个巴掌,打得我耳朵嗡一声。

我大骂:“我操,他妈的打上瘾了你?”

老痒大叫:“不打你行吗,管住脑子,千万别乱想啊——”

我大叫:“我乱想什么了?”

话还没说完,“嘣”的一声巨响,我们回头一看,整只椁室突然鼓了起来,裂开了好几条缝,一条黑色的巨蛇探出头来,那条触手就是蛇的尾巴,但是这条独眼巨蛇,鳞片非常细小,看上去更像一条巨大的虫子。

独眼巨蛇爬出来之后,巨大的眼睛马上转向我们,老痒一看不妙,猛地从我腰上拔出长柄猎刀,用力一挥,将登山绳砍断,我们人猿泰山一样划过一道摆线,撞上一边的栈道,这一次我有了经验,就地一滚,缓冲了很多撞击。

老痒落地之后,抽出背包边上跨着的短步枪,对着那巨蛇的眼睛就是一枪。子弹打进去一个大洞,那巨蛇疼得猛地蜷成一团,尾巴一扫,将我们头上那一排栈道全部扫飞。

老痒避过砸下来的木头碎片,站起来对着那蛇,一边开枪,一边拉着我往下跑,我知道这种枪能装五发子弹,但是老痒拿在手里,子弹如流水一样打了出去,根本不需要装弹。

可惜这枪的口径还是太小,这蛇刚才中了一弹,现在学乖了,缠绕起来,用身体护住自己的眼睛,子弹全部打在它的尾巴上,鳞片犹如铁甲一般,毫无用处。

我一看枪对它没用,就招呼老痒快跑,一路跑到了栈道的断口,我刚想爬上悬壁,老痒一把拉住我,说:“什么时候了,还爬?”说着拉着我往下一跃,我们从断口直接落到了下一层的栈道,就听底下的木板喀嚓一声,哪里经得起这样的撞击,立即裂成几十块,我们透板而下,又撞破一层,摔在栈道地上的平台上。

这一次摔得十分严重,我起来的时候,嘴里鼻子里全是鲜血,老痒一把拉起我,说到:“好像估计得太乐观了,你没事吧?”

我只觉得天旋地转,也不知道回答了他些什么,黑色巨蛇已经闪电一般顺着青铜树爬了下来。老痒说道:“打是打不过,逃也逃不掉了,我们到下面找个岩洞躲一下。”

我往下一看,再往下走已经没有栈道,只剩下我们刚才休息过的那种小岩洞,密密麻麻的有很多。那蛇体积很大,我们随便找一个进去,应该可以暂时避一下,再想对策。

当下被老痒拉着就往下爬去,就着最近一个直径一米都不到的岩洞爬了进去,还没爬到底,突然巨蛇的眼睛就出现在了洞口,朝我们看了看,然后猛地一冲,试图想钻进来。

老痒打了好几枪,想将它逼退,但是子弹打在蛇头上,只崩飞了几片鳞片,一点效果也没有。

黑蛇的巨头有解放卡车那么大,钻了几次钻不进来,突然甩脑袋往洞口一撞,一时间乱石纷飞,我们赶紧往后退去,免得给塌下来的石头压住。

黑蛇见我们退到洞的内部,大为恼怒,又是一撞,整个岩洞一阵震动,只听到岩石开裂的声音,从洞口一直传到我们头顶上。

这里的玄武岩,因为里面的地下河道过度地开挖,已经十分不稳固,给这么一撞,岩石内部的细微平衡被破坏,里面缝隙发生连锁反应,一条裂缝突然出现在我们头顶上。老痒一看不好,拉着我就往洞的底部退,我惊魂未定,才往里爬了几步,就听到一连串轰鸣,一时间沙尘满目,碎石四溅,不知道哪里塌了。

出于本能,我反射性地蜷成一团,护住脑子,石头下雨一样从上面掉下来,身上和背上连中了十几下,慌乱间,老痒一把拉住我,将我拖到他的那一边,同时一声巨响,一块写字台一样的石头塌了下来,将洞口完全塞住了。

这下子黑蛇不但进不来,连我们也看不到了,然而它似乎并不死心,又连着撞了十几下,石头不停地塌下来,四周的岩壁也开始出现裂缝。

老痒说:“这样下去不是办法,这家伙不弄死我们恐怕不会罢休,再撞几下,山都要塌了。”

我转头一看,我们已经退到洞的最里面,退无可退,再塌进来一点,大罗神仙也救不了我们了。

此时已然到了绝境,就算有炸药,在这么小的空间也不能使用,看着四周的裂缝一点一点地延伸开去,我心急如焚。

就在这时候,忽然一条裂缝碎了开来,一段岩壁不堪重负,整个塌了下去,我们往边上一贴,勉强留得全身,却看见岩壁塌了以后,后面竟然出现了一个岩洞。

我心中大喜,心说天不忘我,肯定是两个岩洞之间的岩石碎裂,使得中间出现了一条石道,忙转头招呼老痒,就要往里爬。

老痒却一下子拦在我的面前,说道:“不能进去!”

\chapter{日记}

岩洞坍塌在即,大石头小石头不管三七二十一就往我脑袋上砸,再多待一秒都有葬身乱石之下的危险。这种情况下,眼前有路已经不错,还怎么能管其他,我一把将他拉住,一边对他大叫:“什么不能进去,不进去难道在外面等死?”

老痒说道:“里面情况未明,你先看看再说!”

我对他说道:“管不了这么多了,你看这种情况,里面是龙潭虎穴也得闯了。”说着拉着他就往洞里猫去。

老痒硬扯住自己的手,不让我拉他进去,说道:“拜托你也听我一次,这洞真不能进去!”

说着还要将我往外拉,我大怒,刚想问他是想寻死还是怎的,忽然一块石头猛地塌了下来,我赶紧松手,两个人都往后一跌。石头“轰隆”一声横在了我们中间,塌出的洞口一下子被堵住了。

我吓得够戗,忙大叫着问他有没有事,过了好久,才听到他呻吟一声,回道:“没事,他娘的头上给砸了一下,这里已经不塌了,你怎么样?”

我告诉他我也没事,随手推了推石头,纹丝不动,知道来路已断,于是观察四周,本来我以为这是岩壁上的另一个岩洞,一边必然有一个出口,然而现在一看,却是一个封闭的空间,非常狭窄,似乎是一处自然的山体缝隙,看情形总觉得眼熟。

垫着碎石头爬了几步,我忽然醒悟,这里原来也是一处坍塌后的洞穴,不过这里的坍塌有些年头,该塌的都已经塌了,地上全是碎石。

我刚才还在奇怪,为何这巨蛇如此有力,几次撞击就把坚硬的岩石撞成这样,现在想来,原来这里早已有过一次坍塌,那上一次事故必然对周围的岩层损害很大,表面看上去坚固的岩石,其实里面早已经开裂,给巨蛇一撞,终于爆裂,塌出了这一条通道。

我看了看头顶,发现这里是两块坍下的巨石中间的缝隙,看契合的程度应该十分坚固,纵使外面还在不断撞击,这里也只有灰尘洒落下来。

那巨蛇看来力气也用得差不多了,撞得一下比一下轻,最后终于安静下来。

我惊魂未定,想起老痒刚才扯着我,要不是我放手得及时,现在已经成肉饼了,气不打一处来,在石头后面怒道:“你刚才他娘的吃错了什么药了?差点给你害死。”

老痒被石头堵在外面,想进也进不来,也说道:“什么我吃错药了,你怎么不说自己别扭,你看现在可好,怎么办?”

我扒了几块石头,看到老痒的手电光从石头的缝隙里透进来,然而最大的那块石头最起码有一张八仙桌那么大,之间的缝隙有限,我能把手伸出去,但是人决计钻不出去。

我拿石头敲了几下,砸出几个白茬子,两种石头硬度相同,砸起来很费劲。老痒见我砸得上头的碎石头又开始松动,忙让我别弄了,说:“你悠着点,再敲这里又得塌了。”

我说道:“伸头一刀缩头也是一刀,反正不是压死就是饿死,少顾虑这么多。”

老痒说道:“你还是别,咱们没到山穷水尽的时候,你先四处看看,有没有什么特别的东西,发现马上就叫我。”

我环视一周,这里黑咕隆咚,能看见的只有碎石,就对他说里面什么都没有。

他听了沉默了一下,问道:“真的什么都没有?你再仔细看看。”

我说道:“骗你干什么,这就屁股大点地方,有什么肯定看见了。”

老痒说道:“那好,你再看仔细点,我也先到前面去看看,是不是堵得这么结实,说不定还有缝隙能爬出去。”

说着他的手电光就移开了,我靠在石头上休息了一下,爬进缝隙里面,四处一看,就知道这里不会有出口,架在头上的石头又重达数吨,困在这里,恐怕一年半载是出不去了。

再往里面走了走,就没路了,正想返头,忽然看到石壁上好像画了点什么东西,赶紧凑过去看。

第一眼看时,我以为那是一些涂鸦一样的洞穴壁画,非常原始,可能是铸造青铜树的先民留下的。再仔细一看,却发现不是,这些涂鸦上的图案是一架飞机和几个英文字母,这是现代人的作品。

什么人会在这种地方搞这些东西?我感到十分疑惑。

涂鸦的一半压在我脚下的碎石头堆里,我搬开那些石头,想看看到底画了些什么,移开一块大石头后,出现了一团黑乎乎的破布,好像是一件衣服的碎片。

我扯开这团破布,一只干瘪并已经腐烂得露出骨头的人手赫然露了出来。手呈爪状,似乎想从这些碎石中爬出来,而终于力竭而死。

我吓了一跳,几乎要叫出来,心说这里怎么会埋着一个死人?该不会是这洞坍塌的时候,给活埋在这里的?那这人又是谁呢?

我继续搬开那些石头,很快,一具尸体便呈现了出来。尸体已经完全腐烂,看来埋在这里也有些年头了,身上的衣服破成一团一团的,看质地也不知道原来是什么颜色,不过从他脖子上挂的护身符来看,这人可能和我们一样,也是来倒斗的。

想起在瀑布水底看到的那一具尸体,也腐烂得和他差不多,那这两个人也许是一伙的,真是人为财死,鸟为食亡,这两人也许就是我的下场。

我继续挖掘,把整具尸体挖了出来,又找到一只背包,烂得不能再烂了,里面几乎空了,只有一些黑色的残渣,不知道是什么东西腐烂成的,又翻了翻背面,从夹层里面掉出来一本笔记本。

笔记本也快散架了,好在纸质好,上面用蓝色圆珠笔写的字还清楚。我捡起来看了看,前面记的是一些地理位置和电话号码,我翻到后面,忽然愣了一下,这里有一些日记,看第一篇的时间,好像是三年前开始记录的。

这个人字体比较幼稚,应该不是很擅长写字,每一篇日记只有百来字,我快速翻了几页,直看得背脊发凉。

从日记上的记载来看,这人应该是三年前来到这里的。日记上没有写他来的过程,而是从他困在这个岩洞起开始记录的,不过在后面的内容中,偶尔提到了一下他进来之前的经历。

他们一伙人应该总共有十八个,因为在其中一篇里面提到:十八个人只剩下我一个了。里面还提到,他们并不是由我们的路线进入的,而是自山顶的榕树林子中,一个给气生根裹住的巨大的树洞里面进来的。

这应该是老痒提过的那一片榕树林子,我们没有机会进去,没想到里面竟然还有这么大的蹊跷,早知道如此,就不用费那么多周折了。

但是看下去,又不由庆幸没有走那一条路,因为里面记着,他们下来的路极度凶险,十八人进去,从底下出来的时候只剩下了六个,其他全部死在路上了。

估计那一个树洞应该开在林子中间、老痒说的那几棵十几个人环抱不住的榕树老祖宗的一棵上,但是榕树独木成林,那一片林子到底是几棵还是一棵,现在也说不清楚。这些人下来之后,应该和我们正好相反,我们是从青铜树底向上直接爬了上去,而他们应该是直接落到了青铜树顶上。

出乎我意料的是,他还说道,他们在祭祀台上没有发现什么后,顺着四周的栈道而下,栈道的底部,却全是水,有如一个极深的水潭,水是碧绿的,根本看不到底。

他们跳入水潭中,发现深度极深,没有设备无法潜入下去,他们带的潜水设备太小,尝试了一下后,只好放弃,六个人浮上水面,一看,却傻了眼。

原来在他们潜水那一当口,水位极度下降,等他们出来,他们放着装备的栈道竟然离开他们六七米远。他们没想到这一茬,绳子全在包里,没带在身上,一下子全慌了。

水位迅速下降,他们有一批人爬到了青铜树上,有一批人跑进了岩壁上露出的洞里。这一本日记的主人,就在那个时候进入了我所在的岩洞,但是不巧的是,他还没进入岩洞多久,从水里突然盘出一条黑龙一样的巨蟒,顺着青铜树直追上去,他只听到同伴的惨号声和枪声,吓得躲在洞里不敢出去。

这次灾难猝不及防,他的同伴全是亡命之徒,其中一个在和巨蟒搏斗中,临死前启动了炸药,他们预备着开山炸墓,所以炸药分量很多,一下子炸得天崩地裂,连他藏身的洞穴也给冲击波轰塌了。

日记的主人给炸得暂时晕了过去,醒过来的时候发现自己已经给困住了,他料想如此剧烈的爆炸,外面的人肯定无人生还,自己来盗墓的本来就无目标性,指望有人救援也不可能,一时间心灰意冷。

接下来的内容就开始有点无聊起来。

他在缝隙里困了七天,身上带的食物不多,一下子就吃完了,他又渴又饿,电池又电能耗尽,在一片黑暗中,他知道自己大限将到,想起自己的老娘无人照顾,不由痛不欲生。

后来几天,他因为饥饿,神志恍惚,一天他醒了过来,也不知道现在是什么时候,只觉得口渴到了极限,恍惚间,他拿起早就干涸的水壶猛灌了几口,这个时候奇迹发生了,水壶里面突然涌出了甘甜的清水。他也不知道怎么回事,贪婪地连喝了十几分钟,水却丝毫不见少。

他以为自己是在做梦,心说自己肯定是快死了,出现幻觉了,那索性就这样死好了,又想到既然是做梦的话,包里也许还有吃的,一掏,果然原来放食物的那些袋子全满了,他大喜,拼命地吃着,结果吃得几乎噎死。

逐渐地,他发现这一切不是梦,刚开始他以为上帝显灵了,来搭救他了,后来越来越觉得不对,终于,他发现了,这一切的产生和他的思想有一定的联系,但又不是万试万灵,比如说,他一心想吃一样东西的时候,那东西却不会出现,但是他随手去摸包里的吃的时,却往往会摸到自己喜欢吃的东西,虽然包里什么都没有。

他开始有意识地去分析,做思维的实验,逐渐地,他发现了自己的物质化能力。这一段他写了很多,实验的过程非常复杂,最后他并没有得出物质化能力的结论,而是认为,自己成了“恍惚的上帝”。

石头上的那些涂鸦,就是在这段时间里画上去的,恐怕是他穷极无聊的时候画着玩的。

日记的最后,他写道他要用这种能力尝试着从这里出去,如果成功了,他就可以出去做一个超人,如果失败了,他就会死在这里,我不知道他最后做了一个什么实验,反正现在看来最后是失败了。

不过一个有这样能力的人来到现实社会,也不知道是一件好事还是坏事。

看到这具尸体,想到我自己的处境,我不由感觉心寒起来,我身边根本没有食物,恐怕连七天都撑不到,再说就算有食物,无休止地在这里困下去,还不如死了痛快。

我放下日记,又翻找尸体身上的口袋,找出一只手机,早已经没电了,我扔到一边,又翻出一只钱包,里面有一些钱,心说什么都烂,就是人民币不会烂,这叫什么事儿。

钱包里还有这人的身份证,我扯出来,想看看这倒霉鬼叫什么,打着手电一看,只见人的照片已经模糊掉了,名字倒还是清楚,叫做“解子扬”。

这个姓还真少见,死在海底墓中的解连环也是这个姓,我看了看这人的生日,还颇年轻,只叫可惜。

忽然间,后面手电光一闪,老痒已经爬了回来,在石头后面问我道:“老吴!你在看什么?”

\chapter{真像}

我正在看尸体的身份证件,老痒突然问了我一句,吓了我一跳,当下含糊地应了他一声,继续看手里的东西。

从这简短的日记来看,这人是三年前到这里来的,老痒他们第一次进这里也是三年前,这人会不会就是和老痒一伙的?我想了想,又觉得不对,他日记写的和老痒说的虽然有一点吻合,但是大部分还是不同,应该是两批人。

只是不知道为什么,总觉得“解子扬”这个名字很熟悉,解这个姓比较少见,同名的应该很少,哪里听过呢?

我仔细地回忆,但是最近奇怪的事情发生得太多了,脑子不太好使,想来想去也想不清楚。

继续翻他的东西,就没什么发现了,我将他的日记本收起来,以便等一下仔细看看。

老痒看我蹲在那里不说话,以为我出了什么事,又叫了我一声,我回头一看,他的半张脸正往缝里挤,眼睛直往我手里瞟,但是石头和我的位置有一个死角,他看不见我,我能看得见他,只觉得他样子古怪,好像恨不得钻进来一样。

我暗骂了一声,心说你小子刚才死也不进来,现在后悔了吧?对他说:“别吵吵,我找到有趣的东西,正在看。”

老痒皱了皱眉头,忙问:“找到什么了?”

我把刚才发现尸体的经过和他说了一遍,叹了口气对他说:“这家伙可能就是我们的下场,要找不到路,我们恐怕比他死得还快,不过我觉得这个人的名字有些耳熟啊,你记不记得我们小时候有没有什么同学叫这个名字的?”

说着我退到那块巨石边上,想把身份证从缝隙里传出去给他看看。可是我抬头一看,却突然看到老痒的脸上一点血色也没有,惨白惨白,正直勾勾地盯着我的脸看。

我心里陡然出现了一种异样的感觉,心说怎么了?怎么一下子变成这样的表情,难不成我们小时候还真有个同学叫解子扬?

又闭上眼睛想了想,实在想不起来了,现在人情淡薄,大学的同学有些都已经不认识了,小时候的更是没有记忆。我看老痒不说话,又低头看了看手里的身份证号码,说道:“我是真的想不起来,不过这人年纪和我们差——”

刚说到这里,突然一道闪电掠过我的大脑,一下子我整个人愣在那里。

解子扬,解子扬,解子扬,解子扬!

不过啊,这名字好像不是什么陌生的名字——这是老痒的本名啊!

我的头皮猛地一炸,几乎打了个寒战,忙仔细地去看身份证上的生日,一看不由得一阵晕眩,我的天,真的是老痒的生日,可这……这是不可能啊。这张身份证,难道竟然是老痒的!

那难道,这具已经腐烂成骨头的尸体,是老痒……

可是这不对啊,如果老痒三年前就死在这里了,那,在石头外面看着我的,是谁?

我的脖子都硬了,几乎是机械地转过头去,看着石头缝隙里透出的那半张脸,忽然感觉到一股莫名的恐惧。老痒的脸在手电光的闪烁下,显得鬼气森森,看上去竟然和外面看到的那条黑色巨蛇有几分相似了。

我不由自主地向洞的内部退去,不敢再靠近那块石头,老痒却一动不动,还是直勾勾地看着我,我也不说话,好像一座石刻的雕像一样。

以他的脾气,看到我这个样子,肯定将我骂得像孙子一样,如今这个样子,难道真的是因为身份败露,不知道如何反应?

此时我心里越发怀疑,外面的这个人,虽然长相脾气和老痒一样,可能却不是老痒,我从杭州来到这里,之间的经过犹如放电影一样在我脑海中闪过,那一个个谎言,闪烁其词,他在青铜树顶和我说的话,都历历在目,那在其中一点点积累起来的怀疑,也在这个时候逐渐清晰起来。

我一向认为,老痒的城府不可能会有这么深,一来我和他的关系,他根本不需要骗我,二来,他说那些谎言的时候,无不真切到了极点,如果不是我这个人过于谨慎,根本发现不了。可是,看其他方面,这个人和老痒太像了,我找不出一丝的破绽,虽然我心里已经百般怀疑,还是只认为他的性格改变了,没有想到他根本不是老痒。

这个时候,“老痒”终于开口说话了,他的脸缩回到后面,对我说道:“老吴,我刚才不让你进去,你就是不听,只能怪你自己太固执,你没听别人说过,有些事情,知道了并不一定是好事。”

我心里咯噔了一声,心说果然有问题,一边努力让自己的声音不要发抖,说道:“你不是老痒……你到底是谁?”

老痒很古怪地笑了几声,“我是谁?我就是老痒,解子扬,从小和你一起长大、坐了三年牢的解子扬啊,你要不信,可以去查我的案底啊!”

我冷笑一声,“胡说,老痒的尸体就在我边上,他死了已经有三年了,他根本没出去坐牢,你他娘的到底是谁?”

“老痒”的半张脸又无声息地出现在了岩石间的缝隙里,森然一笑,“不错,他是死了三年了,但是我活着,有什么区别吗?”

我看着他的表情,突然感觉到了什么,皱起眉头一想,突然张大了嘴巴,结巴道:“我操,你不是人!你……难道是他物质化出来的——”

“老痒”冷冷地哼了一声,说道:“你怎么不说他是我物质化出来的呢?谁知道呢?我和他一模一样,谁知道是哪个先哪个后?”

我几乎失控,捡起一块石头就朝他扔去,他的脸往后一闪,又说道:“老吴,其实我和他是一模一样的,你不用介意。”我大叫道:“当然有区别,谁知道用那种力量物质化出来的,他娘的是什么东西!”

“老痒”突然沉默了,脸色变得很难看,盯了我一会,突然狰狞地说道:“放你妈的狗屁,老子就是老痒,你和他是一路货色,那就怪不得我了。”

我心里顿感不妙,忽然一支枪管就从缝隙里伸了进来,我赶紧翻身到死角里,“老痒”一枪打在石头上,削掉了一大片,接着枪头马上就瞄向我在的那个死角,又是一枪,子弹几乎是贴着我的脖子飞了过去。

这个缝隙空间实在太小,就算有死角也无法保护我所有的身体,我一看情况不对,忙一下子关掉自己的手电,让他看不到我。他慌乱间开了几枪,都没有打到我,我翻身冲到岩石边上,拿起石头就去砸伸进来的枪管子,几下,便给我砸得变成了九十度。

“老痒”拔又拔不出去,气得大骂,我冷笑道:“什么一模一样,我不认为老痒会朝我开枪,你他娘的就是个劣质的仿冒品!”我自“老痒”和我提起物质化活人之后,心里就一直有一个疙瘩,总有一股感觉,这棵古老的青铜树在这里,不会没什么目的,这种几乎恐怖的能力所带来的生物,会是正常的人吗?真的和我们一样吗?会不会是某种妖怪呢?

现在看来这个“人”虽然不知道是不是和我们一样,但是他显然知道自己是被物质化出来的,不知道为什么,我总觉得事情大大的不妙起来。

“老痒”和我对骂了一会儿,突然好像想到了什么,就不说话了,接着,他将手电关了,一下子整个空间一暗,无尽的黑暗压来,在这一点光源都没有的狭小空间里,显得格外沉重。

我提防着他有什么诡计,缩到死角里躲好,就听他道:“老吴,我记得你小时候最怕黑了,现在怕不怕?不过你可千万别乱想哦,记得我刚才和你说的话,在这个地方胡思乱想的话,小心你的灯一开,你面前出现一张死人的脸哦。”

我心里直骂该死,这家伙是想我因为对黑暗的恐惧,而自己实化出什么怪物。

我心里告诉自己绝对不能让他得逞,但是内心反而害怕起来,他刚才说的手电一开眼前便出现一张死人脸,一下子使我的神经吊了起来,我马上就感觉到自己的面前,只有几厘米的距离,好像出现了什么东西,我呼出去的热气,撞在那东西上,反冲到我的脸上,带着一股腥臭的味道。

没这么灵吧,我想,从那“老痒”刚才的表现来看,物质化能力非常难以管制,否则我们刚才也不会给巨眼黑蛇撞得如此狼狈,照道理不可能这么容易就弄出个怪物来。

错觉,我对自己说,千万不要上他的当,在这么封闭的一个黑色窨里,恐惧是肯定有的。

我深吸了一口气,忽然,脸上一湿,好像有一条冰冷的东西一掠而过,我一下子浑身冒冷汗,几乎要尿裤子,小心翼翼地摸了摸胸口,心脏狂跳,只觉得全身发软,他娘的这下子没错了,妈的,黑暗里果然多了什么东西。

我不敢打开手电,人缓缓地往后靠,想紧贴住石壁,可是我的背一靠到后面,我马上发现那不是石头,而好像是一片一片的鳞片……我甚至能感觉到鳞片下面筋肉的蠕动。

天哪,我在胡思乱想什么,背后怎么会有鳞片?我赶紧闭了闭眼睛,紧紧抓着自己的手电,举到自己面前,刚想打开,突然听到“老痒”做作地惊叫了一声,“老吴,怎么不开手电啊?我帮你照照!”

接着他的手电就亮了,我猛地看见就贴着我的鼻子尖,一个巨大的蟒蛇头昂了起来,它犹如水桶一样的身体盘绕在洞穴里,我的头顶、背后的岩石全变成了鳞片的墙壁,黑得犹如宝石,被老痒的手电一惊扰,四周鳞片搐动,身体缓缓摩擦,发出令人胆寒的嘶嘶声。

\chapter{烛九阴}

贴着鼻子的巨大舌头,满眼蠕动的鳞片,我不知道怎么来和别人说这种震撼,一下子我的心脏好像停止了跳动,浑身僵硬得犹如石头一样。

第一次实际领略这种能力的巨大威力,让我仅有的一丝怀疑也一扫而光,可是这条巨大的黑色蟒蛇是如此的真实,每一片鳞片,空气中的气味,那种无处不在的摩擦声都毫无破绽,我实在想象不出这东西是怎么突然产生的,如果刚才亮着灯,难道会“砰”一声凭空就变出来?

“老痒”还在外面叫着什么,我也没有心情理会他,只觉得那种爬行动物毫无感情的目光在我身上徘徊。本来我所处的岩石缝隙就小,现在突然出现了这一条黑龙一样的巨蟒,连做广播体操的空间都没了,这个时候,只要那条蟒蛇随便一张嘴巴往边上一咧,我就马上嗝儿屁着凉,什么都完蛋了。

我心里闪电一般盘算了一下,蟒蛇的嗅觉和视觉都很灵敏,没道理看不到我,现在只有一个希望,就是它对于我这样的体形不感兴趣,蟒蛇是不会捕食体积太小的东西的,我只要坐着不动,不引起它的恐慌,它可能就会放任我不管,但是如果这一招不管用,那这一次就真的无计可施了。

我咽了口唾沫,尽量不让自己发抖,巨大的舌头在我耳边舔过,留下极其难闻的唾液,但是,幸运的是,它只是抬起头注视了我一下,马上转头去看在石头后面的“老痒”的手电光源。

“老痒”躲在挡住洞口的巨石后面,看到蟒蛇没攻击我,反而转头向他探了过来,马上意识到不对劲,封住通道口的巨石,相对于巨蟒只有它的脑袋一样大,根本挡不住它,我听到老痒骂了一声,忙缩回石头后面,喀嚓一声关了手电。

四周一下子黑了下来,巨蟒两只黄色的眼睛在黑暗中发出荧光,我仍旧大气也不敢出,隐约看见巨蟒轻轻顶了两下,见石头没动静,突然缩起了脖子,做了一个攻击的姿态。

我脑子里出现了电视里蟒蛇捕食的动作,马上知道接下来会发生什么事情,刹那间,蟒蛇缩起的脖子犹如子弹一样撞了出去,就听一声闷响,整个山洞一震,堵门的巨石像风筝一样给撞飞,我听到“老痒”一声惨叫,接着就是石头互相撞击的声音接连不断地传了过来。

虽然知道外面不是真正的老痒,但是这一声惨叫还是让我条件反射地心里一慌。巨蟒发现了石头后面的空洞,但是它的脑袋太大了,怎么也钻不出去,它的身体在缠绕中不停地弓起来,我左躲右闪不给它卷进去,不然给它两边的蛇鳞一夹肯定骨头尽断。

几次尝试不行,蟒蛇开始烦躁起来,甩着脑袋开始撞向那洞口边上的石壁。蟒蛇的身体盘起来看上去已经非常吓人,如今龙一样舞动起来,更是壮观得离谱。几下子那洞口就给它撞裂了一个口子,巨蟒用力一转,脑袋便钻了出去,鳞片摩擦着石壁,把整块石头都挤出了裂缝。

巨蟒将前面挡路的石头尽数向外推去,我跟着蟒蛇出去,看到“老痒”躺在碎石头堆里,几乎全部身体给压在石头后面,气息微弱。看到我,咳嗽了几声,似乎想说什么,可是嘴巴一开,血就从嘴角流了下来。

我检查了一下他的伤势,试着搬动了一下石头,可是一眼看下去,下半身已经全部压烂了,实在连看都不能看,我叹了口气,问他道:“你……你还有什么话说?”

他看了我一眼,咬了咬牙,从岩石缝里扯出他从王老板那里弄来的背包,甩给我。

我接过包,心里也不知道是什么感觉,他咳嗽了几口,吐出很多血,然后也不再说话,闭上了眼睛。

我顿了顿,想问问他当天到底是怎么一个经过,突然“轰”的一阵巨响,整个山洞狂震,我几乎连坐也坐不稳,撞到岩壁上,顶上又是悠长的一连串石头开裂的声音。

我吓得够戗,心说难不成外面那条巨眼蛇又开始撞了,忙猫着腰向洞外爬去。“老痒”这时候突然嘶哑地叫了一声:“老吴!”

我愣了一下,不知道他还想说什么,回头一看,只见他对我张了张嘴巴,突然他所在的那块地方坍塌了下去,上面的石头瀑布一样翻落下来,一闪之间他就像陷入泥沼一样消失在碎石堆里。

我心中一悸,竟然有一种撕心的感觉,但是此时也没有时间调整情绪,几个翻滚避开落石冲到洞外,正赶上一团黑影又撞了过来,我赶紧往边上一翻,黑影子撞到山体上,整块山壁都给撞得震动起来,石块纷飞,山体裂出了一条裂缝,一直从我站的位置延伸下去。

我看到撞得如此厉害,不由得奇怪,这蛇难道不要命了?转头一看,原来不是这样,只见刚才爬出去的那条黑蛇巨蟒,已经和从青铜树中爬出的细鳞巨蛇缠绕在了一起,斗得难解难分。那细鳞巨蛇体形比蟒蛇大出不少,但是打斗起来却丝毫占不得一丝上风,加上两条都是黑色,一时间也看不出谁是谁,只见两团黑色的旋风在青铜树上不停地缠绕,尾巴乱扫,将四周的石笋石乳拍得像炮弹一样乱飞。

我从没见过如此惊心动魄的场面,只看得呆了,突然一条尾巴直扫在我的脚边上,我站的整块石头给扫成了石粉,情急之下忙往四周一抓,却没料到边上的石头全部都已经给撞得松动了,一下子没抓牢,整个人向下面的深渊栽了下去。

几分钟内几次经历大生大死,一下子我也反应不过来,大叫一声,忽然听到了隆隆的水声,接着浑身一凉,耳边一静,整个人竟然摔进了水里。

他娘的,哪来的水?

我一直刺进水里六七米才停了下来,入水的姿势根本无法调整,就听见脖子咯嗒了一声,不知道是不是断了。浑身用不上力气,人直往水里沉去。

正在无计可施的时候,一个人影从背后游了过来,将我托住,把我往上带去。

我回头一看,原来是一直躲在下面岩洞里的凉师爷,大概也是给不断上涨的水逼了出来,看到有人掉下来,过来拉了我一把。

冲出水面一看,只见我们刚才爬上来的深渊不知道何时变成了一个水潭,水里有水流涌动,不知道由哪个地方涌进来,水位还在迅速地上升。

我看着四周,心说难道他们三年前来这里的时候,这里会是一个水潭,但他娘的这样一来,岂不是回不去了。

我的水性比凉师爷好,他将我拉上来后自己没了力气,直往下沉去,我将他拉到青铜树边上,也不想和他计较以前的事情,问道:“这是怎么回事?”

凉师爷咳嗽了几声,这才说道:“外面肯定下过一场雨,这是山洪,这里这个季节经常有山洪。洪水泻进我们过来时的地下河里,那条河肯定和这里墙上的几个岩洞有连通,高海拔上的洪水冲下来,水位上升,水就倒灌进来了!山洪一过,水位马上就会降下去。”

我心里暗骂一声,这样一来上下不着边际,也不知道该从哪里出去好了,抬头一看,只见一团巨大的黑色影子还在上面缠斗,心说乖乖,现在已经斗成这样了,待会儿要掉进水里,不真成龙潭虎穴了,我们还不给折腾死?

还没想完,耳边呼啸一声,黑色巨蟒已经摔了下来,直摔进水里,一时间水花四溅,不大的水潭像开水一样沸腾了起来。

紧接着细鳞巨蛇也顺着青铜树爬了下来,凉师爷看到那蛇巨大的紫色眼睛,吓得整个人往水里沉,我把他拉起来,他哆嗦着说道:“我的天!这东西是哪里来的?这……这条是烛九阴啊!”

我听这名字怎么这么熟悉,拉着他直往青铜树后面躲,问他怎么回事。

凉师爷咬着舌头轻声说道:“烛九阴是龙,古时候叫做烛龙,其实是一种远古时代的巨大毒蛇,帝舜时代用这种东西来炼油做烛照明,几千年前就灭绝了,怎么这里还有一条?”

我从来不知道这些事情,当下感觉到奇怪,既然我不知道,那这不可能是我幻想出来的,那难道是真的,这青铜古树里真的有一条远古时候的巨大毒蛇?

凉师爷继续说道:“这么大的烛九阴不知道活了多少年了,你发现没有,从这里看只能看到它一只眼睛,烛九阴的眼睛是横着长的,你现在看到的这一只应该是本眼,还有一只眼睛长在这只眼睛上面,叫做阴眼。传说千年的烛九阴阴眼连着地狱,给它看一眼就会被恶鬼附身,久之就会变成人头蛇身的怪物。”

我想起那老痒那种毒蛇一样的表情,心里一阵发寒,回头偷偷看了一眼,所幸烛九阴的注意力完全不在我们身上,我感觉到水下的水流变得极度混乱,知道黑色巨蟒还在水下,烛九阴盯着水里,恐怕是怕巨蟒突然袭击。

水位不停地上涨,我们越来越靠近烛九阴的身体,凉师爷紧张得要命,我看了看头上,这岩洞的顶上应该有一处出口,只要水位上升得够高,我们就能爬到那上面出去,只是不知道这水位能上到多少,毕竟这里非常靠近山顶,过千棺阵的时候,棺材没有给水浸过的痕迹,水位不可能高过那一边,具体能到哪里我也不知道,只好浮一点是一点了。

我将自己的想法轻声告诉凉师爷,他完全听不进去,这个时候,几只白色的面具从水里浮了上来,那是螭蛊的壳。我心里突然感觉到不妙,拿起一只一看,嘴巴部分的空腔是空的,里面的蛊虫不见了。

“妈的!”我骂了一声,突然意识到为什么那条蟒蛇在水里潜了这么久都不上来了,打起手电潜进水里一照,只见无数螃蟹腿一样的虫子,有些还带着面具,有些只剩下身体,犹如蚂蟥一样附在那条黑色巨蟒的身上,白花花的一大片,黑色巨蟒肚皮朝天,还在不停地翻滚,但显然没办法甩掉这些虫子。它的身体撞在岩石上,蛊虫的面具给蹭掉,但是虫身还是牢牢地吸在蛇身上,看起来古怪异常。

一些蛊虫无法抢到位置,在蛇身的四周游荡,行动非常的敏捷,不妙的是,一看到我手里的手电,所有的蛊虫突然都顿了一下,然后迅速从蟒蛇身上弹开,我还没反应过来,眼前一花,所有的虫子犹如海里的巨型鱼群一样向我直围过来。

这些东西游得极快,我一看不好,已经来不及反应,情急之下,我往后一贴,狠狠地咬了自己的手心一口,这一口连我自己都不明白为什么咬得那么狠,一下子鲜血涌了出来,我把手在水里挥动,将血均匀开来。

蛊虫忌讳我的血,一下子冲到我面前又游了开去,不敢靠近。成群的白色虫子在我面前形成一道虫墙,我甚至还隐约觉得这些虫子排列的起伏有点像人的脸。

凉师爷吓得要命,二话不说就往青铜树上爬去,我知道在水里待着也不是办法,就探头出水,回头一看,烛九阴已经发现了我们,巨大的蛇头对着我们的方向,那只紫色的眼睛已经闭上,取而代之的是一只血红色的眼睛,不知道什么时候张了开来,怨毒地注视着我们。

\chapter{脱出}

这只红色的眼睛里布满了跳动的血丝,看上去诡异异常,我一给它对视,突然有一股灵魂被抽离的感觉,只觉得强烈的恶心和头晕,马上把脸转过去。

凉师爷却好像中了邪一样,眼睛直勾勾盯着那只血眼,一动也不动,我朝他叫了两声,没有反应。

凉师爷说过“烛九阴”的阴眼通着地狱,我知道肯定不对劲了,忙掬起一捧水就泼向他。

可不知道是烛九阴突然往前探了探还是如何,那捧水竟然没有泼到凉师爷的身上,而是泼到了烛九阴的脑袋上。

烛九阴给我泼起的水花吓了一跳,眼睛一闭,蛇头往后一缩,就想发动攻击。我赶紧贴到铜树后面,蛇头撞在青铜树上,将那些枝丫全部都撞弯了。这个时候,我想到了我从“老痒”那里拿来的背包,里面可能有什么武器,急忙将背包翻到前面。

他的包里肯定没有枪了,但是我记得有几根他们原本用来炸墓墙的雷管子,现在我手无寸铁,有点大威力的东西威慑一下也好。

烛九阴从青铜树的一边盘绕过来,我一边移动不让它看到我,一边连滚带爬地爬上去,抓住背包,就往里掏。

那背包塞满了东西,我把那些食物全部都拿出来丢进水里,终于摸出来我认为的雷管,一看,不由一呆,他妈的刚才看的时候太马虎了,那一捆东西,竟然是黑色的蜡烛。

这时候蛇头已经探了过来,看见我又突然折起蛇脖,又做出了攻击的姿势。

蛇的平均攻击速度只有四分之一秒,这条虽然大了一点,估计也慢不到哪里去。我一看再耽搁一秒就完蛋了,扯起背包就往水里跳。

但是我落下的速度还是太慢,突然黑影一闪,射出的蛇头一下子凌空将我咬住,然后蛇身一卷,就想把我缠绕进它的身体里。

我的手在包里乱摸,这个时候,突然摸到了他们用的那种信号枪,一下子手忙脚乱,下意识之下就扣动了扳机,背包给轰出了一个大洞,混乱间也不知道是不是信号弹在蛇嘴巴里爆了开来,只觉得虎口一热,然后就是天旋地转。

我“啪”的一声又落到水里,浮出水面,回头一看,烛九阴嘴巴里的信号弹正发出炽热的白光,空气中竟然弥漫着一股蜡的味道,而且不知道为什么,它的全身都开始冒出青烟来了。

这种蛇本身体内的油脂就非常容易燃烧,不然古人也不会捕猎它来做蜡烛了,但没想到竟然能够这样就烧起来,它体内流的到底是什么东西?

烛九阴极度痛苦,再也管不了我们,不停地扭动着身体,巨大的尾巴拍打着岩石,那一边本来就已经出现了一条巨大的裂缝,给它继续拍打着,一条裂缝扩散出好几条小裂缝,整块山面不停地开裂,似乎整个岩洞都可能崩塌了。

我不知道烛九阴会不会这么容易就死,继续翻动那只背包,再也没有有用的东西,就将背包往水里一扔,这个时候,突然水下激流溢滚,潭水竟然向烛九阴撞出来的裂缝涌了过去。

这里的山体里面洞系众多,看样子裂缝后面的山体已经给撞穿了,水不知道涌到哪里去了。我最后看了一眼青铜古树,四处去找凉师爷,已然不见了踪迹,眼看着上面的石头开始给涌出的水冲得大块大块地塌下来,烛九阴更是发了狂一样乱舞,忙往后一仰,顺着水流就给卷进了缝隙里面。

缝隙极深,里面一片漆黑,因为是坍塌出来的通道,里面石头很不规则,水流撞出不少漩涡,我打着转儿在里面东撞西擦,勉强感觉到自己应该是在往下游漂去。

大概转了有十几分钟,突然我感觉到自由落体,接着就一头栽进水里,忙挣扎出来看,发现已经给水流带到了来时的地下河里。这里的水流比我们刚才看到的还要湍急很多,应该是和凉师爷说的一样,外面下过一场大雨。

这里水流虽然非常快,但是没有岩缝里那么多的漩涡,而且水有一点温度,我得以控制了一下自己的肢体,心里开始盘算前面的情况。

这条地下河由上而下,不知道通到什么地方去,要是直冲入到几十米深的地下,我真是无话可说,不过按照来时的方向,如果它中途没有变换大的方向,我估计应该会给冲到来时渡过的那条河里。

当然前提是这一路上顺利,我紧张地看着前面,唯恐出现什么岔口,这个时候眼角的余光一闪,我看到地下河的河壁上刻着什么东西。

这里的地下河道,看岩石的冲刷情况,历史应该与这座山一样古老,上面有什么东西,应该不会是近代刻上去的。我看准了一个机会,拉住从顶上垂下来的一根石柱,停住身体,用手电一照,我惊呆了。

河壁的两边,全是和我们在青铜树顶上的棺椁内看到的一样的浮雕,连续成画,有些已经塌落,但是大部分还是保存得很好,线条明快流畅,衣纹飘逸,每幅各异,形象生动,极具动感。

我一眼看上去,就知道这些浮雕描绘的是古代少数民族祭奠青铜树的过程,其中的场景极其生动,有一幅浮雕上,是那棵巨大的青铜树上挂满了奴隶的尸体,奴隶的血流入青铜树内,顺着上面的沟壑一直汇流而下;有一幅则是他们将奴隶的尸体抛入青铜树的内部。

浮雕有很大一部分淹没在水里,最底下的一切已经给水冲平了,看来他们雕刻的时候这里还没有水。

从这里的浮雕来看,这种祭祀青铜树的祭奠规模很大,我一直看下去,却越看越觉得奇怪,有一些浮雕描绘的场景和祭祀又不相同,我无法理解。

其中有一幅浮雕,表现的是古时候的那些先民将一些液体倒进青铜树的情形。接着下一幅,就有一条和刚才看到的一模一样的“烛九阴”从青铜树里出来,很多穿着像战士一样的先民用弓箭和长矛围着它,显然是一种狩猎的场景。

按照我刚才的理解,这棵青铜树应该是古时候一种特殊的神权象征,那青铜树中的“烛九阴”在古代是一种龙,在一些笔记小说里,“烛九阴”甚至给抬到了盘古一样的高度,应该会给人当成神兽来顶礼膜拜,这里的人怎么会狩猎它呢?

我继续往下看去,希望能从后面看到答案。后面还有一些仪式的内容,我可以看到所有的先民都是带着面具,面容呆滞,但是,每一幅浮雕中,总是有一个人雕刻得特别魁梧。看这人的服饰和神态,我可以基本肯定,这个人应该就是他们的首领,而且应该就是我在夹子沟的悬崖上看到的那一座雕像的原形。

那一座雕像的脑袋给炸弹炸没了,我那时候总觉得不太对劲,但是一路过来始终没看到他的脑袋,这一次正好可以看个仔细。

我拉住顶上的钟乳柱,贴近地上的岩石,抹掉上面的污渍,凑过去看。

浮雕里的首领图像,比其他人都几乎大了一倍,就如一个巨人一样。如果按照我以前的设想,这里的雕刻都是按照正式比例,那这个首领可能真的有如此高大。

可是离奇的是,所有这些浮雕上,这个首领的脖子上都长着一个蛇头,看上去也不像是带着面具什么的。

我虽然有一定的考古知识,但是这些需要大量阅读而积累的东西,我还是没什么头绪,只知道单从这些浮雕的表面意思来看,我感觉凉师爷当时的判断可能有一些偏差,这棵青铜树可能不是单纯用来祭祀的,而是用来进行某种狩猎仪式,那些牺牲的奴隶,可能就是将“烛九阴”从地底下引出来的诱饵。

青铜树深入地下不知道多深,这些“烛九阴”应该是生活在极其深的地底,怎么在那种地方生活也不是我能考虑的事情,我只是好奇,这些先民搞这么大的阵仗捕猎“烛九阴”是为了什么?

浮雕上面并没有给我答案,我看到最后只是一些庆典的场面,“烛九阴”被捕猎上来怎么处理,并没有雕刻出来。

基本的情况我已经知道了,我看了看水位,有继续上涨的趋势,只好放掉双手,继续随着水流向下漂去。

手电在经历了这么长时间后,已经变得非常的暗淡,最后淡到完全没有照明的作用,我索性关掉,在黑暗中随流而动。

这一段时间非常的难熬,我几次都给冲下一些小的瀑布,虽然不致命,但是难免给撞得鼻青脸肿。足足有好几个小时,我不知道周围是什么,不知道自己要到哪里去了。

我逐渐感觉到绝望起来,也不知道自己刚才有没有转弯或者进入岔口,如果自己判断错误,那我现在说不定正在给带入无尽的地下河深处,也不知道这条河通到什么地方去,难道会冲到“烛九阴”生活的底层去?

那到底是一个什么地方,说回来,会不会有什么帝王的陵墓修建在地下河的深处,这倒是一个好创意。

就在我胡思乱想的时候,前面突然看到一丝光亮,看得我浑身一震动,接着我就听到隆隆的水声,我心中大喜,知道前面肯定是出口了,十几个小时没见到自然光了,我扔掉手电就向前游去。

我的速度非常快,只是几分钟的工夫,我的眼前突然一闪,然后一片白光,什么都看不见了,那是太久没看到光线的视觉迟钝,我心中大叫,可是那一刹那,一种熟悉的感觉突然从我身下传来。

又是自由落体!又是一个瀑布!

而且从水冲出的劲道和底下传来的声音来看,这瀑布肯定不小,不知道下面是什么,如果水太浅,那我死得真是太冤枉了。

我的耳边一片呼啸,电光火石之间,没等我的视力恢复,我已经一头栽进水里。

那一刹那我手往下一伸,马上摸到了一块石头,糟糕,太浅了!我刚意识到这一点,脑袋已经磕到了什么上面,眼前一黑,就什么都不知道了。

(《秦岭神树篇》完)

◆ 第四卷 云顶天宫(上) ◆

\chapter{新的消息}

我昏迷了三天时间,醒过来的时候,已经给你送到了医院里面,刚睁开眼睛的那一刹那,我什么都记不起来,只觉得天旋地转,止不住的恶心和头晕。

两天后,这种情况才一点一点好转起来,但是,我的语言能力全部丧失,无论我想说什么,我发出来的声音全部都是怪叫。

我以为自己的脑子摔坏了,影响了语言的神经,非常害怕,不过医生告诉我,这只是剧烈脑震荡的后遗症,叫我不要担心。

我像哑巴一样用手势和别人交流,直到第四天,我才能勉强开口去问医生,我现在在什么地方,他告诉我,这里西安市碑林区的红十字会医院,我是几个武警带回来了,具体怎么发现我的,他也说不清楚,只说我全身大概断了十二根骨头,应该是从高外坠崖导致的。

我胸口和左手打着石膏,但是不知道自己伤的多重,听他一说,才知道自己命大。我又问他大概什么时候能出院,他对我笑笑,说没十天半个月,连床都下不了。

当天晚上,送我过来的武警听说我能说话了,带了水果蓝过来看我,我又问了他问医生同样的话,他也不知道如何回答我,只说有几个村民在蓝田的一条溪边找到了我,我是给放在一个竹筏上,身上的伤口已经简单处理过了,医生说道,要不是这些处理,我早就死了。

我觉得奇怪,我最后的记忆是落进水里的那一刹那,按道理最多也是应该给冲到河摊上,怎么给放到竹筏上去了,二来,蓝田那里离夹子沟那一带有七八十里路呢,难道,我们在地下河走过的路。不知不觉已经有这么长一段距离了?

我编了一个登山堕崖的诺话,千恩万谢的送走了武警,马上给王盟打了电话。让他到西安来一趟,带一些钱和我的衣服来,第二天王盟就到了。我把医药费付清,然后重新买了手机和手提电脑。

我问王明最近生意怎么样?他说没什么重要事情,就是我老爸找了我很多次,我心说出来的时候没想到要这么长时间,可能担心,于是给家里报了平安,不过我老爸不在。我和我老妈说了几句。顺便问了问三叔的消息,还是没有音信。

看来一切还如我来时一样,我感叹了一声。

接下来几天,我百无聊赖,忽然想到老痒。心里发酸,便躺在病床上,翻着我坠山时候穿地那件已经完全破烂的登山服,寻找老痒的那本日记,日记倒还在,只是给水泡地什么都看不清楚了。我勉强辩认的读了一会儿。再看不出什么,又连上医院的电话,上网打发时间。

我查了很多资料,不过网上关于古董地信息到底是不多的,我只有将我脑子里青铜树的景象简略的描绘了出来,发给一些朋友去看,后来陆续收到回信,大部分也都不知道这是什么东西,而且他们对我的描述也不相信,然而也有几封信对我挺有启发。

其中有一封是从美国发来的,我父亲的一个朋友,和我挺聊地来,他在电邮写道,这一种青铜树,叫做“篪柱”,因为形状像篪(古乐器),八四年的时候,攀枝花一座矿山里也发现过一根,但是远没有我说的这么大,只有一截,深入地下的那一段已经完全锈化了。

到现在为止还没有任何文献资料能够解释这东西是用来干什么的,不过根据山海经和一些文字记录下来的少数民族叙事诗,这东西地确和远古时期的捕“地龙(蛇)”活动有关。

“烛九阴”应该是生活在极深地脉里一种蛇类,因为长年在陡峭的岩石缝隙中自下而上它几乎没有正视的机会,所以两只眼睛像比目鱼一样变异了。古人用鲜血将其从极深的地脉中引出来,然后射杀,做成蜡烛。听起来很冤枉,但是那个时候,持久光源是极其珍贵的东西,特别是对一些晚上活动或者生活在漆黑一片的岩洞里地人来说,更是如此。

我觉得他分析的有点道理,不过还是不能解释,为什么碰到所谓的“柱”,会产生那种奇妙又恐怖的能力,我回信过去,问他历史上还有没有类似的事情发生过?

他回信过来,还付上了一份残卷,是一本笔记体小说,里面记录了清朝乾隆年间发生的一件事情,里面提到了西安矿山挖出青白石龙纹盒,乾隆皇帝打开一看,当夜就秘招几个大臣入宫,秘谈到了音半夜,之后就有乾清宫失火。那几个大臣,除了一人有名的之外,其他几个,全部也没有善终,最后都给莫名其妙的杀了。

我看时间,大概也就是李琵琶《河木集》写的那一件事情发生的时间,也就是应该有关联,看样子,最后挖出那只白石龙纹盒的人和了解这件事情的人,都糟到了灭口,皇帝下了这么大决心,要保守一个秘密,那这白石龙纹盒里放的到底是什么东西?会不会就是这棵青铜古树的来历呢?

我再一次回信征求他的意见,他只回了一句话,要挖下去才知道。

我苦笑一声,知道这是不太可能了,谁知道下面还有多深,也许当初他们铸造这东西,花了几个世纪时间,就算有人愿意挖,我绝对是看不到挖出来的时候了。

还有几封信,是我二叔发给我的,他说,那个时候少数民族,文化传承西周时期的装饰风格,介是那个时候忆族交流有限,而且交通和通讯极度不发达,所以应该有一个时滞,也就是说,我反时间估计得太早了,按照一般规律,那个时候,中原地区应该改已经是秦后期。

那个时候,几乎所有的活动都和秦始皇修建陵墓有关,他们捕猎烛九阴,可能是为了提炼“龙油”。进贡给皇帝炼单或者类似的活动。而且根据地质探测,秦始皇陵的最底层,也有巨大的金属物体。环绕整个陵墓,按照道理,当时的冶金技术应该无完成如浩大的工程。这一部分的修建者,应该是冶金技术特别发达的外来民族。

二叔是秦始皇的忠实Fans,凡事都能扯到那一段去,我对他的推测不以为然。

一个月后,我出院回到家里,整理了一下后,我开始收拾心情。从新投入生活。我整理了已经几乎撑爆的信箱,理出一些杂志和报纸后,我找到了一封没有署名的快件。

老吴:

猜到我是谁吗?

对,我没死,或者说。我又活了。

我很抱歉把你卷进这件事情来,不过毕竟你是我唯一能信任的人,我没有其他选择。

现在整件事情已经完成了,我们的关系,也必须到此结束了,我很高兴能和你做过朋友,但是现在这一切已经不重要了。

你是不是很想知道三年前到底发生了什么事情?

三年前。我和一群辽边佬到秦岭那一带踩盘子,我们根据当地人的传说,在山顶的榕树林子找到了一个树洞,我们考虑再三准备冒险下去,过程你全部都知道了,后来我就困在了石洞里。

当时,我已经绝望,虽然我不会这么快死,但是活着对我来说更可怕,永远生活在狭窄的,一片漆黑的大山深处,永无出头之日,那种痛苦,你应该也体会过了。

我在黑暗中整整呆了四个月,这四个月简直就是地狱,不过,在这段时间里,我不停的思考,我知道了,这种能力在和潜意识有关,比如说,我想要在石头上开一个门,我必须让自己相信石头上本身就有一个门,否则,就算你想破了头,门也不会出现。

人自己是无法欺骗潜意识的,所以使用这种能力,必须要引导,这非常难,我跟你说过了,一旦引导失败或者出现偏差,你物质化出来的就不知道是什么东西,非常地可怕。

我不停的做事情,逐渐掌握了一些窍门,但是,这个时候我发现,这种能力会随着时间的减退而逐渐消失。这种感觉非常明显,就好像人一点一点感觉到疲劳一样,我意识到,如果再不采取办法出去,我可能会饿死在这里。

我走投无路,尝试着用那种能力,复制了一个自己,我没想到这会成功了,自己也吓了一跳,一下子,我突然发现我出现在了山洞的外面。

那时候我并没有意识到我是复制出来的,我和本我的所有记忆都完全一样,所以当他叫我的时候,我完全不认同我是复制品,他开始骂我,说我想代替他存在于这个世界,说要让我消失。我很害怕,我觉得洞里的那个是怪物,所以,我不管洞里的本我如何的呼号,还是找来了炸药,将这个洞完全炸塌了。

事实上,我的确知道自己是给复制出来的,但是我潜意识不愿意相信这件事情,所以我选择了一种受破坏的状态,我把本我杀了,然后告诉自己,我只是杀了一个替代品。

青铜树给人的能力,时间很短,所以我取下了一根青铜枝桠,从表铜树底上的暗道出去,希望带上青铜树的一部分,能够使我的能力持久一点,这样我才有可能逃到外面去,后来证明我的想法没错,我回到外面,挖出我们到这里之前挖到的东西,又怕青铜枝桠太碍眼,将他埋了进去,然后回到西安,想找个地方把手里的东西卖了。

可惜的是,做买卖的时候,我在古懂摊上给便衣给抓了,后来,你也知道了,我回到家里,我妈已经走了,这些事情,我没有骗你。

还有一些事情,我也必须要告诉你,拥有这种能力,并不是没有代价的,我的记忆力非常的差,很多事情必须预先写下来,才能够记得,那就是使用能力的后遗症,我一路上,本可以很好将你安顿好,让你不知不觉的就帮我完成这一次的探险,但是遗憾的是,这三年来,我忘记了很多东西,我怎么出来的,我都记得不清楚了,所以破绽百出,我估计,再有两三年的功夫,我可能完全失去记忆的能力。

你身上也有那种奇特的能量,我不知道对你会不会有影响,你要多保重了,按照我的计算,这种力量也许会在你身上残留好几年,但是十分微弱,几乎感觉不到。

老痒

我看完整封信,长出了一口气,不知道说什么好,信封里面,还有一张照片,是他和他妈妈坐在般上照的,后面是大海,应该是到国外去了,她妈妈很漂亮,很年轻,和他站在一起,反倒是像情侣,我仔细看了看,却总觉得,她妈妈的脸上,有一股妖气,一种说不出的狰狞,也许是心理作用吧。

不知不觉冬天来临了,窝在空调房里,整个下午都庸懒的连打瞌睡都没力气,我躺在“西冷印社”内堂的躺椅上,双脚冰冰凉,不知道干什么好,正在半梦半醒之间,王盟走了进来,对我说:“老大,有人找。”

我勉强反应过来,打了哈欠,心说三九天的,还有人逛古玩店,这位也算是积极了,不过再怎么说也算生意,爬起来拍了拍脸,抖擞精神走了出去。

外面空调小,冷风一吹,人打了激灵,一看,原来是济南海叔手下那小姑娘,正冻得直打哆嗦,我心想估计是给我带支票来了,心里一热,忙叫王盟去泡茶,自己问她道:“怎么,丫头,海叔让你来的?”

小丫头叫秦海婷,是海叔的亲戚,才十七岁,已经是古玩界的老手了,她点点头,说道:“哎呀我的妈,怎么杭州比我们北方还冷呢。”

王盟笑道:“南方那是干冷天气,感觉刺骨一点,而且你们济南也不算太北啊。”

我看秦海婷只打牙花子,忙拉她到内堂去,里面空调暖和,把热手的水袋递给她,问道:“你也太怕冷了,这么样,暖和点没?”

她喝了几口热茶缓过劲来,还是在房里直剁脚,“稍微好了一点,人说杭州多美多美,俺叔不让我过来我还抢着来呢,谁知道这么冷,哎呀我下回再也不来了。”

我问道:“你叔叫你来啥事情啊?怎么也没个电话通知一声啊。”

秦海婷解下自己的围巾,从自己的皮包里掏出一封东西来,说:“当然是正事,给,现金支票,那块鱼眼石的钱。”

我一听果然是,接过来瞄了一眼,价钱不错,当即放进口袋里,说道:“那替我谢谢他。”

她又拿出一张请贴,递给我:“我海叔后天也来杭州,参加一个古董鉴定会,他说让你也去,有要紧事情和你谈。”

我问道:“后天?我不知道有没有时间啊,怎么不在电话里说,神神秘秘的?”其实我是不想去,古董鉴定,太无聊的事情,对行内人来说,说是一帮老头子在那里聊天,其实哪有这么多典故,是真是假,几秒钟就看出来了。

秦海婷凑到我的耳朵边上,小声说道:“俺叔说,和那条青铜鱼有关系,不去自己后悔。”

\chapter{二零零七年第一炮}

我和海叔的关系还没有好到无话不谈,平时也就是一些生意上的沟通,熟络之后我叫他声叔给他面子,他突然要和我套近乎,我感觉到有一些奇怪。不过小姑娘在我不好表现出来。随口答应了一声,问她:“怎么说?他查到什么消息了?”

秦海婷坏坏的一笑,“俺叔说,到时候再告诉你,俺也不清楚是怎么回事情,你别打听咧。”

我心里暗骂了一声,这个老奸商,估计是又想来敲我的竹杠了。

第三天老海果然到了,我把他从火车站接出来,带他上高架去预定的酒店,在车上我就问他,到底听来了什么消息,要是蒙我,我可不饶他。

老海冷的直发抖,说道:“强龙不压地头蛇,都到你的地盘了,我怎么敢蒙您呢,不过咱们别在这儿说,我都快冻死了。”

我给他带到酒店里,放下东西,去饭堂里找了个包厢。烫了壶酒,几杯下肚,总算缓过气来。

我看他酒劲一直到脖子,知道差不多了,问他:“行了,你喝也喝了,吃也吃了,该说了吧,到底查到什么了?”

他眨巴眨巴嘴巴,嘿嘿一笑,从包里拿出一叠纸,往桌子上一拍,“看这个。”

我拿起来一看。是一份泛黄的旧报纸,看日期是一九七四年的,他圈出了一条新闻,有一张大好的黑白照片。虽然不是很清晰,但我还是认了出来,照片拍的,是一条蛇眉铜鱼,边上还有很多小件文物,像佛珠一类的东西。

不过这条鱼的样子和我手里的和三叔手里的那一条都不一样。海底墓里墓道雕像额头上有三条鱼的浮雕,这一条应该就是最上面的那一条。这样一来,可以说三条鱼都现世了。我问老海:“你怎么找这报纸的?后面有什么隐情不?”

老海道:“我最近在帮一个大老板捣鼓旧报纸,你知道,有钱人收集啥的都有,你看,这是七四年的广西文化晚报。他要我一月到十二月都给他找到,我找了两个月才凑齐,这几天要交货了,在核对呢,一看,正巧给我看到了这条新闻。您说巧不巧?这份报纸就七四年出了一年,七五年就关门了,世面上难找啊。算您运气不错,我眼睛再快点就没了。”

我的眼睛向下瞄去,照片下有三百字左右的新闻,说这条鱼是在广西一座佛庙塔基里发现的。塔因为年代久远,自然坍塌了,清理废墟的时候挖出了地宫。里面有一些已经泡烂的经书和宝函,其中一只宝函里就放了这条鱼。专家推测是北宋后期僧人的遗物。

北宋?我点起一只烟,靠到椅背上,心里犯起嘀咕来,这种蛇眉铜鱼,第一条鱼,出现在战国后期的诸侯墓里;第二条鱼在元末明初的海底墓中;第三条鱼在北宋佛塔地宫里。搞什么飞机,时间上完全不搭界啊。

我翻了翻报纸的其他部分,只有这一条新闻是关于这条鱼的,这些个内容,其实没有什么新东西,等于没说。对于这条鱼,我还是一无所知,想着人也郁闷起来。

老海看我的表情,说:“你别泄气,我还没说完呢,这后面的故事还精彩着呢。”

我皱了皱眉头,“怎么说?难道这报纸还能衍生出什么来?”

老海点点头,说道,“那是,要是光找到一张报纸,我也没必要来杭州找你,是吧?这事情,还得从头说起。对了,你也是行里混的,知道不知道一个人,叫做陈皮阿四?”

我听了一惊,陈皮阿四是老时长沙有名的土夫子,老瓢把子,和我爷爷同代的人物,听说现在已经九十多岁了,在十年浩劫的时候眼睛瞎了,之后就一直没出现过,也不知道是死是活,但是他的名字在我爷爷嘴巴里,还是响当当地。

不过这个人和爷爷不一样,他是刀口上过生活的,就是不单单盗墓,杀人放火什么事情,只要是能弄到钱的,他都干,所以解放前人家都叫他剃头阿四,意思是他杀人像剃头一样,不带犹豫的。

老海提到这个人,我有点意外,因为他不是和我们同时代的人物,我也从来没和他接触过,这鱼难道会和他扯上关系?那这条鱼背后的故事,即使和我没关系,也绝对值得听上一听了。

老海看我不说话,以为我不知道,说道:“陈四爷的事情你不知道也不奇怪,到底和我们不是同一辈人,不过我得告诉你,这报纸上的这条铜鱼,就是他从那佛塔地宫里带出来的,事情还真没这报纸上说的这么简单。”说着,他就把当年的事情,简要的和我说了一遍。

原来,七四年的时候,陈皮阿四也有将近六十了,他的眼睛还没有瞎,当时正是十年动乱时期,他因为解放初期在国民党军队中当过排长,后来给化整为零当了几年土匪,所以没合法身份,这在当时给抓住是要弄死的,他只能在广西一带的少数民族地区活动,连县城都不敢踏入。

早几年除四旧,很多古迹都给砸得差不多了,陈皮阿四去过广西不少地方,因为广西在古时候不算中原,并没有多少古墓,他那几年过的还算老实。可是不巧的是,那年,他正巧在驾桥岭盘货经过,和当地几个苗民聊天,那几个人喝的多了,就说起猫儿山有座庙里的塔塌了的事情,说是动静很大,连地也陷了下去,塌出了一个大坑,坍塌的当晚,很多人还听到一声非常诡异的惨叫声。

陈皮阿四一听觉得不对,猫儿山他去过很多次,那地方的庙宇修建的都很坚固,怎么可能说塌就塌了?仔细一问,才知道这座塔并不是在猫儿山上,而是边上一条叫“卧佛岭”的山脉中心。这个地方很奇怪,四周都是村落,就是中间一块大概十几平方公里的盆地,海拔很低,里面植被茂密,树盖遮天蔽日,村落在悬崖上面,树林在悬崖下面。落差一百多米,就是两个完全不同的世界,而且从村落没有路下去,要下到这个盆地,只有用绳索。

当地人说,这个盆地肯定是有其他的进出口的,但是地下的植被实在太茂密了,行走都困难。以前下到下面打猎的和采药的苗民,经常会在里面失踪,所以一般没事没人愿意下去。

那古塔就是修建在这样一个地方。几乎就是在盆地的中心位置,平时人们从悬崖上看下去,只能看到一个非常小的塔尖露出茂密的树冠,而且给植物附着满了,下面是什么也看不清楚。苗民说,他们十几代前就知道这里有座塔,但是谁也没想到下去看过,现在也习惯了。最近有一天,突然一阵巨响,出来一看,塔尖没了,才知道塔塌了。关于这神秘的古塔,当地人还有很多传说。据一些老人说,这塔是古时候的一个高僧修建来镇妖用的,现在塔一倒,妖怪就要出来做恶了,那一声怪叫,就是妖怪挣脱束缚的叫声。

陈皮阿四听了之后,觉得很有意思,他隐约感觉这塔修建的位置和半夜苗民听到的那声音,有点不太对劲。但凡是他们这种人,可能都有一种奇特的直觉,可以从别人的叙述和一些传说中本能的找出信息。这一点,在我们这一代人中已经很难找到。

陈皮阿四思索片刻,决定去看看再说。

广西山脉分布众多,可堪称全国之首。猫儿山是其中重要的一个源头,地跨兴安、资源、龙胜三县,是漓江、资江、浔江的发源地,连接着长江、珠江两大水系。那地方有着大片的原始丛林,红军长征翻越的第一座大山老山界就在其中。二战期间援华飞虎队的好几架轰炸机在此神秘失踪,所以这地方一直给人传的有点玄乎。

陈皮阿四几经波折,来到“卧佛岭”上的一个村落里,站在土岗上往山脉中间的盆地一看,我操,那塔比他想像的要大多了,倒下去的时候砸倒了好几棵树,所以森林的绿色树盖上出现了一个缺口。在“卧佛岭”上,看不到缺口里有什么,但是陈皮阿四几乎立即发现了,在塔倒塌地方的一周,所有的树木都因为地面下陷,显得非常凌乱,看样子,塔的下面,果然有什么东西,而且体积比塔基还要大。

我听到这里,已经知道那是一座“镜儿宫”。“镜儿宫”是长沙一带解放前的方言了,就是说地上建筑的下面,有和地上建筑规模一样的地下部分,看上去就像是地上建筑在湖面上的倒影一样,上下两头是对称的。

这在北派也叫做“阴阳梭”,就是指整体建筑就像一只梭子插在地里,一面是阴间,一面是阳间。不过这样的古墓或者古建筑已经很少见了,大部分地面的遗迹已经毁坏干净,所以这种说法,在解放前十年内几乎已经没人提起。

陈皮阿四单单看着树木的排列变化,就能知道底下埋着“镜儿宫”,这种判断力没有极其丰富的经验是不可能做到的。我不由暗叹一声,宁神静气,听老海继续说下去。

陈皮阿四打定主意之后,心里已经起了贪念,佛塔的地宫里,只会有三样东西,要不就是舍利子,要不就是高僧的金身,要不就是大量的佛经,随便什么都是价值连城的东西。

但是他这么一个外乡人,在这里活动不太方便,一来自己身份特殊,出身又不好,二来苗汉两族那个时候纷争不断,这里几个村子都是苗寨,贸然进去,可能会引起别人怀疑。

考虑再三,他想出了一个计策,他出高价找了一个当地的苗人向导,他告诉向导他是从外面过来的支边老知识份子,过来的时候他的一个学生从悬崖上掉下去了。苗人民风淳朴,不谙世事,怎么会想到里面有诡计,一听有人坠崖,马上通知了全寨的人。年轻的苗族汉子用绳索扎了吊篮,将陈皮阿四连同几个帮忙的青年放到悬崖下面。

据陈皮阿四自己事后回忆,通过这一百多米的落差简直是地狱一样的经历。悬崖非常险峻,人的体重完全靠一条藤绳拉伸,屁股包在一个篮子里,风一吹,整个人陀螺一样打转圈,极度不稳。等他通过浓密的树盖,下到丛林底部,已经只剩下半条人命了。

森林的内部几乎没有什么阳光,光线极度昏暗,空气中弥漫着沼气的味道。这里树木的种类非常多,但是无一例外的,所有的地方都长着绿藓,泥巴非常松软,几乎站立不住。

陈皮阿四下来之后,装出体力透支的样子(其实是真的吓蒙了),坐在那里喘气。苗族首领看他年纪也不小了,一副小老头的样子,就让他在原地等他们回来,自己打起火把招呼其他人按照他指的方向去搜索。

等他们一走,陈皮阿四马上掏出罗盘,按照事先记下的方位,往丛林深处钻去,他估计着,这么大的区域,苗民们来回也要一个晚上的时间,以他的本事,应该足够找到“镜儿宫”的入口,来一个来回。可惜的是,他这一次来没有带足装备,能不能入得宫内,还得看自己的造化。

在丛林里没头没脑的走了整整四个小时,靠着罗盘和他这些年走南闯北的魄力,陈皮阿四终于来到了自己在“卧佛岭”上规划出的那片区域,也就是那一座塔四周的寺院遗迹。

在丛林里没头没脑的走了整整四个小时,靠着罗盘和他这些年走南闯北的魄力,陈皮阿四终于来到了自己在“卧佛岭”上规划出的那片区域,也就是那一座塔四周的寺院遗迹。

随着不断的深入,陈皮阿四看到越来越多的残檐断壁,显然这里的古建筑已经荡然无存了,只剩下一些地基和断墙,几乎和那些植被混合在了一起,也看不清楚原来到底是什么。但是看规模,这寺院面积极大,那座塔虽然倒在这一大片范围内,但是具体在哪个地方,也很难看的清楚。

陈皮阿四到底年纪不小了,四处一走,觉得有些气短,正想坐下来休息,突然眼前一闪,边上包着整面墙的植草丛里,突然收缩了一下,里面好像裹着什么东西。陈皮阿四吓了一跳,他一个打滚翻了出去,同时手里翻出一颗铁弹,回头一看,只见裹着墙壁的藤蔓草被里,有一具苗人的尸体,已经几乎干瘪了,但是尸体的肚子,不知道为什么,正在微微的鼓动,似乎里面有什么东西一样。

\chapter{镜儿宫}

要说死人对于陈皮阿四来说,是最平常不过的东西,不说墓穴里出来的干尸粽子,就是他杀过的人,随便数数恐怕也数不清楚。他翻身一看是具尸体,心里已经一松,心说哪里来的倒霉鬼死在这里,都成鱿鱼干了还吓唬人。

虽然这样想着,陈皮阿四手里还是卡着那颗铁弹,他这一手空手打铁弹子的功夫是他从小自己锻炼出来,可说是百发百中,而且他甩出铁弹的速度极快,普通人可能连他手里的动作都没有看到就已经给打瞎眼睛了。

看这苗人的装扮,死了没有十年也有两三年了,衣服基本上都已经破烂,亏的给大量的蕨类植物和爬地细藤裹住,苗人服饰的特征才保存下来。可是日晒雨淋的,怎么这尸身就没有烂光,反而有一点脱水的感觉?

尸体的肚子还在鼓动,陈皮阿四越看越觉得不妥。他这种人,有自己一套特别的行事方式,如果是我,当时肯定砖头就跑了。可是陈皮阿四从小就信奉先下手为强,心里转念一想间,手里已经“啪啪啪”连打出三颗铁弹子,全数打中尸体的肚子,心说管你是什么,打死再说。

铁弹子力道极大,几乎将尸体打成两截,下半身一脱落,陈皮阿四就看到里面一团黄色的不知名黏液,裹着大量的卵,不少卵已经孵化了出来,成堆的白色虫子在里面扭动,四周还挂着一些他非常熟悉的东西——蜂房,紧接着从尸体身上的破口处爬出了大量的地黄蜂。

陈皮阿四骂了一声,心说倒霉,原来是地黄蜂在尸体里做了窝。地黄蜂毒性猛烈,而且非常凶横,这下子他要倒霉了。

眼看着一层黑雾腾起,地黄蜂开始密集起来。陈皮阿四急中生智,从包里翻出他随身携带的解放军折叠铲,猛地从地上铲起一把湿泥,往那尸体的断口一拍,将涌出的地黄蜂全部封住,然后转身便跑。

已经冲出的地黄蜂蜂拥而上,他一边用衣服拍打,一边没头没脑的四处乱跑。幸亏他一铲子速度很快,才只付出了几个包的代价。等他喘着气停下来,拍掉身上残余的地黄蜂,已经不知道自己跑到哪个地方了。

陈皮阿四将身上中的蜂刺拔出,疼得他直咧嘴巴,心里还在奇怪,怎么会有地黄蜂在人的尸体里面做窝。这种毒蜂一般都是在地下,像蚂蚁一样,在广西的雨林深处,有时候还能看到像山包一样的蜂窝。别人以为是蚂蚁窝,翻开去找蚂蚁,还没等明白过来,就给裹成蜂球了。

广西云南这种地方,对于虫子的事情,不被世人了解的太多了,陈皮阿四只能怪自己倒霉。他一边处理蛰伤,一边四处查看。翻过一个山丘后,他突然愣住了。

只见一座巨大的石塔,就倒在他前面的山丘根部,塔身估计是六角形(无法辨认),气势磅礴,密檐宽梁。用刀刮开上面的青苔和缠绕植物,塔身上的浮雕石刻非常精美,但是明显这座塔给人焚烧过,所有的部分都有黑色的灼烧痕迹,可能是发生过火灾。

塔身、塔顶和塔刹全部已经开裂倒在地上,并且断成了n节。因为塔身太重,很大一部分压进了雨林的泥土里,塔下面给压倒的树木更是不计其数。

陈皮阿四经验丰富,知道塔一般由地宫、塔基、塔身、塔顶和塔刹组成。最上头的塔刹,因该有须弥座、仰莲、覆钵、相轮和宝珠,也有在相轮之上加宝盖、圆光、仰月和宝珠的,总之塔上面应该有一个珠形的东西,颇有价值。

他顺着塔身来到塔刹边上,塔刹在倒下的时候,中途可能撞倒了一棵巨大的“云杉”树,结果塔刹在半空就断了,塔刹头朝下插进了地里,须弥座碎裂。陈皮阿四看了看损坏程度,确定宝珠肯定成“宝饼”了,报废了。

他回到塔基出,半截断墙还在,爬进去,里面一片乱石头,下面肯定就是地宫。可惜这里不仅在修建宝塔的时候已经给人封死,而且上面还压了坍塌时候散落的大量碎石和碎砖,自己一把折叠铲,挖进地宫可能要半年时间。

陈皮阿四看了看罗盘,他下来的时候是傍晚,天色已经非常昏暗,现在月亮已经挂了上来,自己没打火炬,走了这么远,也不知道如何回去,看样子还是装成迷路的样子,等那些苗民来救好。想着,他先在塔基用撞断的树枝和枯叶,烧起一大团篝火,来吸引别人的注意力,一边爬到塔基参与部分的最高点,想看看,四周到底是一个什么样子的情况。

根据从“卧佛岭”上看下来的和他现在所见的,此时他所处的区域,应该就是树木长势非常凌乱的那一片地带。地面应该是比四周要低一点,那是因为回填地宫“杂填土”的时候,因为广西的特殊气候,土层水分太多,没有结实,随着水分的下渗,泥土里面形成很多气泡,一发生大的震动,像发泡馒头一样的泥层就塌了。

如此说来,陈皮阿四判断出了两件事情,一就是,地宫很大,但是不深,不出二十分钟肯定能挖到。二就是,泥土应该比较松软,不会耗费太大的体力。

此时他陷入了犹豫,到底是现在就进这个地宫,还是以后再来。现在看来,再回来一次也不是太困难。但是,陈皮阿四和所有的盗墓人一样,明知道下面有东西,是绝对无法忍住好奇心。

最后他一咬牙,妈拉个b的,管他娘的,这下面的东西老子要定了,要是等一下那群庙蛮子找到这里来,老子就把他们全杀了,丢进地宫里去,谁也不会知道。

陈皮阿四拆开折叠铲子,他没有带洛阳铲,也没办法定位,而且佛塔到底是罕见之物。里面没有棺材,定出来也没有,他凭着直觉,贴着塔基就开始挖盗洞。

很快他便挖到地宫的顶板,不是石头的,是曲木的整条树干割方了做的木顶。他心中大喜,用线锯开掉一个角,凋落的木块落入地宫之中,不久便传来落地声,他忙不迭的用手电往里照。

“镜儿宫”上下是对称的,就是说上面有多少层塔,下面也应该有多少层地宫,所以地宫极其深。从上往下望去,每一层之间没有楼板,最下面一片漆黑。

手电照过去,有一团白白的雾气一样的东西,实在无法说出是什么。

陈皮阿四想起那几个苗人说的,塔下面镇着妖怪的说法,不由得也有了一丝担心。但是这一丝担心转瞬即逝,他现在头热血涨。当下感觉地宫空气没问题,一边双脚搭住曲木宫顶,以一个倒挂金钩,头朝下倒进了地宫里,全身的力量全部压在了两只脚上。

倒进去后,他先调整了一下动作,先照了照地宫曲木宫顶的另一面,这种地宫是功能性的,不讳像古墓一样设置机关,或者搞很多装饰。陈皮阿四照了一圈后,却发现曲木宫顶的另一面,天花板的位置,有着大量的经文。

经文是刻在曲木上的,里面封了朱漆,是梵文。陈皮阿四汉字都不认识几个,是什么经文当然看不懂。

但是他本能感觉到,这应该是镇魔或者是伏妖的那一类东西,心里也不由的犯起嘀咕,难不成这下面真封着什么东西?

再看下面,他看的更清楚,每一层,都有一圈突起的外延,从上往下,一层一层看上去有点像楼梯,每一层上都有一圈等身的僧袍彩雕罗汉像,颜色流光溢彩,非常精致。所有的雕像面部向下,俯视着地宫的最底部,整个地宫一共有十几层,摆满了各种动作的罗汉像,足有百来具。

最近的罗汉像离他并不远,陈皮阿四倒挂着,看到罗汉像的表情时,突然感觉到一阵寒意,原来所有的罗汉像竟然都翻着眼白,表情有一种说不出的森然,和平时看到的那些不一样。

仔细一看,才知道是眼睛的眼拄因为图色太过真实,给手电光一照,好享空间反光太强烈,造成的错觉。但是他的手电光扫过,那些罗汉像瞬间变得狰狞无比,好比他们的表情发生了变化一样,看上去无比的骇人,真怀疑当初他们设计的时候是不是就是这样考虑的。

所以陈皮阿四看着这些罗汉,心里非常的不自在,但是他又不明白自己到底在怕什么,不由产生了退却的念头。

他的手电继续在地下划动,想看到一些出了罗汉像之外的东西。这个时候,他的手突然一僵,手电的光斑停在一个位置。

在离他大概有六七层的那一层突起处,他照到了一个奇怪的罗汉像,这个罗汉像和其他的都不同,他的脸不是俯视的,而是抬着头,脸正对着陈皮阿四,直勾勾盯着他的眼睛好享空间。手电光照上去,一闪间露出了一张狰狞的白脸,要不是一动不动,几乎要以为遇到鬼了。

陈皮阿四顿时吓得浑身冰凉,一下子连动也动不了,直觉得自己的双脚开始发软,人开始往下滑去。

说到鬼,陈皮阿四倒是真的不怕,自己杀了这么多人,可以说罪大好享空间恶极,怎么也不见一个半个来报复?但是他们那个年代的人,或多或少都有写迷信思想,陈皮阿四就认为自己这么多年能够混下来,是靠祖先保佑。

(人总要有点信仰,外八行的人是拜关公的。盗墓的人,北派拜的是钟馗,南派一般不来这一套,但是长沙那一带有说法,说是拜过一段时间的“黄王”。)

(黄王是什么?黄王就是黄巢,“满城尽带黄金甲”那位。为什么拜此人?听长辈们说,有几个理由,一是,这人可以说是杀人冠军。民间流传:黄巢杀人八百万,在数者在劫命难逃。什么意思?就是他杀人是有指标的,不杀到八百万,他不算完成任务。还有不知道是笔记小说还是中国特色化的民间传说,黄巢是目楗连罗汉(不是易建联)转世,这位主为救老妈放尽地狱八百万饿鬼,所以佛祖让他转世,一个一个杀回来,也就是说他回去是给佛祖招聘农民工的。)

这具雕像脸朝上他并不害怕,但是这张脸这么巧正对着他,他就觉得不对劲了。难道当时的修建者,算准了他会从这个位置开盗洞下来,特地摆了这么个东西在这里吓唬他?

\chapter{多了一个}

可这样一来就麻烦了,只要脚一着地,就算你眇履如烟,但是搬动这么一座小塔,在如此小的空间里,不惊动这些地蜂是不可能的。

陈皮阿四只是一个琢磨,就知道下去是不可能了,要把东西弄上来,只剩下一个办法。

在这里不得不提一下陈皮阿四这个人的来历,这个人自小在浙江沿海的渔村长大,日本人打来才逃难到了长沙,所以他一口长沙话很不“地道”,但是这人非常的聪明,自古时候起土夫子基本上不传手艺给外省人,他是难得的一个。

陈皮阿四在海盐的时候,已经有了一手绝活,那就是在滩涂上抓螃蟹,当然不是用手抓,陈皮阿四抓螃蟹的东西,叫做“九爪钩”。

这东西就是类似于武侠片里的飞虎爪,或者特种部队用来攀岩用的三钩爪子,但是这种爪子有九个钩子,成一个环行,排的很密,抓螃蟹的时候,就用绳子绑在钩子的尾巴上,然后看见螃蟹在滩涂上一冒头,就一把甩出去,一钩就是一只螃蟹。然后一扯,螃蟹就飞回来,自己掉进筐里。

据我爷的笔记上记录,这种功夫能精准到什么地步,二十米一只生鸡蛋,一甩手就能勾过来,落地不破,简直是神技。再远一点,就要用棒子甩,也是十分的准确。

陈皮阿四此时无计可施展,没有办法,只好一咬牙使出看家本事,他先荡到一边,顺着罗汉像,一层一层地爬下去。等到距离差不多了,他掏出九爪钩,提起一头气,一个角度级小的弧线。爪子就钩到了宝帐上,幸好这东西不是常见的青石的,十分轻盈,陈皮阿四一提将宝帐甩起,架到一边的罗汉脑袋上,手上力道一变,钩子脱出又回到他手里。

接下来是把这玉石或是牙塔去掉,不过无论是什么材料。用九爪钩是提不上来的,陈皮阿四甩出九爪钩,勾住袖珍的塔刹,扯了几下,纹丝不动。

没半吨也有五百斤,陈皮阿四心里暗骂。

他用手电扫了一遍塔身,看到塔基入有四根袖珍的柱子。这塔必然是按照头顶上塌掉的这一座等比复制的,那结构也应该差不多,这四根柱子支撑着塔身所有的重量。宝函就在柱子中间,只不过角度不对,不然仔细去勾,也应该能勾上来。

这时候陈皮阿四心里已经有点急噪,他估计着下来也有四个小时了,刚才隐约听到几声哨声,弄不好那帮苗人已经在附近了,没时间再犹豫想办法了。

他心里一压,脑子一热,心里恶念已起,甩手啪啪又打出两棵铁弹子,弹子打在塔基上的小柱子上,柱子应声而碎,接着他纵身一跃,一下子踩到塔的一边,然后一使缓劲,顺着自己的冲力将塔带的往一边斜倒,另两边地柱子本来就受力不平衡,一下子断裂,塔往下一沉,塔身和塔基裂了开来。

陈皮阿四趴在塔上,控制着力度,塔重力量缓,倾斜的很慢,等到陈皮阿四看到塔下的宝函一露一个角,一甩九爪勾,一下将这东西从塔下勾了上来,然后收钩子再甩出去,勾住一边的罗汉像,像拉起纤绳一样把自己稳住。

这一系列动作只有3秒就全部完成了,但是他没想到那罗汉像根本拉不住塔身和他的体重,一拉之下,罗汉像首先不稳,竟然从墙上掉了下来。

这下面一圈几乎都是蜂包,要是这样掉下去,等于直接摔进蜂包里面,那不死也不可能了。

闪电间陈皮阿四使尽全身的力气用力一扯,将罗汉像扯向自己的方向,一手将八重宝函丢向空中,如此闪电般一换手,罗汉像给他稳稳接到了手里,但是无法避免的,宝塔顶也重重撞上了地宫壁,更多的罗汉像给倾斜的塔刹拨落下来。

这一次陈皮阿四再也无技可施展,眼看着一排的罗汉像砸进地黄蜂巢里,顿时灰尘四起,黄蜂巢给压的几乎完全凹陷裂开。

混乱中他只得丢下手里罗汉,又转接住宝函,条件反射的手电去照那蜂包,心说完了,老命交代了,没死在战场上,还是死在地宫里,应了祖宗的老话了。

手电一照间,却见那些裂缝处却没有他想象的大量的黄蜂涌出来,反而他看到蜂巢的裂缝里面干涸没有一点水分,似乎是一个废弃的蜂包。

但是,让他浑身冰凉的是,有一道裂缝里面有一驮黑呼呼的东西,看样子是修巢的时候裹进去了,不知道是死人还是什么动物的尸体。

他跳下去,掰开一看,是一座和这里样式相同的罗汉像给裹在了里面,已经摔的成了几片,估计是蜂巢还没形成的时候就从上面摔下来碎了,结果给包进去。

陈皮阿四抬头看去,他刚才下来的时候虽然没注意,但是他感觉并没有发现哪里少了一尊罗汉像啊,这一座是从哪个位置上掉下来的呢?

\chapter{最初的迷题}

此时整个地宫内是极端的黑暗的,向上看去,手电光斑所照满眼都是俯视的罗汉,百双眼睛注视着陈皮阿四,罗汉的瞳孔因为光线的变化,一刹那露出狰狞的表情,气氛一下子变得十分诡异。

陈皮阿四心里又骂了几句秃驴,心说这些和尚肯定是故意的,此时他也顾不得那么多了,又找了几圈,却他旧没有发现哪里缺了一座雕像来。

他心里灵光闪动,慢慢知道了问题所在,手电也移向那座给他打裂双眼的白面望天罗汉的位置。只有这一座罗汉像明显和其他的不同,问题应该是出在这里。有可能是什么人将上面某尊罗汉推倒下来,然后将那尊脑袋向上的白面望天罗汉放了上去,所以那一尊罗汉才和其他的有如此大的不同。他妈的那底是谁那么无聊要这么干呢?而且能够准确的知道他下来的位置,将雕像的头对准他下锔的地方,不是行内人也不可能做到啊?难道自己这次是二进宫?这里已经有人来过了,还摆下这么个东西来寒掺我?

陈皮阿四的手电光照在那胖胖的白面望天罗汉身上,又掂了掂手里沉甸甸的八重宝函,如果是二进宫,干什么不把这东西带走,不可能人去不留空,肯定是自己多考虑了,这里是那些秃驴设下的圈套,好让他们这些人往歧路上想。

陈皮阿四缓下心神,一大把年纪,经过这么一折腾,已经到了极限了,他咳嗽几口,就想把手电光从那罗汉上移开,去照一下四周,看看如何回去最省力。这时候骇人的一幕发生了。

在手电光从罗汉上移开的那一刹那,陈皮阿四突然看到那张惨白的脸突然间扭了过来!

手电移的太快,这场景一下子就没了。但是陈皮阿四却看得真切,他不是那种会怀疑自己看错的人,当下就觉得脑子一炸,几乎就要坐倒在地上,闪电之间他大吼一声,给自己壮胆子,同时一翻手,把铁弹子机关枪一样甩了出去。

他凭着刚才的记忆,连发10几颗,10几颗铁弹在头顶上四处弹来弹去,他还以为是那妖怪一样的白面罗汉蹦下来了,慌乱间乱了阵脚,把早年的一把王八盒子掏了出来。他是真怕了,这枪解放后几年就从来没用过,他也不敢轻易拿出来,现在掏出来了,明知道没用也用来壮胆子,那是真的慌的找不到北了。

你说掏个几十年的沙,碰到各把粽子的机会已经少之又少,这样的场面就算我爷爷在也难以应付,陈皮阿四虽然是老手中的老手,但是主要的经验还在于和人在生死关头的较量,一碰上什么摸不着边际的事情,还是照样慌。

慌乱之中,他看到了那一边毫不起眼的矮石门,这爬上去从盗洞回去是不可能了,还是找路跑吧!

他猫腰钻进矮门,里面便是一间石室,山包一样的地黄蜂巢从墙上一直长过来,规模实在不小,这石室里原本摆着什么东西也不知道了,跑了几步,脚嵌进蜂包里,一下子整个人摔了个狗吃屎,手电飞出去老远,他也顾不得捡了,抱起那个盒子就往前冲去。

过了石室就是漫道,目测就有10几米长,尽头就是地宫的正规出入口,一片火光很微弱,出口应该是给什么堵住了,他咬着牙深一脚浅一脚的也不知道踩到了什么东西,终于地势开始向上,他又跑了十几步,头晕脑涨已经赶到火光面前,一头撞到了什么东西,只听一阵倾倒撞击的声音,他已经冲了出去,滚倒在地。

外在的火光熊熊,他站起来四处一看,自己竟然从一处断墙里撞了出来,看到隐藏的浮屠地宫入口竟然在一面墙里,正在诧异,几把苗人的苗刀已经架在了他的脖子上,同时手里的东西也给人接了过去。

陈皮阿四体力到达极限,也无法反抗,一看不给,踉跄跑了几步,给人一脚踢了后膝盖,跪倒在地上,抬头一看,那几个他骗下来的苗人小伙子举着火把围着他,为首的首领有点恼怒的看着他,看样子他们找了一圈什么也没发现,已经知道自己被骗了。

陈皮阿四知道要糟糕了,这解放初期在苗人的地盘上犯事,是要给处私刑的,这下子自己的处境极端不妙。

苗人首领看了看从陈皮阿四手里拿来的宝函,又看了看断墙里黑漆漆的暗洞,心中已然知道了怎么回事,面露厌恶的神色,给其中一个苗人做了一个遮着双目的动作,又用苗语说了几句,陈皮阿四喘的厉害,这倒不是装的,但是他为了麻痹别人,加重了自己的表现,还不停的咳嗽,看到苗人的动作,心中一凉,他在广西生活了这么多年,知道那是要挖他的眼睛。

受命的苗人点了点头,折下边上一种锋利的草叶,蹲到他面前,用苗语问他问题,陈皮阿四不停的摆手,装成自己气太急的样子。苗人看他如此疲惫,互相看了看,不知道如何是好,另几个苗人好奇他出来的地方,打起火把探头进去看。

陈皮阿四缓了几分钟,不见那妖怪一样的白面罗汉追出来,不由心生疑问,这时候他体力有一定程度的恢复,见有两个苗人上前要按他的双手,知道再不反抗就完了,一咧嘴角,突然翻出了一把铁弹,跳起来啪啪啪啪,一瞬间便把所有的火把打落在地。

苗人一下子惊惶失措,陈皮阿四冷笑一声,杀意已起,一脚踢翻面前的苗人,同时另一只手翻出王八盒子就想杀人,就在这个时候,就听边上冷风一响,自己手里一凉,一摸,扣扳机的手指头已经没了。

陈皮阿四何时吃过这样的亏,心里大骇,可没等他反应过来,接着又是一道冷风,他最后看到的就是那苗人首领淡定地眸子和他身上5动的麒麟纹身,这是他最后看到的景象了,因为下一秒他的两只眼睛已经给一刀划瞎,苗人首领的土刀自左眼间横劈进去,划断鼻梁骨头,横刀过右眼而出,两只眼睛一下子就报废了。

老海说,“那几个苗人总算没杀了他,他们将陈皮阿四和那宝函交给当地的联防队,他一个起义的战友那几年正好在那里负责联防,把他保了下来,他才没给枪毙,不过眼睛就此瞎了,后来那宝函给送到了博物馆,那里人一听,就派人去现场看了,也不知道有没有结果,不过那宝函开启一看,最后一层却不是什么舍利,而是这条铜鱼。”他敲了敲报纸,“怪不怪,这在当时是天打雷劈的事还必须,那陈四爷知道后,破口大骂,说自己给人耍了,这宝函可能早在几代前已给人打开过,里面的东西给掉包了。”

我此时听老海讲故事,已经不知不觉喝了盅酒下去,人有点飘,问道:“他有什么根据?”老海一边吸了口螺蛳一边说:“我不知道,陈皮阿四后来当了和尚了,在广西挂单,这些事情我可是托了老关系才打听来的,小哥,这消息不便宜啊,以后你有啥好处也别忘了便宜我了。”

我暗骂了一声,心说就知道这老家伙没这么好心,看来也就是想和我笼络一下关系,当下见他没其他消息了,又问他这次来杭州那个拍卖会是怎么回事。

老海把最后一只螺蛳解决,砸吧咂吧嘴巴说道:“当年乱七八糟的,这条鱼也不知道流落到什么地方去了,这不,今儿个竟然有人拿出来拍卖了,我参加拍卖会是常事,在业内有点名气,他们就给我发了本手册和请贴,你看,这鱼在拍卖品名单上呢,我看着你对这鱼也挺有兴趣,就顺便给你弄了张请贴,甭管有用没用,去看看谁想买这鱼,也是件好事情。”我一看起拍价格,1000万,神经病才会去买呢,我手上还有两条,要是有人买,我不是有2000万,现在的拍卖行自我炒作也太利害了,也要别人相信才行啊。

老海的消息虽然不错,但是并不是我想知道的那些,一时无话,我们各自点上一只烟,各自想着各自的事情,服务员看我们赖着不走,想上来收盘子,我只好又寒暄的问了问老海地生意咋样,老海说起他也想跟我怎么去见识一下这种话,也看不出是不是真心的,我说还是免了,我自己都不打算再下地,你一把老骨头就别搀和了,免得拖累了自己又拖累我。

我酒也喝得差不多了,问他拿了请贴,就让他先休息,晚上,秦海婷吵着要出去玩,我是地主,不好推辞,就开车带他们四处跑了一下,吃了点小吃,不过天气实在太冷,他们也就早早的回去睡觉了。

我开车到家里,没上楼,忽然觉得家徒四壁很凄凉,以前一直都没有这种感觉,觉得很奇怪,难道这几次经历让我沧桑了?想着自己也觉得好笑,于是开车径直到二叔开的茶馆,跑去喝晚茶。

在茶馆里一边喝一边看爷爷的笔记,一边想着发生的事情,只觉得还是一头雾水,主要的问题,是这三条鱼不在同一个朝代啊,而且地理位置差这么远,暂且不管这三条鱼的用处,就是它们发掘的地方,也丝毫没有一点可以让人猜测的头绪。

古人做这一件事情,必然会有目的,不然这阵仗太大了,不是一般人能玩得起的,我左思右想,觉得关键还是不知道他的目的是什么,只要知道了目的,查起来也有方向得多。

如果爷爷还活着就好了,我叹了口气,或者三叔在,至少也有个商量的人,现在一个人,这些问题我真的想得有点厌烦起来了。忽然闻到一股焦臭,低头一看,借阅的杂志里有一张中国的旅游地图,我边想边用香烟在上面比画,下意识的把那三个地方都烫出了一个洞,等我反应过来已经晚了,我赶紧把烟头掐了,看了看四周,服务员没注意到我搞破坏,不由松了口气。二叔虽然是我的亲戚,但是为人很乖张,弄坏了他的东西,他是要翻脸的,特别是这里的杂志,每一本都很珍贵,是他的收藏品,弄坏了更是要给他说几年都不止。我装成什么都没有发生的样子,将杂志还了回去,刚放下,就有一个老头子拿了过去,站在那里翻起来,我担心他发现我搞破坏,没敢走远,落到一边的沙发上,看那老头子一翻便翻到我烫坏的那一页,一看,不由嗯了一声。我一听糟了,被他发现了,正准备开溜,就听他轻声笑道:“谁给烫出了个风水局在这里,真缺德。”

\chapter{简单答案}

这老头子讲话的声音清晰,带着长沙那边的腔调,加上他说话的内容,引得我一奇。偷偷打量这老头,相貌很陌生,大概70多岁,干瘦干瘦的,身材不高,眉宇间有一丝阴靡,穿着有点皱的老旧棉袄,超级啤酒瓶底似的老花眼镜,估计拿了就是半瞎子。这样的打扮不像是这里的客人,不过二叔的茶馆里能人很多,所以服务员也不见怪,这年头,什么人都有。

我不动声色,看他有何举动,只见他拿起那本书,背着手就回到他的座位上,腰板挺的很直,步履生风,如果不是个练家子,以前必然当过兵。他的座位上还有几个人,都上了年纪,正在聊天,一看到老头回来,都露出恭敬的神色,显然这家伙是头。我偷偷把自己的茶端了过去,坐到他们身后的位置上,耳朵竖立起来,听那老头会说什么。刚开始那几个老头聊了会股票,我听着很不是味道,半个小时后,那老头才想到自己拿了杂志了,只听那老头说:“对了,来来来,让你们看件有趣的事情。”

说着,他展开那本杂志,翻到我烫坏的那一页,我一听有门啊,这家伙可能真知道什么,连大气也不敢出,听那老头又道:“你们来看看,这张地图有啥特别的,考考你们。”老头子们看来看去,唧唧喳喳说了一堆,你说一张被香烟烫了个洞的地图有啥特别的啊,那几个老头还真能扯,有几个还扯到什么三足鼎立上去,为首的那老头摇头,通通不对。

我听得肠子都痒了,心里盼着快公布答案,我投降了还不成吗。见没人能说上来,那老头呵呵一笑,忽然压低了声音,说了一句我听不懂的话,另几个人马上激动起来,都要抢着看那本杂志。

我一下子心里郁闷,没事情你说什么方言啊,难道该的我就是没缘分知道这事情?老头们看了很久,都发出恍然大悟的声音,我心里急得几乎烧起来,盼着他们有讨论一下,让我也知道点细节,按我的能力,知道一些就应该能推个大概了。

没想到的是,接下来,这帮人所有的对话,全部都用起了那种奇怪的语言,我仔细听了很久,只能确定不是汉语的方言,他妈的那几个老头到底是哪里来的人?

我听了很久,实在听不下去了,脑子也热起来,心说你不让我听懂是吧,我他妈的自己去问你们,总奈何不了我了吧,把心一横,站起来走到他们一边,装成好学少年的样子,问道:“几位老爷子哪里人呢,怎么我觉得这话听起来这么怪呢?”这在杭州是十分唐突的,不像在北京,茶馆四合院子大家多少都认识,我这话一出就后悔了,心说该不会给我眼色看吧。

没想到那几个老头子都楞了楞,大笑起来,其中拿了书的那个道:“小娃子,你听不懂是正常的,这是老苗话,全国加起来能说的不超过千号人了。”我惊讶道:“那几位都是苗人?怎么看着不像啊。”

老头子们又哄堂大笑,也不回答我,我看这几个人都健谈,不是这一带人,搞不好能问出什么来,忙顺着势头问道:“几位别笑啊,刚听这位老太爷说,什么风水局,这地图是给我烫的,难不成还烫出了啥噱头不成?”为首那老头子打量了我一下,说道:“小伙子也对风水感兴趣?这学问你可懂不了啊。”

“能懂!能懂!”我恨不得去舔他的脚让他快说出来,“要不您说说,让我也开开眼?”那老头和其他几个相视一笑,说道:“其实也没什么,你看,你烫出的三个点,位置都很特别,把他们连起来,然后横过来看,你看到什么?”我拿起杂志,一看这下,忽然浑身发凉,“这是!”我张大嘴巴。

原来,祁蒙山西周陵,广西地卧佛岭浮屠地宫和西沙的海底墓,三条鱼出土的地方,由曲线贴着中国海岸线连起来,形状非常熟悉,仔细一看,那赫然是一条若隐若现的龙形脉络!我恨不得抽自己一个巴掌,心说吴邪,你咋就这么笨呢!也不会在地图上比画比画,只顾着这几个地方的朝代不同了,咋就没想过位置的关系呢。

那老头子看到我吃惊,知道我已经看出端倪,颇有几分赞赏地感觉,说道:“是条不太明显的‘出水龙’,说得好听点,叫做潜龙出海,不过,这一局还少了一点,缺了个龙头。”说着,他拿起自己地香烟,朝杂志上一点,正点在长白山的位置上。

杂志滋滋冒烟,我却一点也反应不过来,楞了片刻,忙问他:“这~这个,大师,这局有什么用意吗?”老头子呵呵一笑,“你看,这叫横看成峰竖成岭,你看这几个点,连着长白山脉,秦岭,祁蒙山系,昆仑山脉入地的地方,这叫做千龙压尾,中国的几条龙脉在地下都是连着的,这整合着看风水,整个一条线上聚气藏机的地方自然多不胜数,你烫下的这几个点,都是很关键的宝眼,因为这一条线一头在水里,一头在岸上,所以叫做出水龙。不过这种大头风水是不是实用,用这种风水看出来的龙脉,比较抽象,我们叫大头龙,古时候用来占卜看天下运势,北京城的位置,都是靠这个确定的,而给皇帝选陵,这风水就太大了,我也是只懂点皮毛,要说大师,还属明初时候的那个汪藏海,大头风水是他的拿手好戏啊。”

我听到这里,眼睛一花,直觉得七窍都通了,所有想不通的事情,全部都从脑子里涌了上来,为什么鲁王宫外五坟岭尸洞内的六角铃铛会出现在海底墓里,为什么西周墓里会有如此精巧的迷宫盒子,为什么广西浮屠“镜儿宫”里的佛骨舍利会变成蛇眉铜鱼,理由太简单了,因为这些个地方,汪藏海全去过了。

出水龙的宝眼处一般都是当条龙脉的藏风聚气之地,一般都已经修筑了建筑或陵墓,虽然现在还不知道把铜鱼放在这些宝眼处是什么用意,但是按风水学的一般惯例,这一条风水线大龙头,是为了长白山上的龙头而设。这一切布置都是为了云顶天宫,难怪他会如此着迷,他花了如此巨大的心血。

那这雪层下的天宫里,到底埋着的是谁?

老头子看我出神,大概也不知道我在想什么,就招呼其他几个起身,将杂志塞进我手里,就招手结账。我想着事情一下没反应过来,等我想起要他联系信息,他已经走出了茶馆,我追出去,正看到他把眼镜一摘,我一看他的眼睛的五官轮廓,咯噔一声,人不由站住了。

只见一道极其可怕的伤疤从他的眼角开始,划过鼻子,一直到另一边的眼角,鼻梁骨有一处凹陷,似乎给什么利器割伤过。我看到他的眼睛,人又给吓了一下,忘记去追,结果他们一群人上车走了。我转念一想,感觉这老头子谈吐不凡,而且中气十足,很可能是老海今天说的,陈皮阿四!

刚才吃饭刚谈到他,咋现在就在茶馆碰到了,这也太巧了吧。我想了想,猛然觉得老海莫名其妙的来杭州和我说起故事有点唐突,难不成这老头子和老海有什么猫腻在?布了这么个套想引我入局?这老头看上去有一点狡狯,不可不防啊。

我心里暗骂,又不知道这一套戏扯的什么路子,心里顿生疑惑,回忆起老海的叙述,这老头子不是已经瞎了吗?怎么还能看得见呢,而且说话中气十足,也不像90岁的人。不过想通了大头潜龙的局,心里舒服了很多,那股阴靡的感觉也不扫而空,我结了账,回去舒舒服服结结实实的睡了一觉。

醒来是第二天中午,一看请贴,娘的,已经结束了,打电话给老海,他也没什么说的,只说那条鱼没什么人拍,我心里大乐,傻B才去买这东西呢,又交待了几句,听老海那边好像很忙,看样了买下了不少东西,就不和他罗嗦了。下午也不想去铺子,想去茶馆去待那人,三叔的店里却打来电话,说有人找我。

我心里说该不是老痒又出现了吧,七上八下的开车过去,走进店里一看,只见一个人坐在客座沙发上,我几乎眼睛一酸,眼泪差点下来,立即大叫了起来,“潘子!”

\chapter{潘子}

我和潘子在三叔的铺子里坐了一个下午,互相讲了一些自己的情况。

原来潘子在我去海南之前已经有一点恢复意识,但是当时我走的太急,只给医院留了一个手机,我出海后自然找不到我。

潘子的体质很好,恢复的很快,就算这样他还是在床上躺了将近一个月,等他能够下地来找我们,却一个也联系不到。算起来那个时候我应该是在陕西,而三叔就更不用说了,全世界都在找他。

我看到潘子臂上带着黑纱,就问他干什么?他说大奎一场兄弟,头七没赶上,现在带一下心里也舒服一点,我给他一提,想起去山东那段日子,心里也唏嘘起来,说到底,那件事情还是因我而起,如果当时不去多这个事情,将帛书给三叔看,各人现在的近况自然大不相同。

潘子看我脸色变化,猜到我在想什么,拍了我一下道:“小三爷,我们这一行,这该来的逃不了,怪不得别人。”

我叹了口气,心说你说的简单,打死大奎的又不是你。

唏嘘了一阵,我又把我这一边最近的一些情况和潘子说了,听的他眉头直皱,听到后来我们的猜测,他面色一变,摇着头说他和三叔这么多年下来,他能肯定三叔绝对不是那种人,叫我别听别人乱讲。

潘子跟随三叔多年,感情深厚,有些话自然听不进去,我不再说什么,转移话题,问他有什么打算。

潘子想了想。说本来他打算还是回长沙继续混饭吃,那里三叔的生意都还在,人他都认识,回去不怕没事情做。现在听我这么一说,他觉得这事情不简单,恐怕得再查查才能安心。

我点点头,虽然这里我基本上都查过了,但是潘子和三叔的关系不一般,有很多我不知道的关系在里面,他能去查查是最好不过。

潘子打了好几个电话,对方都让他等消息,我以为要等个十天八天的,没想到才五分钟就都回了电话。潘子听完之后,皱着眉头对我说道:“小三爷,恐怕你得跟我走一趟了。”

我一楞,心说怎么回事情,该不会是出事情了。

潘子接着道:“三爷在长沙找一个人,给你留了话。不过得亲自和你讲,那一边的人叫我带你过去。”

“三叔留了话给我?”我几乎跳了起来,长沙那边我也不是没联络过。怎么从来没人和我提起这个事情?

潘子表情非常严肃,也没想给我解释,对我道:“那边很急,您看怎么样,什么时候能够出发?”

潘子非常急,我隐约觉得事情不简单,但是我也没想到他会急成这样,结果当天晚上我就上了去长沙的绿皮火车,什么都没交代。

上了火车之后。我还问潘子,要是急干啥不坐飞机,还坐个火车,这不是笑话吗?

潘子魂不守舍的,只拍了拍我说等一下就知道了。我看他脑门上都冒了汗了,越发觉得奇怪。心说他到底在紧张什么。

火车从杭州出发,先到了杭州的另一个火车站。三个小时后到达金华站前,此时我已经有点忍耐不住要问个究竟了,这时候,火车突然临时停车了。

绿皮车临时停车是常有的事情,当时在买票的时候我想这么远的距离,你不坐飞机至少也要坐个特快,干什么要买绿皮的硬坐啊,可是潘子的心思根本不在这个上面,现在车一停,我心里幸哉乐祸呢——你急是吧,临时停车,急死你!

没想到车才一停,潘子就拍了一下,示意我跟上,我站起来想问他去哪里,结果他突然一个打滚,从车窗跳了出去。

我一看我操这是干什么啊,车里的人一看也都吓了一跳,都站起来看,潘子在外面大叫:“小三爷你还等什么,快下来!”

我看了看四周,所有人都站起来看着我,心说这下子明天要上都市快报头条了,一咬牙也滚了出去。

绿皮很高,我下来翻了个跟头,摔进一边的路枕上,潘子一把把我扶起来,就拉着我跑。

一直跑进边上的田野里,上了个田埂,然后翻上大道,那里竟然已经有了一辆皮卡在等我们,潘子拉我进了皮卡,车子马上发动。

我累的上气不接下气,等车开上省道,才缓过来,骂道:“你他妈的搞什么飞机。”

潘子也累的够戗,看我的样子,笑道:“别生气,我是第一次这么狼狈,娘的也不知道什么时候招惹上的,不知道能不能甩掉。”

说着他看了看车后面,一片漆黑,似乎没人追来。

我没听明白,看样子这些事情他都计划过了,忙问他怎么回事情,他点上一只烟,用长沙话道:“车上那哈有警调子,三爷他不在,长沙那哈乌焦巴功,地里的帮老倌晨出了鬼老二咧。”

这话的意思是火车上有警察,我三叔不在长沙,长沙那边的生意乱七八遭了,有做活儿的帮工里可能有警察的人了。

他说话的时候眼睛瞟了瞟开车的人,我意识到这司机可能是临时找来的,不能透露太多,也就不在问了,心里却打翻了五味瓶一样,心说那我现在算什么,我不是成逃犯了啊。

我的爷爷,当年到底怎么回事?早几个月我还是小商贩,突然变盗墓贼和粽子搞外交就不说了,现在又成逃犯了,人生真是太刺激性……

车开到金华边上一个小县城里,我们下了车付了钱。潘子带我去随便买了几件比较旧款式的小一号的西装换上,一照镜子,比较寒酸,然后又赶到火车站。买了我们刚才跳下来那辆车的票,那车临时停车到现在才到这个站。

我们重新上车,这次买了卧铺,潘子看了车厢,明显放松下来,说道:“刚才那些警调子应该在金华站就下了,现在高速公路省道两头都有卡,他们绝对想不到我们会重新上火车。”

我第一次做逃犯,手脚都不知道怎么放,几乎紧张的发抖。轻声问道:“到底怎么回事情,怎么我们就给警察盯上了?我可没干……哦不对,应该说我干的那些事情一般人发现不了啊?”

“我也不知道,”潘子说道,“下午我给长沙我们的地下钱庄电话,结果那老板一听是我的声音,只说了两句话。一是让我马上把你带去长沙,三叔有话留,二是长沙出了状况。叫我们小心警调子,然后就挂了,这老板是三叔三十年的合作伙伴,绝对靠的牢,我想了一下,杭州我不熟悉,呆久了会出事情,怎么样也先回长沙再说。”

他看我担心,又道:“我上了车之后马上就发现几个便衣。就联系了个朋友,叫了辆车,让他尽量跟着铁轨走,刚才临时停车,我看到司机给我们打信号就知道机会来了,所以才拖着你下来,看你那司机一路上一句话也没说。就也是咱们道上混的,在这种人面前你不能说太多。不过这些个条子没抓我们,说明我们和长沙的事情关系不大,肯定是长沙那里有大头的给逮住了,咱们这些小虾米都是萝卜带出的泥,你也不用太害怕,和你做的那些事情无关,最多就是一个消赃。”

我听了稍微舒服一点,刚想说谢天谢地,没想到他又道:“长沙一但出事情,千丝万缕的,三爷肯定脱不了关系,那老板也不说清楚,他娘的也不知道到底是什么事情,其实我们这几年已经很收敛了,几乎都没怎么直接下地,以前的事情也不可能给翻的这么大,真是想不明白。”

“那你现在怎么打算?”我试探着问,我可不想亡命天涯啊。

潘子道:“我们不能直接去长沙,出了浙江我们就下车,然后长途大巴到长沙边上的山里,三爷在外面有几个收古董的点,那里有人接头,那钱庄老板到时候会过来。”

我点点头,这时候车又到了一个站,开始上客,我们那卧铺间里又来了一个人,潘子打了个眼色,我马上转移话题。

聊着聊着,我不知不觉就说到了陈皮阿四的事情,这人的名气在长沙倒是很响,潘子还听说过他,对我说道:“这人在我们那里也有自己的生意,听说他瞎了以后就不在自己做活了,文化大革命结束后收了几个徒弟倒卖古董给外国人,这人很阴,他几个最先跟他的徒弟几乎都已经给枪毙了,他还逍遥在外,传言很多,最好和他保持距离。”

我想起陈皮阿四的样子,不像瞎子,觉得越发奇怪起来。

我们按照潘子的计划,几经波折,来到长沙附近福寿山一带,那里果然好地方,沿途风景迷人,潘子长年在这一带活动,倒也习惯了,我们来到镇上一处杂货市场,好象旧社会地下党接头一样,东拐西勾的,来到一处一看就知道不会有生意的铺子里,铺了里外面卖的是旧电脑,里面推开后墙,就是一小间,再往里面豁然开朗,是两间铺面之间背靠背留出的一道建筑缝隙,大概能容纳两个人并排的走,现在上面拉起了雨布,里面两边一排架子,上面全是刚出士的明器。

有向个人正在那里挑货,负责人认识潘子,看见他过来,放下手里的东西,以潘子道:“怎么才到?基本的东西都备好了,你们什么时候走?”

“东西?什么东西?”潘子楞了一下,一脸迷惑。

那人也楞了一下:“你不知道?”

潘子回头看了看我,我瞪了他一眼,心说你的地盘你看我干什么?他转头道:“准备什么?”

那人道:“三爷吩咐的,五人装备,做活儿啊?你不知道?”

\chapter{新的团伙}

潘子皱起眉头道:“我怎么不知道?三爷回来过了?什么时候吩咐的?”

那人看我们两个的样子,还以为潘子拿他开涮,耸了耸肩笑道:“少跟我装八咪子喃(装傻)东西是给你的哈,你能不晓得?”

潘子火了,骂了一声:“我骗你做啥子?三爷怎么说的,啥时候说的?”

那人一看我们两个样子,才知道我们真不知道,也觉得奇怪,说道:“具体我也不清楚,我也是听钱庄的楚老板交代的,他就在后头,你们去问他吧。”

潘子闷哼一声,带着我穿过这条窄道,尽头还有道铁门,没锁,一推打开,里面是一个简陋的办公室,一边的客坐沙发上,我看到里面有个光头的油光满面的中年人正在抽烟。看到我们进来,把烟头往地上一扔,踩熄了站了起来。

潘子打了声招呼,“楚哥。”态度一下子变的恭敬起来,我马上意思到这个人就是为三叔带话给我的人。

他看了看潘子又看了看我,说道:“怎么现在才到,等你们两天了。”

潘子把路上的事情和他说了,不等他反映,急着问他道:“楚哥,到底出了什么事情,我们哪里招惹号子里的人了?”

楚哥不紧不慢,说道:“先别慌,没出事,这是你三爷的意思,他让我把他前几年做地一些买卖的消息放出去的。给号子里来点刺激的,现在厅里已经立专案组侦察了。我也不知道他是什么用意,不过看样子他是在给另一批人设置障碍。”

“另一批人?”

“对,因为这一招,现在整个古董市场都受了牵连。凡是和你三叔有生意关系的人全部都给监控了,这样一来,没提前做准备的人,现在就很难开展活动。你三叔在给你们争取时间。”

我看了看潘子,并不是很听的懂这光头说的话,“什么时间?”

光头耸了耸肩膀,表示他也不知道。“你三叔是老江湖了,他的套路我是猜不透的。”

潘子问他道:“那刚才听外面的九四说,什么装备准备好了,说是您安排的,这又是怎么回事情?”

楚哥道:“刚才说了,只要我一把消息放出去,凡是做这一行的人,无论什么活动都很难开展。所以你三爷让我在放消息前,把我们该做的事情都做了,所以我提早去买了装备,要是现在去,世面上没人敢出手,连铲头都买不到一支。”

我问道:“可是买这些装备干什么呢?我们又没打算做活?”

楚哥道:“这就是你三叔给你带话的原因,”他让我们坐下:“其实你三叔的那些装备,刚开始没算上你的份,也就是说,他准备了五份装备,其中一份是留给他自己的。”

他顿了顿,又道:“不过他当时打电话给我的时候也说了,他做的事情,并不是只有他一个人在做,还有人在和他‘枪胡’,对方也不是省油的灯,所以如果他回不来,这份装备就给你用,无论如何,你要把他的事情继续下去,不能让另一批人登先。”

另一批人?我忽然想到了阿宁所属的那个公司,难道三叔在海斗里摆了他们一道,就是因为这个原因?

潘子问道:“三爷有没有说另一批人是什么人?”

光头摇头道:“没有,不过我想现在三爷有可能是已经落在他们的手上了,不然他早应该出现了,可惜我们现在什么都不知道,不然我想对方来头再大,我们也不至于摆不平。”

我心里哎呀了一声,那光头又道:“你们要去的那个地方,是吉林长白山的横山山脉,具体地方只有用坐标来标,不过我已经准备了当地向导带你们过去。”

长白山的话,我们现阶段所有的记忆和长白山有关的,只有汪藏海的云顶天宫,毫无疑问,横山山脉的某处,应该就是云顶天宫的所在。

只是,我为什么要到那种地方去?没有任何理由,我就要到这种莫名其妙的地方去,而且还是冬天?

光头看我的脸色已经变的绿色,突然叹了口气,说道:“说实话我也很迷惑,不过我自己也仔细想过,唯今之计,你们唯一能做的,是跟着你三爷准备好的计划走下去,才能找到线索。不然,我估计你三叔恐怕过不能这一关。”

潘子拍了拍我,转头继续问道:“那,三爷计划里,下一步我们应该怎么样?”

光头道:“你们一共五个人,先上火车去吉林,行李我们会通过办法托到那边的,基本上都到了。”

我和潘子对视了一眼,吉林,那看样子真要去爬雪山不可耻下场。

光头说他会负责我们全程的所有细节,所以我们不用担心,只管上路,只要小心路上别给警察盯上就行了,时间安排的很紧,在长沙休息一晚,明天就直接送我们上火车,车票连洗漱用品都全部打包准备好了。所有的细节问题,另三个人都知道了,有问题只要明天问他们就行了。

这个光头行政能力之强出乎人的意料,三叔托他来传话,这一次计划,恐怕计划了很长时间。不知道他的目的到底是什么?

我们又问了些问题,光头也是只知道其一,不知道其二,不过听他的口气,三叔的安排真是天衣无缝,这一次老江湖总算是显现出功力来了。

我们原路出来,我看到铺子外面运来了很多二手电脑的显示器,潘子告诉我,明器就是藏在里面运输的,一般关卡检查,这样的包装是查不出来的。那光头说的运我们的装备去吉林,应该就是通过这个方式。

潘子是这里的地头蛇,傍晚我跟他去吃了长沙的饺饵,我我来长沙不是一次两次了,也不觉得新鲜,我们一边吃一边讨论今天光头给我们传的消息,潘子想了半天,对我道:“小三爷,我思前想后,总觉得你和我说的,去西沙给你们准备的那个什么什么资源公司有可疑,他娘的你说三爷说的那一批人会不会就是他们?”

我道:“这我早就想到了,不过我觉得问题不在那个公司,而在于公司背后的人,咱们也别想,反正到了那边我们不去找他们,他们也会找上门来。只是,那个楚哥靠不靠的住?”

潘子说道:“小三爷,你别看我潘子当兵的,看人准的很,这人你绝对放心,我就是担心,那人说一起去的有五个人,其他三个是什么货色。”

我说道:“三叔安排的总不会错。”

潘子摇头道:“难说,三爷常说看人要365天的看,少看一天都不行,人是会变的,你一个星期不见他,说不定他已经想着要害你了,特别是我们这一行里那些没文化的,说的不好听点,他娘的那个手里没几条人命债,心横横,老娘都能埋到土里。三爷这么久没回来,这里的伙计,人心肯定起变。”

我说你要求太高也不行,咱们走一步是一步吧。

街上晚上冷起来,吃完后二话不说就回潘子以前住的房子里睡了,早上起来吃了早饭,光头的车就来接我们,我远远背起自己的贴身行李,看了看车里,发现座位上已经坐了个人了。

仔细一看,发现是个老头,人很面熟起来,好象哪里见过,而且还是不久前。

我并不在意,和潘子开着玩笑走过去,靠近一看,突然人蒙了。

那老头,看身形和那身古怪的装扮,不是别人,竟然是在杭州二叔茶寥里看到的陈皮阿四!

我张大嘴巴几乎脖子僵硬,心说他坐在车里干什么?难不成这老头子也是五个人的一个?总不会这么离谱吧?

潘子这时候也看到了,嘟囔了一声,也是一脸的诧异。

光头招呼我们快点,我们一头雾水的上了车,潘子认识陈皮阿四,给他打了个招呼,那老头闭目养神,只是略微点了点头,潘子马上转向开车的光头,呲着牙用嘴形问他怎回事情?

光头无奈的一笑,用嘴形回道他也不知道,三爷就是这样安排的。

不会吧,我心说,这老家伙不是个瞎子吗,而且年龄加起来比我和潘子加起来还大,三叔这是玩什么花样?

我们在忐忑不安中来到火车站,我心里在盘算,三叔给我们安排工作的第一个人是一个近100百岁的老头,那第二个人是什么货色就真不好估计了,难保不会是个大肚子的孕妇或者坐轮椅的残疾人。

难道三叔想试探我们的爱心吗?

庆幸的是陈皮阿四身体很硬朗,背着手就下了车,光头对他很尊敬,帮他提着行李,我们为了便于应变,还是选了比较差的卧铺,一个房间可以睡六个人,正好一个床可以放行李。

我们来到自己的房间,我探头往里看了看,先看见一个胖子在吃方便面,看到我,一扬眉毛,诧异道:“他娘的,又是你?”

我顿时头疼起来,心里一个咯噔,心说三叔怎么找了他,难不成还是以前那只的队伍吗?马上转向胖子的上铺,果然,一双淡然的一点波澜也没有的眼睛正看着我。

我松了口气,闷油瓶眯起眼睛看了看我,又转过去睡着了。

\chapter{九龙抬尸}

光头给我们的计划是走旅游路线,从长沙先到山海关,然后转车到敦化,全程火车,整个旅程大约两天时间,经过近3000公里。在这段时间里,我们无事可做,只能通过一只手机和几本杂志打发时间。

我把那鱼眼珠的支票带给了胖子,他看到我还是很开心的。看他心情不错,我就偷偷问他,怎么会到这里来?

胖子和我说,这道上,有些事情非扎堆做不可。比如说有些深山老林里的大斗,你一个人绝办不掉,一来太多必要的装备你一个人背不进去,二来好东西太多你一个人也带不出来。这种古墓一般环境极端险恶,你能走运活着打一个来回也不错了,再要两三次的冒险进去,恐怕谁都不愿意。所以,一有这种情况,就会有一个人出来牵头,古时候叫“捉斗”,民国时候的行话叫“夹喇嘛”。

这东西就好比现在的包工头,手里有项目,自己找水电工来做,解放初期的考古队也用类似的招数来找能人异士。

这一次“夹喇嘛”的是光头。那光头人脉很广,认识胖子一个北京的土瓢子朋友,而胖子很多路子都是他那土瓢子朋友给搭上,这一来二去,胖子就上了这车了。至于具体的情况,一般的常例子,不到目的地“夹喇嘛”的人是不是透露的,不然给别人提前知道了,有可能引起内杠。所以我问起胖子我三叔的事情,胖子头直摇,说:“奶奶个熊,你还问我,你胖爷我要知道这事情又和你那狗屁三叔有关系,再多票子我也不来干。”

我心里叹了口气,心说那闷油瓶必然也是光头联系的,估计也问不出什么来。这里了解情况最多的,除了我和潘子,要么就还有个陈皮阿四。

闷油瓶一如既往的闷,也不和我打招呼,一直就在那里打瞌睡。我想故作殷勤的和他叙叙旧,说了几句发现他根本没在听。胖子让我别废力气了,说他上车来后一直都在睡觉。

车开了以后,我和胖子和潘子一起锄大D,打跑的快消磨时间。我一边打着一边琢磨着陈皮阿四,这老头上了车后就一直没讲过话,潘子跟他套近乎他也只是嗯一声,车一开就自顾自走了出去,到现在还没回来,胖子还低声问我,“这瘦老头是谁啊,拽得二五八万似的。”

潘子轻声和胖子讲了一下陈皮阿四的事情,胖子听到他九十多岁了,脸都绿了,说道:“你可别告诉我这老家伙也得跟我们上山,要真这样,到没人的地方我先把他给人道毁灭,谁也别拦我,反正他进去了横竖是一死。”

潘子赶紧压住他的嘴巴,轻声道:“你他娘的少说几句,老家伙精得很,给他听到了没到地方就把你害了。”

我回忆了一下陈皮阿四,在茶馆里他给我的感觉就像一个高深莫测的国学大师,旁边一群人围着,以这种人的势力和造诣,再加上这么大年纪,怎么会一个人来“夹喇嘛”?不怕给我们害了吗?

和潘子一提,潘子笑道:“这你就不懂了,咱们现在都是三爷夹来的喇嘛,不管是小沙弥还是方丈,现在都给三爷夹着呢,这是江湖规矩。他要分这杯羹就得按规矩来,他来头再大都没办法。”他想了想,又道:“不过他娘的我们是得小心着这老头,表面上他是一个人,其实他这样的人,肯定有安排自己的人在四周。”

胖子听了骂道:“老子就搞不懂了,你那三爷整这么个人出来干什么?这不存心添乱吗?要这人真这么邪,我看着咱们得先下手为强,要么绑了要么做了。”

潘子看了看门口,说道:“我警告你别乱来啊。三爷提这个人来肯定有用意,咱们就买他的面子,反正他这么一把年纪了,年轻时候再厉害也没用,到时候要真——哎呀!”

他话还没说完,闷油瓶子的手突然从上铺垂了下来,一把捏住了潘子的肩膀,力气极大,几乎把他捏得叫起来。

潘子给他弄得呲牙,后半句话就没说出来。我们都愣了一下,潘子对闷油瓶子没什么好感,刚想说话,门嘎吱一声,陈皮阿四走了进来。

我们互相看了一眼,忙低头继续玩牌,就好象读书的时候考试作弊被都是察觉一样。

老头子看了看我们,也不说话,回到自己的床上,也不知道是不是睡着了。

他在这里,我们也不敢商量事情,只好集中精神打牌。就这样时间一点一点的过去,第二天晚上将近零点,我们的车停靠在了山海关。

山海关是天下第一关,不过是人造景点,大部分是1986年重修的。我们要转的下一班车还有两个小时才到,胖子说要不要去看看,我说都凌晨了,又没月亮,看个鸟啊。于是我们几个人跟着同样转车的一大批天南地北的人走向车站候车室。

现在正是春运前夕,人已经很多了,车站里面气味难闻,各种各样过夜的人都有,有的还卷铺盖睡在地上。我们小心翼翼的顺着人流进去,生怕踩到别人。

人很多,走的极乱,一会儿工夫我们几个人就给冲开了。闷油瓶和陈皮阿四给冲到离我们很远的地方,胖子给几个人踩了脚,在那里直骂。我想招呼他们别走散了,举手让他们看我的位置,潘子一把拉住了我的手,将我拉的蹲下腰去。

我心中奇怪,就听他道:“有警调子!悠着点。”

我一听赶紧顺着势头坐到一边的地上,左右都是人。我用眼角的余光一看,大门口,几个穿制服的警察和几个协警正在查身份证。

我低头对潘子用杭州话轻声说:“没事吧,杭州也经常有,查身份证而已。我们也没带装备在身上,又没被通缉,怕什么?”

潘子用下巴指了指人群中很不起眼的几个男人,说道:“门口的是看门的,便衣在人堆里,在找人呢。把头低下,别给认出来。”

我抬起头闪电般一看,那几个人之间好像还夹着个面熟的人,那人还直往我们那方向张望。我还想看的仔细点,那人已经猛的挣起来,指着我大叫:“那里!”

我看到那人的手上还带着手铐,心里咯噔了一下,再仔细一看那人,我靠,那不是楚光头吗?怎么两天不见,已经给拷进去了!

“妈拉个X!”潘子大骂,拉起我跳起来就跑,后面一帮便衣猛的冲过来,大叫:“站住!”

我们连滚带爬地翻过好几排座位,用力推开人群。潘子一路过去,人全部都纷纷让开,可是我一过去,那些人都围过来。我心里大叫,这叫什么事,我看着这么好欺负吗?

眼看着前面的把我堵住,后面的警察也到了,突然啪一声,候车室大厅头顶上的一盏日光灯碎了,所有人都吓了一跳,紧接着,啪一声又是一盏。我乘机猫腰从两个人之间钻了过去,在人堆里挤来挤去,想挤到门口的位置。

忽然,一个人就抓住了我,将我拉到一边。我一看,是潘子,他一甩头,意思是,咱们从铁轨那里出去。

头上的日光灯,啪啪啪啪连续碎掉,候车厅越来越暗,破碎的玻璃直掉下来。一下子吵闹声,小孩子的哭声,惊叫声乱成一团,很多人都往进口处挤,我们顺着人流又挤了出去。

我远远看见胖子朝我们打手势,朝他靠了过去,刚想问闷油瓶呢,那家伙突然幽灵一样冒了出来。胖子问潘子:“你那‘夹喇嘛’的筷子给雷子折了,现在怎么办?”

潘子骂了一声:“那个龟儿子,这么容易就把我们抖出来了,现在人真他妈靠不住,要有机会,我敲死他去!”

胖子道:“你现在起什么劲,你得说怎么办啊?”

潘子挠着头,他也不知道怎么办好了,又来看我。我刚想骂他,闷油瓶一拍我们的肩膀,说道:“跟着老头。”

我们顺着他的目光看去,陈皮阿四正在不远处看着我们,旁边还站着几个不知道哪里冒出来的中年人。

闷油瓶径直朝他走了过去,我们这时候也没办法商量,只好硬着头皮跟他走过去。陈皮阿四看到我们走过来了,给旁边几个人打了个手势。那几个人一下子就散开在了人群里,他自己也一转头往人群中走去。

我们在人群的掩护下,终于逃出了山海关火车站,摸黑来到一处公园里,我们停了下来,互相看了一眼,所有人脸色都不好。这真是出师不利,原本以为按照光头的计划,我们可以自己不用动脑筋就到达目的地。没想到没出两天,光头竟然给逮住了,还亲自带着雷子来逮我们,就这义气,还三十年的老关系,看来三叔的眼光也不怎么样啊。

我们蹲在草丛里休息了一会儿,陈皮阿四看了看我们,突然冷笑了一声,用沙哑的喉咙道:“就凭你们这几个货色,还想去挖东夏皇帝的九龙抬尸棺,吴三省老糊涂了吗?”

\chapter{营山村}

我们心情都很不爽,突然给骂了这么一句,一下子就更郁闷了。胖子呸了一口,破口大骂到道:“老爷子你这话说错了,这他娘不关我们的事啊,是那个什么三爷他眼光有问题啊,妈的这事情能怪我们吗?老子我混了这么久,第一次给雷子撵的满街跑,真他妈的憋气。”

我看他说的太过,赶紧把他拦住,打了个眼色,潘子听不得别人说三叔不好,一句两句还能忍忍,这个时候最好别说这么多了,不然可能会打起来。

胖子还算卖我面子,闭上嘴巴点上一只烟狠狠的抽起来。潘子转头问陈皮阿四道:“陈家阿公,咱们也算打过交道,现在也不是批评我们的时候,你是这里辈分最大的,现在夹喇嘛的筷子断了,您看这事怎么着吧?我们听您的。”

胖子瞪起眼睛,看样子想叫起来:凭什么要听他的?给潘子一把按住没叫出来,我知道潘子肯定有什么打算,忙拉住胖子,拍他后背让他镇定点。

陈皮阿四眯着眼睛打量了一下潘子,沉默了很久,说道:“算你懂点规矩,我就提点你们几句。这火车是不能坐了,我安排了其他车子,想跟来的等一下跟我上车,不服气的,哪儿来回哪儿去!不过我事先告诉你们,这次要去的地方,没那么简单,吴三省当初找我,就是要我这个老家伙给你们提点着,那地方,当今世上,除了我,恐怕没第二个人能进去了。”

胖子冷笑一声,“我呸!老爷子你别吓唬人,你小胖爷我什么世面没见过?我告诉你,我们几个上天摘过月,下海捉过鳖,玉皇大帝的尿壶我们都拿着颠倒过,不就是一个九龙抬尸棺吗,能有多厉害?老子过去一巴掌能把里面的粽子打的自己跳出来。还有这位,你知道他是谁吗?他是长沙狗王的孙子,想当年在山东的时候……”

我赶紧捏了一下胖子,笑道:“老爷子,别听他胡说,这家伙说一句话,你得掰一半扔茅坑里去。”

陈皮阿四看了看我,说道:“你也别否认,我知道你是吴老狗的孙子。你老爸的满月酒我去喝过,算起来你还要叫我一声阿公。”

吴老狗是我爷爷在道上几个走的近的人称呼的,我爷爷说和这人有打过交道,果然不错。

我忙点头,千穿万穿马屁不穿,叫道:“四阿公。”

陈皮阿四古怪的笑了笑,也不知道是什么意思。潘子问道:“陈家阿公,那现在,我们怎么办?是先找个地方落脚,还是……”

话音未落,远处传来一长两短的汽车喇叭声,陈皮阿四说道:“我的车来了,是来是去你们自己考虑。要上山的,就跟着我过来。”说着直起身,迈步就向喇叭响起的地方走去。

我们一下子都没跟上去,等他走远,几个人互相看了看。潘子轻声道:“这老家伙早有准备,好像早知道我们在这里会出事,我敢肯定是他卖了光头。现在敦化那边接头的人肯定也没了,装备趁早也别指望了,要弄清楚怎么回事,他妈的咱们非得跟着他不可。这一招真他妈狠。无论如何,三爷交代的事情我一定要做下去,你们去不去,自己考虑吧。”说着已经站起来,向陈皮阿四追去了。

闷油瓶看了我和胖子一眼,也站起来追了过去。

一下子只剩下我和胖子两个,我看了看胖子,胖子也看了看我,胖子问道:“对了,他刚才说的东夏皇帝的九龙抬尸棺是什么东西?”我摇了摇头,道:“我也不知道。”胖子把烟一掐,想了想,道:“那,要不,咱们追上去问问?”我失笑了一声,点点头,两个人站了起来追了过去。

在车站碰到的跟着陈皮阿四的中年人,果然是陈皮阿四安排在附近的人,安排车的就是他们。来接我们的是一辆解放卡车,我们上了车斗后,外面就堆上了货物,车子一直开出去山海关,上了省道,直开往二道白河。

这一路睡的昏天暗地,醒过来的时候已经是第二天中午,汽车没火车那么方便,到现在还有大半天的路程,这里的温度已经比杭州不知道要低多少,车斗虽然有篷布,但是风还是直往里钻,我冷的直发抖。陈皮阿四裹在军大衣里,有几次不经意间露出了老人的疲态,但是这样的表情一瞬就消失了。我心中暗叹,年纪果然还是大了一点,不知道这样一个已经知天命的老人,还要图谋些什么。

我们商量了进山的进程,按照陈皮阿四从光头那里得来的消息,到了敦化后,我们也是通过汽车进二道白河,然后那里有当地的向导和装备在等着我们。我们从那里再进一个叫栗子沟的小村子,在那个地方,他会透露给我们目的地的信息,然后向导会带着我们去那里,找到地方及出来的事情就是我们自己的了。

栗子沟我们肯定不能去,雷子可能已经守在那里了,而且那地方离二道白河还太近,我们看了看,决定不进栗子沟,直接再进去,里面还有几个村子,开到没路为止。

我们不知道光头到底知道多少关于天宫位置的信息,现在他已经不在了,事情自然就难办的多。长白山很大,还有一部分在朝鲜境内,要一寸一寸的找,恐怕也不现实。不过我们推测,既然是去栗子沟,地方必然在它附近。我们按老路子来,先到附近山村子里去踩踩盘子,打听打听消息,应该会有收获。

一切按计划进行。到了二道白河。陈皮阿四的人弄来了装备,我想着现在全国都查的那么严,怎么这些人就这么神通广大。打开一看,就蒙了,心说这是什么装备,没铲子没军火,我举目看去最多的,竟然是护舒宝卫生巾。然后还有绳子,普通的工具,巧克力,一大包辣椒,脸盆等等日用品。

胖子问怎么回事,咱们这是去发妇女劳保用品还是怎么地。陈皮阿四说用起来你就知道是怎么回事了。

四天后,我们来到横山林区比较靠里的营山村。卡车能开到这里真是奇迹,有几段路,外面三十厘米就是万丈深渊,只要司机稍微一个疏忽,我们就摔成肉泥了。到了那里找当地人一问,才知道这里原来有过一个边防岗哨,后来给撤消了,所以路才修到这里,不然得用雪爬犁才能过的来。不过正因为有了路,这里现在偶尔会有一些游客自驾游,村里的人也习惯了外来的人。

跟我们一起来的,陈皮阿四有三个伙计,一个叫郭风,就是开车的,大个子,一个叫华和尚,带着眼镜,不过身上全是刀疤,还有一个三十多岁年纪比较小的。一路上话一句也没停过,叫叶成。

我们下了车,环视四周的雪山,我想找出记忆里和海底墓中影画相似的山景,但是显然站的地方不对,看上去,雪山几乎都是一个样子。

陈皮阿四说,寻龙容易点穴难。《葬经》上说,三年寻龙,十年点穴,定一条龙脉最起码要三年时间,但是找到宝眼要十年。这一过程是非常严格的,既然我们知道了龙头在横山,只要进到山里,自然能够找到宝眼的位置。问题是,怎么进到山里去,这里不比其他地方,雪山太高,一般猎户不会去那种地方,采参人也到不了雪顶,要找一个向导恐怕很难。

村里没招待所,没找到地方住,只好去敲村委会的门。村支书倒是很热情,给我们找了间守林人的临时空木房子。我们付了钱安顿了下来,在村里呆了几天,租好了马,几经辛苦,找到了一个当地的朝鲜族退伍兵顺子愿意做我们的向导。

这人告诉我们,一般人不会上雪山,由于风雪变化,基本上每天的路都不一样,而且上去了也没东西,只有他们当兵的,巡逻的时候要上去。这里的几座峰他都能上,所以我们真想上去,他能带我们去,不过进了雪区之后得听他的。

我们商量好了价钱,事情就拍板下来,整顿了装备,又按顺子的要求买了不少东西,九个人十四匹马浩浩荡荡就往林区的深处走去。

长白山风景很美,举目望去山的每一段都有不同的颜色,因为山高的让人心寒,我们也没有太多去注意四周的森林景色,所有的精力都放在保证自己不掉下马上,但是偶然一瞥,整个天穹和山峰的那种巍然还是让人忍不住心潮澎湃。

长白山是火山体,有大量的温泉和小型的火山湖。从营山村进林区,顺着林子工人的山道一直往上四个小时,就是“阿盖西”湖,朝鲜话就是姑娘湖,湖水如镜,一点波澜都没有,把整个长白山都倒影在里面。

为了让顺子认为我们是游客,我们在湖边留影,然后继续出发。我们刚进去的那一段是在山脉的低部,越往里走低米那就越陡起来。最后我们发现自己已经行进在60度左右的斜山坡上,这里的树都是笔直的,但是地面是斜的,每一步都显得非常惊险。顺子告诉我们再往上那里面还有个荒村,就是边防哨所在的地方,那里现在已经没人了,我们在那里过第一夜,然后第二天,我们就要过雪线了。

此时“阿盖西”湖已经在我们的下方,我们由上往下俯视,刚才若大的湖面就犹如一个水池一般大小了。这个时候,我们所有人都发现,另一只马队出现在了湖边,这只队伍的人数远远超过了我们。

我们觉得有点意外,胖子拿出望远镜,朝下面看了看,然后递给我道:“我们有麻烦了。”

\chapter{困境}

我一边策马前进,一边顺着胖子指示的方向看去,透过稀疏的树木,我看到下面湖边上熙熙攘攘的大概有三十几个人,五十多匹马,是一支很大的马队。

那些人正在湖边搭建帐篷,看来想在湖边上过夜。其中有一个女人正在张开一个雷达一样的东西调试,我用望远镜一看,那女人不是别人,正是在海南的阿宁。

我骂了一声,这个女人也来了这里,那说明我们的推断没错。三叔想要拖延的人,恐怕就是这一帮,不知道捞泥船的公司,来到内陆干什么。

华和尚也看到了下面的马队,脸色变了办,轻声问陈皮阿四怎么办。

陈皮阿四看了看,轻蔑的笑了笑,说道:“来的好,说明我们的路没走错,继续走,别管他们。”

我拿着望远镜一个人一个人看过来,没看见三叔,不过三叔既然是可能落在了他们手里,不太可能有太多自由,有可能给关在帐篷里了。令我觉得不舒服的是,下面的人当中,有一半几乎都背着五六式步枪,我还看到了卫星电话和很多先进设备。胖子看着枪眼馋,对陈皮阿四道:“老爷子,你说不买枪不买枪,你看人家荷枪实弹的撵上来了,要交上手了怎么应付?难不成拿脸盆当盾牌,用卫生巾去抽他们?”

陈皮阿四看了他一眼,甩了甩手笑道:“做我们这一行从来不靠人多,过了雪线你就知道跟着我跟对了。”

我们的对话全是用方言交谈,汉语都讲不利索的顺子听不太明白,不过他做向导好多年了,自己也知道客人说的话别听,听太多了,人家说不定把你灭口。

我们继续往上走,直看到前面出现一些破旧的木头房子和铁丝门,上面还写着标语“祖国领土神圣不可侵犯”。

顺子告诉我们,这里是雪山前哨战的补给站。多边会谈后,这里的几个哨站都换了地方,这里也荒废了,雪线上的几个哨站也都没人了,咱们要上去的话,到时候有机会去看看。

当夜无话,我们在这里凑合过了一夜,第二天一大早就起床继续赶路。顺子觉得奇怪,少有旅游的人这么拼命的,不过收人钱财也由的我们。

我们起床的时候已经开始下雪,气温陡然下降。南方人很少能适应这样的天气。除了胖子和叶成,其他几个人无一不冻的僵硬。

再往上过了雪线,我们终于看到了积雪。一开始是稀稀落落的,越往上就越厚,树越来越少,各种石头多起来,陈皮阿四说这是这儿有工程进行过的痕迹。

到了中午的时候我们四周已经全是白色,地上的雪厚的已经根本没路可走,全靠顺子在前面带着马开道。这时候忽然刮起了大风,顺子看了看云彩,问我们,要不今天就到这里吧,看这天可能有大风,看雪山过瘾就过一下,再往上就有危险了。

陈皮阿四呵起气摆了摆手,让他等等。我们停下来休息,吃了点干粮,几个人四处去看风景。

我们现在在一处矮山的山脊上,可以看到我们来时候走过的原始怎林,他极目眺望,然后指着一大片洼地,对我们说:“古时候建陵一般就地取材,你看这一大片林子明显比旁边的奚落,百年之前肯定给人砍伐过,而且我们一路上来虽然步履艰难,但是没有什么特别难过的障碍,这里附近肯定有过古代的大工程,这一带山体给修过了,咱们大方向没错,还得往上。”

叶成问道:“老爷子,这山脉有十几座山峰,都是从这里上,我们怎么找?”

陈皮阿四道:“走走看看,龙头所在肯定有异象。地脉停顿之处为龙穴。这里山多,但是地脉只有一条,我们现在是沿着地脉走,不怕我不到,最多花点时间而已。”

我顺着他的目光看去,只看到一片一片的树,也看不出有什么区别,不由自惭形秽。

转头去看闷油瓶,却见他眼睛只看着前面的雪山。眉头微微的皱了起来,好像在担心什么事情。我知道问他肯定是白问,转身去找胖子聊天。

顺子听说我们还要往上,叹了口气,摇头说套再往上马不能骑了,要用马拉雪耙犁。长白山的冬天其实是交通最方便的地方,除了暴风雪天气,一般用马拉雪耙犁能爬到任何马能到的地方,但是一旦风起,我们任何事情都得听他的,他说回来就回来,绝对不能有任何异议。

我们都点头答应,将行李从马上卸下来,放到耙犁上,准备妥当,顺子叫着抽鞭子在前面带路,我们的马自动跟在后面,一行人在雪地里飞驰。

刚坐雪耙犁的时候觉得挺有趣的,和狗拉雪橇一样。不一会儿,不知道是因为风大起来的关系还是在耙犁上不好动弹,身体的肢端冷的厉害,人好像没了知觉一样。因为是山路,马跑的不稳起来,胖子因为太重,好几次都侧翻摔进雪里,弄的我们好几次停下来等他。

就这样一直跑到天灰起来,风越来越大,马越走越慢。我们不得不戴上风镜才能往前看,到处是白色的雪花,不知道是从天上掉下来的还是雪山上刮下来的。满耳是风声,想说句话,嘴巴张开,冰凉的风就直往里灌,用胖子的话说,骂娘的话都给冻在喉咙里了。

跑着跑着,顺子的马在前面停了下来,我隐约觉得不妙,现在才下午两点。怎么天就灰了。我们顶着风赶到顺子身边,看到他一边揉着脖子一边看四周,眉头都皱进鼻孔里去了。

我们围上来问他怎么回事,他啧了一声,说道:“风太大了,这里好像发生过雪崩,地貌不一样了,我有点不认识了。还有,你们看,前面压的都是上面山上的雪,太深太松,一脚下去就到马肚子了,马不肯过去。这种雪地下面有气泡,很容易滑塌,非常危险,走的时候不能扎堆走。”

“那怎么办?”潘子看了看天,“看这天气,好像不太妙,回的去吗?”

顺子看了看天又看了看我们。说道:“说不准。不过这风一旦刮起来,没两天两夜是不会停的,咱们在这里肯定是死路一条,前面离那座废弃的边防岗哨不远了,到了那里能避避风雪,我看回去已经来不及了,我们可以徒步过去。”

胖子压着自己的盖耳毡帽,试探性的走了一步,结果人一下字就捂进了雪里,一直到大腿。他艰难的往前走了一步,骂道:“他奶奶的,有的罪受了。”

我们穿上雪鞋,顶着风,自己拉着爬犁在雪地里困难的行进,这地方是一风口,就是两边山脊的中间,风特别大,难怪会雪崩。我们往风口里走,顺子说着哨岗一个小时就能到,但是不知道是我们走的太慢。还是顺子压根就带错路了,走到傍晚六点多,还是没见到哨岗的影子。

顺子转来转去摸不着头脑,再一想,忽然哎呀了一声:“完了,我知道这哨岗在什么地方了!”

我们围上去,他脸色极度难看,道:“我怎么就没想到,这表示小雪崩,哨岗肯定给雪埋了,就在我们脚下,难怪转了半天都找不到!”

潘子叹了口气,说了句话,看他的嘴型是:“妈拉个B的!”

胖子大叫着,问顺子:“那现在怎么办?马也没了,难不成我们要死在这里?”

顺子指了指前面,说道:“还有最后一个希望,我记得附近应该有一个温泉,是在一山包里,温度很高。如果能到那里,以我们的食物可以生活好几天,那温泉海拔比这里高,应该没给雪埋住。要真找不到,那只有求生意志了,一步一步再走回去了。”

“你确定不确定啊?”胖子对顺子不信任起来。

顺子点头:“这次绝对不会错,要找不到,你扣我工钱。”

我心里苦笑,你娘的要真的扣你工钱,恐怕呀下辈子才有这机会了。

众人都哭丧着脸,跟着顺子继续往上走。天越走越黑,顺子拉起绳子让我们每个人都绑在身上,因为能见度太低了,根本看不到人,叫也听不见,只能靠这绳子才能让我们集中在一起。

我走着走着眼睛就开始花了,怎么也看不清楚。前面的人越走越远,后面的人越拖越后,我一发现两面都看不到人,心里不免咯噔了一声,心说是不是现在这个时候进山犯了个错误,难道会死在这里。不像,顺子走的还挺稳,虽然我看不见他,但是感觉到这绳子的走向很坚定,折中风雪他一定已经习惯了,跟着他准没事。

我一边安慰自己,一边继续往前,忽然看到前面的雪雾中出现了一个黑影,迷迷糊糊的我也看不清楚是是谁。走了几步,那黑影子忽然一歪,倒在了雪地里。我赶紧跑过去一看,竟然是顺子扑倒在雪里。

后面闷油瓶追了上来,看到顺子,赶紧扶了起来。我们背着他,一边拉紧绳子,让其他人先聚集过来。

胖子一看到顺子,做了一个非常古怪的表情,大吼道:“这他娘是什么向导啊?不认识路不说,我们还没晕他先晕了,叫我们怎么办?”他还想再骂,但是后面话全给风吹到哪里都不知道了。

我看了看四周,我的天,四周的情形已经完全失控了。强烈的夹着大量雪花的风被岩石撞击着在我们四周盘旋,一米之外什么也看不见,我们来时候的脚印几乎一下子就给风吹没了。我们东南西北都分不清楚,强风压过,连头也抬不起来,站起来就会给吹倒。

所有人的脸色全是惨白,陈皮阿四眼睛米粒,看样子老头子在这样的极限环境下,已经进入半昏迷状态了。就算顺子不倒下,他肯定也坚持不了多久。

潘子道:“我们不能停下来等死,温泉可能就在附近,我们拉长绳子,分散了去找找,找到了就拉绳子做信号。”

我们四处散开,我也不知道自己选了哪个方向,一边走人就直打晕呼,只觉得一种麻木感从四肢传递到全身。以前看过不少电影里都说,在雪山上,人会越来越困,如果睡着就永远醒不过来了,人还会产生很多幻觉,比如说热腾腾的饭。

我拼命提醒自己,可是却一点也坚持不住。每走一步,眼皮就像多灌了一块铅一样,沉重的直往下耷拉。

正在一筹莫展之时,忽然听见胖子叫了一声,风太大了叫了什么没听清。我回头一看,只见他的影子一闪就没了,闷油瓶马上转过头去,发现地上的绳子突然拉动起来,脸色一变,大吼:“不好!解绳子,有人塌进雪坑里去了!”

话还没说完,他脚下的雪突然也塌了,整个人给绳子一下子扯进了雪里,接着就是离他最近的我。

我们就像一串葡萄一样一个接一个被胖子拉进了雪地里,翻来滚去,不知道滚了多久才停住。

我眼睛里全是雪,根本睁不开,只听到潘子叫我们都别动,他是最尾巴上的,他先爬下去再说。

这个时候,突然听到叶成叫了一声:“等等等等!操家伙!都别下去,那雪里盘的是什么东西?”

\chapter{百足龙}

我拍掉眼睛上的雪珠,一时间也不知道自己在什么地方,只感觉背上顶着石头尖,叶成就在我下面,在那里大叫。

我定了定神,下意识的去看叶成在害怕什么东西,往下仔细一看,发同我们现在正靠在一面陡峭的乱石坡上,离坡底还有五六米,腰里的绳子挂在了一块岩角上,我们才没直接滚下去,坡底全是刚才随我们一起滑下来的雪块和石头,雪堆里面,露出了好几截黑色细长的爪子。

我感觉到一阵窒息,不自觉的把背贴紧后面的石头,顺着爪子看上去,雪堆里若隐若显,盘绕着一条黑色的,水桶粗细的东西,环节状的身体上全是鳞片,一些藏在雪里,一些露在雪外,我咋一看还以为是条冬眠的蛇,仔细看又像是蜈蚣。这东西贴着石头,一动也不动,不知道是死是活,看不到头和尾巴,也不知道有多长。

我心里奇怪,这已是雪线以上,本来活物就很少,这到底是什么生物,看着那些蜈蚣一样的爪子和它的个头,心里本能的不舒服起来。

潘子执意要下去,叶成不停的叫,胖子也看见了下面的东西,拿自己边上的雪捏个球砸在叶成后脑上,轻声骂道:“你他娘的给我轻点声,想把它吵醒?!”

我看了看四周,这里应该是一处封闭的小山谷,被雪崩填满了,但是因为这里石头堆砌太凌乱,产生了大量气泡,胖子走到上面,把脆弱的雪层踩断,引起连锁反映,雪层一下子塌了。结果我们全部给他带了下来。上面的雪还在不断的坍下来,很多时候这样的塌方之后,四周的积雪会像流沙一样汇拢过来,将塌出的地方重新埋住,这一过程极其快。很多高山探险队就是在这样的情况下减员,几秒钟整个队伍就消失了。

幸好这一次边上的雪还算结实,可能也是因为我们是给绳子拴在一起的。

这里是背风面,风明显小了很多,不像刚才那么冷了。我得以畅快的呼吸了几口,小心翼翼地坐起来,往下挪了几步,这里虽然很陡峭,但是坡体表面上都是碎石头,有些有解放卡车头那么大,有些只有乒乓球大小,攀爬很方便,往上往下都不困难。

潘子和闷油瓶已经解开了绳子,因为离底不远,他们两个哗啦一声,带着雪跳了下去,落地之后打了滚缓冲力道,滚到了坡底。两个人蹑手蹑脚的爬起来,一前一后朝那黑色的东西摸过去。我们一下子心提到了嗓子眼上。

走了几步,闷油瓶和潘子都直起了腰,明显放松下来,潘子看了看闷油瓶子,耸了耸肩膀,做了个手势让我们下来。

我们奇怪,胖子解开绳子也滚了下去,闷油瓶已经把石雕地上的积雪扫掉,原来那是一条伏石而卧的石头盘崖石龙,用黑色的石头雕的,磨崖石雕非常传神,如果藏在雪里,还真看不出来。

我们陆续下去,陈皮阿四看见石雕,人明显脸色变化,他站立不稳,招呼华和尚扶着他,径直走到磨崖石雕的前面,摸了起来,这条龙有和其他的龙不同之处,它的身子下面,刻了无数只和蜈蚣一样的脚,显然不是中原的雕刻,应该是附近游牧民族异化的龙。

胖子问我道:“怎么这龙这么难看,像条虫一样,看上去邪气冲天,比故宫龙璧上的难看多了,该不是刻坏了。”

华和尚道:“不懂别乱说,这条是百足龙,不是蟠龙,东夏国早期的龙雕都是这个样子的。在中国早些时候,中国远古的龙有着迥异的形态,有的龙还有猪鼻子呢,这不奇怪。”中国龙的演变非常漫长,刚开始的龙是匍匐爬行,随便找个兽头放在蛇身上就是龙了,那个时候每个部落都有自己的龙图腾,各部落分别演变,到最后龙的形态也各不相同。后来汉文化传播,夷夏文化大融合,汉龙的形象才和各少数民族的龙开始混合,到最后龙逐步统一成现在这个样子的蟠龙。

这条百足龙,就是龙和蜈蚣的混合体,可是不知道为什么,看一条普通的龙身下长了这么多只节肢动物的脚,不但无法给人威武的感觉,反而让人觉得非常的不舒服,让人觉得有一丝诡异。

胖子听了华和尚的介绍,笑道:“刀疤兄,看不出你还挺文学的,那这块石头,应该是东夏国的东西了?”

华和尚看了看石雕,又抬头看了看山坡的上面,疑惑道:“没错,只不过,这块雕龙的石头是从哪里来的?”

此时天已经入黑,我们各自打开手电,边用手电边把石头上的雪全部扫掉,发现这块石头几乎是一块五米高三米宽的巨大平板子,靠在一边的乱石坡上,石头极平整,而且是黑色的,和这里的其他石头明显不同。

我看了看石头断裂处的痕迹,说道:“可能是从上面塌下来的,四阿公说的没错,我们要去的地方还在上面。你看这龙的形体不对称,这是双龙戏珠,这样的石雕应该还有一边,一般是刻在石门上的,两面各一。”

陈皮阿四咳嗽了一声,有气无力和说道:“放屁,一知半解,大放撅词,什么石门,这块是墓道里的封石。”

说着他指了指龙嘴巴,华和尚马上过去,把手抻进龙嘴巴里,一扯,竟然给他扯出一条黑色手腕粗细铁链来,胖子一看,说道:“哎呀完了,龙肠子给你扯出来了。”

陈皮阿四道:“这是封墓的时候用来拉动封石的马链,这一面是朝里的一面。”

我给他说的脸红,左顾右盼道:“啊,果然是,我看错了,可是封石怎么会出现在这里?”

华和尚用力扯了扯铁链,石头纹丝不动,陈皮阿四脸上也闪过一丝疑惑,摇了摇头,抬头看了看上面,我心里哎呀了一声,知道他在担心什么,如果这块封石是从上面塌下来的,那说明上面的墓道毁坏很严重了,我们就算找到了,还能不能进去?

头顶上风雪肆虐,天已经黑的基本上入夜,我看了看表,不知道这暴风雪要刮到什么时候。

发现了这块石雕,增长了我们找到天宫的信心,但是我也不知道该高兴还是沮丧。华和尚给石雕拍了照片,陈皮阿四精神恢复过来,让我们先把自己的东西顾好,该休息的休息一下,这里正好避风,什么事情等风停了再说。

我们将装备整理出来,华和尚去照顾那个伤兵。我在翻东西,他跑过来告诉我,有点麻烦,顺子已经基本上没反应了。

我们将顺子放倒,摇了摇他的头,他只能迟钝的“嗯”一声,意识模糊,一看就知道是低体温症。

“我们得生点火,不然他熬不了多久。”潘子走过来说,“睡过去就醒不过来了。”

我看了看四周,根本没有任何柴火,要点起火来,恐怕要烧爬犁了。可是上雪山需要很多装备,没有爬犁,下面的路恐怕走不下去。

华和尚看了看陈皮阿四,显然不敢自己做主,后者的脸色很阴糜,不知道是给冻的还是怎么的,皱了皱眉头,说道:“暂时别让他死,我还有事情问他。”

我松了口气,华和尚将爬犁上的东西卸掉,准备把木条子扯出来当柴火,不过现在的爬犁也都给雪浸湿了,不知道还点不点的起来,正在担心的时候,我忽然闻到一股硫磺的味道。

这味道不知道从哪里冒出来的,我脑子一跳,让华和尚先别动,自己站起来仔细的闻。其他人也同时闻到了,都停下下手里的事情,胖子猛吸了一口,道:“同志们,好象有温泉的味道!”

陈皮阿四给叶成和郎风打了眼色,让他们出去找,胖子背起背包也说要去,结果三个全给潘子拦住了,胖子问:“干什么”,潘子用下巴指了指闷油瓶,说道:“慌什么,别忘了咱们有高手在。”

这时候闷油瓶已经俯下身子,用他奇长的两根手指逐一摸了摸了底下的石头,忽然皱了皱眉头,“嗯?”了一声,转向一边的百足盘龙封石。

我们来到那块盘龙石面前,这里刚才还没有什么味道,现在的硫磺味已经很明显了。闷油瓶摸了摸龙头,又看了看石头后面,将手往龙头上一放,一压,说道:“奇怪,龙头后面是空的。”

\chapter{缝隙(上)}

长白山是潜在的活火山,根据史料记载,最后一次小规模的喷发应该是在1000年前,现在虽然火山归于沉寂,但是附近地热极其丰富,不少火山时期的地质缝隙和熔岩口都保持着极高的温度,这盘龙封石的后面,说不定就压着一条冒着热气的地缝,才会冒出硫磺的味道。

这对于我们来说无疑是一个好消息,在这样的环境里,能有一个稳定的热源肯定比点篝火要经济实在,可是黑色的巨大盘龙封石压在上面,目测一下少说也有十几吨重,我们没有任何开山设备,要把它翻覆过来,实在有点难度。

胖子是行动派,撩起胳臂招呼我们去搬石头,几个人上去尝试性的扛了两下,一群人抬得满头大汗,面红耳赤,石头却纹丝不动。

胖子气喘吁吁,骂道:“不成啊,老爷子,早说咱们装备不行,你看现在这情况,要有点炸药多好。”

华和尚说你不懂就不要乱说,我们老爷子过的桥比你走的路多,不带炸药来是对的,你说我们现在谷底,你头顶上白雪皑皑,你随便那里放个炮眼,把上面的雪震下来,一下就给活埋了。

胖子没话反驳,这时候我看到盘龙石的下沿,卡着很多大小不一的石头,灵机一动,对他们说道:“可能不需要炸药,让我来。”

说着我从行李上拿出一把石工锤,走到盘龙封石的一边,仔细检查了一下下面几块比较大的石头,然后对准其中一块用里一敲,那块石头一方面受着十几吨的压力,又收到我侧向锤击,马上裂开一条缝,紧接着卡拉拉一连串石头磨擦声,上面的盘龙封石因为支撑力突然变化,顺着石坡开始滑动。

我们赶紧向后退去,盘龙封石向下滑了几寸,又开始倾斜,可是这块石头实在太重了,滑动了一点点位置就停了下来,虽然如此,我们还是看见封石的后面,露出了山体上的一条岩缝。

岩缝有脑袋宽,人勉强能通过。看洞口的边缘,呈岩层撕裂状,没有人工开凿的痕迹,一阵阵的硫磺味道就是从里面传出来的。

胖子调亮手电,伸手进去看了看,转头道:“里面很暖和,不过角度太难受了,照不到什么,而且,里面的石壁上好象有字。”

“写着什么?”我问道。

胖子眯起眼睛仔细看了看,道:“看不懂,妈的不知道写些什么。”

说着他试图猫腰钻进去,但是胖子的确太胖了,这个洞显然不适合他,挤了几次,挤不进去。最后他把外面的大衣脱了,才勉强钻了进去。

陈皮阿四让叶成,郎风和潘子留在外面,有什么事情好照应。我们跟在胖子后面,钻进缝隙里。

这里整个儿就是条山体运动时候裂开的岩缝,进去之后,发现缝隙是一个陡峭的向下的走向,里面非常黑。看样子极其深,恐怕通到这山内部。

缝隙开口处的空间不大,两个人无法并排,而且缝隙里面非常难以行走,底下全是大块的石头,棱角分明,洞里的硫磺味道非常浓,温度起码有三十度,摸了摸,连石头都是烫的。

我们手脚并用的往前走了几步,胖子用手电照了照一边,说道:“你们看,这些是什么字?”

我转过头去,字不是刻在缝隙的壁上,而是刻在一块横在的底部乱石上,都是几个陌生的文字,有点像中文,又有点像韩文,刻的很凌乱。

华和尚凑过去看了看,确定道:“这是女真字。”

“写的什么?”胖子问。

华和尚道:“等等,我没那么厉害,要看看才知道,我先把它描下来。”

我们等了片刻,华和尚把这些字抄到本子上,胖子打头,我们排成一队,继续往洞的深处走去。

说是走,其实用手的机会比脚还多,整条缝隙几乎是三十度向下,又没有阶梯,几乎全靠爬着下去,里面时宽时窄,时高时低,有些地方人要坐着才能通过。

唯一让人舒服的是,这里面暖和很多,我们爬着爬着,都开始出汗,只好解开衣服扣子。这时候胖子问道:“老爷子,你说会不会那封石堵着这条缝,不是偶然啊?”

陈皮阿四吟道:“开同建陵,就地取材,这里的外面这么多乱石头,应该是修建陵墓时候用来采石的石场,可能这条缝是他们采石的时候发现的,不知道为什么,最后要用封石压住。”

下了不到一百米,硫磺的味道越来越浓,岩石也越来越黑,都开始呈现琉璃的光彩,那是云母高温融化过的痕迹,我哎呀一声,心里已经在想,这里应该是一处火山的熔岩口啊,长白山是潜在的活火山,要是突然间喷发了,岩浆从山体内部喷出来,我们不就死定了。

胡思乱想着,忽然,打头阵的两个人停了下来,手电照去,原来前面裂缝陡然收缩,乱石重叠,只剩下一个极小的缝隙能够下去。

我蹲下去用手电照了照里面,这里是缝隙坍塌造成的,里面空隙很小,看样子要匍匐着才能进去。

陈皮阿四看了看这个洞口,知道自己的体力是爬不进去了,商量了一下,我让华和尚陪着他等我们。我,胖子和闷油瓶进去看看,里面还有什么。

我们脱掉外衣,让自己的体积尽量减小,这一次是闷油瓶打头,三个人前后下去,一点一点挤进那条缝里。

我以为这一段坍塌只是暂时的,向前爬个几步,必然会有出口,如果是实的,我们也可及时掉头回去,没想到这一段空隙很长,爬了很久,前面还能通行,深得出乎意料。

里面的石头尖子非常锋利,我爬了几步,身上的衣服已经勾破了好几处。岩石挤压着我的肺部,加上温度越来越高,我逐渐感觉到呼吸困难起来。

后面的胖子和我感觉一样,拉住我的脚道:“不成,这里的空气质量可能有问题,咱们探也没探就进来,太莽撞了。”

我想回头看看,空间太小,实在没办法,想着刚才爬过来很长一段距离,要回去也舍不得,而且现在这个局面,倒着爬恐怕比来时要更加痛苦,于是道:“咱们再往前几步,如果还没底再退出去。”

胖子应了一声,这时候,忽然,前面的闷油瓶子叫了一声:“嗯?”

我转头向前看去,前面却空空荡荡,刚才还在堵着我的闷油瓶子,前面却不见了,只剩下一个黑漆漆的石隙通道,不知道通向何方。

\chapter{缝隙(下)}

从我听到闷油瓶说话,到发现他在我面前消失,绝对不超过五秒钟,就算是一只老鼠也无法在这种环境下迅速的在我眼前消失,更何况是一个人。

我顿时感觉到不妙,下意识的往后退了一步,想再看仔细了,一恍神间,却看到闷油瓶子又出现在我的前方。

胖子就在我后面,给我退后了的一步,吓了一跳,问道:“怎么回事?”

我一时间丈二和尚摸不着头脑,支吾道:“没……没事。”

闷油瓶子似乎并不知道自己刚才出了异状,顿了一下,招呼了我们一声,开始加快速度向前爬去。

这一隐一出在一瞬之间,虽然我感觉的十分真切,但是看到面前的景象,又突然没有了十足的把握,心里非常疑惑,难不成是这的空气,让我产生了幻觉?

情况不容我多考虑,胖子在后面拉我的脚催我,我一边纳闷一边又跟着爬了一段距离,爬过刚才闷油瓶消失的那一段的时候,我特别留意看了看四周,也没有任何凹陷和可以让我产生错觉的地方,心里隐约觉得不妥起来。

通过这一段,又前进了大概十分钟,闷油瓶子忽然身形一松,整个人探了出去,我看前面变得宽敞,知道出口到了。

缝隙的尽头是大量的乱石,爬出去后,闷油瓶子打出数只荧光棒,扔到四周,黄色的暖光将整个地方照亮起来,我转头看去,发现这里应该是整条山体裂缝中比较宽敞的地方,大概有四五辆金杯小面包的宽度。长大概有一个半篮球场,底下全是大大小小的碎石,都是这条裂缝形成的时候给地质活动撕裂下来的。

胖子扩大手电的光圈,四处观察。说道:“怪了,这里竟然还有壁画,看来我们不是第一批来这里的人。”

我们走上去,发现裂缝的山壁上果然有着大幅的彩色壁画,但是壁画的保存情况十分差,颜色黯淡,上面的图案勉强可以分辨出是类似天女飞天的情形。

进入这里的人口给一块巨大的封石压住了,里面还有壁画。这里到底是什么地方?我在一次感到疑惑。

来回走了走,在碎石之间,我们发现了几处小的温泉眼,都很浅,但是热气腾腾,说不出的诱惑。但是却没有发现其他人活动过的痕迹。

再往里面,缝隙里不时吹出热风,我走到一边向里照了照,深不见底,不知道通到那里。

我们交换了意见,认为没有必要再进去,这里已经是躲避暴风雪的好地方。胖子测试了空气没有太大问题,打起持久照明用的风灯,闷油瓶又爬回来时候的狭小缝隙通知外面的人。

不一会儿,华和尚和叶成先后进来,顺子也给潘子拉了进来,我马上去检查他的情况,发现因为这里温度的关系,他的脸色已经开始红润,但是手脚依然是冰凉。不知道能不能挺过来。

上来的路都是由他带的,如果他死了,虽然不至于说下不去,但是总归会很多困难,再加上我也挺喜欢这个人,真不希望他因为我们而这么无辜的死去。

华和尚检查他的心跳和脉搏,然后让我让开,用毛巾浸满温泉水,放在石头上稍微冷却后,给顺子搽身,等全身都给搽的血红后,才给他灌了点热水进去,顺子开始剧烈的咳嗽,眼皮跳动。

我们稍微松了一口气,华和尚说道:“行了,死不了了。”

气氛缓和下来,胖子和叶成都掏出烟,点上抽了起来。这时候陈皮阿四也给潘子搀扶着进来。

经过这一连串变故,我们的筋疲力尽,也没力气说话,各自找一个舒服的地方靠下来。

身上的雪因为温度的变化融化成水,衣服和鞋子开始变的潮湿,我们脱下衣服放在干燥的石头上蒸干。叶成拿出压缩的罐头,扔进温泉水里热过分给众人。

我一边吃一边和华和尚去看刚才发现的壁画,这里非常明显是天然形成的而且空间狭窄,为什么要在这里画上壁画,刚才闷油瓶突然在我面前消失,和洞口的巨大封石,给我一种很不自然的感觉。

和古物打交道的人,对于壁画和浮雕这种传承大量信息的东西,总是非常感兴趣的,其他人看我们在看,也逐渐走了过来。

然而壁画上却没有太多的信息,天女飞天的壁画多处于华丽的宫廷或者礼器之上,只是表现一种美好的歌舞升平的景象,并没有实际的意义。这里的壁画残片,大部分都是这样的东西。这里都是古墓里爬出来的人,见的多了,一看便失去了兴趣。

我正想回去揉揉我的脚趾头,这一路过来出了不少汗,脚趾头都冻麻痛了,这个时候,却听见胖子“啧”了一声,伸出自己的大拇指,开始用手指剥起壁画来。

我问他怎么回事,虽然这东西没什么价值,但也是前人遗物,你也不能去破坏它啊。

胖子说道:“你胡扯什么,我的指甲就没价值了?一般东西我还不剥呢,你自己过来看,这壁画有两层!”

“两层?”我嗯了一声,皱起眉头,心说什么意思?

众人又围了上去,走过去看他到底说的是什么,他让我们看了看他的手指,只见上面有红色的朱砂料给刮了下来,再看他面前的那一块地方,果然,壁画的角落里有一块构图显然和边上的不同,画的东西也不同,只是这一块地方极不起眼,要不是胖子的眼睛尖,绝对看不到。

这显然是有人在一幅壁画上重新画了一层,将原来的壁画遮住,而造成的情形。

这上面一层因为暴露在空气之中逐渐脱落,将后面的壁画露了出来,这在油画里,是经常的事情。

胖子继续用手指刮了刮壁画,发现这表面一层,似乎并没有完成所有的工序,所以胖子随便一刮,就可以简单的将颜色搽掉,不然如果按照完整的步骤,唐以后的壁画外面会上一层特殊的清料,这层东西会像清漆一样保护壁画,使颜色没有那么容易褪色和剥落。

陈皮阿四的眉头皱的很紧,很快,一大片脸盆大的壁画被剥了下来,在这壁画之后,出现了有五彩颜料画的半辆马车,马车的主人,是一个肥胖的男人,这个男人的服饰,我却从来没有见过。

这是叙事的壁画,我忽然紧张起来。

显然有人先画了一幅叙事的壁画,但是因为某种原因,有非常匆忙的用另一幅替代掉了,而且当时的时间可能十分紧张,所以这外面的壁画连最后的工序都没有完成。

陈皮阿四看了看这整幅壁画,又看了看周围的环境,对我们说道:“这……和天宫有关系,把整面墙都清掉,看看壁画里讲的是什么。”

我早就想动手了当下和其他一起,祭出自己的指甲,开始精细作业,去剥石壁上的壁画。

壁画大片大片的剥落,不一会儿,一幅色彩绚丽,气势磅礴的画卷,逐渐在我们面前展了开来……

\chapter{双层壁画}

四周静的吓人,风灯给提到了岩壁的一边,加强照明,昏黄的灯光照在岩石上,给人一种古老神秘的感觉。

壁画的颜色非常鲜艳,用了大量的鲜血一样的红色,在不定光源下,闪现出琉璃的光彩,好像是整块岩石正在渗出鲜血一般,掩藏在另一曾颜料下面的壁画能保存的这么好,真是不可思议。

然而真正让我们惊讶的,却是壁画的内容,我很难用语言来形容上面画的是什么,壁画分为二个部分,分别记述了不同的事情,然而整合在一起,又看上去十分完整,可谓美伦美幻。

华和尚看的眼睛发亮,自言自语道:“这应该是东夏万奴皇帝,和蒙古人之间的战争的场景,你看这个人,这个人应该就是万奴王本人,这很可能是传说中东夏灭国的那一场战争。”

我对东夏的了解非常少,其他人显然也并不精通,都没有说话,听他继续说下去。

他来回一边惊叹,一边看着上面图案指着壁画的一边,大量披带着犰皮和盔甲的士兵,说道:“这是万奴王的军队。”又指了指一边的骑兵,说道:“这是蒙古人的军队,你们看,人数远远多过东夏的军度一,这是一场压倒性的战争。”

我看着他指的方向,看到了箭石纷飞的画面,胖子看了看,不知道觉得哪里奇怪,问道:“为什么东夏的军队,那些人的脸都像是娘们?”

我看者也觉得奇怪,难道东夏人靠女人打仗吗?那不亡国就没天理了。华和尚说道:“不是,这是东夏壁画的一个特征,你看所以的人,都是非常清秀的,我在典故上也查到过一些奇怪的现象,似乎所以和东夏国打交道的人,都说,在东夏国,见不到老人,所有的人都很年轻。朝鲜人说,东真的人就连死的时候,也保持着年轻的容貌。”

胖子皱着眉头,似乎想不通为什么会是这样,我感觉这可能和一些少数民族的习俗有关系,有些民族,老人是不能见客人的。我不以为意,和其他人又继续看下去。

华和尚又指了指到壁画的第二部分,说道:“这一块就记载着战斗的情形。你们看,东夏人以一敌三,还是陆续个蒙古人射死,这场战斗最后变成了屠杀。”

壁画上用了大量的红色表现战争的惨烈,代入感极强,我仿佛看东夏兵一批一批的倒在血泊里,蒙古的铁骑从他们的尸体上踏了过去,开始焚烧房屋和屠杀男人。

壁画的第三部分。给压在了一块巨大的石头后面,我们无法移开,但是估计,也应该是这里内容的延续。

此时我感觉到疑惑,打断他道:“不对啊,东夏这个国家,不是老早就给蒙古人灭了?我看资料说,他们才存在了七十多年,一直再打仗,如果说云顶天宫是他们造的,在当时的情况下,这么小一个国家,如何有能力建造这么大规模陵墓?”

我这话一出,不少人都露出了赞同的神色,东夏是女真被灭国时期,在吉林和黑龙江一带突然出现的一个政权,我记忆里他的开国皇帝万奴王甚至没时间传位给下一代,就给蒙古人绕道朝鲜给灭了,那个时候蒙古正是极端强悍的时候,遇神杀神,遇佛杀佛,壁画上的景象如果真是那一场决战,以蒙古人的性格,应该被灭的十分彻底才对。

而那个时候女真各部之间的生产力还是十分低下的,没有大量劳动力,就算没灭国,也根本没可能建造如此巨大的陵墓。

陈皮阿四所说的,云顶天宫里真的埋着东夏的皇帝,怎么想都是不可能的事情,因为他们没有这个时间也没有这个实力。

更没有理由的是,如果按照在海底墓穴中我们看到的东西推断,这座传说中的陵墓是汪藏海建造的,那修建的朝代怎么样也应该是元末,那个时候,东夏国已经被灭了几百年了,哪里还会有东夏皇帝能用来下葬。

我们都将目光投向陈皮阿四,说云顶天宫中葬的是东夏皇帝的是他,但是现在看来,似乎绝对没有这个可能。

陈皮阿四知道我们在想什么,四面无表情的扫了一眼壁画,冷笑一声,然后看了华和尚一眼,说道:“既然他们不信,和尚,你就给他们说说。”

华和尚答应了一声,转头对我们笑道:“我知道你们在怀疑什么,我敢说你们都想错了,你们看到的关于东夏的资料,大部分都是根据一些不完整的古书推断出来的,实际上东夏国留下的资料实在太少了,在国外,甚至不承认有这么一个国家存在过,所以你们现在所看到的信息,实际有多少是真实的,很难说。”

胖子说道:“既然如此,你凭什么说你的资料就是对的。”

华和尚道:“是这样,因为我们的资料更直接。”他从贴身的衣服口袋里,掏出一块白鹃部,在我们的面前展开,我一看,不由心里咯噔了一下。

那竟然是那条拍卖会上的蛇眉铜鱼!

怎么会在他们手上,不是说没人买吗?我皱起眉头,忽然意识到了什么。

既然没人买,鱼又在陈皮阿四手上,那难道说,陈皮阿四是这条鱼的出售者?

我浑身颤动,竭力稳住自己的身体,不让自己表现出太过于惊讶的表情来。但是心里已经乱成一团,无数的问题在脑海里炸了出来,一时间也不知道是感觉到恐惧还是兴奋,只觉得手脚的突然的凉好像失去了血液一样。

华和尚并没有注意我的表情,继续道:“这种铜鱼,是龙的一种异型,是我们老爷子机缘巧合之下得到的,我相信,他应该是一个知道东夏国内情的人制作的,奇特的是,他通过一种非常巧妙的手段,因此了一段绝密的信息在这条铜鱼的身上,你们看。”

他将铜鱼放到风灯的一边,镏金的鱼鳞片反射出金色的光芒,在壁画上射出很多细细的光斑,华和尚转动鱼身。光斑便开始变化,渐渐的,竟然变成几个文字样式的斑点。

“秘密就在这里,这条鱼的鳞片里,一共藏了四十七个女真字。”

我心里啊了一声,心说竟然还有这种技巧,捏住我口袋里的另两条铜鱼,有的颤抖的问他:“是……是什么内容?”

“因为这上面的资料并不完整,我还没全部破译出来。不过我能肯定做这条鱼的人,想把某些事情纪录下来,而又不想让别人发现,这里。记载了真实的东夏历史。”华和尚有点得意的说,“其实,早在我看到这个东西前,根据很多的蛛丝马迹,已经推断东夏国这个政权一直存在着,只不过他们退回了大山的深处,而且在几百年里不知道依靠什么,这个极度弱小饿政权,在一边极端强大的蒙古和一边虎视眈眈的高丽之间留存了下来。我研究过高丽志,直到明朝建立之前,还有采参人在这里的雪山看到过穿着奇服的人活动。我想应该就是东夏国残存的部分居民。”

他又指了指铜鱼,说道:“这里的零星记载,证明了我的想法,东夏国在与蒙古决战后,退到了吉林与朝鲜的边界,一直隐秘的存在了几百年,总共有过十四个皇帝,蒙古和高丽不止一次的想把这个小国灭了,但是却因为一个奇怪的理由,全部失败了。”

“什么理由?”潘子问道:“和尚你讲话能不能痛快点?”

华和尚耸了耸肩膀,“我不知道,那鱼上的资料不完全,肯定还有其他的东西记载了另外一些部分,不过根据我手上的这几个字,我敢说东夏国能够存在下来,可能有非常离奇的事情发生过,后面就没有了内容。我们一直想找,但是很遗憾,我们老爷子找了很多年,都没有找到其他的部分。”他顿了顿,又说:“你们知道不知道,这几个女真字的最后一句是什么意思?”

我心说当然不知道,叶成接过去,问道:“什么?”

华和尚看着我们,说道:“上面说,历代的万奴王,都不是人。”

“不是人,那是什么?”胖子说道。

华和尚把铜鱼收了起来,“上面说,他们都是一种地底下爬出来的怪物!”

不是吧,我心里想,众人互相看了看,估计心里都有点毛起来,叶城问道:“那也不能这么说,会不会是说,皇帝是龙,而不是人这样的比喻?”

“我原本以为他是指真龙天子这样的比喻,但是后来研究起来,我发现这些人应该只是想一些秘密记录下来,对东夏的历史记录的比较客观,所以应该不会用这么恭敬的语言,而且,如果是你说的那样,你想像一下,如果你给皇帝贺寿,你先一句,陛下,您真不是人,恐怕你第二句没出就给剐了。没人会这么写。”他什么的笑了笑:“而且,后面这一句,写的非常清楚,非常唐突,我一直很介怀,如果能拿到另外的部分,这句话到底是什么意思,也许就能破译出来。”

胖子和闷油瓶都知道其实另外两条铜鱼在我手上,但是处于谨慎的关系,他们都没有出声,我抓紧口袋里的铜鱼,忽然觉得他们变的沉重起来。

一时间我也不知道自己应该不应该把这两条鱼拿出来,实际上这两条鱼对于我并没有意思,我并不会女真的文字,给我看我也看不懂,但是如果交给他们,我又感觉到十分的不妥当。

潘子盯着壁画,自言自语,壁画上可能是万奴王的那个人,人模人样,似乎并不是怪物,胖子拍了拍他,对华和尚说道:“刀疤兄,我说你破译什么啊,咱们是实在人,别搞知识分子那一套,到时候棺材一开,是人是狗,一清二楚。”

华和尚笑笑说道:“我的意思是,知己知彼,总是好一点的。”

“不过,画这壁画的人干什么要把这些东西画在这里?”胖子问道:“不忘国仇家恨?”

华和尚摇摇头,显然也不清楚,我想了想,说道:“有可能是想在这画好壁画后,将石头整块采下,或者干脆就是画来消磨时间的,你看这里这么暖和,可能当时的工匠利用这里来休息。”

没人给我说服,华和尚开始拍摄这些东西,以留做资料。

我们休息够了,精神逐渐恢复,开始轮流休息,陈皮阿四让他的人轮流出去在外面呆着,如果雪停了就爬进来叫我们,我们则开始轮流睡觉。

我睡醒的时候,顺子也已经苏醒了过来,一个劲儿的给我们道歉,胖子都懒的理他,我拿了东西给他吃,让他好好休息,我还得靠他继续上去。

在里面没有日月轮替,也不知道时间过了多久,大概是两到三天的样子,雪终于停了,我们陆续怕出了这条裂缝,外面已经放晴,到处是一片广翱的白色世界。

整顿装备,发现我们这几天吃掉了太多的东西,估计没有补给,不到我们要到的地方,就会断粮。问顺子有没有办法,他说雪线之上真的没什么办法,要不就回去在回来,要不分配食物,尽量少吃一点。

在缝隙里,陈皮阿四教了我们很多在雪山上的小技巧,比如说把卫生巾当成鞋垫,可以吸收脚汗,脚保持干燥,全身就会暖和,我们按他的方法,确实不错,不过我自己又觉得很别扭,想到如果进入古墓只中,将这些东西丢弃,若干年后考古队发现,看到棺材边上有这种东西是什么表情。

我们用绳索爬上滚下来的陡坡,地面上有不少新印的马蹄印子,胖子蹲下看了看,说道:“那阿宁那帮人看来超过我们了,跑到我们前面去了。”

我们二话不说,戴上护目镜,马上起程赶路。两个小时后,我们在一个山坡上,看到了阿宁的队伍。他们显然也遭到了非常大的损失,三十个人只剩下二十来个,马也只有一半数量,其中还是没有看到三叔的影子。

我们不动声色潜伏起来,观察他们。我看到阿宁正在用望远镜凝视一个方向,也想她那个方向看去,忽然眼皮一跳。

只见远处的不知道是雪气还是云雾中,一座雪封的大山巍然而立,与其他山脉连成一体,又显的非常的突兀,那正是我在海底墓中,看到的那一座山峰。他的形状,几乎和影画中的如出一辙。

“就是这里了。”我心里暗道,指着那山,转头问顺子道:“那里是什么山?要怎么样才能过去?”

顺子手搭凉棚,看了看,变色道:“原来你们要去那里?!那里不能去的!”

◆ 第五卷 云顶天宫(下) ◆

\chapter{五圣雪山}

躲过了暴风雪之后,我们再次起程赶路,在一处斜坡下发现了阿宁他们的马队,同时也发现了海底墓穴影画之中的那一座神秘雪山,赫然出现在了我们的视野尽头。就在我们询问向导如何才能到达那里的时候,顺子却摇头,说我们绝对无法过去。

“为什么?”我奇怪道,心说你不是说这八百里雪山,你每一座都上的去吗?怎么这一座又不能去了?

顺子解释道:“那座山叫三圣山,这山只有非常小的一部分在我们这一边,雪线以上到那一边,都在朝鲜的边境里,我们进不去。”

胖子愣了一下,问道:“我靠!不会吧!三圣山,难道就是当年彭总司令抗美援朝的时候,志愿军后勤部队建设战后生命线时候翻的第一座雪山?”

顺子点头道:“对,就是那山,海拔3400多米,翻过这山,就是朝鲜的丘陵地带。”

我一听,就心说坏了。

三圣山这个地方,当过兵的或对近代中国历史感兴趣的都知道,天下最难过的三条边境线,一条是印度和巴基斯坦,一条是以色列和黎巴嫩,还有一条,就是三圣山的这一条只有14公里长的边防线。

其实,中国和朝鲜两国历来是友好国家,熟悉的人都知道,在长白山的西坡可以非常轻松的越过边境线,并没有太多的关卡,在96年左右中国长白山林区萧条的时候,有很多人经常越境挖掘一种叫做“高山红景天”的中草药赚钱,虽然朝鲜兵也抓,但是中国人跑的溜,大打游记战,加上很多来偷挖草药的都但着烟酒,给转了也能用烟酒脱身。所以一段时间下来,西坡的这条边境线已经名存实亡了。

惟独三圣山的这一段边防线,却仍然封锁的非常严密。原因没有人知道,据说是因为这段边境线是中国与朝鲜的老边界,雪线以上就是朝鲜国境,抗美援朝的时候为了快速运输战略物资进朝鲜,山上修了很多的临时战略通道和地下工事。两芳都能通过这些通道迅速派兵,所以不严防不行。

现在我们的食物储备,不允许我们从边上海拔非常高的那几段边境绕过去,那唯一能赶上进度的办法,就是走直线从三圣山口直接过中朝边境然后进入雪顶。

那我们的麻烦,就不是什么玄之又玄的奇淫巧术和粽子,而是非常实在的81式自动步枪的子弹和少则排多则连的正规军。

其他几个人或多或少的也知道三圣山的情况,也都面露愁色,我们交换了一下眼色,合计着下一步怎么办?

潘子安慰我们道:“你们别急,边境上偷偷过境的路肯定有,在这里当过兵的顺子肯定知道,我们可以说服他带我们过去,到时候多给他点钱就行了。”

说着就去问顺子,没想到顺子竟然坚决地摇了摇头,说道:“不行,没可能,那边能上山的道路就这么几条,全部都是高岗,十米一个探照灯,从山脚上就全是军事禁区,虽然人不多,但是岗哨很密集,别说过境,你要靠近我们自己那边的哨子都不可能。我服役当时接到的命令,看到任何陌生人进入视野,马上就会朝天开一枪警告你,如果你还不退,第二枪就直接打你腿了,不带一点理由的。”

胖子问:“那咱们买点水果带上去,装成老百姓来慰问行不行?”

顺子笑道:“老板你也太会说笑话了。当然不行,一来这不是能混水摸鱼的地方,二来这里哪里去找水果,冰天雪地,我们提着水果到长白山的雪线以上,比空手还可疑。”

胖子啧了一声,说道:“那怎么办?这条破线就打死过不去了?我就不信。马其顿防线都给突破了,这还能有马其顿防线强?你他娘的是不是嫌钱少?需要多少你就直接说。”

顺子为难的挠头:“哎呀,这不是钱不钱的问题,要是真有办法,我还会和钱过不去?你们要想到朝鲜去,早说我就不带你们走这条道了,现在既然来到了这里,我真没有办法。”

顺子说的没有一点商量的余地,我们都有点意外,不过这一带并不富裕,这个边境也不是什么大罪,如果真有办法顺子应该不会瞒我们。

华和尚他们没什么主见,走到陈皮阿四边上,问老头子怎么看。

其实也就是继续走还是回去的问题,继续走的话,就必须象顺子说的,绕道其他的边防线,时间可能要延长一倍,而且最后几天得饿肚子爬山,不继续走就是回去休整,重新再来,也就是说这几天都白爬了,各种辛苦全部白费。

我自己倾向于继续走,不知道三叔部署了如此急迫行动的目的,阿宁他们的队伍又给了我很大的压力,脑子就希望能够早点见到三叔是完。当然当时有这样的想法,是完全不知道在饥饿中攀爬雪山的痛苦。

陈皮阿四叹了口气,显然也没有预料到这事情会这么麻烦,这些个长沙的老瓢把子,在自己的行里只手遮天,杀人放火什么都敢干,但是一碰到和官面上扯上联系的事情就蔫了,所以说贫不与富斗,富不与官争,他想了半天,也不说话,眉头就越皱越紧。

我有点着急,看了闷油瓶一眼,想问问他的意见,他却完全不参与我们的讨论,只是看着远处的雪山,不知道在思考什么东西,好象这一切都和他没有关系。

商量来商量去,一下子谁也拿不出个办法来,正在一筹莫展的时候,一边的叶成叫了我们一声。

我们停止说话,往山下一看,发现阿宁的马队又开始向前面移动了,看他们出发的方向,目标毫无疑问就是那三圣山。

很多的物资从马上卸了下来,随意丢弃在雪地里,大概是为了减重加快行动速度,山下的雪地里看上去一片狼籍。

叶成奇怪的说:“奇怪了,这些家伙不知道前面是边境线吗?他们的向导吃什么的?要真象顺子说的。背着这么多武器过去,不是给人家练实弹射击吗?”

我摇头表示不可能,我知道他们公司的习惯,肯定有当地的向导,而且也许不止一个。这样专业的私人考察公司最擅长的就是调研和公关,这里的形式他们了解的绝对比我们清楚,而且肯定在来之前就定下了固定的路线,不会轻易更改。

胖子怀疑顺子的业务能力,就问他这怎么解释?是不是有别人知道的路他不知道。

顺子眯着眼睛看了看道:“这样走只有一个可能。就是他们是想从前面的山口,绕到其他山上,然后饶过那段边境线,在朝鲜境内再转向三圣山,风险虽然也大,但是比冲击边防线要好很多。他们的队伍比我们庞大,食物充足的话,或者朝鲜方面打通关节的话,的确有这个实力做长途的跋涉。”

“那怎么办?要不要跟上他们再说?”叶成转头问陈皮阿四。

陈皮阿四摇了摇头,也不说话。突然指了指另一边,三圣山边上的一座白雪蔼蔼的小山头,问顺子,“那是什么山?”

顺子拿起望远镜看了看,道:“那是小圣雪山,那一座山是在我国境内的,三圣山和小圣山,加上还有那一边的大圣山,通称五圣。”

陈皮阿四又问道:“从这里走,能不能上到这小圣山上去。”

话音一落,所有人都一愣,都不知道这老头子想干什么,顺子也有点奇怪,道:“问题是没有,一天就到了,而且那里离岗哨很远,风景不错,就是路不太好走。”

陈皮阿四拍了拍裤子上的雪,站起来,对顺子道:“行,带我们去那里就行了。”

众人摸不着头脑,华和尚马上提醒道:“怎么了?老爷子,到那里去,太浪费时间,咱们没食物能维持这么久了——”

陈皮阿四摆了摆手,指了指一边连绵的山脉,道:“这里山势延棉,终年积雪而又三面环顾,是一条罕见的三头老龙,大风水上说这就是所谓的‘群龙坐’,这三座山都是龙头,非常适合群葬。如果这天宫是在中间的三圣山的悬崖峭壁上的,那边上的两个小龙头,应该会有皇后或者近丞的陪葬陵。”

三头龙的格局非常奇特,三个头必须连通。不然三龙各飞其天,龙就没有方向,会乱成一团,葬在这里的子孙就会兄弟残杀,所以如果有陪葬陵,陵墓之下必然会有和中间天宫主陵相通的秘道。

历史上有很多三头龙的古墓。比如说87年发掘的邙山的战国三子连葬,就是三个有关系的古墓分列同一条山脉的三个山头,两边的两个古墓本来都有大概半米直径的甬道通向中间的主墓,可惜当时发掘的时候,这些甬道都已经坍塌了,考古队不知道这些甬道是不是真的是相连,还是只是一个象征性的摆设。

我们顺着他的手看去,只见三座雪山山脉横亘在天地尽头,与四周的雪山毫无区别,不知道陈皮阿四的判断从何而来。

陈皮阿四说完,看了一眼闷油瓶,问他道:“小哥,我说的对不对?”

闷油瓶破天荒的对另人问话产生了反应,回头也看了一眼陈皮阿四,不过什么也没说,又转回头去继续看远处的雪山。

我们都不懂大头风水,听的云里雾里,心里感觉有点玄,不过既然老头子这么说,闷油瓶似乎也同意,那这一套最好还是别怀疑。

下到山下阿宁他们呆过的地方的时候,我们看到满地的废弃行李散在雪地里,很多都给翻掠过了,里面一点食物都没留下。显然所有的装备经过了重新的筛选,一些无用的,或者重复的东西都给舍弃了。

胖子甚至还找到了几把抢,但是里面子弹都给退干净带走了,只剩下空的枪身。胖子好着这枪,背起一把想带着走,被顺子拦住了,说你背着枪,在这里碰到边防军你就不好说话,如果没枪,给查到他能帮我们混过去,搞的胖子直叫可惜。

过了山下阿宁呆过的这片平坦的坡道,后面就山山谷,我们看到阿宁马队的足迹朝着山谷的深处延伸了过去。

我们也在这里整顿了一下。顺子就带着我们往另一个方向的小圣山口走去。很快,我们就走进了一片白色的世界,眼里看到的,就是满无天际的雪和难得看见的裸岩和冰锥。

长白山可能是世界上唯一一座可以走上去的雪山。这里比起昆仑山的冰川来说,环境要好上很多,没有那种有裂隙地巨大冰盖,不用担心脚下突然断裂,但是长白山的冰川也是典型的古冰川,山的连贯性不好,什么冰蚀地貌,臼洞,巨型冰斗,深不见底的冰井。反正我雪山地貌也没学好,说不出什么道理来,只知道经常一走就是前面没路了,万丈悬崖,得从边上绕或者趴着过去,走的也是惊险万分。

一路无话,看上去几个小时就到的直线距离,我们居然走了将近一天的时间才到。到达小圣雪山下山谷的时候,已经是当天的傍晚。

我们在山谷之上大概五六百米的雪坡上打了雪洞扎营,吃了点热的东西。高海拔处的星空无比璀璨清晰,陈皮阿四使用指北针,配合心里的天文罗盘已经天上的星宿排列大致定出了第二天走的路线。

一路走的人困马乏,但是天色尚早,胖子缠着顺子,问四周还有没有温泉。

顺子也惦记着温泉,不过他说这里海拔已经太高了,他也不常来,要找温泉有点困难,要是觉得无聊,倒是可以四处去走走找找,顺便还可以去看看古代先民冰葬的地方,在离我们扎营的地方一公里多的地方。

倒斗的总是对尸体有一种特别的感情,反正闲着也是闲着,听到有死人,我们都好奇起来。

陈皮阿四体力不行了,华和尚照顾他,其他人就跟着顺子往营地的左边的山谷走去,走了不到半个小时,来到一处悬崖,下面就是冰谷所在,一片漆黑,什么也看不到。

顺子找了个好地方停下来,打起一只冷烟火丢下去。

只见冰谷底部的冰层里,果然有很多蜷缩成一团的黑影子,密密麻麻,有的可以明显看出人的形状来,有些则只剩下小黑点,冰谷的四周,甚至还有一些祭祀的痕迹。

顺子说古代山里的村民都流行冰葬,解放初期都还有人葬入这座冰崖,所以现在有时候还有一些老人来这里拜祭。这里的冰川是逐年加厚的,所以你看最里面的尸体,那些几乎看不清楚的小点,恐怕有上千年的历史了,而最外面的就是几十年的。

我粗略数了一下我能看到的黑点,发现成千上万,显然这块冰冻的墓地在几千年的岁月中不知道累计了多少的死人,象这样的冰谷,小圣山谷内应该还有,那这座雪山岂不是就是一座特大号的坟山。

“这些尸体当中,会不会有当时修建灵宫时候的东夏奴隶?”胖子突然问。

“保不准有。”闷油瓶看着冰谷的深处,逐渐黯淡的冷光,不知道在想些什么。

尸体埋在冰中,也不可能去挖掘,我们看了一圈,索然无味,又去寻找温泉,倒是真给我们找到了以处小的,几个人在温泉中洗了脚和脸,浑身暖烘烘的回到营地,把情况一说,说的华和尚羡慕不以。

在雪山上,说来也奇怪,人一暖就犯困,人冻的要死的时候也犯困,晚饭是挂面,出完后困意袭来,外头又起了风,我们早早都进入睡袋休息,顺子守第一班岗,我们人多,不需要一天把人轮换完,今天轮岗的就是顺子、郎风和潘子三个人。

我很疲倦,很快就睡着了,满以为能睡一个甜觉,没想到没睡上一个小时,华和尚、胖子、郎风、潘子同时开始打起了呼噜,此起彼伏,就象交响乐一样,我做着噩梦就醒了过来。

这一下子就再也睡不着了,躺着又难受,我爬出帐篷,对顺子说我和你换换,你这一班我来,你先去睡一会儿。

顺子正自顾自在那里抽烟,看着一边月光下巨大的黑色山体发呆。听到我要换班摇头说不用,拿了我们的钱,这点还做不到就不好了。

我心说那随便你,掏出烟去乏,上去问他借了个火,然后一边往炉子里添了点燃料,一边和他开始闲聊。

与向导聊天是一件长见识的事情,我和他讲了很多古墓方面的事情,他很感兴趣。他也给我说了很多当地的风土人情和山林趣事,听的我一点也不觉得困,两个人越聊越精神。

后来就聊到了这一次的探险身上。顺子告诉我,他是七年的边防兵,不过有四年是预备役,在当兵之前,他是采草药的,所以对雪山很熟悉,他的战友都叫他“阿郎材”,意思是雪山的儿子。所以我们跟着他绝对可以放心,象这里的山,能带人进来的人不多,他算是其中一个了。

我心中怀疑,心说那你怎么还没进山就晕了,这肯定是吹牛,但看他说的一本正经,无谓去拆他的台,就顺着他的话听。

聊着聊着,话题多了起来,我们感觉之间的距离也拉近了,这时候,顺子突然就问我:“吴老板……其实,你们到底进山来是干什么的,你能不能告诉我?”

我听了就一楞,一下子不知道怎么回答,两个人就又静了下来。

我们的目的,我怎么说呢,说是来找云顶天宫的,你能信吗?说是来盗墓的也不行,说旅游的又摆明不是,这还真不好说,我想了好久,最后还是叹了口气:“你管这个干嘛,我不能说。”

顺子似乎预料到我会这么回答,笑了笑:“没关系,我只是随便问问。”

我心里觉得不舒服,因为我不想骗他,就随便转移了一个话题,聊别的。我问他既然以前是采草药的,为什么后来做了雪山向导了。

在长白山采草药很赚钱,比做这吃力不讨好的向导舒服多了,现在雪山向导这么少,也是这个原因。

顺子看了我一眼,突然说了一句让我几乎吐血的话。

他道:“我不是专业向导,我退伍之后一直在采草药,难得带几次人上山,也不会走的如此深,一般在姑娘湖那边就折返了,这里还是我第一次带队伍进来。”

我笑道:“别开玩笑了。”

“真的,吴先生,我实话实说,这个季节,没有专业向导会带你们进雪山,如果我不带你们进来,你们只有自己进来。”他朝我笑笑:“太危险了,如果不是菩萨保佑,其实我们已经死了,能一个不缺的到达这里,已经是奇迹了。不过你不用担心,虽然我没带人进来过,但是自己走过很多次,熟悉的很,不会出事情的。”

他说话的表情非常严肃,一看就不是在开玩笑,我心中暗骂,又奇怪道:“那既然这么危险,你还带我们来?你就这么缺这点钱吗?”

顺子意味深长的看了我一眼,道:“钱也是一个因素吧,还有一个原因……是因为我的父亲,他……十年前失踪了,当时他也是带一批人进雪山,和你们要走的路线差不多,但是最后整批人都消失在了山里,我隐隐约约就记得,当时找他的那几个游客,和你们的装扮很象,也是在冬天,也是非上山不可,所以我看到你们,就突然感觉到自己一定要跟着你们来,一来我不希望你们象我父亲一样死在里面,二来,我有一种很幼稚的想法,也许你们进山的目的,和十年前那批人是一样的,那也许我能够知道我父亲到底出了什么事情。当然,这只是我的臆想。”他自嘲的笑了笑:“我的父亲也许只是单纯的遇上了雪崩,给掩埋在这一片雪山里了。”

我领悟道:“所以你才问我们进山的目的……?”顺子不好意思的点了点头:“哎,你不明白,那种知道父亲就长眠在这片雪山里,却无法见到的感觉。”

我没想到顺子的内心还有如此细腻的时候,不禁有点刮目相看,以前一直以为他只是一个油嘴滑舌的普通导游而已。

不过十年前进入雪山失踪的游客,和我们打扮的很象,难道也是来找云顶天宫的?我心里咯噔了一下。不过随即又否定了自己的想法,不,不可能。在长白山里,能让一个人失踪的地方太多了,不可能有这样的巧合的,他的父亲,可能遇到了什么意外而在山里遇难了。

顺子看我不说话,以为自己刚才的那个问题问的有点过分了,对我道:“吴老板,我看你和其他人不一样,才和你说这些,希望这些东西你别和其他人讲。我怕他们会有顾虑。”

我心说我肯定不会讲你是第一次带人来这里,不说陈皮阿四会拿你怎么样,胖子都可能会打死你。

于是点头答应,这时候第二班的郎风从帐篷里走了出来,打了和哈欠,看到我们两个在聊天,很意外。顺子收拾收拾东西,在雪地里放了泡尿就去睡觉了。我和郎风无话可说,也打了和招呼回去睡觉。

在震耳欲聋的呼噜声中,我半梦半醒,梦到了十年前顺子的父亲,一个长着大胡子的顺子带着一群人上山的情形,离奇的是,在梦中,我总觉得那几个人我在哪里见过,翻来覆去,睡的很不踏实。

第二天天不亮,开始顺山脉走势继续往上走。

从昨天顺子的问题来看,他应该早已经知道我们不是普通的登山客,我知道我们伪装的也不好。最起码,没有哪个旅游的人会这么丧心病狂的赶路。但是我们也管不了这么多,反正他做长白山的导游,早有接待各种神秘团队的觉悟,这里每年的偷猎者,大韩民族朝圣者,偷渡采药人,没有一千也有八百,每个人都有秘密,我们是干什么的,就让他去猜吧。

山腰之上的路更加难走,很多地方的路都是斜的,头顶上又是万丈高的积雪山峦,极容易雪崩,不能大声说话。路上的雪又实在太厚了,几千年的雪层,下面几乎是空的,有时候一下人就捂进雪里,没到胸口,没人帮忙自己就出不来,我们只能小心翼翼的用长冰锥一点一点的打着脚窝,犹如在走雷区。

胖子脚程最快,这和他以前有过雪地探险的经验有关,他一路走在最前,因为高山反应,我们的舌头开始发麻,除了陈皮阿四偶然修改行进的方向,最后四周只剩下喘大气的声音,整个世界安静的似乎已经没有了生命。

过了山腰的雪路,我们走入到了一处两面都有巨型雪坡的冰封带,这里常年照不到阳光,雪都呈现冻土状,山的坡度越走越陡,温度极低,在里面,我们终于看到了陈皮阿四定的龙头宝穴所在,那是一处几乎与山成六十度锐角的陡坡峭壁,上面覆盖着皑皑白雪。

我们继续向上,一个接一个,尽量错开身形,开始使用冰锥冰锤,向那陡坡爬去。

这小圣山不在长白十六峰之列,所以我们来时候并没有太过注意,但是也不是无名的小峰,此峰和对面的大圣峰遥遥相对,中间形成一道山谷,矗立于三圣雪山的前面,犹如两个守门的卫士,这一景观被称呼为天兵守仙门。

从小风水来说,仙门两山虎踞龙盘,气吞万向,要不是处在中韩边境,历来纷争不断,这里也必然是一个皇宫贵胄墓葬的积聚之地。刚才一路走来,连我这样的水平,也看出这里山脉的奇特走势,但着一股劲道十足的龙气,我们对于山上有陵的假设,也更加的有信心。

爬陡坡不同走路,体力消耗更大,陈皮阿四爬了一会儿,体力到了极限,再也爬不动,郎风只好背起那老头子,我们走的就更慢。

又经过了大约三个小时的跋涉,我们终于登上雪坡,此时我已经完全失去神智,完全依靠条件反射跟着胖子。

胖子第一个到达,体力好如他也已经到达了极限,踩在上面的雪后,有点神智不清,装模作样的用力踩了个脚印,张开双手对我们说:“这对于我个人来说只是一小步,但是对于摸金校尉来说,是他娘的一次飞跃。”接着就趴进了雪里,一动不动。

我几乎虚脱了,双腿开始不自主的发软,人开始下滑,潘子想把我拉起来,但是拉了几下我都使不上力气,他自己也滚倒在地。

我用冰镐子用力敲了一记雪地,这才卡住自己,其他人也纷纷倒地,大口的喘白气,向四周看去。

这雪坡是一片巨大的区域,左右几乎看到不分界线,如果没有陈皮阿四指路。你绝感觉不到有什么特别的。上面雪覆盖的非常平整,只有几块黑色裸岩突兀而出。三圣雪山此时就在我们的左侧,比昨天看,近了很多很多,圣山的顶上覆盖着皑皑的白雪,整个巨大犹如怪兽的山体巍峨而立,白顶黑岩,显得比四周其他的山峰更加的陡峭,由于夕阳的关系,一股奇怪的淡蓝色雾气笼罩着整个山体,仙气飘渺,景色非常的震撼人心。

叶成一边喘气,一边感慨道:“太美了,难怪他们说蓬莱仙境,不及长白一眺,爬了这么久,也值得了。”

几个人都是粗人,但也都被四周的夕阳中的美景陶醉了,特别是在这雪山山峦,那种立于天顶之下的感觉就更加的让人感叹。

就在我想掏出相机,把这里的景色拍下来的时候,突然胖子拍了我一下,让我看他那边。

我顺着他的手指指的方向一看,只见一边的闷游瓶已经跪了下来,朝着远处的三圣雪山,十分恭敬的低下了头。原本面无表情的脸上,显露出了一种淡淡的,十分悲切的神情。

\chapter{自杀行为}

经过了两天的奔波,我们终于来到了小圣雪山的冰川山谷之中,所有人都是满头的雪沫,疲惫不堪,不过正值夕阳西下,登高眺望,四周的景色却让我们大大的陶醉了一把。

然而此时闷油瓶子的举动却让我们大吃了一惊,不知道为何他对着雪山跪了下来,行了一个十分恭敬的大礼,似乎对于这一座山,有着什么特殊的感情。

叩拜完之后,他又恢复了那种完事不关心,只睡我的觉的表情,爬上一边的裸岩,闭目养神。我不禁又好奇起来,真的是无法看透,他那混黑比见低的眸子里,到底隐藏了些什么呢?

一路过来大家都知道了他的为人,特别是我们几个,所以都没人去问他怎么回事,料想他也不会回答,不过从陈皮阿四几个人的眼神来看,显然也是觉得十分的纳闷。只有顺子不以为意,大概是以为闷油瓶也是朝鲜人了。

众人各有心思,一边看风景一边休息,片刻之后,我们的体力都有所恢复,胖子点起无烟炉,我们围过去烧茶取暖,同时顺子也开始做他的功课,喝了几口热酒缓过了劲来,便指了指周围的几座雪山,向我们解释了它们的由来。

这是他做导游的本分,这小子十分的敬业。

他说在长白山的传说中,这里的小圣雪峰,大圣雪峰和神秘的三圣山,在洪荒时代是一座雪山,大禹治水的时候,路过这里,用一把神刃劈了两下,才使得一座山变成了三座。

原本解放前的时候,这里还没有开放,他听他祖父说,这三座雪山上去之后,看到的四周的风景就截然不同。比如说在小圣雪峰之上,可以看到三圣和大圣两峰,而在大圣峰上,却只能看到三圣峰,看不到小圣峰,非常奇怪。而最奇特的,还是在三圣山上。除了能看到两边的两座大小圣峰之外,还可以看到在三圣山的后边,和其遥遥相对的,有一座比三圣山更加巍峨的雪山,叫做天梯峰。那一座山终年被云雾笼罩,不见真面目,传说山上有一道天梯,可以直达天宫,是人间和仙境的通道,如果天高气爽的时候,就能看到天梯峰与大小圣山之间会出现彩虹一样的霞光,犹如仙笔描绘,美轮美奂,奇异万分。

胖子听了,对我们道:“这传说肯定搞错了,天宫明明是在三圣山上,怎么会跑到天梯峰去了,传下这个传说的人肯定眼神有问题。”

华和尚想了想摇头,解释说:“不是,我来之前研究过这个传说,我猜这也许是云顶天宫修建的时候,天梯峰和四周的雪山的白雪产生折射形成的海市蜃楼,因为天梯峰终年有雾,大雾就成了反射的幕布,印出的云顶天宫的形象隐在雾中,好象天宫真的在天上一样。”

海市蜃楼这种现象大多发生在沙漠湖泊之中,雪山之中发生非常罕见,恐怕还不是偶然,可能是因为这里是龙脉的源头有关,这种现象在风水上叫做“影宫”,我只在一本古书上看到过一次,也不知道具体有什么讲究,宝穴这里,异向丛生,发生什么都不奇怪。

我们用方言对话,我用的是杭州话,华和尚他们用的是长沙话,顺子听不明白意思,也没留意去听。讲完风景之后,他站起来对我们道:“几位老板,你们先休息一下,吃点东西,然后想干什么干什么,但是得抓紧时间,天快黑了,这里也没办法打帐篷,天一黑路就不好走了,我们还得连夜找个比较平坦的地方,晚上还可能起风。”说着就倒了茶水,分给我们,自己很识相的站到远远的休息。

我看了看表,离太阳下山还有一个多小时,时间说长不长,说短不短,休息也休息够了,似乎该干正事了。

不过四周一片白雪覆盖,没有一点特别的痕迹,这里如果有陪葬陵,也肯定是被埋在了雪里,入口应该在我们脚下的雪层中。雪山环境和地面上太不同,我们都没有经验,不知道如何下手。

我们都站了起来,围到陈皮阿四身边,想商量下一步如何是好。

陈皮阿四经过这么强度的跋涉,还是没有缓过来,郎风拿着酒葫芦递给他,让他泯了两口。华和尚给他揉了揉后背,促进他血液的流动,他的脸色才逐渐缓和,但是整个人看上去还是非常的萎靡,听到我们问他,只是略微看了看四周的山势,对我们道:“宝穴的方位就在我们脚下,我也没有好办法,下下几个铲子看看雪下面有什么再做打算吧。”

众人点头,其实我也知道没有更好的办法,倒斗倒斗,万变不离其宗,寻龙点穴之后就是探穴定位,历代不同的只是探穴用的工具,过程都几乎是一样的,所以说如果没有开棺那一刻的兴奋,盗墓其实是一项枯燥的活儿。

雪比泥软的多,探铲打的很顺,华和尚他们手脚极快,很快雪地里就多出了十几个探洞,不过,几乎所有的铲子敲进去雪坡中五六米左右,就怎么也敲不动了,胖子以为叶成瘦猴一样没力气,跑去帮忙,用了蛮力,也还是只打进去一点,每次拔出来一看,铲子什么也没带上来。

华和尚看了看铲头,发现铲尖上粘着一点点的冰晶,就知道了怎么回事情,下面是冻土和冰形成的冰川面,和混凝土一样硬,铲子穿不透,自己也带不上什么来。

“这里下了几千年的雪了,雪积压多了就会成冰,你说会不会陪葬陵给冻在下面的冰里了?”胖子问。

我们都点头,很有这个可能,但是洛阳铲打不进冰里,就算知道东西在下面,我们也找不到。

潘子对我们道:“主要这些雪太碍事了,咱们有没有炸药,我当兵的时候听几个兄弟说,他们在大兴安岭的时候,那里的生产大队有定期的上雪山雪坡清血。只要一个炮眼,就能把这些雪全炸下去,省心的很。把雪炸了,雪下的情况就一目了然了,咱们再找就方便很多,也省得挖盗洞了。”

我知道他说的情况,每一次下雪形成的雪层。中间都有缝隙的,只要一个小爆炸,整个山体一震,整片的雪层都会滑下来,形成连锁反应,最后一层带一层的往下塌。

大兴安岭林区有几座小雪山,为免积雪太厚发生雪崩危害林区,当地的工兵队经常要在大雪之后人工清雪,当时条件简陋都是人员自己上去放炮眼,有一定的危险性,现在都是直接用迫击炮轰了。

华和尚捏了捏雪,“炸药我是有,但是你看咱们头顶,在这里放炮会不会是自杀?”

我们抬头去看,上面是高耸的万丈雪崖。前后一直延伸,连着整条雪龙一样的横山山脉,我们在这底下,犹如几只蚂蚁,实在太过渺小了,上面只要撒下一点点雪儿,我们几个就要长眠在这里了。

潘子看着也有点发晕,不过还是坚持道:“长白山是旅游景点,这里每点也会进行清雪,我感觉问题不大。你不大这些雪清掉,那咱们就乘早回去,你想在雪山里挖藏在雪里的东西,和大海捞针不是一样嘛,我告诉你藏地雪山找落难的登山队,也是这么做的,没有别的好办法,老头子醒了,就算定准了穴,你还得用炸药,不然盗洞也绝对打不下去。这一炮,你还就放定了。”

我想想不妥,不同意道:“这风险冒的太大了,我宁可花点时间用铲子来铲。”

潘子道:“小三爷,我们就是因为没时间了才用炸药,要有时间我们就等到夏天再来了,无论如何得试一试,三爷还等着我们去呢。”

华和尚摆了摆手让我们停下,指了指郎风:“你们不用吵,咱们说的都不作数,听听专业人士的意见。”

我楞了一下,一路过来,我并不知道郎风在他们几个人中是扮演什么角色,听华和尚这么说,我还有点奇怪,难道他是这方面的专家吗?几个人都看向郎风,看他如何反应。

郎风看我们看着他,有点不自在,对我们道:“我认为老潘的说法,应该可行,其实来之前我已经预料到会有这样的情况,我有一定的准备,而且这个我专业,我可以控制炸药的威力,声音也不会太响,其实只要在雪下面有一个很小的震动,就可以达到目的了,有的时候只要一个鞭炮就行了。”

“你确定?”胖子问道:“这可不是炸墓,咱们现在相当于在豆腐里放鞭炮,让你在豆腐里炸个洞,但是表面上又不能看出来,这可是个精细活。”

郎风点头:“我做矿工的时候,放炮眼放了不下一万个,这不算有难度的。”

华和尚看向我们,指了指郎风:“你们别看他平时不说话,这家伙是二十年的老矿工,十四岁开始放炮眼,炸平的山头不下二十座,给老爷子看中进到行内才一年,已经给人叫做炮神,说起炸药没人比他内行了。”

“你就是炮神?”一边的潘子睁大眼睛,显然听说过这个名号。

郎风有点不好意思的挠了挠头,一改前几日的冷酷劲:“都是同僚给捧的,一个外号而已。”

华和尚对他道:“你也不用谦虚,在这种场合你得发挥你的专长。”然后转头对我们道:“郎风到现在还没失过手,炸东西他说炸成几片就是几片,我绝对相信他,他既然这么说,我认为可以试一下,你们有没有意见?”

胖子看了看我,他有雪地探险的经验,听说过很多关于雪崩的事情,显然也觉得有点玄,不过他是我这一边,他看我是想我表态。

我想了想,陈皮阿四他们是真正的集团化职业盗墓贼,不象三叔还是比较传统,喜欢用老办法进古墓的人,这些人对于炸药的依赖程度是我们所无法想象的,而且华和尚这么说了,应该这郎风有相当的能耐。

于是心一横,就对胖子点了点头,拼了吧,此时其实已经没有退路了,我说不准炸没他们还能真听我的?

我们同意之后,郎风和叶成取出一只特别的洛阳铲,开始拧上一个特殊的铲头,在雪地上打了几个探洞。

然后郎风用几种粉末配比出了一种炸药,往里面深深的埋进去几个低威力雷管,我知道这种炸药本来就是专门调制的,威力大概只有十个炮仗左右,是用来钻孔破坏古墓的封石的,给郎风重新调过配方之后,威力肯定更小。

现代化的盗墓贼,大多都有相当的工程学知识,只要几和个很小威力的雷管,就能在任何地方炸出一个能容纳人通过的洞,这一点我早就知道了,不过亲眼看到倒还是第一次。

顺子在边上喝茶,一看这情景就疯了,他见过朝圣者插国旗的,见过偷猎的晚上偷跑出去的,也见过偷渡过境,但是千辛万苦跑上来,掏出雷管来炸山的,肯定还是第一次。跑过来一下拦住华和尚,大叫:“你们干什么?老板,你们疯了——!”

还没说完,郎风在他身后一镐子就把他敲晕了过去,顺子摔倒在雪地里,给拖到一边。

我看着觉得后脑发疼,心说当我们的导游也够惨的,路走的多不说,还要挨这个。不过想想也实在没办法了,顺子能容忍一切,也绝对不会容忍我们炸山,他毕竟还要在这里混下去。不把他敲昏没法继续开展工作。

胖子问华和尚:“我们以后拿这小子怎么办?”

华和尚道:“先不管他,我们还得要靠他回去,把他带到地宫里去,丢在一边就行了,到时候多塞点钱给他,他还能怎么样?”

郎风的表现极其专业,几乎就没让我们插手,他自己一个人干活,雷管根据一种受力结构的模型排列好,他挥手让所有人都爬到裸岩上去,以防等一下连锁反应把我们一起裹下去。

我原本以为雷管爆炸的声音会很大,至少得翻起一声雪浪,没想到郎风一按起爆器,我根本什么声音也没有听到,就看到平整的雪面一下子开裂了,然后大片大片的雪块开始象瀑布一样向坡下倾泻而去,坡度也一下子变得更加陡峭,我们脚下一下子空空如也。

不过这样的倾泻并没有持续多远,滚下去的雪片就停止了,雪坡下露出了一大片浑浊的,凹凸不平的白色冰雪混合层,这就是课本上说的四世纪古冰川表面。

胖子在边上闭着眼睛,一直还以为没有爆破,我摇里了摇他,他睁开眼睛一看,惊讶道:“亚哈,这真没什么声音,神了。”接着马上忐忑不安的看了看头顶。

不知道是幸运还是郎风的技术过硬,除了我们上方一点点的雪因为下面失去支撑而下滑之外,似乎没什么问题。等一会儿,都不见大的松动,我们逐渐松下心来。

我朝郎风竖起了大拇指,潘子也拍了拍他,做了个你厉害的手势。

郎风不好意思的笑了起来,可还没等他的嘴角裂的足够大,突然一块雪块就砸到了他的头上。

几个人脸色都一变,胖子急忙对我们挥了挥手,低声道:“嘘!”

我们下意识的就全静了下来,几个人又抬头一看,只见我们头顶上大概一百多米的高处,雪坡上,逐渐出现了一条不起眼,但是让人心寒的黑色裂缝,正在缓慢的爆裂,无数细小的裂缝在雪层上蔓延。随着裂缝的蔓延,细小的雪块滚落下来,打在我们的四周。

我顿时就浑身冰凉,知道出了什么事情了。

看来郎风“炮神”的这个名号,今天是要到头了。

\chapter{雪崩}

“所有人不准说话,连屁也不准放。”胖子用极其轻的声音对我们道:“大家找找附近有没有什么突出的岩石或者冰缝,我们要倒霉了。”

“不可能啊。”郎风在那里傻了眼:“我算准了分量……”

华和尚捂住了郎风的嘴,示意他有话以后再说。几个人都是一头冷汗,一边看着头顶,一边蹑手蹑脚的背上自己的装备,四处寻找可以避难的地方。这上面的雪层并不厚,就算雪崩了,也是小范围的坍塌,但是我们站的地方实在太不妙了,离断裂面太近,雪潮冲下来,很容易我们就会裹下去,下面又是高度极高的陡坡,连逃的地方都没有。

此时最好的办法,就是如胖子说的,找一块突起的山岩,躲到山岩底下,或者找一块冰裂缝,不过这应该从电影《垂直极限》里看来的,不知道事实管不管用。

我们所在的这一块裸岩太平缓,躲在下面还是会给雪直接冲击到,胖子指了指边上的那一块巨大的犹如核桃一样的石头,那下面和山岩有一个夹角,应该比较合适。

我们离那块山岩之间的雪坡已经全没了,剩下的是冰川的冰面,滑的要命,这时候也没有时间换冰鞋了,硬着头皮上吧。胖子把绳子系在自己腰上,一头给我们,自己就咬着呀踩到冰层上。

一步,两步,三步,每一次迈腿都象踩在鸡蛋上,我就等着“喀嚓”蛋黄飞溅的那一声。但是胖子这人总是时不时让人刮目相看,三步之后,他已经稳稳爬到了对面的石头上,拽着腰里的绳子,看了看头顶,招手让我们过去。

我们几个拉着绳子,先是潘子和闷油瓶,接着是背着陈皮阿四的郎风,再就是背着顺子的叶成,我是最后。看他们都平安的过去了,我心里也安了很多。此时上面已经有大如西瓜的雪块砸下来,那条雪缝已经支持不住,胖子挥手让我快。

我拍了拍自己的脸,把绳子的另一头系在自己腰上,然后踩上了第一脚,站上去稳了稳。

我自小平衡性就差,滑冰骑车样样都非要摔到遍体鳞伤才能学会。此时就更慌了,只觉得脚下的冰面,似乎随时都有可能消失一样,不由自主的,脚就开始发起抖来。

胖子一看就知道我是最难搞的货色,低声道:“别想这么多,才两步而已,跳过来也行啊。”

我看了看胖子离我的距离。果然,只要能够充分发力,绝对可以跳过去。想着我一咬牙,就垫步拧腰想一跃而起。

可没想到的是,就在一使劲的时候,脚下突然就一陷,我踩的那块冰,因为刚才踩的人太多,一下子碎了。我的脚在斜坡上打了个滑,接着整个人就滑了下去。

我手脚乱抓,但是冰上根本就没有什么地方能着力,一下子我就直接摔到绳子绷紧,挂在了冰崖上,就听登山扣子咔嚓一声,低头一看,卡头竟然开了,眼看身子就要脱钩。

我心里大骂,他娘的这西贝货,肯定是义务生产的!

胖子给我一拉,几乎就给我从石头上面拽下去,幸好潘子抓住他的裤腰带,几个人把他扯住才没事情,他们用力拉住绳子,就把我往上扯。

但是每扯一下,绳子就松一下,我心急如焚,我双脚想蹬个地方,重新系上扣子,但是冰实在太滑,每次只踩上几秒就滑下来,人根本无法借力。

眼看着这扣子就要脱了,万般无奈之下,我扯出了登山镐,用力往冰崖上一敲,狠狠定在里面。然后左脚一踩,这才找到一个可以支撑的地方,忙低头换登山扣,还没扣死,突然一阵古怪的震动从我头顶上传来。

我抬头一看,就看到上面的几个人用一种白痴的眼神看我。还没等我反应过来怎么回事情,霎时间,只见一片白色的雪雾一下子炸到了半空,几乎遮挡了我的整个视野。

雪崩了!

没有惊叫,没有时间诧异,那一瞬间我不知道自己在想什么,只听到胖子在边上大叫了一声:“抓住登山镐!贴着冰面!”然后一下子我的四周就全黑了,我的身子猛的一沉,似乎突然十几个人拉住我全身的衣服往下猛扯,腰部的绳子顿时就扣进我的肉里,然后大量的雪气就呛进了我的肺部。

接着,我就陷入到了一片混沌之中,巨大的冲力撞击着我身上的每一个地方,我连头都抬不起来,很快喉咙开始发紧,极度的窒息感觉从我肺部传来,我只感觉我就象是被扔在糖炒栗子机里,无数冰冷的东西从四面八方积压我,砸我,一瞬间,鼻子、嘴巴里全是雪沫的味道。

这时候我才想起来,冰是绝好的传震导体,特别是极其厚的冰,有极其强的共鸣性,刚才那一镐子,终于催化了雪崩的形成。

我几乎想抽自己一巴掌,但是此时后悔已经没用了,整个人象陀螺一样给撞的到处打转,我想抓住登山镐,但是连我的手在哪里都感觉不到。

就在脑子发蒙,不知道怎么办的时候,突然,我感觉到绳子竟然给人往上提了一提,接着我的身体竟然也朝上拉起了一点。

我心中一惊,那是胖子他们在那一头拉我,我顿时燃起了希望,绳子还能反应,说明雪崩下来的雪量不是很厚,他们的力气还能传导到我这里来。

我忙用力扒拉四周的雪流,把身体往上钻,几次趔趄之后,借着绳子的拉动,我的耳朵突然一阵轰鸣,眼前一亮,探出了雪流的表面。

胖子他们躲在一边的岩石夹角下,雪流从石头上面冲过去,在他们面前形成了一个雪瀑,几个人都安然无恙。胖子和郎风扯着绳子,看见把我拉了出来,大叫了一声,问我:“没事情吧?”

我大口的喘气,点了点头,一边的还是漫天的雪雾扑头盖脑的朝我砸下来,我用力扯着绳子,顶着雪流开始向他们那边靠拢。但是雪流力量太大,我根本无法站起来,两只手再用力也无法移动半分,胖子只好拉着我,等待雪流过去。

雪崩来的快,去的也快,半分钟不到,雪流就从我的身边倾泻而过,只留下大量的碎雪。我朝下看看,脚下整个山谷都给白雾笼罩了,不由后怕,要给冲了下去,现在哪还有命在。

我给拉到岩石之下,几个人都心有余悸的喘着大气。胖子拍了拍我道:“你小子真的算是命大了的,幸好这只是坍塌,雪量少,不然这一次不仅是你,我也估计得给你扯下去。”

我也不知道自己是什么表情,登山帽都掉了,耳朵冻的发红,什么也听不清楚,只好拍了拍他,转头去看一边的冰川表面。

整片的雪坡已经全部倾泻到了山谷的下方,一大块巨型的陡坡冰川暴露在了我们面前,不时还有碎雪从上头滚落下来,提醒我们还有二次雪崩的危险。

冰川的表面都是千年雪层底下受压而成的雪成冰,也就是我们常说的“重力冰”。这种冰是自然形成的,在高海拔山区会包裹在整个山体上,形成冰川,一般雪山上都有,处于雪层和山体之间,不会太厚。冰层之上还有大量的碎雪。

除了胖子,我们从来都没见到过实际的冰川,在雪山山谷中,见到如此巨大的一块冰崖暴露出来,在夕阳的照耀下,犹如一块巨型雕牌超能皂,实在是一件让人震撼的事情,我们看到都有点发痴了。

叶成在一边喃喃道:“郎大这一炮,倒也不是没有成果。”

看了片刻,众人逐渐反应过来。华和尚亮起几只手电,朝冰里照下去,想寻找陪葬陵的痕迹。里面混混沌沌,深不见底,一般的雪山冰川几乎只有一二十米的厚度,这块冰川的厚度似乎有点异常。

胖子眼睛很毒,这时候,突然咦了一声,似乎发现了什么,从华和尚抢过手电去照。

我们吃力的顺着他的手电看去,在微弱手电光线的穿透下,我看到胖子照的方向下,呈现暗青色的半透明的冰川深处,竟然有一个若隐若现的巨大影子,几乎占了半壁冰崖,看形状,象是一个蜷缩的大头婴儿。

\chapter{昆仑胎}

夕阳逐渐西下,只有一点点的太阳还冒在云头上,整块冰层已经逐渐变成了黑色,里面的巨大影子模糊不清。

影子的形状非常奇怪,不伦不类,诡异非常,象是什么冻死的动物幼胎,脑袋大的要命,浑身还长着长刺,看着心里就发毛。

叶成张大嘴巴问我道:“他娘的,出来没拜菩萨,老是撞邪,这是什么鬼东西?”

我和胖子摇头,我们也从来没见过,看大小,这东西足有一幢五层小楼的大小,冻在冰川深处,要是陪葬陵,是怎么修进去的呢?又或者难道是远古时候的生物?

传说长白山地带在几十万年前还是汪洋一片,是靠主火山体喷发,才从海中隆起,这么大的东西,会不会是当时巨型海洋生物的尸体呢?

想来也不对,古冰川形成的时候,山早就在了,有尸体也早成化石了。

虽然经历了一次惊心动魄的雪崩,但是说实在这样的雪崩其实只能叫积雪滑坡,并没有雷霆万钧之势,去的速度又快,几个人虽然也心有余悸,但是此时都恢复了过来,看到冰中的影子,好奇心都给勾起。

我们使用冰锥,在冰川上打上立足的地方,套上绳子,穿上冰鞋,下到冰川的表面,仔细去看冰川内冻的诡异黑影,但是几个人怎么都看不出门道来。

此时陈皮阿四也恢复了意识,华和尚和叶成扶着他也从上面下来,我们小心翼翼的搀扶他到了跟前。

陈皮阿四反应还是不快,揉了揉眼睛,蹲了下来,盯着那冰盖里的影子看了半天,突然嗯了一声:“这影子……难道是‘昆仑胎’?”随即又摇了摇头。

“什么是‘昆仑胎’?”我们都没听说过,看他如此激动,简直莫名其妙。

“‘昆仑胎’是一种奇怪的自然现象。指在龙脉的源头,也就是俗话说的,集天地之灵气的地方。往往在岩石、冰川、树木之内,会自己孕育出一些奇怪的婴儿状的东西出来,这些古籍里就叫做‘地生胎’。传说经过万年的衍化,有些‘地生胎’就会成精,比如说西游记里的孙悟空。”华和尚给我们解释。“我记得在唐朝的一本笔记里提到过。西汉末年,传说在昆仑山的巨型冰斗下底下,当地藏民发现过一个巨型冰胎,大如山斗,五官已经具备,还是一个女婴,栩栩如生,于是‘地生胎’就被叫做‘昆仑胎’,后来还在那女婴的肚脐眼上修了个庙,叫做昆仑童子庙。风水中,‘昆仑胎’是天定的宝穴,和人为推断出来的风水穴位是不同。要找到一条龙脉中可能生成‘昆仑胎’的地方,是不可能的,只有等到‘昆仑胎’开始形成,偶然给人发现,然后将胎形挖出,再把陵墓修建其中。这样的宝穴是可遇不可求的。传说只有通天的人才有资格。历史唯一记载埋在‘昆仑胎’位里的人,只有一个人,那就是黄帝。”

“还有这么邪门的事情?”胖子蹲下来,看着那个影子。“不过,这个‘昆仑胎’不型是人的胎啊。”

陈皮阿四也似乎并不能肯定,点头道:“我也是猜测,‘昆仑胎’是神定胎位,地生神物,如果这个是‘昆仑胎’,那陪葬陵,必然会修建在了‘昆仑胎’位内,不过这样一来的话……”他看远处的三圣雪山,眼睛里现出极端的迷惑。

我知道他的顾虑,接道:“这里是天生的宝穴‘昆仑胎位’。但是这里只是一座陪葬陵啊,那这样,云顶天宫主陵所在的三圣山,风水要好到什么程度才算完?再怎么样也不能比‘昆仑胎’差啊。”

“是啊,没有比‘昆仑胎’更好的风水了,‘昆仑胎’是大地灵气汇聚的地方,如果要比这里更好,那只有一个可能。”陈皮阿四很疑惑,叹气道,“天宫,真的是修建在天上!”

陈皮阿四说这句话的表情很真切,我看的出不是戏谑之言,我给他说的浑身发寒。胖子就道:“怎么可能!”

“是不可能,所以这里出现‘昆仑胎’,绝对有问题,难道山川的走势,给他改了,汪藏海竟然神通到了这样的地步?”陈皮阿四又四处去看周围的山势。

“不,不应该这么样想。”我突然有了一个想法,问道。“会不会这个胎形的影子——是假的?人工修出来的?一种象征性的手法,在古墓葬的设计中很常见。象武则天的城形,就象女人的乳房,说不定这影子,只是陪葬陵的影子。”

我是很自然有这样的想发,因为我们做古董的,平常的工作就是与假的东西作斗争。我们采购的时候,所以的东西第一感觉都是假的,所以我听到陈皮阿四说的这么厉害,第一印象就是:会不会作假的?这也算是职业病了。

况且,把陵墓的入口冻在土里,修成婴儿状,的确符合汪藏海事不惊死不休的性格。

陈皮阿四注意力全在了四周的山脉上,根本没听我说。我转头看向闷油凭,后者也脸带疑惑,表情复杂的盯着那影子,也不吱声。不过华和尚很同意我的说法(看样子他也应该是采购第一线的人员,和我一样有着职业病)。他道:“你说的有可能,看着‘胎影’之中还有浅淡之分,显然不是一个单纯东西,似乎有高低高矮,而且四周还有刺,无法解释是什么东西,可能真的是建筑。”

我心里泛起一股奇妙的感觉,汪藏海把陵墓,修成了胎儿的形状,难道是希望这座陵墓象“昆仑胎”一样成精吗。

这事情如果是真的,那就太匪夷所思了。

胖子道:“还是不要猜了,反正不挖出来,怎么猜也都是猜,有这闲工夫,不如想个办法下去。”

“那要是挖下去,看到的不是陪葬陵,而是一具真的巨型冰——”叶成有点害怕。牙齿打结:“那怎么办?”

胖子拍了拍他:“那你就留在上面,我们下去确认了,再叫你下来。”

我也道:“如果真是个冰胎,那真是天作的奇迹,能看到一眼也是值得的。”

华和尚拍了拍叶成,道:“就你胆子小,学着点这几位大哥……现在的问题不是去不去,而是怎么下去?”他目测了一下冰的厚度。道:“用镐子挖,半个月都不一定挖的到那里。”

我们又不是冰夫子,在冰上作业完全不同于一般的地面。要考虑到非常多的因素,平时身手再好也没有了。

胖子盯着脚下冰川中巨大的影子,对我们摆了摆手道,“这有什么难的?就交在我胖子身上。”

我看他似乎有点眉目的样子,心中好奇,胖子在队伍中一直是充当急先锋的角色,很少在技术方面发表意见,但是一但他发表意见,所提出的东西就非常关键,说明这个人的心思其实相当的细腻。我在海底已经深切的感受到了这一点,这恐怕也是他如此贪财却还能够屡次化险为夷的品质之一。但是于胖子这个人说话需要技巧,他是属于软硬不吃的那一种人,大多数时候,激他比奉承他有用多了。于是对他道:“你能有什么办法?”

他果然就有点不爽,对我道:“什么话,就许你大学生有想法?我去过昆仑山,昆仑山上多冰,比这厚的冰川多的是,经验比你丰富多。”

我笑道:“那你说出来听听。”

胖子就哼哼着和我们讲了他当时的向导和他讲过的。很多关于冰的故事。昆仑山的海拔比这里要高的多,是真正的高山冰川,那里大大型冰缝因为气温和山体运动会频繁发生开裂,有时候裂缝中就会发现古时候奇怪的先民遗骸,甚至有人发现过冻在冰川深处的房子,但是这些东西都是坍塌的,只是残骸。

他当时问为什么这么冷的环境下这些古代遗迹都保存不下来,那向导就对他说,把一座建筑完整的冻在冰里是不可能的,特别是木结构的房屋,遭遇冰崩或者雪崩的时候,肯定会先坍塌。

现在我们脚下冰川中的建筑必然是修建在悬崖上的,这里面的黑影看上去如此的完整,轮廓象极了婴儿,就说明下面没有坍塌的迹象,不然那种架空的建筑,一塌就完全不成样子了。所以,除非冰川中的不是陪葬陵而是石头,不然,这陪葬陵冻在冰里就肯定不是雪崩,而是人为造成的。

胖子的理由非常充分,我点头同意他的说法,不过其他人并没有听出胖子这个假设的意义来,潘子问他道:“那又怎么样?”

胖子摆手道:“如果不是雪崩,那修建陵墓是在九百多年以前,按照道理,九百年累积的雪压冰绝对不可能这么厚,所以这些冰肯定是人为的,我们脚下肯定是一片非常厚的人工冰墙,这冰墙又不可能直接压在建筑上,那肯定有一个弧度,形成一个天然冰穹,压在斜坡上,保护着下面的建筑。类似于冰做的封土堆,冰没有我们想象的厚,你看,这里的冰透明度很好,也是一个证据。”

胖子一说,众人哗然,一个个都对他刮目相看,同时就突然感觉脚下不稳当了很多。

胖子还惦记着我刚才看轻他,又知道我是学建筑的,就问我他说的说法可能不可能。

我点了点头,说理论上解释的通的,而且有可行性。用冰来构架房屋,世界上很早就有了。三国的时候曹操一夜城就是冰加稻草造的,爱斯基摩人也早就用冰来搭建自己的房屋,最近在丹麦好象还有现代的冰建筑出现,说明冰的硬度在建筑学上是绝对没问题的。

不过曹操一夜城是在平原上,要在峭壁上搭建如此宏伟的冰穹,真的可以实现吗。我又有点保留,毕竟是在1000多年前左右的时候,汪藏海就算能超越他们那个时代很多,也不应该牛B到这种程度。

胖子听我同意他的看法,马上就得意起来,甩了甩头发,道:“瞧,胖爷我这就叫人才。”

叶成问我道:“吴家少爷,那能不能根据建筑学,算出这冰穹的可能厚度?”

我大学里大部分学的都还给老师了。不过单位体积冰的重量我还知道。心里默算,套用了几个公式一算就出来一个数字,对他道:“如果象胖子说的,假设使用木头的支撑结构,那我们脚底下冰层的厚度不会超过十米,不然自重太重,会自我坍塌,用什么都撑不住。”

“十米。”几个人面面相觑,潘子道:“我操,那也够呛了。这儿的冰和其他的地方不一样,硬多了,我们没专业设备。刚才我和郎风用铲子用力敲过冰锥,敲了几下,手都麻了,只敲出几个白印,要打穿十米恐怕得花上点时间,一个星期可能都不够。”

重力冰和其他河床上的冰不同,河床冰的原料是河水,里面有杂质而且含有大量气泡。河床的温度也不会太低,但是重力冰是给千年雪一层一层压成的,不仅杂质少,而且雪层底下的冰温可能有零下50多度,在这个温度和纯度下,冰的硬度和密度是非常可怕的。

胖子道:“我们不是有炸药吗?干脆我们爬到石头上去,再放个炮眼得了。”

华和尚和我马上摇头,我想着刚才差点就死在雪里,没好气的对他道:“你还真不长记性,刚才还没尝够味道啊?而且,如果冰川是空心的,再小威力的一个爆炸,也可能把整个冰穹给炸裂了——如果你的假设是正确的,破坏力太大的方法来打洞就不能考虑,挖到关键的地方,可能连冰铲都不能用,一旦弄不好就是连锁反应。”

胖子对理论科学非常反感,道:“你这是本本主义,冰铲都不能用,那怎么办?难道用调羹来挖?你不要仗着自己是大学生,在这里危言耸听,人为给咱们制造难题。”

我道我比你还急呢,但是事实就是事实,谁要是不信,大可以试验一下。

一个问题想通了又来一个问题,一下子气氛又沉闷起来,众人都不说话,开始想解决办法。正犹豫不决,突然闷油瓶拿着顺子烧茶的无烟炉走到了我们边上,往边上一放,滚烫的炉身马上和冰冷的冰面起了反应,发出啪啪的声音,问我道:“这样行不行行?”

我一看,心里说哎呀,对啊,他娘都冻驴了,没想到这办法,用火不就行了嘛。

冰的硬度和温度直接相关,温度一升高,硬度就会下降,冰墙表面就开始变脆,冰铲敲击造成的连锁反应就会减弱。我们可以一步一步来,先把表面的冰烘软,然后整块的敲下来,露出更里面冻的严实的冰芯,然后继续用无烟炉烤,重复直到砸通为止。

实践是检验真理的唯一标准,我们马上做试验。掏出自己的无烟炉,点起来放到冰上,一分钟后用铲子削冰。果然,书上说的没错,脆化的高温冰会整块的裂开。

不过因为四周气温太低了,这样做的进展非常慢,我们轮流尝试,直到将近三个小时,天几乎全黑的时候,墙上才给我们捣鼓出了一个半米宽,七八米深的凹陷,下面冰层的颜色明显变化,冰的纯度也清澈了很多,已经可以肯定胖子的说法对了一半,绝对不是自然形成的冰。

胖子腰上绑着绳子双脚撑在冰井两边,最后用无烟炉烤了一下井底的冰面,然后用短柄锤子一砸,想再砸下一块来,没想到“啪”一声,冰穹裂开了一条缝,一下子我们感觉外面的空气涌向那个破洞,吹起了一阵风,温度陡然就凉了很多。

胖子又一砸,将底下的冰块砸碎,碎冰跌落而下,果然出现了一个洞口,下面是空的!

众人都松了口气。连胖子自己也惊讶了一声,叫道:“还真给我猜对了。”

我们将他拉了上来,所有人围拢到洞口,争先恐后的拿起手电筒朝里面照去。

冰井之内,是一个灰蒙蒙的巨大空间,整个冰穹犹如一个透明的碗扣在一道峭壁上,无数挂满冰棱的木梁从峭壁的山岩上竖起来,交错在一起,形成类似于脚手架的结构,撑着外面的“冰碗”,这些就是胎影身上的刺,峭壁之下是看不到底,漆黑一片的深渊。

而在大概一百多米落差下的峭壁山腰,我们看到了那黑色胎影的真身,那是一个巨大的胎形山洞,也不知道是人工修造的还是天然形成的,洞口足足有一个标准游泳池这么大,乍一看,象极了一个黑色的巨大婴儿。

我们看的惊呆了,几个人都几乎说不出话来,胖子眼睛很毒,抓住我的手电,移向一个方向。“看这里!”

在他的引导下,我们眯起眼睛仔细去找,这才看到在那山洞之中,竟然还修建有一座横檐飞梁的巨大宫殿,有一部分建筑探出了洞口,用木头廊子支撑在峭壁上,犹如悬空的空中楼阁,而大部分的建筑修建在山洞之中,看不到全貌。

因为常年在低温中,到处凝结着冰屑,露出洞口的那部分建筑看上去灰惨惨的,并不明显,所以粗看并不容易发现。

这是陪葬陵的灵宫,也就是摸金校尉口中常提的龙楼宝殿,陵墓中的“陵”这一部分,而埋着墓主人的墓,应该是在这灵宫的底下,山体之内。

我不禁感慨,还以为这里最多只有一个隐蔽的地宫入口,没想到万奴王的排场这么大,陪葬陵都设了如此巨大的灵宫。那如果云顶天宫没有给大雪覆盖,将是怎么一幅壮观的景象?真的无法想象,古人的智慧无法不让人感到畏惧。

胖子首先反应过来,大笑了起来,接着其他人都笑了,大家互相击掌庆贺,我给胖子的屁股一撞,差点从冰上滑下去。

华和尚急忙阻止了我们,他指了指头顶的雪崖,意思是小心再塌方一次,我们全部都在冰崖之上,一个也逃不了。

我们这才强忍住了心头的激动,安静下来,但是几个人的脸上全是按耐不住的狂喜。

现在想想,盗墓贼,就算是天大的盗墓贼,有几个人能盗掘到皇陵这种档次的,如果能进入皇陵一次又能安生出来,已经不会去在乎里面有什么宝贝,就这腕儿你就大了,不说吹牛能吹多少年,自己的心态肯定就不同,这种吸引力,谁也抗拒不了。就连还没有自定是盗墓贼的我,也有一股极度的冲动在心里涌上来,简直迫不及待想下到下面去看看。

华和尚拍了拍脸,想让自己放松下来,然后转头问陈皮阿四,我们是现在下去,还是明天下去。

陈皮阿四阴阴的看了我们一眼,问道:“明天下去,你们忍的住吗?”

\chapter{胎洞灵宫}

我们整顿装备,把无烟炉熄灭收好,所有的镐子、铲子都折叠起来,几个人都似乎有了默契,速度非常快,很快都收拾妥当,集中到了我们挖出的破口周围。

这是人有了共同目标时候的典型表现,其实说起起来很幼稚,收拾的再快,与是不是能早点下去一点关系也没有,因为谁也没有碰过皇陵,再怎么样也要经历一个熟悉的过程,不过当时就是觉得不能让别人抢先了。

所以就出现了可笑的一幕:围到破口周围之后,大家突然都不知道怎么办了,就好象很多人商量了半天去哪里玩,决定之后发现谁也不认识路一样。几个人面面相觑,都有点愕然。

我看着洞内,心里稍微分析了一下,其他倒还好,有一个致命的问题是,我们所在的位置开在深渊的正上方,离灵宫所在的胎洞有一百多米的落差和二十多米的横向距离,我们虽然有足够的绳索,但是无法越过这横向二十米——靠荡是荡不过去的。

身后的陈皮阿四看到我们这个样子,冷笑一声:“一群没出息的。”说着站了起来,让我们都让开。

我在心中暗笑,陈皮阿四的老人心态还是无法避免,一直以来我们都以他马首是瞻,刚才胖子露了一手之后,他难免心里不舒服,这时候看到我们这样,就忍不住要口出恶言,来挽回自己的地位,这是很多老人普遍的心态。

我们给他让开一个缺口,华和尚自嘲的一笑,道:“老爷子,小的们不是都乐昏了嘛,没见过这么大的阵势。您说这斗……该怎么个倒法?”

陈皮阿四给叶成搀扶着蹲下来,看了看破洞之内,道:“万变不离其宗。小心为上,咱们先找一个人上这些撑着冰穹的木头廊柱,顺着廊柱爬到山洞的上方,然后用绳子下到外面架空的建筑瓦顶上。”

我们看向结满冰的木头廊柱,每一根廊柱足有100多米长,绝对不是一棵树的原木,肯定有木锲子把几根木头连起来,这样的结构承压不成问题,但是不知道能不能承受拉力。如果不行,那就完蛋了,一根木头廊子坍塌之后,下落的过程当中,必然会砸到其他的廊柱,到时候整个冰穹都可能会塌,这样的方法还是十分的冒险。

但在当时,大家都急着想下去,也没有过多的考虑这些事情,而且,似乎其他也没有更好的办法。

这里适合趟雷的只有潘子,其他都无论身手体重都不合规矩,所以潘子只好挑起这个大梁。

我们在他腰上绑上蝴蝶扣的绳子,身上只带一些轻量的装备,潘子看上去有点兴奋。陈皮阿四给他传了一口酒喝,让他镇定一下,道:“千万别乐昏了头,咱们目标不是这里,下去招子给我放亮点。”

潘子点点头,深呼吸了口气,就小心翼翼的爬入冰井,然后用飞虎抓子绕上一边的木头廊子,象特种兵荡绳一样荡了过去,一下子爬上木廊柱之上。

一踩上去,木头廊柱就发出一连串让人十分不舒服的冰块爆裂声,我们顿时都屏住了呼吸,潘子也脸色惨白的一动不动,惟恐廊柱解体断裂。

然而幸运的是,等了有十几分钟,廊柱的那种爆裂声停住了,四周又恢复到一片平静,受力又重新恢复了平衡。

我也想也是,可能是自己多虑了,上面的冰穹如此沉重,木廊子之间的压力非常大,我们就象蚂蚁一样,应该问题不大。

几个人都松了口气,给这么一吓,我们都清醒了一点,那种莫名的激动有一定程度的减退。

潘子继续向前,走的更加小心,几乎是在跳一种节奏极其缓慢的舞蹈。我们的心也跟着他的步伐跳动。好不容易,终于走到了廊柱尽头的山崖石上,下面一百多米,就是山洞的所在。

我们给他打下去的手电光太发散了,潘子打起五六直荧光棒,一只一只往下丢去。

黑暗中几道光直落向下,有几道象流星一样消失在了深渊的尽头,有几只掉落十几米后,撞在了瓦顶上,弹了几下停了下来。同时荧光棒里面的化学物质因为剧烈震动而发生反应,光线越来越亮,隐约照亮了冰穹里面的情形。

接着潘子丢下绳子,一只垂到了下面瓦顶,然后迅速的滑了下去。

看着潘子稳稳的落在了瓦顶之上,我们的心才放下,潘子朝我们打了几个手势,意思大概是这样的过程安全。

我们又开始兴奋起来,接下来第二个就是华和尚,我们陆续小心翼翼照葫芦画瓢,一拨一拨有惊无险的下到了瓦顶之上。

一百米的平衡木和一百多米的绳索攀爬不是儿戏,我到下面之后几乎站不稳,要潘子扶住我才能在琉璃瓦上站定。会议起在冰木廊柱上的感觉,我的腿不由自主的就开始发软。

七只手电四处去照,发现这一座冰穹中的斜坡峭壁大概30度的近垂直,山洞很深,宫殿直入山体内部,看不到最里面的情况,山顶和灵宫之顶几乎贴合,我们所站的瓦顶是其中最外面一层架空“大殿”的屋顶,檐头的飞檐都是朝凤龙头,屋脊两边是镇宅的鸱吻,黄瓦红梁很有皇气。我们几个人歪歪扭扭的站在上面,大有周星驰版决战紫禁之巅的感觉。

胖子想去掀一片瓦片看看,却发现瓦片和瓦梁冻的死死的,根本掰不下来,只得作罢。我们又一个一个小心翼翼的扒着飞檐的龙头,用绳子下到灵宫的正门外的门廊处。

门廊是类似于祭祀台的地方,架空铺平的地面都是石板,常年的寒冷让石头脆化,脚踩上去嘎嘣作响,随时可能断裂。这里应该是当年修建进入山栈道的尽头,现在栈道已经给拆毁了,一边就是一片漆黑的万丈深渊,而左右两边是一排铜制的覆盖着冰屑的鼎,里面全是黑色的不知名的古老灰烬。

铜器的风格和宫殿的样子,都有非常明显的汉风格,看样子汪藏海到那里承包工程,设计方面还是无法超出他自己的民族和时代限制,或者说,也可能以当时东夏的国力,只能去掠夺边境汉族的东西来凑合了。

另一边就是灵殿的殿门,门前立着一块无字王八石碑,石碑后面就是弄宫的白玉石门,门很大,几乎有三个人多高,两人宽。石门上雕刻着很多在云中舞蹈的人面怪鸟,说不出名字,在门上方的黄铜门卷是一只虎头,门缝和门轴全给浇了水,现在两边门板冻的犹如一个整体。

站在这里看上面的冰穹,微弱的光线从上面透下来,我的眼睛都似乎蒙了一层雾,看出来的东西都古老了很多,这种感觉很难用语言来表达。

华和尚要在这里先拍摄一些照片,我们趁机喘口气,四处看看。叶成四处走了一圈,看到下面的悬崖后,感慨道:“我真他娘的搞不明白,这万奴皇帝为什么非要把陵寝搞在这种鸟不拉屎的地方,平地上不好吗?这不是折腾人嘛?”

我道:“做皇帝的想法和平常人是不一样的,也许是和他们宗教有什么特别的关系,你看西藏有很多的庙宇,全部都是建在一些根本人无法到达的地方,为的就是要接近天灵,这个我们这种俗人无法了解。”

胖子摇头表示不同意:“我感觉修建在这里的原因很简单,就是不想让别人上来,这皇陵里面肯定有什么好东西,万奴皇这老小子捂着当宝贝,死了也不给人,咱们这次得好好教育教育他。”说着和郎风一起拿出撬杆去撬殿门。

我听着好笑,胖子这人就是实在,要是他做皇帝,不知道会把自己陵墓设在哪里。

玉石石门后面没有自来石,用撬杠用力一卡,两边门轴的冰就爆裂,我们用凿子将门缝里的冰砸碎了,门才勉强可以推开一条缝隙。一道黑气都涌了出来,我们赶紧躲开,华和尚说没事,这是粘在门背后的防潮的漆,现在都冻成粉了。

殿门拉开一条缝,就再动不了了,似乎是门轴锈死了。拿手电往里面照了照,空旷的灵殿里什么都看不见,里面的黑暗好象能吸收光线一般。

叶成迫不及待的就想进去,却给胖子拦住了,他转头问闷油瓶:“小哥,你先看看,这地方会不会有什么巧簧机关?”

闷油瓶摸了摸门,又看了看门上的浮雕,看了半天,摇头表示不能肯定:“你们跟在我后面,别说话。”

这人说的话一定要停,已经是我们的共识了。我和胖子大力点头,几个人都掏出防身的东西。

闷油瓶闪身,跨过高达膝盖的门槛,一马当先走了进去。我们紧跟起后,越过门槛,忐忑不安的走入到里面黑暗中的那一刹那,我突然就感觉到一股极度的异样向我袭来。

我突然想道,近一千年来,我们可能是踏入的第一批人,想想这一千年里,这座无人注视的巨大的宫殿中发生过什么呢?

\chapter{灵宫大殿}

灵宫大殿是整个陵墓地上建筑的主体部分,规模最大,进入之后,第一眼看到的就是灵宫中间灵道两边的石墩大柱子,大概五米一根,我想起影画上他们用“飞来剪”吊棺椁时候的情形,想必这里所有的东西,都是用这样的方式一点一点从我们现在认为最不可能的悬崖上吊上来的。

石柱中间的黑暗里,可以隐约看到黑色的大型灯奴,再后面就是漆黑一片,不知道为什么手电照过去,竟然没有任何光线的反射,似乎那里是一片虚空一样,也没有看到任何的陪葬品。

胖子打起火折子,想去尝试点燃灯奴,我对他说不可,这一座建筑还矗立在这里没有倒塌,这里的低温是一个很重要的因素,如果点燃大量的灯奴,造成瓦顶的冰晶融化,可能要造成一些小坍塌,所以还是不要了。

我们只能靠手电在黑暗中前进,给环境影响,所有人都不说话,似乎怕吵醒了这灵宫里的什么东西,四周静的吓人,空气中只剩下我们的脚步的回声和四周人沉重的呼吸声。

叶成是几个人里最没见过世面的,走了几步就忍不住说道:“真他娘的安静,怎么感觉浑身凉飕飕的,越没声音我就越慌,咱们说话,别搞的很做贼——”

话没说完,闷油瓶做了个轻声的手势,让他闭嘴。胖子轻声对叶成道:“你他妈的别出馊主意,咱们不就是贼吗?这位小哥的耳朵灵着呢,你一说话,咱们踩到了机关都听不出来,你担当的起吗?”

叶成一听这里可能有机关,忙捂住嘴巴,紧张的看向四周,惟恐有什么暗器飞来。

华和尚道:“也不用这么紧张,这里是祭祀用的。东夏的政权,很可能每年还来这里祭祀,有机关的机会不大。而且这里也有点年头,不用担心。”

“胡说。”胖子一听,想反驳华和尚。

我拍了他一下,让他别多事,刚才还说让别人别说话,自己说起来没完了。

外面如此厚的冰穹,一旦封闭就很难再打开了,外面的栈道也早就烧了,这说明灵宫封闭之后压根就没人打算回来,华和尚不可能没想到,不过这种事情上无谓增加不必要的恐慌。

我们继续往前,走了大概不到五分钟,已经来到了灵宫大殿的中央,前面就出现一座玉台,四周围着有几只人头鸟身的巨大铜尊,这雕像雕的不是人不是佛,就象一根爬满地衣的扭捏的柱子,谁也说不出那是什么,看上去非常诡异。

胖子问华和尚道:“这他娘的是什么?灵殿里不是放墓主的坐像的吗?难道墓主是长的这个德行的?这……不是一只大蚂蝗吗?”

华和尚道,“这可能是东夏宗教中被异化的‘长生天’……他们的主神。”

“这神长的也太没溜了吧。”胖子喃喃道。“和洗衣服的棒槌有什么区别?”

我又拍了一下胖子让他积点口德,咱们现在还在它的地盘上呢,他就不怕现眼报应。

不过这诡异的黑色图腾,我知道并不是长生天,我对萨满虽然不了解,但是我知道长生天是没有形象的,长生天代表一种无处不在,无限的力量,是一种宇宙崇拜。华和尚这么说要么是在晃点胖子,或者在掩饰自己的心虚。

这里的环境的确给人一种莫名的紧张感,除了陈皮阿四和闷油瓶子还是那副臭脸,其他人都或多或少的有点异样的表现。

但是如果灵殿之中放的不是崇拜的神龛,那应该放着的就是墓主人的坐像,难道真如胖子说的,东夏皇族长的是这个样子的?不可能啊?这——这根本不是人的形状,这看上去,更象海地拜物教中的邪神,我在上海看展览的时候看过一次,那边的神才是这么一陀一陀的象锅巴一样,犹如巨型的软体动物一般。

我突然想起那条铜鱼之中的记载:东夏皇族都是地底挖出来的怪物,难道就是这东西?不会,这东西只能说是个妖孽,我相信东夏人不会矬到认块锅巴当皇帝。

如果能看到另外两条铜鱼中记载的东西就好了。我心道,就不用猜的如此辛苦了,不知道什么时候才有这个机会。

正胡思乱想着,一边的潘子叫了我们一声:“你们看这里。”

我们转过头去,发现潘子已经攀上一座铜尊,在人面鸟的嘴巴里,小心翼翼的捧出了一个东西。

潘子也是个闯祸精,我紧张道:“小心机关。”

潘子点点头,十分小心的去捧,很快,一只鎏金青面獠牙的铜猴给启了出来,身上还雕刻着无数奇特的花纹,犹如纹身的小鬼。

我们都很好奇,从来没有见过这样设计的尊器。潘子跳下来,捧到我们中间,几个人围过去看。看来看去,只发现这东西竟然是青铜的,其他一点也说不出个所以然。

在考古中这种事是常见的,因为墓葬一方面是有着严格规定的神秘学,一方面又是墓主个人的事情,有很多墓葬中都出现过无法言喻的陪葬品,那些既定规则的东西你可以去收集和整理,无限接近事实,但是个性话的东西就只能猜了,有很多的东西,历史上只出现过一次,除了墓内的苦主,谁也无法去知道这是什么意图。

华和尚检查了一遍其他四只铜尊,也发现了相同的东西,他推测说如果这一根棒槌如果是他们的主神的话,四周的应该是主神的守护兽。这可能和当地非常地域化的神话传说有关系,咱们不在那个朝代,已经无法了解真实的情况了。只不过让他想不通的是,为什么会是青铜的材料,明朝的时候已经是十分发达的铁器时代了。

在图腾的四周查看了一圈没什么发现。我们又往后走了走,后面一片黑暗,不知道有多深。

此时让我有点奇怪的是,灵宫大殿之内,一般放的祭祀用的巨鼎和长明往生烛,设暖阁、宝床、宝座和神位,现在这些都没有踪迹,有点奇怪。不过形势大过形式,东夏国一直蜗居在长白山密林深处,也不知道是个怎么样的生活状态,这些东西也许女真习俗里并没有也说不定。

胖子此时已经有点烦躁了。他来这里的目的,就是为了摸东西。跑了一路却没见到任何可以带走的明器,如何能不郁闷。走着他就问我们,能不能让他去那些灯奴后面看看,看看后面的黑暗中有什么。

闷油瓶对他摆了摆手,意思是不行,他取出一只荧光棒,往那边上一扔密植见一道绿光闪了过去,掉落到灯奴后面的黑暗里,绿光一下子便消失了,好象是扔进了黑色的棉花里一样。

胖子看着咋舌,轻声问道:“怎么回事?”

闷油瓶摇了药头,表示不知道。

我对他道:“我们在外面看大殿没这么大,我们的手电没反光,殿墙肯定有吸光的涂料,离群独走,我保证你回不来,还是不要轻举妄动的好。”

胖子道:“那你们拴根绳子在我腰上,摸到东西算你们一份,算你技术入股。百分之……十,如何?”

我最烦胖子这德行,怒道:“你要疯等我们都出去了,现在别连累我们。”

潘子也道:“你他娘的猴急什么,这才到哪里啊,要是等一下你拴根绳子进去了,拉出来就剩条大腿了,你说我们是进去找你还是不找你?你看人家陈老爷子的队伍多齐心,你安了,别给我们三爷丢脸。”

胖子哎了一声,失望道:“得,你们人多,说不过你,胖爷我服从组织安排就是了,在没有查明敌情之前,绝对不背叛组织。”

“查明了也不准背叛,你他娘的现在就开始捞油水,进了地宫怎么办?你能装的了多少?”我怒目道。

胖子举手表示投降,嬉皮笑脸,我知道他的脾气,现在说什么也没用,拿他没有办法,只好提醒自己留一个心眼看着他,免的他闯祸。

再往里走,我们就看到了大殿的尽头,那里还有一道玉门,是用四块汉白玉片嵌接而成,门轴盘着琉璃烧制的百足蟠龙,门楣浮雕乐舞百戏图,门上雕刻着两个守门的童子,门后同样没有自来石,门是用哨兵浇死,我们撬开之后,发现门后是通往灵宫后殿的走廊,漆黑一片。

胖子看到门上的两条龙,顿时又来精神了,眼睛发亮,对我们道:“我在一拍卖会上见过这种门。这叫做蟠龙轴琉璃栓,整一扇门拍到了两亿,还是港币呢,哎呀,这门看上去也不是很重……”

我知道他想鼓动什么,泼他冷水道:“你省点心吧,那是炒作,现在现金的古董交易,能超过2000万就是天价了,这门最多就值四十万。”

“不会吧。”胖子不信:“40万炒到2亿?有这么离谱的事情?”

我心说我口袋里的两条铜鱼都值2000万呢,但是真卖的时候谁会买,现在拍卖行的勾当谁不知道,都是想着三年不开张,开张就吃一辈子,碰到个楞头青真掏2亿买扇门,下辈子的工作就只剩下花钱了。

胖子的世界观顿时就被我摧毁了,看着门神情有点呆滞,我们不去理他,走入走廊,向后殿走去。

后殿一般就是地宫的入口所在的地方,一般都会放一只装饰性的棺椁,点着长生蜡烛,终年不灭,或者是堆积大量的祭品,由守陵人定期更换。东夏这种常年战争状态下的隐秘边境小国,料想也不会有太多的好东西,不过地宫入口一般设在里面,我们必须去看。

进入走廊,两边加上头顶,前是壁画,壁画上蒙着一层冰,冻的灰蒙蒙的。我在缝隙中看过那一块双层壁画之后,一直对这种记述性的东西很感兴趣,于是打起手电看起来。

一看却看的浑身发凉,只见壁画之上,画的几乎都是盘绕在云雾之中的百足龙,盘起的,飞腾的,满墙都是,乍一看就象爬满了蜈蚣一样。

壁画分成好几个部分,有的壁画上还画着很多穿着裘皮的士兵,朝天上的百足龙叩拜。

头的还画着两条百足龙缠绕在一起,不知道是在交媾,还是在争斗。

每幅壁画之上,百足龙必然是主体部分。四周的人物都显得非常渺小,而且谦卑之极,显然东夏人对于这种蜈蚣龙的崇拜,比我们汉人对蟠龙的崇拜有过之而无不及。

叶成掏出相机把壁画全部都拍了下来,这在卖明器的时候可以用到,因为东夏是不确定政权,有陵墓的照片,价格能翻上好几倍。

“你们说这陪葬陵里葬的是什么人,万奴的老婆还是手下,怎么尽画这种壁画?”叶成边拍边问。

我也不知道,心里也觉得有点异样。

一般来说,陪葬陵的墓主人会有两种,一种是自己的子嗣和亲属,一种是自己的宠丞,子嗣和亲属的话壁画的内容因该多是生活场景,宠丞的话一般就是在朝的场景,比如说文官治水,武官伐兵之类的画面。画着如此多的神化龙形,如果在主陵里看到还可以说正常,在这里就不对劲了。而且……壁画之中看不见陵墓主人的形象。

就算以龙为主体,这些画突出龙的威严,那在下面虔诚叩首的应该会有一个领头人,因为是陪葬陵,带头人必然是万奴王,而这座陵的主人应该在万奴王的左右祀奉,但在壁画上面所有的人都是奴隶或者士兵的打扮,没有任何的领头人。

这在皇陵壁画之中,简直不合常理,不符合三规五常的壁画,画在这里等于没画。

胖子突然问道:“会不会这里的壁画也是双层的?”

我摸了一下,这里的壁画有些已经脱落了,之下并没有发现有什么特殊的面,摇头说不是,那道火山缝隙中的壁画,背后肯定有一个故事,不然在这么一个地方有着两层壁画,实在说不过去。

我一边胡思乱想,走了大概有二百多米,壁画却突然停止了,走廊到了尽头,后殿的出口出现在了前方。

出口处无门,不过中央摆着一只青铜鹤脚的灯台,有半人高,造型很奇特,上面起了一层白色的冰膜,使得颜色看起来偏黑。

我们走出走廊,来到后殿之内,胖子打起一只冷烟火四处观望,发现后殿的格局和大殿几乎相同,但是小了很多,我们可以直接看到四周的殿墙,墙上仍旧还是漫天的百足龙壁画,颜色当初应该都是鲜艳的红色,现在都冻成灰的了。

后殿之内空空如也,没有任何的陪葬品,就连搜索都不需要,一目了然。中间横放三张黑色的雷文盘龙石床,台上覆盖着雕刻有云边的木籉,都已经给冻的开裂了。

这叫停棺台,棺椁抬进来之后,暂时就是放在这里,这里有三张,显然当时入殓的时候并不是只有一只棺材,陪葬者的妻儿也同时陪着他下葬了。

当陪葬折者的陪葬,听起来就感觉非常不幸,但是在那个年代,也没有办法。

三张石床的后边的地上,凸出有一块四方形的巨大石板,石板上雕刻两只人面怪鸟,呈现环绕状,石板的中间浮雕着太极八卦图。这是封墓石,地宫的入口必然是在这块石板之下。

除此之外,后殿真的是啥也没有,空旷到了过分的地步。

胖子看了一圈道:“万奴老儿真他娘吝啬,舍的钱给手下盖房子,舍不得钱买家具,这叫人怎么过啊,肯定好东西全给他一人占了。”

华和尚道:“别胡说,能盖这么大一个陵墓,还会舍不得几个祭品?这他娘的肯定有什么特别的原因。”

我也感觉没这么简单,这后殿之中的情形,是有点不太对劲,即便是一个边陲的小国,如我们所预料的国力不足,但再怎么说,破船也有三分钉。没有金银,一般的铜器总会有几件的。

又搜索一圈,四周也没有通道通往其他地方。就来到封墓石板的一边,胖子甩开膀子上去用力抬了一抬,纹丝不动,忙招呼别人来帮忙。

为防石台下面有毒沙毒水之类的陷阱,闷油瓶仔细的检查了封墓石板边上的青砖地面。确定并无问题之后,郎风把顺子往一边的停棺台上一放,就和化和尚、叶成他们上去推动石板。

几个彪形大汉力气真不是盖的,就听嘎嘣一声,石板给移开了少许,他们继续用力,缓缓讲整个石板推到一边。

我们往石台下面一看,却吃了一惊,石台之下并没有任何秘道入口的痕迹(没有封墓门的条石),而是如边上一样的青砖,只不过,因为石板压在上面长达百年,地上有一个四方形的印子,用脚一搽,有凹凸感,石板下的青砖已经被压入底下几毫。

“怎么回事情?”潘子奇怪:“这封墓石是假的,摆设?”

“不可能,这是最基本的葬式,玩什么都不会玩这个,入口肯定就在这里。”华和尚道。

“会不会封在这层青砖下面了?”叶成问。

我皱起眉头,这些砖头只见没有铁浆封死,看上去似乎有点问题,但是要我下结论,我又不知道怎么说。

胖子道:“管他呢。反正没人,难得倒一回皇陵,拆了砖头看看就知道了。”

叶成马上附和。其实我也是这么想的,我们这些人现在已经不能说是在盗墓了,我们现在干脆就叫明抢。盗墓的时候还怕惊动四周的居民,怕遇到巡逻的警察,但是现在最近的警察局也在八百里外,我们根本就不用怕什么。

我们全部都开始肾上腺素过度分泌,挖掘和开地宫永远是令人兴奋的时刻,有时候开棺都没这一刻紧张,这一点谁也无法否认。

闷油瓶蹲下身子,用他奇长的手指夹住一块青砖,用力一拔,硬生生将砖头从地面上拔了起来,叶成和华和尚看的目瞪口呆,嘴巴都合不拢。

胖子很得意,脸上大有看见没,咱们兄弟厉害不的表情。闷油瓶却不给他面子,看也不看他。有了一个缺口就好办了,我们上去帮忙,用登山镐将砖头挖出来。

让人奇怪的是,下面的砖头仍旧没有铁浆的痕迹,全部是交错结构,并不难挖。

我不详的预感又重了一点。因为地宫的入口处是堡垒最森严的部分,当年孙麻子挖慈禧墓,要不是有炸药,连地宫石封的皮都铲不掉。这里如此轻松就能起青砖,肯定就不对了,会不会下面有什么蹊跷。

但是闷油瓶却不说话,一般如果有问题他肯定能马上发现,他不说话,我说话又觉得似乎没这个资格。

半支烟的工夫,我们很快就挖出了一个大坑,最后一层青砖被启出,数来只有七层,大概是因为这里的建筑的高度是固定的,要想不撞到洞顶,只有牺牲底下铺地砖的数量。坑底下面,竟然露出了一块黑色的,似乎类似于布满花纹龟壳的石头。

“是不是封条石?”叶成兴奋起来。

“不是。”最下面的华和尚敲了敲,把黑色石头四周的砖头都启出来,砖头下面,出现了一只八仙桌大小的,黑色的双头石雕龟,龟的壳上的花纹,现在看来,竟然雕刻的是一张女人的脸。

“这是怎么回事?”众人不解,这应该是地宫入口的地方,竟然埋着一只石头乌龟。

“怎么没有墓门?”潘子刚才出力最多,喘着气纳闷。

“先搬出来再说!看看龟下面是什么。”华和尚也摸不着头脑,开始乱指挥。

其实不用搬就知道乌龟下面肯定什么都没有,我已经看到乌龟底下的黑色山岩,我们已经挖到了洞底了。

几个人手忙脚乱跳入坑内,想将石龟抬起来。才蹲下身子,胖子就“嗯”了一声,似乎发现有什么不妥。

我凑过去一看,只见胖子挂在腰上的工兵铲,不知道为什么竟然粘在了龟的背上,胖子用力一掰掰了下来,一放手,那工兵铲又给吸了过去。

我看着奇怪,难道这龟,是磁石雕刻吗?

几个人围过去看,都啧啧称奇。胖子掏出一枚硬币往乌龟背上一扔,“当”一声,吸的牢牢的。自言自语道:“嘿,这他娘的逗啊,这么大的磁铁,这墓主人是收废铁的?”

陈皮阿四在上边休息,看我们发现了什么,以为找到入口,问怎么回事情,华和尚把情况向他汇报。

还没说完,陈皮阿四的脸色就变了,他忙叫叶成搀扶他下来,走近那只龟,从自己口袋里拿出指北针,一看之下,他脸色几乎绿了,狠狠把那指北针一砸,冷声道:“糟糕,我们给骗了!这个陪葬陵是个陷阱,我们中计了!”

\chapter{博弈}

我看着陈皮阿四的表情,顿时觉得不妙,这个老家伙一路过来,一直闷声不响,只在关键的时候说几句话,从来都没有什么恼火的表情,但是现在,明显他是真的大怒了。

华和尚也察觉到了这一点,也紧张起来,问道:“老爷子,怎么回事?”

陈皮阿四脸色非常难看,对我们道:“这里的龙脉给人做了手脚,这条三头龙是假的,龙头的方向错了。”

我心里一个咯噔,忙掏出自己的指北针去看,果然,无论怎么转动,指针就是指着那黑色的石龟,显然,这古怪的东西磁性极强。

我马上明白了陈皮阿四的意思:看风水脉络的,方位非常重要,刚才一路过来,陈皮阿四都是靠这个指北针配合自己的心里熟背的罗盘来确定龙脉的走向和方位,但是这里埋着一只磁石雕刻的东西,这么大的体积,那我们靠近这座山的时候,指北针里的南北指向肯定会受到影响,那他当时用来判断龙脉走向依据就是完全错误的!

这三头龙的格局是在这错误的前提下判断出来的,那肯定也是假的了!

也就是说这里根本不是龙头,什么“昆仑胎”,外面巨大的冰穹,都没有了存在的理论依据。都是一种假象!都是引导我们走入这个陷阱的心理暗示!

汪藏海肯定是想到了以后能找到这里来的人,必然有相当的风水造诣,所以早就做好了准备。在我们还没有进入陵墓,还没有提高警惕四五时候,早就进了他的套。

我突然感觉到一种无力感觉,“昆仑胎”,冰穹,如此巧妙的设计,竟然只是为了一个陷阱!汪藏海果然对于盗墓有着深刻的了解。一直以来我都嘲笑那些笃信风水的建筑师,风水没有给墓主人带来任何的荫福,反而成为了盗墓贼指明了无形的方向。但是我们却犯了同样的错误,给一个古人硬生生摆了一道。

现在是和一个死了有几百年的人博弈,结果第一局还没开始我们就给将军了,真是出师不利。

胖子和潘子还不明白。我把事情给他们一解释,胖子还不是很相信,说:“不可能啊,那时候怎么可能有这么大的磁铁?”

我感慨。“这只石龟,肯定是用磁性陨石雕刻而成的。这东西的价值非比寻常,可是汪藏海却用它来压墓,看来为了保护云顶天宫,老汪是下了死力气了。”

“我操,不可能。”胖子还是不肯相信,道:“这里修的这么正规……”

说到一半他也意识到了,这座灵宫建筑制式的确正规,但是里面一点灵宫的必须品都没有,其实我们早就发现破绽了,只是谁也没想到整座灵宫都会是一个圈套。只因为他的制式太正规了。

陈皮阿四脸色铁青,也不说话,只是狠狠的盯着那石龟,眼神非常的可怕。

我和华和尚他们在那里合计,这一下子算是完蛋了,咱们的粮食肯定不够再转向去三圣山,这一次我们恐怕要先回山村补给。那这一趟来回,算是完全白走,而且我们几个损伤都很大,估计回到村里还得花时间休息一下,这时间损失不起,阿宁他们就算走的再慢,也到了。

现在还不知道三叔这些安排的目的,但是无论从什么角度来讲,我们都已经处在下风。

想到这里,人不由有一些烦躁,这件事情其实谁都没有责任,不过人在遇到挫折的时候,有人是祸头总是有好处的,不然火没处发,只好在那里郁闷。其他人的脸色也不好看,但是如今也没有任何办法了。

胖子看我们都有点泄气,说道:“算了,那我们快回去,不过是走错路了,咱们出去再来,阿宁他们才这么几个人,不可能把东西全运出来,咱们动作快一点,还有洋落好捡!”

我一听他脑子里全是洋落,突然一股无名业火,冷笑摇头说你知道什么,三叔几乎是牺牲了自己的生意来拖慢阿宁他们的进度,但是我们还是慢了一拍,如果回去再回来,不知道要给他们拉下多少,三叔可能就会凶多吉少。你他娘的只知道明器,什么都不关心,别在这里瞎叫。

胖子听了也不爽,破口就想呛我,叶成把他按住,“好了好了,现在不是吵架的时候。”

气氛一下子很尴尬,胖子甩开叶成,骂了一声,走到一边就抽烟。华和尚摆了摆手,道:“白走一趟,大家都不好受,现在主要是想办法补救,咱们镇定点,想想怎么办吧?”

胖子道:“什么补救,我认为没关系,这么大一磁石杵在这儿,谁到这里来都要倒霉,你们就敢说阿宁那帮人没中招,说不定他们的方位也全错了,现在已经给边防打成蜂窝煤了。我们应该把这里摸一遍,把能带的都带走,然后用最快的速度折返,在山下重整装备再来,别浪费时间,既然已经中招了,不面对现实怎么行。”

我知道胖子其实说的没错,可能我们到最后还是不得不按他说的原路回去再来,但是现在他这样的论调在这里是不受欢迎的。

潘子马上摇头:“说的轻松,要你现在原路回去,你有把握回的去吗?就算你认识路,咱们走了一天了,你皮糙肉厚的不觉得累,我们可吃不消。就算要回去也肯定是明天早上,小三爷的担心是有道理的,这样耽搁时间,三爷做的部署就全白费了。”

胖子一听马上就抓狂了:“三爷三爷,去TMD三爷!你们TNND连那老瘪三在想什么都不知道,还扯什么jb蛋,胖爷我为什么非得掺合到你们的家务事里来,老子是来摸明器的,TNND不管了,老子自己摸完自己走,你们陪那不阴不阳的老鬼一起去死吧。”

说着胖子就扯起自己的包,打亮手电,往走廊走回去。不过才走了两步,闷油瓶就拦到了他的面前,不让他继续走。

胖子对闷油瓶有点忌讳,不好对他发作,但是又不好下面子,问道:“干什么,TNND别拦着胖爷我发财。”

闷油瓶道:“你不觉得奇怪吗?我们到了这里,好象情绪都很焦躁,连吴邪都发火了。”

闷油瓶一说,胖子就一楞。马上转过头来看着我,众人都脸色一变。我心里也咯噔了一声。

是啊,刚才的无名业火他妈的就是突然起来的,发的一点道理也没有,我不知怎么的,突然就有一股烦躁从心里散发出来,胖子他以前就是这么样一个人,再不靠谱的话我都听过了,我怎么就发飙了,这不是我的性格啊?

以我的做事情方式,就算真的有人说不中听的话,我也不会在这种场合去挤兑他,而且刚才胖子的反应也太大了。

难道真是给四周的环境影响了?我转头看向四周,四面一片漆黑,手电照过去,整个黑暗的空间里面只有我们几个手电是亮的,其他地方的黑暗就犹如黑色雾气一样把我们团团围在里面,非常的压抑。但是压抑归压抑,我感觉这不是那种莫明焦躁的源头。

“怎么回事?好象刚才真的有点邪门,突然就发火了。”胖子也醒悟过来,问闷油瓶道。

闷油瓶对我们道:“我也不清楚,不过我看这里不仅仅是一块磁铁这么简单。现在一定要冷静,你们刚才争论也没有用,这里既然是陷阱……”他顿了顿:“汪藏海花了这么大的精力设置了这里,既然能放我们进来,我看我们不一定能出去。”

我心里的烦躁一下子又浮了上来,一想到闷油瓶的话,我硬把怒火压了下去,道:“那现在怎么办?”

闷油瓶不说话,只是看了一眼陈皮阿四,后者也看了他一眼。道:“既然已经入了套了,我们只能走一步是一步,现在下结论能不能出去还太早,不过不管怎么样,我们必须把这只乌龟毁掉,然后在这里搜索一下,确定再也没有同样的东西,不然我们来几次都是一样。”

众人都怒目看向那只乌龟,显然都从来没有比现在更恨过这种动物。

大磁铁打碎了,也只是变成小磁铁而已,还是会对指北针有影响。要完全消除磁性,只有用火烧。

我们掏出无烟炉的燃料,浇在乌龟身上,然后胖子点起一根烟,猛吸了一口往里面一扔,火就烧了起来。无烟炉燃料的热量极其大,一下子我们就感觉炽热的气浪轰了过来。

华和尚拿出指北针,看里面的指针转动。

很快乌龟给烧的通红,就连四周的砖头也都烧成了红色,我们都趁机靠到砖坑边上取暖。

这里没有任何可以用来焚烧的木头,用高纯度的燃料,很快就烧完,大概半支烟的工夫,底下只剩下了滚烫的砖头和通红的乌龟。

“怎么样?”我问华和尚,凑过去一看,只见指针已经不再指着那只乌龟了,磁性已经消失了。他又拿着指北针走了几圈,确定地下再无其他的磁石,才点头说搞定。

此地不宜久留,既然是个陷阱,我们再无留恋。几个人收拾了一下,我想着闷油瓶说的话:能放我们进来,不一定能出去的话,心中已经有了一点不详的预感。会不会我们进到这个后殿来之后,外面已经发生了什么变化?有什么不可知的变故正在等待我们?

我脑子里闪过几个不太好的画面,马上否定掉,现在也只是推测,没必要自己吓自己,走一步是一步就行了。

不过我的预感总是在倒霉时候出奇的准确。就在我们准备重新走入走廊的时候,突然,不知道从后殿的哪个角落里,传来了一连串“喀啦喀啦”的声音。

“喀啦喀啦”的声音极脆,十分刺耳,我们全部都听到了。马上我们都停住了脚步,转头去看。

声音并没有停止,而是一直在延续。我听了一会儿,发现竟然是从我们焚烧过的那个砖坑里传出来的。

我们心里奇怪是什么声音,小心翼翼的走回去。探头一看,只见坑底的那只乌龟,竟然裂了开来,大量的裂缝在乌龟壳上蔓延。同时我们就看到一股奇怪的黑气,从裂缝中飘了出来,速度很快,瞬间膨胀上升到了空中,犹如一个巨大的软体生物,从乌龟的体内挤了出来。

接着,黑气和头顶的黑暗连在了一起,不停的蠕动,看形状,竟然和我们刚才在外面大殿之中看到的黑色图腾相似起来。

“这是……长生天!”胖子脸色惨白大叫道。

“你别吓人。”华和尚道,“可能这乌龟是空心的,热胀冷缩,就裂开了,里面什么东西烧焦糊。”

胖子变色道:“空心的?那这黑烟会不会有毒?”

“应该不会,没这个先——”华和尚道,话没说完,闷油瓶突然做了禁声的手势,让我们不要说话。

我给他的动作弄的一下冷汗都下来了,忙捂住嘴巴,所有人都屏住了呼吸,四处去看,想知道又出什么事情了。

我四处转头,听到我的心在“砰砰”作响,就象打鼓一样,四周却没有什么异样,倒是听到了,在这极度安静的后殿中,除了石龟的爆裂声,还有一种非常非常轻微的“稀疏”声,不知道从什么角落里传了过来。

我听了半天,没有听出那是什么声音,连它的方位都感觉不出来,好象这声音是直接进入我的大脑的。

说着话的时候,我下意识的回头去看了看,此时灵宫的玉门已经自己关上了,身后一片漆黑,手电照过去,整个黑暗地空间里面只有我们几个手电是亮的,其他地方的黑暗就犹如黑色雾气一样把我们团团围在里面。

这种黑暗非常的压抑,不知道是心理作用还是什么。我刚想对他们说“此地不宜久留!我们最好赶快出去!”忽然闷油瓶做了禁声的手势,让我们全部不要说话。

我给他的动作弄的一下冷汗都下来,忙捂住嘴巴,所有人都屏住呼吸。

我听到我心在碰碰作响,就像打鼓一样,但是同时也听到了,在这极度安静的四周,某一个地方,传来了非常轻微的“稀疏”的声音。

我听了半天,没有听出那是什么声音,连他的方位都感觉不出来,好像这声音是直接进入我的大脑的,这座灵宫在冰穹里面,不可能被风吹到,这声音肯定不是风声。

上方的黑烟越来越浓,那种稀疏声也越来越密集,很快,四面八方全部都传来这种声音,听的人浑身发痒起来。

闷油瓶的脸色越变越难看,不停的转声,看着积聚在头顶上的黑气,自言自语道:“烟里面,有东西!”

华和尚听着那“稀疏”的声音,又看了看那只石头龟,似乎也意识到了什么,脸色一下子变了。“这烟是虫香玉?乌龟里面有虫香玉!汪藏海想我们死。”

“虫香玉是什么东西?”我问道。

没人回答我,但是我知道我很快就会知道,闷油瓶指了指一边的棺床上躺着的顺子,示意郎风背上,然后一指前面走廊:“跑,不要回头!不管什么东西掉到你身上,也不要停,一直到出去,快!”

\chapter{骚动}

我一看闷油瓶的脸色,就知道他绝对不是开玩笑,在鲁王宫碰到血尸的时候,他都没露出这种表情来,事情肯定很严重。

但是此时我又不好去问他到底出了什么事,只得答应一声,拔腿就准备招呼别人跑路。

我认为我对于闷油瓶的指示贯彻的已经是非常彻底了,没想一回头,发现叶成和胖子他们已经跑进走廊里了,暗脉一声没良心,忙跟了上去。

冲过了走廊,撞开玉门来到大殿,那种“稀疏”的声音不减反增,此时已经明显可以感觉声音来自房顶的所有方向,就好象无数只脚在头顶磨擦横梁,听着直起鸡皮疙瘩。

但是抬头向上看去,却是无尽的棉花一样的黑暗,什么都看不到,更不知道是什么发出的声音。我们站在这样的黑暗和不安底下,简直是如坐针毡,恨不得马上离开这里,所以跑起来就几乎是拼了命。

相信所有的人都有体会,在黑暗遇到自己恐惧的东西,你一个人逃跑。你跑不了多远就会停下来,但是如果大家一起跑,到后来就肯定一发而不可收拾,你的想象力和落单的恐惧不会让你停下来。

不过人跑步的速度终归有差别,叶成已经吓破了胆子,跑的比兔子还快,胖子跑的也不慢,两个人速度最快,一下子就飞了出去,我们几乎跟不上,加上黑暗中看背影几乎不能分辨出谁和谁,很快几个人就给拉开了距离,我在后面勉强追着,只能凭借手电的光点来分辨方向。

也不知道跑了多久,力气几乎都用光了,脚步不由自主的慢了下来。我看着前面的几个手电光点,也逐渐变慢,似乎是目的地快到了,也松下劲来。心里庆幸,幸亏我的体力比以前已经好了不少,不然肯定就给他们落下了。

跑过去一看,前面几个人都停了下来,撑着膝盖大口的喘气,然而却不见出去的石门,前面还是一片黑暗。

我问怎么回事情,怎么不跑了?

叶成上气不接下气,脸上青筋开始爆出。道:“不对……不对劲——我刚才留意过,大殿一共是五百步距。我的步长是一米,以这样百米狂奔的速度,估计两分钟不到就到了,可是现在,我肯定我已经跑出了远远超过了那个时间,至少应该看到玉门了,但是前面还是什么都没有,有问题!”

胖子道:“会不会你数错了?哪有人每一步绝对是一米的?”

叶成自豪的笑起来:“绝对不会错,我的一步就是一米,不超过一厘米的误差,你要不信,咱们可以打赌。我们回来我已经跑了快一千米了,肯定有问题。”

后面的人也跟了上来,看到我们不跑了,速度慢了下来,跑到我们身边停了下来。几个人都背着沉重的装备和厚衣服,这一通跑下来,全部都累的气喘如牛,几乎都要摔倒了。华和尚大口喘气道:“怎么停下来了,快跑啊,一口气跑出去再休息。”

叶成一口气一句话的把情况一说。华和尚脸色也变了,抹了抹头上的汗道:“怎么回事情,我们进来的时候没走岔路啊,怎么一往回走就找不到路了?”

我想了向道,心道肯定有是中招了,这里必然用了什么我们不知道的手段,对他们道:“果然小哥说的没错,汪藏海根本就没想让我们出去。”

“那怎么办?”胖子问。“我们换个方向,往左跑!”

我四处转了转头:“不行,既然原路都回不去了,肯定是朝任何地方跑,都会跑到四处不着边的地方,永远到不了头,不要白费这个力气。”

叶成骇然道:“我靠,那我们不是要在这困死了?”

我在海底墓中领教过这些机关的厉害,但是也摸到他的一些门路,对叶成说那倒不至于,我们有这么多人在,肯定能出去的,只要集思广益,就不会有问题。到底汪藏海只能在他的能力范围内动手脚,机关再精密,也肯定是有破绽的。怕只怕汪藏海困住我们不是本意,那头顶上的怪声,才是我们要担心的东西。

又抬头看了看上边,“稀疏”之声已经密集到让人发痒的地步,心中骇然。叶成用手电扫来扫去,上面灰蒙蒙一片,隐约只能看到彩绘的房梁,快要把人逼疯了。

华和尚道:“呆在这里不动也不是办法,要不我们兵分四队,朝两个方向跑,这样总归有一队能先出去,不至于全军覆没。”

胖子大叫道:“你看看清楚,人还没到齐,我们就这么几个人,怎么兵分四队?”

众人一听,忙四处一看,一数手电,果然几个人顿时就蒙了。

闷油瓶,陈皮阿四,还有背着顺子的朗风,还有潘子都没赶上来,我操,一半的人都没了,我还以为他们都在我们四周。

刚才跑的时候乱成一团,也没有注意他们是什么时候掉队的,现在回头去看,四周看不出有一盏光线,根本无从寻找他们的下落。难道是刚才跑的时候跑岔了路,跑进了这里的黑暗当中,那就麻烦了,在这种情况走散几乎等于是自杀。

我捏了捏自己的眉头,仔细回忆了一下刚才的细节,我们并不是跑在最后的,那些人,比如说潘子,令他一向的习惯就是在最后,这是他当兵养成的习惯,这样可以监视所有人的行动,陈皮阿四年纪大了,也是早我们后面,朗风背着个人,行动不便,也跑不快,而闷油瓶是职业级别的突然失踪人员,他在遇到情况的时候一直会习惯性的殿后,然后突然失踪,是非常正常的事情。

这些人都是在我们后面,显然他们失踪的时候离我们并不远,刚才我们跑的太疯狂了,一点也没有察觉到。

华和尚他们一下子没了头,不知道怎么办才好,胖子扯起嗓子就大吼了一声:“老潘!你们在那里?”

他的声音一落,忽然就听到一边传来了朗风的声音,这声音根本无法辨别方向,但是却叫的极其响,只听郎风大叫道:“我操,和尚!快把手电灭了!看头顶!”

\chapter{墙串子}

“灭手电?”我一听蒙了,已经少了这么多人,还灭手电,要是再少了怎么办?这不是找倒霉嘛——忙看向华和尚,想他老成些,看他怎么反应。

华和尚也紧张的要命,看见我看向他,竟然还问我道:“灭不灭?”

胖子关掉手电道:“听他的,灭了看看!”

我马上关掉手电,华和尚他们也陆续关掉,一下子四周陷入到绝对的黑暗当中,我们赶紧抬头看房顶,一开始还是一片漆黑,什么也看不到,胖子正想骂人,忽然上面就亮了起来,我们马上看到,无数绿色的小光点密密麻麻的聚集在房顶上,咋一看,好像看到了漫天的星海一样。

“是五十星图。”

我听到边上华和尚的声音,我抬头再一看,果然,上方的绿色光点组成的形状,隐约是一个五十星图的样子,但是又不是很象,因为,这些绿色的光点,竟然是在移动的。

“这下发财了,这么多夜明珠!”胖子惊讶道。

“不是,夜明珠哪有这么小。”我冷汗都下来了:“在动,是虫子!”

“虫?什么虫?”胖子一下就紧张了,大概是想起了尸蟞:“萤火虫?”

“不是,荧火虫是一闪闪的,我没——”话还没说完,我突然感觉到脖子里痒了起来,好像什么掉进了我的领子,忙用手一摸。摸到了一团东西,一捏就给我捏死了。

当时凭借着手感,我就感觉到不妙,这是节肢昆虫,而且好象长了很多的腿。

我把这东西用手指从我脖子里捏出来。打起手电一看,心里忽然一毛,忙把那东西扔在地上。

那是一只巴掌长的,长的非常像蜈蚣的昆虫,前后的触须很长,身体细长分成九节,每一节的背上都有一个绿点,但是它和蜈蚣明显不同的是,这虫子的脚非常长,几乎和它身体等长,而且非常的多,犹如很多长毛在躯干两侧。

我知道这种虫子叫做“蚰蜓”。有的地方叫“墙串子”或者“蚵蛸”,这东西非常邪门。我小时候什么都敢碰,但是就是不敢碰它,总觉得这东西让人一看就不舒服,我们家乡的传说。这东西只要一爬过你的身上,给它爬过的地方全部都会腐烂。最可怕的是,这东西会往人的耳朵里钻,现在看到,一下子就浑身发麻。

“墙串子”在聊斋里面都有记载,最大能长到三尺,而且和蜈蚣蜘蛛一样,都是妖性很重的东西。

我看到这虫子就全身发紧起来,突然头上又痒了起来,一摸又是一只,是从上面掉下来的。

我顿时大叫起来,忙把它拍掉,然后带起了登山服的帽子,一照地上,我操,不知道什么时候,地上已经爬了好几只这种虫子,而且还有更多的不停的从上面掉下来。

下面的人无可避免的中招,华和尚反应没我这么快,已经跳将起来,不停的将他脖子里的东西拍出来,但是一点用也没有,那东西见缝就钻,很快就钻到了他的衣服里面。而且地上的虫子也不知道怎么回事情,全部都围向我们,从我们的鞋上爬上来。

胖子拿出脸盆子罩在头上,另一只手用工兵铲不停的拍打,我看到叶成抱住了脑袋,赶紧去帮他,拉开他的手一看,只见他的耳朵里已经爬进去了好几只。

有些“墙串子”和蜈蚣一样有剧毒,甚至毒过蜈蚣。我宁可我身上爬满蝎子也不愿意爬这种东西。我让他侧转头低下,拍打他的脑袋,把虫子拍出来。

我们边拍边跑,但是哪里都是下雨一样的“墙串子”掉下来,正在就要抓狂的时候,忽然啪一声,远处的一盏灯奴亮了起来,不知道是谁给点燃了。

我正纳闷这时候谁还有心思去点灯,忽然地上的“墙串子”就起了反应,开始向灯奴的方向爬了过去。

远处传来顺子的声音:“几位老板,点起火!这些虫子会在温暖的东西上产卵,不要让你的身体成为四周最暖的东西。”

原来是顺子这小子,我心道,看样子他醒了过来了。

我和胖子一听,赶紧爬上一边的灯奴,这东西是用石头雕刻而成的,造型是一个人背着一个盆子,盆子里面就灯芯,灯奴有一人多高,我爬上去一看,盆子的万年油都冻成肥皂了,里面爬满了虫子。

我拿起打火机烧了烧灯芯,火苗一开始很小,但是随着里面万年油的熔化,慢慢旺盛起来。油盆子的“墙串子”一看到火苗,竟然毫不犹豫的围了上去,几只“墙串子”缠绕在一去,被火烧的噼叭作响。

我再一次打开手电,向屋顶照去,上面的横梁彩画已经变化了,似乎刚才的图案是由这些虫子排列而成的。这时候其他地方也点起了灯奴,火光透过黑暗后非常的灰暗,我还是看不到边上的人,但是已经知道了他们的方位。看来这里的黑暗,并不是不能用光来穿越,而是我们的光线不够强而已。

头上还是有零星的虫子掉下来,但是大部分就掉到灯奴那里的方向去了,我们几个用护着头,向一边的灯光跑去,跑了几步就看到郎风还有顺子,郎风倒在了地上,不停的抽搐,顺子一边摸着后脑,一边给他拍身上的虫子。

我跑上去,问他怎么了,他道:“完了,虫子跑进脑子里去了,进的太深,挖不出来了。”

胖子啊了一声,不由自主地挖了挖耳朵,自言自语道:“幸亏我耳屎多。”

我问顺子道:“还有没有得救?”

顺子摇头道:“不知道了,在我们村子里,一旦中了这雪毛子,死活是听天由命的。”

我翻了翻郎风眼睛,已经没有知觉了,真的够呛,不过这其实也是好事情。因为这家伙是陈皮阿四手下的人,到时候如果和陈皮阿四翻脸的时候——这是迟早的事情——肯定非常难对付,现在中了招了,我们手里就多了一分胜算。

潘子远处叫了一声,让我们全部围过去,我对顺子道:“先不管了,人集合到一起再说吧,你脑袋没事情吧。”

顺子点了点头,纳闷道:“我怎么突然就昏过去了?我记得——你们要炸山!这里是什么地方?”

我一时反应不过来,胖子马上道:“你真糊涂了,我们本来只是想放个礼炮。没想到雪崩了,有山石掉下来,砸到你头上了,把你砸晕了,我们给雪裹到了这里,好象是座庙,我们也弄不清楚是怎么回事。”

顺子想了想道:“我没一点记忆了,不过你们怎么可以在雪山上炸东西,简直太乱来了……不行,你们回去得给我加钱,这买卖不合算。”

胖子还想说话,我怕胖子扯到哪里去都不知道了,拍了拍他道:“别扯jb蛋,这事情咱们待会儿再说,快点过去。”

说着和胖子两个人一起抬起郎风,就想把他抬到潘子那里去。郎风个子太大了,我们两个几乎用尽了全部的力气,才勉强把他推的坐起来。

郎风抽搐着,脑袋已经挺不直了,拖拉在那边。我想着怎么把他抬的立起来。这个时候,胖子突然皱了皱眉头。

我顺着他的目光看去,发现原来郎风的后脑有一块明显给打过的痕迹,虽然不是很明显,但是仔细一看能发现。

我心里咯噔了一声,这说明郎风并不是中了毒,而是给人打晕了。我看了一眼正在背郎风背包的顺子,刚想问他怎么回事情,胖子嘘了一声,示意我别说话。

我看了看胖子的眼色,不知道他有什么用意,只好还了一个眼色过去,然后抬起朗风,吃力地将他过的肩膀上,搀扶着就往潘子的方向去了。

经过一段黑暗,我们到达了潘子的那盏灯奴之前,陈皮阿四和潘子都等在那里,惟独不见闷油瓶子。我问潘子:“那小哥呢?”

潘子道:“不是在和你们在一起吗?我一直没有看到他。”

我向四周望去,除了我们点起的那盏灯奴和顺子点起的那盏灯奴的灯火,没有第四盏灯奴亮起来,远处只有两点朦胧的火光幽幽的毫无生气的立在那里。

按照道理来说,在那种环境下,听到顺子的叫声,怎么样也会点上边上的灯奴,不然肯定会给这些墙串子围死的,如果他不点上,难道是在听到顺子叫之前,已经像郎风一样中招失去知觉了?

也不可能啊,象他这样的蚊香体质,应该什么虫子都见了怕怕才对。

胖子对这四周的黑暗大叫了一声,声音一路回旋,在空旷的灵宫里面绕了很久,可是没有人回答,好像闷油瓶根本没有进来过一样。静下来一听,也没有任何呼吸声和脚步声。

我心里明了,以这个家伙的身手,应该没有什么东西能够在毫无声响的情况下制住他,如果他这样无声息的消失了,肯定是他有什么特殊的理由,或者发现了什么东西,自己离开了队伍。那即使我们现在给他跪下来磕头,他也不会出现的。

潘子和胖子又叫了几声,确定没有回应,就打起手电准备去找,我把他们拦住,道:“这时候千万别走散了,我们先把伤员处理好。然后一起去。”

众人一想也对,马上围到了郎风边上,陈皮阿四检查了一下郎风的伤势,以他这种老狐狸的性格,我看到他几乎立即就发现了郎风后脑的伤口,但是他一点惊讶的表情也没有露出来,而是看了我一眼,不知道在想什么。

我忽然感觉到不对劲的地方,不对啊,刚才背着郎风回来的人,是胖子和我,按照一般的逻辑关系,陈皮阿四不可能会怀疑在山村里临时找来的顺子,那他就很可能认为,击伤郎风的是我和胖子中的一个,或者两个都是。

那他以后会对我们采取什么策略,这事情就不好说了。这真是把枪口往自己身上拽啊。

话说回来,顺子是退伍兵,怎么说也是边防第一线的正规军。要说他打昏一个郎风也应该不是什么困难的事情,他可能是忌讳着我们,到底我们的身份不明,又明显都不是好东西,所以暂时装傻来迷惑我们,这我也不能去拆穿他,这里环境这么复杂,多一个朋友好过多一个敌人。

当时就不应该找个当兵的来做向导啊,我自己在心理嘀咕。心里感觉到关系乱成一团。不知道怎么处理才好。

一边的顺子将郎风放倒,然后从口袋里拿出两只牙签,将他的耳朵撑开,将里面的“墙串子”剔了出来,拍到地上,胖子马上一脚踩死。

顺子和我们道,这种虫子他们叫做“雪毛”,是非常罕见的中药。虫子一般是在雪线下活动的,在雪线上从来没有见到过,不知道这里怎么会这么多,郎风脑子里种了虫子了,估计坚持了不了多少时间了。

一般来说通过耳朵进入大脑,那是扯蛋。我摸了摸郎风的下颚,发现红肿,肯定是“墙串子”在他耳朵的里面咬了一口中毒了,没有顺子说的那么严重。只不过这些虫子到底是哪里来的,真的让人搞不懂。

胖子看着头顶道:“肯定是藏在屋顶的瓦片里,给那个什么虫香玉一熏,就醒了过来,这一招还真他娘的狠。不过,那老汪难道知道我们会烧磁龟?”

我心说那是肯定的,既然把磁龟埋在封墓石的最下方,必然是希望盗墓贼会发现,然后对它进行破坏,不论是烧还是砸,估计都会导致虫香玉的挥发,熏醒隐藏在宫殿瓦顶上的蚰蜒,但是如果对于灵宫有所敬的人如果不破坏,那磁龟在这里,就能永远保护云顶天宫的安全。

华和尚有带了一些药品,给郎风注射了一支,说是暂时可以保他的命。注射完了之后,我们将郎风的外衣脱掉,将衣服里面蜷缩着的虫子拍掉,清理干净了。潘子对陈皮阿四道:“四阿公,这虫子的毒性很厉害,我们最好快点离开这里,要再有人给蜇一下,药品就不够了。”

陈皮阿四看了看四周,皱起眉头,叶成叹了口气,把刚才我们发现自己被困的事情说了一遍给潘子听。潘子一听之下也是疑惑到了极点:“你确定,不会是我们走岔了?”

叶成刚想说话,顺子“嗯”了一声,说道:“奇怪。”我回头一看,原来是他刚才点燃的第一盏灯奴的火光,在远处的黑暗里消失了。

灯奴里面的灯油几百年没用了,现在能点着已经谢天谢地了,我对他说这没有什么好奇怪的,但是顺子却还是皱着眉头,又拍了我一下,让我再看。

我有点不耐烦了,这个时候,我却看到我点燃的那盏灯奴的第二盏火苗,抖动了起来,似乎有什么人在他边上走过,带动了风吹动火苗。

大殿之中绝对没有风,如果边上没有东西经过,绝对不会发生这样的事情。

我以为是闷油瓶回来,想叫一声,胖子却捂住了我的嘴巴,我看到火苗的光影,隐约照出了一个人的轮廓,肯定不是闷油瓶,因为这个影子太高大了。

我有点感觉不对,但是影子太模糊了,实在连个轮廓也照不清楚,陈皮阿四看了几眼,突然手一扬,打出一颗铁弹子,直掠过原处灯奴的火苗边上,劲风带起火苗,一下子亮了一下,马上我们就看到了一个脖子长的有点异样的人影,站在灯奴的边上。

\chapter{百足龙神}

陈皮阿四的铁弹子飞过之后,闪动的火苗瞬间又黯淡了下来,那边的人影子恢复模糊,一下子又什么都看不清楚。铁弹子最后不知道打在什么地方,发出一声脆响,滚落地上,声音在空旷的灵宫里回荡,让人直起鸡皮疙瘩。

火光闪起的那一刹那,所有人都给这影子吓了一跳,顺子更是惊慌,吓的轻声叫道:“这是什么东西!”

华和尚马上把他的嘴巴捂住,不让他继续说话,几个人的手都下意识的按到了自己的刀上。

陈皮阿四对我们摆了摆手,让我们不要这么紧张,然后给华和尚使了一个眼色,后者马上几步跳上一边的灯奴,一手将火拍灭了。

我不得不佩服陈皮阿四的冷静,在这么诡异的环境下,任谁也不会想到把自己身边的光源拍灭,都是希望自己身边越亮越好,但是其实,这种情况身处在黑暗中才是最安全的。

灯奴一灭,四周又一下子暗了下来,如浓雾一般的黑暗一下子将我们包围了起来,另一边的灯奴却显得更加的明亮。

我们“啪啪啪”把自己的手电也灭了,一起屏住呼吸,看着那边的影子。身边的黑暗一下子了刺激了我的神经,我一下子我感觉到心脏跳动的非常厉害。

这影子明显是一个人的,大部分的身体还是隐没在黑暗中,让人觉得非常异样的,是他奇长的脖子,和身上一些让人无法言语的似乎是刺或是触须一样的东西,看上去竟然不是像是人类。而是一种……一种鸟类。

我本能地感觉到一阵寒意,心里直跳,除了闷油瓶之外,其他人都在四周了。这影子看着又肯定不是闷油瓶,难道这里还有其他人?

那他是什么人?怎么会出现在雪山顶上的灵宫中的呢?难道刚才这里的雪崩引起边防的注意了,这人是探路的解放军?

也不会,不说雪崩发生在山谷里,就是真发现了,赶过来起码也要一天时间,不会这么快到达。

我突然想到,这个灵宫,是汪藏海设下的一个陷阱,既然是一个陷阱,必然是险恶万分。中陷阱的人绝想不到陷阱里等着他的是什么,这个影子,会不会就是汪藏海设立这个陷阱的时候,安排在这里的怪物呢?

我们大气也不敢出,死盯着那个影子,指望着能从它的动作和形状中推断出什么。最起码能让我们知道这东西到底是人还是其他的东西。

但是奇怪的是,那个影子也是直直的站在那里,犹如一座泥雕,连晃也不晃。似乎根本不是活物。

等了片刻,双方都没动静,胖子开始沉不住气了,轻声说道:“不对劲啊。是活物他就得动,这东西一动不动,是不是我们看花眼了,那是那些灯奴印在柱子上的影子?”

叶成道:“胡说,灯奴不是都在边上站着吗?他怎么能自己走到这边来?”

胖子轻声道:“不是说天地灵气,琵琶都能成精吗?说不定这里的石头灯奴就成了精了,自己就能走动。”

我给他说的浑身不舒服,一下子也没有多余的智慧来判断胖子说的话,早几个月的时候我连粽子都不信。现在我见过的粽子可以搓上两桌麻将,要说是有没有妖怪,我真不敢判断。但是胖子说是这石头灯奴成精,我感觉更多的还是一句玩笑话,胖子越是在危险的时候说话越是不靠谱,这也和他的性格非常有关系。

但是胖子有一样说的没错,只要是活的物体,他肯定得动啊,就算是只粽子,他也不可能像石头那样站在那里,这影子一动不动,就太过奇怪了。

不管是什么东西,我们也不能一直在这里僵持着,虽然我看不见,但是我知道这里的四周爬满着“墙串子”,如果再有人被咬,虽然不致命,但是雪山上缺衣少药,也是要命的事情。

我们轻声一合计,几个人想法基本和我相同,胖子用非常低的声音道:“那咱们就别在这里欣赏它的身材了,偷偷摸过去看看,要真是个人,他娘的按倒就揍!”

几个人答应了一声,我感觉到身边有风一闪,心急的已经摸了过去,一片漆黑也没办法布队形什么的,我硬着头皮朝着那唯一的火光就去了。

那灯奴离我们也不是很远,走了几步那影子就越来越清晰,我看着也越来越怪,不自觉的,一种不祥的感觉越来越强烈起来,在几乎走近那灯光能照到的区域的时候,我下意识就放慢了速度,埋伏在黑暗里,缓慢的轻轻的靠过去。

那黑色的影子几乎就在我的十步之外,我眯起眼睛看着他,一点一点地,我的心跳越来越快,冷汗开始不停的冒出来,一边祈祷着老天不要让我看到我不想看到的东西。

可是,随着越来越靠近的视野,我逐渐已经意识到老天可能不会保佑我们这种盗墓掘坟的人,眼前的那东西越来越清晰,一下子我连脚步也迈不动,只觉得浑身发软,最后竟然整个人都僵在那里无法动弹。

我都无法用言语来形容我看到的,那只能说是一条巨大地“蚰蜒”形状的东西,但是是不是我就不知道了,因为它太大了,我知道最大的蚰蜒能长到一米多,但是这一条显然更大,蚰蜒像蛇一样扬起着半个身体,缠绕在一座灯奴上。我们看到脖子,其实只是它的两只大毒鄂和长触须形成的影子。无数的长脚垂着,整条巨虫一动不动,似乎正在吸从灯油里挥发出来的气体。

在蜈蚣科里,加勒比海加拉帕格斯蜈蚣能够长到40~60cm,但是长到一米多的至今没有发现过,这么巨大的体形,这条虫子的寿命,恐怕有上千年了。

四周传来了几个人的呻吟声,我甚至听到胖子非常轻的说了一句:“你大爷的!”显然是其他几个摸过来的人也看到了,开始不相信自己的眼睛,我想到我们在半路上看到的那块刻着蜈蚣龙的黑色巨型墓道封石,忽然明白了为什么东夏人的龙会长着蜈蚣的千足!

看样子是他们退入到深山之后,看到了这么巨大的蚰蜒,把它神化为龙的化身了。

脑子一片混乱间,我听到有人打了几声呼哨,意思是:“退回去!”当时也不知道这话是谁说的了,我不自觉的就往后退去。一直退一直退,也不知道退到了什么地方,四周一看,一片漆黑。

原来华和尚把我们那边的灯奴灭了之后,我们没有了后退的目标,一退之下,全都走散了。

我重新打起手电,想着点起一盏灯奴,来吸引他们的注意力,却看到不远处那巨型蚰蜒的影子晃动了一下,它边上的灯奴一下子熄灭了,一下子巨型蚰蜒就消失在了黑暗里。

我忽然想起顺子说的蚰蜒有趋热的习性,顿时感觉不妙,同时在很远的地方,华和尚打起了一只冷烟火,叫道:“大家千万别点灯奴,所有人看着我的冷烟火到我这里集合。”

暗中我就听到许多只脚在地板上爬动的声音,频率极快,我一听也不知道它在哪里爬。反正声音是越来越响,赶紧撒腿就跑。

混乱中,我听到胖子在另一个方向叫到:“为什么不点?点上这个东西暂时拖一下那大虫子。不点它就直奔我们来了?它这么多脚我们跑不过它啊。”

华和尚道:“不行!我闻了那灯油,那油里面也有虫香玉,味道一散发出去,更多的这种——这种东西就会爬出来,到时候更麻烦。”他顿了一下,显然不知道怎么称呼这种巨大的蚰蜓。

我一听可能还不止这一条,顿时心里就毛了,一边朝华和尚的冷烟火快跑,一边也大叫:“那我们拿这条怎么办?”

华和尚道:“到了那里我自有办法,小心自己的身后,这种虫子爬的非常快!”

很快我就根据着冷烟火冲到了华和尚的身边,一下子四周出现了很多手电的光斑点,几个人从黑暗里冲了出来,我们跑的上气不接下气,连话都说不上来。胖子捂着胸口一边看着周围的黑暗,一边就问华和尚:“好了,到地方了,有什么办法,快说!那东西马上就要过来了。”

说着就去听一边那种让人觉得很抽筋的爬动声,但是这一听,那声音却消失了,似乎那大虫子并没跟过来,而是停了下来。

华和尚也是喘的非常厉害,一边咳嗽一边拍了拍背包里。道:“其实也不是什么特别的方法,我们还有炸药,炸死它。”

胖子一听失声笑道:“那好吧,这光荣的任务就交给你了。你去吧,我会帮你照顾老爷子的。”

华和尚说道:“不用我去,我已经安置好了,自然有人会去。”

我忽然从他脸上感觉到一丝寒意,同时也意识到了什么,转头一看,色变道:“郎风呢?”

华和尚不说话,只是看向一边的黑暗,道:“准备好。就要来了。”

话音未落,忽然“轰”一声巨响,一边的黑暗里忽然闪出一团耀眼的火光,我们条件反射地全部扑倒在地,一下子大量的木头碎屑雨一般落到我们头上,整个地板狂震,弹起木板子几乎撞到我的鼻子,冲击波不大,但是声音很响。震的我的耳膜翁翁叫,一时间什么都听不清楚。

我抬起头一看爆炸的方向,只见地板已经给炸出了一个大坑,边缘已经烧起来,那条巨大的“千足蚰蜓龙”整个脑袋给炸碎了,还在不停的扭动,而爆炸的地方,竟然是我们刚才安置郎风地地方。

我顿时就明白华和尚做了什么。难怪刚才那“千足蚰蜓龙”没有追到我们这里来,它是给一边昏迷的郎风给引了过去,而华和尚又把炸药按在了他的身上——

我简直就不敢相信自己的眼睛,转头看了看华和尚和陈皮阿四,几个人都没有表情,似乎这事情和他一点关系也没有。

陈皮阿四看到我的表情,拍了拍我,轻声对我道:“前走三后走四,你爷爷没教你吗?如果是我,他们同样也会这么对我,做这一行,就要有这样的觉悟。”

前走三后走四,是土夫子的土语。意思是做事情,做之前要考虑三步,做之后要考虑四步。土夫子在地下,每动一样的东西都是性命悠关的,所以你在做任何事情前,都必须考虑到后三步会发生的事情和该处理的办法,如果发现你无法解决,你这事情就不能做。而且这样的考虑必须养成习惯。

陈皮阿四这样说的目的,我也明白,其实像郎风这样的情况,他跟着我们活着出去的机会已经非常渺茫了。他的意思就是,早晚是死,不如让他死的痛快点。

我爷爷也曾今在他的笔记里提过。在地下的时候,有时候等你意识到危险的时候已经晚了,所以在危险产生之前就考虑到它。盗墓是个细致活儿,又要胆子,古来不知道多少半调子脑子一热就下古墓的,直接就成了陪葬。

但是话虽然这么说,郎风这样就死了,实在是太冤枉了,让我一下子觉得连站在陈皮阿四的边上,都觉得害怕。

可惜此时也无法表达自己的心情,只好深呼吸一口,尽量装成什么事情也没有发生。

前面的火光逐渐熄灭,这里的木头板子都经过长年的冰冻,空隙里面全是空气中水分凝集的冰颗粒,越烧就越多,越多就越烧不起来。

我们几个向着那个地板上炸出来的坑走过去,我的脚步迈的十分的沉重,很害怕会突然看到什么郎风的肢体。胖子和潘子却没有什么大的反应,似乎也很习惯了这种事情,或者说,他们可能认为把郁闷表现出来也没有用。胖子看我有点无法释怀,还拍了拍我,轻声道:“算了,反正是他们的人,说不定手上还背着人命债呢,出来混总归要还的。”

顺子还不明白发生了什么事情,几乎吓的有点傻呆呆的跟着我们。

走到地板被炸出的破洞处,用手电往下面一照,木头的地板下面的砖头给炸飞出了一个大坑,地下用黑色的石条做了加固的廊子,也给炸裂了,露出一道缝隙,下面是空的。

我知道下面是什么地方,因为这座灵宫的这一部分是修建在陡坡上,但是地板是平的,下面肯定就会产生用梁柱撑起来的一个三角形空间,所有修建在陡坡或者悬崖上的建筑,比如说布达拉宫,就是这样一个结构。

缝隙中有冷风刮出,显然与外界相连,我回忆了一下,下面的三角空间四周也用白浆墙围着,不知道是一个什么情形,但是有风吹出来,似乎可以从这里出去。

这里四周显然有什么问题,地面上布满了蚰蜒,如果硬要从正门出去,恐怕会越走越危险,此时炸出了一个坑洞,正好可以让我们脱身。

胖子跳入炸出的坑中,下面的洞还不能容纳一个人通过,要挖大才行,华和尚也跳了进来帮忙。胖子问我,这样挖下去有没有关系?

我让他们不要乱来,冬天的石廊子本来就冻的发脆,刚才的爆炸肯定已经把下面的承压结构完全破坏了,这下面不知道有多高,万一突然塌掉下去,不是塌一个人两个人,很可能这里附近整块地面都会凹陷下去,到时候灵宫就会成为我们的封土。

于是在胖子和华和尚腰里系了绳子,另一头系在一边一根巨大的柱子上,我们全部把扣子扣到绳子上,这样一旦发生坍塌,可以互相照应。

准备妥当,胖子开始用锤子砸下面的石板,没想到才砸了一下,突然“喀吧”一声从他脚下传来,下面碗口粗的梁子,竟然给他踩断了一根,一下子把他的脚陷了下去,一直没到了大腿根。

我给吓了一跳,还以为说塌就塌了,幸好只是脚陷了下去,胖子骂了一声非常难听的粗话,一边想把脚扯出来。

扯了半天,脚扯到膝盖却怎么也扯不出来,胖子自己也有点奇怪,突然他脸色就变了,大叫道:“不好,有东西在扯我的脚!”说着人就直往下滑去。

华和尚忙下去拉住他的两只手,用力往上扯,其他人一拥过来帮忙,把他的脚拔了出来,但是却没法把他拉到砖坑上面来,似乎下面有什么东西真的把他抓住了。

叶成打起手电,往下一照,众人顿时吸了口凉气。只见从胖子踩塌的石廊子的洞,竟然伸出来一只青紫色的干手,死死的抓在了胖子的脚腕上。

\chapter{夹层}

这真是万万也想不到的情况,所有人都慌了。

潘子一手翻起自己腰间的折叠铲,已经跳入坑中,轮起来就砍,但是胖子的脚甩来甩去,却没砍中,一下子批在一边的石头上,火星四溅。胖子一看潘子用的力气这么大,大叫:“你他娘的砍准点,别砍到胖爷我的脚!”

潘子也大叫:“你他娘的别动,不然劳资从你大腿那截算!”说着轮起来又是一下,没想到这一下还是没砍中。

胖子大叫:“换人换人,这小子看我不顺眼,要公报私仇了。”

一边的叶成和华和尚跳下去帮忙,想按住胖子的腿,没想到叶成下去还没站稳,突然人也一陷,下面整个石廊子又塌了一块,他整个人都缩了下去。

这他娘的简直是添乱,华和尚忙上去一把抓他,自己又没站稳,一个趔趄撞到了拉着胖子的我的手,我的角度本来就不好用力气,一撞就脱手了,胖子整个人就给拖了下去。

事情发生的太快,加上光线不佳,所以才如此慌乱。几个人滚成一团,胖子象头肉球一样,一下子摔进了坑底,我个潘子给他带的重重的摔倒在砖坑的斜坡上。当时我就感觉有点不妙,还没站起来,就听一连串接“喀啦啦”的声音从砖层下面传了上来。

我一听脸色就白了,这声音我太熟悉了,这是我们做建筑受力实验的时候,受冻石质材料大范围纵向开裂的声音。

还没等我想明白,四周就突然一震,整个坑往下猛的一陷,坑下面那部分的石廊子就坍塌了。所有人都没反应过来,突然就失去了平衡。都象坐滑梯一样顺着斜坡滚了下去,裹在砖头里摔到了木头廊塌出的凹陷里。

我还没来得及庆幸自己有先见之明,屁股就一麻,已经摔到了一处斜坡上,然后人就直往下滑去。幸好有绳子绑着,给硬生生绷住了才没滑下去,接着四周的砖头劈头盖脑就往脑袋上砸下来。

我屁股摔的生疼,捂着脑袋想坐起来,但是屁股底下的斜坡太陡峭了,脚根本借不到力气。用手挡开砖头,问其他人有没有事情。没人回答我,只听到一连串的咒骂声和砖头的碰撞声。

好不容易砖头停下来,我才能抬起头,看了看四周,一片狼籍,有几只手电全给裹到砖头里去了,有几只沿着斜坡摔到了很下面的黑暗中。幸好这些登山用的德国货结实,一盏也没碎。不过一点点光从人和砖头的缝隙里透出来,仍旧是什么都照不清楚,边上一片漆黑。头上隐约可以看到一个大洞,是石廊子的破口,我们就是从上面滑下来的。

这里应该就是灵宫大殿的下面,陡坡山岩上架空的那一块空间中。我们正摔在陡坡上,要不是有绳子,我们早就滚下去到底了。

叶成就挂我头边上,给砸的不轻,我拉住他问有没有事情,他回答我说吃过了中饭了。给砸傻了。

华和尚在黑暗中就叫:“小心这里可能有只粽子,抄家伙。有蹄子都把蹄子拿出来!胖子,你在哪儿?抓你脚那玩意还在吗?”

胖子是最下面的,我们和砖头全摔他身上,实在够呛。就听他的呻吟从砖头堆里传出来:“还抓着呢,都快摸到我大腿根了,老子把他夹住了,他娘的快把我拉出来,不然你胖爷我的老二要保不住了!”

“那是我的手!”一边的潘子大骂!

“我靠!”胖子怒道:“你他娘的耍流氓也不会挑个时候?”

没有手电,几个没给压住的人只好摸黑扒拉砖头,将砖头往斜坡下扒拉下去。潘子先给我挖了出来,不过他的手给胖子夹只了,拔不出来。我们又继续挖,很快胖子也挖了出来,如释重负,喘着大气就说:“你们这些挨千刀还真舍得压我,幸好老子带着神膘,不然这一次就正归位了。”

潘子没空和他斗嘴说:“你脚上那东西呢?”

黑暗中胖子动了动脚,似乎感觉了一下,道:“没了!摔成这样还能抓着不太可能,可能给我们撞到斜坡下面去了,他娘这种地方怎么会有粽子?”

华和尚道:“肯定还在附近,都小心点,拿好黑驴蹄子,先把手电找出来!”

我忙去砖头下摸手电,摸来摸去摸不到,倒是一边的叶成摸到了。拉出砖头堆,顿时四周就亮了起来,他拿起来马上就朝下面照。

我正在他下面,厌恶挡住手电光,刚想让他调暗一点,忽然,我看到叶成的脸色瞬间就绿了。

我一看他的表情,顿时就开始出冷汗,心说难道又在我边上?忙咬牙转头一看,猛看到我的肩膀边上,离我的鼻子只有一尺距离的地方,赫然探出了一张青紫色的干涸怪脸。

我吓的“哎呀”了一声,人往后一缩,左手抄起一块青砖就拍了过去,也不知道拍中了没有,转身就往上爬。

这时候另外几只手电都给挖了出来,一下子四周全亮了。我往上爬了几步,因为上面就是叶成,根本让不开,又滑了下来,往边上一看,不由倒吸了一口凉气。

只见在这灵宫大殿下的陡坡悬崖上,给修成了一层一层简陋的梯田一样的突起,在这些突起之上,几乎整齐的坐满了这样的冰冻青紫色古尸,一层一层,看上去好象庙里的罗汉堂,缩在一起,密密麻麻的,面目狰狞,看体形显然都是冻死的,全部都是象和尚一样打坐在这里,黑影错错看不到头,也不知道到底有多少。

叶成是这里胆子最小,发抖道:“我操,这里是和尚的堆金身的藏尸阁?”说着竟然有点浑身发软。

陈皮阿四按住他,摆了摆手,对他道:“不用怕,只是尸体而已。”说着指了指我的脚下。我低头一看,只见我们的脚下的砖块中,竟然也有一具已经被踩成粉末样的木乃伊。

“这里的死人都冻的和石头似的,一碰就碎。”陈皮阿四道:“这些东西已经不可能尸变了,这里应该没有粽子。”

“那刚才抓我脚的是什么东西?”胖子问。

陈皮阿四道:“你的脚,刚才可能是正巧给尸体的手勾住了,不然要是粽子,你以为你还有腿在?不信你看看自己的裤管。”

胖子低头看了看自己刚才给抓的裤管,果然有一个破洞,一只呈现勾状的干手,就在他脚下不远处的砖堆里。我捡起来一看,坚硬无比,不可能伸缩去抓人家的腿。

顿时,所有人都松了口气。潘子还夸张的唉了一声:“胆子这么小,看也会看错。”

胖子大怒,想反驳又实在找不到理由,只好在那里生闷气,喃喃道:“刚才那手真的是抓了我的脚了,被勾了被抓了我还分不清楚?他娘的不信拉倒。”

我们用手电向四周照去,这里是大殿之下,空间很大。因为尸体排的很密,我们也看不到尽头,不过除了尸体之外,倒没有什么其他令人起疑的物体。

潘子问华和尚:“这里怎么会有这么多死人?老子连听说都没听说过。”

“看情形应该是个殉葬的隔层,这个……我完全看不懂了,没有任何朝代的皇陵是这个样子的……这些死人到底是什么人?”华和尚自言自语道。

我压住恐惧,用手电照其中一个死人,发现尸体的五官保存的还算完好。眼睛都闭着,脸上皱纹横亘,却都没有胡子,浑身都覆盖着一层薄冰,让人害怕的是这些古尸的皮肤都是青紫色的,嘴巴张的很大,里面长的竟然是獠牙。

“这些可能不是人类。”胖子看着道:“你看这口牙,打个波儿能把人家脸批给捎了去。”

“不是人类?”叶成的脸色又白了,“那是什么?妖怪?”

“有可能就是传说中的雪人,只不过这些没毛。”胖子开始胡扯。

“放屁!”华和尚喝道:“什么妖怪雪人的,这些尸体的牙齿是自己磨尖的,这是古萨满教的一个习俗。后来因为太麻烦,用面具代替了。这些肯定不是明朝那个时代的女真人,你看这些尸体的衣服,都非常原始,不是女真或者蒙古的样式,还有你看,尸体外面有的还裹着麻布。这是冰葬形成的木乃伊。”

我想起在小圣山谷扎营那一晚,看过的冰葬坑,道:“难道这些尸体是汪藏海挖山修陵的时候,挖出来的冰葬的先人遗骨?”

华和尚点头:“肯定没错,这一处胎形山洞,以前可能是个墓地,当地上古先民在这里进行冰葬,不过给汪藏海土地规划成假陪葬陵了,这些尸体肯定是挖掘山洞的时候挖出来的。”

胖子问:“如果真象你说的,为什么不直接烧掉,把这些尸体摆在这里的作用是什么?”

“谁知道,你看这些木乃伊这么可怕,萨满教有很多原始诡异的行巫仪式和诅咒,据说都需要借助于尸体。这里的布置,可能和萨满巫术有关,也许会有什么诡异的事情发生,说不定我们在上面怎么走也走不出去,就是因为这些尸体,咱们还得小心一点。”

我想起秦岭之中的尸阵,似乎有着大量尸体的地方,总会发生这种类似于鬼打墙的事情,难道真的是邪术在作怪?

萨满教并不是完全的宗教,它其实是一种原始巫术,也就是说它是有实用价值的,和药理、精神崇拜有着相当的联系。我对于萨满的了解仅限于清宫戏里跳舞的萨满法师。不过据说萨满巫术和中国的奇门遁甲一样,在历史上分段的失传了,一部分好的东西引入了藏传佛教,一部分邪恶的东西,则突然消失。从古籍上可以看到,远古早期萨满巫术很多仪式极其阴邪乖张,有着大量关于诅咒、尸体方面的内容,和蛊术有着千丝万缕的联系,而库人就是信奉蛊术的,这两者之间是不是有什么共同点?

胖子听了华和尚的话,恍然大悟道:“难怪,进到这个灵宫总感觉脚下直烧,浑身不自在,原来底下埋了这么些个粽子,万奴老儿的良心真的大大大的坏了。”

华和尚道:“我也是推测,现在最重要的是怎么出去,咱们分头找找,四周有没有出口?”

说完华和尚又道:“但是要小心,怎么说这里也看着有点邪门,总归会有安全的隐患,而这里的山崖太陡了,一旦出事情,想跑也跑不了。”

众人答应,胖子早就等不及了。几个人解开登山扣,拿起手电,就分散了开去,开始小心翼翼的在这陡峭的峡壁上寻找。

在这么多尸体中行进并不是一件让人愉快的事情,但是有点奇怪的是,尸体越多的地方,你倒越觉得不慌,可能是害怕到了一定程度后就会有一种逆反式的情绪。

尸体排的极密,每一排中间并没有留下供人行走的空间,我们几乎都是从尸体和尸体的缝隙中挤过去的。尸体有老有少,全部都已经冻的犹如青紫色的岩石,我看到有些人还带着铜制的法器,都已经完全锈绿,几乎所有的尸体的下半身都和下面的岩石溶合在了一起,你要搬动他,除非将他敲碎。

找了半天,我的这个方向并没有收获,看着自己离其他人越来越远,总觉得心里不安,正想假装找完了回到破洞处问其他人的结果,就听潘子叫了一声:“死胖子,你在干什么?”

我们顺着潘子的声音,朝刚才胖子寻找的那个方向望去,只见胖子不知什么时候停止了搜索,反而是在下面的尸堆中,面向我们阴阴的蹲在那里,面无表情的张着嘴巴,乍一看上去,脸上竟然泛起一股青紫色,和边上的尸体无异,不知道在搞什么鬼。

\chapter{藏尸阁}

整个藏尸阁里一片漆黑,几盏手电的光斑交叉在一起,光线凌乱,胖子所在的角落离我们几个人都很远,手电照到那边,四周的尸体遮挡,影子一层叠了一层,纵使照的透彻,我们也看不太清楚。

只不过胖子脸上的那种青紫色,却不会看错,那种诡异的,木然到阴森的表情,实在和边上的尸体太象了,更是让人直起白毛汗。

潘子原本以为胖子又在瞎闹,又叫了一声,胖子却还是毫无反应,犹如雕塑一般一动也不动,潘子也看出了苗头不对,对我们道:“好象是出事情了?”

我皱起眉头,不知道怎么说好,胖子的表情和动作和这里的尸体如此相象,如果不是他在耍我们,就肯定有不妥的事情发生了。但是其他人都没事情,怎么偏偏又是他。看他这副德性,难道是给这里的鬼儿附身了?还是中了萨满的诅咒了?

我们逐渐顺着陡坡滑下去,靠近胖子蹲的那个地方,也没看到他周围有什么和其他地方不同的东西,全是青紫獠牙的尸体。走到大概离他还有四五米的距离的时候,潘子摆了摆手,让我们别动,给华和尚打了个手势。

在陈皮阿四的团队里,郎风是胖子这样的先锋类型,华和尚是师爷,叶成是类似于打杂的。几个人还都有自己特别的能力,现在郎风死了,但是华和尚的能力也不弱,所以潘子会给他打手势。

我感到了差距,如果是我们这一队,打先锋的人倒是很多,但是勉强可以成为师爷的我就太弱了,想想少了闷油瓶之后,如果对方没有华和尚,那有事情就得我上了,我和华和尚的能力就相差太远了。

华和尚看到潘子的手势,点了点头。他们两个人各自翻出猎刀,反手拿住,就向胖子摸了过去。

两个人很快就摸到了胖子边上,而胖子却没有转头看他们,好象那些搞行为艺术的街头卖艺人假扮的雕像一样,巍然不动。

我的手心里全是汗,不知道为什么总感觉有点不对。这时候,前面的两个人停了下来,其中潘子已经离胖子非常近了,几乎抬手就可以碰到他。可是这两个人却突然向后面摆手,让我们别靠近了,自己也开始后退了。

我的心脏开始狂跳,又不知道他们看到了什么景象,只见潘子退到我的身边,转头对我们道:“麻烦了,他身后的那具尸体有问题。”

“什么麻烦了?”我问道:“是在大粽子?”

潘子让我们别问,做了个手势让我们跟着他。

我们跟着他穿过几具尸体,下了几层梯田,来到了胖子的侧面。他一指,我顺着他的方向一看,只见胖子后面,盘坐着好几具青黑色的尸体,但是其中有一具,却和其他的与众不同!

只见这一具尸体的脑袋极大,几乎有普通人的三倍大,五官都看不清楚,犹如一个大头的还未发育成熟的婴儿。一条奇怪的舌头,从那具尸体的嘴巴里伸了出来,竟然盘绕在胖子的脖子上。

我顿时就头皮发炸,心狂跳起来,几乎脖子都僵硬住了,捂住嘴巴不让自己惊叫起来,轻声道:“那是什么的?”

“这可能是一只尸胎,那尸体所在的位置,肯定是整个灵宫的养尸穴,这具尸体正好在这个点上,时间一长,就起了变化,变成了这个样子,再有个几百年,恐怕就要成精了。”陈皮阿四在另一边轻声道。说完后,表情突然变的很奇怪,好象想到了什么事情,又道:“不对!可是这条龙脉不是假的吗?怎么会出现养尸穴,这……”

华和尚一看陈皮阿四的表情奇怪,似乎也突然明白了,表情一变(我感觉华和尚其实早就想到了,但是为了照顾陈皮阿四的面子,所以经常等到陈皮阿四想到之后才做出反应)。问陈皮阿四道:“老爷子,难道,这是个‘连环扣’?”

“连环扣”是一种骗术,是外八行了老千一个“雀”字局里的伎俩。讲的是把真的东西做成假的,再做成真的,然后留一点破绽,让其他人看的时候,看到破绽,看破最外面“真”的面纱,以为这东西是假的,其实这东西确实是真的,也就是空城计的一种。

陈皮阿四冷笑了一声:“是啊,假的,假的龙脉上怎么会有养尸穴呢?汪藏海这老家伙,‘连环扣’玩的很绝,可惜你百密一疏,终于还是出了破绽了。”

我还没听懂,问华和尚到底是怎么回事。

华和尚解释说,“真是太悬了,我们差点就给骗了,幸亏摔到了这里来。你记不记得,我们刚才发现方位被做假了之后,一直以为这条龙脉是假的,但是这里出现了尸胎,假龙脉上没有宝穴,是不可能出现尸胎的,这样就出现破绽了,看样子那磁龟也是汪藏海陷阱的一部分,是想让我们误以为自己上当了,误以为整条龙脉都是假的,其实龙脉是真的,只不过格局并不是三头龙,那只磁龟,只是将一条普通的龙脉格局,修改成了群龙座的极品大局,这其实很容易。”

我哦了一声,顿时有了点眉目。风水方位其实在决定一条龙脉的好坏上非常关键,比如有一条独眼龙,自西向东,那就是腾龙,自动向西,就是伏龙,你埋一只磁龟,改变一下当地的风水方位,那伏龙就可以伪装变成腾龙。

(后来我查了一下群龙座,原来三只龙头全部朝东,才能叫做群龙座,而长白山三条圣山山脉全部都是朝北,那只有中间的三圣山才是龙头,其他边上两条叫做双蛇盘护,也是风水佳穴,但是不宜葬人,而是适合修建庙宇,而磁龟一放在那里,北就变成了东,陈皮阿四才会做出了错误的判断。)

我不禁感慨,这样的复杂的设局,这种斗智的程度,简直不可思议。想想我们刚才完全已经被骗了,如果不是发现了这里的尸胎,我们肯定是灰溜溜的回去了。

我们和汪藏海,中间隔了一千年的岁月。但是我突然就感觉到他的思想几乎就在我的面前流淌,他在一千年前的定下的计策,竟然还能够把我们玩的团团转,这个人到底是什么来路?

潘子在一边轻声说:“你们竟然还有心思说这些,现在死胖子怎么办?对付这东西,黑驴蹄子管用不管用?”

陈皮阿四摇了摇头,表示不知道,华和尚也皱起了眉头,显然都不知道怎么对付。

这种时候是最讨厌的时候,我们不知道胖子这样给舌头绕着,会不会有什么危险,但是贸然去救又怕导致形势恶化,两边都无计可施,潘子和我急的满头是汗,又不知道如何是好。

没想到的是,我们这边没动,胖子那边倒是先有了反应,就看到胖子突然摔倒在地上,然后就给拖着动了起来。那大头尸胎蜷缩着爬动,用舌头扯着胖子,开始朝陡坡的下方迅速的拉去,胖子僵的和石头一样,一点反抗也没有。

要是给它扯到下面去,那胖子就死定了。形势一下就升级,潘子叫了一声追,我们马上就冲了下去。

那大头尸胎一见我们冲了下来,马上加快了速度,顿时胖子就在坡道上滚了起来,一路把那些尸体撞的七零八落,我们根本在斜坡上也不能跑,干脆象坐滑梯一样顺着就滑了下去。

很快就追下去十几米。突然我们看到胖子就在斜坡上消失了,一瞬间就不见了,大惊失色下冲到那边,马上就看到斜坡之上竟然有一个洞,胖子已经给拖进了洞了,只剩下两或只脚在外面。

潘子一跃而起,猛虎扑食一样扑了过去,一下子抓住胖子的两只脚,然后用力去拉。我哗啦着滑过去,又双脚乱蹬爬回去,也去帮手,接着叶成、顺子和华和尚也冲了下来。华和尚扯下一条登山绳绑在胖子的腿上,这样除非把胖子拉断,否则那尸胎怎么也拉不赢我们。

我们这么多人,很快胖子就给硬生生扯了上来,那条舌头紧紧勒在胖子的喉咙里,几乎扣进了肉里。胖子青筋直爆,双眼翻白,几乎就不行了,潘子翻出军刀就是一刀,顿时洞里传来一声女人的尖叫,舌头断裂,胖子就一松,给我们拉了出来。

我们赶紧扯开那条断舌,丢下洞里去,给胖子按胸口,胖子的身体马上就能动了,开始摸着脖子大口的喘气和咳嗽。潘子怕那东西又窜出来,猛扯出工兵铲就到洞口,用手电照着洞里,不过照了一会儿就放下了武器,似乎是尸胎已经钻下去了。

我们都松了口气,忙给胖子捶背。捶了半天他才缓过来,心有余悸的看着那个破洞,道:“谢谢,谢谢各位好汉。”我问胖子到底是怎么回事,怎么一动不动象弥勒佛一样。

胖子自己也不知道,说就感觉找着找着脖子一凉,就不能动了,看和听都行,但是身体就怎么也动不了,好象是给冻在了冰里,他在那里用力的使力气,但是连转一下眼珠子都不行,可把他急的。

潘子大笑:“听刚才那尖叫,这是只女尸胎,估计是在这里太寂寞,看你和她体型相似,想拖你下去陪她了。这叫做来自地狱的搭讪。”

胖子苦笑,推了他一把,“你他娘的才和她长的象呢。”

潘子笑着躲他的推手,人往后一仰,所有人都没有想到的是,就在这个时候,突然那只巨大的胎头又从洞里探了出来,满嘴是血,一下子咬住了潘子的脚,潘子根本没反应过来,猛的就给拖进那个洞里去。

\chapter{排道}

我们猛冲过去的时候已经来不及了,潘子已经跌的没影子了,洞里有转完,手电照不到最底下的情形,不知道是死是活。

我脑子一热,就想跳下去,但是胖子比我更快,扯住自己脚上的绳子拔出军刀就跳入了洞里,一瞬间就滑的没影子了。我还想再跳,给华和尚拉住了,说直径太小了,连你也跳下去,下面打都没法打,如果有用,胖子一个人就能把人救上来,如果没应你跳下去也是送死。

我咳了一声,探头看洞里,却什么也看不到,就听到胖子不断滑落的声音。上面的绳子迅速的给拉进洞里,不由心急如焚。

直过了一分钟,突然绳子就停了,接着绳子的那头传来了震动,接着胖子突然就在下面很深的地方大叫了一声:“拉绳子!”

我们赶紧拉动绳子,拼了命的往上扯,很快胖子就拖着潘子出现了,潘子还在不停的踢脚,显然那尸胎还是没松口。

陈皮阿四让我们让开,自己皱起眉头,翻出一手一颗铁弹,对着潘子的脚踝就一颗,狠狠就打在尸胎的大头上,尸胎这才尖叫一声松口,但是松了之后马上就想冲上来。

陈皮阿四就不给它机会了,又一颗铁弹,把它打了个跟头,它翻身再冲,又是一颗,这一次把它打的滚了下去。

我们趁机把他们两个都拉出了洞来,几个人马上远离了洞口。华和尚轮起工兵铲,就等在一边,果然不出几秒,这东西猛的又窜了上来,华和尚“当”一声活活把它拍了下去,我们就听一声惨叫迅速就跌落到了石洞的深处。

胖子脸色苍白,一边喘气一边对潘子道:“瞧见没有,看来你家媳妇还是喜欢你多一点。”

潘子吓的够呛,摆了摆手:“不说了,咱们扯平。”又问华和尚:“他娘的这个洞,是不是尸胎的窝,要是的话,老子炸了它,让它早日投胎。”

华和尚摆手:“不是,尸胎又不是动物,哪来的窝。这个洞确实奇怪,你们刚刚跌下去的时候,在里面看到什么了?”

胖子道:“又没带手电,什么也看不到,不过摸到了好几块石板,这洞应该是人工修的。”

人工修的?华和尚看上去有点在意,我也把目光重新投向这个大洞。

洞口看上去有点象井,还他妈妈的有点深度。我以为这是个废弃的桩孔井,看看又不是,这个井口的直径有点大,当时的桩孔井不可能打到这种程度,井洞的边缘有修凿的痕迹,有不是天然形成的那种火山熔岩孔。照了照,里面的尸胎已经不见了,看样子摔到里面去了,不知道是不是已经被华和尚拍死了。这东西除了长的可怕一点,倒也不是很厉害。

里面的井壁刚开始还有一些石板镶嵌,到后面就没有,而且非常的不平整,有点像人的十二指肠的内壁,有风从井里吹进来,夹杂着一丝潮湿的味道,探头进去几米,里面一片漆黑。不知道通往何处。

胖子看着就奇怪倒:“有点象东北的地窖口子?该不是修这座灵宫的时候,工匠用来腌白菜的地方。”

华和尚没去理他,用手感觉了一下洞口:“风是从这里吹出来的,这井不是实心的,肯定能通到什么地方去。”

胖子问,“会不会就是通到天宫地宫里去的后门,你们说的三头龙之间的秘道?”

我轻声说:“三头龙局已经证明是假的了,而且就算是真的,秘道应该开在地宫里,怎能开到这里来?”

胖子道:“你不懂。这叫声东击西,你没听毛主席说吗?最危险的地方就是最安全的地方。说不定这就是那‘汪汪叫’的计策。”

胖子一时记不住汪藏海的名字,随口就给他起了个外号,我听了差点笑出来。没好气的说:“拜托你放尊重点,怎么说汪藏海也是这一派的大师。你见了也得叫声祖师爷。而且那话哪里是毛主席说的,这是楚留香说的。”

胖子道:“你少给我认祖宗,什么祖师爷,他要是认我我还不认他呢,咱们别扯这个了,拿这个洞怎么办?要不要进去看看?说不定还真让我说中呢,那尸讨将来也是个祸害,要是在这洞里做了窝儿可能还会害人,咱们下去把它干掉。”

华和尚摇头说不可能:“既然群龙座是假的,就没有不要挖通三座山,这样倒也合情理,他们根本就没有人力和精力做这么巨大的工程,修一个云顶天宫恐怕就够呛了。这个洞在这里,恐怕大有学问了。”

我看他眼睛有点放光,显然有想法,就让他说出来,大家也好商量商量。

华和尚道:“我只是初步的一个构想,说出来你们可能不信。”

胖子道了:“没事,先说出来再说,如果有错误,同志们会帮你改正的。”

华和尚失笑,点头道:“好,那我就来说说。其他先不说,暂说这洞的口子开在这灵殿下面的这块地方,就非常耐人寻味,你想,把这口子开在这里,肯定是为了隐蔽的考虑,又有风吹出,说名这个洞是通往什么地方的一条通道,再看,洞壁上有的修凿痕迹全是反凿子,就是说这个洞是从里面开出来的,而不是从这里打进去的,三个要点,按照我们的经验,我们可以推断出这可能是一条排道,可能是修墓的工匠给自己留的后路,如果古墓被封,可以从这里逃跑。”

我奇怪道:“排道?不会吧,这么说,这下面还是有地宫的?虽然这里不是三头龙,但是还是修建了陪葬陵?”

华和尚却摇头:“可能性不大,我们在封墓石下面没有发现地宫的入口,有地宫入口必然是在那里,如果没有入口,就肯定没有地宫,这是万古不变的真理,把入口修在风水位之外,于主大不利。”

胖子道:“汪汪叫这个人,做事情很乖张的,也许他就是把入口修在了别的地方。”

华和尚摆手:“千万别想的这么复杂,汪藏海还是有时代局限性的,要是他连葬经都不遵守,乱来一气的话,我们死一万次都不够。”

我一想也是,如果连葬经都不遵守了,那就不用看风水了。象成吉思汗一样随便找个地方刨坑埋了,万马一踏,到现在都没人找到。问他道:“那既然下面没地宫,你说这条排道,是通向什么地方的?”

华和尚道:“排除法,第一,这条排道修在这假陪葬陵的下面,那么肯定是和云顶天宫的工匠有关系;第二,附近什么地方可能会修建这样的排道?毫无疑问,只有云顶天宫的地宫!所以我的结论——排道十有八久,是从三圣山下天宫地宫一路挖过来的。”

我马上叫道:“这怎么可能,这也太远了,他们如果真是要挖一条排道来做后路,也不用把口子挖到这里来,大可以直接做到三圣山上,那样不是可以省不少力气吗?而且在山里挖出这么长一条排道,需要多少时间,少说也要二三十年吧?这样的工程是人能做到的吗?”

华和尚解释道:“云顶天宫这样的浩大工程,在古时候肯定需要花费六七十年,甚至几代人才能修建起来,我想里面的工匠知道自己最后必死的情况下,偷偷要挖一条排道出来,并不是不可能。至于他们为什么要把洞的出口开在如此远的另一座山上面,肯定有他们自己不得已的原因,我们下去看看,必然能知道。”

一直听着的叶成问道:“和尚,你这想法,你自己有没有把握?”

华和尚顿了一下,道:“说实话,我不敢说,不过我觉的值得我们去尝试一下。总比咱们出去之后再跑一趟的强,现在所有的迹象都表明这是一条排道,如果我料错了,那下面是其他地方,进入也不是坏事情,这种排道,咱们也不是第一次见了,应该不会有什么危险,没人会在自己逃命的路上设机关的。”

我一琢磨,华和尚的说法实在是非常有吸引力,一来这里风水的说法太乱,我已经搞不清楚陈皮阿四他们说的话了这里的风水是好是坏我也没兴趣。二来另一边阿宁他们的进展不知道怎么样了,我们已经浪费了很多时间,到现在我们还不知道三叔安排这一次“下地”的目的,要是因为这一个来回全盘皆输,我真是对不起他老人家了。

另一个方面还有一个考虑就是顺子现在可能基本上知道了我们是干什么的,现在闷声不响的站在一边,也不说话,但是这人不笨,我一直看着他手从来就没有离开过他的刀超过两尺,说明这个人已经在戒备我们了,这人一旦回到村里,谁知道他会干什么,说不定马上就会把我们卖了,陈皮阿四肯定考虑到了这一点,如果我们不得已要出山重来,那进村之前第一件事肯定是杀人灭口或者重金收买,到时候再找向导,就不一定能找的到,你们去一次雪山,自己回来了向导没回来,谁还会再带我们进去,二来,村里能带人上雪山的人,恐怕也不多了。

几个人商量一下,权衡再三,意见却不统一,叶成怕那尸胎坚决不赞成下去,潘子也觉得邪乎,胖子和我就觉得可以试一下。华和尚就去请示陈皮阿四,说老爷子我们要不就走一招?

陈皮阿四一直坐在那闭着眼睛听我们说话,华和尚问了几遍,不知道为什么,他一点反应也没有,似乎是睡着了。

胖子有点按奈不住,就去拍他道:“老爷子,你倒是说句话,别在这装酷啊。”一推之下,陈皮阿四晃了一下,却仍旧没有睁开眼。

华和尚一看,脸色一变,猛跑上去一抓老头子的手,一下子脸就唰一下白了,胖子一看也跑了过去,一摸老头子的脖子,也顿时变色道:“我操,死了!”

众人一听,都楞了一下,什么?死了?怎么可能,几分钟前不是还好好的吗?但是一看到胖子的脸色,华和尚脑门上的汗,和毫无反应的陈皮阿四,我们都意识到了不对劲,众人马上围了过去。

一边的老头子像是僵直了一样,闭着眼睛,一动也不动的坐着,犹如冰雕一般。

我摸了摸陈皮阿四的手腕,一下子也摸不到脉搏在什么地方,只是感觉他的皮肤又干又涩,而且凉的可怕。而且里面的肉似乎都僵了。

难道真是死了?我心里骇然,就在我们在那里研究那坑的时候,陈皮阿四就坐在这里,心脏慢慢停止了跳动?

虽然这很符合低体温症的死亡方式,但是低体温症起码需要在低温度下二十分钟才会真正断气,我们才坐了五分钟都不到,他怎么会就突然死了?这也说不通啊。

我心里还存着一丝希望,胖子这人说话不靠谱,他是只摸了摸陈皮阿四的脖子,判断死亡太武断了。有可能只是休克了,刚才一路跑的太快了,九十岁的老人怎么可能受的了?

然而华和尚皱着眉头,掰开老头子的眼睛,用手电去照后。脸色越来越难看,最后他回头看了一眼叶成,摇了摇头。

华和尚有一定的医学知识,看到他摇头,我们顿时就吸了口凉气,知道不会错了,真的是死了。

潘子轻声问道:“怎么回事,怎么死的?”

华和尚叹了口气。不知是说不知道还是不想说话,阴着脸一下子瘫坐在地上。胖子就拉了潘子一下道:“这么大年纪了,怎么死都行啊。”

我不禁一叹,果然对于九十来岁的老头,来到这里,实在是太勉强了,发生这种事情说是意外,也在情理之中,这陈皮阿四大概自己也想不到,自己竟然会这样死掉。也算是他的报应了。

我的爷爷最后也是这样突然就去世的,当时我在吃饭,前一分钟他还在让我给他拿酒,后一分钟他就去了,我父亲说,很多盗墓的人因为早年接触了大量的墓气,所以心脏都会受到一定程度的损害,所以老了大部分都是这样死的,也好,这是最舒服的死法。

我们都有点不知所措,一方面陈皮阿四是他们的瓢把子,现在他死了,叶成和华和尚呆在这里就没意义了。二来,我们是得了一个大便宜,顺利到达这儿,但是陈皮阿四一死,闷油瓶又不在。靠胖子和潘子两个带我们,恐怕也够呛啊。

就在我飞快琢磨的时候,陈皮阿四忽然一颤,我一惊,以为是条件反射的尸动,谁知道“啪”一声我的脖子就给他死死的捏住了,同时他人猛的一直,眼睛睁了开来。

我们全给吓了一大跳,叶成就直接一滑摔下去五六米,胖子和潘子也忙往后一退,胖子惊叫道:“诈尸!”

我赶紧想把手给掰开来,没想到这老头枯萎树枝一样的手力气极其大,象老虎钳子一样,连动也动不了。忙咳嗽着大叫:“拿……蹄子来,快快!”

话还没说完,陈皮阿四突然就松开我的脖子,把我一推,骂道:“你在胡扯什么?”

我脑子已经混沌了,赶紧退到胖子身后,却给胖子卡住不让我过去,这时候忽然一想,不对啊,诈尸还会说话?再一看陈皮阿四,明显人的精神也上来了,呼吸也恢复了。

我们几个一脸疑惑的看着陈皮阿四,也不知道刚才到底发生了什么事情,胖子更是眼睛直瞟向陈皮阿四,非常的疑惑,但是这一下子陈皮阿四好像又恢复了正常一样,一点也看不出刚才脉搏停止跳动过。似乎刚才的那一刹那我们看到的都是幻觉。

华和尚呆了半响,才反应过来,问道:“老爷子你没事情吧?你刚才这是……”

陈皮阿四似乎一点也不知道自己刚才死过一次了,莫名其妙的看了他一眼,点上一只烟,说道:“什么?”

华和尚看着陈皮阿四的表情,也有点犯晕,不知道说什么好。

陈皮阿四冷冷看了他一眼道:“你放心,老头子我没这么容易死。”

我看着陈皮阿四的样子和语气,和刚才无异,也不象似乎给什么鬼借尸还魂的,忽然感觉刚才是不是被他耍了?但是他干什么要玩这种把戏啊,一把年纪了。

陈皮阿四一下子“复活”,一下子谁也没反应过来,但是看他的样子,我们也不能把他按倒解剖看看是怎么回事。我心里又逐渐怀疑是不是刚才华和尚和胖子弄错了,老年人的脉搏本来就很难摸,两个赤脚医生可能根本就没摸对地方。而陈皮阿四到底年纪大了,偶然发一下呆,是很正常的事情。

几个人都是一脸疑惑,但是都没办法表露。

华和尚虽然奇怪,但是一看陈皮阿四没事情,也就放下心来,于是把刚才我们讨论的事情又说了一次,陈皮阿四看着那冰洞琢磨了一会儿,说道:“有点道理,似乎值得试一下。”

\chapter{进入排道}

我们在洞口停留了很久,讨论这个洞的可能性,期间陈皮阿四突然僵死了一段时间后又奇迹般的复原了。我们莫名其妙,但是陈皮阿四似乎一点也不想提起刚才的事情,也没有办法,只好将注意力转移到了这奇怪的冰洞上。

我们围到这个洞边上,讨论下洞具体的问题,我们几个虽然都经历过不少洞穴的探险,但是都是在平原和山区,和这里大不相同,需要从长计议。

这个洞刚开始是斜着四十度左右下去的,底下很深,并不好走,刚才胖子他们摔下去,要是控制不住姿势,也是十分危险,很可能会在洞里打起滚,那摔到低脑袋可能就撞扁了。

潘子甩下去一根荧光棒子,黄色的冷光迅速滚落,在很远的地方弹跳几下,消失不见。

如果华和尚的说法是对的,回忆我们两座雪山之间的走势,这个洞穴肯定是一路向下然后再往上的“V”字路线,两座雪山因为属于同一条山脉,所以山峰之间的峡谷海拔也很高,这条“V”字路线的距离,应该不会超过5公里。

当然如果当时的工匠秀逗喜欢“Z”字形挖掘,那我们也没有办法,不过这种情理之外的事情应该不用考虑。

既然是人工挖掘出来的通道,那就不用担心氧气的问题,我们商量完之后,决定先由潘子探洞。这次准备好了武器,万一那尸胎还在里面,就地就把它解决掉。

潘子刚才拖了进去,很没面子,在手上吐了口唾沫就掏了登山绳子。一边系在胖子的腰上,一边就扔下洞口,一马当先爬了下去。过了一只烟的功夫,才听到他的叫声,让我们下去。

我们也陆续地爬下洞口。坑道修凿的非常粗糙,石头里进外出,一路滑下去屁股给割的生疼。我一边爬一边观察边上的岩石。这些都是火山喷发的时候,涌出的玄武岩,上面有大量的气泡,这些石头密度很不稳定,有些硬的像铁一样,有些就软的像豆腐。不知道当年开凿的时候是什么样的情形。

我们一个一个的下去,胖子最后一个下来,一下子一堆人挤在了上面石道斜坡的尽头,大口喘着气。我们在这里看到很多黑色的液体,肯定是尸胎的体液,但是却不见尸胎的影子,可能往洞穴的更深处去了。

这个冰封下的狭洞倾斜着下去,到了下面转弯的地方,变的竖立着狭长起来,再往里面,洞穴的高度似乎继续在增加,豁然开朗,空间似乎变的很大,但是一片漆黑,手电照不进去。

我一开始还以为这是他们在挖隧道的时候故意再设置了一段比较宽的隧道,这在我们开盗洞的时候也有讲究,叫做鸽子间,这地方是用来囤积空气和放置“土”的,当然鸽子间的做法复杂,你在地里挖出这么大一个可以让两个人转身的空间,而不从盗洞口翻出一点土星子,有一个非常非常巧妙的窍门。

但是手电一照,我们就一呆,原来这条排道到了这里竟然已经到了头了,到了前面急速收缩变窄,最后前面只剩下一条大概只能供一个人侧身进入的石缝隙,犹如一道不规则的剑痕,深深刺进山岩里面。

潘子问道:“不是说这是工匠逃生的秘道吗?变成条瞄人缝了?这还走的过去嘛?”

华和尚想了想,忽然做了恍然大悟的表情,道:“我想这一条排道可能是利用了天然的火山溶洞,火山洞在火山地带的岩层里面非常常见,四通八达,最长的火山溶洞全长可以达到几千公里,就像蜘蛛网一样密布地下,可能这一条火山缝隙能够一直通到对面的三圣山,正因为这样,他们才可以挖通这么长的一条秘道,原来是利用了大自然预先设好的通道。”

胖子道:“那难不成我们也得进这缝里?你们都还行,我这体形可够呛啊。”胖子在海底墓里就说着要减肥,但是到现在也没见成效,看着他的身材,还是真够呛。

华和尚道:“这应该不用担心,这种火山溶洞都是树枝状结构的,这些孔洞应该都通到更大的缝隙里,这在地质学上就叫做地下走廊,有的地下走廊规模非常大,里面甚至会形成自己独特的生态系统,我相信进去不久缝隙肯定会宽起来,因为这种地质破坏都是从内部开始的。”

华和尚言之凿凿,我却不是很相信他,不过这时候确实也没有理由反驳他,于是大家休息片刻,整顿装备,由胖子打头,继续朝着缝隙内爬去。

缝隙里面是一片漆黑,而且手电都没有用处,因为那种黑是全方位的。在欧洲,人们认为所有的这种缝隙都是通向地狱的通道,藏民也认为洞穴是恶魔的地盘而从不进入。我虽然有过很多这样的经历,但是进入缝隙的那一刹那,心脏还是不安的跳动了起来。

一个接一个收着腹部进入了缝隙之后,我们侧着像螃蟹一样走,这个地方的洞壁已经没有了人工的痕迹,里面几乎不能转头,看着前面,满眼都是琉璃花的痕迹,大量的各种颜色的岩溶滴瘤覆盖着所有的岩石,上面结满结晶透明的冰霜,像凝聚的水柱均匀排列。

我在学建筑的时候学过一点地质学,我脑子里有模糊的记忆,眼前的东西应该是火山喷出岩,和我们在遭遇暴风雪的时候进去避难的那一道火山缝隙一样。这种地貌的产生又不同于常见的火山岩洞,这种缝隙是在火山喷发的一刹那形成的,然后给火山碎屑流以极高的速度冲出来,它的特点是形成的火山缝隙道极长,但是隧道单一,不会形成火山岩洞一样的迷宫洞群。

缝隙的刚开始段非常狭窄,我们不得不学着霹雳舞的动作挪动。没十五分钟已经累的浑身酸痛,想着当年那些逃难的工匠,爬出来也挺不容易。不过走着走着,缝隙真的如华和尚所说,逐渐变宽,最后竟然转过身子前进。

缝隙里面一片漆黑,但是四周的琉璃和融化的云母反射着我们的手电光,使得四周的光线产生一种魔幻的效果,加上大面积的火山碎屑覆盖的熔岩刺、绳状结壳熔岩、熔岩钟乳让人目不暇接,非常漂亮。

走着走着,我们就逐渐发现了一些人类活动过的迹象。比如说废弃的铁锈工具、篝火的痕迹,都非常古老。

一路上没有碰到任何奇怪的东西。缝隙里面非常干净,只走了将近六个小时,我们已经到达了华和尚所说的地下走廊规模的隧道,这里面的缝隙已经非常非常的宽阔。

缝隙到了这里,我又发现了大量人工修造的迹象,在一边的缝隙壁上,给修凿出了很多简陋的台阶,一直向上,这台阶说是台阶,其实只是一些突出的石头,要是脚大如胖子的,恐怕走几步就要晕。

我们停下来休息,我略微计算了一下,我们行走的距离和下来的坡度,发现我们这个时候所处的海拔高度已经低于雪线,可能已经位于两座山峰之间的峡谷下方。这两座山峰如果在地表上行走,就算是直线行走,最起码也要花八个小时,现在在地下行走,我们节约了不少时间,而上面的边防线,要是知道有这么一条地下走廊,肯定会大跌眼镜。

那到了这里,如果继续在缝隙的底部行走,那我们可能就要走到地心去了,这些简陋的台阶,估计是说明这条隧道进入了第二个阶段,台阶的尽头,也许就是云顶天宫的神秘地宫。

休息了片刻,几个人都按奈不住自己的心情,于是马上再次起程,不过这一次,路走起来就没有这么顺利了。

我们几个都用登山绳子互相连起来,然后尽量贴着一边的峭壁,踩着开玩笑一样的“石阶梯”,一点一点走上去,刚开始还好,等到爬到一定的高度后,马上就觉得自己像一个攀岩运动员一样,但是自己又没有半点攀岩的经验,这种感觉别提多慌了。

胖子的脚大,这些阶梯他踩着就像踩高跷一样,所以没走几步,脚已经开始发抖,我看他直念阿弥陀佛。

所幸一路走的小心,几乎是像女人做针线活一样,一点一点的向上爬去。很快,下面已经是一片漆黑的深渊,无法估计出高度,看着就会头晕,要不是刚才我们是从下面上来,我包准真的会以为下面是通着地狱的。

随后这几个小时,我们越走越高,最后都无法判断自己是在哪个位置,也无法判断时间,几个人进入到一种茫然的状态。但是却没有一个人提出来休息,不知道是盗墓人天生的贪欲,还是因为这里的环境实在无法休息,你可以想象你的一只脚踩在一块巴掌大的石头上,一只脚悬空,下面是万丈悬崖,如何能休息的进去?

走着走着,忽然四周传来了水声,打起手电一照,原来一边的峭壁上竟然有好几处泉水瀑布,顺着峭壁流淌,大小不一。看到上面的水气,看样子还是温泉,温泉水不知道是从哪里流出来的,但是水声却很大,似乎这附近有地下水脉的活动。

胖子问我们爬过去洗把脸舒服一下,最近的温泉离他只有一个手的距离,其实我们这一路来已经很累了,加上上次有过在温泉边上休息的记忆,几个人都想在这边停一下,可是顺子却摇头道:“不行!”说着指了指温泉边上的岩石,我们一看,第一眼没有发现什么,但是仔细一看,却几乎打了一个寒颤。

只见温泉边上的岩石上,有很多的非常奇怪的纹路,我第一眼以为是火山纹,但是仔细一看,却寒毛直竖,原来这些纹路不是岩石上的,而竟然是一条一条的手臂粗细的蚰蜒,扒在上面。这些蚰蜒的颜色和边上的琉璃火山石一模一样,不仔细看根本分辨不出来。

我们四处看去,才发现这边的石头上面几乎爬满了这些东西,一动不动的,似乎都死了一样。

一下子我们都安静了下来,胖子轻声道:“怎么回事,咱们怎么进虫子窝了?”

顺子轻声道:“雪山上的生物一般都集中在温泉边上,所以不要一看到温泉就想着下去舒服,有些温泉里甚至都是蚂蟥。不过现在气温还偏冷,这些东西扒在这里是处于半死状态,没有特别强烈的刺激,他们是不会醒过来的。我们快走,过了这一段就没事了。”

几个人马上开动,胖子掂着脚,边挪边问道:“特别强烈的刺激是指什么?”

话音未落,顺子突然摆了摆手,又让我们全部都别动。

我们不知道又出了什么事情,马上就停止不动,像木头人一样呆在了那里,都盯着顺子看,但是顺子却是看着一片漆黑的峭壁深渊。

静了片刻,我们逐渐就听到一种让人发毛的“稀疏”声,似乎有无数只脚正在摩擦峭壁的岩石,向我们靠拢而来。

“关手电。”顺子轻声道。

我们马上关掉手电,转头一看,我操,几乎四周整个峭壁,目力能及的地方全是大大小小幽幽绿色的光点,数量之多,浩如星海。在这黑暗中,这亿万的光电犹如魔幻,而我们就犹如置身于群星宇宙之中,那种壮观,无法用言语来表现万一。

然而低头一看,又突然发现身边的景象实在不算什么,只见深渊底下的虚无黑色中,一条绿色的银河蜿蜒而去,宛如深黑色幕布上华丽的翡翠流苏,穿过无边的黑暗,从天的这一头,一直甩到另一个尽头。

我张大了嘴巴,不敢相信自己看到的,这底下蚰蜒的数量,恐怕要以亿万来计。

就在我们被这壮观无比,简直可以用仙境来形容的景色震撼的时候,忽然从下面的光点中,闪出了几点巨大的红色荧光,那几点荧光扭曲着,在星海之中挪动,一下又消失在了黑暗中,显然下面的蚰蜒,有一些块头不会太小……

\chapter{火山口(上)}

黑暗中传来顺子的声音:“这种虫子在我们这里被当成神来膜拜,因为它可以活很长时间,而且一只蚰蜒死了之后,它的尸体会吸引很多很多的同类,所以我们走的时候要特别小心,千万不要踩到它们。”

说着他打开手电,手电一开,四周的绿色星海马上便消失了,一下子又变成无边际的黑暗。

这些蚰蜒的保护色太过厉害,如果我们不关掉手电,根本无法察觉,我不由一阵后怕,要是刚才爬的时候,不小心按死一只,恶心不说,弄不好就死在了这里。

我们收敛心神继续顺着石头的阶梯缓慢的向上爬去,小心翼翼地过了温泉的这一段区域,石纹蚰蜒逐渐减少,到了后来就看不到了,显然就如顺子说的,雪山的生态链接,都是围绕着温泉。

不过刚才的那种景象,真是太壮观了,如果有机会,我真的很想多看几眼,很难想象这么丑陋的虫子能够组成如此美丽的景象,这个世界真的是非常奇怪。

没有了石纹蚰蜒,我们的速度也相对的快了起来,但是上方的黑暗似乎是无穷无尽,不知道什么时候我们才能走出缝隙,走到这条天然排道的另一个出口。

胖子边爬边问道:“对了,老爷子,我问你个事儿。在车站那哈儿,你和我们讲的,那九龙抬尸是怎么回事儿?老子一直听着,可就没听你再提起过?”

陈皮阿四停下来看了他一眼,又看了一眼华和尚,示意他来说。华和尚就解释道:“我们也不知道,我们所有的信息都是那条龙鱼上来的。九龙抬尸可能是一种失传的丧葬制度,那原文字的记载,似乎是说万奴皇帝的棺材,是由九条龙抬着,九条龙守着他的尸体,没有任何人可以靠近,不过女真语言几乎要失传了,我翻译的东西。也不知道是不是那个意思。”

接着他把原文念出来给我们听了一遍,女真的发音实在是太陌生,我压根什么都没听懂。

“哇,要是这鱼上面的字是真的,那我们要开那万奴皇帝的棺椁,岂不是还得先学哪咤,大战龙王三太子?”叶成开玩笑道。

“那你就别操这份心了,我看这九龙抬尸棺,大概也就是棺材下面雕刻着九条龙这样的性质,意思一下。”胖子道:“要真有龙,那咱们就发财了,逮他一条回去,往故宫里一放,保管人山人海,光收票子钱就得好几万。”

我道:“就你这点出息,光惦记钱了,你要真逮的到龙,那你就是孙大圣,我还没见过孙大圣是你这身材的。”

胖子听了大怒,骂道:“胖又怎么样?胖爷我上天下地,靠的就是这身神膘。晃一晃风云骤变,抖一抖地动山摇——哎呀……”

胖子话还没说完,忽然就是阵乱风从峭壁的一边吹了过来,吹的他几乎摔下去,我赶紧扯住他,把他拉回贴到悬崖上。转头一看,原来是缝隙到前面到头了,阶梯已经到了缝隙的尽头,再走过去,外面似乎是一个很大的空间,但是一片漆黑,什么也看不清楚。

到了!我心里突然一阵激动。

几个人不再说话,蒙着头向着边缘的极限靠拢,那里有一个突出的山岩,我们爬了上去。华和尚先打起一个冷烟火,四周照了照,除了我们站的地方的峭壁,前面什么都照不到。

然后他把冷烟火往峭壁下一扔,冷烟火直线坠下,一下子就变成一个小点,看着它一直变小一直变小,掉落到地的时候,几乎都看不到了。

我们不由咋舌,前面到底是什么地方?怎么好像是一个被悬崖包围的巨大的盆地一样。

“照明弹。”陈皮阿四说道。

“砰!”一声,马上,流星一样的照明弹滑过一道悠长的弧线,射入面前的黑暗里面,直射出去一百六七十米,开始下降,然后一团耀眼的白色炽球炸了起来,光线一下子把前面整个黑暗照了起来。

我想举起望远镜往前看,但是手举到一半,我就呆住了,一下子我的耳朵听不见任何的声音,时间也好像凝固了一样。

白色光线的照耀下,一个无比巨大,直径最起码有3公里的火山口,出现在了我们的面前,巨型的灰色玄武岩形成的巨大盆地,犹如一个巨型的石碗,而我们立在一边的碗壁上,犹如几只小蚂蚁,无比的渺小。

“想不到直接就连到火山里来了。”边上传来一个人的声音,但是这个人是谁我已经分不清楚了,脑子里只剩下了眼前的壮观景象。

如果说九头蛇柏和青铜古树只是给我一种奇迹的感觉的话,那这个埋藏在地下的火山口盆地,简直就是神的痕迹了。

盆地里面覆盖着大量已经死去的树木,显然这个火山口曾经暴露于大气中,这里原先必然是一个“地下森林”,可能是由于火山喷发,或者突然的火山活动,这里的树木都硫化而死,现在森林的遗骸还矗立在盆地之中。

“看那里。”继续有人叫道,我已经分不清楚是谁。接着又是两发信号弹打了出来,飞向火山口的上方。

在加强光线的照耀下,我们看到一片宏伟的建筑群,出现在了火山盆地的中央地下森林的深处,黑幽幽的巨大黑色石城,无法看清楚全貌。

那难道就是我们这一次的目的地,万奴皇帝万世的陵寝?云顶天宫的地宫,竟然会是在火山口之中?

\chapter{火山口(下)}

建筑群的规模之大,出乎我的想象,要是这些建筑下面就是地宫的话,那这里的规模恐怕可以跟秦皇陵一拼了。

按照海底墓穴影画里的景象,真正的云顶天宫本来应该是在我们的头顶上,那雪崩之后,上面的灵宫可能给全部压垮,不知道我们头顶上到底有多少深的积雪作为这地宫的封土。

重新打出的信号弹都熄灭在了黑暗里,黑暗重新包围过来,我们的光线又变成手里几盏明显电力不足的手电。

除了顺子之外,所有人的脸上都带着一种近乎狂热的兴奋。盗墓代表着人类一种最原始的欲望,求得财富和探询死亡,这种刺激,恐怕是人就无法避免的。

足足过了十分钟,我们才缓过来,就准备下去,陈皮阿四对华和尚道:“把没用的东西留下,准备绳子,我们轻装上阵。”

华和尚马上开始准备。我们整顿了一下装备,把抛弃的一部分没有必要的东西,放在这里的平台上,免得负重攀岩,产生不必要的风险。

接着我们全部带上防毒面具,然后用标准登山的步骤,一步一步的用绳索爬下悬崖去。

下面是大量死去的树木,弥漫着奇怪的气味,就连防毒面具,也无法过滤掉。所有人下来之后,就听到潘子说道:“这里是个死坑子,我们得快点,呆久了,可能会缺氧而死。我在部队的时候听过,这种地方鸟都飞不过去。”

那是火山活动所挥发出的含硫毒气。毒性之烈,很难想象。

华和尚打起照明力度很强的冷烟火,照亮四周的环境,我们环视了一圈,脚下是石板子铺成的两车宽的石道,几乎是笔直地就通向前方,这是陵墓的神道,直通向陵墓的正门。这里隐隐约约就能看到尽头一片黑色的巨大影子。

华和尚问陈皮阿四:“咱们怎么走?”

“顺着神道,先进皇陵再说。”陈皮阿四回道。

我们都没有对付皇陵的经验,此时也没有其他的想法,于是不作废话,跟在后面,一路小跑走了过去。

翻过很多倒塌在神道上的死树,很快来到了一处石门处,石头很高,有点像我们在古村中的牌坊。这是皇陵的第一道石门,叫做天门,过了石门之后,神道两边便会出现大量的石头雕刻。

经过石门的时候,陈皮阿四就道:“出来的时候,记得倒走,免的撞了断头门。”

我在爷爷笔记上看到过这个讲究,这第一道石头门,有着很诡异的身份,这门之前,就是屠杀抬棺和送殡队伍的地方,入殓大典完成之后,所有人出这道门的时候,就会被喀嚓一刀,所以这一道门等于就是阴阳之门,盗墓者要是顺着神道而入,或者是进入地宫的第一道大门,那出来的时候,必须倒着出来,不然就很麻烦。

当然几乎没有土夫子有机会能顺着神道进入皇陵盗掘,我们可能是极其稀少的几个之一。历代能够盗掘皇陵的人,不是军阀就是枭雄,他们当然不怕所谓的断头门。

过了天门,神道两边每隔五米就是白色石人石马,我们不考古,这东西也搬不走,一路看也不看,就直奔前方而去。

跑着跑着,忽然,跑在我前面的胖子停了下来,我跟在胖子后面,撞了个满怀,摔倒在地。

这一下实在突然,胖子也给我撞的差点扑倒,我忙问他干什么。

胖子转头看了看身后,脸色苍白,轻声说道:“好像路边站着个人。”

前面几个人发现我们停下来了,都折返了回来。潘子问道:“怎么回事?”

胖子把他看到的一说,其他几个人都有点不信,潘子就道:“是石头人吧,你看错了吧?”

胖子摇头,“一闪就过去了,我刚反应过来,你看我,一下子一身冷汗,应该没看错。”

“有没有看清楚?”

“好像是个女人,也不能肯定。”胖子道,“跑的太快,我没看清楚。”

我们都把手电照向后面的几个石头人,石头人每隔五米一个,刚才一瞬已经跑过六七个了,手电能照到的范围内,没有胖子说的那个女人,也许还在更后面。

华和尚问:“老爷子,要不要回去看看?说不定是那帮人里面的那个女的?”

华和尚指的是阿宁,我心说怎么可能,他们走的是云顶天宫的正门,就算他们已经成功的越过边防,那现在也应该是在我们头顶上打盗洞,绝对没有我们这么快的。

胖子也道:“那肯定不是,要是那娘们,老子肯定一眼就能认出来。”

陈皮阿四犹豫了一下,马上对华和尚道:“你和其他人先过去,”然后拍了顺子一把:“你陪我去看看。”

\chapter{门殿(一)}

顺子给拍的一楞,不知道怎么回事情,不仅是他,其他人也都楞了一下,不知道陈皮阿四怎么了。

我当时一刹那,甚至以为陈皮阿四想支开我们,杀顺子灭口,但是一想又不对。一来他90多岁,要杀一个退伍的壮年正规军,就算是偷袭,也未必能得手;二来,我们的回路还是靠着顺子,所以应该不会借这个机会杀他灭口。

我对于陈皮阿四的这个举动不是很理解,于是拍了顺子一下让他小心。

顺子也不知道有没有意识到,看了我一眼就跟着陈皮阿四走了过去。

我们马上回头,顺着神道继续向前跑去,身上的装备幸亏放掉了很多,不然这样的运动强度,恐怕没人能坚持住。

这一条神道一共有六道石门,这是汉家佛教的六道轮回,而女真信奉萨满,这汉人设计的痕迹随处可见。

我跑的飞快,不由的已经有点晕眩的感觉,身上裸露的皮肤可开始瘙痒起来,可见四周的空气实在是不妙。

不知不觉,手电的光圈中已经可以照射到一些黑色残檐断壁,很快神道尽头的祭坛到了。祭坛的后面,六十阶破败的石阶之上,便是皇陵的正门。

在与传统的墓葬观念中,陵和墓经常是混为一谈,其实陵墓,是两种不同的东西,陵就是用来祭祀和入殓仪式的地上建筑,而墓,才是指地下的地宫。

陵墓并不一定要是同在一起,很多的陵墓相差十万八千里,就如成吉思汗陵就在内蒙古鄂尔多斯草原中部,但是陵中的棺木只有附着成吉思汗灵魂的驼毛,他的尸体和陪葬品藏于草原的何处,无人知晓。

这云顶天宫用三层的结构,我们头顶上在海底墓中看到的那些宫殿是象征性的灵宫,和地下的皇陵和地底的王墓,构成三千世界,也象征着万奴王神人鬼的身份变化。

整片皇陵的建筑风格和明宫很像,在峭壁上看的时候,规模巨大,皇气逼人,由于大量使用那种黑色的石料,所以在壮观之余,还显得有一丝诡异和神秘。但是我们一进入陵宫,这种感觉就消失了,满眼是萧索和残破,如果不是一些大型的犹如庙宇一样的楼殿还耸立在那里,我们不免就要失望。

这里空气不流通,也没有狂风日晒,这里的建筑应该保存的非常好才对,怎么会残破成这个样子?

我们踩着巨大的可以并驰十辆马车的陵阶,走入皇陵的正门之内。那巨大的陵门早已坍塌,打满乳头钉的巨大门板倒在地上,我们踩着门旁若无人的就走了进去。

正门进去,是陵宫的门殿,古代葬书皇陵篇,四道龙楼盘宝殿,九尾仙车入黄泉,这就是四道龙楼里面的第一殿。此时候我已经觉得口鼻的内部犹如灼烧一样的难受,招呼他们几个动作快点。

门殿大概有两个篮球场大,两边是迎驾的铜马车。在后面的深墙边上,左右各是两座黑色雕像,已经蒙尘。雕像面目狰狞,冷面怒目,似乎是萨满的图腾,上面的辅梁柱已经倒塌,瓦片云当摔了一地,幸亏这里不会下雨,不然这里早就淹了。

我们见没有什么特别起眼的东西,就想穿过门殿,向皇陵的中心走去。才走了几步,忽然胖子脚下一滑,不知道踩到了什么东西,“哎哟”一声,摔了个四脚朝天,门殿地板上全是碎瓦片,这一跤摔的他就要了命了,疼的直吡牙。

我一下子觉得奇怪,这地面这样,要是绊一跤还可以说说,怎么会滑倒?胖子自己也觉得奇怪,一边捂着屁股一边就走回去,看自己踩的地方。

那地方只有他摔倒时候划出来的一条痕迹,他踩到的东西已经不见了,他顺着痕迹看过去,翻了几片瓦片,也没有。

“你别不是鬼绊脚了??”潘子问胖子。

胖子摇头,忽然感觉到了什么,招手让我们停下,自己蹲了下来,翻起了自己的一只鞋。

我们围过去一看,原来他那登山鞋的鞋钉里面,竟然卡着一枚子弹壳。

众人脸色就一变,潘子接过来,闻了闻,随即想到自己带着防毒面具,又用手捏了捏,道:“有温度,他娘的还是刚从枪膛里打出来不久的。”

“有人先到了?”我一愣,难道阿宁他们这么神通广大,竟然能够比我们还要快?

但是,为什么要在这里开枪呢?

“点个火,四周看看还有什么?”潘子道。

华和尚马上打起冷烟火,打大照明的力度。我们四处查看,门殿里面一片混乱。我们分散开来,很快我们就在一根柱子上,发现了一大串连续射击的子弹孔,直射着就上去了。

“看上去好像是有什么东西顺着这柱子下来,然后子弹就跟着它扫下来啊。”

潘子走上去,看了看子弹孔,挖了一下,摇头道:“不是,正好相反,看这子弹偏移的角度,枪口是顺着柱子往上甩。”

胖子用手电照着子弹孔,一点一点的看上去,最后一直看到了高高在上的横梁上,我们马上看到一个黑色的影子,悬挂在横梁上。

看影子的姿势,那应该是一个死人,似乎是阿宁队伍中的,因为我看到一把56式老步枪挂在他的肩膀上,整个人无力的垂在那里。

众人都吓了一跳,不明白这个人怎么会死在横梁上,我们的手电照过去,看到了那人的脸。死的是个男人,脸上带着小型的鼻吸式防毒面具(这东西非常先进,重量很轻,效果也比我们脸上的好,我最后才听说有这个东西,没想到今天就给看见了),由脸形判断应该有斯拉夫血统,不知道是怎么死的,眼睛瞪的牛大,因为面具的关系,看不到他的表情。

尸体由一根什么东西吊在悬梁上的,距离太远了,也不知道是不是绳子。

几个人想爬上去,给潘子拦住,这个人死的那么怪,肯定有问题。这时候胖子拍了拍我,指了指横梁的其他地方:“各位,还不止一个。”

我们看过去,只见上面横梁的其他地方,还有六七具尸体,都是悬空挂在上面,犹如吊死鬼一样。

这些人都是清一色的登山装,身上都挂着56式的国产步枪,我不由心里感觉到一股异样,五六步枪的破坏力很强,有这东西在手,粽子也吃不消十几发子弹,是什么东西杀了他们,而且就算这里有过枪战,这些尸体怎么会跑到横梁上去?

越想越觉得不对,此地不宜久留,我招呼几个人,快点通过门殿,这地方邪门。

可是转头一看,却发现胖子不见了,再用手电一打,发现胖子不知道什么时候已经踩着一边的雕像正往横梁上爬。

\chapter{门殿(二)}

“你搞什么?快下来!”我急的大叫。这样的局面,他竟然还会往横梁上爬,我真不知道他脑子是怎么长的。

胖子不理我,他的身手很快,几步便已经探到横梁之上,回头道:“慌什么?你胖爷我又不是三岁小孩子,有不对劲我自然会下来。”说着便顺着横梁,向离他最近的尸体走去。

我一下醒悟,知道胖子是盯上那把56式了,这家伙手里没枪,一路上一直不自在,现在看到这么好的枪还不兴奋。这家伙无组织无纪律我是习惯了,现在气的七窍生烟,也拿他没有办法。

胖子在小心翼翼的走了几步,他的体重很厉害,整个门殿的檐顶都顺着他脚步的震动,发出一种让人不安的声音,同时大量的碎木屑从上面掉了下来。我们条件反射的就往后直退,怕胖子把头顶整个结构给踩塌了。

潘子拍着身上的垃圾骂道:“你他娘的给我悠着点儿,等一下咱们几个都给你断送了。”

胖子做了个抱歉的手势,大跨步走到那尸体的边上,第一件事情就是把尸体身上的56式勾了上来,拿到手,马上退膛看子弹,然后从上面扔给潘子,又把尸体身上的子弹包挑了过来。背到自己身上,最后才去看那尸体。

我看着胖子一点一点的把尸体的防毒面具解了下来。面具里面是一张中年老外的脸,整张脸扭曲着,脸色发青,嘴巴张地离奇的大,似乎死的时候正在大叫。死亡应该是瞬间的,所以死的时候的表情才会凝固的如此强烈。

我看他脸色发青,大叫:“别碰他,看他脸色,应该是中毒死的。”

胖子点了点头,带上手套。然后去看吊着尸体的“绳子”,这些人肯定不会是自己吊在上面的,那这些绳子是怎么回事?我们都很想知道。

然而胖子上去看了一眼,脸色却仍然很是疑惑。

我问道:“发现什么了?”

胖子道:“这些他娘的好象是头发啊……”

“头发?”我奇怪道。

胖子点了点头。道:“还他娘的挺长,怎么这些人难道都是娘们?”胖子将尸体提起来一点,“不对……这头发是从他脖子里出来的,不是头发,我靠,他娘的难道是嘎吱窝毛?这老外就是厉害,嘎吱窝毛都这么长。”

说着已经掏出匕首,想把着那死人的“头发”切断,把尸体放下来让我看。可是他用匕首划了两下,那“头发”却没有断,似乎非常的坚韧,又拿出打火机,想烧一烧看看。

我心说我可不想看这种尸体,就对他大叫:“算了,我没兴趣看尸体,你快点下来,别搞了,万一有毒就麻烦了。”

胖子一想也是,收起打火机回道:“再等一下,马上马上!”说着却向另一具尸体跑去,看样子他是一把枪也不想放过了。

我看着这尸体似乎也没有什么危险,也就不去阻止他了,他还是老样子,到了尸体边上先把枪勾了下来,丢给我,然后又想挑那尸体的子弹袋,就在这个时候,我忽然看到这一具尸体的手,忽然动了一下。

我脑子一紧,忽然意识到不对,胖子正要去摘他的防毒面具,我忙大叫:“等等!这个好象还活着!别摘他面具!”

胖子啊了一声,“真的?”说着按一下尸体的脉搏,脸色也一变,忙拿出打火机,将上面的“头发”烧断,这尸体马上就从横梁上掉了下来,我和华和尚将他接住,放倒在地上。华和尚带上手套一翻他的脖子,只见这吊着尸体的“头发”果然似乎是从这人的背上长出来的。

华和尚又翻了翻他的眼皮,摇了摇头道:“死是没死,不过也差不多了,瞳孔几乎放大了。”

我看着这人似乎是中国人,习惯性的问道:“还有没有救?”

华和尚摇了摇头:“咱们犯不着救他,一来也只能让他多撑一会儿,死的时候更难受,二来带着走麻烦。”

我道:“那他还没死,把他丢在这里好象不太好吧?”

华和尚笑着摇头,似乎觉得我很好笑,一边抽出腰里的军刀,把那人的脖子扯起来,我一看顿觉不妙,忙一把把他拉住,道:“你干什么?”

“他现在中毒了,死的时候很难受的,我给他放血,可以死的舒服点。”

我一听傻了,这是什么逻辑?刚想摇头说不行,突然那“尸体”一下子痉挛了一下,手猛的就拉住了华和尚的手,睁开了眼睛,人还在不停的发抖。

华和尚吓了一跳,忙把手挣开,退后了好几步。

那人看了看我,又看了看华和尚,显然意识有所恢复。突然就挺起身子,痛苦的叫了起来,我一点也听不清楚他在叫什么,忙去压他,但是这人力气很大,我和华和尚都给甩了开去。那人在地上翻来滚去,撕心裂肺的大叫,嘴巴越张越大,竟然张到了人类绝对不可能张到的极限,而且脖子也膨胀起来,好象要爆炸一样,大量的腥臭的液体从他嘴巴里吐了出来。

潘子看不下去,拉上枪栓,“砰”一声,送了那人一程。

枪声之响简直出乎我的意料,我一下子耳朵就一疼,只见潘子这一枪直接打中他的心脏,大量的血从尸体上涌了出来,尸体扭动了两下,停下来不动了。

“他刚才在叫什么?”华和尚一头冷汗,问:“有人听懂了吗?”

“客家话,他叫成这样,我也听不懂多少,不过似乎是在叫‘背上、背上’”叶成道。

“背上,难道他背上有什么蹊跷?”华和尚将尸体翻了过来,想割开他的衣服,看看背上到底是怎么回事。

我看着到处是血,感觉头开始无晕起来,转过头不去看,让胖子快点下来。

胖子还蹲在房梁上,居高临下的看着我们,这时候已经点起来烟。看我转过来,马上道:“别催了,你他娘的快和我老娘一样了,我向毛主席保证,抽完这烟我就下来。”

我心说你带着防毒面具怎么抽,一看他,却突然一愣,随即头皮一炸,一声大叫就摔倒在地上。

只见胖子的肩膀后面,竟然冒出了一张陌生的瞪着眼睛的白脸,仔细一看,竟然是一个奇怪的人趴在胖子的背上。而胖子似乎一点也没有察觉到。

\chapter{门殿(三)}

几个人都给我叫声吸引,转头一看,叶成就怪叫了一声,都不约而同的后退了一大步。潘子条件反射,“喀嚓”一声上弹,枪就抬了起来,但是却没有开枪。

(说来也奇怪,不知道为什么,我从来没有听说过有人在墓穴里用手枪打粽子,从来没有,不知道这是祖宗的规矩,还是如果这样做了,会有什么我们不知道的后果,后来问了华和尚,他说表面上的原因是很多尸体都有尸毒,要是只霉粽子,这一枪下去,尸水溅到哪里哪里就废了,而且枪的声音太容易招惹麻烦了,但是实际上是怎么回事情,他也不清楚。)

胖子正琢磨着怎么把烟塞到防毒面具里去,一下子给我们的动静吓了一跳,不知道怎么回事情。但是一看潘子的枪指的地方,是他头边上,就知道瞄的不是他,他反应马上意识到了什么,转头就往肩膀后看去。

这一看,他就和那脸对上了,胖子一下子就蒙了,手里的香烟一下子掉到梁上,僵在那里。

趴在胖子背上的人,鬼气森森的缩在胖子的肩膀后面,也没有因为胖子的转头做出任何反应,两个人就这样大眼瞪小眼,含情脉脉的看着。

这人是从什么地方冒出来的!?我脑子里炸开了锅,刚才我们进来的时候都用手电扫过一遍了,虽然没有如何如何的仔细,但是这么大一个人,肯定是躲不掉的,也就是说刚才我们进来的时候,这“人”还不在这里,那他怎么就能突然就出现在胖子背上?

会不会是阿宁他们一伙的,在这里着了什么道了?还是干脆他娘的就是死在这皇陵中的冤魂?

我古怪的事情见多了,可是处于阴森的地下皇陵之中,一下子也是头皮发炸,寒毛直竖,不由也感觉自己的背上好像有什么东西趴着一样,浑身的不自在。

胖子脸色惨白,冷汗直流,不过他到底是个人物,这时候已经反映过来,人不敢动,但是我看到他的手缓缓的做了一个手枪的手势,估计是让潘子开枪。

潘子摆了摆手,让他把头移过去点,自己好瞄准。这时候华和尚举起两只手,轻声说:“等等,先看看,别是个活人。”

“长成这样都能叫活人?”潘子轻声道。

华和尚摆手让他别说话,自己用手电一点一点移向胖子肩膀后面的地方。手电照了上去,那人被光线一照,头一下子转向我们。我看到一张无法形容的脸,整张脸是凹陷下去的。鼻子的地方只有一个大洞,眼窝深地畸形,两只眼睛犹如电筒一样反射着手电的光芒,嘴巴的地方,看上去竟然像一只猫头鹰。

潘子就犹豫了,枪就松了下来,看向我们,“他妈的是只夜猫子?”

我心说怎么可能,这里的空气质量这样,基本上不可能存在生物,夜猫子不可能在这里生存。而且要这是夜猫子,那他娘的也太大了。

可是单看这张脸,还真是非常像,夜猫子飞翔的时候是没有声音,难道就是这样,他从瓦顶上无声息的飞了下来,停到了胖子的身上?那胖子怎么会一点感觉也没有?

胖子脸上的汗就像瀑布一样,一边还在让潘子开枪,一边手开始往腰上的匕首伸去。大概是看我们没反应,自己忍不住要动手了,我忙向胖子一摆手,让他别动,没搞清楚之前,万万不要硬来。

胖子朝我疵牙,表示抗议?

没想到他才一露牙齿,背上人突然似乎受了刺激,一下子凹陷的脸扭曲了起来,人往后一仰,突然嘴巴就张了开来,我靠!一口的2寸长的獠牙,那肯定就不是鸟了,而且越张越大,很快就超过了人类所能张的极限。

我一看糟糕,胖子要倒霉了!潘子猛把枪托压紧自己的肩膀,一瞄那嘴巴,刚想开枪,突然“嗖”一声,一道劲风在我面前飞过,一个东西就从门殿外面扔了进来,一下打在潘子的枪上,枪头一偏,一连串子弹就贴胖子的耳朵扫了上去。胖子吓的大骂:“你他娘的打哪里啊?”

我转头一看,陈皮阿四和顺子冲了进来,陈皮阿四对潘子大叫:“放下枪!”

上面那东西一口已经朝胖子的脖子咬下来了,胖子脑袋一撞,把那家伙的脑袋撞开,然后扭过身子就用反手掰住后面那东西的嘴巴,想把它给甩下去,但是那东西不知道是怎么趴在他背上的,怎么甩都甩不掉,胖子大叫:“他娘的,快上个人来帮忙!”

叶成拔出刀就想上去,我大叫:“不行!梁要塌了!胖子你快跳下来!”

胖子根本没听见,还在那里大叫:“你们几个没良心的,快点!”

陈皮阿四猛一甩手,一颗铁弹子就打在胖子脚上,胖子“哎呀”一声吃痛,脚一松一滑,整条梁柱因为他的动作喀嚓一声往下一斜,胖子一下子就平衡不住了,人一倒就摔了下来。

横梁离地的高度,摔下来不是说没事情就没事情的,幸好下面挂着一具尸体,他下来的时候用力扯了一下,在半空缓冲了一下力道,重重就摔进地上的瓦砾堆里。我们赶紧冲过去一看,几个人都一愣:胖子背后那东西不见了,什么都没有……

我一下子想起柱子上那些弹孔了,马上意识到不对,一甩手道:“那东西没掉下来!当心头顶!”话还没说完,头上一个影子闪电般掠过,一边的顺子一个就地打滚,左肩膀上已经多了三道血痕。

我马上端起枪,但是老56比我想象的要重多了,我端的不是很稳,抬了两下,枪口竟然没抬起来。胖子爬起来,一把夺过我的枪,凭着感觉就朝顶上扫了一圈。大量的瓦片稀里哗啦的掉了下来。我们的手电全部举上去给他照明,但是等枪雾散尽,顶上却什么都没有,刚才那东西不知道到什么地方了。

“这他娘的到底是什么东西?”胖子心有余悸。

“你还问我们,它趴在你身上你都没感觉,你干什么吃的?”潘子大骂。

胖子大怒,刚想骂回去,忽然人一顿,我们转头一看,我靠,那张怪脸,不知道什么时候竟然从潘子的肩膀后面探了出来,幽幽的看着我们,而潘子自己也一点都没有发觉。

我们马上全部退开潘子,潘子一看我们的反应,脸马上绿了,叫道:“你们干什么?”还没等他回头,肩膀后的那东西猛的就张大嘴巴,一下子一口的獠牙。

胖子抬枪一个点射,砰一声那东西半边脑袋就给轰飞了,顿时绿水四溅,溅了我们一身,一股极度难闻的味道弥漫了开来。

我一下子以为搞定了,一看又不对,那张半个脑袋的巨大的嘴巴里面,竟然还隐约有着一张脸!

“该死!”我听见一边的顺子轻声叫了一声,一个飞扑就撞到了潘子身上,潘子给撞的飞了出去。他倒地后一个转身就坐了起来,不知道什么时候军刀已经在手,反手就往身后捅。

但是他身后的东西却已经不见了,坐在他后的是刚撞他的顺子,那一刀就直了过去。幸好顺子反应快,一把压住他的手,把他手给扭了过来,同时大叫:“刚才谁开过枪?!”

\chapter{门殿(四)}

胖子马上举手,“我!”

“还有我!”潘子也举起了手。

顺子不知道为什么,突然眉宇中多了一股不容质疑的气质,一甩手:“开过枪的人留下!其他人跑!一直往前跑!绝对不能回头!”

我一看,一数,哎呀,我们的人全都留下了,那我怎么办,跟着陈皮阿四岂不是等宰吗?忙也一举手:“我……我忘记了,我也开了!”

叶成他们一下子也不知道怎么反应,这时候我们就听到门殿顶上传来了瓦片碎裂的声音,似乎有很多东西正在爬上殿上的瓦顶,数量之多,难以想象。几个人都大惊失色,不知道到底出了什么事情。

“来不及了,还不走!”顺子大叫。

陈皮阿四看了我们一眼,一甩手,对华和尚他们说:“走!”说着三个人快速跑出了前殿。

我心里觉的奇怪,但是形势已经不容我多想,头顶上的瓦片碎裂声越来越多,胖子甩出自己的子弹袋子给潘子,两把枪都上镗,我们围成一个圈,问顺子:“上面到底是什么东西,我们怎么办?”

顺子沉声道:“不知道。”

“那你让他们跑什么?”潘子掉眼睛。

顺子道,“我只是想让你们和那老头子分开来,这不是我的主意,你们三叔的吩咐。”

我们一听,全部都转头看向他,心说什么,我三叔吩咐的?潘子就问道:“那你是什么人?”

“别问这么多了。”顺子道:“我现在带你们去见你们的三叔,到时候你们自己去问他吧。”

我浑身一紧,刚想问:我三叔现在也在这皇陵里?突然头顶上发出一连串破碎声,瓦片下雨一样直往下掉,我们护住头全部都往上看去。只见在手电的光斑里,无数的影子在挪动,似乎都是刚才的那种东西。

顺子甩手道:“刚才你们枪声一响,这死树林里面到处都是声音,都向这里围过来了。”

“那我们为什么不跑?”潘子听着四周已经密集的让人无法分辨数量的爬动声,紧张问:“在这里不是等死吗?”

“等他们再走远一点。”顺子一边看了看身后,陈皮阿四似乎已经跑远了,转头对他道:“走!”说着一拍我们,一马当先向着前殿的出口跑去,我们紧跟其后。

门殿之外可以看到神道的衍生殿,前面出现一道汉白玉二十拱长桥,桥上吊着两条不知什么材质的盘龙,顺着桥两边的栏杆缠绕着,玉色极好,竟然没有一丝缝隙,似乎是整体雕刻而成,桥下就是内皇陵的护城河,地下不知道有没有水。

我们才跑出几步,后面劲风就起,我们几个全部就势一滚,胖子回手就是一个无目标的点射,黑暗中听到一声轻微的嘶叫,不知道打中了什么东西,一团东西就摔进了桥下的深渊里。

黑暗中弥漫着一种躁动,我隐约感觉到这种东西似乎能飞,但是手电狂扫却什么也扫不到。

我们爬起来继续往前跑,头顶一阵一阵,似乎有东西在贴着我们的头皮盘旋,胖子对着天上边跑边扫射,很快我们便跑到了桥上,突然我就感觉背上被什么东西带了一下,一下子就摔了出去,我一个反身爬起来,还没反应过来,胖子一个枪托就从我耳朵边上砸了过去,我就感觉一个东西从我背上摔了出去。

回头一看,只看见一个黑色的影子正挣扎的爬起来。潘子抬手就是一枪把它打成两截,接着胖子就对着天上狂扫了几枪,子弹的曳光闪过,无数的影子盘旋在我们头上。

“这些到底是什么?”我咋舌道。

“太多了,打不光,我们怎么走?”潘子大叫道,问顺子。“三爷到底在哪里?我们怎么走!”

再往前就是四道龙楼殿的第二殿,到了那里免不了又要和陈皮阿四碰头,说实在的他们几个人在一起我心理压力很大,而且现在已经有了三叔的下落,我恨不得马上找到他,问问到底整件事情是怎么回事。

“你们三爷应该在地下玄宫了。”顺子道。

“地宫?”胖子又是一枪托,也不知道打下什么东西,“太好了,妈的省事了,地宫的入口在什么地方?”

顺子道:“我不知道。”

一下子几个人都楞了一下,看下顺子,一看他的表情就知道他不是开玩笑。胖子就骂:“你不知道你说带我们去见他,这皇陵这么大,我们怎么找?”

一般来说地宫的正规入口就是顺着神道进入的第三道龙楼——天殿之内,但是必然是压在铜鼎之下,有七十多道青砖加上铅浆铁水的装甲等着我们,现代工兵团没有十天半月也挖不开,但是地宫肯定有秘密入口,而且应该就在皇陵建筑之内,中轴线上。慈禧陵的地宫入口就是在陵宫影壁里,但是现在这情形哪有时间去挖洞。

顺子非常镇静,矮着身子,对我道:“你三叔说,这里是‘玄武拒尸’之地,他说告诉你这话,你自然就知道是在什么地方,你想想有没有印象。”

我一听奇怪,“玄武拒尸”是玩笑之说,也就是风水理论中,集合了世界上最差的风水的地方,这种地方和理论中极品宝穴“九龙盘花”相同,是理论中的东西,世界上是不会有的。我问道:“他真这么说?还有没有说什么?”

葬书上说:“地有四势,气从八方,故砂以左为青龙,右为白虎,前为朱雀,后为玄武。玄武垂头,朱雀翔舞,青龙蜿蜒,白虎顺俯。形势反此,法当破死。故虎蹲谓之衔尸,龙踞谓之嫉主,玄武不垂者拒尸,朱雀不舞者腾去……”

顺子矮着头看着四周,急促道,“没了,当时你三叔似乎在躲避什么人,所以非常匆忙,你三叔是安排我在村子里面接应你们,带你们进山,然后就是带这几句话。”

我听着,忽然站定,心里哑然。如果这里真的是“玄武拒尸”,那葬在这里,后代死绝,老婆偷人,发生任何事情都不奇怪,汪藏海和万奴皇帝这么大仇?

而按照陈皮老头的说法,这里的风水应该是极其好才对,怎么会是“玄武拒尸”呢?

我一下子很后悔以前没有好好的留意这些东西,如果来此时候能看懂一些东西,现在应该一下就能领悟出什么意思了。

胖子也懂这些东西,甚至有些方面比我还知道的多一点,这时候也很疑惑,叫道:“放屁,不可能,皇陵玄宫所在,怎么可能是‘玄武拒尸’的地方。”

潘子一边又是一个扫射,将逼下来的东西逼开,回头道:“也不是不可能啊,风水对人来讲的,你没听那和尚说吗?这皇陵里埋的不是人啊,说不定这种奇怪的格局差异,和这有关系!”

我知道潘子的话纯属气话,以东夏国薄微的国力,建造这些建筑应该已经倾注了全部的力量和资源,能够发动如此巨大的工程的,只有万奴王一个人,而且我不相信当时的末代万奴王还有如此的威信,建造这座皇陵,必然夹杂某种宗教的成份,那个时期,万奴王很可能是人神一体的宗教偶像。

铜鱼上说历代的万奴王都是从地里来的妖孽,我认为不能直白的去理解,铜鱼之上的信息应该另有隐讳,具体是指什么,可能要破译了我手上的那两条铜鱼才能够知道。

但是胖子不买帐,一听潘子这么说,怒道:“你他娘的别不懂装懂,不是人,难道会是条狗吗?不论陵墓里葬的是什么东西,按照风水上的讲法,都不应该选择‘败穴’之地,你以为棺材里是妖怪,那葬它的风水就该人相反吗?没这回事情!而且你看这里的规模,少说也是个城邦级别的,何必为妖孽修建如此规格的陵寝?”

潘子的业务知识没胖子丰富,一下子语塞,不知道反驳什么好。

顺子对我们道:“几位老板,我听不懂你们说什么,别扯这些个jb蛋了,到底是怎么回事情?谁懂谁说,快点!”

胖子道:“这还不简单,葬经看过没?你知道什么叫:‘地有四势,气从八方,前为朱雀,后为玄武’。玄武就是后面的意思,拒尸,就是拒绝尸体,拒绝了那就是没尸体的意思,合起来说就是后面没尸体,那不就摆明了吗?尸体在前面!”

我一听心说我靠,这句话是这个意思嘛?要给郭璞(葬经作者)听到,还不从坟墓里爬出来把你掐死。

顺子不懂这些,还真信了,道:“这范围也太广了点,要说在前面,是在什么的前面?就凭这个也找不到入口啊?”

我对他说别听胖子胡扯,哪有这么解释葬经的,道:“三叔既然没有直接把玄宫入口的方位说出来。肯定是因为照直说,你反而无法转达。那就不能单纯从字面意思去理解他的话。像胖子这么猜是没用的。”

胖子不服气。问道:“那你有什么眉目?”

我摇头表示暂时也没有头绪,需要好好想想,三叔精通古代密码和密文,应该从那方面去下功夫,而且既然他认为我能理解,肯定有他的理由,但是现在显然不是思考的时候。

说话间,我们已经退到了石桥的末端。再过去就是皇陵的广场,黑暗中可以看到石桥的末端的地方竖了两块并排的石碑,都有10米多高,一块已经断了,底下由黑色的巨大赑屃驮着,石碑后面的不远处。是一片高耸的巨大黑影。

我知道这里是“皇陵界碑”,石碑之后应该就是通往“往生殿”长生阶,也就是通往幽冥的大门,“皇陵界碑”可以说是真正的人间与幽冥的分界线。因为“皇陵界碑”之后的地方,守陵人都无法进入,几百年前,皇陵封闭地那一刻起,就没有人再踏足界碑对面的那一片区域了。

看见石碑的那一刹那,我突然有了一种非常不祥的预感,似乎前方那一团巨大的黑影中,在这死寂的皇陵内城的某个角落里,有什么东西正在等着我们。

就在这个时候,跑在前面的胖子突然停了下来,一下张开双手,把我们都挡了下来,我上去一看,原来石桥的末端,竟然已经坍塌了,石桥和对面“皇陵界碑”之间,出现了一道大概三米多宽的深渊,手电照下去一片黑气蒙蒙,似乎有水,但是不知道有多深。

“怎么办?”我看向潘子,潘子想也不想,端起枪就道:“还能怎么办?一个一个跳过去,快!”

我一看这距离,不由咽了口唾沫,奥运会那些人能跳多少,八米左右?三米多不算太远,但是对于我这样整天不运动的人来说,想要轻松跳过去还真有点难度。

一边的胖子已经把枪交给顺子,然后自己退后几步,助跑一段后猛的一跃,在空中漫步而过,滚倒在对面的石地上。顺子子把枪再甩给他,然后把我们身上的装备也先甩过去。接着顺子也跳了过去,潘子要给我殿后,让我先跳,我看着前面的深渊,心里一横说死就死吧,对对面的胖子大叫了一声,拉着我点。

胖子满口答应,我退后几步,定了定神,猛的一阵加速,可倒霉的是,就在我想起跳的时候,潘子突然就在后面大叫:“等——!”

此时我已经刹不住车了,一下子高高跃起,猛的向对岸跳去,还下意识的回头一看,奇怪潘子为什么要叫我。

这一看,就看到一个巨大的黑色影子从我左上方俯冲了下来,凌空就抓住我的后领子,一下子爪子勾住了我的衣服,把我往边上一带,我在空中的姿势就失控了,接着爪子就一松,我整个人就翻了一个跟头,就往深渊里掉去。

一刹那间我脑子里一片空白,也不知道怎么办好了。眼前的一切就好像慢动作,看着胖子冲过来,一跃而起想在空中拉住我,但是他的手就在我的领子边上擦了过去,接着潘子举起枪,对着我的头顶“啪啪啪”就是三个点射,子弹呼啸而过,然后我就掉进入了一片黑暗之中,他们的手电光瞬间就消失了。

下落的过程极快,我在空中打了几个转,同时脑子瞬时闪过一连串的念头,这下面是什么?下面是护城河道。一般的护城河有多深,有水吗?我会摔死,或者给这里硫化的水融成一堆骨头?

还没等我想到这些问题的答案,我的背就撞到了一根类似于铁链的物体,整个人差点给拗断了,疼的我眼前一花,接着身体绕这铁链打了一个转,又往下摔去,还没等我缓过来,又撞上另一跟铁链,这一次因为刚才的缓冲,撞的不重,我伸手想去抓,但是抓了个空,我继续下落。

这一连串的撞击把我撞的晕头转向,连坠落时蜷缩身体的姿势也摔没了,接着我就脸朝上重重的摔在了地上,我自己都听到我全身的骨头发出一声闷响,接着耳朵就嗡的一声什么也听不见了。

\chapter{护城河}

落地好几分钟,我完全蒙了,脑子还不知道是怎么回事情,也不知道自己是死了还是没死,接着就有一股辛辣的液体从喉咙喷了出来,倒流进气管,我不停的咳嗽起来,血从我的鼻子里喷出来,流到下巴上。

足足花了半只烟的功夫,我才缓过来,感觉一点一点回归到身上,我颤颤悠悠的坐起来,四周一片漆黑,什么都看不见,我摸了摸地上,都是干燥的石头和沙子,这护城河底是干涸的,幸亏这些石头还算平整,不然我就是不摔死也磕死了。

防毒面具已经裂了,镜片一只碎了,我摸了一下,发现整个防毒面具都凹了进去,再一摸前面,发现我脸摔的地方有一快很尖锐的石头,看样子是幸亏了这面具的保护,我的脸才没摔烂,不过这一下子,防毒面具算是完全已经没用了。

我艰难的扯掉后扣,小心翼翼的把它从脸上解下来,才拿到手上,面具就裂成了四瓣,再也带不起来。

没有了面具,四周空气中的硫磺味道更加浓郁,但是吸了几口似乎没有什么大的不适,看样子潘子所说的这里毒气的厉害程度,并不真是,或者在护城河底下的空气质量还可以。我暗骂了一声,把面具扔到地上,吐掉残留在嘴巴里的血,抬头去看上边。

护城河最起码能有十几米高,上面是灰蒙蒙一片,我只能看到胖子他们的手电从上面照下来,四处划动,似乎在搜索我,还能听到一些叫声,但是也不知道是不是摔着的缘故,我的耳朵里满是刚才落地一刹那的嗡嗡声,实在分辨不出他们在说什么。

我尝试着用力叫了几声,但是一用气,一股撕裂的剧痛就从我的胸口扩散到四周,声音一下子就变成了呻吟,自己也不知道自己在说什么,甚至不知道自己到底有没有发出声音来。

为了让胖子他们知道我还活着,我捡起刚才扔掉的防毒面具,用力敲击地面,发出“啪啪啪”的声音。声音不大,但是在安静的护城河底,却反弹出了回音,十分醒耳。

敲了一会儿,突然一只冷烟火从上面扔了下来,落在我的边上,我骂了一声躲开,接着,我就看到上面一个人的头探出了桥的断面,看脑袋的大小似乎是胖子。

我爬过去,捡起冷烟火对他挥了挥,他马上就看到了,大叫了一声,但是我一点也听不出他到底在说什么,只好发出几声毫无意义的声音,胖子把头缩了回去,不一会儿,从上面就扔下一根绳子,晃晃悠悠垂到河床底部,胖子背着自动步枪开始往下爬。

十几米也就是四五楼的高度,说高不高,说短不短,胖子一下子就滑溜到了底部,放开绳子先用枪指了指四周,看没有什么动静,才跑过来,蹲下来问道:“你他娘的没事情吧?”

我嘶哑着,有气无力道:“没事?你摔一次试试看?”

胖子一看我还能开玩笑,松了口气,对上面打了个呼哨,马上,潘子和顺子背着装备也从上面爬了下来。

他们扶起我,先把我扶到一边的一块石头上,让我靠在哪里,接着让顺子按住我,拿出医药包,给我检查身体。

我看到医药包,心里就稍微安心了一点,心说幸好准备还充分,潘子确定我没有骨折,拿出一些绷带,帮我包扎了一下比较大的伤口,然后骂道:“叫你停你怎么还跳,也亏的你命大,不然你死了我怎么和三爷交代?”

我一听大怒,骂道:“你还说我,我都在半空了,你才叫停,这他奶奶又不是放录像带,还能倒回去——”还没说完。突然胸口一阵绞痛,人几乎就扭曲了起来。

潘子一看吓了一跳,忙按住我,让我别动。

我咬牙切齿,还想骂他一句,但是实在疼的不行,连话也说不出来,只能在那里喘气。

胖子在一边递给我水壶,道:“不过你也算命大了,这样的高度,下面又是石头,一般人下来绝对不死也残废。”

我接过水壶,心说这应该叫做命贱才对,刚才肯定是因为撞到那两根铁链子,自己才没死,也不知道是走运还是倒霉,最近老是碰到高空坠落这种事情,而且还都死不了,真是要了老命了。

喝了几口水,嘴巴里的血都冲掉了,喉咙也好受了一点,我就问他刚才那到底是什么东西,潘子说这次他们看清楚,肯定是一只怪鸟,而且个头很大,有一个人这么高,可惜没打中,不然就能看看到底是什么。

胖子道:“他娘的邪乎,刚才我在神道那边看到的人,可能就是这东西,人头鸟,可能是种猫头鹰。”

顺子看了看上面,道:“奇怪,那些怪鸟好象不再飞下来了。”

我也看了看头顶,果然,刚才那种无形的压力明显消失了,也没有什么东西再俯冲下来。

“是不是这里有什么蹊跷,它们不敢下来?”

潘子也有点犹豫,胖子道:“这样吧,我先四处去看看,要是这里可能有问题,我们还是马上上去,你们呆在这里,小吴你先休息一下。”

我点点头,潘子说我和你一起去,两个人往两个方向走去。

不多久,一边在搜索的胖子就朝我们打了个呼哨,似乎是发现了什么。

潘子横起枪,朝胖子的方向看去,只见胖子已经顺着桥走出去老远,手电光都模糊了,在他手电的照射范围里,我们看到他的身后有一大片黑色影子,似乎有很多的人站立在远处的黑暗里,黑影交错,连绵了一片,数不清到底有多少。

我们全部都戒备起来,潘子“咔嚓”一声上栓,顺子拔出了猎刀。潘子就对着胖子叫道:“怎么回事?什么东西?”

胖子在那边叫道:“你们过来看看就知道了。”

从刚才我们在桥上的感觉来看,护城河有将近六十多米宽,纵横都非常深远。相比河的绝对宽度,胖子站的地方,其实离我们并不远,但是因为四周浓稠的黑暗,我们根本看不清楚他手电照出来的东西。

不过,听胖子的语气,那里似乎没有危险。

顺子看了看我,问我能不能走,要不要去看看?我点了点头,他扶着我将我拉起来,三个一瘸一拐,就往胖子的呆的地方走去。

护城河底全是高低不平的黑色石头,有些石头的大小十分骇人,看的出原来修凿的时候,肯定是十分巨大的工程,胖子照出来的那一大片交错的黑色影子,正好是位于上边石桥的桥墩下。

艰难的走到胖子的边上,那些影子也清晰起来,我走到近前,从胖子手里接过手电去照,才看清那是些什么东西。

胖子站的地方,河床出现了一个断层,断层之下是一条大概一米深的沟渠,沟渠大概有二十米宽,无数黑色的真人高的古代人俑和马俑,夹杂着青铜的马车残骸排列在沟渠之内,连绵一片,凑近其中几个,可以发现人俑的表面被严重腐蚀,面目模糊,五官都无法分辨,很多人俑还拿着铜器,更是烂的一片绿色斑澜。

这些人俑大部分都是站立着,靠的极密,也有很多已经倒塌碎裂,东倒西歪的堆在一起。从我这里看去,目力加上手电的光线所及的地方,似乎全是这些东西,一大片的黑蒙蒙的影子,在阴森的皇陵底部,看上去如何不让人感觉毛骨悚然。

“这些是什么东西?”顺子第一次见到,看的目瞪口呆。

“这好像是殉葬俑,这些是车马俑,象征的是迎宾的或者帝王出行时候的队伍——”我结巴道。“奇怪,他娘的这里怎么会有这些东西?不是应该放在地下玄宫或者陪葬坑里的吗?”

胖子也知道这茬,也觉得奇怪,这地方是皇陵,不是儿戏的地方,地下玄宫中的东西的数量,陪葬坑中所有殉葬品的摆设,都是有相当的讲究,不像一般皇宫贵胄的陵墓,可以随性而来。皇陵讲究一个气,一个势,这种把殉葬品堆在露天的做法,相当于一块上等白玉上的一块老鼠斑,大忌中的大忌,在当时要是给皇帝看见,肯定是要抄家的。虽然当时东夏是一边陲的隐秘小国,但是既然修建陵墓的鼎鼎大名的汪藏海,肯定不会犯这种低级的错误。

胖子爬下沟渠,一手戒备的端起枪,一手用手电照着一具无头的人俑,对我道:“看服饰是好像是元服,是少数民族的衣服。”说着就想用手去碰。

我提醒他道:“别乱动,这东西神神秘秘的,摆在这里,只不定有什么蹊跷。”

胖子不以为然:“怕个球,难道还能活过来不成?”不过我的话还是有点作用,他把手缩了回来,背起枪,一手拿手电,一手就抽出了腰里的猎刀,用力杵了那人俑几下,人俑毫无反应,他转头道:“货真价实,石头人。”

潘子看着好奇,也爬下了沟渠,走到胖子身边,我看着还是有点不舒服,道:“你们小心点。”

胖子摆了摆手,表示不屑与我交谈。他把猎刀插回皮套里,尝试着抬了抬最近的一座人俑,问道:“小吴,你是干这一行的,这些玩意儿,值钱不值钱?”

我点点头,告诉他:“这东西有点花头,不说整个,就是局部也有人要,我知道一个兵马俑的头就值200万,还是美子,那些马头比人头少,更珍贵,价格就说不好了。”

胖子惋惜的看了一眼四周,露出痛心疾首的表情,道:“可惜可惜,这东西不好带——”

我心里还是感觉到很奇怪,这些东西,实在不应该出现在这里。人说,古墓中每一件东西,背后都是一个故事,这些东西在这里,应该有着什么讲究,或者故事在,那么当时的设计者到底有什么用意呢?

按照两边的距离来看,这些人俑站的沟渠,位于护城河的中央最深的地方,在皇陵刚修建完成的时候,这些东西应该都是沉在护城河的水底,给水面所掩盖,人俑模糊的面部也是它们曾经长期浸没在水中的证据。也就是说,当时皇陵修建完成之后,上面的人,是看不到这些东西的存在的。

那把这些人俑放在这里,有什么意义呢?难道这些是建筑废料,人俑的次品?工匠偷懒把这些垃圾沉到护城河里了?又不像,摆的如此工整,不像是堆放次品的方式。

当真是无法揣测古人的心思啊,我心里感慨,要不是我摔下来,在桥上根本就看不到桥下的东西,也算是机缘巧合,这是不是上天想昭示我什么?

这时候,胖子突然“啧”了一声,说道:“你们有没有发现,这里所有的人俑,都是面朝着一个方向,做着走路的动作,和咱们在市场上看到的很不相同。”

我本来没有注意到,但是胖子一说,我也就顺着他的意思去看,果然是如此。

本来陪葬俑朝一个地方排列,是很平常的事情,从来没见过乱七八糟面向的情形过,但是胖子说的走路的动作,倒是十分的奇特,我从来没见到过。我用手电仔细的照了照人俑的下部分,突然,一股奇怪的感觉涌了上来。

“这些人——”我皱起眉头道:“好像是在行军。”

“行军?”潘子看向我。

我点点头:“从马车个人物的衣饰来看,这是一只帝王出行的队伍,你看这些马,这些车,这些人的动作,他们都在朝同一个地方走,这些人俑这样摆列,他们的动作,似乎是在表示这样一种动态情景。”

我们都朝人俑队列朝向的方向看去,只见这支诡异人俑的长队,延伸到了护城河深处的黑暗中,无法窥知它们的“目的地”是哪里。

\chapter{殉葬渠}

如果没摔蒙了,我可能还想说咱们过去看看,但是看到远处那种深邃的黑暗,这句话就没说出口,胖子没感觉出我的胆怯来,问道:“你说的有点道理,那它们是去哪儿呢?咱们要不去看看,反正这河也不长。”

潘子马上摇头,不同意,道:“咱们耽搁不了时间,小三爷受了伤,要再出点什么事情,跑都不行,咱们还是别把经历花在这里,三爷给我们传的话儿,咱们都还不知道是什么意思,与其节外生枝,不如趁这个时候好好想想,三叔说的地宫入口究竟在什么地方,正巧那些怪鸟似乎也不飞下来。”

这话正合我意,我马上点点头,然后咳嗽了几声,表示自己受伤严重,顺子也不表态,胖子看我们这样,不由有点悻然,耸了耸肩说那算了。

顺子把他们拉上石俑渠,我们又回到了我摔下来的地方,潘子从背包里拿出风灯,点燃了给我们取暖,我一算到这里已经快一天没吃东西了,肚子马上就叫了起来,于是四个人坐下来吃了一点干粮。

翻开我们的行李,我们才发现,我们大部分的食物,竟然都是在陈皮阿四那伙人的包里,我们身上带的食物,明显已经十分不够了,特别是胖子,这一顿下来,他包里基本就没吃的东西了。但是,几乎所有的装备却全部都在我们这里,像绳子,爪钩子,火具等等必须的探险用品。

潘子查看了一下,对我们道:“看样子陈皮阿四在分配我们装备的时候,已经下了功夫了,装备全部都是我们的人背,食物都是他们的人来背,这样两边谁也拉不下谁,谁也不能自个儿跑掉,这一招我还真没注意到。”

胖子嘲笑道:“你他娘的注意到什么了?幸好我也没指望你和你们那个三爷,每次碰到你们,一定做亏本买卖,在火车上我就料到有这一天了。”

潘子呸了一口,道:“你他娘的少说风凉话,你也不是什么好东西,你不给我们闯祸我就阿弥陀佛了。”

顺子怕他们吵起来,道:“几位老板,有力气吵架,不如快点想想你们那个三叔说的那话是什么意思?”

我也拍了潘子一下,让他别动气。问顺子道:“当时三叔来找你,是个什么情况,你要不详细和我们说说,那一句话太笼统了,我们连皇陵都没进呢,真不知道该怎么去想。”

我一问,胖子和潘子也静了下来,一起看向顺子。

顺子坐了下来,皱起来眉头道:“那是大概是一个月以前,当时我也是带客人上山,当然没你们上的这么厉害,就是四周走走,看看雪山。你们三叔当时是混在那些客人当中,后来在山上过夜的时候,他突然就把我叫出去,神神秘秘的,说他现在要自己一个人上雪山去了,让我别给其他任何人说,然后给我点钱,让我大概在这个时间,在山脚下等一个叫吴邪的人。然后带你们进山,只要能把你们带到他面前,就能给我一大笔钱。他就是在那个时候和我说的这一句提示,他很强调的是,只要是‘你’,一听就马上懂。”

“他确实这么说?”我问道。

顺子点了点头,表情很肯定。

我就感觉到有一点奇怪,这话似乎是在强调听的人,而不是话的内容,只要是“我”听了就能马上懂,难道我身上有不同于其他几个人的特质吗?

“那你怎么懂得支开陈皮阿四之后才告诉我们这些东西?”胖子问。

顺子嘿嘿一笑,露出了与以前截然不同的一种表情,道:“我也不是傻子,你三叔告诉过我你们的人数,说如果人数不对,就只能把话传给你一个人听。我一看到你们,当时就感觉到你们这一队人气氛有问题,似乎有两股不同的人混在一起,当时我又不知道你们是干什么的,只好先装傻看看。到底我收了别人的钱了,万一弄的不好,耽误了你们的事情就不好。”

我看着顺子的表情,就感觉到一种狡狯,心中就一个疙瘩,心说原来从上山开始,他的那种憨厚都是装的?那乖乖,真是人不可貌相,难怪越走到后来,这小子就越镇定,原来是露出本来面目来了。

潘子是老江湖了,这时候就沉下了脸,道:“没这么简单吧,我看你好像还知道什么?”

顺子幽幽的一笑:“我退役前是在这里当兵的,雪山我走的多了,我的父母是土生土长的鲜族人,718动乱的时候从北朝鲜逃到这里来的,在山里躲了好几年。这山里,古时候的传说多了,我们碰到的怪事情也多了,每年怀着各种奇怪目的进山的人数不胜数,你要说我什么都不知道,我总归是知道一些东西的,所以我一看你们往这山头走,就猜出你们想干什么了。”他顿了顿,意味深长的看了我一眼,“要不是有你们三叔的嘱咐,在山腰雪崩的那个地方,我就绝对不会让你们再往前走了。”

潘子看了看我,又看了看胖子,一下子也讲不出话了。

呆了半饷,潘子拿出一只烟,递过去,道:“顺哥,有眼不识泰山了,那咱们现在是自己人,来,抽一根。”

顺子没接那烟,抬头道:“我是个实在人,别说废话,我帮你们不是喜欢你们,我是求财。你们那个三叔,答应给我的数目,够我用两辈子了,所以我怎么样也得把你们带到他面前,你们还是快点想那句话是什么意思。”

潘子给他弄的很尴尬,只好把烟叼到自己嘴巴里,苦笑着看了看我。

我问顺子道:“那你把三叔当时的原话,重复一遍给我听听。”

顺子回忆了一下,道:“当时他似乎是这么说的:‘等吴邪到了,你告诉他,地宫的入口在玄武拒尸之地’,然后我就问他那是什么意思,他说只要这么说,如果是你,就肯定能知道了。”

“还是同样。”我叹了一口气,心说,整句话听下来,关键还是“我”,但是这句话我明明是一点也听不懂,三叔他娘的到底哪里来的这种对我的信心,这不是坑我吗?

几个人都看向我,眼里露出殷切的表情,我摇了摇头,直叹大气。

胖子看我想不通,问道:“会不会是这样,这个提示和你们以前自己家里发生的事情有关系?所有只有你们吴家的人才知道?”

“不能这么说。”我道:“我了解三叔的个性,他不是那种讲一个超级复杂的暗号,然后让我们来猜的人,他既然是让顺子传话,那这句话绝对是意思非常明确,肯定是哪里岔了,我们想错了。”

“不过三爷既然说,是‘你’一听就能知道,而不是‘我们’一听就能知道,那肯定是一个关乎你们之间共同点的暗号。”潘子道:“不如想想你们之间有什么共同点就好了。”

我感觉这也不太靠谱,不过此时也没有别的办法,就摆开手指头琢磨起来。

我和三叔的共同点,其实也不太多,而且还必须是我和三叔的,潘子他们如果也是就得排除,比如说大家都是男人,潘子也是男人,那就不算了,算起来,我们都姓吴,应该算一个,但是这和那暗号应该没关系吧。

还有就是,我和他看到女人都有点不着调,不过这也比较模糊,他自己是打死都不承认,除了这些,要说能算共同点的,就是我和他都住在杭州,现在主要的生活地盘是杭州。

等等!我想到这一点的时候,突然人就打了一个激灵,好像脑子里出现了什么东西,好象脑子里出现了什么东西,一丝灵感突然就出现在了我的脑海里——

“玄武拒尸”——三叔的暗示——杭州——“我”一定能听懂——

我突然恍然大悟,这四个字,竟然是这个意思!

\chapter{无聊暗号}

“玄武拒尸”。狗屁的“玄武拒尸”!

我想通了之后,一切都豁然开朗,不由得笑起来,这完全是一个误会,三叔说的四个字,根本就不是这四个字,因为我们对于葬经的先入为主的概念,一听到发音相近的四个字,就把它对号入座了,而且正如我预料的,这个暗号其实根本就不是暗号,三叔用了一个非常巧妙的办法,使得他这一句几乎是直白的话,可以在别人面前传达,但是真实的意思却只有我能知道。

看来三叔早就想到了,可能与我一起来到的这皇陵之中的,不一定都是他安排的人。

几个人看我的脸色剧烈变化,马上就知道了我已经有所醒悟,忙问我想到了什么。

我解释道:“我们真的想错了,三叔说这句话‘我’能听懂,最重要的原因是不是我和他的共同点,而是因为,我是一个从小在杭州长大的人。”

几个人还是不明白,胖子问:“这么说,这话和杭州的风景有关系?不会啊,你胖爷我去过杭州啊,没听过有叫‘玄武拒尸’的景点啊?”

潘子摇头,道:“你扯哪儿去了,肯定和风景没关系,从小在杭州长大的人,也不一定熟悉杭州的名胜古迹,你看我们家三爷,在杭州也定居快十年了,他就知道个西湖,上次带我们去宝石山上喝茶,还给我们带迷路了呢,最后走到天黑一看,到玉泉了。”

我点点头,确实,我也是这样的人,谁说做古董的就得喜欢古迹,我也没走过多少景点。

胖子皱起眉头,对我道:“和风景也没关系?那他娘的是什么,你还是直接说吧,我都急死了我。”说着就擦汗。

我也不想卖关子,对他道:“这很简单,在杭州长大的,虽然不一定熟悉风景,但是,绝对——能听的懂杭州土话,这一点才是关键。”

几个人都一愣,呆了好久,显然有一些感觉了,还是不了解。胖子问道:“是发音?”

我点点头,在这里几个人中,只有我是精通杭土话的,潘子常年在长沙,杭州话能说能听懂点,但是你要说到深处去,就不行了,胖子京片子,一听就知道常年混在北京城,顺子就更不用说了,普通话都说不利落,如果三叔用杭州话说一句,确实只有我能听懂。

可惜的是,顺子因为汉语不好,只记得了发音,没听出前面的话和后面的语调变化了,所以用他那嘴巴念出来就成了一句完全不着调的话。

潘子拍了拍自己的脑袋,说:“我操,这我还真想不到,那‘玄武拒尸’,用杭土话来念,是什么意思?这好像也难念啊。”

我笑道:“听我来分析就行了,其实三叔的暗语不是四个字,而是‘玄武拒尸之地’,这六个字,第一个字‘玄’,杭州话的发音同‘圆’,又相似于‘沿’,‘武’的发音,和‘湖’的发音是一样的,但是在杭州,‘湖’这个发音,即可以说是湖,又可以说是河,‘拒’和‘渠’,发音是一样的,‘尸’和‘水’同音,‘之’和‘至’同音,‘地’和‘底’同音,连起来就是——沿河渠水至底!”

我一解释完,几个人“啊”了一声,都露出了恍然大悟的神色,胖子点了点头,显然我这样的翻译,十分合理,没有什么破绽。

潘子“啧”道,喃喃道,三爷就是三爷,这句话要是陈皮阿四听见,他打死都想不到是这个意思,肯定磕破脑子去琢磨“玄武拒尸”的意思。

“河渠水?”半饷,胖子就道,“可是。这里没有河渠啊?皇陵中会有河吗?”

我道:“陵墓中肯定没有,陵墓中可以有泉,但是应该不能有河,因为河的水位不受控制,水太高了会淹,水太小就会破势,而且河水会暴露古墓的位置。这里说的河渠,可能就是指这条护城河。”

潘子脸上的肉都激动的抖了起来,道:“那咱们是误打误撞,还走对了路了?”

“也不好说。”我摇头,毕竟没进过皇陵,不知道里面的情况,不过按照现在的迹象和以前看过的一切资料推断,我的分析还是有道理的。

“如果说河就是护城河,那渠,他娘的该不会就是我们刚才看到那条——”胖子站起来,看向一边那条全是石俑的殉葬沟,那简直就是贴合三叔的暗号出现的,我们有都转过头去,心跳加速起来。

“不过,”潘子有点不确定,“那渠里没水。”

我摇头,道:“三叔当时还没进这个皇陵,他说的这句话应该也只是他从其他什么地方得到的提示,有可能是什么古籍或者地图,而当时制作这种地图或者古籍的人,大概也想不到,有朝一日,护城河里会一点水也没有。”

这里河壁堆砌的岩石上有着给腐蚀的痕迹,这条河里原来肯定也有水,但是经过千年的岁月,引入河水的源头,或是地下河,或者温泉,可能干涸了,河水得不到补充就逐渐渗入地下,最后一点也没剩下。

胖子沉不住气了,“咔嚓”一声拉上枪栓,对我们歪了歪脖子:“同志们,难得咱们的个人利益和革命利益高度统一了,还等什么,他娘的一起上吧。”

这一次胖子的提议,我们都找不出理由来反驳。但是马上出发,他显然太过猴急了,潘子把他拉下来,道:“既然有眉目了,现在倒是不急,你看看小三爷受这么重的伤,还没缓过劲来,你是想一个人去,还是让我们把他扔在这里等死?”

胖子呆了呆,想想也是在理,但是他实在欲火焚身,拍了顺子,道:“那咱哥两儿先去探探,勤鸟吃头菜,让他们两在这里歇着,保证拿到的不比那个老三爷给你的少。”

谁知道顺子也摇头,道:“老板,我的任务是把他,”指了指我:“带到你们那个三爷面前,之后你们的死活都不管我的事情,但是现在我得看着他。”

我听了嘿嘿笑,对胖子道:“现在知道这里谁是大人物了吧?”

胖子呸了一声,不爽道:“得,你们都在这里休息,胖爷我自己去,等我摸几只宝贝回来,看你们眼红不眼红。我丑话说在前面,摸到就是我的,可不带分的,你们谁也没份!”说着端起枪就走。

可走了几步,他突然停住了,顿了顿,转头又走了回来了,我们几个都哈哈大笑,问他干什么,又不敢了?

胖子哼着气,一脚踢开自己的背包,坐到风灯对面,道:“什么不敢,你们还真想我去了,胖爷我没这么笨,等一下我东西摸出来,你们三个人上来抢,我猛虎难敌群狼啊,给你占便宜,直不定还给你们谋财害命,我才不干这缺心眼的买卖呢。”

潘子看胖子一直不爽,这时候乘机奚落道:“你这叫小人之心,你以为我们都跟你似的。”

我怕他较了真了,打断他们道:“行了,都别说了,现在算起来也该半夜了,虽然这里看不到天,但是我们也得抓紧时间休息。”

潘子看了看表,就点了点头,把风灯调大,一下子四周暖和起来,然后扯出充气的睡袋,吹了气,几个人都睡了进去。

胖子点起一只烟,说自己睡不着,他来守第一班。我看了他一眼,对他说千万可别半夜自己摸出去找东西,进了玄宫随便你拿,这里就消停掉,你他娘的别给我看扁了。

胖子大怒说自己是这样的人吗?他守夜,保证我们安全。

路途疲倦,算起来上到雪顶已经是傍晚,进的冰盖中的宫殿,一路过来,已经快用了10个小时,相当于强体力劳动一天一夜,其中包括攀岩、狂奔、跳远,以及跳远失败摔楼,我想着都累,一进睡袋,很快就睡着了。

一觉睡的很香,因为我是伤员,没让我守夜,我醒过来的时候,四周还是一片漆黑,风灯暗了很多,守夜的人已经换了潘子,他正靠在石头上在抽烟,一边胖子的呼噜打的象雷一样。

我看了看表,也只有睡了五个小时,不过大伤的时候,睡眠质量一般都非常好,因为身体强烈的修补,人基本都处于半昏迷状态了,但是醒过来脑子是清爽的,身体却更累,腰酸背疼的厉害。

我揉了揉脸爬出睡袋,一边活动手脚,一边让潘子去睡一会,说我来守会儿,潘子说不用,在越南习惯了,不在床上,一天都睡不了三个钟头。

我也不去理他,坐到另一边的石头上,也要了一支烟抽,吸着醒脑子。

两个人沉默了一会儿,突然潘子就问我,能不能估计出三叔现在怎么样了?会不会有什么事儿?

我看他表情,是真的关切和担心,心里有一丝感慨。按照道理,潘子这种战场上下来的人,看惯了枪林弹雨,生离死别,不应该有这么深沉的感情,但是事实上,潘子会对于这个老头子的忠心和信任,让我这样的亲侄子都感觉到惭愧,也不知道潘子和三叔以前发生过什么,有机会真的要问问他。

我安慰他道:“你放心吧,那只老狐狸绝对不会亏待自己的,他这种人命硬,要是出事,也不会等到现在才出事了,咱们现在只要顾好自己就行了,现阶段,让别人担心的应该是我们,因为我们还什么都不知道。”

潘子点了点头,叹了口气,有点懊恼道:“可惜我脑子不行,三爷做的事情,我总搞不懂,不然这种危险的事情,也不用他亲自去做,我去就行了。”

我心中苦笑,心说三叔做的事情也不见得非常危险,我反而感觉最危险的是我们,老是跟在三叔后面猜三叔的意思,然后被他牵着鼻子走,这样下去,运气再好也有中招的时候。

就比如这一次,从三叔可以提前给我们地下玄宫入口的线索来看,似乎他身上有什么东西,让他预先知道了这里地宫的结构,“沿河渠水至底”这是一句文言文,三叔讲话不是这种腔调的,这句话肯定是来自古籍。而顺子所说的,三叔他是一个人进入雪山来看,显然他并没有落在阿宁他们手里,如果他顺利进入了这个火山口,那他很可能已经在皇陵的地下玄宫之中了。

可以推测的是,这那让他预先知道地宫结构的“东西”,应该就是他前几个月去西沙的目的,也可以解释为什么阿宁的公司竟然会在这里出现,他们的目标应该也不是海底墓穴,而是这里的云顶天宫,和三叔合作去西沙,只不过是在海底墓穴中寻找这座长白山地下皇陵的线索。

而阿宁在海底古墓中,和我们分开过很长的时间,在我们疲于奔命,给那些机关陷阱弄的抓狂的时候,这个女人在后殿中干了什么?是不是也和三叔一样,拿到了通往这里地下皇陵的关键?这个我们就不得而知了,不过刚才在前殿看到的装备精良的尸体,证明阿宁的队伍已经先我们到达了这里。根据顺子所说的,他们这么庞大的队伍是无论如何也通不过边防线的,可是他们却毫发未伤的过来了,表明他们必然知道一条谁也不知道隐秘道路。

这至少可以证明,阿宁他们也知道我们不知道的事情。

这就是我们和他们的绝对差异了,我们是完全的“无知”,地下玄宫之中有什么等着我们,我们根本无法估计,这其实是最糟糕的处境了,然而我们还必须继续前进,不能选择后退,这是糟糕之中的糟糕。

这些我都没有和他们说,因为对于潘子来说,三叔就是一切,三叔要他做的事情他就必须去做,不用管动机。对于顺子来讲,他完全是局外人,这就是一比买卖,他只关心最后的结果。而胖子就更简单,他是为了“夹喇嘛”而来的,陵墓中的东西才是关键,我们的三叔,对于他来讲只是一个麻烦的代名词而已。这些分析的出来的东西,似乎只对我自己有用,只有一个人是在扑朔迷离之中的。

其他人都活的如此简单,第一次让我感觉到有点羡慕。

又聊了一回儿其他的,潘子就问我身体行不行,我感觉了一下,经过睡眠,我的身体已经好转了很多,此时不用人搀扶应该也能够勉强走动,只是显然,打架还是不行的。潘子说还是再休息一下的好,难得这里这么安静,似乎也很安全,恐怕进了地宫之后,就再没这种机会了。

我一想也是,就想再进睡袋睡个回笼觉,然而却睡不着了,一边的胖子不停的用一种我听不懂的方言说梦话,似乎是在和别人讨价还价,在他说的最激动的时候,潘子就拿石头丢他,一中石头,胖子马上就老实了,但是等一会儿又会开始,十分吵人,我疲倦的时候完全听不到这些,但是现在要入睡,就给这搞的够呛。

闭着眼睛,又硬挨了两个小时,潘子一块石头挑的太大,把胖子砸的醒了过来,这一下子谁也别想睡了,顺子也给吵醒了。

整理好东西,又随便吃了一点干粮,我们重新走回到刚才看到的殉葬渠处,糜烂的黑色石头人俑还是无声的矗立在那里,长长的队列,一直衍伸至两边的无尽的黑暗之内。

我给搀扶着爬下殉葬渠,一下子就走入了人俑之中,在上面是俯视着人俑,所以感觉并不是很强烈的,但是一到下面,人俑就变得和我一般高,四周的错错黑影,让一股强烈的不安从里我心里产生了。

胖子用手电照了照两边的方向,问我道:“你们的三爷让我们跟着水走,但是这里现在没水了,咱们该往哪里?”

我看向潘子,他在对越自卫反击战的时候,参加过特种战争阶段,应该对这种东西有点研究。

潘子走近一座人俑,摸了摸上面的裂缝,指了指人俑朝向的方向,“看石头上水流的痕迹,那边应该是下游。”

胖子凑过去,却看不出什么所以然来,不信任道:“人命关天,你可别胡说。”

潘子不去理他,说着招呼我们小心点,几个人开始顺着沟渠,向护城河的黑暗处走去。

护城河的长度,我一点概念也没有,在悬崖上用照明弹看的时候,整个皇城是一个远景,我们大概只看到建筑物的顶部,护城河给四周茂密的死树林遮挡着。而在上面桥的时候,手电的光芒又不足以照出黑暗中的全部。所以沿着殉葬渠直走了有半个小时,万般寂静的护城河底,却还是没有到头。

殉葬渠高底不平,有几段,里面的人俑碎裂的十分严重,似乎给什么巨大的东西踩过,那种坚硬的不知名的石料,都裂的粉碎,我甚至发现在沟渠的底下,不时还有人俑的头颅的四肢出现,似乎殉葬渠底下的土里,还埋着一层这样的东西。

或者可以这么想,这条沟渠是不是原本是要被埋藏的,但是因为某种原因,工程停顿了,所以还有这么多的人俑没有掩埋。

越走越黑,本来手电照在一边的河壁上,还有一点反光,至少还有参照物,走着走着,就连一边高耸的河壁都找不到了,四面都是黑咕隆东的,我们不由放慢了脚步,潘子提醒我们机灵一点,千万不要分神。

这个时候,走在最前面的胖子停了下来,我们正要上前,看到他做了一个让我们停下的手势。

我走到他的身边,顺着他的手电看去,只见殉葬渠的尽头已经到了,人俑的队伍消失了,面前是一块巨大的石头河壁,应该是到了护城河的另一面了,河壁上似乎有雕刻着一个乐山大佛一样的巨大的东西,因为手电根本照不出全貌,也不知道是什么,只看到河壁的根底下,有一道被碎石掩盖的方洞,现在石头已经给搬开了不少,露出了一个黑漆漆的洞口。

这和刚才我们进来的排道一样,这个洞也是当年修陵的工匠们偷偷挖掘的通道之一,这是他们在地宫封闭之后逃出的唯一通道。

“又是一个反打的坑道?”潘子惊讶道:“开口怎么会在这里?这不可能啊。”

“怎么会不可能?”胖子问。“又不是你修的。”

潘子道:“这里当年是在水下,你以为那些工匠全是鱼吗?”

我摆了摆手让他们别吵,这时候顺子“嘿”了一声,说道:“过来看,这里有东西。”

说着用手电照过去,我们一看,只见方洞一边的石头上,有人刻了几个字。

\chapter{水下的排道}

方洞有半人高,四方形,打的非常粗糙,边上全是大概西瓜大小的碎石头,里面也有不少,显然有人曾经把这个洞堵上过,而方洞内黑漆漆一片,不知道通向哪里,有点像我们在南方经常看见的水库涵洞。

在方洞一边的碎石头堆里,有一块比较平整的,上面很粗劣的刻了几个字,是非常仓促刻上去的,刻的非常浅,要不是那几个字是英文字母,在这种皇陵里面看着非常刺眼,顺子还不一定能发现。可惜刻的什么,根本无法拼出来。

是三叔刻上去给我们认路的吗?我当时就这么想,但是三叔的洋文很不靠谱,他这种脑子怎么会想出来刻洋文当暗号,这实在不是他的风格。

胖子好奇走近去看一看,突然就咦了一声,招手招呼我道:“小吴,这几个扭曲曲的洋文,咱们好像在哪里见过。”

我也走过去,才看了一眼,心里就不由一跳。

不是好像,这几个符号我们的确见过,这是我和胖子在海底墓穴之中,下到碑池之中的时候,胖子在池壁上看到的。看到这个符号之后,闷油瓶突然就冲下那个碑池,之后他就想起了海底墓穴中发生的事情。怎么突然又出现了在了这里?

当时,我一直以为这符号是当年三叔带文锦他们下来的时候,那几个人中的人刻上去的,但是突然又在这里出现,显然就不对了。

看雕刻的痕迹,是用登山镐胡乱敲的,而且痕迹如此新,那要不就是三叔留下的,要不就是闷油瓶子或者阿宁留下的,因为这里也就这几个人能有登山镐,留这个符号的人,肯定也已经进到方洞里去了。

此时我突然有了一个念头,心说会不会,海底墓穴中的那个洋文符号,是闷油瓶刻下的,所以他看到这个符号之后,才会知到道:“这个地方我来过。”

还真是有这个可能,他再出现的时候,我得问问。

潘子看我发呆,问我怎么回事情,我把我和胖子在海底看到符号的事情和他们一说,潘子也感觉到很新奇。不过他道:“我跟三爷十年了,往少了说也倒了不下五十个墓,其中大的也有几个,没见过他留过暗号,而且三爷abcd都认不全,这肯定不是三爷留下的。”

我心说那就是阿宁或者闷油瓶了,转头对他们说:“不管怎么说,看样子路没错,这洞已经有人进去过了,地宫的入口应该就在这下面,咱们是不是马上进去?”

“进!”胖子马上道:“还等什么?几番人马都在我们前头,胖爷我向来都是打先锋的,碰上你们几个倒霉孩子才混的给人殿后,咱们就别磨蹭了,等会儿人家都办完事出来了,咱们都没脸跟他们抢。”

潘子对我道:“你别问我们,你身体行不行?”

我点头表示没问题,“胖子说的对,咱们不能拖了。反正碰到粽子,我就是没受伤也是死,现在受伤了,也就死的快一点而已,不怕。”

胖子一边已经卸下自己的背包,听我这么说,“啧”了一声:“你他娘的就不会说点吉利的事情?也不看看咱们现在要去什么地方?”

我瞪了他一眼道:“有你在脑门上贴两个门神都没用,你先管好你那手。”

我们各自准备自己的装备,刚才我们是行军的打包方式,现在我们把风灯,燃料这些东西全部放进包里,然后把冷烟火,冷光棒,炸药全部拿出来,系在武装带上,胖子和潘子各自拉开枪栓,退下子弹匣子,把子弹带上的子弹退下来装枪,上满弹药后,猎刀匕首都归位。

五四枪太长,在方洞之中可能无法转身,于是胖子把枪给了顺子,自己拿出登山镐子,几个人测试了一下手电的光度,胖子拿出自己的摸金符,捏在手里朝天拜了拜。

顺子也是用枪的行家,拿过枪,“咔嚓”几下熟悉了一下,大有怀念之感,然后对我们道:“几位老板,我不懂你们这行,不过我要提醒一句,在长白山上钻洞,要小心雪毛子,如果看到苗头不对,先用棉花塞自己的耳朵,这东西现在这个季节脑壳还没硬,只能钻耳朵,等到了夏天,壳硬了之后,能直接从你皮里钻进去,就露出两根后须,你一扯后须就断,整只虫子就断在里面了,你得挖开伤口才能挖出来,还有,这东西也钻肛门,坐的时候千万小心。”

胖子厌恶的看了一眼顺子,下意思的勒紧了皮带,道:“现在虫子也有这嗜好了?”

顺子道:“我不和你们开玩笑,中招了自己想办法拉,别来问我。”

我们感到下半身发凉,都点了点头,胖子当下一马当先,探身爬进了方洞之中,我们进跟其后,鱼贯进入,向着地下终极的未知世界开始前进。

方洞之中必须猫着腰走,洞是平行挖掘的没,边走边看四周的情况。因为高度太低,走的很慢,这里的岩底非常结实,看敲凿的痕迹,这条坑道,显然用了最原始的办法挖掘。我猜想修这么大规模的皇陵用了多少时间?怎么样也要二十多年吧,很多皇帝在登基的时候就开始着手修坟墓了,二十多年,挖掘这条坑道也是十分的勉强,看样子当年外逃的人应该是很大规模的一批人。

越往里走,越看到很多人到过的痕迹,登山鞋子的鞋印就不止一处,没有出现雪毛子,不过,我却发现在坑道的顶上,有一些奇怪的岔洞。

这些洞都不大,只能够容纳一个人,而且洞是180度弯曲的,笔直向上一段后,就会向下大转弯,形成犹如数字“9”形状的弯曲管道。这样的洞,每隔十米,大概就有一个。

自从涉足这一行以来,爬洞不知道爬了几次,还从来没见过这样的结构,从建筑核算学的角度来说,打这些洞的工程量几乎和打整条坑道一样多,那这些洞必然有不得不打的绝对理由,不然就是不经济的,可是又实在看不出这些洞有什么存在的价值。

潘子在后面对我说:“小三爷,你有没有发现,这条坑道有点眼熟?”

“眼熟?”我顿了顿,转头问他为什么这么问?

潘子道:“咱们在山东瓜子庙的时候,过的那尸洞,进洞的隧道,不是也是这个德性的,那老头子不就是躲到上面的洞里来害咱们几个?”

他这么说,我又仔细看了看洞的顶上,在山东的那时候,我慌都慌死了,并没有太过注意那尸洞水盗洞的头顶,现在也无法比较。不过潘子既然这么说,那就应该不会有错,也心生奇怪,问他道:“你确定?”

潘子倒也不确定,说:“我们也是听了那老头的话才知道上面有洞,自己过的时候一片漆黑,并没有发觉。”

我停了下来,仔细看了看这些岔洞,马上就明白了它的作用,道:“当时那个尸洞也是个水盗洞吧?”

潘子点头说是,我道:“这些岔洞其实是用来呼吸的,你看,水灌入这条排道的时候,因为岔洞的弯曲结构,会在岔洞中留有空气,这样只要游一段,然后头探入岔洞中呼吸一口,再继续前进就可以了。”

潘子一下惊讶道:“这么巧妙的办法,这么说,当年这一条排道,的确是在水下的?”

我道:“差不离吧,看样子,瓜子庙的那一道水盗洞,说不定也是汪藏海的人挖的。”想想又不对,那条盗洞之古老,三叔推断是在战国时期,可能是鲁殇王进山修陵的时候挖的,难道是汪藏海去了之后看到,借鉴了古人的技术?倒也十分有可能。

走了很长时间,也不知道走了多少距离了,排道逐渐变宽,终于看到了出口。我们爬了出去,面前竟然是一跳极深的河渠,大概十几米深,五六米宽,河渠中已经没有了水。

我看了看河渠修凿的情况,道:“这是引水渠,护城河的水从这里引出去,保持水是活水,不会发臭,而且防止了水位的倒灌。”

河渠两边都有供一人行走的河埂,上头还架着一座石桥。我们小心翼翼的走过去,来到河的另一岸,胖子问现在怎么走?

我道:“这条渠和外面的渠是相通的,应该算一条渠,我们跟着水走。”

潘子蹲下去看了看水流向的痕迹,指了指一边,“那里。”

我们继续往前,不多久,前方的河埂边上的石壁上,出现了一个四方形非常规则的方洞。

胖子打起冷烟火,丢了出去,照出了方洞外面地面上黑色的石板,显然这是地宫的封墙石。胖子钻了出去,连续打起很多冷烟火扔到四周,接着给我们打招呼,我们才从坑道中爬了出来。

出来的地方是一间黑色岩石修建的墓室,不高,人勉强能站直,但是很宽阔,墓室的四周整齐的摆放着很多的瓦罐,可能是用来殉葬的酒罐,每一只都有半人高。粗略估计有一千多罐,看样子万奴皇帝可能是个酒鬼。

四面黑色的墙上,有一些简单的浮雕,雕刻着皇帝设宴时候的情形,浮雕保存的并不好,可能和这里于外界相通有关系,这里的火山气体虽然没潘子说的那么致命,但是腐蚀性肯定比一般的空气强,这里的壁画能保存下来,已经是一个奇迹了。可惜保存下来的那些画面只能看一个大概。

在墓室的左右两面墙上,各有一道石头闸,后面是黑漆漆的甬道。一股阴冷的风从里面吹出来,胖子捡起两只冷烟火,一边扔进去一只,都没看到头。

\chapter{猴头烧}

潘子看我脸色不对,让我休息一下,我实在有点吃不消了,就坐到酒缸上喘气,其他人重新收拾了一下装备,顺子从来没进过这种地方,捡起一只冷烟火,就四处好奇的看。说道:“还真是不来不知道,这长白山里竟然还埋着这样的地方。这次算是长眼了。”

“再走下去还有你没见过的呢。”潘子在一边道:“我估计当年大金国掠夺南北宋得来的这些东西,和南宋岁供的宝贝,要不就是落在成吉思汗的手里,要不,就肯定在这个地方。”

“别想的太美。”胖子道:“当年南宋进贡的大部分都是绫罗绸缎,这种东西不经放,又不好出手,我看就算有也烂的差不多了。咱们别老是惦记地宫里的东西,还是多考虑考虑眼前的利益比较好。”说着就去研究那些酒缸,想去搬动一罐,看看罐底写着些什么。

我对他道:“这种缸子太糙了,你别折腾了,送给别人卖羊杂碎腌菜别人都不要。”

胖子道:“谁说我惦记这缸了,别以为你胖爷爷我只好明器。”他用匕首敲开一罐酒的封泥,顿时一股奇特的味道就飘了出来,说香不香,说臭又不臭,闻多了还挺过瘾,也不知道是什么酒。

古墓藏酒,我在大量的典籍中都看过,但是亲眼见到还是第一次,这时候也好奇起来,就凑过去看。

酒是黑色的,很纯,里面的水份已经基本上没了,只剩下半缸,懂酒的人都知道这就是陈年酒的特征,这半缸就是酒的精华所在,实在是诱人,不过再怎么说,这东西也放了太久了,不知道当年的保质期是多少。

我记得中国最古老的酒是1980年在河南商代后期古墓出土的酒,现存故宫博物院,大概有3000多年的历史了,听说开灌之后酒香立马就熏倒了好几个人,也不知道这帮人当时有没有喝过,不然也有个借鉴。

胖子用刀蘸了一点,想尝一口,我拉住他:“你不要命了,过期食品,小心食物中毒。”

胖子道:“你不懂,窖藏酒放几千年都不会坏的,千年陈酒下面的酒漕吃了听说还能长生不老呢,咱们老祖宗倒斗,有的还就为那酒去的,尝尝味道不会有事的,最多拉个肚子。”

还没说完,潘子过来,“当”一脚就把那酒潭子踢翻了,黑色的酒液和罐子底下的酒漕子全撒了一地。一股浓郁的奇香顿时扑鼻而来。胖子刚想大怒,潘子对他道:“先别发火,你看看那酒漕里面是什么?”

我和胖子转头一看,只见黑色犹如泥浆的酒槽里面,有很多暗红色的絮状物,犹如劣质的棉被的碎片,这种东西我们在浸水的棺材里经常看到。

胖子用匕首拨弄了一下,脸色就变了,我凑过去一看,顿时头皮就一麻,感觉一阵剧烈的恶心,几乎就吐了出来。

那些红色的絮状物,是一具还未完全泡烂的婴儿的尸体,肉已经完全融解于酒中了,但是皮和骨头都在,所以形成破棉絮状的一团。

潘子看着目瞪口呆的我们,蹲下道:“这种酒叫做‘猴头烧’,这不是人,这是未足月的猴子,是广西那边的酒,可能是女真的大金还鼎盛的时候,南宋进贡的窖藏酒。”说着拍了拍胖子,用匕首挑起那团“棉絮”,做了一个请用的手势:“能不能长生不老我不知道,不过听说壮阳的功效不错,你别客气了。”

胖子恶心的用刀拍掉,骂了声娘,问潘子道:“你小子怎么知道的怎么清楚?你他娘的喝过这酒?”

“我在山西的南宫见过这种瓦罐,当时大奎和我们另一个伙计取了一罐出来,我始终是认为不妥当的,就没碰,但是他们不在乎,结果喝到见了底才发现下面的东西,后来为这事情大奎在医院躺了两个月。”说起大奎,潘子又有些感慨:“我对你们实在算不错了,要是有心害你,我等你舔上一口再踢翻罐子,有你好看的。”

胖子脸上直抽动,想发作又没借口,样子非常好笑。

此时冷烟火都陆续灭了,黑暗袭来,我们重新开启手电,四周的气氛一下子压抑起来。

休息了片刻,重新开路,胖子要回他的宝贝步枪,又拉枪上栓,这其实是有枪的人给自己的壮胆的习惯动作。他看了看两边两条墓道,小声问道:“往那边走?”

我们都定了定,这时候顺子指了指左边,“这边比较稳妥一点。”

一般这种情况都是潘子和我回答,现在顺子鱼肉冒出来一句,胖子莫名其妙,“为什么?”

顺子用手电照了左边甬道口子的地面,我们看到,在甬道的一边一个很隐秘的地方,又刻着一个洋文的符号。“我刚才偶然看到的,我想这是有人在为你们引路。”他对我们道。

\chapter{记号}

我蹲下身子来,再一次试图辨认这几个奇怪的洋文符号,但是同样无果,线条过于凌乱,虽然能够看出和我们刚才在方洞口看到的是同一个词语,但是到底是哪几个字母组成的,无法拆解,我甚至怀疑起这到底是不是英文。

胖子也很好奇:“你确定这不是你们那个三爷留下的?”

潘子点头,表示绝对肯定,“三爷没这么花哨,他要留记号,一般就是敲出个崁就行了。这肯定不是三爷留下的,我觉得小心点好,记号不一定全是用来引路的。”

我明白的他意思,如果这记号不是引路的,那就可能是一种危险的警告。

不过我在海底墓穴里看到那符号之后并没有发生什么危险的事情,而且甬道就两条,不是走这一条就是那一条,两条都没把握,随便选哪条都一样,此时犹豫似乎没什么意义。

还是胖子在前面带头,我跟在胖子后面,走进甬道。

里面非常宽,足可以并排开两辆解放卡车,胖子一进去,就说里这是条骡道,就是施工的时候走骡车的道,这确实有可能,因为我从来没见过这么宽阔的墓道。地面上还隐约可以看到当年的车辙痕迹,但是离奇是,甬道里面竟然很冷,温度不知道降了多少度,而且还有冷风从里面吹过来,似乎是通着外面,我们都知道无论什么古墓都很讲究密封性,这风从哪里吹来的?

“这是自来风,”潘子给气氛感染,压低声音对我说:“咱们老祖宗说这叫鬼喘气,在大墓里经常有这种事情,不过没什么危险。”

“有解释吗?怎么产生的?”我问道。

潘子摇头,“传下来大多数只有个说法,没人去研究过,而且这事情最好也别去研究。”

我心说也是,在那个时代,盗墓都是为了温饱,只要知道危险不危险就行了,各种奇怪的现象到底是怎么产生的,实在无暇顾及。

甬道刚开始的一段还算平整,到后来就开始发现坍塌和地面碎裂的情况,很多黑色的石板都从地上翘了起来,使得地面高低起伏,这是地壳运动造成的自然破坏。甬道的两边每隔一段距离都有一种加固的拱梁,上面都雕着单龙盘柱,很多都开裂了,我想如果没有这个加固的措施,这条甬道早就塌了。

一路无话,几个人安静的走了七八十米,胖子突然停了下来,在前面道:“门?”

我们都停了下来,手电照向前面,只见甬道的尽头,出现了一道黑色的石头墓门,门上飞檐和瓦当上都雕刻着云龙、草龙和双狮戏球的图案,门卷好像是金属的,左门上雕刻着一只羊,右门上雕刻着另一只不知名的东西。走近一看,石门关的紧紧的,门缝和门栓的地方都用铜浆封死了,但是左边的门上,羊的肚子上,给人炸开了一个脸盆大的破洞,冷风就是从这里面吹出来的。

“这不是门。”我推了推:“打不开的就不是门,这是封石,是用大块的黑石头垒砌,然后用铜水封死冻结成一个整体,做成门的样子,胖子说的没错,这条甬道是骡道,修的这么宽,是为了便于骡子拖动这些石头。”

胖子蹲下来看了看墓门上的破洞:“墓道里有封石,看样子这条墓道应该挺重要,能通到地宫的中心,路算是没错,那标记看来真的是给我们引路的。而且洞都开好了,他们已经进去了。”说着探入半个头,把手电伸进去,照里面的情形。

我们问他怎么样,里面有什么东西?

他说:“还是墓道,里面还有一道封石,看样子万奴皇帝从小缺少安全感。”

我说:“扯蛋,你家的门还三保险呢,封石最少也有三块,三千世界,你懂吗。”

胖子没听到我说什么,他把手电往里面一放,缩身窜进了门上的洞里,到了封石的对面。我听到他打了个磕巴,自言自语道:“我操,好冷。”

潘子把枪给他递进去,跟着他也爬进去,我跟在后面,顺子殿后,都爬进了洞里,果然后面还是墓道,温度比另一面更低,人马上就有浑身发紧的感觉。正前面还是一道封石,不过这一道就比较简陋,没有外面的飞檐。封石上同样给炸了一个洞,比刚才那个更大。

我们不做停留,继续爬了过去,后面还是一样,墓道继续延续,面前又是封石,上面还有洞。

“我操,他娘的还没完没了。”胖子嘀咕道。

我道:“这很正常,一般的封石都七八吨重,长一点的墓道会有六七重封石,这些算是好的,厚度可能只有一半。咱们的老祖宗没炸药,对于这种封石塞道的古墓是一点办法都没有的。”

说话间我们穿过了最后一道封石,我们面前出现了一个十字路口,另一条和我们所在这条甬道垂直交叉的墓道从我们面前穿过,而这条交叉的墓道比我们所在的甬道宽度还要宽一半,高度更是高的多。

我们陆续走到十字路口中央,发现这一条墓道不是刚才的那种黑色,而是一片丹红,上面是大量鲜艳的壁画长卷,几乎连成一体,一直覆盖到手电照不到地方,连墓道的顶上也全是彩色的壁画。

我赞叹了一声,“这条肯定是主墓道了,直接通到椁殿的直道,整座地下玄宫的中轴线,不然不会修饰的如此华丽。”

“别感慨了,咱们是贼,还是老问题,往哪里走?”胖子问道,“快找找,附近还有引路的标记没有?”

我们经过几次在狭窄坑道中的穿越,早已经失去了方向感,要分辨这条主墓道,哪一头是通往地宫中心,哪一头是通往主墓门,只有靠前人的提醒,不然只有丢硬币来猜了。

我们的手电光点在墓道里划来划去,寻找那种符号,红色的壁画发射出一种让人感觉十分不安全的光线,这里的壁画就是我们在入山之前,在温泉缝隙中看到的那种风格,全是在腾云的仙车和仕女,似乎没有什么特殊的意义,当然如果让考古的人来说,还是可以说出一些名堂,但是在我们看来,没有叙述性质的壁画就纯粹是装饰性,我们看不懂象征意义。

才找了一会,一边的潘子突然就“嗯”了一声,招呼我们过去。

我们凑过去,果然又发现了一个符号,给雕刻在一边的墓道墙角。

“这他娘的省事情了,碰到倒斗界的活雷峰同志了。”胖子道。“咱们一路顺着走就行了。”

我这时却摇了摇头,因为看发现,这一个符号,和我们以前看到的那几个,已经不同了。

\chapter{一个新的记号}

在海底墓中的符号的样子,我已经记不清楚了,但是刚才刻在护城河底和甬道口子上的两个符号我还记忆犹新,现在这一个符号,和那两个完全不同。

胖子潘子他们,对于英文字母实在是没有概念,只要是英文,他们就认不出区别来,所以刚才没有在意,但是我这个上过大学,考过四六级的人,虽然成绩再不济,也至少知道这两个是不同的单词。

我一直认为这只是一个单纯的引路符号,类似于任何一种简单的图形,只有“往这边走”的意思。但是如果单纯就是引路,符号是不应该会变化,按照人的一般心里,进入墓道之后,注意力应该完全在四周的环境上,雕刻符号的时候,不可能有意的去变换花样,而且符号雕刻的也非常匆忙,说明这个留记号的人,并不是在非常从容的情况下做这件事情,这也更排除了他心血来潮变化符号的可能行。

那现在这种现象,就只有一个理由,那就是这些符号,他是有不同的意义的,他在引路的同时,也似乎在告诉我们什么信息。

问题是,那到底是什么信息呢?这洋文不是洋文,但是却是英文字母组成的单词,实在看不出是什么语言。但是常见相似的如德语法语就肯定不是,因为字母的排列太没规章了。

而且我们在河底和甬道口看到的那个符号,进入之后没有遇到什么危险,那么如果假设意义是:可以安全进入,那现在这一个不同的符号,刻在这里,意思肯定不同了,难保不会就是一种警告,表示墓道的这个方向,有什么可怕的危险?

胖子他们听了我的想法也觉得有点问题,我们停在原地,暂时不敢轻举妄动。

不过,到了这里已经是一个不大不小的突破,可以说已经成功了一半,此时墓道走哪边这种问题显的并不重要,就算没有符号指路,我们也并不惊慌。

只不过进入地宫,特别是主墓道之后,凡事就必须特别小心了,因为只要古墓之中有机关陷阱,那肯定就在这一段了,在这里花点时间是必须的。

潘子对我道:“小三爷,咱们这里也就你有点洋文知识,连你也不认识,那就没法认识了,你要不把这几个英文字翻译成中文,咱们不知道整句话的意思,咱们也能猜啊?”

潘子一点英文都不会,他大概是认为英文实际和中国字一样,是一个字母一个意思,我懒的给他扫盲,对他们道:“说要猜的话,不如猜这符号是谁留下的,以及他留下来的目的,这样猜到意义的可能性还大一点。”

胖子奇怪道:“谁留下的我们不知道,但是留下的目的我们还用猜吗?这肯定是给我们引路的啊?”

我摇头道:“我以前也这么想,但是现在就非也,如果真是为了我们留的,至少该写我们看的懂的符号,雕刻这些符号的人用的形式如此晦涩,现在看来目的并不是帮助我们,我们可能只是捡了个便宜,这符号是给别人看的。”

潘子想了想,觉得有点道理,又问道:“那别人是谁呢?”

“阿宁他们人多,可能分批行动了,这符号可能是他们几个小队之间的暗号。”胖子道。

我点头,表示有这个可能,但是没有根据,实际情况就无法猜了。道:“也有可能是其他原因,这个现在猜也没用。”

最让我在意的还是这个符号里包含的信息,这种符号应该是类似于国际探险地图的图列,有的原始丛林小道,在地图上的标示都有危险等级之分,一个符号除了告诉你这里可以走之外,也可以知道这条道路上会碰上什么东西,比如河道中有河马,就会有河马意义的暗号。

到了这里,这个符号竟然改变了,那这个特殊的符号意义就让人不得不上心了。会不会是表示这条墓道中有粽子呢,这真是让人郁闷。

我想起越野车上面的“熊出没注意”,也许留下这个符号的人也有着探险理论化的做事情方式,这个符号,也许就是“粽出没注意”的意思。随即我又想到如果能活着出去,是不是该在我的金杯小面包上贴一个,以表示我的个性。

潘子不知道我已经在胡思乱想,突然对我道:“也不对,我觉得这个符号表示的信息不可能有什么危险方面的提示,你想,墓道之中有没有危险,要走过才知道,没理由他们走过之后,再返回来刻这个符号,也就是说,这个符号是那人即将要进入这个墓道的时候刻的,表示自己走了这个方向了,告诉后来人自己的行走顺序,至于里面是什么,当时他刻的时候是并不知道的。这其实有讲究,叫做‘追踪语言’。”

我没听说过这东西,胖子问他:“什么叫追踪语言?”

潘子道:“我打越南猴子之前,当兵的时候学文化课,因为是在丛林里服役,所以学过很多关于救险的东西,‘追踪’语言,就是一旦在丛林里遇险迷路,你在自己找出路的同时,必须标志你的行走路线,这种表示的方法是有特别的规律的,后来的救援队看到你的标识,就知道你在这一带做了什么事情,比如说食物充足的情况是一种标识,食物吃完了的情况又是一种表示,队伍中有人遇难了,又是一种标识,救援队跟着你的标识走,就可以一路知道你的近况,如果事情极度恶化,他们就可以用这个标记作为依据升级营救策略,这听说是老美打越南人的时候发明的东西。”

胖子问他:“那你学过,你能看懂吗?”

潘子摇头道:“我是说也许,这个暗号和我当时学的东西完全不同,我也认不出来,但是我相信这应该是追踪语言的一种。我们没有必要去破译他,这个符号的变化,也许是只是说他在这里扭了脚。”

胖子叹了口气,道:“情况不妙啊,如果真是‘追踪’语言,那说明留下这个符号的人他娘的并不是志在必得,他是为了自己的第二梯队做准备,也就是说,他并没有信心自己这一次进入这里能活着出来。”

潘子道:“对!所以说了这么多,也没有实际作用,我看,既然这符号不是留给咱们看的,咱们就当没看到这标记,我们现在的主要任务就是找到三爷,符号不是三爷刻的,也就是说三爷不一定是走的这一条道,跟着走就算走得再顺也没有。我们走我们自己的,以前倒过不少斗了,也不是没碰到过这种情况,我就不信咱们连探个墓道都摆不平。”

这论调符合胖子的胃口,胖子点头同意,对我们道:“老潘,这句像是人话了,那不如我们兵分两路,你和小吴走那一边,我和小顺子走这一边,咱们看看谁的彩头亮,反正是直路,如果走到底发现不对,折回来就是了,另一对走对的,就在椁殿外等其他。在这里犹豫,也不是办法。”

我感觉这样不妥当,道:“话是这么说没错,只怕这主墓道不是这么好走,你看地下的四尺石板,这种墓道很可能装着流矢和翻板的机关,别是两队走到最后,都死在墓道里,咱们一分开就永别了。”

胖子嘲笑我道:“照你这么说,你就不该来,你吃饱了空,下这儿来干什么,既然下了地宫了,这点儿破事就不该怕。”

我心说这是我想来的嘛,老子的志愿一直是当一个腰缠万贯的小市民,也不知道今年走的是什么运,犯的尽是粽子,现在我倒是已经不怕粽子了,但是小心都不让我小心,这叫什么事儿。

潘子的想法和我相同,对胖子道:“不,小三爷说的对,就说一个理由,阿宁马队里的人肯定就在附近了,咱们不防范着粽子,也要防范人,两把枪的火力总比一把强,而且万一一队人出去就消失了,没回来,那另一队怎么办?咱们还是在一起好,有个照映。”

一直没说话的顺子也表态:“不管怎么样,我必须把吴老板送到,我肯定得跟着他。”

胖子举手向我们三个投降:“你们两个这是搞个人崇拜啊,他娘的孤立我一个啊,算我倒霉,那你们说怎么样就怎么样吧,大不了一起死。”

潘子道:“我们就先走这个刻了记号的方向,如果不对,再回头,事事小心就对了。”

我们点头答应,我心里明白的很,反正事以至此,我们在这里讨论的再好也无用,现在走哪边,怎么走,全要靠运气了。

于是起身,潘子扯出类似于盲人棒的折叠探路棍,一边敲着地面,我们就向刻了符号的那个方向走去。

一路走的是极其小心,我其实心中已经非常厌烦这一种走路都不得安宁的地方,但是有没有办法,既然来到这里了,总不能少了这一步骤,否则之前的千辛万苦,不就白费了。

本以为会在这墓道中消耗至少半个小时的时间,没想到的是,这一段墓道极短,不到二百米,便陡然变阔,尽头处出现了一道巨大的玉门。

我一眼便认出了这是冥殿的大门,因为墓道口的墓门不会用如此好的石料。门的下半截已经给炸飞了,露出了很大一个空洞。显然已经有人进入过了,不知道是阿宁他们,还是其他人。

我心中暗喜,这么说我们还是走对了路了,门后面就是整个地宫的核心部分,我的脑子里马上浮现出很多经典陵墓的结构,这里虽然是东夏的皇陵,但是由汉人主持建造,想必和中原的墓葬不会有太大的区别,进入之后会看到什么呢?我不禁有一些紧张,不知道万奴王的棺椁是什么样子,四周有没有陪葬的棺材。

墓室的玉门十有八九会有机关,两边的石墙很可能是空里,里面灌着毒石粉,而且这种机关往往没有破解的办法,因为墓室一关就没打算再开,就算你是设计这门的工匠,关上之后你也进不去。

不过这门已经给炸成这样了,估计有机关也给破坏了,这一点到不用担心,我们几个俯下身子,鱼贯而入,进入了门后的墓室之中。胖子谨慎起见,打起了冷烟火,让我们的照明力度加大,好一下就看清楚墓室里的布置。

在冷烟火亮起的一瞬间,我们就看到一幅让人窒息的情景出现在了我们的面前,所有人都没有想到自己会看到如此的情景,几乎都冻立在了原地无法动弹。

\chapter{无法言喻的棺椁}

这个墓室比刚才看到的葬酒室,高度和宽度都差了将近十倍,四根满是浮雕的巨形廊柱立在墓室的四个角落里,墓室的地面上到处堆着很多东西,冷烟火一亮,我们就发现那是小山一样的金银器皿,宝石琉璃,珍珠美玉,我们的手电照上去,流光溢彩,简直让人不能正视。

“我的爷爷——”胖子眼睛瞪的比牛还大,脸都扭曲了。

我也惊的够呛,几乎站立不住,潘子喃喃道:“我说什么来着,女真的国库,南宋的岁供,我他娘的没说错吧。”

涉足这一行这么久,见到的都是破铜烂铁,我以为这一次也逃不过宿命,没想这小小的边荒弱国的皇陵内,竟然会有如此多的宝贝,难道真的如潘子所说,大金灭国之后的宝贝,全都给屯到这里来了,那就是不是发财的问题,这里的东西,随便拿几样出去,就可以吃一辈子了。

胖子想滚到金银器堆里去了,我都有上去滚滚的冲动,但是心中还有一丝理智,拉住胖子让他不要得意忘形,很多墓葬的金器上都喷着剧毒,滚到里面去被毒死,太傻了,这些东西最好还是不要碰。

可是我拉住了胖子,却没拉住潘子,他已经冲进金器堆里,抓起了一大把金器,目瞪口呆的看着,反射出的金光照的他的脸都是金色的了,浑身都在发抖。接着他松开手,那些东西就从他的手指缝里摔落下去,发出金属撞击的声音。

我看潘子抓了几把也安然无恙,知道金器并没有毒,一下放宽了心,忍不住也上去抓了一把,那种沉甸甸的感觉,几乎让我控制不住的大笑起来,我不知道是谁说的,人类对于黄金的喜爱,已经写入了基因中,变成了与生俱来的,不可抗拒的本能了,他真他娘的说对了。

就算如我这样,虽然表面上道貌岸然,但是看到黄金的那种悸动,却是由心里发出来的,我就想骗自己也骗不了,我喜欢这些东西。

几个人一下子就把什么都忘记了,我们一下跑到这一堆里,捧起一堆东西来,又跑到那一堆里,拿出一只镶满宝石的头箍仔细的看,这些东西都是真正的极品,只要有一件,放到博物馆里就是镇馆之宝,现在这里却有这么多,随便拿,随便的踩,都不觉得可惜。

胖子在一边已经开始往他的包里装东西了,他把他的装备都倒了出来,什么都不要了,用力往包里塞,塞满了,又觉得不对,全部倒出来,又去塞其他的东西,一边装一边还一边放出毫无意义的声音。

但是很快我们都发现,无论怎么装,都带不走这宝藏的万一,装了这些,马上又会发现更好更珍贵的东西出现在他下面,装了那更珍贵,又发现从来没见过的真品,一下子简直无从下手。

疯狂了很久,直到我们筋疲力尽,人从极度的兴奋中平静下来,我才感到不对劲,何以进来之后就,没有没有听到顺子的声音。

我擦了擦头上的虚汗,从珠宝堆里站了起来,用手电四处照,看到顺子正站在一座金器堆上,不知道在看什么东西。

我走过去,问他在干什么,看到这些黄金不兴奋吗?

他没有说话,而是指了指下面,我用手电顺着他的手电照去,发现在几堆金器的中间,无数财宝围绕的地方,里面竟然蜷缩着几个人,一动不动,似乎已经死了。

我顿时就吓了一跳,刚才的兴奋突然就消失了,起一声鸡皮疙瘩。

胖子和潘子看到我和顺子都呆立在了那里,以为我们又发现了什么宝贝,飞奔过来一看,却是几只粽子,不由也吃惊不少。

我们走下金器堆中的那个凹陷,反手握住手电,仔细照了照,发现确实是死人,而且死了有一段时间了,尸体的皮肤冰冻脱水成了橘皮状,只是奇怪的,这几个人穿的,竟然是腐烂的呢子大衣,是现代人的衣服,身边还有几只烂的不成样子的老式行军包。

胖子奇怪道:“怎么回事情?这些是什么人?咱们的同行?”

我摇头,带上手套翻了翻那些人的背包和衣服,这种装扮,应该是在80到90年代比较流行的衣服,现在东北的农村大概四五十岁的人也会穿,我们在营山村就见过不少这样的半大老头。看腐烂的程度,这些人也应该在这里死了五到二十年了。

潘子问:“会不会是长白山的采药人或者猎户,误进到这里,走不出去死了?”

“不太可能。”我扯开一具尸体的衣服,那是一具女尸,又看了看女尸的耳朵,上面挂着老式的耳环,手上还有手表,早就锈停了,“你看,这是梅花表,老款式,当时就算市长级别的人也不一定搞的到,这女的是来头不小,不像是农村里的人。”

“那会不会是以前80年代的迷路游客?”潘子又问,“我们一路跟过来的记号,是他们刻的?”

我摇头,记号是他们刻的是不可能,因为那记号我在海底墓穴中看到过,肯定是相关的人刻的,不是阿宁他们,就是闷油瓶,说是迷路游客倒有可能是,但是真的迷路可以迷路到这种地方来?地宫墓道,没有相当的胆量,普通人是不敢下去的。

不过如果这女人有点来头,比如说是什么领导人的子女,或者和地方的官僚有点联系,失踪了说不定会在当地影响很大,顺子年级不小,当时可能会听到,就想转头问他,五到二十年间,他们这里有没有出过什么比较轰动的失踪事故。

转头一看,顺子却没有跟着我们跑下来,还是呆在那金器堆上,表情十分的僵硬。

我心说奇怪,难道顺子也象胖子一样中了尸胎的舌头了,又没看到他的脖子上有东西啊,我看他竟然还有点发抖,就感觉到不对。

胖子对他道:“怎么了,怕死人啊?刚才怎么没见你怕啊。”

顺子不理胖子,脚步沉重的一步一步走下来,来到其中一具尸体之前,蹲了下来,我发现他紧张的几乎要摔倒。

突然就想到了是怎么一回事情。

胖子还想去拍他,我拦住胖子,对他摆了摆手,胖子轻声问我:“他怎么了?中邪了?”

我摇了摇头,这几具尸体,如果我猜的没错,可能就是顺子和我提起的,他父亲十年前带入长白山的队伍。而顺子现在看着的那具尸体,有可能就是他的父亲了,所以他才会出现如此紧张的举动。

想不到,真的给他料中,跟着我们,真的可以找到他父亲的遗体……

可是,这是巧合还是什么?十年前的队伍,是误入了这里?还是有这其他我们不知道的隐情呢?

\chapter{十年前的探险队}

顺子最后并没有哭,激动了片刻后,人也放松下来,恭敬的给他的父亲整理了头发,但是尸体已经严重脱水了,头发一碰就往下掉。好不容易整理好了,他父亲也就变成葛优的样子了。我知道这小子心里肯定还是不好受的,也许他十年内还有着父亲还活着的侥幸,现在侥幸破灭,人可以说轻松了,也可以说绝望了。

胖子和潘子不知道怎么回事情,看的莫名其妙,直冒冷汗,我就简单把我猜的事情和胖子潘子说了,相信我也没猜错。

胖子听了也流眼泪,说:“我家老头子也去的早,给国家干了一辈子革命,最后还给扣上反革命的帽子。顺子你的心情我可以理解,不过人嘛总要往好的方面想,十年后父子还能重逢,老天也算照顾你的了,看开点。”

胖子一哭潘子眼眶也湿了,说好了好了,你们都还有老爹,我老爹的面都没见到过,三爷一直象我爹一样,现在也生死未明。

我忙道:“你们有病啊,顺子都没哭你们两个凑什么热闹,快看看他们为什么会死在这里。”

他们既然能走进这里,没有道理出不去,死在这里肯定是发生了什么意外。我们现在也身处于这个墓室之中,我可不想我们步他们的后尘。同时我也感觉着几具尸体出现在这里有一点蹊跷,顺子的父亲不说,只是一个领路人,其他几个人,按照顺子说起来也是在不适宜进山的非要进山,应该不是普通游客,是不是也有什么不可告人的目的,进这里是巧合吗?我一定要知道。

我们去翻找这些人的背包,背包里还什么东西都有。翻出来象腐烂的松夸夸的小说、笔记本、铅笔、牛筋绳索、行军帐篷、老式手电,老版瑞士军刀(竟然还能用),韩中辞典1986版的。泡泡糖、老式打火机、酒壶、口红、卫生带。医药盒子包括纱布、酒精、棉花和几种药酒,军用指南针等等等等。

小说是《钢铁是怎样练成的》,老书了,我都不敢去翻,一翻肯定就散架了,笔记本也都是老时候的工作笔记,我小学的时候见过老爹用过,一共有三本,翻开来一看,都是记录了一些账和电话号码,当时的笔记也就是这些功能。此外,也没有任何东西能证明他们的身份,最主要的是,没有一个人带了身份证。

我们把这些东西全部摆成一列,几乎设备齐全,虽然没我们的先进,但是要出去应该不成问题,再险恶的环境,这些装备也可以应付个大不离了。

这就奇怪了。我心里琢磨,无论怎么样,在有能力离开的前提下,这些人要死,也应该死在出去的路上,而不应该是坐在这里,似乎是等死一样的,难道是舍不得这里的宝贝?这更不可能。

那如果是这样,难道死在这里是另有蹊跷?我心里突然涌起一股不详的感觉,突然感觉到这个墓室之中似乎有什么东西在看着我们,不由打了个寒战。

一边的胖子看这这些我们陈列出来的东西,突然“啧”了一声,道:“同志们?你们有没有发现这些东西里面,少了什么?”

我们都在琢磨,听胖子这么问,又仔细看了看那些东西,但是在我的概念里,我感觉所有不可缺的东西都在了,实在想不出缺了什么,问他道:“少了什么?”

胖子道:“食物!没有食物!所有人包里都没有食物。”

他一说,我们顿时就一个激灵,再看向这一排东西,果然,全部都是装备,没有任何可以用来充饥的东西。

我奇怪道:“真的没有食物,这说明这些人不是因为意外死的,如果是因为意外死亡,可能不会这么巧,所有人都没有食物,不对啊,那他们难道是……吃光了食物,在这里饿死的?”

这又说不通了,人从没有食物到饿死,只要有水,体型正常的人足够可以坚持一个月的时间(你2米27却只有90斤的人就不要来找我抬杠了),只要他们有心出去,也不会在这里饿死了,这些人如果饿死在这里,那只有一个解释,他们出不去。

想到这里我就想起了海底墓穴中会消失墓道门,忙跳起来跑上金器堆去照我们进来的墓门,那墓门却还在,根本没有消失,才松了口气。又怕那门突然消失,有点不知所措起来。

胖子知道我在担心什么,对我道如果真的遇上了那种情况,咱们这一次有炸药在身上,也不用怕,我才觉得心安了很多。

“会不会是这样?”想来想去想不明白的时候潘子问顺子道:“你知道不知道你父亲带的探险队是几个人?”

“好象只有七个人,我母亲说,但是这只是她看到的,实际有几个人她也不知道,反正我父亲临走,是和七个人一起出发的。”

“那这里有……1,2,3,4,5,6,六具尸体,还有至少两个人不见了。”潘子道:“这些人死在这里,会不会是那两个人见财起意,把人杀了,有两个人跑了。”

我摇头表示否定,这些人一点也没有打斗的迹象,看临死时候的动作和表情,是蜷缩在一起,也不象是中毒,又不象是受外力死亡的。最让我感觉到不妥当,一定要弄清他们死因的是,尸体的表情,十分的统一,无以不透露出一种深切的绝望,似乎陷入到了一个毫无希望的境地之中。

我是第一次看到这样的尸体,心中无法释怀,我有一种预感,当年在这里发生的事情,肯定很不简单,而越往深入去推测我越觉得四周开始笼罩起一股无法言语的寒冷和不安,这堆金山之中,有什么东西正在注视着我们的那种毛骨悚然的感觉也越来越明显起来。

琢磨了半天也没琢磨出什么名堂来,胖子他们就按捺不住了,又想开始去捣鼓那些金器,我这一次很冷静把他们都拦住了,说这几个死人死在金器堆里,我实在感觉放不下,我们先不要动了,别忘了我们来这里的目的。

我一说他们才醒悟过来,一下子胖子就想到了什么,道:“我还真晕了,忘了来这里干什么了,那记号引我们到这里来,门也给炸开了,但是里面只是一个藏宝室?没有棺椁,我看那个记号的意思也知道了,就是有明器的意思,记号肯定是阿宁他们留的,以便他们的第二梯队来运宝贝。”

我道:“门倒可能是这几具尸体炸的,不过这里只是一个放陪葬品的墓室,那棺椁肯定不在这里,我们要向相反的方向走。”

虽然不合情理,我一直以为这条墓道是主墓道,一边是墓门,一边是地宫中心,现在看来却不是,那难道这一条仍旧不是主墓道?那这地宫到底有多大啊?别是迷宫一样,一想到是想起那些记号,难道真的是因为地宫太复杂,他们才留下这些记号的?

“那些东西怎么办?”胖子有的舍不得。

我道:“你随便拿一样走就足够你过半辈子无忧无虑的生活了,也不用太贪心,而且以后也不是不能回来。”

胖子看到那几具尸体只后,显然心中也犯着嘀咕,但是什么也不带走又不可能。于是挑了几样小样的金器揣到兜里,顺子坚持要把他父亲的尸体带出去,用背包袋子把尸体背到了身上,尸体已经脱水,没有什么份量,也不难背。

我们最后看了一眼金光璀璨的金山玛瑙堆,狠了狠心,又鱼贯走出了墓门下的炸口。

才一出墓门,我就又听到胖子“嗯”了一声,我心里早就有点预感,忙打起手电四处一照,不由就一身白毛汗。

外面墓道上的壁画,竟然和刚才走的时候不同了,不知道何时,红色的壁画全部变成了一个个黑色的,脑袋奇大的人的影子。

\chapter{影子的道路}

顺子和潘子看的膛目结舌,自言自语道:“我操,怎么回事?走错门了?”

“不是!”我和胖子都有经验了,马上就知道了是怎么一回事情:“这墓道移位了,我们在墓室里面的时候,老的墓道移到了其他的地方,一条新的墓道移动到了这里。”

“这样都能做到?”潘子张大嘴巴。

“能!”我和胖子都用力的点了点头,心说何止这些,在汪藏海设计的慕穴中、发生什么事情、也不用奇怪。

我心里有点害怕,但是又有点安心,因为墓道一改变,我就突然明白为什么那顺子的父亲和另外几具干尸会活活困死在了黄金之中、如果不是通晓汪藏海的计策,那这里诡异的墓室墓道变化,足可以把人逼疯,我们在海底墓中就几乎给骗的丧失了理智,但是一旦我们知道了这里墓道突然变化的原理,这就一点也不可怕了。

这墓道一变化,我们来时候的十字路口必然就不存在了,要回去也不可能了,虽然不知道这条新的墓道尽头是什么,但是如果我们留在这里不走,那下场必然就和那几具尸体一样了。

我当时琢磨的,是最多也就是墓道尽头什么都没有,是死路,这也没什么大不了的、那些尸体困在这里也至多是这样的原因,没有炸药,来时候的路突然又消失,自然会不知所措,露出那种绝望的表情。

事后想起来,我到底还是太年轻了,尸体脸上的那种绝望的表情之深切,预示着他们遇到的事情比我想象的要匪夷所思得多,而我当时想地实在是太简单了。

我把想法和其他人说了,又给潘子和顺子解释了墓道变化地原理,他们才醒悟过来,露出了不过如此的表情,不过潘子就想的远了一点,道:“如果是这样的,理论上这个地下玄宫的结构会无限复杂,我们会不会像深陷入魔方中一样,走进入就怎么也走不出来?”

我让他放心:“应该不会,汪藏海的伎俩说实话也只是给盗墓贼施加心理压力,真的要做到困人到死,也不容易,我估计最后很多人都是给折磨的精神崩溃才死的。”

总之这条新出现的墓道,栽们必须要走一走,然后想想办法,实在出不去,就如胖子说地,可以先确定一个方位,然后一步一步炸出去,我们现在有了炸药,腰板就硬了很多。

说着我就带头走入了墓道中,胖子他们紧跟其后,一下子我就感觉到不对。这四周的壁画太寒人了,这么多大头影子,筒直就好像四周站满了这样的东西一样,让人极度不舒服,我突然想到,是不是这秘道的尽头就是有这么一个东西,它的影子照到墙上的时候,我们肯定发现不了。

不过走也走进来了,再退回去太丢脸了,我只好硬着头皮走在最前面,尽量不去想这些东西,很快,身后的墓门就看不到了,我们走到了两头不着边的地方。

身后的潘子边走边问顺子父亲和探险队的事情,顺子和他讲了一些,潘子就对我们说:“刚才我们一路过来,所有的封石都是用定向爆破炸出洞口地,是最新的技术,说明他们不是顺着我们进来的路线进来的,看来这里肯定有不止一条路出去。”

我道:“肯定的,你看阿宁他们走的这么快,他们走原路竟然可以比我们先到就知道了、我们还是输在情报太少上。”

只不过不知道阿宁他们现在到哪里去了,他们应该也到过刚才的那个藏宝室,是不是也出来碰到了墓道移动、是不是和我们进的墓道一样,更加,三叔是不是也是这样?

我心里实在没底,我们已经按照三叔的暗号来到了地宫之内了,他没有后续的暗号给我们、看样子进入地宫之后,他可能也是没头苍蝇了。

边说边走,走了大概二十分钟,照向前面地手电光出现了反光,证明墓道的尽头到了,我们不由都紧张起来、马上安静下来,放慢了速度,一点一点地走过去,很快,墓道的尽头又出现了一道有玉门。

玉门刚出现的时候,我猛然就给震了一下,因为这道玉门和刚才那道实在是一模一样,随即一想,古墓中的门大部分都是一个工匠负责的,当然会很象,门的石料质地还是很好,门下方也有一个破洞,也是给人炸出来的。

看样子还是有人来过了,那就好,不管是谁来过,对我们都是好事情,至少证明没有机关陷空。

我们再一次鱼贯而入、因为没多少冷烟火了、这一次胖子没舍得点冷烟火,而是打起了几只火折子。我们四处一看,不由一愣。

墓门后面是和刚才的藏宝室一模一样的房间的,墓室内成堆的金银宝器堆成小山一样,墓室的四个角落里四根巨大的柱子,格局几乎一样。

我心说这地宫中这样的房间还不止一间,那堆积的财宝到底有多少,难怪东夏王朝这么盈弱却仍旧可以修建如此雄伟的陵墓地宫,原来囤积了如此多的宝贝、想来独载政权都有这个习惯,成吉思汗的灵藏在蒙古的草原之下,希特勒的纳粹黄金听说是埋在了西藏,女真大金耶律兄弟的就在这里了。

正胡思乱想着,突然一边的胖子大叫了我一声,声音之大,吓了一跳。

我以为出了什么事情,朝他看去,只见他张大嘴巴,站在一座金山上,不停的想说话,却一口气卡住什么都说不出来,我忙跑上去一看,不由也大吃了一惊,只见在这里的宝藏包围中,也蜷缩着几具尸体。

我奇怪的问道:“顺子,你有几个父亲……啊不,你父亲的队伍到底有几个人?”话还没问完,我就突然看到一个让我毛骨悚然的现象,只见那堆尸体边上的金器堆里,给人整齐的摆放着一串东西,我用手电一照,正是我们刚才在另一间藏宝室里整理的出来的一些,顺序、类别都一模一样。

胖子再也忍受不住,在一边打起了冷烟火,一下子就把整个墓室照亮了,我们走了下去,仔细一看,这些东西分明就是我们刚才拿出来的东西。

胖子骇然道:“怎么回事?这……有人模仿我们的行为……?”

我皱起了眉头,站起来,环视了一圈四周的、一股熟悉感觉袭来,哑然道:“不是……是我们自己又走了回来,这里就是我们刚才出发的地方!”

\chapter{永无止境的死循环}

几个人的脸色都是铁青的,我们四处去看,越看就越确定,地上到处还有我们的脚印,这里的确就是刚才我们发现顺子父亲的那间墓室,只不过奇怪的是,我们怎么走回来的?

墓道是笔直的,我们走的时候,没有转一个弯,四个人一条尸,都可以证明,按照道理,绝对不会走了二十分钟,却回到了原点。这简直太匪夷所思,简直是鬼打墙嘛。

胖子有点犯嘀咕,看了看来时候的墓道口,道:“难道我们走的时候,不知不觉,就走了回头路了?他娘的这邪门啊。”

潘子道:“不会吧,要是走了回头路,咱们四个人不可能都不知道,我记忆里面一直就是笔直走,这墓道又不长,也没有叉路,没有理由记错啊。”

胖子道:“那他娘的就是鬼打墙了,顺子,是不是你老爹和咱们开玩笑啊?你可得教育教育它,咱们在办正事呢。”

顺子给胖子气的够呛:“你少胡说。”

我拦住他们,现在这个时候实在不适合扯皮,我浑身都出了冷汗,因为我感觉到,最不想发生的事情,可能已经发生了,但是我心里还是不敢完全肯定,道:“你们不要吵,要看看是不是真的是走了回头路,只有一个办法,我们再走一遍看看。”

几个人面面相觑,看到我的表情,他们大概都感觉到了不妙。

当时我心里想的已经是那几具干尸的表情了,那种绝望的表情,难道他们就是在这里,被这种方式困死的?没有了食物,但是又怎么走都会回到原来的房间,这也太匪夷所思了。但是我的直觉告诉我我可能猜对了,而且困死他们的事情,现在已经同样发生到了我们的身上。

我现在必须要做的,就是证明我的这个预感,或者说我心里想否定我这种恐怖的预感,所以我迫不及待的走进了墓道里,其他人忙跟上了我。

因为走过了一次,确定没有机关陷阱,这一次我们走的非常快,我几乎是一遛小跑的冲在最前面,眼睛死死就看着两边的路,确定没有任何的叉路,我也没有莫名其妙地转回头。

这一次不到十分钟,我们就跑完了全程,在感觉即将要看到墓道尽头的时候,我几乎在不停的祈祷、希望自己的预感不要实现。但是最终,当我看到那扇几乎一模一样的玉石大门的时候,我的心顿时就凉了,冷汗就不由自主的往外冒。

走入大门,胖子就冲上了那座金山。然后他就跪了下来,捂住了自己的脸。我冲上去一看,六具尸体,我们排列开的东西全在……我们又回来了。

我的预感应验了,在100%全神贯注地确定没有叉路和回头的前提下,我们一路直走,竟然还是走回了起点。

胖子跑的累了,大喘气道:“这是鬼打墙,这绝对是鬼打墙,咱们怎么走都是一个循环,这墓道的两头都是这墓室,咱们这一次要去见顺子的爹了,顺子你倒是和你爹说说,别玩我们,不然咱们就把他扔这儿自己走了。”

顺子已经惊讶地够呛,没工夫和他绊嘴了。我也心慌意乱,不住的转身看四周的墙壁,但是又不知道自己在看什么。

“冷静!冷静!”潘子在一边大口的喘着气,“千万不要乱,小三爷你自己不是说汪藏海的东西充其量还只是制造心理压力的小伎俩吗?我们千万不要知道这一点还中招,现在一定要冷静,肯定有什么地方不对了。”

给潘子一说。我突然倒是醍醐灌顶,一下子人清醒了不少,那种绝望的感觉顿时淡了,忙点头,道:“你说的对,这肯定是机关,我们在海底墓穴已经证实了,没有什么鬼打墙的事情,汪藏海善于使用巧妙的机关,来营造诡异的气氛,如果不知道底细,很容易就给他牵着鼻子走。”说着忙用力揉自己的脸,让自己从那种窒息的感觉中脱离出来。

说这些话其实是说给我自己听的,我说完之后都不知道我说了什么。

事后我想起这时候,感觉当时我应该是已经感觉事情超出了我的控制,想用这些话来暗示自己不要放弃。

因为刚才走那条墓道的时候,感觉太真切。我其实根本无法想象用机关怎么来实现这个现象,脑子里首先出现的就是墓室或者墓道地移动,但是这不可能,马上就给我否定了,我们走的并不慢,墓室如果能移动,他需要多快的速度?墓道就更不可能,我们在其中、只要有一点震动,我们绝对可以知道。但是如果不是墓道和墓室移动,那这就无法解释了。

虽然我不停的告诉自己这是机关,但是其实我的心里已经知道不对了,这用机关无法解释迹,但是这样说出来,对其他人还是有好处的,至少可以减少恐慌。

不过我是小看胖子他们的心理承受能力了,潘子比我要镇定得多,擦了擦汗,问我道:“不管是鬼打墙还是机关,都得解决,现在怎么办?要不要再走一次?”

我一咬牙,“再走!他娘的这一次咱们走慢一点,好好感觉一下脚下或者四周的动静,我就不信没破绽。”

于是我们又走进了墓道之内,这一次走了四十分钟,还没走到底我们就知道失败了,因为墓门一模一样,一路上什么也没有感觉到。

其后我们不知道又走进去了几次,全部都以失败告终。我逐渐就感觉到了那些尸体的绝望,几个人的脸色也越来越差。

我感觉到这样折腾下去不是办法,回到墓室之后,我让他们别走了,既然走了这么多次,我们基本上什么都排除了,这个机关肯定是用了我们根本想不到的办法来设计的。

胖子累的几乎虚脱,但是还是坚持想继续走,他的想法是,也许某时某刻,以前的那条墓道会回来,那时候我们就可以脱身了。

潘子听了他这话,只说了一句:“你死了这条心吧,那条墓道绝对不可能回来了。”

说着就看了看一边的那几具干尸体。意思很明显,那几具干尸走入墓道的次数,绝对比我们多的多,但是他们还是被困死了,所以走墓道是没有用的,再走一万次也没有用,我们不用去考虑这么走运的事情。

胖子顿时就歇了气了,坐下来,道:“照你这么说,咱们不是死定了?这几个人在这里,肯定什么尝试都做过了,我们再做一遍,也没有用啊。”

潘子道:“你少想这些,现在就这样想,那你干脆自己撞死好了,等到我们把能做的做了,再来想绝望的事情,现在趁还有力气,不如想想办法。”

我想起尸体食物的事情,问道:“要不要现在把食物限量一下,我们要做好长期作战的准备,能够活的时间越长,我们出去的机会也就越大。”

潘子叹了气摇头:“小三爷,不瞒你说,我们其实还不如他们,我们的食物不多了,我看最多也只能吃两顿,还不管饱,我看不用限量了,该怎么吃就怎么吃,保持精力充沛,我估计着,如果两天之内我们还出不去,估计什么办法都没了,那就该用炸药了,如果炸药也没用,那就等这别人来给我收尸吧。”

两天,我心里抖了一下,这几具干尸在这里呆了多长时间,我们能在两天内出去吗?这真的是一点把握也没有。

胖子的肚子已经在叫了,就问潘子:“那炊事员同志,咱们能不能提早开饭,我先把分散我注意力的事情先解决了,才有力气来想别的事情。”

胖子一说我们都觉得饿了,潘子没有办法,只好点上炉子做饭,我们的食物其实只剩下挂面了,刚吃下去的时候还可以,但是时间撑不了多久,胖子埋怨没有肉食,我说有速冻排骨,你要吃的下去,顺子不介意,我们就不介意。

吃完之后,浑身发暖,人的精神头也很足,几个人就开始琢磨这东西,我回忆整个下地宫的过程,惊险万分,没想到下到地宫之后仍旧不安稳,这个地宫,汪藏海肯定有一个设计的主旨,到底是什么呢?

地宫都是回字形的,灵殿在最中间,是制式最严格的地方,汪藏海必然不敢动手脚,其他地方,回字地宫周边是殉葬坑,陪葬坑,排水系统和错综复杂的甬道和墓道,这么说我们现在还在地宫中心的外延。

我尝试估计出我们下来的垂直距离和水平距离,凭借我对地宫大大小的估计来判断自己的位置,但是这似乎非常困难,我们在那条水下排道中已经昏了头,不知道方向,鬼知道我们最后出来的洞口是朝什么方位的。

整在飞速转动大脑的时候,一边也装模做样想事情的胖子,突然做出了一个恍然大悟的表情,对我们道:“我想到了!”

\chapter{胖子的枚举法}

胖子忽然说他想到了,我们都大吃了一惊,但是随即做好了胖子胡扯的准备。胖子这人的不靠谱,我们几乎都习惯了,与其每次挤兑他,还不如任他胡说算了!而且有时侯他的思维方式与我们不同,听一听到是也无妨。

其实我当时倒也不是非常慌。因为还没有到真正弹尽粮绝的时候,只不过有这几具尸体在这里,心里难免会有点不好的东西,事实上,像我这样的人,面对这种智力上的挑战,心里甚至还有一点庆幸。这是在比遇到若干粽子轻松多了。

潘子和我想法一样,没当回事情,随口问胖子道:“什么?你可别胡扯啊,老子们没功夫!”

胖子凑到我们身边,确是对潘子道:“你他娘的就是歧视我!老子哪一次乱七八糟了,这一次我想到的绝对关键!”

潘子打了个哈哈道:“就你那小脑子,你说你想到了什么?”

胖子这次却出奇的认真,正色道:“其实也不是关键,我刚才是灵犀一动,想到海底墓里的机关了。你想,当时我们也是想的很复杂,但是事实上事情多简单?我就琢磨这一次是不是也是想太多了,而且让海底墓穴里的机关搞得先入为主了。一遇到这种事情就想着房间会动啥的,也许这里的问题根本就跟墓室就没关系,这里就是个普通的墓室而已。”

潘子咧嘴道:“胡扯,要是普通,老子怎么走不出去……”

我看胖子没说完,知道还有下文,就对潘子摆了摆手,让胖子继续说下去。

胖子道:“其实事情就是很简单,你们想啊,如果这条道和这个墓室一点问题都没有,可是我们还是一直都走不出去,那问题出在哪里?肯定是出在我们自己身上了啊。”

这一下,我和潘子都愣了。我说:“你是说,这里的死循环是我们自己出了问题?”

胖子点点头道:“虽然是什么问题我们还不知道,但是差不离,我是想,是不是我们被那些壁画催眠暗示了?或者这里干脆有什么致幻气体。我们都中毒了。我就知道有一种蘑菇,吃了后方位器官错乱,自己一直在转圈,但是不知道。”

胖子和我说过以前他小时候猎熊的陷阱就用这种毒蘑菇,中了招后拿熊就一直转圈直到累死。

我一下子陷入了沉思,潘子也不说话了,皱起眉头开始考虑胖子的话。

是我们自己的问题吗?如果是这样,那事情的棘手程度就完全不同了。不过我略为考虑一下,就感觉不是很对。

事实上,胖子的说法很有启发性。也许试试离他说的很就近,但是却有一个很致命的不合理,就是我们自己的感觉,中了毒的人会是我们这样的样子吗,我不是没中过毒。中毒的人肯定会有强烈的不适反应。

而催眠,我一直不是很相信这种东西。因为他的针对性太强了,说胖子容易催命倒是可信的,但是我何潘子实在不太可能。

但是如果还是回归到奇淫巧术的范畴来,的确很难想出什么东西。其实刚才我构想了大概十几种方法,其中有两三种建筑结构完全可以实现这样的布局,但是这几种方法的要求太高了,就是说必须要有绝对的前提,比如说三个人不许一齐行动,我们行走的速度必须固定等等,汪藏海绝对不会设计这样低成功率的陷阱。

我突然先到自己是不是太挫了一点?想这汪藏海是多少年前的人了,为什么咱们一直以来就一点上风也占不到?

我们一下子各自思考问题,都入了定,胖子看我们听他说完就不说话了。一下子也不知道怎么办,只好继续装模作样地也沉思起来。后来,也不知道怎么地,我越想越困,越来越疲倦,接着竟然睡了过去。

不过大概只睡了三四个消失,迷迷糊糊的其实也没有睡死,就听到胖子和潘子说话的声音,又给吵醒了,起来发现他们又在走那条墓道,顺子显然刚跑回来,气喘吁吁的,看胖子的脸色,显然结果还是一样,并没有进展。

我揉了揉眼睛,问他们在干什么,胖子说想了半天也没有头绪,不如试验也好,他们刚才让一个人闭着眼睛在前面走,另一个人在后面看着,两个人用绳子连着,看看会不会走到一半,那个睁着眼的人会忽然转身。

我听着不寒而栗,这简直是让人崩溃的试验方法,也亏的这几个人神经大条,要是让我这么干,鬼知道走到一半那绳子另一头拉着的还是不是原来那个人。

不过最后走下来的结果还是一样,不管是蒙着眼睛,还是闭着眼睛,都是感觉自己走的是直线,但是两个最后还是走回了这个墓室。因为顺子是闭着眼睛那一个,所以走的格外吃力,脸色惨白。

几个人又坐回到自己的座位上,都是唉声叹气,我让他们省点力气,其实这样盲目的试验,反而会导致思维的中断。接着事情又回到我睡觉前,我们又开始毫无意义的讨论起来。

讨论种总是有人睡过去,但是好在一个人睡觉,其他几个人都能继续思考,就这样,我们东一个想法,西一个想法,提出来,然后否决掉,一开始说法还很多,后来几个人话就越来越少,时间不知不觉就过去了六,七个小时,我们的肚子又开始叫起来。

最后胖子点起一只烟,想了想,对我们道:“不行,咱们这么零散的想办法是很浪费时间的,我们不如这样,我们吧所有的可能性全部都写出来,然后归纳成几条,直接把这几条验证不就行了。”

我点点头。其实说到最后很多问题我们都在重复的讨论,几个人都进入到一种混乱状态了。

他在金器铺满的地面上整理湖一块石头面,然后写下来几个数字,1、2、3、4。说:“我们想想我们现在有几种假设。你们都回忆一下,不要具体的,要大概的方向就行了。”

潘子就道:“最有可能就是有机关。”

胖子在1那个地方写了机关。然后顺子就说道:“我的想法,可能有东西在影响我们的感觉,不如说心理暗示或者催眠,让我们自己不知不觉的走回来。”

胖子对他说:“不同说这么详细。”接着在2的后面写了错觉。然后看向我。

我道:“要说理论上,也有可能是空间折叠。”

“你这个不可能,太玄乎了。”潘子道。

胖子道:“不管,有万分之一的可能性。我们就承认,我们只是列一个备忘录而已。”说着也写了上去。然后自己说:“也有可能有鬼。”说着写了个4,有鬼。

“你这样写出来有什么意义?”潘子不理解的问。

胖子道:“你们念的书多。不懂。老子读书少,凡事都他娘的必须用笔写下来,但是这样有个好处,不如说有几件事情,你可以一起做。你事先一理就能知道,可以节省不少时间。咱们不是只有两天了吗?还是省着点,对了,还有5吗?谁还有5?”

我看了看着四点,这确实已经包括了量子力学到玄学到心理学到工程学,四大学科都齐了,第五点一时半会儿还真想不出来,我们刚才的讨论,其实也只是讨论一和二,三和四简直就是不可能的嘛。

胖子看我们都没发应,道:“好,咱们先来验证一和二,这两点正好可以一起处理。”

“你用什么方法验证?”我奇怪道。

事实上我们能做的试验大部分都做了,但是因为墓道过长的关系,很多试验其实都没有用处。

胖子突然笑了笑:“其实我刚才想到了一个好办法,要证明到底是1还是2影响我们,估计是不可能的,但是要证明不是还是有办法的,你看好把。”

我看着胖子得意满满,大有成竹在胸的感觉,顿时觉得不妙,这家伙是不是有什么打算了。只看他拾起地上的步枪,对我们道:“这条墓道大概1000m到2000m,56式射程是400m,但是子弹能打到3000m外,我在这里放一枪,看看会有什么结果。”

我一听顿时就醍醐灌顶了,心里哎呀了一声:这天才啊!

如果是因为我们自己感觉上出问题,那子弹是没有感觉的,墓道能够影响我们,但是影响不了子弹,如果这里的情况用常理还可以解释,那么子弹必然会消失在墓道的尽头不会回来。

这个试验之完美的地方,就是子弹的速度,这么短的墓道,2、3秒之内,子弹就能完全走完,没有任何地机关陷阱可以在这么短的时间内发挥作用。

但是如果这里的情况真的超出了常理可以解释的范围,进入玄学的范围了,那么子弹就会像我们一样,在笔直的墓道中超越空间而180度转向。

简单而漂亮,非常符合科学精神,我实在有点惭愧为什么我这个大学生想不出这种办法来。

不过一想,这一招也只有他这样的人才能想的出来,这是最简单的逻辑思维。

要判断是不是有错觉的影响,就要我不会受错觉的影响的东西,要找东西就要就近找,三段式一考虑,马上就出来了这个办法,也并不复杂。我突然就感觉到了,汪藏海可能遇到对手了,像他这么处心积虑的人,可能就怕胖子这种单板的思考方法,任何诡计都会被最简单化。

胖子说做就做,我们跟了过去,他走到墓道理,拉上枪诠,就想对着墓道开枪。

我忙大叫:“等等!”

“怎么了?”他问道。

“不要这样。”我道,“如果,我是说如果,这里真的邪门到那种地步,那你开枪出去,几乎是一瞬间,自己就会中弹。”

胖子的脸色变了变,显然他刚才认为其实1和2的可能性很大,根本没有考虑到3和4会不会是真的,不过给我一说他就点了点头,把枪往边上挪了挪,子弹是抛物线,子弹如果射回来,应该落在枪口偏下的地方。

我们全都躲在门口,还没做好心理准备,胖子突然就开枪了,砰一声巨响在墓道里炸起,接着是一连串回音,但是几乎就是同时,我们看到墓门剧烈一抖,炸起了一连串灰尘。

我脑子就嗡的一声,心说不妙,忙探头一看,胖子僵直的还是维持开枪的姿势,但是他的枪下边五六公分的地方的门上,出现一个弹孔,炸起的烟雾还没有散紧。

\chapter{倒斗和量子力学}

到藏宝墓室中作下,气氛和刚慈爱就完全不同了,所有都不说话,脸色也不知道是白还是绿,无烟炉的反射出的黄色光竟然让我感觉到十分的厌恶。

没有人再提出任何问题出来,大家都是一副沉思的样子,但是我知道他们都和我一样,脑子里绝对是一片空白。

事情以及功能超出了我们的控制,甚至我认为这是机关的假设,现在也不存在了,我们进入到了一种无法言喻的状态中去。任何科学的推理经过了这么一个简单的试验,宣告完全失效。

因为没有任何人的力量,能够使得一颗子弹,能在几秒的瞬间,转如此巨大的一个弯。要用科学俩解释这种现象,恐怕般出量子力学都不一定摆得平。

“这是真的鬼打墙!”顺子的脸色季度难看,有看向防在一边的句子的父亲,露除了十分悲切又恐惧的表情。

我知道他此时想到了什么,他也明白了,那几具珠宝中的干尸,连上为什么会有如此绝望的神情,在这样的境地下,一次又一次的尝试,一次又一次的回到起点,知道弹尽粮绝,如何能不绝望,恐怕他们四的时候已经万念俱灰,仍旧没有琢磨出一点眉目。

而我们,可能就是下一批,很快这里就会多出四具干瘪的实体,同样是一脸深切的绝望,让后面的牺牲着来猜测我们死前所想。

我之前之所以没有绝望,没有想到这一步,是因为我认为一自己的智慧,只要是机关陷阱之类智力的东西,我就一定不会困住,但是现在事情已经不同了,显然我们面临的情况,要诡异的多的多。

“要不要继续?”静了大概十几分钟,一边的潘子用干涩的声音问。

没有人回答。不过几个人的目光都投向了胖子的方向。

胖子面前的地面上还剩下两个我们的假设,第三个是我随口胡说的想法:空间折叠。

我刚才之所以突然提出这一点,是我刚才突然想起在火山缝隙地时候,闷油瓶曾经在我面前消失过几秒钟。我当时百思不得其解,现在想来,也许真的和空间折叠有关系,因为刚才的试验,实在太可怕,简直是一种伪科学试验。一下子,我的玄之有玄的空间折叠,变成了最有可能的解释。

如果不是胖子把这些东西列了出来,我恐怕看到这依次试验之后,肯定慌得什么都忘了。

沉默了很久,胖子才道:“好吧,咱们都亲眼看到了。就不说什么废话了,咱们怎么来证明第三条。”

“不!不用证明。”突然一边的潘子又说话了。

潘子看问题非常的透彻,总是能够直接看到好似请的本质,就象刚才胖子还奢望墓道会出现。潘子立即完全否定一样。这和潘子是从战场上下来的也有关系,他思考问题是不带一丝侥幸心里的,所以我一听他说话,就很害怕,怕他说出很多事实但是不应该说出的话来。

只听他道:“这里只有具尸体。我们假设一共进来的是8个人,那有6个人必然是出去了。虽然不知道他门是怎么出去的,但是如果是象小三爷说的第三条,绝对是一个人也出不去,所以我们不用考虑,考虑第三条就等于承认自己死定了。”这话说的几人都全身发凉,胖子就坑议道:“你怎么能确定进来的是8个人,说不定进来的时候就只有6个了呢。”

潘子叹了口气道:“胖子,你还不明白,他们进来几个人其实不重要。”

这就无法证明了,吵也没有用处,我心道:“现在他们到底进来几个人对我们的处境是一点也不重要,但是对于我们的志气非常重要,如果有来年个个人成功的出去了,那我们的心境就完全不同,我们就可以思考他们出去的方法。至少还有一点希望。”

不过刚才看笔记本的时候是粗略翻了翻,没有大篇幅的文字,小篇幅的文字又多是记账或者是短小的信息,看不出什么名堂来。

我琢磨着这些人死到临头的时候,还会不会写东西呢,也许他们临死的时候,恐怕连等都没有了,电池早就耗尽,也没有取暖的东西,所以他们才会在黑暗中蜷缩成一团挤在一起。那如果是八个人进来,那最后两个人会是在什么时候出去的呢?肯定不会是在他们清醒的时候,如果是那样的花,其他人也应该能出,那难道是他们已经饿得神智不清,且没有灯光,一片漆黑的时候?所以走了两个人其他人不知道?

那走出去的关键,难是黑暗,不用灯走?

想着我就一片寒意,想起这里是古墓,如果是在黑暗中,走古墓中如此狭长的墓道,这真是要了任命了。

其他人看我来找资料,也围了过来,开始帮忙找起来,老是作在那里空想总不是办法,有时候也需要看点东西刺激一下。

我想着最后没有光的事情,就让他们不要浪费了,把手电都关了,声下取暖的炉子也可以照明,我们围在炉子面前,三本笔记和一本小说,每个人翻了开来,逐字逐句的找起了线索。

我翻的这一本笔记本里面字体娟秀,应该是一个女人写的,反了好几页。写的都是人名和电话号码,后面还有请客吃饭的名单,还有长白山旅馆的电话,我看到在1994年的时候,好像这个女人还生过病,住过院,这里写着要复诊。

再往后翻就是白纸了,但我还是一页一页的反,希望她能写点什么,正翻着,一边的胖子道:“这里有一条线索。”说着就念道:“今天,卖掉了从海里带出来的最后一件东西,拿了3000块钱,1500还给老李,欠款还清,和着这家伙是打渔的。”

我苦笑摇头,再去看一边的潘子,他的笔记最薄几乎什么都没有,已经看完了,又去看顺子,只见他正津津有味的看着小说,显然是跳到主人公走前最激情的那一页去看了。

胖子看了不爽,一下就抢了过来,骂到:“让你找线索,你看黄书,你的良心,大大的坏了!充公!”

一抢之下,突然小说就散了架了,一边打开手电去拣,突然潘子就道:“唉,这里有张照片。”说着从纸里拾起一张发黄的黑白照片出来。

我接过来一看,突然觉得眼熟,再一看,顿时脑子就嗡了一声,几乎背过去气——这照片不是其他,正是三书他们去西沙之前,在码头的合照!

\chapter{来自海底的人们}

我身上还有着内伤,如今一看之下,几乎就一口血喷出来,把其他几个人吓了一大跳,潘子他们没见过这张照片,虽然听我提过,但是看到了并不认识,所以觉得很奇怪,胖子忙给我顺血,问我怎么回事情。

我发着抖拿起照片,把照片上的闷油瓶和三叔指给他们看,一看之下,另外几个人顿时脸色别我还要难看,谁也说不出话来。

我们简直是不敢相信这是真的,转头看着一边的几具干尸,心里乱成了一团。

这张照片不会出现在无关人等身上,难道这十年前进入长白山,给困死在这里的神秘的队伍,竟然就是海底的那一帮人?这几具干尸,就是文锦和李四地他们?

我发着抖翻转照片,看到后面还有一行模糊的字:西沙考古队,李四地留念。

看来是没错了,要说是其他人带着这张照片来到这里,实在是不太可能,带着这种留念照片的,应该就是当事人……难怪三叔怎么找也找不到他们,原来早就死在了这里!

看着的服装,的确是吻合,还有这照片,但是这些人为什么要来这里呢?难道也在海底墓穴中发现了什么东西,给吸引到长白山了?

等等,不对啊,我突然想到了三叔,想到闷油瓶,天哪,几乎海底墓穴中的所有人,现在都在云顶天宫中了,这帮人十年前就来了,而三叔也闷油瓶也在最近赶到,他们到底为什么非来这里不可?

我心中那些已经给我淡忘的谜团顿时复活了起来,无数的问题涌向我都大脑。

潘子他们不知道三叔的往事,看到照片的震惊程度,还在我之上,我只好又耐心地解释了一遍。听的其他几个人目瞪口呆,胖子道:“不会吧,等等,我想到更多,似乎去到海底墓穴的所有人,包括阿宁,还有我们,也都到这里来了,难道海底墓穴中有一个诅咒。只要是到了那里的人必须爬长白山……不对,好像说不通?”

胖子当然是胡说,但是我却感觉不寒而栗,心中有一些东西也明朗化了,看来海地墓穴倒不是关键,关键是在这里。海底墓似乎只是一个跳板而已……

我翻找了尸体上所能找到的一切,但是再无任何线索,这些人谁是谁,我也搞不清楚,我心乱如麻,昏头转向的就往墓道里走去,连手电也都没有拿。

胖子忙拉住我让我冷静。说急也没用,这些人还不是困死在这里,你四了倒是可以问问他们的灵魂是怎么回事情,但是那时候已经晚了。

我给拉住按坐下来喘气,逐渐安静了下来,心里只剩下了一个念头:“我一定要出去,我一定要找到三叔问个明白,不然我死也不会闭眼的。”

胖子道:“可是到现在还没有找到任何线索证明他们之间有人成功出去了,搞不好这里根本就出不去,是一个封闭的空间。你就算闭眼也没有用。”

胖子的话一说,其他人就无话可说,本来我们是想在这些尸体身上找到线索,一下子却发现了如此大的一个秘密,真是一波未平一波又起。

大家都在考虑自己的事情,气氛差到了极点。我脑子昏昏沉沉,根本不敢再去看那张照片,恐怕其中会有神秘怪物把我吸进去,喉咙也开始痒了起来,似乎感冒了,开始咳嗽起来,又咳出了血。

潘子看我这样,对我们道:“今天先休息吧,反正一时半会儿也出不去,不如好好地睡一觉,这样脑子更清醒,小三爷你也不要想太多事情了,我知道你心里的疑团太多了,但是要弄清也不是一时半会儿的事情。”

我摆手,怎么睡得着,还不如在这里继续想,想到实在坚持不住了,才能睡着,不然只能越睡越累。

胖子也不知道在抽第几根烟了,一边抽一边喃喃道:“其实,我想起来,早知道刚才就不要按那个记号走了,听我哦大多好,一帮人困住了,另一帮人还能想办法……那记号,现在想起来倒可能是这几具尸体留了下地了吧,你看,事情都赶巧了。也许他们也像我想的一样,分队走了,那两人压根走地就是墓道的另一边。”

我摇头说不会,一帮人被困了,另一帮人回来找,还不是同样中招,到时候更郁闷,而去说不定走没有记号那一边更凶险,不知道有什么等着我们。

不过深入去想又不可能,因为既然已经给困住了,那另一帮人回来的时候,墓道已经变化了,他们无法找到这个墓室了。那几个记号,是不是另一边的幸存者留下的,这里队失踪地记号?

想着想着,忽然我浑身一抖……突然一道闪电从我的脑子里闪了过去……记号……

我猛地就坐了起来,对他们道:“我突然想到一个很诡异的破绽,这墓道,是一个悖论!”

“什么?”

我皱眉头想想自己应该怎么说,“我怕你们听不懂,比如说,我们走着出去,在黑暗中,无论什么原因导致了我们这样,我们都必须有一个调转方向的过程,尽管这个过程我们自己一点都不知道,对不对?”

其他几个人点了点头,我继续道:“比如我说,拿着一支笔,在墙上一边划一边往前走,那这出口处的墙上,肯定有会留下一道长长的痕迹,一直跟着我,那等我在无意中调转方向的那一刹那,你们猜会发现什么?”

胖子几乎跳了起来:“你会看到前面的墓道墙壁上,已经有你的划过的痕迹了!”

“不只这样!”我道:“最关键的是什么?就是我转过身之后,左右就发生变化了,那我拿着笔的手,就会在墙壁的另一边开始划道。”

“这!”潘子也皱起眉头道。

“这是逻辑推论。”我道:“也就是说,如果按照逻辑来解释,墓道中间必然会有一个转折点!在转折点上,我们就像走入一面镜子一样,直线走到自己的相反方向,你们承认不承认?”

众人都点头,只要是符合逻辑,就肯定是我说的那样。

我道:“好,那你们再想一下,如果我们这么走过去,真的碰到了我说的那个‘反射面’,那么这个反射面有多厚?”

“多厚?”几个人还在消化我前面的话,一头雾水。

“是啊,肯定会有一个厚度,如果没有厚度,那么,你身体前一半通过的时候,你身体的后一半,就会……”

潘子瞬间就理解了我的意思,一下子冒出了一身的冷汗,下意识的接口道:“互相重叠!”

“对!因为在那个位置上,你的前半部分已经给反射回来,但是你的后半部分又没有通过‘镜面’,所以,如果我的说法是正确的,那我们在通过反射‘镜’的同时,必死无疑!会变成一陀怪物!我的脸会撞到你的后脑勺!”

“可是,我们走了这么多次,都没有死啊?”胖子奇怪道。

“这就是我要说的,这个镜子面肯定有一个远大于人的厚度,一个反射的过度段,我们走入这一段之后,从这一头进去,在里面行走一段距离后,再从另一头出来,完成了空间的折叠。”

众人又点头表示同意,这推论天衣无缝。

“问题是,我们不知道这段距离有多少,我们假定只有两三步路,我举一个例子,比如我们走进了那一段‘镜子空间’之中,但是胖子不走进去,而是呆在镜子空间之外,而镜子空间只有两三步,你前后两边都能看到,你猜会发生什么事情?”

潘子理解的最快,喉咙几乎都僵直了,“会……看到前后出现了两个同样的胖子。”

“好,这里出现了一个悖论,在你后面的胖子,往你前面看的时候,能不能看到你前面的那个胖子呢?又或者你去牵其中一个胖子的手,会发生什么事情?”

潘子赶紧做了个打住的手势:“别……别说了!”

“这说明什么?”一边的胖子也是脸色惨白。

“我们不用继续试验,也可以确定,这个所谓的‘镜子空间’,是不存在的!而去这个墓道反射,怎么也走不出去的逻辑基础也是不存在的,这个墓道的存在是不符合逻辑的。”我压低了声音:“汪藏海不是神,他不可能自己创造物理规则,这里的机关,和汪藏海无关,这些人也不是因为这个而困死的,我们现在面临的情况,是一个特例,是一种新的情况!我们给这些尸体误导了,而最可能造成我们这种状况的,似乎只有一个可能性了……”

我把手指小心翼翼地指到了胖子写的第四条上去,动了动嘴巴,用唇语道:“我们身边有鬼!”

\chapter{犀照}

现在想想,当时如此一本正经的说出这几个字,又用唇语来说,怕那鬼听到,说明自己的神经已经给折磨成什么样子了也不知道了,要是平时,或者压力再小一点的时候,根本就不可能有这种想法。

这其实也是必然的,我们几个花了多少时间,经历了多少事情,才到达这一步,却陷入这种没有原因可找的绝境,且不说前路漫漫,且不说怎么回去,眼前的事情就已经使的我们思维堵塞,很多问题都想不到看不到了。

事后去想的时候,其实还能想出很多办法出来,比如说拿着指北针,看着指数的变化去走那条墓道,只要我们发生反转的一刹那,指南针的指针就绝对会移动,等等,但是当时脑子里除了几个固定的思维之外,简直是一片空白,以致于竟然会把可能性指到鬼打墙上面来,而且当时一点也不觉得可笑和荒谬、甚至有点悚然的感觉。

胖子、顺子他们比我还不如,此时完全给我的表情所感染,几乎一个一个脸色发白,咽了口唾沫,胖子也用唇语说道:“你确定吗?我早说嘛……那现在怎么办?”

我心中当时的想法是,这条墓道的逻辑基础是不成立的,那么形成这种现象的原因必然和逻辑无关,但是如果不是做梦的话,其他的东西都无法逃脱逻辑的束缚,也就是说我们现在看到的,或者听到的,很可能都是假象。那么我们周围是什么景象就很难说了,而能够让四个人同时产生假象的,我认为只有“恶鬼”的力量,只有“恶鬼”才可以不讲逻辑,才可以毫无破绽的把人困成这样的地步。

这里恶鬼其实只是一个比较让人明白的代意词,泛指一切我们无法理解的力量,这种力量是显然是必然存在的了。

但是如果真的有“鬼”的话,我们又变地束手无策,因为我们根本看不到他,自然也无法去对付他,就算我们去骂,或者随便用什么方法都好,都对他们一点用也没有,这样就变成了我最讨厌的情况之二,明知道问题出在我们四周,我们却对付不了,无处着力。

当时还有一个很幼稚的想法,而且也不知道这种力量是什么类别的,如果是无意识地就麻烦了,他自己没有思维,就算我们用计都没用,只有硬碰硬找到它才行,如果是冤鬼就好办了,他能够思考,我们就可以将他逼出来,逼他犯一些错误。

我和他们考虑再三,胖子就一口咬定,感觉这鬼很有可能就是我们面前这几具干尸中的一具,可能这里有人的魂魄放不下凡尘俗事,还在这里游荡,看到有人来陪,自然想作弄一番,但是又不知道是哪一具。

胖子先排除顺子的父亲,老爹十年不见儿子,自然不会拿儿子的命来开玩笑,那就是另外的六具。

我此时已经有点感觉自己荒唐了,不过我们已经走投无路了,什么事情都要尝试一下。于是我走到尸体之前、让他们都跪下,然后用废指折了几个金元宝,给他们每人烧,一边烧我就一边磕头:“我是吴三省的侄子,我找我三叔有急事,你们哪位在施法,请笑纳纸钱之后就放过我们吧,我们真的赶时间,要不留下这个胖子陪你们玩,其他人放我们出去。”

胖子一听大怒,潘子和顺子马上一人一个挟持住他,不让他动弹,胖子大骂:“吴邪,这你卑鄙小人,老子咬死你!”

我念完之后,四处看了看,四周一点变化都没有,尸体也没有变化,意识到没用,挥手让他们放开胖子,胖子紧张的瞪着四周,也发现什么变化也没有,不由就冷笑:“你者,鬼大叔还是公平的,看不上你这几个臭钱。”

我道:“也许人家看不上你呢,真是的。”

顺子这时候在一边道:“不对,咱们是不是应该怎么想,你看我父亲在,就算有人对我们不利,我父亲也会帮忙的,如今没用,是不是作恶的不是这几个人?”

如果平时,如此幼稚的话我肯定已经笑出来了,可是现在我却听的一本正经,还去考虑他的可能性,考虑之后,我道:“说不定你父亲已经走了,或者作恶的不只一个,他打不过。不过我也感觉可能不是这里的几个,这些人都是成年人了,而且和我三叔关系都不错,我想不会做恶作剧,搞这种花样的,可能是小鬼,尸体并不在这里。”

说是这么说,可是如果真地是我说的那样,就难办了,因为我们看不到这鬼在哪里,说不定就趴在我们背上,我们都不知道,看不到就无从下手。想着我就叹了口气,问:“你们谁有什么办法,偏方也行,有能看到鬼的没有?”

潘子道:“我听说只要在眼晴上涂上牛的眼泪,就能看到鬼了。”

胖子打了个哈哈:“那寻找牛的任务,就托付给你了。”

“不,也许不需要牛的眼泪,也能看到。”我突然想起了一个办法。“但是要胖子牺牲一下。”

胖子一下又紧张起来,“你该不是想杀了我,让我的灵魂去和鬼谈判,我可不干,要是你们把我杀了,我肯定和那鬼合谋,把你们整的更惨!”

这家伙倒是又想出了一个办法,我大怒,“你想到哪里去了,我要你的摸金符用一下。”

“你想干什么?”胖子捂住胸口:“这可是真货,弄坏了你陪的起吗。”

“摸金符是天下最辟邪的东西,要是真货,咱们怎么会落到如此田地,我刚才已经看过了,这东西是假的。”我道,“快拿来给我。”

“假的?”胖子摘下来仔细看了看:“你确定?”

“当然,这是犀牛角做的,老子是专门做这一行的,能不知道?你看,穿山甲的摸金符是越带越黑,你自己看你的犀牛角,已经开始发绿了,我不会骗你的。”

“妈的!我说怎么这么倒霉!”胖子大怒:“那鬼儿子又他娘的晃点了我一次,难怪每次都不灵,胖爷我这次要是有命出去,不把他那铺子给拆了,我就不姓王。”

我从胖子手里接过他的摸金符,安慰了他几句,他又问我打算怎么用?是不是用来按在尸体的脑门上。

我道:“自古有一个传说,叫做‘犀照通灵’,你听说过没有?”

胖子不解道:“该不是前几年放的香港片子?”

“差不多,就是那个意思。”我点头:“只要烧了这个东西,用这个光,你就能看到鬼了,当然我也没试验过,不知道是不是真的。”

我当时自己都觉得自己荒唐的要命,不过牛眼泪都拿出来说了,犀照有何不可,这也是病急乱投医了,在胖子那5出现之前,我的想法是唯一可行的了,不试也不行。

晋书中曾经有这样的记载:“峤旋于武昌,至牛渚矶,水深不可测,其下多怪物,峤遂燃犀角而照之,须臾,见水族覆出,奇形怪状。其夜梦人谓之曰:‘与君幽明道别,同意相照也!’”大意是说:中国古人通过燃烧犀牛角、利用犀角发出的光芒,可以照得见神怪之类。古人的说法总归能有点用吧。

说着我拿出了无烟炉,就将摸金符放到上面焚烧了,一开始还烧不着,后来就有一股奇怪的味道散发出来,绿色的火苗中闪烁出奇异的光亮。

我举起这一只无烟炉,举高让它照亮到尽量多的地方,我们都四处转头,寻找四周是不是出现了什么刚才没有的东西。我在墓室中走了一圈,却什么都没有,其他人也都看不到什么。

“也许那鬼躲的远远的。”顺子道。

“不会,传说如果是鬼打墙,鬼是趴在人的背上的。”

我们又看了看各自的背上,仍旧什么都没有,胖子喃喃道:“他娘的我早说传说是不作数的,浪费我的摸金符,就说是假的,那也是犀牛角的啊,结果浪费了也什么都没照出来。”

潘子泄下气来:“看来这一招也没用了,恐怕也没有鬼,咱们碰到的是第五种情况,也就是无理可寻,一点都没有头绪的情况,连一点参考都没有的情况,现在应该怎么办好?这一次恐怕真的要歇菜了。”

我心里叹了口气,刚想说话,突然胖子给我做一个禁声的手势,潘子也做了一个别说话的动作。我眼皮一跳,顺着胖子的眼神抬头一看,只见在我们的上方,墓室的顶上,隐隐出现了一个黑色的“小孩”。

\chapter{出口}

我的血液一下子就结冰了,潘子一手去拿枪,胖子则一点一点把手里的犀照灯举高。

随着犀牛角越烧越亮,那黑色的“小孩”也越来越清晰起来,我仔细一看,这……这不是我们在藏尸阁中看到的那只大头尸胎吗!怎么跟到这里来了?难道它一直跟着我们?

“妈的!原来是这东西在捣鬼!”胖子大吼了一声,“咔嚓”一声就把枪端了起来,无处发泄的怒气顿时就爆发了出来,一连开了几次扫射,顿时把那东四打得黑汁四溅,一下子摔落到地上。

我们马上后退了好几步,尸体发出一种类似于婴儿的尖叫声,猛撞翻了无烟炉,闪电一般向着黑暗中逃去。

“不能让它跑了,不然我们还会中招!”潘子叫道,“追!”

四个人爬起来就狂迫过去,几乎是一瞬间,我们突然看到了外面的墓道壁画已经变成了原来的图案,鬼打墙失效了!

“出来了!”胖子大喜,“不用困死了!”

尸胎跑得飞快,以惊人的速度冲入了墓道的黑暗之中,向墓道的另一头跑去,我们知道自己绝对不能停下来,一旦停下来,百分之百就会重新回到那种境地中去,我真是死也不想再经历一次了,而且也不可能有第二只犀牛角给我烧了。所以四个人儿乎拼了命一样地跟在它后面,竟然设有给它落下。

说时迟那时快,几乎也就是跑了七八分钟的时间,一千米左右的墓道就跑完了,尽头出现在于我们向前,那是一道阶梯,直通向下,尸胎闪电一般冲了下去。

我们狂奔着鱼贯而入,什么机关陷阱都不管,要死就死吧。就算四十人只剩下一个,也要把这东西干掉,以解心头之恨!

几乎是十级并成一级,我们如袋鼠一样狂窜而下,但是我们跑楼梯总归要比跑步慢上半拍,而那尸胎却一点也不减速,几乎一瞬间就消失在了楼梯下的黑暗中。我明知道追上无望了,可是却刹不住车,想停下来,结果左脚绊了右脚,一连几滚就掉到了石阶的尽头,摔得头破血流,手电都飞掉了。

心中暗骂,刚想站起来,却听到枪声从一边传来,而且非常密集,不像是胖子和顺子两把枪能发出的声音。

我爬起来就看到一边传来的光线,但是光线又不强,正想走出去,跑在我后面的潘子和胖子就赶到了。

我奇怪他们怎么跑得这么慢,胖子道,顺子路过十字路口的时候,按原路回去了,他父亲也找到了,也摸到这么多金子,根本不想再跟着我们冒险。他说他在外面的雪山上等我们一个星期,如果一个星期后我们还不出来,他就自己回去了。

我暗骂声这个没良心的,不过他也够了,跟着我们吃了这么多的苦头。这时候胖子也听到了枪声,一下子警觉了起来。

我们用手电照了照四周,发现这墓道另一边楼梯的尽头是一个楼台,外面是几道长廊子,也就是说,这是一个两层的巨大墓室的一个入口,但是两层的墓室之间并没有天花板,而只有几道架空的长廊,在长廊上可以直接看到一下层的景象。

这叫做连天廊,看上去雕龙刻凤,其实是功能性的,是在巨大的墓室中吊人棺椁的设备,看样子外面连天廊的下面可能就是一个棺室了,现密集的枪声正从下面传来,而且外面到处都闪动着手电的光芒。

我们心里奇怪到底发生了什么事情,难道是刚才那只尸胎跳下去了造成的?那这么多枪在扫射,大象也放倒了,还打不中一只尸胎?

三个人排着队去了楼台,外面的连天廊很窄,我们小心翼冀地爬上去,往下一看,发现下面竟然是一个巨大圆形墓室,足有五六百平方米,有点意外的是,阿宁的队伍就在我们廊下,七几只冷烟火扔在四周,把整个墓室照得通明。只见他们围成一圈,不停地用枪在扫射周围的东西,但是我又看不清楚是什么,仔细一看,才发现那都是手臂粗的蚰蜒,满墓室都是,密密麻麻,简直就像海洋一样把阿宁他们围在了中间。

而在墓室的中央,有一个倒金字塔形的棺井,井底有八只巨大的黑棺,围着中间一只半透明的巨型玉石棺椁,玉石棺椁已经被打开了,在下面的冷烟火映照下,玉石棺椁流光溢彩,反射出诡异的光芒。我看到蚰蜒似乎就是从这棺椁之中源源不断地爬出来的。

我心里一个咯噔,心说这难道就是蛇眉铜鱼上记载的九龙抬尸棺?盛殓万奴王的宝匣?看样子这帮外行触动了什么机关,或者干脆就只是踩死了一只蚰蜒。

此时也管不了这么多了,下面的十几个人已经疲于应付,但是蚰蜒潮水一样涌上来,根本就没有用,打死一只其他的就更疯狂。

“我们要不要帮忙?”胖子问我道。

潘于摇头:“等他们再死掉几个。”

胖子笑道:“你不如现在直接扫射他们,死得更快。”

我心里也很矛盾,这倒也不是救不救的问题,问题是救了之后他们会怎么对我们。阿宁在海底墓穴中就要置我们于死地,我们命大才侥幸逃过,而我之前也救过她,不见得她买了我的面子,不过不救,看着如此多的人全部在我们眼皮底下死去,我恐怕要内疚一辈子。

另外就是救不救得了的事情,我们在上面开枪于事无补,要救他们只有用绳索将他们拉上来,但是他们现在全力扫射才勉强能够全身而退,绳子一垂一停,下面肯定有人伤亡。

正在犹豫的时候,突然我就看到在阿宁的队伍中,有一个老外正背着一个人,看上去非常的面熟,我马上拍了拍潘子,指给他。一指之下,他顿时就惊叫了一声:“那是三叔!”

“你确定?”我也看着像,但是自己不敢确定,潘子一说我心里就更觉得像了,忙往这个人上方走近了几步,想仔细去看。

没想到才走了一步,我的脚就感觉不对劲,低头一看,只见刚才逃下来的那只尸胎,竟然吊在石廊的下方,正好我就这么巧,走到了它的上面,它干枯的手一下就抓住了我的脚,用力地往下拽。

我心里大怒,心说这东西肯定就是记上仇了,老是找我们的麻烦,但是人在石廊上,我的平衡感又差,被它一拉,我的人就站不稳,顿时趴在了廊子上。

潘于和胖子同时举枪,这家伙真是不长记性,这么近的距离顿时脑袋就给打烂了,大脑袋只剩下一半,接着抓着廊子下部的爪子就脱手了,整只尸胎摔入了廊下,同时拽着我的脚。

我被这么重的东西一拉,惨叫了一声,也摔了下去,接着尸胎就先落在了阿宁他们的人群中,其他人早就全神贯注边上的蚰蜒,哪里顾得上头上,顿时就吓得屁滚尿流,四散摔倒,接着我也从空中落了下去。

后来据胖子说,我落下去的动作就似乎是自己跳下来的一样。但是我确实是不得已摔下去,接着我就狠狠踩在那只尸胎已经打烂的脑袋上,顿时黑血四溅。

幸亏这石廊不算太高,不然我这样硬生生摔下去,肯定得崴脚。但是摔下去之后我只是一个轻微的趔趄就站住了,向四周一看,顿时发现四周的蚰蜒像见了鬼一样地四处逃窜。一瞬间,潮水一样的蚰蜒潮水一样地退去,很快地上只剩下了蚰蜒的尸体。

我吓得够戗,好久才回过神来,也不明白发生了什么事,抬头一看,却见所有的人都看着我,脸上满是惊骇的表情,好像看到了什么怪物一样。

\chapter{闷油瓶第二}

我坐在自己的背包上,阿宁队伍中的医生帮我包扎了伤口——我手上的伤特别严重,缝了三针才算缝合了起来,这是被尸胎从石梁上拽下来的时候割破的。我自幼虽然不是娇生惯养,但是也没有做过什么粗重活儿,所以这样的磕磕碰碰就很容易受伤,换成潘子恐怕就不会有什么事。医生给我消了毒,让我不要碰水,也不要用这手去做任何的事情了,我点点头谢了谢他,他就去照看别人。

从石廊上掉下来之后,阿宁他们对于我这种“出场方式”吃惊到了极点。阿宁一开始竟然还没有认出我来(事实上我当时蓬头垢面,她最后能认出是我已经很了不起了),直到胖子在石梁上招呼他们一声,她才反应过来,更是惊讶得说不出话来,还用一种不可置信的眼神看着我。

两帮人僵立了很久,才逐渐有所反应,我走动了一下,着急想看看那人背的是不是我的三叔,可是我一动,围着我的人突然就全部自动后退了好几步,好像见了鬼一样,有几个还条件反射地又端起了枪。

胖子和潘子在横梁上刚松了口气,一看只好又迅速把枪端了起来,我赶紧举起双手表示自己没有敌意,阿宁也忙挥了挥手,对她的手下道:“自己人,合作过,放下枪。”直说了好几遍,她的手下才将信将疑地把枪口放下来,但是几个老外还是非常的紧张,眼睛死死盯着我。

我看到他们脸上的筋都鼓得老高,显然情绪已经受到强烈的刺激,再有一点惊吓,这些人可能就会崩溃了,于是也不敢再有什么动作,就站在原地不知道怎么办。

阿宁皱着眉头,从她的表情看,显然是不知道我们也在这里,抬头问我:“你们……怎么会在这里……”

胖子在上边嘿嘿一笑:“这叫白娘子找对象,有缘的千里来相会,无缘的脱光了搂在一起还嫌对方毛糙——我说我们路过你信吗?”

胖子说着和潘子从石梁上跳了下来。这时候阿宁队伍中有几个人显然认出了胖子,都惊讶地叫了起来,显然胖子在这里出现,触动了他们某些糟糕的记忆。

胖子走到我们面前,大概是因为他和这些人合作过,气氛这才稍微缓和和下来,几个神经绷紧的人这才松了口气,放下枪上的保险咒骂,有个人还自言自语:“这下好了,在糟糕的地方碰上了糟糕的人。”

我想起第一次遇见胖子的情景,感觉这一句话还真是贴切,不由就想笑。

胖子瞪了那人一眼,又和其他几个可能比较熟悉的人打了招呼,阿宁还想问他问题,我和潘子已经忍不住了,就跑向那背着人的老外那里,翻看他背着的人,看看到底是不是三叔。

老外似乎对我非常顾及,我跑过去他们都远远走开,那背人的老外倒似乎不怕,看到我的目标是他背上的人,便将人放到了地上,我上去急急地翻开他头上的登山帽。

登山帽中是一张十分憔悴、胡子邋遢的脸,我几乎没认出来,只觉得像是三叔,仔细一看之下,我才“哎呀”了一声,几乎没吼出来。

果然真的是失踪多时的三叔,那个老贼!只几个月不见,这老浑蛋竟然似乎老了十多岁,头发都斑白了,乍一看根本就无法认出来。

这样的见面说实话我真的没有做好心里准备,我认为我最后会在一间墓室中见到三叔,然后三叔会说给我一切,或者在我危险的时候,他会出现来搭救我……但是他竟然就这样马马虎虎地突然出现在了阿宁的队伍里,我看着真切,却突然不相信起来。

我真的又看到三叔了?我找到他了?我僵在那里不知道该作什么反应,也不知道自己是在做梦还是产生了幻觉。

三叔似乎神志不太清楚,眯着眼睛,也不知道能否看见我,但是我看现他所到我叫的时候,突然浑身有一丝轻微的反应,干裂的嘴唇微微动了一下,好像在问:“大侄子?”但是随即就没有动静了。

我突然心里一酸,一种无法言语的感觉涌了上来,看到这老家伙平安,我顿时放下了心来,那种没了主心骨的焦躁的感觉顿时消失了,可是又有一股极度的愤怒涌了上来,想上去把他推倒狠揍一顿。两种感觉混合在一起,脸上不知道出现了什么表情,但肯定十分好笑。

一边的胖子不知道和阿宁在说些什么,似乎吵了起来,我也无暇顾及了。潘子看着三叔这个样子,上去就摇了他好几下,又解开他的衣服。我一看就蒙了,只见三叔的衣服里面竟然全是黏浓,仔细一看,他的胸口都是烂疮,无数的硬头蚰蜒挤在了他的皮肤之下,显然三叔想把它们扯出来,但是蚰蜒的尾巴一碰就断,蚰蜒就断在了里面,伤口也不会愈合,时间一久全部化脓了。

潘子一把就扯住边上的老外、就要揍他,被其他人抱了起来。潘子一边挣扎一边大叫:“你们他娘的对三爷做了什么!竟然把他搞成这个样子?”

我看着那老外看到伤口的惊骇表情,知道他们肯定也是不知情,但是三叔这样子也太惨了,我发着抖问那老外道:“是在什么地方找到他的?他怎么会这个样子?”

那老外几乎要吐了,转头过去道:“就是在这里的棺井下面,我们刚发现他,还以为他已经死了,后来发现他还活着,领队说这老头知道很多事情,一定要带着他走——我不知道他身上有这些东西,不然我死也不会背他!”

“一定是你们!”潘子在一边大怒,“老子在越南见过,那些越南人审问犯人就是用这一招,就是从你们美国人那里学来的,你们他娘的肯定逼问过三爷,老子杀了你们!”

其他人都围在我们的四周了,我摆了摆手让潘子冷静一点,道:“和他们没关系,如果是他们干的,他们不会不知道死蚰蜒会吸引同类而这么惊慌。”

阿宁走过来一看,也倒吸了一口冷气,马上招来了队医,几个人手忙脚乱地把三叔弄正了。就在这个时候,我突然感觉三叔偷偷地往我的口袋里放了什么东西,动作很快,一瞬间我感觉口袋动了一下,我呆了一下,心中一动。

一瞬间我的脑子嗡的一声,马上知道了:三叔可能是清醒的!心里顿时一惊又一安,惊的是他假装昏迷,不知道有什么目的;安的是,能做这种小动作,说明这老家伙死期还不近。我用眼角一看四周,其他人都被他的伤口震到了,没有注意到,于是不动声色地继续扶着他,但是手用力捏了捏他的肩膀,表示自己知道口袋里有东西了。

三叔的眼神又涣散起来,队医用酒精给他擦了伤口,然后用烧过的军刀划开皮肤,用镊子将里面的蚰蜒夹出来,再放出脓水。因为这里太冷了,很容易结冰,我和潘子就打起无烟炉,不停地烘烤三叔。

伤口一共有十六处,有几只蚰蜒拉出来的时候还是活的,直接扔进火里烧死,最后把伤口缝合起来。潘子全神贯注地看着整个过程,我想给他打眼色都不行,我心里有事,但是这样的情况我突然走开也是不妥当,想知道我口袋三叔到底放了什么东西,只有硬等着。

好不容易所有的问题都处理好了,队医给他盖上了毯子,让他睡在一边,潘子就问他怎么样了,队医叹了口气道:“我能做的都做了,现在他是伤口感染,我等一下给他打一针抗生素,但是他现在已经有点高烧了,我不知道能不能撑到出去,要看他的个人意志,你们不要去吵他,让他睡觉。”我这才有借口将潘子拉开,这时一动才发现自己滚下来的时候也是浑身是伤,竟然站不起来。

队医给我也包扎好伤口后,就去看其他人,阿宁的队伍大概有十六七个人,冷烟火都逐渐熄灭了,四周黑得过分,实在数不清楚,胖子又被阿宁拉在一边不停地在说着什么,我也看不清那里的情形。我想拉着潘子到个没人的地方,但是潘子竟然有点懵了,只顾着坐在三叔的边上,有点反应不过来。

我心里实在恼火,关键时候一点忙也帮不上,只好自己想办法避开四周的人。阿宁的队伍分成了两批人,一批受伤的休整,一批下到棺井之下,这些人似乎对我没有恶意,这可能和胖子与这些人都认识有关系。但是可能因为我刚才震退蚰蜒的关系,我走到哪里,他们都用一种奇怪的眼光来打量我,这圆形的墓室又是如此之空旷,实在没有地方能让我躲。

我心一横,就走到被我踩烂的胎尸那里,假装蹲下去看它,这才没人围上来看我。

尸胎就像一只巨大的虾蛄,五官都被我踩得模糊了,一看我就头皮发麻,但是也管不了这么多,掏出口袋里的东西一看,竟然是一张小纸条。回头看了看没人在身后,我就紧张地展开一看,里面写了几行字,一看我就惊讶了一声,这些字的前半部分不是三叔的笔迹,看写字的形体,竟然好像是闷油瓶写的,上面写的是:

我下去了。

到此为止,你们快回去,再往下走,已经不是你们能应付的地方。

你们想知道的一切,都在蛇眉铜鱼里。

署名更是让我吃了一惊,竟然就是我们看到的那个奇怪的符号……这果然是闷油瓶留下的,这到底是什么意思呢?

再下面才是三叔非常潦草的文字,看样子竟然是用指甲刻出来,但是还算清晰,只写了一行。

我们离真相只有一步了,把铜鱼给阿宁下面的乌老四,让他破译出来,没关系,最关键的东西在我这里,他们不敢拿我们怎么样。

显然三叔到了这里的时候,肯定在什么地方发现了闷油瓶的这张纸条,而且这张纸条肯定是写给我们的,闷油瓶看来想阻止我们下去,看字条里的意思,似乎还有什么通道,他去了一个十分危险的地方。而三叔显然不领情,这真是要命了,这老家伙到底想干什么?到底三叔那里还有什么关键的东西?闷油瓶既然不想我们下去,那记号是留给谁的?难道是留给自己的?

我的脑子顿时神游天外,其实这一段时间我感觉越来越多的眉目出现了,但是因为之前的谜团都太杂乱,所以一旦有新的想法就特别的混乱。

我想到海底墓穴中的标记,闷油瓶看到这个,才知道自己来过那里,如今他刻下记号,难道……他知道自己会丧失记忆?所以事先留下了自己的记号,以便下一次到来的时候,能够凭借记号想起来?

太乱了,我的头又开始疼起来。这时候,阿宁和胖子向我招呼了一声,我被吓了一跳,回头一看,他们正在让我过去,于是索性不想了,把纸条一折,塞回口袋里,就走了过去。

阿宁给我递了壶水,我喝了一口,她道:“我和王先生谈了一下,我们正式准备合作,你怎么看?”

合作?我看到她紧身衣服里面的胸形,想起了在船上的事情,有点不敢正视,想起闷油瓶的警告和三叔的话,一下子真不知道怎么说好。

找到了三叔,我心里一安,这一安中也有自私的成分在,就是可以出去了,其实我心里所想的还是自己能够摆脱这个地方。但是正如三叔说的,我们似乎离真相非常近了,看样子三叔自己也有谜题,如此救他出去,说不定他自己也是一问三不知。如果我们能够忘记还好,如果不行,以三叔的性格,必然还要再来一次,我能坐视不理吗?

想了想,我还是咬了咬牙,道:“怎么合作法?你说说看,说实在话,和你合作我真的要考虑考虑。”

她看到我的样子,笑着摇了摇头:“那个,在岛上来不及向你们道别了,现在谢谢你救了我,我在海里……那是有苦衷的,我没想过要害你们。”

我想起海底墓里的事情,叹了口气,心说鬼才信你。我点上一支烟道:“真想合作的话,就告诉我是怎么一回事,你们在海底到底要找什么东西?你们来这里又是干什么?”

胖子在一边道:“对,大家坦荡荡的才好做事情。”

阿宁露出了惊讶的表情:“你不知道,你三叔没有把事情告诉你吗?你们……什么都不知道就这样拼了命地乱跑?”

我苦笑了一声,心说要是三叔把事情告诉了我,我才不理他的死活呢,摇了摇头:“他没说,我一直是个无头苍蝇。”

阿宁皱起秀眉看着我,看了很久,似乎发现我没在说谎,道:“难怪,我一直以为你是个特别厉害的角色,一点也看不出你在撒谎的样子,原来你的确什么都不知道。”

我这个时候突然感觉有点异样,为什么这女人突然来找我们合作?他们这么多人,兵多粮足,我们只有三个人,何必与我们合作呢?就算是因为我能够震退蚰蜒,大不了绑我就行了。难道——我看了看四周——他们的处境不妙,或者有什么不得已的理由吗?

阿宁看我的表情,大概猜出了我的想法,也不点破,叹了气:“其实,我们这些小角色知道的也不多,只不过给老板卖命而已。”说着让我们坐下,招呼了另外一个老外过来,阿宁给我介绍,说是这老外叫柯克,是汉学专家,专攻的就是东夏,整件事情他知道得最多,可以问他。那老外和我握了握手,道:“本来我们是严格保密的,但是现在这种情况……你想问什么,就问吧。”

我心里“咯噔”了一下。

他继续说道:“很遗憾,关于我们老板的目的,我无法告诉你,说实在话,我也是个领队而已,我和阿宁只知道我们需要进入一个地方,拿一件东西出来,然后就完成了,具体高层要这些做什么,我真的不知道,所以我们在海底墓的目标可以说一共有两个,一个是一只玉玺,你们中国人把它叫做鬼玺,听说可以召唤阴间的军队;另外就是这里地宫的机构图,可惜的是,我们都没有弄到手,最后还是我们阿宁出马,才拿回来应该得到一些东西。”

“鬼玺?”我听了几乎跳了起来,“你是说鲁殇王的鬼玺?在海底墓穴中?”听到我们说起了鬼玺,胖子也挺感兴趣,凑了过来,阿宁似乎很厌恶胖子,但也没有办法。

那个柯克点头道:“是的,相信你们也知道了一些吧,鲁殇王陵被汪藏海盗掘了之后,后者用蛇眉铜鱼替换了鬼玺,我们一直以为鬼玺被他拿到自己的坟墓里去了,但是却怎么也找不到;而那天宫的机构图,恐怕就是落在了你们的三叔手里。我们到现在都不知道被这只老狐狸摆了多少道了,但还是得和他合作,他的情报比我们准确得多。”

我点头苦笑,这个我也深有感触。那胖子在一边道:“那你说阿宁和我们去海底的那一次,她带出来了什么东西?”

柯克张嘴就想说,阿宁却拦住了他,对他道:“该说的说,不该说的你别多话。”

胖子怒道:“你这是什么意思?”

柯克却似乎不太领阿宁的情,大笑一下,道:“你就算现在不告诉他们,总归还是要拿出来的,况且你现在就算有这些东西也没有用。”

阿宁看了我们一眼,跺了一下脚,似乎很不甘心:“我千卑万苦弄出来的东西,真是便宜你们了。”

我这个时候感觉非常奇怪,阿宁他们怎么这么合作,后来和三叔聊起这个事情,三叔就说那个时候其实阿宁他们已经走投无路了。她除了和你合作别无其他办法,因为他们到底是业余的,就算技术设备再好,也比不上我这个半桶水的土夫子。但是她又非常聪明,她其实已经巴不得把所有的事情都告诉你,但还是一点一点和你抠,想从你嘴巴里也抠出一点东西来交换,这就叫老江湖。幸亏我有意什么都没告诉你,不然你肯定给她全套去,那三叔我的计划就全完蛋了。

柯克道:“就是你们一起下海的那一次,从主墓室拍下来的,这是叙事壁画,非常关键,你可以看看,里面画的是什么内容。”

我数了一下,一共是十五张壁画,上面都有变化,显然都是有联系的,但是壁画之间却没有什么必然的情节联系。我看到有画着攀登雪山的情形,有画着俯视山陵的情形,有画着攀岩的情形,有画着士兵战斗的情形。每幅壁画的画面,都没有什么必然的联系之处。

柯克看我的表情就知道我看不懂,就拿出一张给我看,道:“你看看这是第一张,你看到的是什么?”

画面是几个女真打扮的人,正在捆绑一个汉人。我道:“是不是在战场上抓俘虏?”

“可以这么说,但是你猜这俘虏是谁?”柯克故作神秘地笑了笑。

我仔细地看了看壁画照片,发现这俘虏的样子竟然和瓷画上的汪藏海形象逼近,惊讶道:“这是汪藏海?女真人在抓他?”

柯克道:“对,这是第一张,就是这样的画面,说明什么?说明汪藏海修建这里,可能是被迫的,他是被掳来的。”

我顿时看出了点苗头采,又去看其他几张,道:“那这些照片?”

“都是汪藏海被掳去之后,他在东夏人手里经历的事情。我们虽然无法完全迹看懂,但是从前面的照片上也猜了个八九不离十。”

我仔细去看了其中一张,突然又发现了不对的地方:“这一张……”

柯克一看,也点了点头:“你眼睛很厉害,这一张也很关键,你发现没有,这就是那火山口里的皇陵,当时汪藏海被掳去的时候,那皇陵就已经存在了,而且已经非常破败了。”

我“啊”了一声,那难道我们头顶的皇陵不是他修建的?

柯克道:“我们研究过,上面皇陵的整体样式,是殷商时期的,但是被他硬改成了明式。东夏人掳他来,不是让他修皇陵,而是让他来改造皇陵,因为皇陵经过了实在太多的年份,已经无法再用下去了。”

“那这里的地宫什么的,也是早就存在了?”胖子问。

柯克点丁点头:“我们就是靠这些照片,找出了通往这里的旧路,但是,还是有些照片无法理解,比如说这一张。”

那是一张无数恶鬼从石头中窜出的壁画,是倒数第三张,还有一张,竟然是描绘了一团黑色的软体生物一样的东西,是从什么巨大的悬崖爬上来,而上面有人往下倾倒什么东西。

我看得神经紧张,松了口气,正想坐下来仔细看看,这时候,阿宁却突然向我伸出了手,道:“好了,我们的事情说完了,照片你随时可以看,现在你是不是也得告诉我们什么?”

“告诉什么?”我莫名其妙。

“我的事情我都和盘说了,你们和吴三省的事情,”阿宁看着我,“你不会比我这个女人还小气吧?”

我心说你说的那些是什么狗屁啊,说了等于没说,重点根本就没提,你他娘的还以为我是以前那个什么都不懂的吴邪,便脑筋一转,就问她道:“你们这里是不是有一个叫乌老四的人?”

阿宁点了点头,奇怪道:“怎么?你认识?”

我从口袋里掏出了两条铜鱼,在他们面前一晃:“你们要知道的事情全在里面,乌老四如果没死,就让他出来!”

一刹那,我看到柯克几乎摔倒在地,阿宁的眼神也都直了,结巴道:“天!你竟然有两……条……”我一移动手臂,他们的眼睛就跟着我转。

\chapter{蛇眉铜鱼}

我实在是不想把蛇眉铜鱼交出去,但是想起三叔的交代,脑子一热就拿了出来,没想到阿宁他们的反应这么大。

隔了好久,其中一个才反应过来,问我道:“你哪里弄来的?你……简直是神仙,难道说你们在鲁五宫里……这是龙鱼密文!我一直以为只有一条,没想到……”

我没心思和他们说这些,摆了摆手,道:“你们这里有人会看吗?”

阿宁马上大叫了一声:“乌老四!”边上一个中国人走过来,一看我手上的鱼,脸色也变了,忙冲过来,大叫了一声:“天!”

我对他道:“能翻译吗?”他猛点头,像接神物一样接了过去,那手电开始照鱼的鳞片,很快大师的女真字就显示在了地上,边上马上就有人帮忙抄写下来。

阿宁的手下到底厉害,一边抄,一边就能翻译,比华和尚强多了,抄完之后,基本意思我也懂了,我听得莫名其妙,根本是似是而非的意思,但是越听到后来就越清晰,有点像叙述诗。我也无法去全部都记录下来,但是其中有几段让我记忆深刻。

全篇的内容非常精简,开头就是几句话,表明了这篇龙鱼密文所隐藏的秘密十分重大,汪藏海刻录下来,本希望永世不见天日,但是如果有人看见,希望此人是汉人而不是女真人这样的说法云云。

后面就记录了他被掳获到东夏之后的事情,和他壁画上的记录非常相似,但也是提到了几句,他为了拿到一些东夏没有的宝物,先后带人盗掘很多的古墓,而在灵气最盛的地方,偷偷将铜鱼放入,以使得这个秘密有机会让人发现。

我看着就“啊”了一声,心说竟然是这样。再往后看,后面的内容就让我匪夷所思到了极点——里面记录的,是他在改造东夏皇陵的过程中,竟然逐步发现了东夏王的一个诡异秘密。

之所以让我感觉非常惊骇,是因为华和尚曾经和我说过了这一段的前半段,也就是东夏的万奴王是从地底爬出来的怪物,是妖孽,而我听到这一段,正好和华和尚说的有关。

里面说的是,汪藏海在这里被困了长达十年的时间,曾经被领去看一扇被称为神迹的地底之门,传说历代的万奴王,不是世袭的,而都是在前一代死亡之后,从那道地底之门中爬出来的。而那道地底之门,也只有在前任万奴王去世的时候才能够打开,否则,地狱的业火就会烧尽那个开门者一切,使得长白山没有白头。我听着感觉像是火山爆发,心说难道万奴王是从火山里爬出来的?

而他有幸目睹了一次这种王位的更替,让他感觉到恐惧非常的是,从地底之门中爬出来的万奴王,竟然是妖怪,根本不是人。

上面记载,这地底之门就在皇陵之下,长白山底,年代源于上古,恐怕是夏时的产物,而通往地底之门的通道,由一种长着人头的鸟守卫。

我想起那种怪鸟就直冒冷汗,但是更诡异的内容却还在后头。

在另一条铜鱼上,竟然记载了他偷偷潜入地底之门的经过,这些我完全看不懂,不知道他们在说些什么,显然是他回来之后,在极度惊骇的时候刻的,有些语无伦次。

胖子也听着,这时忍不住插嘴道:“不是说地狱的业火会烧尽那个开门者一切,怎么他进去就没事?这他娘就是胡扯。”

我心说他肯定用了什么我们不知道的方法,但是这里的记录实在太乱了,这时候,突然有人过来汇报,说是又发现了记号。

我们走过去一看,只见在棺井中的几只棺材都给开了,里面的东西全给罗列了出来,在棺井的一边,有人竟然开启了一道暗门,暗门内又出现了一个记号。

“这记号不是你们留下的?”阿宁问道。

“不是,我们也很纳闷。”我假装不知道。

旁边一个人报告说:“这里的棺椁全是影棺,是假的,里面只有玉做的尸体,真的棺椁不在这里,我们刚才一开,开启了虫香玉的机关,结果全是蚰蜒爬了出来,现在小心地找了找,没想到这里还有一条密道,而且也有人进去了,看样子是个双层墓,真的棺椁可能还在这下面,这是元朝进修比较流行的墓葬方式。”

我看着这宝石琉璃制成的巨大棺椁,心中骇然,又往开启的暗门看去,发现这条暗门非常的不同寻常,因为这条暗道非常陡峭,似乎以挖掘深度为目标的。心中“哎呀”了一声,看样子,闷油瓶不让我去的地方,就是这里了。

阿宁看了看我,看来心中和我所想的一样,也挥了挥手想让人下去,但是所有的人都没动,他们都看向我和胖子。

\chapter{唯一的出口}

墓道倾斜向下,角度越来越陡,我和胖子手电直射下去,看不到一点到头的迹象,尽头处永远是深沉的漆黑一片。

我有点慌起来,我们一路往下已经走了很长的距离,已经深入了长白山的内部,如果再这样一直走下去,我们会走到哪里?地心吗?

可是就算是地心,我们也必须走下去,因为闷油瓶留下的引路符号明白无误地指示我们,他就是朝这个方向走的,我们每走一步,都是靠近事实的真相一点。

我们别无选择,只得硬着头皮走下去。借着手电的灯光走了有二十多分钟,胖子对我道:“小吴,你有没有发现,这条墓道里有点暖和起来了?”

我点点头,道:“也许我们的目的地靠近火山的地层活动区域,那里有熔岩或者温泉活动,温度才会逐渐升高,汪藏海当年到底挖到了什么地方?”

胖子也无法回答这个问题。

又前进了一段时间,胖子突然回头问我:“你老实告诉我,你和那小哥有什么特殊的关系?”

我被胖子问得呛下一声,不知道该怎么回答,随即想到是自己理解错误了,他问的不是我想的那种关系。

刚才的一系列事情发生得太快,我其实自己也没有完全反应过来,现在想想,胖子并不知道我的血在秦岭中已经出现了和老闷宝血一样的现象,他第一次看到如何能不吃惊。为了不在阿宁面前露短,所以当时没问出来,现在只有我们两个人,他自然要问上一问。不过以他的性格,让他正儿八经地来问也是不可能的,他问的我和闷油瓶的关系,应该只是在奇怪,为什么我的血也可以“驱虫”。

按照凉师爷的说法,我的血的奇特能力应该是和吃了熏尸的麒麟血有关系,但是我实在想不起我是否吃过这种东西,对麒麟血又一点也不了解,无法确切回答他,而臣我刚才自己也是意外,根本无意识的行为,也不能单单就断定。是我的血在起作用。

胖子见我不回答,以为我认同他的想法,道:“他说不定是你失散多年的哥哥、弟弟或者表亲之类的,或者是你父亲的私生子,你们家都遗传了这一种特殊的能力。”

我骂道:“你别胡说,我老爹就我一个儿子,他那种学究要是有私生子,那世上就真没男人靠得住了。”

胖子还是认为其中肯定有蹊跷,我实在不想和他讨论这些,就把话题岔开。

走了很久,墓道终于到了尽头,走出墓道,突然就是一阵暖风吹来,让我精神一振。我忙打亮手电向四周看去,发现这里是一处修建在悬崖上的廊台,就和我们来的时候在冰穹中看到的假灵宫的祭祀台一样,脚下的地板是用廊柱架空在悬崖上的,廊台的中间立着一只巨大的黑鼎,鼎的一脚已经陷入到石头地板中去了,呈现一个要倾倒的姿势,显然这个平台我们走动的时候也得小心,底下的石头都老化了。

而平台的边缘都是悬崖,上面也是一片漆黑,看不到头顶。

胖子发牢骚道:“怎么又……到头了,没路走了,还是役有棺椁,这万奴王到底躲什么地方去了?”

我道:“这还不是最奇怪的,我们是沿着那小哥的记号来的,一路上有非常明显的线索,但是你看,这里一个人也没有,难道说,这些人发现这里是死路,都回去了?还是……”我看向一边的黑暗,“飞走了?”

我们走到廊台的边上,信号弹在高空突然燃烧,在这无比漆黑的空间中,就如同一个小太阳,一下子就照亮了我们眼前的情形。

借着信号弹的镁光,我看到这里其实是一处巨大的山体裂缝,我们所在的廊台修建在一边的裂缝峭壁上,而我们对面两百多米处,是巨缝另一边的峭壁,遥遥相对,给人的感觉就像身处在非洲巨大地表裂谷中的悬崖上。我们都不禁发出了一声惊呼。

信号弹逐渐下落,落到了廊台之下,照出了我们下方情形,又是一幅让人震惊的景象出现在我们面前,只见下方深不见底的裂谷中,无数碗口粗细的青铜锁链横贯两边,将裂谷连在了一起。

随着下落的光源,在廊台下二十米,到一片混沌的裂谷深处,也不知道有多少这样的锁链架在那里,几乎看不到稀疏的地方,而在深处的锁链上,还密密麻麻地挂着很多的东西,好像很多的铃铛一样,实在太远,看不清楚。

这时候胖子在廊台的一端找到了一根攀岩绳子,从平台的一端垂了下去,一直垂到下面最近的一根青铜锁链上方,系在了那里。

胖子皱起眉头道:“够戗,那小哥倒也锲而不舍,看样子他爬下去了。我们是不是也得跟下去学猴子?”

我道:“看下面锁链的密集程度,想必不会难爬,只不过这里到底是什么地方?为什么会有这么奇怪的设置?难道万奴王的棺椁会在这裂谷下面?你有没有想过,当时他们如何能将那巨大的棺椁运下去?”

胖子道:“真有这个可能,不过古人总会有他们自己的巧妙办法,我们是上去通知那几个老外,还是自个儿先下去?”

我道:“下去之后还不知道能不能再上来,咱们犯不着给他们当探路狗,把他们叫下来,他们的装备和技术都比我们好,还能有好处,况且潘子也还在上边,反正三叔也已经找到了,多花一点时间就多花一点时间。”

胖子想起柯克那满身的肌肉,也同意了我的说法,我们又从原路返回,因为知道路颇长,走的时候不知不觉都加快了速度。

阿宁他们早就等得心急了,还以为我们出了事,见到我们回来了,才松了口气。我接过潘子的水,喝了一口,就把看到的东西说了一遍。

一听说下面有横亘的青铜锁链,阿宁忙掏出了她从海底墓中拍下的照片,指着其中的一张,只见照片里的壁画上,很多东夏勇士正背着弓箭,攀爬在一道悬崖峭壁上,而背景就是无数类似于锁链的东西,显然描绘的就是东夏人探索那遭巨型地下裂谷时候的情形。

我道:“看样子,那些锁链也不是东夏人设置的,他们当时也应该很好奇,这些用锁链封锁着的裂谷底下是什么情形。”

胖子道:“这些铁链条,会不会是修建这里的先民的什么防御措施,用来防止下面的什么东西爬上来?比如说,壁画中描绘的那种巨型黑色软体东西。”

我点头道:“有可能。”又问阿宁,“这一幅壁画是第几张?下一张是什么?”

阿宁道:“按照叙述壁画的一般规律,这应该是倒数第六张壁画,后面还有五张,依次是……”

阿宁将最后五张照片摊开,我看到后一张照片里的壁画,是很多东夏勇士搭弓射箭的情形,似乎有一场惨烈的战斗,但是壁画上又看不到敌人,不知道他们在和什么搏斗。我想起那种在空中飞行的时候看不到身形的怪鸟,心中就一紧,心说难道下面也有这种东西?

而再下一张,就是很多恶鬼从石头中钻出的情形。

壁画和壁画之间似乎并没有太多情节上的联系,但是看上去又给人无限的联想,很有意识的感觉。

阿宁问我:“是不是又看出什么蹊跷了?吴超人?”

我自嘲地笑了笑:“倒也不是看出了什么来,你看,在攀爬悬崖的壁画后面就是战斗的画面,我感觉这也许是告诉我们,下到裂谷中之后会遇到什么危险,有武器的人把武器都准备好。”

几个人都当我是精神领袖,我说什么就是什么,柯克忙端起自己的M16,做了个包在他身上的手势。我们收拾起行囊,向深切入长白山内的墓道走去。

我跟在队伍的最后,去看三叔怎么样了,却还是昏迷不醒,也不知道他在这里看到了什么骇人的东西,潘子很让我放心,他说就算是爬,他也要把三叔一起爬着拖出这个鬼地方。

在墓道中走着,看着前面神经紧张的众人,心里也有一种奇怪的感觉,在陈皮阿四和三叔都不在的情况下,我不得不但当起了这些人的领袖,这种感觉是我从来没有感受过的,有一种莫名的快感。但是,我的想法和我的决定真的是正确的吗,会不会我正在将这些人全部推向死亡呢?想到这里,我又感觉自己犹豫不决起来。

不久所有人都来到了廊台上,胖子又打了一个信号弹,让众人看裂谷四周的壮观景色,我和潘子掏出绳子准备攀爬到下面,这是一个极度冒险的决定,但是我们的去路已经被完全封死了,一点别的选择也没有。

不知道下面是一个什么情景,阿宁这一批人也不是好货,我掏出所有的绳子后,将胖子拉过来,告诉他要小心一点,现在我们都落了难,大家看上去都很合作,一旦到了下面出路有了眉目,要小心那臭女人翻脸不认人。

胖子拍了拍我的肩膀,给我打了个眼神,意思是早就留子一手了。怕我不放心,他又扯开衣服的一角让我看了看,他的腰间绑着剩下的十根雷管。

我们试验了一下,闷油瓶的那根绳子非常结实,潘子还是做先锋,第一个爬了下去,下到锁链上之后,他像单杠运动员一样,挂下自己的身体,轻松地就跳到了下面的另一根锁链上,这样重复五六次,已经下去了十多米,给我打了个OK的手势。

阿于他们的装备比我们好得多,柯克带上发散式的指引头灯,把自己变成一只移动的灯泡,第二个爬了下去,我们以柯克的脑袋为指引,陆续爬下廊台,来到悬空的锁链世界中。

不过锁链的密集程度颇高,攀爬还是十分的方便,不论青铜锁链设置在这里原来的目的是什么,反倒是给了我们这样的人一架方便的梯子。爬得久了,各种动作都熟练起来,也掌握了一些窍门,大有蜘蛛侠再世的感觉。

如蜘蛛一般,十几个人缓慢地向下,一路上并无突发事情发生,锁链的牢固程度也让我们叹为观止。四个小时后,头顶的廊台已经变得很小,我们进入到了裂谷深处,已是我们在上面目力所不能及的范围。那些在上面看不清楚的、挂着巨型铃铛一样的青铜锁链出现在了我的视野里。

谨慎起见,我吹了一下蝙蝠哨,让最下面的柯克和潘子停了下来,用阿宁的夜视望远镜向下看去,幽幽的绿色视野中,我看到那些挂在锁链上的东西,原来都是一些吊死在那里的死人,一条黑色的头发般的丝线从他们后颈深处延伸出来,挂在锁链上。看数量,底下的锁链上密密麻麻,几乎无法尽数。

汪藏海的龙鱼密文中透露出,我们所遇到的那种在空中飞行时候看不到、只有在落地的时候才会出现的人头怪鸟,喜欢将猎物挂在枝头上风干备用,这里有这么多尸体,难道下面竟然是它们的巢穴吗?

难怪闷油瓶让我们千万不要下去,可是我们现在也役有其他路可走,不下去搏一把,还不是一样死?我压下心头的恐惧,打了个手势,示意所有人戒备,继续向下。

已经走到了这里,就算下面是地狱,我们也得硬着头皮下去了。

\chapter{守护神的巢穴}

尸体都穿着破烂的盔甲,有些被风干成木乃伊了,有些则已经成了半骷髅状,这些应该都是当时的女真勇士,被猎杀在了探路的途中。不过他们当时的武器太简陋了,我们现在有这么多的M16和五六式,火力非常猛,想到这一点,我就心安了不少。

进入挂尸锁链的范围之后,又向下爬了将近五十多米,上下左右都是尸体,那种腐烂空洞的眼神望着你,着实让人不舒服。气氛一下子阴郁起来。

为防止出现视线死角,或者驱散这种恐惧,有几个人打起了冷烟火,四周的亮度达到了空前的强度。

有点意外的是,并没有什么怪鸟出现,我也没有感觉到那种它们在空中飞行时候的躁动,四周出奇的安静。

胖子指着一边悬挂起来的尸体,轻声问我:“都是老尸体,没有新鲜的,会不会这里已经被荒废了?”

我摇头让他别说话,有这个可能,但是既然这里的怪鸟能够出去狩猎,那说明附近肯定有出口,我们希望大了很多。

当然也有可能就是它们暂时不在,像成群的蝙蝠,都是在同一时间飞出洞口去狩猎的,这样想的话,我们应该快速通过这一段区域。于是我打了个招呼,催促加快速度。

这个时候,前面的柯克和潘子却停了下来,潘子转身招手让我过去。

我让其他人原地休息,几个跳跃连爬下去十几根,来到了柯克边上往下一看,原来他的强力手电已经能够照到裂谷的巨大底部,我们的蜘蛛侠生涯看样子即将结束了。

不过手电光圈发散得太厉害,看不清底下有什么东西,在经历了中国古墓的诡异墓名之后,这个德国人显然对自己的判断力丧失了信心,凡事都要我看过才能作决定。

这条地下裂谷太大,用手电去看一点用处都没有,你只能知道下面有东西,但是什么一概看不出来,用夜视望远镜也只能看到模糊的绿色影像。

我们还剩几颗信号弹,本想省点用,但在这种场合也省不下。我让胖子想个办法,在这种环境下发射一颗照明弹,尽量能让照明的时间长一点。

阿宁他们的照明弹比我们先进,胖子懂行,知道怎么用,就做了个OK的手势。

他把一根荧光棒打亮了,用刀切开,把里面的涂料点在照明弹的弹头上,然后把照明弹丢到下面深渊中,我们只见一个荧光小点像流星一样滑落,掉到裂谷的底部,摔了两下不动了。

接着胖子端起五六步枪,一个三点射,打中了下面的弹头,顿时照明弹就烧了起来,整个谷底给照得清清楚楚。

确实已经到达了谷底,底下全是极度不平整的黑色火山岩块和从上面跌落的尸骨,层层叠叠也不知道有多少骨头和黑色的粪便,几乎把这些岩块都覆盖了,而在裂谷底下一边的崖壁上,有一扇两面的青铜巨门。

我都无法来形容这一扇巨门的宏伟程度,门高在三十米左右,宽度将近六十米,折算成三米一层的现代楼房,这门光高度就有十层楼这么高。

整扇门面看上去竟然像是整体铸造而成,这绝对不是古人能铸造出来的青铜制品,也绝对不是给人用的,因为这样的门有上万吨重,压在岩石之上,什么人能够打开?

阿宁道:“这一定就是东夏传说中,历代万奴皇帝出现的地底巨门,每次王朝替换之后,他们就再次用人牲的活皮,将门封闭起来,你猜……这里面是什么地方?”

我摇头,脑子根本在其他地方,心说这么一扇巨门,到底是什么人铸在这里的?万奴王是怎么出来的?难道他真的是神,拥有能够推动万吨巨石的神力?我喃喃道:“不管里面是什么地方,我们绝对进不去。”

同样的巨型青铜器,还有我在秦岭的深山中看到的巨型青铜神木,同样也是深深地埋在山脉的底端。这些巨型的、人力无法修造的青铜神器,是不是有什么联系?又或是其他的巨型山脉,比如昆仑、喜马拉雅,它们巨大的山体中,会不会也有这样的东西存在呢?

我隐约间感觉自己似乎正在靠近一个远古的巨大谜团,一种极度渺小的自卑感油然而生,和这些神迹的古老神秘相比,我一个人实在是不值一提,就连知道真相的希望都一点也看不到。

照明弹逐渐熄灭,地下又重新被黑暗笼罩,但我还是呆在了那里,直到一边的潘子拍了拍我,道“下去吧”,我才回过神来。

我们陆续爬下了锁链,很快来到子谷底,小心翼翼地踩着脚下的骨头,走到青铜巨门之前,顿时自己的渺小感就更加强烈,我简直有跪下来的冲动。

以我们这个年代的人,到了这里都有这种感觉,更不难想象当年的东夏勇士千辛万苦带着汪藏海来到这里的时候,会是怎么样的震惊,也难怪他们会对在这里的经历念念不忘,以至于拼死也要将这里的一切记录下来,传达给后世的人。我甚至能够感觉汪藏海的痛苦,他那种原本以为自己已经通彻宇宙的规律,又突然发现自己什么都不懂的恐惧。

正在胡思乱想,胖子在一边打断了我的思考。

他正用手电照向裂谷的中间,这条地下裂谷谷底足有五六百米宽,地上的碎石都像小山包一样,胖子走得很远,看到裂谷中间的地方,一块巨石山给整个儿打成一个一个平台,就像一座小型的金字塔一样,一条长长的石阶修造在石头的一边,每一级阶梯两侧都有一盏小灯奴。

引起胖子注意的,是石台上摆放的东西,那是一只巨大的犹如轿车大小的白石棺椁,九条石雕的百足龙盘绕在棺椁的底下,形成莲花的形象,四周还立着四个黑色的石人,面朝四方,做跪拜状。

棺椁之前有一只盛放祭品的大鼎,后面有一座影壁,看不清上面雕刻了什么,这些东西从上往下看的时候,都和普通的石头一样,不容易看清楚,所以刚才都没有看到。

我倒吸了一口冷气:“难道这是……万奴王的九龙抬尸棺?汪藏海龙鱼密文中说的?”

胖子道:“绝对就是,那个谁不是说嘛,万奴王的棺材下由九条神龙守护着,你看这棺椁下面,不是正好就九条蜈蚣嘛,我还以为陈皮阿四当时是在晃点我们,没想到是真的!”

一直以为万奴王只有墓室地宫中的影棺,尸体实行了天葬,早已经放弃了找到真正王棺的希望,没想到在这里居然被我们发现了真正的九龙抬尸棺,我们全部都激动起来,几个心急的已经跑了过去。一边的阿宁忙急急叫住了他们,大叫:“不要过去,危险!”

跑过去的人一听,马上停住了脚步。阿宁大叫:“你们没看到棺材下面的蚰蜒龙吗?”

胖子道:“我的姑奶奶,那是石雕的,有个屁危险,你他娘的是什么眼神啊?”

阿宁娇眉倒竖道:“你他娘的才是什么眼神,我说的不是那些石雕,你好好看那石台边上!”

石台边上?我看阿宁的表情很严肃,但是石台边上,我左看右看,又实在看不出什么东西来,不知道她到底在紧张什么东西,就让她指给我看。

阿宁用手电当成教棒,当下一指,初时我仍旧什么都没发现,正在极度纳闷的时候,我突然发现石台竟然动了一下,顿时发现,原来在石台之上,竟然盘绕着一条巨大的火山蚰蜒,足足有五六米长,因为实在太大了,加上它甲壳的颜色和火山颜色几乎一样,所以粗略一看,根本发现不了有这么一只东西趴在上面。发现了第一条后,马上第二条、第三条、第四条……一共九条巨型蚰蜒给我们数了出来,全部盘绕在那座石台上,好比石头上的浮雕,几乎与石台融为了一体。

九龙抬尸,真的是名副其实的九龙抬尸!

阿宁道:“你们如果一爬上石台,还没明白怎么回事,肯定就被咬成两截了,火山蚰蜒是食肉性昆虫,非常的凶狠迅捷,我们这样的体形,正是它们最喜欢捕食的对象。”

我已经算经历过很多古怪的事情了,如果这几条蚰蜒长到一米,我也还能原谅,毕竟这里是火山中的地下裂谷,环境和空气成分大多不相同,世界上其他地方也有过发现,但是大到如此超出常理的昆虫,我还是第一次见到,这简直是美国恐怖片里被辐射变异了的怪物。

边上阿宁队伍中一个华裔的专家自言自语道:“太奇怪,这种蚰蜒的寿命一般也只有两三年,虫子在只有手指这么长的时候就应该死了,这几条能长到这么大,难道已经活了几千年了?”

\chapter{谍中谍}

看到九条巨大的蚰蜒盘绕在裂谷底部的棺台之上,尽管一动不动,但我们还是感觉到了巨大的压力,一个一个脸色惨白,一边后退一边将武器举了出来。

那个华裔专家说:“你们不用这么紧张,现在是冬天,这里的气温还偏低,蚰蜒还在冬眠期,这些巨虫子不会这么容易醒。”

阿宁道:“不容易醒,总归也有醒的可能,我们这些人,是绝好的冬眠点心。”

胖子杀心又起,说道:“管它醒不醒,老子摸过去顶着它们的脑袋来几枪,就算它再大十倍也立马死定了,接着我们就去看看这个从地底爬出来的、不衰老的万奴王到底是人还是妖怪。”

潘子摆手道:“绝对不行,你还记得不记得顺子说过,死去的蚰蜒会惊醒其他冬眠的同伴,这条裂谷左右贯通了整条长白山系,你知道里面有多少的蚰蜒,到时候别有更大的家伙出来替它的徒子徒孙报仇。”

我举起夜视望远镜,想再真切地看一下,这么大的蚰蜒,说不定是古代昆虫的化石,我实在说服不了自己这些是活的。举起来一看,却看到棺椁之后的影壁上,原本看不清楚的浮雕,竟然是很多的女真文字。我当即就一愣,心里激动起来。

影壁浮雕之上的文字非常多,非常多的文字聚集在一起的地方,必然就是有一定的叙述内容,汪藏诲修建的建筑当中,很少出现文字,但是这里却出现厂这么多,那就很可能是古墓中最珍贵的资料之一的墓主人志。

我忙把阿宁手下那个会读女真文字的小个子拉了过来,把望远镜递给他,让他帮我看看上面写的是什么。

那小个子一看,一脸的迷惑,说虽然这些字和女真文字的形体很像,但却不是女真字,是另一种相同语系的文字,一时半会儿他也不知道写的什么。

我顿时又泄气,心中暗骂,这万奴王也太狡猾了,简直不给我留一点破绽。

也难怪,像汪藏海这样处心积虑到了极点的人,在这里二十年,直接参与了上古皇陵的改造,也无法探到万奴王朝想隐藏秘密的那个核心,那万奴王为他设置了一个不可逾越的障碍,更不用说我们这些靠猜来行事的人了。

可惜华和尚不在,他浸淫其中多年,有着别人不具备的思维习惯,他在这里,说不定还能说点名堂出来。

想想又觉得不对,如果华和尚也在这里,那局势之复杂就不是我能控制的了。

一会儿一个念头,一会儿又是一个念头,脑子都不知道在想什么,一边我又听到潘子在叫:“胖子?你行不行,要不换人?”

我最不爱听到潘子叫胖子的名字,心中一跳,举头一看,只见胖子和那个柯克已经爬上了一条锁链,小心翼翼地走到了棺台的上空,胖子正在腰上系绳索,大概想像汤姆·克鲁斯一样,从锁链上挂下去,悬空到棺椁上方,而且其他人竟然没有阻止,还在一边指示胖子的位置。

我问阿宁怎么回事,这些人准备看九龙戏胖珠吗?

阿宁道:“没事,一般来说这样的方式不会惊动蚰蜒冬眠,而且我刚才发现蚰蜒的尾巴都被青铜锁链锁在了石台下的石桩上,它们的活动范围有限,只有步行靠近的人才有危险。他们来这里都想看看万奴王的棺椁中有什么,现在找到了棺椁又不能看,谁也忍不住。”

我说就算胖子能垂下去,也不能翻开这么巨大棺椁的石头盖子,你也是看他出丑而已。阿宁说:“他不是去翻棺椁盖,他是把启棺钩卡进棺椁的缝隙中,我们在上面的一根青铜锁链上挂上一个滑轮,然后我们在这里将棺椁盖子吊起来。”

我心里感觉到很不舒服,阿宁她还是在履行公司的工作义务,寻找棺椁中的某样东西,就算到了这样的地步,她还是没有放弃,虽然我不知道她要寻找的是什么,但是我觉得没有理由有一样东西会让人觉得比自己的生命还重要。而且棺床之上有如此多的青铜锁链,汪藏梅设计的时候不会想不到他们的招数,肯定有什么蹊跷使得他认为上面不需要防范。胖子傻乎乎地做先锋,肯定是想第一个开棺的可以捞点好处,我必须要阻止他。

谁也不知道柯克发生了什么事情,胖子正在调整自己蹦极的位置,一看柯克竟然跳得比他还快,一下子愣住了不知所措。接着突然他自己也飞了起来,在空中竟然手舞足蹈地盘旋了一阵,就直往下掉去,幸亏他腰上有绳子,在脑袋快撞上棺椁的时候绳子蹦直了,停了下来,脑袋下面就是柯克的尸体。

我几乎吓晕过去,这景象太诡异了,难道锁链上有什么东西把他们椎了下来?

想到这里,我忙对一边呆若木鸡的潘子叫道:“照明弹!所有人操家伙!”

众人顿时反应过来,我们也没工夫去顾及胖子了,潘子一颗照明弹打上半空,炸了开来。顿时我们看到无数只影子在我们头顶上盘旋,好几只已经倒挂到了锁链之上,好奇地看着我们这些闯入巢穴的怪东西。

原来是那种怪鸟不知道何时已经无声无息地开始归巢了,我甚至看到天空飞翔的怪鸟中,有几只还抓着什么东西,显然有猎物到手。我举手让那些几乎箭在弦卜的人千万不要开枪。

这些怪鸟是半瞎子,在这么强烈的光下,根本看不见我们,但是它们对声音非常敏感,就是我们在前殿之中开了一枪,才引得大量的怪鸟从四面八方飞来。显然在一点光线都没有的地下火山口里生活的这种生物,早已适应了黑暗中的生活。

然而我说不要开枪不要开枪,却还是还有人开厂枪,而且还不是一声,而是一连串的扫射,枪声在空旷的裂谷底部极其响亮,响彻云霄,上空顿时一片骚乱,无数的影子盘旋着就开始俯冲下来。

我怒目转头看是哪个王八蛋不听命令,却看见石台上的胖子正在试图爬上绳子,柯克的M16被他拿了过去,此时他正在对着下面的棺椁不停地扫射。

我仔细一看,发现万奴王的巨大棺椁,不知道什么时候竟然启开了一条缝,三只青紫色类似于手臂的东西,注意,是三只,从棺椁中伸了出来,奇长的指甲在空中划动,想要抓住上方的胖子。

\chapter{千手观音}

天空中的照明弹熄灭,黑暗迅速笼罩了下来,潘子随即又打出了一发照明弹,在空中炸亮。接着下面的人全部都开火了,十几条火舌向上空倾泻,很快天宫中飞翔的影子就有几只中弹,从空中摔落下来。

强光可以使得这些东西产生暂时的错觉,就像你在“狗熊”面前做“鸭子”叫和走路,它会一时分不清你到底是人还是鸭子一样。但这只是暂时的,如果我记得没错,这是我们最后一发照明弹了。

如此多的怪鸟,一旦这一颗照明弹也熄灭了,我们将面临在黑暗中被无情捕杀。

怪鸟越压越低,有的甚至已经从我们的头顶掠了过去,我们的子弹根本不够这样大强度的扫射,很快几把枪就告罄了。胖子的情况又极其危急,如果没人去救他,他这一次命再硬也得完蛋。

正左右为难、不知所措的时候,胖子一枪打在了我的脚下,把我吓了一跳,我抬头看他的嘴形,知道他的意思是让我们跑吧!

我心一横,对潘子道:“你带着三叔和其他人往裂谷的尽头跑,这里是它们的巢穴,它们肯定是顺着裂谷飞行出去觅食的,你看它们飞来的方向是哪一边,就一路跑下去,不要管我了,我去救胖子!”

潘子抓住我道:“你行不行啊,要不我去救胖子,你带三爷走!”

我道:“我背不动那老头子!”扬起手让他看我的伤口,“老子有宝血,绝对不会有事!”

潘子看到我的伤口,稍微安心了一点,用力点了点头,道:“小心点,我们在外面等你!”当下背起不能行动的三叔,对着其他人大叫丁一声“跟着我跑!”就往裂谷的一边退去。

我接过潘子扔给我的枪,“咔嚓”一声看了看子弹,三发,真他娘的慷慨,其他人在我身边狂奔而过,大叫着叫我跟上,我都没理,这时候我看到阿宁也站在原地,脸色惨白,但是没有动。

我上去拍了她一下,让她快走,她甩开我的手,“咔嚓”一声也端起了枪,不知道又有了什么打算。

我知道这种人劝也没用,不去理会她,端着枪就朝石台上跑击。

走运的是、就算如此混乱的环境,棺台四周蛰伏的巨大蚰蜒还是没有苏醒,也许经过了这么多年代的沉睡,这些巨大的昆虫早就死了。

此时我也管不了三七二十一,大叫:“胖子,把五六的子弹给我!我掩护你!”

胖子自己的枪是五六式的,身上全是五六式的子弹,但是他攀爬的时候减重没拿上枪,所以用柯克的M16来,但是M16的子弹不多,要是打完了他在上面就完了,只有下到地面上才有一线生机。

胖子听到我叫他,马上单手持枪,另一只手扯下几个子弹便丢给我,我接住一个,其他几个也不要了,换上弹匣端起枪来就射。胖子在我的火力掩护下顺着锁链一路狂爬,爬到他上去的地方,然后一溜烟儿滑丁下来,对我招手让我快跑。

我转头去找阿宁,人已经不知道去向,不知道是跑了还是被怪鸟叼飞了,心里暗叹绝色佳人何必如此执著,又一看棺台上,只见棺椁板子已经翻到了一边,一具巨大的黑色男尸站立起来,身上穿着已经褪色腐烂的女真铠甲。让我大吃一惊的是,这具男尸竟然长着十二只手,呈环形排列在身后,而且十二只手都在扭动,就像庙中的千手观音一样。

我马上想起了海底墓穴中看到的十二手蜡尸,不由惊讶万分,难道东夏的皇族不是人?这具十二手男尸就是万奴王?

胖子一边点射,将俯冲下来的怪鸟逼退,一边到我面前来拉我,大叫:“你在发什么呆?”

我不理胖子,对他道:“你看……他想干什么?”

只见千手观音尸舞动着他的十二只手,对我们并没有一点兴趣,快步跳下石台之后,径直就向青铜巨门走了过去。

胖子惊讶道:“难道他是想进入巨门之内?”

我顿时想起汪藏海龙鱼密文上的最后一句,如果时间不对,打开地底巨门就会遭受天谴,地下的业火就会通过巨门涌出地狱,焚烧整个天空。

当时我们认为这一句预言的灾难,是汪藏海进入巨门之后,看到了火山内部情景之后的臆想,但是也有可能这道青铜门的设置者为了防止青铜门内的秘密被发现,设置了什么威力巨大的机关。

此时我们就在青铜巨门之前,如果有任何的机关,我们肯定是首当其冲,成为第一批牺牲者,不管是不是真的,我们也必须阻止这只畸形粽子。

我追着千手观音尸几个扫射,但是子弹打在尸体上犹如打进橡胶里,也不穿透也不炸裂,好像泥牛入海,一点反应都没有,而且最可恶的是他对我们一点反应也没有。我对胖子大叫:“炸药!”

胖子顿时想了起来,他腰上还有准备用来威胁阿宁他们的几根雷管,马上冲上前去,一跃而起跳到千手观音尸的背上,把雷管像黑驴蹄子一样塞进了尸体的嘴巴里,然后赶紧跳了下来。

我眯着眼睛一个扫射,不知道哪颗子弹正射中雷管的引信,顿时雷管就爆炸了,千手观音尸的脑袋连肩膀部分整个儿炸裂了。我们被冲击波掀翻在地,碎片和气浪扑面而来,顿时胸口发闷,满耳朵都是嗡嗡声。

上面的怪鸟被强烈的声波刺激,一下子就疯狂起来,我赶紧爬起来,见千手观音尸已经倒在地上,不由大喜,果然炸药还是无敌的。

没想到胖子还是一脸惊恐的表情,对着我大叫,我什么都听不到,只看到他的嘴巴快速地动,好久才听明白,原来是:“快跑!照明弹要灭了。”

还没反应过来跑的时候,突然头顶上的光线在几秒之内就消失了,黑暗犹如雾气一样迅速笼罩过来,顿时所有的光线只剩下我们手里的手电。

四周一下子竟然安静起来,逃入裂谷深处的人的枪声也逐渐平息了,只剩下我们喘气的声音和响雷一样的心跳声。

我和胖子背靠着背,我解开手上包的绷带,露出里面血淋淋的伤口,一边祈祷我的血对它们也有用处,那个什么教授不是说了,这种麒麟血只对吃尸体的东西有作用,我也不知道这种怪鸟是吃什么的。胖子端起枪,“咔嚓”一声上了子弹,看着天上,问我怎么办。

我说你问我我去问谁,话音未落,突然一只怪鸟抖落着翅膀落了下来,停到了我们前面十几米的地方。这鸟极其大,站起来比我还高,落下来后,丑陋的鸟头转动了几下,就直勾勾地盯着我们,似乎在打量我们这两个人。我隐隐看到它嘴巴里的撩牙闪着寒光,忙举起手,用伤口对着它,但那怪鸟没有什么太大的反应,还是面无表情地立定在那里。

接着又有两只怪鸟飞落下来,一只停在了我们的左边,一只停在了我们的身后,我四处转动伤口对着它们,不知道它们的意图。

逐渐地,怪鸟飞下来越来越多,一只又一只,很快,我们四周围满了这样的鸟,但是这些鸟都没有行动,黑压压的一片。我逐渐感觉到不妙,这些鸟似乎对我的血一点也不感冒,而它们又不马上进攻,似乎有什么阴谋。

\chapter{围攻}

无数的人面怪鸟,犹如雕塑一样将我们围住,降落的时候无声无息,站在那里也不发出一点声音。我突然想起了国外恐怖电影里的石像鬼,那种白天是石像,晚上变成动物的妖怪,难道就是以这种鸟作为原型的?而且从这些鸟的眼神来看,似乎是有智慧的,这样围着我们,是不是有什么诡异的目的?

很快我的预感就应验了。突然有一只鸟从我们上空掠了过去,地下了一个什么东西,“砰”的一声落在我们面前,顿时鲜血四溅,我一看,竟然是叶成,脖子已经被咬断了,正在不停地咳嗽,但是眼睛已经涣散,没救了。

接着又有一具尸体给抛了下来,不知道是谁,但是脑袋已经没了,浑身都是血。

陈皮阿四和我们分手之后,直接冲进了皇陵之中,显然他们也受到了这种怪鸟的袭击,叶成应该就是在皇陵的中心被这种巨鸟捕获的。没有三叔暗号的指引,这些人竟然落得了如此凄凉的下场,我真是想也想不到。

我以为陈皮阿四也不能幸免,但是接下来抛的几具尸体,都是阿宁的手下,显然刚才并不是所有的人都逃脱了,所幸我没有看到三叔和潘子的尸体,总算让我稍稍安心。

胖子此时算是真的有点害怕了,问我说:“这些鸟想拿我们干什么?”

我对他说:“好像正在把猎物集中起来,我不是这方面的专家,不知道它们想干什么,你还有炸药吗?咱们可能得学董存瑞了。”

胖子摇头:“全炸万奴王去了,你又没说还要剩点儿。”

我心说这下麻烦了,我千算万算也算不到,我吴邪竟然会这么死,四周全是鸟,一点空隙都没有,连跑的机会都没有,难道真的要死在这里变成鸟粪?

正在心急如焚的时候,胖子忽然拉着我后退:“这样腹背受敌,太不利了,这里有一条岩缝,我们躲进去,一人挡一面,死也不能这么便宜了这些死鸟。”

我回头一看,是裂谷地下两块巨型山岩之间的夹角,有一条一人宽的缝隙,两边都通的,缩进里面活动可能不便,但是防守倒是一流的地方。

马上死和抵挡一会儿再死,当然后者合算。我们当下解下尸体上的子弹带,快速钻人了缝隙之内,里面空间很小,我尚且可以做一些腾挪,胖子就很勉强,估计这些鸟要钻进来也够戗。

胖子经历过多次生死悬于一线的场面,此时表现得比我镇定得多,一人缝隙之内,马上堆积起几块石头作为掩体,对我道:“它们只能一只一只进来,只要杀掉几只,就能把人口堵住,我们能撑得久一点。”

我心中苦笑,我们子弹根本就不多,而且其实根本没有换子弹的时间,如果子弹匣中的打完,就等于死期到了。不过现在还没有到临死的时候,还是存在一丝侥幸。

脑子还在胡思乱想,突然我听到外面的鸟群开始号叫起来,通过缝隙我看到为首的一只怪鸟突然不成比例地张大了嘴巴,露出了满口的獠牙,接着从它的嘴巴里面,突然吐出了一只猕猴一样的生物,动作极其敏捷,一下于就蹿到地上,先是谨慎地四处看了看,然后跑进尸体堆里,开始撕咬起来。我仔细一看,发现这猴子没有皮,浑身血通通的,竟然似乎是那怪鸟的一种器官。

接着其他的怪鸟也开始吐出这种生物,无数的“口中猴”从鸟群中蹲出,冲往中间的尸体堆,似乎也没有什么阶级之分,上来一拥而食,顷刻间到处都是血和散肉,争食之间,还不时发生冲突。

我和胖子都皱起眉头,几欲作呕,心想到如果等一下我们也是这种下场,自己怎么也接受不了。

“口中猴”数量极多,很快外面的尸体被分食干净,空气中的血腥昧到达了一个让人无法接受的程度。胖子眼睛血红,知道下一步就轮到我们了,他喝了一口白酒,道:“他奶奶的,想吃胖爷我,看看你们有没有这铁板牙。”

我不争气地有点发抖,也接过他的酒咕咚咕咚喝下去一大半,顿时喉咙火烧。酒的确是好东西,男人有了酒和没有酒,感觉真是不同。

外面“口中猴”在残骸中四处搜索,突然有一只就注意到了缝隙中的我们,发出了一声怪异的尖叫,接着其他“猴子”好奇地围了过来,一张张脸探出,打量我们。

我这才能看清楚,那“猴子”竟然没有嘴唇,难怪猿牙如此的锋利,狰狞异常。最让我奇怪的是,所有“口中猴”的脖子上,竟然都挂着一个青铜的六角铃铛,有些还完好,有些已经只剩下半个了。但是这些铃铛随着猴子的行动,一点声音也不发出来。

我当时十分的害怕,也没有去考虑这意味着什么,但是事后我就想到,这些青铜的铃铛,必然和整个谜团有着莫大的关系,虽然似乎这些铃铛并不属于同一种文化。

“口中猴”刚开始还是很谨慎,在洞口围了很久,胖子和我大气也不敢出,端着枪等着它们进来。过了一段时间,有几只就按捺不住了,突然从缝隙顶上悬挂下来,一下跳入缝隙,试探性地朝胖子猛扑过来。

胖子猝不及防,几乎就贴着那怪猴的脑袋开了枪,子弹横贯而出的同时,也将尸体带飞了出去,掉到尸体堆里。接着他的枪就走火了,子弹横扫,猴群里发出惊恐的号叫声,好几只猴子顿时给打得血肉横飞。

顿时所有的猴子都注意到了缝隙之中的我们,场面失控了,为首的那只“口中猴”发出了一声尖锐的叫声,所有的猴子开始向缝隙中钻进来。我咽了口唾沫,知道自己的噩梦就要来了。

没等我祷告一番,两只猴子已经闪电一般跳入了缝隙,挂在缝隙顶上朝我张开了巨大的嘴巴,五六式太长了,没法用枪托去砸,我只好飞起一脚将一只踢了出去,然后两枪将另一只打死,顿时那血就爆了开来,炸了我一脸。然后又是一只狂冲了进来,我根本没有心理准备再去点射,端起枪就开始扫。

五六分钟时间里,我也不知道自己到底是干了什么事,只看到一只又一只狰狞的猴子冲到那里,又被我扫出去,到处是溅飞的血液,猴子发了疯一样根本没有一点畏惧,有时候几只甚至一起挤进缝隙,自己把自己卡住,都被我用脚狠狠踢了出去。然而更多的猴子犹如潮水一样涌了过来,子弹扫过,就算是只剩下半个身体,只要能动,它就还是往缝隙里直钻,简直穷凶极恶。

很快子弹就告罄了,我原本以为坚持个把小时肯定没有问题,但是实际上战斗起来,子弹的消耗量不是你所能控制的。我还有很多子弹带,但是只要猴子不停止冲锋,我们就没有机会换子弹。

胖子的M16首先卡壳,他已经杀红了眼,大骂着丢掉枪,掏出军刀就想出去肉搏,但是人家根本不给他这个机会,一瞬间五六只猴子就已经跳到了他的身上,开口大咬。胖子疼得大叫,把手上的两只敲死,但又是四只一下就扑到了他的脸上。

紧接着我的五六也没子弹了,按着扳机“咔嚓”、“咔嚓”好几声,我的心突然一凉,接着几道红光瞬间就冲到了我的面前,我还没来得及拔刀,肩膀和大腿内侧就中招,下意识的我就用我受伤的手去吓它,但是一点用都没有,挣扎间我脑子只剩下了一个念头:我吴邪和王胖子,恐怕再也走不出这长白山的秘境之中了,命硬如我们,也终归有命丧的一天。

\chapter{天与地的差距}

无数只“口中猴”扑到我的身上,撕咬我的肌肉,我剧烈地挣扎,准备不耗尽最后一点力气决不罢休,但是心中早已经绝望,这样的情况之下,就算神仙老子来了,也救不了我们。

正在负隅顽抗,突然四周一震,我们都被震了一个跟头,抓在我身上的猴子顿时一呆,瞬间,突然全部猴子都从我们身上滑落下去,拼了命地向缝隙的出口逃去。

我转头一看,胖子那边也是同样的景象,顿时“口中猴”瞬间全部退出了缝隙,似乎见了鬼一样。

胖子浑身是伤,也是莫名其妙。我们面面相觑,胖子自言自语道:“怎么了,到手的东西不吃了?难道嫌我太油腻?”

“口中猴”的骚乱还没有结束,围在缝隙外的猴子毫不停留,爬回到人头巨鸟的嘴巴里,人头巨鸟开始动起来,纷纷飞了起来,迅速消失,好像接到了什么指令,或者看到了什么可怕的天敌,疯狂地逃窜。

我将五六式给胖子,让他装填子弹,然后自己小心翼翼地来到缝隙的口子上,也不敢出去,探出头看了看,顿时目瞪口呆,人头怪鸟一只一只地飞上天空,很快我们四周一只都没剩下,全跑了,四周顿时安静下来,只剩下我们两个人。

这真他娘的怪了,我给胖子打了个招呼,示意他出来,我们四处看了看,对临死前的突然转机,感觉有点不太适应。我心说,上帝,你就算真不想我死,你也得找个好点的理由啊。

我自言自语道:“它们到底在怕什么东西?这种怪物竟然还有天敌?”话没说完,胖子就拍了拍我,他看到了什么东西。

我转过头去,只见一边巨型青铜大门上面封门的人皮,不知道什么时候竟然已经全部爆裂脱落,两扇巨大的青铜门竟然向外挪开了一点,一条黝黑无比的细小缝隙,出现在两扇门的中间。

我的心提到了嗓子眼上,一身的冷汗,这么大的巨门竟然自己开了,刚才那一下巨震,肯定是门开时候的反应,如此重的门,是谁打开的?谁在里面?

从汪藏海的叙述中,这个地底巨门给描绘成了一个邪神来往于地狱和先世的通道,地门之内有着万古的邪恶,总之不是好东西,如今地门打开,难道是地狱中的邪神准备出来遛狗了?

这完全是无法预知的景象,一瞬间我脑子转了十几圈,是妖怪还是粽子?跑还是看看再说?跑的话往哪边跑?

此时的思路竟然极端清晰,我自己也开始佩服自己这种被折磨出来的心智了。

可是门开子之后,却没有任何动静,也不见门继续打开,也不见有东西出来。呆立了良久,胖子问我道:“要不要过去看看?”

但是如果进入之后,一旦大门关闭,这么巨大的青铜门,就算有一千个人在这里也无法推动,我们肯定就会困死在里面。那知道了秘密又有什么价值呢?

这其实就是选择安全地离开这里,还是冒险去得到答案。

权衡再三,我还是无法忍受这几乎煎熬了我一年之久的谜团,我一定要进去看看,到底汪藏海当年看到的魔境是怎么样的景象,到底这延续了上千年的、牵扯我们家族三代的秘密背后,是什么神秘的力量。

我看了看胖子,他也和我心意相同。

胖子把五六式给我,自己捡起他的M16,从满地的尸体残骸中调出了几只弹匣,然后擦了擦脸上的血,示意我一起过去。

大门太大了,远处看的一条缝隙,近处几乎可以开进一辆卡车,要将万吨重的巨门移动这一点的距离,需要的力量无法估计。

我压抑着心中的兴奋,走到巨门之前。我闻到从缝隙中吹出了一阵奇怪的味道,心跳陡然加快了起来,一种介于紧张和不安之间的情绪越来越浓厚,我们手上全是冷汗,连脚都有点软。

胖子先用手电照了照,手电光一人巨门之内,就完全消失,什么也照不到。汪藏海提过,当年东夏人带他来这里的时候,刚进入门内的一段是一片虚无,必须要用一种奇怪的照明工具,叫做“真实之火”,我们推测肯定使用的是犀角蜡烛,才能看到里面的情形。

我想到这里,不由一愣,心说不对,我们没有这样的设备,这样就算我们进去,看到的也是一片漆黑,不知道能不能通过那一片虚无的空间,到达魔境之内?

胖子还没想到这一点,看我不动了,以为我又害怕了,问我道:“走不走?”

我刚想说话,突然看到青铜巨门缝内的黑暗中亮起子好几盏灯火,似乎有东西正在走出来。正想拉胖子来看,胖子却也来拉我,我一回头,只见我们身下从裂谷地下的石头缝隙中,不知道什么时候开始冒起一股淡蓝色的薄雾,犹如云浪一样,迅速上升。

\chapter{无法解开的谜团}

我们退后几步,发现四周所有的石头缝隙里都冒出淡蓝色的薄雾来,而且速度惊人,几乎是一瞬间,我们的膝盖以下就开始雾气缭绕,眼前也给蒙了一层雾气一样,而且还在不断地上升。很快手电的光就几乎没有作用丁。

紧接着我们听到了一连串鹿角号声从裂谷的一端传来,悠扬无比,在裂谷中环绕了好几声。无数幽幽的黑影,随着鹿角号声,排成一列长队,出现在裂谷尽头的雾气中。

我要时间反应不过来,这里的人死的死,跑的跑,早就已经不成气候了,怎么突然又出来这么多的人?难道还有其他的队伍在这里?但是又不像,这……人也太多了。

一边的胖子脸色已经白了,似乎已经知道了是怎么回事,嘴巴打结,好久才说全了:“阴兵借道!”

阴兵?我十分不解,还想问他,没想到他捂住了我的嘴巴,做了一个绝对不要说话的手势。我们放下手电,然后直往后退去,躲到了一块大石头后面。

队伍朝着我们不紧不慢地走来,我竟然还看到了前面的人打的番旗的影子,队伍是四人一行,行走极为整齐,很快就从远处的裂谷尽头走到了我们面前,在手电光的照射下,雾气的影子越来越清晰起来。

我看着看着,不由自主头皮就麻了,只见队伍前头的人,穿着殷商时代的破旧盔甲,手上打着旗杆,后面有人抬着号角。虽然负重如此严重,但是这些人走路都像是在飘一样,一点声音也没有,速度也极其快。再一看他们的脸,我几乎要把自己的舌头咬下来,那都是一张张奇长的人脸,整个人脑袋的长度要比普通人长一倍,所有的人都面无表情,脸色极度苍白。

队伍幽灵一般从我们面前通过,并没有发现我们,径直走人青铜巨门的缝隙之内,所有的士兵都是一模一样,好像是纸糊的一样。

我和胖子谁也不敢说话,期望这些人快点过去,这时候,突然胖子按着我嘴巴的手就是一抖,我忙定睛看去,只见闷油瓶竟然也穿着同样的盔甲,走在了队伍中间,他正常的人脸和四周妖怪一样的脸实在差别太大,我们一眼就认了出来。

我几乎要叫出来,难道闷油瓶死了,魂魄给这群阴兵勾去了?

再一看却看到闷油瓶子的身后还架着他那把黑金古刀,走路的动作和边上的阴兵完全不同。我马上就知道他还是活的。

那他想干什么?难道……我突然冒起十分大胆的念头——难道他想混进去?

这小子疯了!我一下子心跳就开始加速,一种久违的恐惧涌上了心头,呼吸开始急促起来,想上去阻止他,但是胖子死死地抓住我,不让我动弹。

我看到闷油瓶注意到了我们这边,把头转了一转,正看到我和胖子的脸,他突然竟味深长地笑了笑,动了动嘴巴,说的是:“再见。”

接着他就走入了青铜巨门之中,瞬间消失在了黑暗中。我目瞪口呆地看着他,脑袋几乎要炸裂了一样。

很快整队的“阴兵”走人了青铜巨门之中,地面猛然一震动,巨型的大门瞬间便合紧成了一个整体。

我坐倒在地,一股无力的感觉瞬间生起,这是怎么回事?闷油瓶他到底想干什么,那些真的是阴兵?

胖子跑过去捡回手电,自己也是一脸惊诧地看着巨门,有点神经错乱。

可是仍旧没有时间给我们发呆,四周的雾气逐渐散去,我们马上听见了零星的怪鸟叫声从裂谷的尽头传了出来,越来越响。

胖子顿时反应过来,对我大叫:“快走!那些鸟又飞回来了,这一次咱们肯定没这么走运了。”

我给胖子一叫,顿时犹如被人泼了一盆冰水,清醒了过来,马上转身,跟着胖子向裂谷的另一头——潘子他们逃跑的方向跑去。

裂谷下的石头犹如丘陵,极度难爬,我们刚爬出不远,怪鸟的叫声已经祖近,我不由心里祈祷,如果刚才死了也就算了,如果逃过一劫后还是死在同样的地方,那真是不值得了。

我们的伤口已经从疼变成了麻,有人说人紧张的时候会忘记疼痛,但是我现在连我自己的脚也感觉不到,连咬牙都跑不快。我和胖子只好互相搀扶,竭力向前跑去,不能停,停下来想要再发力就不可能了。

我们就这样连滚带爬,直往深处跑,我很快就几乎没有了意识,不知道自己在干什么。

翻过一块小山一样的巨石,裂谷的前方出现了三岔口,三条巨大的山体裂缝出现在面前,我有点发蒙,怎么办?走哪一条?我们本以为裂谷会一路到底,能在出口处碰到潘子,我们身上没有任何食物和水,这样的状态就算三条路都能出去,不能和他们会合,也是死路一条。

跑到三苗口的地方,我们赫然看见其中一道巨大裂缝的边上,刻着一个极端难看的箭头。箭头指示着一个方向。

胖子大骂:“那老潘子果然懒惰,连个箭头也不会搞得漂亮点。”

我没想到他们还会留下箭头给我们,道:“你还管这些,管用就行了!”也不能多说,我咬紧牙关就钻入了缝隙之中。

这里的缝隙比裂谷窄上很多,怪鸟飞行得不会太顺畅,进入里面,给狩猎到的机会就小上很多,我们一进去就感觉安心了很多。

很快看到前方有手电的光亮,我心中突然一震,心说按照他们的脚程,应该早就跑得很深了,怎么这里有手电光,难道又遇到意外死在这里了?

才跑几步,却看见潘子和几个老外背满了子弹正往后走,看样子是想回来救我们。一看我们潘子大喜,然后又一呆,问道:“就你们两个?其他人呢?”

我说别提了,太惨了,快点走,后面那些鸟还跟着。

这里能听到叫声,但是上空的情况一点也看不清楚,没有照明弹,用手电去看怪鸟是看不到的。

潘子招手马上又回去,最后的人打起一只冷烟火,在前面带路,一个老外看我伤成这样,就背起了我,一行人迅速退入裂缝的尽头。

我很久没让人背了,觉得很不习惯,但是那冷烟火照起了这条缝隙四周岩壁上的大量壁画,突然又引起了我的兴趣。可惜跑得实在太快,根本无法仔细去看。

凄凉的叫声逐渐减弱,看来怪鸟开始放弃追击了,其实我们一看到潘子,心就安了很多,知道自己恐怕死不了了。他带来的人都是阿宁队伍中的射击好手,就算真的打遭遇战,也不至于会吃亏。

想起阿宁的队伍,就想起阿宁,我问潘子有没有看到她。

潘子说放心吧,那美妞给人敲昏背回来了。

跑了很久很久,缝隙越走越窄,最后只能一个人一个人通过,空气突然暖和起来,我们放慢了速度,这时候前面又出现了两个人,是守夜的警戒人,看到我们回来,都发出了欢呼的声音。

我想问为什么这里的温度会高起来,就已经看到了潘子的营地边上有好几个温泉,顿时我就彻底放松了,一种无力感顿时传遍全身,几乎就当场晕了过去。

\chapter{休整之后}

阿宁队伍的医生给我们检查了伤口,打了消炎针和动物疾病疫苗,撕裂太长的伤口都清洗好缝合了起来,胖子屁股上的伤口最严重,使得他只能趴着吃东西。

我们饿极了,虽然食物不多,但是他们的向导说这里有恬风,肯定有路出去,所以也不用太紧张。我们吃了很多糖类的食物,身体各部分的感觉都有所回归,疼的地方更疼,痒的地方更痒,十分的难受。

三叔还是神志不清,不过高烧已经退了,潘子将他裹在睡袋里,不停地喂一些水给他。

温泉水取之不绝,我们都用它来擦身体,这里的环境远算不上宜人,但是我却感觉这一把身子擦得简直是做神仙一样。

期间我把我看到的毫无保留地讲给了他们听,其他人听了都闷声不响,不发表任何议论。他们这几个老外,这一次算是见识到了中国古老神秘中诡异邪恶的一面,你说要他们再有什么想法,恐怕也困难。

其中一个动物专家说,那种生活在怪鸟嘴巴中的猴子一样的怪物,可能是远古的一种寄生关系,就好比趴在狼背上的狈一样,怪鸟可能无法消化食物,而“口中猴”帮它消化食物,怪鸟靠口中猴子的粪便为生,这在海洋之中很常见。

我不置可否,进入云顶天宫的这一切事情,节奏太快,我们根本无法透过气来,我现在只觉得自己像是做了一场梦一样,实在不想再去考虑这些东西。

不过私下里,我还是和这几个专家作了个约定,大家如果能够活着回去,在这件事情上如果有什么进展,可以通过正E-mail资源共享,希望以后我们可以不再是比快的竞争关系。

我们在原地休整了半天时间,潘子就带着几个人往缝隙的更深处探路,接着我们再次启程,向着山裂隙的深处继续前进。

洞穴专家的意见是这条缝隙应该有通往地面的出口,不然不会有流动的空气,而且出口必然是一个风口。

我当时并不信任他,但是等到我们走了将近一天时间,走着走着,突然发现四周熟悉起来,而胖子张大嘴巴指着一边裂缝上被人剥落的双层壁画的时候,我不由就控制不住地笑了起来。

这条裂隙的出口,竟然就是我们在上山时候躲避暴风雪的那条被封石封死的岩石缝隙。

我看到了我们遗留在里面的生活用品,潘子也苦笑起来。

当时我们来这里,浩浩荡荡,现在都犹如败兵,当时看着双层壁画,猜测云顶天宫中秘密的时候的那种兴奋和神秘,已经变成了无法回避的苦涩和讽刺。而且当时我们怎么也想不到,只要再往这条缝隙中走上几公里,就是九龙抬尸棺的所在。我们竟然绕了如此巨大的一个圈子。

这真是绝大的讽刺了,也不知道这个讽刺,是汪藏海留给我们的最后惊讶,还是连他也不知道的一个天大的巧合。

之后,我们很快走出了缝隙,所有人一个星期来第一次看见了太阳,全都给照得睁不开眼睛。

我们的食物基本上吃完了,不过我们不缺水,精力还算充沛,饿肚子走上一天时间应该不成问题。于是订立了路线,阿宁通过卫星电话,联系好子医生和接应,说在路上就会有人来接应我们。

我们跟着他们的队伍,缓缓下了雪线,碰上山地救援队的时候,已是在营山村外了。

所有的伤员全部被吉普车运到了最近的医院做简单处理,然后再送到吉林大学第三医院。三叔经过检查是剧烈脑震蔼和伤口感染引起的并发症,需要长时间的调理,我和胖子则全是外伤,以致我再也没有羡慕过潘子健壮全是伤疤的肉体,因为我也不会比他逊色多少。

而且,虽然我对于三叔的目的和动机还是完全不知道,但是总算是把他的人找回来,心中也颇有一种自豪感。

三叔一直要在医院治疗,直到病情稳定,我、潘子、胖子和几个老外在吉林放荡happy了大概半个月后也各自告辞。

潘子回了长沙,收拾残局需要大量的精力,后来就没什么联系了。胖子回了北京潘家园,说要休息几个月,几个老外各自回国,我只剩下一个人,一边照顾三叔,一边整理我的想法,试图使用自己先有的线索,理出一点眉目来,但是没有三叔的那一部分信息,实在没有办法把整件事情想透。

其实汪藏海那一部分的谜题都已经很清楚了:

第一,云顶天宫并不是汪藏海建筑的,而是汪藏海改建的。(但是这座殷商时期的巨大遗址,以前到底是谁为了什么目的修建的呢?)

第二,汪藏海参与到这个改建工程并不是自愿的,大部分参与改造工程的汉人工匠,都是东夏人胁迫过来,在改建工程进行当中,总司令汪藏海就开始设计了几乎横贯小圣和三圣两山的逃亡密道,以免地宫封闭时,给异族的万奴王陷葬。

第三,在改建陵寝的过程中,汪藏海逐渐隐藏了在东夏皇陵之底、长自山山体深处的众多秘密。(他在青铜巨门之内,到底看到了什么?)

第四,汪藏海将这些秘密记录在龙鱼密文上,希望有朝一日能够得世人所见。

第五,因为东夏是边境小国,国库不盈,云顶天宫的诸多奇珍异宝,都是从其他墓穴中搜刮而来,汪藏海在指导东夏军队棺倒的时候,偷偷龙鱼密文藏于这些古墓之内,希望能够有人发现。一共放了两条,最后一条,是他自己老死之前,藏入了自己的坟墓中。

第六,他为什么要把古墓修建在海底?是害怕东夏的后人断绝了这个秘密?

第七,海底墓中消失的人,出现在于云顶天宫的密室中。(除了两个人之外,其他人都死去了,但是这两个人是谁?他们到哪里去了?是不是也和闷油瓶一样,进入了巨门之内?他们到底为什么要进去呢?三叔到云顶天宫去,目的是什么呢?)

第八,巨大的青铜古树、巨大的青铜暗门,和几个地方都出现的六角铃铛,这些青铜的东西之间是不是有什么联系?它代表着一种神秘的力量,到底是什么呢?

我逐渐发现,二十年前在海底墓穴中发生的一切,才是关键。

(《云顶天宫篇》完)

◆ 第六卷 蛇沼鬼城(上) ◆

\chapter{三叔的醒来}

云顶天宫的探险结束之后大概一个多月,我一直留在吉林照顾三叔,这一次我留了一个心眼,我怕他醒过来之后又不告而别,所以我干脆就住在医院里,生活在他的病床边上。

后来发生的事情证明我是非常明智的,但是当时,其他人都不这么想。

他的病情稳定之后,却还是没有苏醒的迹象。他呼吸平稳,脸色红润,但就是没有思维反应,医生说这很正常,他伤口感染得非常严重,不知道发烧的时候,有没有伤害到中枢神经,能不能醒过来要看运气。

我没有选择,只有等,期间家里也有人来看过我几次,我都拒绝出去吃饭,因为我怕我一走出医院,回来的时候三叔又会消失。我母亲还说我傻,但是我非常坚持我的想法。不夸张地说,三叔苏醒前的这一个多月,我就几乎没有离开他超过十米。

在漫长的等待中,我也做了不少事情,云顶天宫中的所有线索,我已经整理得差不多了,阿宁公司里的几个顾问回国之后,也将他们手上的资料陆续发给了我,包括阿宁在海底墓穴后殿主棺室拍下来的十几张隐喻壁画、铜鱼之中的全部译文,等等。

所有这些归结起来,我对于汪藏海的那一部分谜团已经全部了解了,心情也逐渐轻松了起来。汪藏梅这个人可以说是一个超越时代的天才,现在他也可以瞑目了,因为他处心积虑流传下来的秘密,已经有人接收到了,虽然就是在我这个时代,我仍然无法去解释他当时看到的景象,但是既然秘密已经传承了下来,就总有解开的一天。

其他令我无法释怀的,就是闷油瓶和三叔的目的。按照我的猜测,二十年前进入海底墓穴的那几个人,似乎都在寻找云顶天宫底下的那扇巨门,似乎都想进去,而我亲眼所见,闷油瓶用一种让人咋舌的方式进去了,而藏宝室中李四地他们的尸体中,缺少的两具(不知道是谁)也可能是进去了。

他们为什么要进去呢?或者进去干什么呢?

所有的谜团都集中到了二十年前海底墓穴中发生的事情,汪藏海应该还在他自己的墓穴中留下了什么东西或者信息,这东西或是信息,是让他们全部都产生一定要去云顶天宫这个念头的原因和关键。可惜,我必须要等着三叔醒过来,才能得到回答。

另外,我还帮助胖子拍卖掉了他身上带出来了六件金器,这一次的活动,他的收益最大,这几件金器的价值十分高,其中一只西域风格的高脚镶嵌玛瑙的金杯,就拍到了四十万美元,胖子还是十分的够义气,分了一点钱给我当佣金,说是下次夹喇嘛的装备钱,我对他发了毒誓,绝对没有下次了。

时间一天一天过去,我隐隐感觉到有一丝无聊,在漫长的等待中,耐心也逐渐消耗,开始几个月还有大量的事情需要我去处理,但是后来的时间,我都是看着三叔电脑上那张黑白照片度过。我常常想,那挨千刀的闷油瓶,他现在在干什么呢?

就在我以为还要遥疆无期地这样生活几个月的时候,突然三叔的主治医生过来找我,说有要紧的事情要和我谈。

我以为三叔的病情有变,就跟他去了他的办公室,没想到到了那里,却看到三叔铺子里的一个伙计在那里。我问他找我什么事,他却吞吞吐吐,说不出来。

我突然感觉到一股不妙,忙跑回病房一看,不由咬牙切齿,三叔已经不在了。

就在懊恼不已、想去揍那医生一顿的时候,却看见三叔正给人提溜着,灰溜溜地押回到病房里来了,那押着他的人不是别人,正是我家的二叔。

我不动声色,也没有拆穿三叔,几个人闲聊了一会儿,我乘机把他昏迷之后发生的事情和我的一切推断都说给了他听,他却并不表态,只是在听到闷油瓶进去了的时候,脸色稍微有了一点变化。

后来二叔就回去了,临走让我看好这个老顽童。二叔一走,我马上就发难,问他到底是什么时候醒的,装昏装了多久?

三叔十分尴尬,但是如此被我识破了,他也没有办法,就说其实也是刚醒,准备出去上个wc而已。我这些就不和他计较了,因为说不定也是真的,我不相信装昏能装一个月,这怎么受得了,但再问他其他的事情,他索性就破罐子破摔了,就是不说,说什么和我没关系。

我急起来就骂开了,我说你这个老家伙,知道不知道我为了你的事,吃了多少苦头,还有像大奎、潘子这些跟着你出生人死的人,你是不是应该尊重一下他们,至少也让他们知道自己为你冒着生命危险,到底是为了什么!

这话已经说得十分严重,我是真的有点发怒了,特别是想起潘子对这老头子情深意重的情景,我真的说不出话来。

三叔这才沉默了下来,苦笑了好几声,叹了口气,摇头遭:“这事和你没关系,知道了说不定更苦恼,我不说,其实是为了你好,你又何必呢?”

我播头,表示就是苦恼也是我自找的,我一定要知道整件事的真相,否则绝对不会罢休。

我说得很坚决,而且是看着三叔说的,就是要让他知道,他这一次绝对逃避不了,让他不要有妄想。

这也是我这。几天学习的成果之一,我已经考虑到会有这种情况,所以看了很多心理学的文章,看怎么样才能让人放弃保守秘密的防线。

三叔想了想,又长叹了一口气,似乎终于打定了主意,揉了揉眼睛道:“唉,想不到想不到,人说儿女是前世债主,我以为不生就没事了,没想到还是给你这家伙搭上了,看来今天你是无论如何都要知道了?”

我怒道:“你还有脸说,不知道谁给谁还债,你有差点在海底被括埋吗?你有差点被猴子吃掉吗?你有……”

三叔做了个投降的手势,道:“好了好了,你既然这么想知道,我这一次就破例告诉你,但是,你必须发一个誓言,听了之后,不准和任何人讲。”

发誓我是当饭吃的,哪能当真,当即就发了一个全家死绝的毒誓。

三叔惊讶于我誓言的狠毒程度,半晌才摇头笑起来,又道:“我丑话说前头,这事不是人人都能相信的,我说了之后,你要是不信也没办法。”

我急得咳嗽道:“我现在还有什么不能信的,你就说吧。”

三叔长叹一口气,摸了半天从兜里掏出半支烟来,也不知道是什么时候的,看了看门外,看没有护士,心疼地点起来吸了一日,才道:“那是很久以前的事了,算起来,整件事情的起因,还是你爷爷在笔记本上写的,从那五十年前的晚上开始的,如果你要知道所有的经过,那我就从这件事情开始讲起好了。”

\chapter{往事不堪回首}

没想到三叔的叙述,竟然要从五十年前说起。这一次我没有把爷爷的笔记本带在身上,但是上面的内容我记得十分清楚。五十年前那天晚上发生的事情,诡异异常,但是爷爷最后却没有记述下去,他昏迷之后的事,我们都一无所知。现在我回忆起里面的文字,还是觉得心中有一种莫名的感觉。

但是三叔这样一说,我却突然有点不相信他,因为爷爷对这件事情讳莫如深,他去世之前,无论我们几个晚辈如何去问,他都没有说,三叔自小和爷爷关系不好,我相信爷爷更不会告诉他。

所以他一说,我就说道:“你他娘的可别糊弄我,五十年前爷爷都还光着屁股,他口风那么紧,你又怎么知道?你别又随便讲点故事来骗我,我绝对不会上当了。”

三叔听了不悦,道:“不和你说你急,和你说你又不信,怎么我就不能知道了?你要不信我就不说了,我还不想说呢。”

我一看他这是顺势就要反悔,马上道:“别别,我信,我只是感叹一下,你快继续说。”

三叔蹬了我一眼,想了想,才继续说了下去。

我听着听着,就发现的确是误会了他。但是事情竟然是这样发展的,我真是没有想到。

事情的起因却是那本笔记,然而过程却复杂得多。

笔记在到我手之前一直是放在老家阁楼的杂物箱里。直到我识字,翻查老东西的时候偶然看见,才到我的手里,而我的父亲和三叔他们年轻的时候,都看过这一本笔记。

三叔第一次看到笔记是什么时候,他自己也记不清楚了,只记得那时他已经出道一段时间,大小也都有过点见识,长辈之间稀奇古怪的传说也听了不少。他知道长沙土夫子中流传着“土带血,尸带金”的说法,所以一看到笔记,想到自己还没有摸到过什么特别拿得出手的东西,就马上被笔记中记录的东西吸引了。

几乎是马上,他就产生了回镖子岭那里看看的想法。古墓是不会走的,就算过再多的年限,应该还在这里才对。加上解放初期山林深处还有土匪横行,不会有很多人进入。他相信古墓中应该还有东西剩下。

但是,镖子岭只是爷爷小时候那个地方的一个土名而已。这种名字可以指一个小土包,也可以是整片山甚至是全部的原始丛林未知区域,所以光靠一个地名去找那座古墓,是不现实的。

那么,怎么才能确定那个地方的准确位置呢?三叔琢磨了很长时间,一直没有头绪,直到他到西沙去的前一年,终于有了线索。

那一年他去了长沙爷爷的老家,老家在山区,他走了四天的山路才到达那个偏僻的农村,在那里和当地人打听云镖子岭的深位置,那一次虽然没有得到直接的信息,但是却大大地熟悉了那边的风土人情。

回来后再一次研究笔记上记录的东西,事情就明朗化了。按照爷爷笔记中的其他内容,和三叔小时候偷听爷爷讲话时的记忆,加上那边打听来的一些事情,他依稀判断出,那座古墓应该坐落在莽山的鬼子寨附近。

因为在笔记上爷爷提到过,太公和爷爷在蟒林中赶路的时候,都被一种“铁头蛇”咬了,这种蛇经常盘在灌木之下,很难发现,当时危害很广,后来开展打蛇运动,却一举把这种蛇打得濒临灭绝。当然这是后话。

那时候的土夫子天生天养,被毒蛇咬过之后,往往只是吸出毒液,拍上点烟叶,吃上几口土药,没有更好的处理办法。这样处理之后、如果过几个时辰,被咬的人没中毒反应,也就没事了;反之,一般来说也就没有挽回的余地,只有认命。

当时咬了他们两个人的蛇都是小蛇,伤口不深,所以爷爷他们也没有在意,简单处理后,也没有感觉到什么特别的不妥,于是二话没说继续赶路。没想到走出两里地去,爷爷就突然摔倒,接着就不省人事了。

他们停下来仔细一看,只见爷爷皮肤发青,不停地痉挛,显然是蛇毒发作了,后来太爷爷赶了几十里山路找来当地的山民,才用草药救了爷爷一命。

爷爷他们于是在原地休息了两天,而根据爷爷当时对瀑布的描述,可以肯定他们休息的地方应该是鬼于寨。

这件事情发生之后的第四天,他们到达了那个叫做镖子岭的地方。那地方地处山谷中的平原,四面都是山,谷中蟒林丛生,特别多的千年老藤,只有山谷最凹陷处的一块,却没有任何的植物,露出一片血红的裸土,那座古墓就在山谷之下。

如今讽刺的是,咬我爷爷的那种蛇已经是濒危动物,其一条成年蛇的价值出口超过百万,远远超过普通明器的价格。

这样一来,找到的希望就大了很多,虽然莽山的原始丛林在那时候幅员辽阔,远比现在鬼于寨瀑布位于丛林的中心部分,但是推测出来的相同地貌却不是很多,并不难找。

三叔整顿行装,再次出发。三叔习惯独来独往,因为他年纪太轻,老人不愿意和他一起出去,同年纪的身手能及上他的又没有。

然而等他历经干辛万苦,穿过当时几乎没有人烟的莽山丛林之后,映入眼帘的,却是他做梦也没有想到的景象。

\chapter{WHO ARE YOU?}

三叔按照当地人的指示,沿着一条不知名的,先民开出的小道在山峦中走了大概四天时间,这条小道有三分之一段都开凿在峭壁腰子上,据他估计已经荒废了几百年,原来可能是属于行军的栈道,现在青苔丛生,草木覆盖,越往里走就修造的越粗糙。

小道一直往森林的深处衍生,外面的一段还经常有山民使用,到了过了鬼子寨一带,更里面的道路就几乎无人涉及,坍塌的坍塌,给树藤覆盖的覆盖,几乎无法前行。

三叔凭着那股偏执的劲,几经辛苦穿过这条古道,来到了悬崖的另一端,他居高临界下,此时笔记中记载的山谷,就在他的身上,经过了二十年的风雨变迁,爷爷他们来时候的足迹早就消失在了极端茂盛的树冠之下,但是山谷中间裸露的一个红色裸土包,却突兀非常,极端的显眼,告诉他此地就是传说中的镖子岭。

同时他也看见,红土包的一边的树冠下头,似乎立着什么奇怪的东西,因为颜色与树冠相近,所以在他的高度,他无法分辨那是什么。

他隐约感觉到不对,这里是人迹罕至的山谷,任何人工的建筑或者活动痕迹的都不应该出现在这里,所以他爬高几步,掏出望远镜去查看。

一看之下,他就愣在了那里,只见土包边上的树冠下面,零零落落的立着几顶军用帐篷,帐篷是迷彩的涂装,所以在远处很难分辨,要不是三叔在明器鉴定中对于那种细小的颜色区别和异样非常敏感,刚才的一撇可能就会看漏。

当时三叔心里打了几个嘟噜,心说这鬼地方怎么会有人在?而且还支起了帐篷,应该不会是猎户,猎户不会来这么深的地方。

正纳闷着。忽然其中一个帐篷一抖,从里面出来了一个人,三叔抬起望远镜一看,一下子就更纳闷了。

原来出来的人,一头棕色的头发,身上是四楞子起金线,竟然是个洋鬼子。

三叔那时候还不能分清东西北欧人种的区别,但是那个年代改革刚开放,来中国的洋人也不多,最多的还是富有冒险精神的美国人,所以他也没考虑,就认定这个洋人是美国的人了。

他当时一琢磨,这地方有人就有问题了,现在不仅有人,还是个洋鬼子,他们在这里干什么呢?难道是美帝来搞破坏了?又或是——也是为了这镖子岭地下的古墓而来?

可是洋鬼子虽然好古董人尽皆知,但是他们也不至于自己来挖啊,他们又没看过老头子的笔记,如何知道这里的地下有墓葬呢?

这简直是八杆子打不到一会儿的事情,三叔根本就无从想起,心里奇怪到了极点。

他怀着疑问爬下悬崖,放下自己的装备,轻身穿过下面的莽林,潜入到帐篷附近。发现这些洋鬼子的营地就在红色土包的边缘,大约有4个帐篷,估计人数不会很多,一边还有几个当地人模样的中国人在吸烟休息,他同时还看到一边的土堆上面已经给开了一个大坑,上面盖着一个用竹子搭起的架子,盖着绿色的防水布,因为这些东西在他视野的北面,所以刚才在悬崖上的时候没有看到。

一边红色的土包应该是当年的封土堆,这些泥土都应该给炒过,添加了一种丹药,使之无法生长植物,但是现在走近一看,还是有很多的杂草长了上去,显然古人低估了植物的适应能力。

三叔看到那个嗽叭口状地的坑,马上明白了这些美国人的目的是和自己一样。

当时三叔地年纪不大,看到这个情形,脑子里勉强想到的是,这可能是中美合作的考古队,跑到这里来做考古挖掘了,这似乎是当时唯一合理的解释。

如果北派,这个时候只有自认为倒霉,因为他们的规矩,私不与官争,如果遇到了考古队,你还能如何,你总不能上去杀光他们,但是三叔不同,他不甘心就这样给人截胡了,看着美国人挖掘的位置和力度,他知道这些人没有土夫子的经验,肯定是就是按照自己国外挖公墓的办法来对付中国的古墓了,这样挖是绝对进不了古墓的,他只要找对地方,下个盗洞下去,神不知道鬼不觉,就能在他们进入古墓之前把东西全部都带出来。

三叔回到自己下来的地方,拿回了自己的装备,此时日渐西斜,他在黄昏中以自己的脚步为尺,穿行了山谷之中,丈量了土丘四周的面积,寻找最合适的打洞位置。

期间过程非常复杂,三叔也没有详细说明,他只告诉我,他当时对自己很有信心的,唯一担心的是古墓之中的情况。

当年爷爷挖出来的盗洞,不会保存很长时间,肯定在几次雨季过后就会坍塌,不知道当时他们到底进到了哪个地步,是不是已经进入墓室地宫的内部,如果是这样,墓室之中可能已经积了雨水,那么除了棺椁里的东西,其他的陪葬品可能已经泡烂了。而棺椁里的东西是否遭殃,还要看棺椁的质地和当时密封的程度。

入夜之后,洋人的营地燃起了篝火,三叔静静的等待着,直到他们全部都睡去,他才小心翼翼的使用自己的“猫铲”开始挖掘。

猫铲是土夫子一种特制的铲子,挖掘起来声音非常小,但是现在工兵铲的锋利程度和声音已经比猫铲还要先进,所以猫铲已经退出历史舞台了,但是当时猫铲却是三叔能使用的最安静的东西了。

即使如此,三叔挖的时候还是非常的紧张,因为无法使用洛阳铲探知地下的情形(一打声音就起,而且不知道为什么,洛阳铲的声音进入地面的声音,特别容易惊飞野鸟。)所以他也没信心能一次就找到古墓的外延。

挖了大概2个小时,盗洞下去大概6米多深,三叔的铲子终于碰到了坚硬的东西,正当他凑过去,想用手电照一照的时候,突然他就感觉不对,泥土下面一阵轻微的蠕动传来,紧接着,整个盗洞就坍塌了,他一声惊叫都来不及发出,口鼻就给泥土盖住,接着他就连着他四周的泥土一下子陷进了地底深处。

\chapter{血尸古墓}

凭着本能,三叔不停的叭啦着四周的泥土,想探出头来呼吸,或是抓住四周的什么东西,然而这是徒劳的,大约也就是两三秒的功夫,他就感觉身下一空,掉入什么空间中,接着浑身一凉,连着裹着他的泥一起掉进了水里。

冰凉的水一下冲掉了他脸上的泥,咳嗽着挣扎爬起来,四周是一片漆黑,他不知道自己掉进了什么地方,他只能感觉腰部以下的部位全部都是水,而且四周弥漫着一股奇特的腐臭味。

手电还亮着,现在掉进了水里,只露出一小点电光,三叔附身将手电摸了上来,因为泡了水,才摸上就暗了,他甩了两下,手电才又亮起来,但是光线明显有点发暗。

他用手电照了照四周,发现自己掉进了一个砖室,四周是四楞青转垒砌的峭壁,往身后一看,只见身后的青砖墙上有一个貌似人工开出的大洞,显然刚才自己就是从这个洞里滑进来的。

三叔看了一圈,就明白是怎么一回事情了,他刚才挖掘的地方有问题,似乎是一个用土掩盖的空洞,他的体重压在上面,下面并没有支撑,所以整个盗洞下方的泥土就坍塌了,和他这些泥一起滚进了下面的墓室中了。

那墓墙上的洞是谁开的呢?难道自己无意中挖到了当年老头子他们进墓穴时候的盗洞?有这么巧合吗?

三叔想了想。觉得还真有可能是这样,自己的本事是老头子教的,老头子的本事又是上一代教的,因为墓葬这种事情从清朝以后就开始退化,所以盗墓技术一直就是吃老本,没有本质上的进一步的发展,盗洞在哪里打,如何打都是死规矩了,同一个师傅教出来的徒弟,很可能就会把盗洞打在同一个位置上。

暂且不去想这些,他仔细观察了一下四周,后面的入口全是倾泻下来的泥土,铲子不知道裹在泥里的哪个位置,要想从原路回去恐怕有点困难。不过他并不担心,身上带着炸药呢,实在出不去,就给他开个天窗吧。

墓室是一个正规的四方型,拱顶,四周都有简单的浮雕,墓室不大,不过高。墓室里的积水到了他的腰部,陪葬品应该在水下。但是一潭黑水,根本看不到下面有什么。

在左边的墙上,开着一道门,似乎是这座古墓的甬道。

单凭这些根本无法判断古墓的朝代和主人的地位,但是看这墓室的高度,这里的墓主人显然并不是王侯等级的人物。

一般的古墓,有墓室的,规格已然不算低,因为古时候能住得起砖头结构的房子的人已经不多,如果要用砖来修墓,墓主人怎么样也需要是一个官宦阶层。迹无深云不过即使是官宦阶层,古墓之中大多数不会有太邪门的机关,因为他们的能力有限,历朝历代,顶级的工匠,特别是掌握陵墓的建筑知识的,都只为皇帝一个人服务的,而且他们一辈子大概也就能服务一次,大批顶级工匠都在皇陵封闭的时候死在里面了,这也是为什么中国有这么多东西失传的原因。

三叔镇定了一下,趟着水向黑暗的甬道中走去,水冰凉而且阻力很大,走起来带着一条条波纹,发出一种让人非常不愉快的声音。

水下的墓室地面并不平坦,好几次他都踩到东西几乎摔倒,这个时候他也无法去思考他踩到的到底是什么,如果这里就是当年笔记中记载的古墓,那他踩到的,除了这里的陪葬品外,还有可能就是长辈们的遗体了,这种事情太刺激了,最好的解决的方法就是不去想。

甬道大概有二十米长,很快就走了过去,甬道的后面是另一间更大的墓室,四周已经没有其他甬道,三叔知道这里已经是后殿,走近几步,墓室的中间有一座棺床,高出水面。

三叔的手电照去,不由咽了口唾沫,脚有点发软起来。

只见棺床上面,摆放着一只石棺,棺材的盖子已经翻到不知哪里去了,这样的情形并不罕见,但是让他有点惊惧的是,另外还有两具腐烂的枯骨,靠在无盖的棺材上面,身上的衣服已经破烂,两具尸体已经完全腐烂,皮肉都已经和石棺粘在了一起,远远的,看不清楚是何朝代,但肯定不是殉葬的奴隶。

三叔愣了一段时间,浑身发凉,不敢过去,心里暗道,这两具,难道就是当时死在古墓中的自己的亲人?

古墓他不是第一次进,古墓中的尸体,他早就练成了无视的心态,对于他来说,这些尸体只不过是物件,但是这一次他遇到的可能是自己亲人的尸体,他心中有一种莫名的恐惧,心跳得厉害。

他缓缓走上墓室中间的石台,人都在发抖,手电都拿不稳,先看了看石棺,只见一片干涸的血块凝结在棺底,里面似乎裹着丝绸,但是却不见尸体。再凑近两具尸体一看,只见尸体腐败殆尽,头已经是骷髅,根本无法判断是不是自己亲人,但是三叔看到其中一具尸体手上,拿着一把匣子炮。

三叔膝盖一软,跪了下来,端端正正的磕了两个头,三叔不是一个感情多细腻的人,这个时候的行为,应该是一种本能。

磕完头之后,三叔顿时觉得轻松了很多,他看了看匣子炮,早已经绣得不能用了,于是扔到一边,去看石棺中的东西,他带上手套探入棺中,按了按棺底的丝绸。

一般很少有人会研究棺材之中明器的摆放,其实棺材里面也分很多层,尸体只是在中间,上下都应该有几层绸缎和天丝棉的被褥,绸缎之间每个位置都摆放着特定的明器。

三叔按了一下之后,就知道尸体并不在腐烂的绸缎下面,反倒给他摸到,在棺材地下一塌糊涂的秽物下面,有一个环状的东西。他伸进去一摸,心里咯噔了一声,竟然是一个铁环,套在棺底。

他把手电放在石棺的边缘,然后双手扣住铁环,用力一拉,只听噶本一声,突然棺材的地板翘起了一边,棺材的底下露出了一道暗门。

三叔脑门跳了起来,想不到这墓穴还不止一层,随即掏出一个火折子,刚想抛入下面的暗门中,查看下面到底是什么地方,没想到手刚探过去,正照到一张满是簸皱的怪脸,从暗门中探了出来。

\chapter{怪脸}

三叔当时就蒙了,脑子嗡的一下,头皮的毛孔都倒竖了起来,大叫一声,一撒手,提起的暗门又摔了下去,就听砰一声,正砸到那怪脸面门。

三叔也顾不得砸的如何,马上条件反射的后退几步,远离石棺,心脏几乎就要从喉咙里跳出来了。

他心说那是什么东西!难道这已经泄了阴气的古墓之中,还有一只血尸?不可能啊,那粽子都是死物,只要墓室一开,里面的墓气一泄,外面的空气一对流,短时间内再厉害的诈尸也必然伏尸,再次开始腐烂。没道理可以尸变几十年这么离谱啊。

而且刚才一瞬间看到的那张怪脸,太难以形容了,三叔从来没见到过如此恐怖的脸孔,那肯定不是普通的粽子。

难道这就是传说中的血粽子?三叔突然想到。但是他琢磨了一下,心里也实在没谱。

长沙一带关于血尸的传说最多,一般是指在红泥地中挖出的古墓,不管墓中情况如何,都被人叫做血尸墓。

红泥地又叫血地,一铲下去泥中带血,谁也无法解释这种地象是怎么形成的,但是所有的风水流派中,对于血地埋尸的说法都是惊人的一致,那就是适宜深埋,葬于此地刹气极重,后代必然极其显贵但是是亲戚死绝,说不定能当皇上,但是家里人全部都会给克死。

即使有着这样的传说,但是很多大户人家还是为了让自己的后代显贵,寻找这一种诡异的地象。

为了逃避煞气,他们会在入葬前,找一户同姓的穷人家,把自己的孩子过继过去,再收入自己家养。

但是天下之大,宝穴和刹穴一样稀有,血地更是世间罕有,比一般的龙脉更加难找,到了后来,大量半桶水的风水先生,看到只要是红泥地,就算是一血地了,以至于红泥地下必有古墓,古墓必是大户人家,陪葬丰富,所以才有“血尸护宝”这样的传说。

在中国近代史上,就有一位极度重要的人物,他的主坟就是血地,当时一位高人设下风水局,为了赶在那一个时间入坟。相传那位人物的爷爷,甚至可能不是正常死亡。

这位人物后来的地位权倾天下,但是正如风水术数中所说,煞气太重,自己的直系亲属,基本上都死绝了。

因为血地其假参半,而且假者居多,所以当时老头子才会冒险下铲,图一个侥幸,没想到这一铲子就挖出了一个真家伙来。

而真正的血尸墓极度地凶险,这从世界上没有任何文宇或者口头记载血尸的情况就可见一斑。见过血尸的人,几乎没有人话着全身而退,爷爷已经是一个特别的例子,而爷爷在笔记里的记述也不清不楚,血尸到底是个什么情况,他自己也不知道,更不知道如何克制,如果这下面的东西真是血尸,暂且不去管它为什么还在这里,如何脱身已径是一个大问题了。

老头子他们上次进这墓穴,必然带着黑驴蹄子,当时他们每人还都带着24响的匣子炮,这样的装备,却一个人都没退出来,说明当时情况凶险到什么地步。自己这一次更不济,只有腰上一把砍刀,拿砍刀砍棕子等于磨刀,是最傻的行为,一点用处也没有。

三叔一退之下的这一秒多钟里,脑子转的像飞一样,但是关于血尸他毫无头绪,一个办法也没想出来。

正骇然不知道怎么办,突然棺材里面发出了一连串石头磨擦的声音,按着,他就看到那盖住暗室的石板,竟然给什么东西顶了起来。

三叔一看不好,这东西要出来!

当时他也是有点蒙了,也不知道是琢磨了什么东西,三叔脑子一热,杀心就起来了。他把心一横,大喝一声,竟然有胆子跳进了棺材之内,双脚一个用力跺,一下子就踩在了抬起来的石板之上。

往下一看,就看见从石板下方伸出来一只酱绿色的干瘪人手,指甲有手指的两倍长,整只手就像生了锈的青铜器一样,长满了绿花。现在给三叔一压,就给夹在了缝隙里。

三叔看到那手,直觉得后背鸡皮疙瘩都起来了,用力一踩就想把它给夹断,但是那手硬如钢铁,猛踩了几下一点反应也没有。

接着石板之下就有一股力量往上猛抬,三叔本来就站立不稳,险些就摔下去,他忙矮身,稳住身体,双手把住棺材的两边,用力就往下顶。

这就是生和死地较量,下面的东西只要一出棺材,在这里的环境下,三叔知道自己的必死无疑,绝对不能让他出来。

但是人的力量是有限度的,下面的力量极大,三叔顶了几下,两只胳膊马上已经到了极限,再多一份力气也使不出来了。而下面的石板却还是一点一点给顶了上来,接着,那张怪脸就从石板下面挤了出来,面无表情的看着三叔。

亏的棺材里面一片漆黑,并不是很看的见,幽幽的鬼魁一样的脸孔也模模糊糊。

三叔此时已经进入一种疯狂的状态中去了,一看那头挤了出来,浑身的毛孔都几乎收缩进肉里去了,牙也越咬越紧,恨不得一刀把那头给跺下来。

就在这个时候,三叔脑子里突然就灵光一闪,想到了一个主意。

他重心一压,左手从包里掏出白酒,一下砸在那怪脸的面门上,瓶子粉碎,酒洒了那怪物一脸。接着,他就掏出了火折子,在边上一擦。然后就伸向那张怪脸,心说,今天老子就替我的爷爷大伯报仇了,你就安心当蜡烛吧。

然而那火折子一靠近那怪脸,三叔在火光下,就突然发现那张脸,有一点不对劲。

\chapter{无法接受的真相}

刚才翻开暗门时候的那一咋呼,和这东西打了一个照面,也就是半秒左右,加上那一下的吓唬,也不可能仔细观瞧那东西的模样,脑子里只有一个大概的印象。可是现在,僵持之下,火光之中,那张诡异的面孔就清晰的印在了三叔的眼前。

三叔咋一看还只觉得慑人,什么粽子他没见过,湿的干的,没脑袋的两个脑袋的,安详的狰狞的,他天生神经就大条,15岁之后就再没怕过这些东西,但是这张脸他娘的太邪门了。

那怪物的脸是青铜色的,皮肉收缩,皮肤都龟裂成鳞片状,一边都剥了起来,两只眼睛没有瞳孔,单是偏偏你又觉得他就是在看着你。

三叔就琢磨着这不像是粽子啊,粽子再难看,至少也得像个人啊,怎么这东西,看着像条蛇呢!这该不是妖怪?

而且最让三叔纳闷的是,越看这张脸,心里好像有一种奇怪的感觉,但是是什么感觉,他又实在说不上来,搞的自己的脖子就不停的冒白毛汗。

他的手越来越没力气,那怪物面无表情的又挤出来了一点,三叔知道不能再瞎琢磨了,当下把火折子往那脸上一扔,火哄一下就起来了。

三叔喜欢的酒,是一种绿色的“烧刀子”,上海人好像叫做绿豆烧,三叔喝的是乡下人自己酿造的,那都是基酒,度数极高,一点就就烧起来。这酒他到现在还喜欢喝,不过对于他这种年级来说,这种酒已经像慢性毒药一样了。

那张怪脸一下淹没在火焰里,再也看不清楚,四边的东西开始滋滋冒起白烟,皮肉都开始融化起来,一股极其难闻的味道扑鼻而来。

棺材里的陪葬品大部分都盖着潮湿的腐烂丝绸,现在也给烧的吱吱响了起来,索性并没有直接点燃。

三叔尽量摒住呼吸。火烧了大概六七分钟,酒精就烧完了,三叔发现这一招起了作用,下面往上顶的力量慢慢消失了,随着火势越来越小,那脸也腐蚀殆尽,露出了里面已经烧的焦黑的骷髅。

三叔恐防有变,还是没有放松脚下的力量,一只手还是撑,另一只手拔出腰间的砍刀,去拨弄那只骷髅。

拨弄了两下,发现并没有什么反应,三叔用力对了脖子砍了两下,把颈骨砍断,才松了一口气,确定这玩意真挂了。

一放松,他浑身就脱了力了,两只手的力气迅速就消失了。脚一软就坐倒在棺材里面大口的喘气。

不过此事还不算完,三叔休息片刻,惦记着石棺下面密室的事情。心说这地方不能久待,整个墓室里已经烟雾弥谩,本来空气就已经不多,这下子更不够用,要抓紧时间看看下面有什么,要是没什么好货色,咱就快点反打盗洞出去吧。

他捡起一边的手电,咬在嘴中,再一次拉起石棺低下的暗门石板。

无头的血粽子就平躺在石板下面,那是一具身材魁梧的男性湿尸,衣物也已经腐烂殆尽,只剩下很多的布条粘在身上,浑身呈现一种青铜的锈色,最恐怖的是,身上长满了很多类似于眼睛的皮肤褶皱。

三叔按了一下它的胸膛,感觉钢硬如铁,不由庆幸,要是刚才自己顶吃不住,肯定是九死一生。

这个时候,一个非常寒人的景象,突然让三叔楞住了,一股极度的寒意,一下子就从他的脚底心冲上了脑门。

他突然发现,这一具血粽子,他缩在石扳下面的右臂,竟然只剩一截!手肘以下的部分已经不翼而飞了。

三叔心里“咯噔”了一声,脑子里顿时就乱了,马上俯身看那断手的断口,只见皮肉果然都是犹如棉絮一样,呈现炸裂的形状,三叔忽然浑身一软坐倒在地。

我本来觉得三叔的叙述过于累赘了,但是一听到那血粽子竟然只有一只手,我顿时就明白三叔为什么要讲得如此详细。

犹如棉絮一样,呈现炸裂形状的伤口,那是枪械短距离扫射才会形成的,也就是说,这血粽子的手是被枪打断的!

这些情况,加上爷爷笔记里的记载,和三叔的表情,我大概已经猜到了事情的发展,顿时我也感觉到一股毛骨悚然的寒意,从我的后背蔓延上来。

但是如果事情真的是这样发展的,那整件事未免也太不可思议了,简直变成神怪小说里的情节,我实在是不太敢相信。

三叔摸了摸身上,想再找根烟抽,但是显然身上已经没有了。我摸了一下屁股口袋,发现还有半包,是在酒吧happy的时候从胖子那里要来的云烟,递给了三叔。

三叔再次点上,狠狠抽了一口,才道:“我看到那具血粽子的时候才明白,我老头子你爷爷,他那破本子上写的东西,可能还有着什么隐情,也突然知道了,为什么我问他当时发生的事情,他不想提起。”

爷爷当时对于他笔记上的东西,无论我们怎么问,他都是一句话,说这不是小孩子能听的故事。当时我们都不知道为什么,现在终于知道了,但是真相却是如此的惊悚。

三叔看了看我,道:“大侄子,你这么机灵,相信我不说,你也知道是怎么一回事了。”

我不敢点头,因为我想到的事情实在太不可思议了。

从爷爷的笔记上可以知道,在爷爷从盗洞之中拉出战国帛书的断手之前,古墓之中响起过一串匣子炮响,也就是说,在古墓中的叔伯,可能就是因为这一梭子盒子炮,给打断了右手。

而古墓暗室中的那具血粽子,竟然也没有右手,而且伤口呈现棉絮炸裂状,那结论就很可能只有一个:那血粽子不是古尸,而是我叔伯尸变而成的!

按照我的推断,事情的经过可能是这样的:

当时他们下到盗洞之下,必然也如三叔一样发现丁棺材下面的密室,以笔记中叔伯的性格,他必然是争着做先锋的人,肯定是抢在其他人前头,第一个下到了棺材底下密室里。

而就是在那间密室之中,叔伯发现了那一卷战国帛书,就在他拿出帛书,准备退出密室的时候,突然发生了什么恐怖的变故。

变故突起的时候,叔伯应该还有应付的余地,所以他还能将手探出密室之外,但是等他自己想出来的时候,可能已经晚了,不得已之下,或是他自己,或是太爷爷,开枪打断了他的手。

断手被在墓室外的爷爷用土耗子拖出了古墓,而叔伯却田在了密室之内,最后,竟然变成了那样可怕的怪物。

而在外面试图救出叔伯的太爷爷和太祖爷爷,也受到了牵连,死在了棺材边上。

然而,最后追出墓室的那血红色的东西,和后来的怪面巨尸是怎么回事,这里就解释不清楚了。我觉得,有可能那血红色的东西,就是中了招、却还残存意识的叔伯,但是爷爷当时绝对想不到那一点,把他当成了古墓中的怪物。

当然,事情是不是如此,只有当事人才知道了,现在这样的推测,就算再说得通,也只是推测而已。

我把我的想法试探性地一说,三叔表情复杂地看着我,点了点头。

这时候我想到一个问题,我问道:“不过,爷爷既然对我们说‘这个故事不是小孩子可以听的’,说明他也知道了当时他开艳打的,可能是自己的哥哥,照理说他不可能知道这件事情啊,那难道爷爷之后也回去过这个古墓?而笔记中没有记载接下来的事情,就是因为事实太过骇人了?”三叔皱起眉头,道:“我也有同样的疑问,但是这一点已经无法追究了,老头子已经死了,我们永远不可能知道当时的真实情况是怎么样的。”

我问道:“那接下来呢?你有没有下到那个棺材地下的密室里去?”三叔又狠狠地吸了一口烟,几乎吸掉了五分之一,说道:“你要是我,会忍得住不下去吗?”

我心中苦笑,心说我要是你,翻暗门的时候就给吓死了,还哪里会有机会琢磨下去不下去。我摇了摇头,道:“我哪能和您比啊,您侄子我的胆子您也不是不知遭,您就别哪壶不开提哪壶了,快说吧,那密室里,到底有什么东西?”

三叔叹了口气,道:“我先给你看样东西,然后再慢慢告诉你。”说着,他从他病床边上的柜于里抽出了他的背包,从里面取出了一个小的象牙盒子。

我接过来一看,盒子是清朝的珐琅原盒,是还没有上珐琅彩的毛坯盒,很沉,打开一看,盒子里面放着一颗黑色的、丑陋的卵石,就好像现在建筑工地上的黄沙堆里经常看到的那一种。

“这是什么?”我奇怪道。

“这石头,就是我从那间密室中拿出来的东西。”三叔道。

我“啊”了一声:“就是这东西?”又仔细地看了看石头,看不出什么蹊跷来,刚想用手去拿,三叔就把盒子盖了回去。“别动,这东西有点危险。”他道。

我把盒子还给他,奇怪道:“这好像只是普通的石头啊,那么诡异的密室里,放的就是这个东西?”

三叔又叹了口气,好像他们上了年纪的人,老是喜欢叹气。他道:“你别看它不起眼,我当时为了拿这个东西,差点就没命了。”

在推断出血尸的真相之后,三叔震惊得失了神,坐在地上愣了很长时间才缓过劲来,他心中乱成了一团,看着离他两尺开外的密室人口,心说那黑暗之内,到底有什么神秘的力量,可以使得一个人变成那种样子。

三叔和我一样,也是命犯太极的人,绝对受不了好奇心的折磨,只不过我胆子太小,经常遭受好奇心和恐惧的双重折磨。而三叔就不同,他只是犹豫了一下,就决定要下到密室之中去看看,知道个究竟。

现在想想,这其实是非常不明智的事情,大概也只有三叔这样的人,在那种情况下还会作这种决定。

爷爷之所以不肯教三叔太多的本事,也是因为他做事冲动,事实证明爷爷看人是相当准的,只可惜,老人的经验,小辈们经常是不听的。

休息了片刻,三叔便开始准备。他先是收拾了先人的尸骨,脱掉外衣,撕开之后,将棺材外的两具骸骨收拾一下,包人衣服之中,然后戴着手套,用捆尸带套住血尸的两腋,拉出了棺材,恭敬地摆到一边,又把砍下的头颅放了回去,对着三具尸骨,叩了三个结结实实的响头,说道:“不肖子孙吴三省,心智愚钝,冒犯先人遗体,请先人见谅。”

磕完头,他就把砍刀别回腰里,又取出雷管插进腰带,纵观全身,确定一切没有什么破绽了。

他收敛心神,来到棺材边上,再一次抬高了暗门,小心地往里面观瞧。

暗门之下,果然是一条暗道倾斜向下,不过,出乎意料的是,暗道很矮,矮得似乎只能匍匐爬进去。

暗道口的长宽和棺材等同,刚才这一具“血粽子”就是躺在暗道之内,也亏得这下面地方狭窄,那“血粽子”就是天生的巨力,也使不出力气,不然就凭三叔的力量,如何能将他压住?

三叔先打起一个火折子,丢了进去。火光一路打滚,直掉进暗道深处,最后停了下来,变成一个小小的光源,照出了一个大概。

接着他摸了摸腰里的砍刀,说了一声祖宗保佑,深吸了一口气,小心翼翼地缩起身子,缓缓钻入了暗道之内。

暗道中弥漫着一股难以言语的恶臭,三叔甸甸趴下,不得不屏住了呼吸,往里面挪动,等身体全部进入之后,脚一带上面的暗门,暗门又自翻下。

四周一下子安静得异样,只剩下前方的火折子燃烧的噼啪声。三叔有点莫名的紧张,身上已经全是白毛汗,他勉强镇定了一下,摸出手电,打亮向前面照去。

手电的光线比起火折子要强上好多,一下子就照得很远,他看到密道是由一块块的黑色石板垒成的,大概三米一截,一截连着一截,一直贯通到深处。整个密道非常清爽,四周的黑色石板也修整得十分平滑,没有任何的装饰,一眼看去,就好像老式中央空调的通风管道。

前面火折子火苗的大小、颜色都很正常,密道里的空气应该和外面是连通的,呼吸应该没有问题。

三叔定子定神,咬着手电,开始向密道的深处爬去。

我也有过在狭窄密道中爬行的经验,知道绝不轻松,三叔虽然体质比我好得多,但是只爬了几步,也感觉到呼吸急促,加上他还要不时提防四周,爬得就更加辛苦。

爬了有十分钟左右,前面有了一个转弯,三叔转了过去,他以为后面还是同样的密道,可是等他一转,却发现他的面前,出现了一面雕刻着浮雕的黑色石墙。

三叔先是一愣,呆了好久才意识到,原来密道已经到头了。

这是怎么回事?他晕了,原本以为密道的尽头应该会有一个出口,然后里面会有一间密室,而所有的秘密,应该都是在这间密室之中。

然而现在什么都没有,密道只不过延伸了一点点,就有一块黑色的石墙挡住了去路。

难道叔伯当年进来的时候触动了什么机关,把密道封闭住了?

三叔敲了敲石墙,发现石墙的后面好像是实心的,又查看了一下四周的结合处,发现这面石墙是封死在这里的,也就是说,不是什么机关,这里就应该已经是密道的尽头了,当年叔伯进来,应该也是爬到了这里。

那就奇怪了,如果这里就是密道的尽头,那这里肯定就是当年叔伯盗出帛书的地方,但是这里什么都没有啊,当年战国帛书放在什么地方?难道就丢在地上?

三叔转了个圈,看了看密道尽头的四周,又打量了一下拦住去路的石墙。

这个时候,石墙上的浮雕引起了他的注意。

那是一个人面鸟身的神灵,鸟身犹如夜枭,而人脸十分古怪,雕刻得十分夸张,脸盘有洗脚盆子大,张着嘴巴,流云行鬓,面无表情,不知道是男是女。

(我听到这里,“啊”了一声。)

三叔注意浮雕的嘴巴处有一点凹陷,比画了一下,发现当时的帛书可能是卷成一卷,放在丁浮雕的嘴巴里。

不过,浮雕的嘴巴是实心的,也就是说,在拿出帛书之后,没有什么机关会被触发。

他又抬头看了看浮雕脸的其他部位,鼻子、耳朵、眼睛,最后,他的目光和浮雕的双眼对上了眼。

浮雕人脸鸟身,有四只眼睛,还雕刻了圆形的瞳孔,但奇怪的是,上边两只眼睛的瞳孔是向外突出的,而下面两只眼睛的瞳孔是向里凹陷的。也就说,分别用了浮雕雕刻方法里的阴刻和阳刻。

这是三叔从来没有碰到过的,不要说是他,就算是我,也知道这是绝对不可能的,所有的浮雕,要不都是阴刻,要不都是阳刻,不可能混在一起乱来。

三叔凑近过去自己去看,不由“啊”了一声。他发现,原来浮雕瞳孔部位的石头,和浮雕并不是一个整体,而是有一块黑色的丑陋卵石镶嵌其中,奇怪的是,上面两只眼睛的卵石还镶嵌在里面,而下边眼睛内的两颗却被人挖走了,只留下两个球形的凹坑。

三叔看着那两只眼睛,心里逐渐明朗了起来,一个大胆的推测出现在了他的心里面。

\chapter{四目九天娘娘}

三叔所说的石墙上这种人头鸟身的神灵,在各地各民族的神话传说中都出现过,我相信应该和我们在云顶天宫中看到的那一种怪鸟是同一种生物。

我后来查过,知道这种神灵在古埃及被称呼为:ba,它代表人不朽的灵魂,也就是说如果你在古埃及,那么他们的鬼都是这德行的。在印度就规范一点,这种神被叫做“迦陵频伽”,传说是雪山上的神鸟,为佛祖的极乐世界所歌唱。

在中国,这种鸟就比前两位更加的有名气,人头鸟身,那就是中国的“九天玄女娘娘”,似乎是《诗经》还是《龙鱼河图》或者其他什么古书之中有记载,给黄帝送来内含奇门遁甲的天篆文册(龙甲神章)的,就是这一位主。

还有很多其他的传说,一传说讲“九天玄女”就是西王母,但是传说大部分都是混乱的,这些无法追究。在六朝时期,道教甚至还有“玄女”传授黄帝房中之术的大量记载,不知道此玄女是否彼玄女,否则,被一只这样的东西来传授房中之术,总不是一件愉快的事情。

那么三叔当时看着浮雕的四只眼睛,想到的是什么呢?他当时的推测是这样的:

浮雕额头上有四个凹陷,显然应该镶嵌了四颗卵石,现在只有两颗,那有两颗明显已经被人取走,古墓之中不可能经常来人,取走那两颗的,十有八九就是当年的叔伯。

这些都是简单的,谁都能想到的,这里就不去解释。

重要的关键就是,那剩下的两颗,为什么还在原来的位置?

土夫子讲究“一次不取后世不尽”,既然叔伯对卵石感兴趣,当时为什么没有把石头全部都拿下来,而要剩下两颗?

三叔又想起了他刚才的结论:让叔伯变成那个样子的变故,必然是发生在这个密室之中。

但是这里又看不出有什么曝跷的地方,只是一个四面封死的空间而已。

唯一奇怪的就是这四颗卵石被取下了两颗,没有道理是叔伯故意留下两颗,如果不是故意留下,那难道是他当年的变故,是发生在他取卵石之时?他取下两颗之后,因为突然发生的事情,没有时间再去取剩下的?

三叔想到这里,心中豁然开朗,卵石的假设和整件事情,天衣无缝地串联到了一起。他忙凑过去,仔细去看那黑色的妖异石头。

卵石深嵌在浮雕之中,整个浮雕犹如一个整体,如果不仔细看,是看不出和石墙是两个部分。三叔之所以一下于就注意到,是因为其中两颗已经被挖走,当时四颗都在的时候,没有相当的注意力是发现不了的,看来,当年的叔伯应该不是个简单人物。

那么,撬出这几颗卵石,会引发什么事呢?石墙之后确定没有机关,难道卵石有毒吗?不会啊,刚才已经碰过了。

三叔犹豫了一下,一种无法抑制的冲动就自他心里冒了上来,他决定也撬下一颗来看看。

三叔抽出了砍刀,在一边的墙上磨了两下,颤抖着凑过去。他用刀尖碰了一下其中一颗。接着,把刀插入一边的缝隙,然后一撬,“咔”一声,其中的一颗就掉到三叔手心里。

卵石一掉下来,三叔马上就后退了一步,警惕地看着四周,唯恐有什么隐秘的机关突然启动。

然而,却一点事也没有,卵石落到了他的手心里,冰凉的,一动不动。四周也没有什么异动,浮雕还是浮雕,墙壁还是墙壁。

三叔又等了一会儿,确定没事才松了口气,心里又纳闷,难道自己刚才的假设错了?又或者当时的变故只能引发一次,现在无论撬多少次,也无法引发了?

他收好这一颗卵石,又去撬另一颗,还是同样的步骤,把刀插入一边的缝隙,此时他镇定了一些,力气也用得大了,一撬,“啪”一声,卵石一动,弹了出来。

三叔忙去接,可是卵石弹得太快,他反应不及,一下掉在地上,“啪”一声,犹如沙球砸在水泥地上,一下摔成了粉末,黑色卵石蓬起青铜色的一层灰尘,一下子飞散在空中。

三叔一个机灵,心说不好,给呛得咳嗽了一声,扇了扇,觉得满口都是辛辣的味道,一想起外面血尸身上的那种颜色,下意识感觉这粉尘可能有毒,忙用衣服捂住口鼻往后退。

退出几步后,马上去看刚才卵石掉落的地方,只见地上卵石碎裂的地方,青铜色的粉末中间,竟然爬出一只红色的小虫,蜷缩成一团,发出“吱吱”的叫声。

三叔一看那虫子,顿时脑子就嗡的一声,人不由自主地就往后退了一步。

因为他一眼就认出了这种虫子,这是一只尸蟞,而且还不是普通的品种。红色的尸蟞,听家里的老人说过,剧毒无比,是恶鬼之虫,见血封喉,稍微一碰就会中毒。

但是这种红色的尸蟞,据说只生存在古尸的体内,几乎没有可能捕捉到,怎么可能会被人裹在一块卵石里面?又给镶嵌在这里?最离谱的是,被裹在石头里的虫子,怎么还是活的?

三叔觉得十分的离奇,不过,他马上意识到自己没工夫再去想这些了,地面上,红色的小虫转了几圈,逐渐伸展了开来,开始抖动翅膀爬动起来,似乎要飞。

之前三叔没见过蟞王,也不知道是不是真的毒得这么厉害,但他知道如果是真的,那在这么狭小的空间里,这种虫子一飞,就等于宣告了自己的死刑。

他小心翼翼地退后了几步,横起砍刀,想趁着它没飞起来,把它拍死。可还没按下去,突然就听到一声“咯咯咯咯”声音从砍刀下传了出来,接着一团红色的影子一下就蹿了出来,竟然飞到了三叔的肩膀上。

那虹光速度太快,三叔根本来不及躲,一个激灵,吓得一身冷汗,手里的刀本能地向后一甩,就拍在了自己的肩膀上。蟞王被吓了一跳,再一次飞起来,停到了一边的墙上。

此时,蟞王已经完全清醒了过来,鼓起了翅膀,不停地发出“咯咯咯咯”类似于青蛙叫的声音,一股辛辣的臭味从它身上不停地散发出来。

三叔一琢磨,心说不行了,这东西他娘的比血尸还难对付,留在这里肯定是死,三十六策,走为上计,还是溜吧,想着缩起身子就小心翼翼地往密道的入口处退去。

密道根本不容转身,他只有倒爬,连滚带爬地退到了暗道的入口处,幸运的是,回头看了看一边,那血色的小虫并没有紧跟过来。

三叔定了定神,就去摸暗门的机销,但是心有点慌,手抖得厉害,几乎就不受控制。

好不容易摸到了机销,推开暗门,三叔刚刚松了一口气,突然一道红光以迅雷不及掩耳的速度,从暗道黑暗中飞了出来,那速度之快,几乎就像是瞬间移动一样,直奔三叔的面门而来,就在转瞬之间,那东西已经到了跟前。

三叔心里叫一声“不好”,再想躲已经晚了。千钧一发之时,他急中生智,把脖子一缩,然后对着那虫就是狠命一吹。

三叔吹灶台的时候练出来的肺活量相当了得,力气很大,一下于,那尸蟞被吹得改变了方向,翻了个跟头,撞到了墙上。

三叔趁着这个机会,一抬手,一翻身就从暗道里翻了出去,反手一下压死暗门。

下面的蟞王几乎同时跟了出来,但是这一次它晚丁一步,暗门已经给盖死了。它“啪”一声撞在石板上又掉了进去,发出了一连串“咯咯咯咯”的叫声。

三叔只觉得头皮发麻,全身都软了,他一下子瘫倒在棺材边上,才发现自己浑身都被汗湿透了,好久才缓过劲来。

我深知蟞王的厉害,听得一身冷汗,忙让他长话短说,不需要讲得如此生动。

之后,三叔意识到此地不宜久留,再也没有什么想法,收拾了东西,反打盗洞,带着几个先人的遗骨爬出了古墓。

回到长沙之后,三叔没有对任何人讲起这件事情,包括爷爷在内,但是他对于战国帛书却有了浓厚的兴趣,开始暗中研究。可是三叔当时的那些朋友,不是地痞就是流氓,没有一个上得了台面,整了有大半年也没有整出什么成果来。从暗道中带出来的黑色卵石也找了很多前辈看过,都说不出一个所以然来。

三叔心灰意冷,逐渐失去了兴趣,直到他到西沙前,在一个机缘巧合之下,事情才有了转机。

当时,他的一个朋友生病死了,请了一个老牛鼻子作法事,那时候的牛鼻子是兼职的,穿上道袍是道士,脱下来就可是任何人,也没讲究,作完法事,一群人就喝酒,三叔自己也忘记当时是怎么回事,似乎是喝醉了吹牛,就把卵石拿了出来炫耀。

没承想,那牛鼻子一看到那东西,就脸色一变,又闻了闻,突然说这不是石头。

三叔没把这人放在眼里,有点嘲笑地问他道:“不是石头,那是什么?”

牛鼻子正色地告诉他道:“这应该是一颗丹药。”

牛鼻子说得言之凿凿,三叔看着不像是瞎说,以为碰到高人了,就把他拉到一边没人的地方,想和他细说,然而这个牛鼻子也是个半桶水,只知道这东西是丹药,却不知道来历和细节。而他之所以知道这东西是丹药,还是因为他们住的道观很古老,据说是五胡乱华的时候就有了,道观中有很多古董,几代下来都给他拿去当了,其中就有很多炼丹的工具,他在其中见过这种石头一样的丹药,也闻过味道,才敢如此肯定。

三叔不免失望,但是总算又看到曙光了,后来又找了几个搞金石研究的人看过,他们也证实了这个说法,这东西的确是一颗“丹”。

不过,丹药这门东西属于玄学,很个人化,几乎每个方士都有自己的炼丹方法,没有古字考寻,在一颗丹药上根本看不出什么来,倒是那个牛鼻子和他说,既然是古墓中发现的,那肯定是古人自认为的长生不老丹,因为只有这种丹药才会用来陪葬。三叔听了感到很迷茫,因为他知道丹药之中包的是蟞王,丹药一般是内服的,这东西吃了肯定是死,而且死得很惨,还长生个屁啊。

三叔百思不得其解,又折腾了大半年,几乎什么渠道都试过了,还是没有任何进展。就在三叔准备彻底放弃,想把那丹药扔抽水马桶里冲掉的时候,一件童想不到的事情发生了。

\chapter{西沙的前奏}

当时是考古湘盗墓潮兴起的时候,大量国外的探险队来到亚洲,想在这第二次考古大发现中分一杯羹。

当时中国的海洋考古几乎是零,眼看着大批国宝被人盗捞走,中国的考古界人士哪能不急,几个老教授一起上书中央,请求采取措施。后来迫于形势的压力,在要钱没钱、要人没人的情况下,终于拼凑出几支“考察队”,其中有一支就给派往了西沙,这就是文锦负责的那一支。

三叔意想不到的事情,就发生在考古队成行之前,大概一个月左右的时间。

当时三叔正在帮文锦准备一些土设备,类似于抽水机、潜水器械这些东西,这些上头都不负责,全是三叔张罗的。那一天中午,三叔正忙着调试设备,忽然有一个学生进来说,外面来了一个人找他。

来人姓解,叫做解连环,大概是取“怨怀无托,嗟情人断绝,信音辽邈。纵妙手能解连环”里面的字。这人是三叔的外家兄弟,也就是相当于我的远房表叔,因为一同住在长沙,所以平日里有来往,但是也不太多。

那年头说起互相来往这种事情,三叔他们还可以,老一辈就只有过年过节去拜会一下,讲究的是淡如水。这样的亲戚突然来找,让三叔有点意外。

不过亲戚来了,自然不能怠慢,也不好马上问他来干什么,三叔就停下手头的活儿,寒喧了一下,拉他到馆子里吃饭,等酒喝到一半的时候,才问他来找自己有什么事。

解家也是大户人家,兄弟有六个,比爷爷家还多,一般来说不会无缺钱迹,来找三叔,必然是有什么事情需要帮忙来着,而且事情可能比较特殊,不然他们自己不至于摆不平。

那解连环扭捏了很久,才对三叔道,其实也不是什么大事,他就是想托三叔的关系,在文锦的考察队里谋一个位置,他想出悔看看。

三叔一听就感觉不对劲了,文锦娇人可爱,大家都喜欢,解家因为是亲戚多少也都见过了,但是文锦自己的度撰得非常好,见过虽是见过,但是都没有深交,平日里就更不要说联系了,解连环莫名其妙地冒出这么一个不着调的要求来,这肯定是有企图的啊。当下他就摇头,问道:“什么出海看看,你想看什么,去杭州看不行吗?”

解连环为难地挠头,说这他不能说,要是一定要知道,就当他有笔买卖在那边。他也是受人之托。

三叔又问他为什么不自己想办法,雇艘渔船又不是很花钱的事。他解释说,现在中国正和越南搞军事对抗,西沙那块地方十分敏感,没有海防的允许普通船只进不去,所以才托三叔帮个忙,混在考察队里行事好方便点,且这事儿对文锦绝对没影响。

三叔越听越怪,这土夫子和西沙摘在一起,怎么想怎么别扭,说是有买卖,西沙那里会有什么买卖?那边说实在的,只有水和沙子,再多就是沉船,你要冲着沉船去的,何必去西沙呢,宁波和渤海海了去了。而且解家在那时候也算有头有脸,几百年的老家族了,不可能突然落魄到要去掏海货的地步啊?

那解连环看三叔的表情有点为难,就说要是不行就算了,他再去想别的办法。

当时如果是我,他这么说我肯定就松一口气,顺水推舟就拒绝了,但是三叔不这么想。他一听,心说不对,这事情里有蹊跷,要是拒绝了,这小于真的会去想别的办法,这一行都不是善类,到时候要做出什么出格的事情来,不好防备。既然已经和文锦扯上关系丁,就不能让他乱来,得查查他到底在摘什么名堂。

于是就说不是不行,他为难是因为这事情不是他一个人说了算,他要先问问文锦,这事情他是拍不了板,便让等上一段时间。

解连环一听,忙说谢谢,还拿出了一堆当时的紧缺洋货,托三叔送给文锦。

两个人各怀鬼胎,又聊了会儿别的,那解连环就走了。三叔马上去找自己认识的几个地痞,给了点钱,让他们去跟着他,查查他最近到底在做什么。

那时候的地痞是消息最灵通的一帮人,不久就有了消息,说跟了这解连环好几天了,发现他就是一个二世祖,平日里也没什么爱好,只喜欢听花鼓戏,朋友也都是三教九流一群,非常平常,要说蹊跷,就只有一个地方奇怪,就是他最近一段时间,不知道为什么和一个洋人来往密切,经常隔三差五地去一个茶馆和一个洋人见面,谈也不谈多少时间,十分钟就走。

三叔一听,心里奇怪,他们这一行和洋人做买卖,那是寻常事情。但是解连环不同,他这种人已经基本上不参与家族生意了,他在家里的工作就是花钱,怎么突然又和洋人打起交道了,三叔觉得这里面有戏,马上决定亲自去看看。

他问清楚了解连环见那个老外的一般规律,自己选了个时间,那一天,他换了一件不起眼的衣服,一大早蹲在解连环门口等他出来。等了有一个小时,解连环就出得门来,三叔摸了上去,远远一路跟着,跟了有半个长沙城,到了老米市那里,前面果然出现了一个茶馆,解连环警惕地看了看后面,没发现三叔,就挑帘子走了进去。

三叔心中大喜,三步并作两步蹿上去,到窗口一看,正看到解连环在一位置上坐了下来,而位置的对面,果然坐了一个老外。

那老外一头白发,虎背熊腰,看不出是哪国人,但是气色极其好,坐在茶馆里就像一只熊一样,现在正似模似样地喝茶,还穿着拖鞋,看这自若的劲儿,肯定在中国混得长了,早就习惯丁长沙的市井生活。

三叔打量了那老外一下,发现这人看着还有点面熟,好像在哪里见过,不由就有点纳闷。

和他做过生意的老外一只手就能数完了,绝对没有这个人。这人肯定不是他的客人,但是那个年代,在长沙见到老外的机会简直是渺茫,肯定也不是平时看到的,那这人是谁呢?

他努力地回忆,把这几年见到老外的场合都想了一遍,突然就打了一个激灵,他马上想了起来:这个老外,竟然是他在一年前镖子岭看到的那一群老外中的一个!那一年前的经历太过震撼,三叔记忆犹新,一扯出线头,马上就全部回忆了起来。

三叔遍体生寒,他看着茶馆里的两个人,突然感觉自己意识到了什么,又抓不住,一种不祥的预感从他心里冒了上来。

说到这里,我举手打断了三叔,让他停了一停,我必须想一想再听下去。

听三叔到现在的叙述,事情已经很清楚了,毫无疑问,解连环想去西沙,是为了帮这个神秘的老外办一件事,而且还是一件比较特殊的事情,因为一般和外国人的买卖,大家都在做,没必要搞得这么神秘。

而这个老外,就是一年前镖子岭外想挖掘血尸墓的那一伙人中的一个,那时候三叔已经感觉十分奇怪,因为镖子岭是中国内陆的深山,不是应该出现老外的地方,而现在,这伙人显然又想托人去中国的西沙海域,这同样是老外不应该出现的地方,因为那时候正在打仗。

当时,三叔还不知道西沙之下有一座古墓,所以很多事情只是疑惑,无从推测。但是我现在已经知道了以后发生的事,根据这些推断,那个老外托解连环要办的事情,应该和那座明朝的海底墓有关。

这么说来,第一个知道海底墓穴存在的人,极有可能是那个老外,而那个老外又告诉了解连环。

那就出现了一个无法解释的怪圈,一个匪夷所思的问题:这个老外是从哪里知道镖子岭古墓和西沙海底墓穴的存在的呢?这两种墓穴之罕见,就算是我爷爷这种人也只能说是略有耳闻,他一个番邦人,如何能这么神通广大?

我又想到解连环死的时候,他手上抓着的蛇眉铜鱼,这是第一条现世的蛇眉铜鱼,显然这东西应该是他从海底古墓中带出来的,那么可不可以这么说,这个神秘的老外,他要解连环做的,就是在古墓中带出这条铜鱼?

也就是说,那老外不仅事先知道梅底有古墓,甚至还知道了古墓里面有什么,这也太符合老美情报至上的原则了。

就连三叔去爷爷的笔记上记载的镖子岭,也是靠寻访当地的山民,几经辛苦才找到的,西沙海底的古墓就更不用说了,我想除了汪藏海,根本就没有人会知道它的存在。

想到这里,我突然打了一个激灵,心说不会吧,人说在没有答案的时候,最不可能的答案就是正确的答案。

既然这些事情是不存在的,那这样说来,唯一的答案就是:难道三叔刚才说的,还是胡说?

这人有过前科,我一下子就心虚了,马上看向他,看他的表情是不是不对。

三叔见我脸色阴晴不定,不知道我想到了什么,一看我看他,就问我怎么了。

我试探道:“三叔,你可不能再骗我了,都说到了这份儿上丁,你再骗我就真不厚道了。”

三叔看着我的表情就奇怪,问我为什么这么想,我把我的顾虑一说,他听了之后,突然皱起了眉头,也看向我。

我一看完了,这反应似乎是被我揭穿了,不知道怎么说了,心里不由就沉丁下去。

没想到他看了我几眼,忽然道:“你想得太绝对了,事情不是这样的。其实,那几个老外当时并不知道那西沙底下到底有什么,他们只是知道,那个地方的下面,必然有什么东西而已。”

我问道:“你从哪里知道的?”

三叔道:“这是他们后来亲口告诉我的,其实这几个老外就是现在阿宁所在那家公司的老板,而这家公司的创始人你知道是谁吗?”

我摇了摇头。三叔道:“就是从你爷爷手里骗走战国帛书的那个美国人。”

我一听几乎下巴掉了下来,道:“是他?”

三叔点子点头,道:“就在这一次去西沙之前,我亲眼见过他一次,他已经快不行了,现在靠机器维持生命。当时他亲口告诉了我他几十年来投入资金在中国活动的目的。”

“那是什么?”我问道。

三叔道:“整个事情的起因,就是当年他骗走的那张战国帛书。当年他还是一个教会的中学教师,偶尔做一些盗卖古董的勾当。那一年,他用云慈善的名义,从爷爷手里骗来战国帛书的真本,当时这个人已经十分精通中国的文化,他为了抬高这份帛书的价值,决定破译上面的信息。”三叔顿了顿,“但是他花了两年时间,破译出来的东西却让他大吃了一惊。”

我心中一动,道:“这个美国人竟然能破泽出我们这么多年都没办法的战国帛书?”

三叔点头:“就因为他是美国人,所以他破译丁出来,因为这份帛书暗字的排列方式,是用一种数学的原理,我们这样的人,就算再精通,也无法从数学的角度来破解这东西。”

“那帛书上写的是什么?”我好奇道。

三叔道:“那帛书上记录的信息,不说出来你绝对想不到——”

三叔正讲到一半,突然门口有人敲门,我心里奇怪,难道又有人来看病?能来的都来了啊,谁他娘的来打扰我听故事,转头一看,竟然是一个快递。

他走进来,问道:“谁是吴邪先生?”

我点了点头:“是我。”

他从包里拿出一大包包裹出来,道:“您的快件。”

三叔也很奇怪,怎么会突然有快件寄来,问我道:“谁寄来的?”

我翻来看了看,信封上写着:张起灵。我顿时心里一慌,心说他怎么会给我寄快件。一看日期,还是不久之前。难道他从地底缝隙中出来了?忙拆开来一看,信封中露出了两块黑色的东西——竟然是两盘录影带。

\chapter{录像带}

就在我和三叔聊天时,突然就有人敲门,随即就走进来一个快递员,问哪个人是我?

我在这里的事情,只有家里人和阿宁方面的一些人知道,所以我以为是家里给我寄来的慰问品或者是国外发来的资料,并没有太在意,就接了过来。等我签了名字仔细看寄件的人时候才发现,包裹上的署名竟然是张起灵。

那一瞬间我呆了一下,接着就浑身一凉。

在这里的这段时间里,我已经把在长白山里的事情逐渐地淡忘了,可以说除了恐惧之外,其他的记忆都基本上被琐碎的事情覆盖,但是这三个字的名字,突然一下子又把我心里迟钝的那根弦扯紧了,不久前的回忆一下子潮水一样涌现在了我的脑海里。

他怎么会给我寄东西?他不是进到那巨大的青铜巨门里去了?难道他已经出来了?……这是什么时候寄出来的,是在他进云顶前还是后?我马上去看包裹上的日期,一看又是眼皮一跳:竟然是四天前。

这么说他真的出来了!他从那巨门里出来了!

我的手都开始发抖起来了。脑海里闪过闷油瓶走入到地底青铜巨门中的情形,看着手里的包裹,心里乱成了一团,心说这会是什么东西?难道,这是他从那青铜门里面带出来的?

那会是什么呢?人头,明器?鬼玉玺?

不知道有多少古怪的念头从我的脑子里闪过,过了好久,才突然意识到我应该马上打开它,忙四处找剪刀。

一边的三叔看我表情大变,不知道我收到了什么,好奇地凑过来看。一看到张起灵这三个字,他也吸了口冷气,露出了极度震惊的神色。

两个人手忙脚乱地翻了半天,最后三叔找到了一把水果刀递给我,我才得以割开了包裹外面的保护盒。

盒子里面裹了一包东西,包裹是四方形的,外面十分工整地用塑料胶带打了几个十字,十分难撕,我废了九牛二虎之力才撕出一个口子,里面露出了两个黑色的物体。我的心跳陡然加快,停了停,深吸一口气,用力一扯,两块黑色的物体被我拔了出来。

那一刹那我已经做好了看到任何可怕东西的准备,然而我看到的东西,还是让我傻了眼——那竟然是两盘黑色的老式录影带。

我刚才脑子里乱成一团,几乎什么都想过了,唯独没有想到,里面会是两盘录像带。因为闷油瓶那个人,你可以很容易把他和什么棺材扯上关系,却实在很难把他和录像带这种过期的现代化设备之间建立什么联想。

我靠,他怎么会寄这种东西给我?里面是什么内容?

我的心一下就悬了起来,脑子里出现了一个念头,该不是他进青铜门后的情形吧,难道他把青铜门后的情形拍摄下来了?

我靠,要是真的那太……不过一想又不可能,当时没见他扛摄像机进去。而且我相信那青铜门之后也不会是什么好地方,应该不至于能轻松地扛摄像机拍摄。

那会是什么呢?我心里顿时好比无数只蚂蚁在爬,直想马上播放出来看看。

不过,这两盘录像带,样子和使用的材料都是很老式的,可以说年代相当久远。我知道必须要老式的放映机才能播放,那种东西现在很难找到了。

三叔示意我翻过来看看,我就把包装丢到一旁,把两盘录像带拿出来,先仔细去看录像带的侧面上有没有标识什么信息。

我对录像带并不陌生,十年前街头还是满布录像带租赁店的时候,看国外的故事片几乎是我唯一的娱乐。那时候假期里一天五盘是肯定的,接触的多了,对这东西的结构自然也有一些了解,知道一般自己录制的录像带,都会在背脊上写点什么,否则无法辨认。

一看却有点奇怪,它的背脊上以前确实贴着标签,然而现在给撕掉了,给撕掉的痕迹很新,显然撕了不长时间,看来,似乎是闷油瓶不想我们看到这边上的标签。

这又是为什么?东西都寄给我们了,还要撕掉边上的标签,这上面有什么我不能知道吗?

“这是怎么回事?”这时三叔拾起地上的包装,甩了甩,确定里面再没有什么东西,问我,“大侄子,你他娘的可不厚道,你怎么没告诉我你和他还有联系?”

我摇头表示绝对没有,三叔拍了拍带子,问那这怎么解释?我说:“你问我我问谁去。”

三叔看我不像撒谎,就皱起了眉头,啧道:“那这小子也算神通广大了,他怎么知道你在这里?”

我也奇怪,我从云顶出来之后,地址只有阿宁那批人和家里人知道,他没有我的信息,却能准确地寄东西给我,这其实是相当困难的事情,没有人为他收集情报是不可能做到的。看样子,这个沉默寡言的人背后的水,真的深不可测。

三叔想了想,又问我面单上有没有写这邮包是从哪里发出来的?我拾起面单看了看就摇头,上面只有发件人和日期,其他真是一片空白。不仅发出的地址没有写,连发出地都没有标明。真不知道这快递是怎么做事情的。

不过日期是在四天前,这里省内快递一般一天就到了,省外比较近的也只需要两天,这份快递寄了四天,寄出地不是离这里很远,就是相当偏僻,交通不便的地方。我可以查查快递公司的电脑系统,如果他们有网络登记,一查就知道了。

说完三叔和我就对视了一眼,都苦笑了一下。这突如其来的东西打乱了三叔的叙述,一下子,我也不知道怎么处理这带子好。三叔就道:“大侄子,要不咱们先暂停,这小哥行事诡秘,他不会莫名其妙寄东西来,这两盘带子可能非同小可,咱们先去找录像机看看里面拍的是什么怎么样?”

我听了一下摇头,忙说不行,虽然我对这录像带里的内容也十分的在意,但是三叔对我叙述的东西还没有一个具体的头绪,现在暂停,等一下他心情变化,还指不定说不说呢。而且录像机这东西停产都快十年了,现在连VCD都淘汰了,旧货市场都很难买到,这带子一时半会儿肯定看不了。

不过,如今如果当这两盘录像带不存在也不可能,我就说咱们继续说咱们的,让你那伙计去问问这市里什么地方有旧货市场,然后去看看,如果有这机器就买下来,如果没有,我晚上上网想想办法。

三叔听了也觉得有道理,道:“也行,反正接下来也会说到这小哥的事情。”说着就挥手让伙计照办。

那伙计听三叔讲事情也听得津津有味,现在把他打发走了,颇有点不情愿,不过给三叔眼睛一瞪,也没脾气了。

伙计走后,三叔就拍了拍脸,道:“那咱们说快一点,刚才我说到哪儿了?”

我把我刚才听到的给他重复了一下,三叔就点头:“对,关键就是那帛书的内容,那老外和战国帛书渊源很深,这事情还挺复杂,还得从头和你讲。大侄子你生意做了也不短时间了,你对战国帛书这东西了解多少?”

我想了想,干一行熟一行,虽然我不太喜欢拓片生意,利太薄而且接触的人都有点古怪,不过这么多年做下来,我对于这一行的了解还是比较深刻的。

战国帛书这东西,不能算是拓片里主要的一种,看名字就知道,战国帛书就是战国的帛书,然而,事实上,这个战国的范围还比较狭窄,正式交易的时候,春秋时期的东西,也算到了战国里面。市面上战国帛书的正本很少,非常珍贵,又因为出土墓点的不同,被分为楚帛书、魏帛书,等等。这些帛书的内容也各不相同,其中最珍贵的是鲁帛,现今公认是鲁帛的,我知道的十个手指都数得过来,而且都不完整,其他混充的东西虽然也有,但是真假难辨,一般官方不承认。

鲁帛书也不是单一的一种,按照字体和拓片的大小,分成几个小类别,其中最珍贵的是一种鲁黄帛,原因很简单,就是它上面的文字,别人看不懂。

记录在这种帛书上的文字语法非常古怪,能知道单字的意思,但是没法阅读。我们知道中国八大天书:《仓颉书》、《夏禹书》、“红岩天书”、“夜郎天书”、“巴蜀符号”、蝌蚪文、“东巴文书”以及“峋嵝碑”,都是文字孤本,没法进行破译。然而鲁帛上的文字,却好像是密码一样,国外考古界把这种鲁黄帛叫做“中国的魔法书”,因为按照排列念出来,就好像是跳大神的咒语一样。

不过这种密码已经在1974的时候,被人破解了,这就是后来被称为“战国书图”的一种图文转换的古代密码。我是在三叔那里听说过这个词后自己查的资料,这是一个大发现,不过1974年当年发生的另一件事情太出名了,所以这个考古事件并没有引起轰动。

现在一般的战国帛书的拓片交易中,这种鲁帛很吃香,找的人很多。前段时间听说根据考古研究,这种鲁帛可能有一百二十卷之多,也不知道是从哪里推测出来的,但是我知道真正在流通的,也就是那四片到五片,那都是真正的专业人士看的东西,在网络上看不到,而且外国人特别喜欢,所以很多掮客在各地淘这东西,希望能发现孤本。而要找稀有的鲁帛书,则需要到拓片店里去扫店,因为我们采购拓片都是一大批弄来,也不会去分类,各种来历的都有,一般都堆在那里,如果有心就说不定能找到冷门的,而且这种人找到了一般也不会张扬,自己回去研究了,所以这个市场的生意还是比较好做的。

我爷爷从古墓里盗出的那一份就是鲁黄帛,不过因为老底子出过事情,这东西我们也不敢拿出来炫,爷爷在江湖上的名气很大,不乏有人问起这事情,也算是我店里压箱的宝贝。

现在我们也知道,这种鲁黄帛,应该就是战国时期铁面生的杂记。这家伙和达·芬奇一样,使用自己创造的文字来书写杂记,非常的神秘主义。从鲁王宫出来之后的那段时间,我也研究过这东西,据说人类历史上,凡是使用密文记述东西的人,都是因为发现了颠覆当时世界观的东西,怕被主流势力(比如说达·芬奇时期的天主教廷)抹杀而不得已采取的措施。

关于帛书,我就知道这些。我把这些和三叔说了,三叔点头道:“说得没错,果然茅坑蹲久了不会拉屎也能哼哼。”说着就从床底下拿出他的破包,从里面摸出了一张发皱的照片,我接过来,发现是在博物馆的橱窗里拍下来的一份战国帛书,看上面的文字排列,应该就是爷爷盗出之后被美国人骗走的那份正本。

“这是本来应该属于咱们家的东西。”三叔道,“老子三年前去美国的时候,在纽约博物馆顺便拍的,整件事情就是因这块东西而起的。想想也真是命里注定,咱们家四代人了,好像给诅咒了一样,都被卷到这事情里头来了。这也是我不想你参与进来的原因,我希望这件事情到我这里就能停了。”

四代人,是啊,我突然感慨了一下,问道:“到底上面写的是什么内容?”

三叔笑了笑道:“刚才我就说过了,不说出来你绝对想不到,其实,帛书上面并没有写任何的东西,帛书翻译出来的并不是文字,而是一幅神秘的图形。”

“图形?”我皱起眉头,想起了七星鲁王宫的那份战国帛书,“难道,也是一幅古墓的地图?”

三叔摇头道:“不是地图,比地图复杂多了。这件事情一言难尽,去西沙之前,那个老外把这些事情全部告诉了我,我转述一遍,你听完就明白了。”

\chapter{裘德考}

(三叔接下来的叙述很是烦琐,牵扯到了很多老长沙的事情,不过这些事情对于我来说十分的有趣,因为我自小就喜欢那种带点土腥子味道的老事情,比较有历史的厚重感,听一听也无妨。)

三叔嘴里的那个传教士当时的名字,叫做考克斯·亨德烈,中文名叫做裘德考,在长沙的教会学校工作,是国民党时期随着当时的东进潮来中国的美国人之一。但是这人自小就六根不清净,洋和尚没什么兴趣当,却对中国的文化很感兴趣,或许在美国人的经济观念里,文物也只是商品之一,能自由买卖,自然也可以出口,所以到了中国的第三年,他就偶尔做一些暗地里的文物走私活动,那一年他才十九岁。

裘德考的走私生意一直做得很小心,生意做得不大。那时候有两种走私商,一种是流水的营盘,走的量大,但是出价很低,玩的是成一笔是一笔的买卖,风险很大。而裘德考是“打铁的买卖”,也就是出价高,东西要得少,但是很安全,来一笔成一笔。他这样的做生意方式,很对爷爷的胃口,所以当时爷爷和他的关系很好。

但是裘德考这个人并不是一个值得交的朋友,从心底里,他并没有把爷爷当成是朋友,甚至他没有把爷爷当成是一个和他平等的人。我爷爷在事后知道,在私底下,他称呼我爷爷为臭虫。

1949年长沙解放,国民党全面溃败,之后是1952年,教会开始退出中国,在中国滞留的很多美国人都开始回国,他也收到了教会的电报,让他在安全的时候返回。

他意识到自己在中国的生意要告一段落了,于是开始做相关的准备工作,转移了自己的财产。在临走之前,他又有了一个险恶的念头,他和他的同党开始大肆收购明器,用中国人信赖老关系的心理,以极其廉价的定金卷走了大量的文物,其中就有我爷爷的战国帛书。

当时我的爷爷并不肯卖这一份父辈们用命换出来的东西,是裘德考谎称这些钱会用来开善堂,爷爷感觉这是积德,才勉强出手的(当然这是我爷爷自己说的,不知道是不是真的,我看他这样的人不太可能有这种善心)。

在这些货物全部上船之后,裘德考知道这批人中有一些并不好惹,为免留下后患,在船上拍了一封电报给当时的警备处,将我爷爷等大概十几个土夫子的形迹全部漏给了当时的长沙解放军临时驻军。

这就是当时十分著名的“战国帛书案”。这不仅仅是文物走私案,因为裘德考和解放前国民党将领的关系,里面牵扯到了间谍、叛国等很多那个年代特有的想也想不通的因素,变得非常复杂,几乎惊动中央。那一天裘德考满载而归,而为他积累财富的那批土夫子,枪毙的枪毙,坐牢的坐牢,哀号一片。

虽说也是罪有应得,但是这样的死去,实在是太过悲惨了一些。后来大跃进和“文革”时期中国的文物走私几乎绝迹,也和当时这一批人的死亡有关系。

当时我爷爷机灵,一看形势不对,就连夜逃进了山里,躲在一座古墓里,和死尸一起睡了两个礼拜,逃过了风头,后来光身逃到了杭州。这件事情对我爷爷的打击很大,以至于战国帛书后来就成了他的一项禁忌。他在世的时候,一直叮嘱我们不可以乱说这方面的事情,所以我们家的人一直对此讳莫如深。

裘德考回到美国之后,拍卖了那批文物,发了大财,战国帛书被高价卖给了纽约大都会博物馆,成为当时拍卖价格最高的文物,而裘德考也一跃成为百万富翁、上流社会的新贵。他在中国的故事写成了传记,广为流传。

富有之后的裘德考,逐渐将兴趣转向社交,大约在1957年,他受邀担任了纽约大都会博物馆远东艺术部顾问,对战国帛书的研究工作提供顾问。当时的博物馆馆长就是臭名昭彰的普艾伦,两个人都是中国通,都是在中国雇用土匪盗掘文物发的家,很快成为朋友。裘德考还赞助了一笔钱给博物馆作为基金,用于收购民间的中国文物。

大概是因为富裕生活的悠闲以及对于中国文化的热爱,之后的裘德考修身养性,逐渐沉迷到了中国文化的研究中,他在大都会博物馆主持研究了几个大型的项目,成果颇为显赫。然而让他真正名留史册的,却是1974年,他解开了战国帛书密文那件事情。

当时他对于战国帛书的研究,已经持续了二十多年,起初他是为了抬高帛书的价格,后来则完全是因为兴趣。

在刚开始,没有任何一个人认为,他这样的一个美国人可以解开中国的古代密码,然而,裘德考却以惊人的毅力做到了。

说来也是巧合,他是借一本中国“绣谱”古本中的灵感,发现了“战国书图”的解码方式。这种解码方式,其实也就是类似于“绣谱”中利用文字记录刺绣程序的办法。在数学上就是点阵成图,说复杂也不复杂,完全在于一个巧,你能想到,就能够解出来,你想不到,即使你对中国古代密码学再精通也没用。

发现解码方式后,裘德考喜出望外,马上召集了人员,对爷爷的那份战国帛书进行了大范围的翻译。一个月后,全部的密文就被解出。

然而出乎裘德考意料的是,当时出现在解码纸上的,不是他原先预计的记载着战国时期占卜历法的古文,而是一幅古怪的、完全没有意义的图案。

这图案古怪成什么样子,很难形容出来,我后来看了三叔给我画的草图也摸不着头绪。描述一下的话,只能说这幅图案十分的简单,只有六条弯曲的线条,和一个不规则的圆组成,线条互相延伸,有点像地图上河流的脉络,或者是什么藤本植物蔓延的茎,但是,给那个圆一围又感觉不是。拿远点看,好像是一个抽象的文字;近看,就完全不知道是什么东西。

此外没有任何的信息,如果你不说这是来自于一本中国的古籍残卷上,所有人都会以为这是刚刚会拿笔的小孩子在纸上乱画出来的线条。

历尽千辛,翻译出来的东西竟然是这么一张莫名其妙的图案,裘德考感觉到十分的诧异。他一度以为自己的翻译方式是错误的,但是反复验证了之后,他发现不可能,如果是错误的,那么不可能成功地将文字天衣无缝地转换成这个图形。显然,用密文记录下的东西,就是这七条线条。

那这七条线代表着什么呢?这帛书的主人为何要将它隐藏在文字当中呢?

凭着在中国这么多年的经历,他的直觉告诉自己,能够被人用密文写在昂贵无比的丝帛中,不会是普通的图案。这线条肯定有什么特别的意义,说不定非同小可。

他对此产生了浓厚的兴趣,立即开始查阅资料。他用了大量的时间,翻了无数的图书馆,同时,拿着这张图案去找了当时大学里的华裔汉学家请教。可是,在美国的那批人水平有限,折腾了大半年没有任何结果,就算有人说了推测,也是不伦不类,完全没有根据,一听就是胡说的东西。

就在他兴趣减退,感觉到没有了指望的时候,有一个大学里的朋友给他指了条明路。他告诉裘德考,这种中国古怪的东西,应该到唐人街里的老人堆里去问,当时是冷战时期,在唐人街,有不少来自台湾的老学者,藏龙卧虎,也许会有线索。

裘德考一听也对,抱着最后的希望,真的去了唐人街求教。

唐人街有一种书馆,是老人聚集的地方,裘德考就专门去这种地方,将那图形发阅,也亏得他就是命好,果然就让他碰到了一个高人。

这高人是一个干瘦的老头,在当地算是个名流,那天他在茶馆听书,正巧碰到裘德考来发图,就要了张拿来看。这一看之后,他就大吃了一惊,问裘德考是从哪里搞到的?

裘德考一看有门,不由大喜,他自然有自己的一套说辞,和那老人说了来龙去脉,就忙问这老人是否知道什么。

那老人摇头说不是,不过他告诉裘德考,虽然自己不知道这图形的来历,但是,他曾经在一个地方见过类似的东西。

裘德考一听,心中也一动,忙问是在什么地方看到的。

那老人说,那是还在大陆的时候,他在山东的祁蒙山一座道观里,看到过一个丹炉,这图形,就是刻在这丹炉之上。

\chapter{青铜的丹炉}

一直以来,这份图形神秘莫测,如何查找都没有一点线索,如今听到这个,裘德考兴奋异常,他马上就请人泡了一壶上好的茶水,恭敬地递上,请那个老学者详细说说。

那个老学者本身就没什么事情,见他十分有兴趣,也来了兴致,就给裘德考讲了当时的经过。

那是三十年前的事情了,当时这个老人是北京大学的国学教授,是国民党员,女婿是张灵甫手下的一个旅长。整编七十四师溃败之后,国民党残军化整为零,他女婿就带着残部逃入了祁蒙山,当了土匪,在山里猫了三年。后来解放军大剿匪,他女婿被逼得走投无路,和国民党特务接上了头,准备逃往美国。

买通了路子之后,老头和家眷就被他女婿接进了山里,等船的消息。因为风声很紧,带着家眷不方便,这段时间,他女婿把他们安顿在了一座道观里,伪装成道士,等特务的接应。

说是道观,其实是那种民间的土庙,不过,和其他山区的庙宇不一样的是,这座道观建筑在两座相距不到五十米的悬崖之间,下面腾空,十分奇特。整个道观类似于一个巨大的阶梯,一层一层,一共有七层,墙壁都是刷着黄漆的泥墙,十分的简陋,最上面四层,就是架在两道悬崖中间的木板,连栏杆也没有。几个神龛上面都是土塑的三清像,也有观音和土地,很有中国的特色。

整个道观由两个老道士打理,老的还是年轻一点道士的父亲,那年代兵荒马乱,香火稀薄,他女婿就给他们一些钱,作为掩护。

那个老教授在道观中生活了两个月,道观是在深山里,爬上爬下不方便,他也无事可做,就开始研究这道观中的古董。就是在那段时间,他发现了一个奇怪的东西。这道观中很多的东西,都是粗制滥造的民间土货,没有什么价值,偶尔有几件古董,也是明朝时候的东西。然而,道观的最顶上那层,却有一个青铜炼丹炉,形状十分的奇特,好比一只倒翻的莲花,看上面的铜锈,年代更加的久远,和这里其他的东西有很大的区别。

老教授不是学历史的,但是当时的老夫子,对于这些都有点阅历,他很感兴趣,就问了老道士,这丹炉是从哪里来的。

那老道就称赞他眼光很厉害,这丹炉确实不普通,是解放前一次地震,从山里塌出来的,当时一起塌出来的还有很多的死人骷髅,村民很害怕,就抬到这里来给神仙镇着,已经是有六十多年了,他当时还小,具体什么情况也不清楚。

老教授听了就觉得越发有趣,然而当时兵荒马乱,自己的身份又特殊,也没法进行更多的调查,他就在道观里琢磨了一段时间,后来也就没有下文了。不过,当时境遇和环境让他对这件事情的记忆非常深刻,对于那个丹炉的形状和花纹,也记得十分清晰,所以一看到裘德考给他看的图形,他就认了出来。

他告诉裘德考,这个花纹是在丹炉的盖子上,形状和这图形一模一样,他绝对不会记错。如果他想知道得更多,可以想办法去那个道观了解一下情况,不过,沧海桑田,现在那地方还在不在,要看你的造化。

裘德考听了之后,又是兴奋又是失望。兴奋的是,显然这份图形背后的东西,比自己想的还要丰富;失望的是,听完这些叙述,他对这个图形仍旧一无所知。

他很想亲眼看看老教授口中的那只青铜丹炉,然而,这在当时几乎是无法实现的。当时一个美国人要到中国去,相当的困难,特别是他这样臭名昭著的文物贩子。

不过裘德考这人是非常自负的,他想做的事情,没有人能阻止。他还是想了办法:自己不能到中国去,但是这么多年的文物活动下来,他在中国有着严密的关系网。他开始设法联系中国的老关系,想办法找人进祁蒙山,到那个深山道观之中去看看,了解一下情况,最好,能够把那个丹炉偷出来,运到美国。

当时的中国刚刚受过十年浩劫,百废待兴,他的老关系已经荡然无存,老一辈的土夫子,都在解放后的清肃中死的死,逃的逃,文物走私这一块,已经完全重新洗牌。他借助自己在国民党中的关系,几乎用尽了所有的渠道,都找不到一个认识的人。

百般无奈之下,他只能冒着风险,求助于几个当时自己不熟悉的文物走私犯,让他们介绍一些长沙这行业里面的新人。

这又是几经波折,不过工夫不负有心人,最后,终于给他联系到了一个肯和他合作的中国人。

这个人,就是解连环。

解连环是怎么进这一行的,三叔当时百思不得其解,因为当时的大环境,连解家老爷子都不敢涉足老本行,只能吃吃老本。这走私文物是大罪,和现在的贩毒一样,是脑袋别在裤腰带上的活儿,一般不是急着要钱救命,谁也不敢去干这个。

而解连环当时就是个纨绔子弟,完全二世祖,解家老爷子有意洗底,从小就不让他接触家族生意,也不让他学东西,所以无论胆量、眼界、阅历还是其他的客观条件,他都不可能会进到这一行来,更加没有理由能够和国外的走私大头联系上。

说得通俗一点,文物走私这一行是要有手艺在手,拿货、鉴货、估价这些技术,没有二三十年的锻炼积累,是成不了气候的,而你没有这些能耐,就算你主观上再想入行,也没法找到门道,你的买主不会理你。所以,如果裘德考能够通过中间人联系到解连环,就说明解连环必然已经和这些人有了生意来往,而且取得了对方的信任。这想来以解连环的本事,是怎么也不太可能的。

这个问题一直困扰着三叔,直到他第一次西沙之行回来,开始调查这件事情,问了解家的老大,才知道了一些来龙去脉。不过,这事情和裘德考的事情并无关系,这里没有必要再提。

解连环和裘德考接上头之后,裘德考就将自己的计划寄给了解连环。那是一份详细的资料,附上了那个老人画的青铜丹炉的草图和一只先进的照相机。他让解连环首先必须要确认那座道观是否还在——在那段时间,古迹庙宇这种东西属于四旧,有可能已经被毁掉——然后,收集这丹炉的信息,拍摄照片,发回美国确认,如果一切无误,那么,再寻找机会将这东西走私出国。

解连环虽然不懂下地的事情,但是去一个地方,看看东西在不在,打听打听事情,还是能做的。他拿到资料之后就去了山东,根据资料上老人的回忆,找到了修建那座古道观的山区。

万幸,因为道观十分的偏僻,并没有受到太多的滋扰,在风云飘摇的十年中奇迹般地保存了下来,不过,老道士已经死了,只剩下老道的儿子,也是风烛残年。解连环拍摄了道观和那个青铜丹炉的情形,发回了美国。裘德考拿出翻译出来的图案一对比,果然那老人说得没错,青铜丹炉盖子上的图形就和帛书上一模一样。不过,对于这丹炉的来历,因为年代过于久远,那老道的儿子也只能说出一个大概,和那老教授说的内容也差不多,得不到更多的线索。

虽然如此,裘德考也已经大喜过望,就发了指令让解连环开始准备,找个办法偷偷将丹炉走私出来。

然而,解连环一准备,就发现这其实是一个不可能完成的任务。

裘德考没有考虑到的是,这个丹炉比他事先预计的要大上很多,时代已经不同了,这样的东西,在当时中国是不可能通过海关运出去的。而要是通过走私船,则要先到达浙江或者广东一带,风险也很大,当时的东南沿海之乱,是普通人无法想象的。

他们尝试了很多种方法,都没有结果,反而引起了雷子的注意。无奈之下,裘德考就出现了一个丧心病狂的念头。他让解连环将整个丹炉砸碎,锯成四十多片,然后标上记号,分批混在当时出口的丝绸里运出去。

这对于考古界来说,简直是令人发指的兽行,但是裘德考完全不在乎,因为这东西的价值对于他来说已经没有意义了,他要的是上面的信息。

这也可以说是无巧不成书,解连环在锯丹炉的时候,就发现了这青铜丹炉的底部,竟然有一个十分巧妙的机关。就是凭借着这个机关,战国帛书上神秘图形的秘密,才最终被解开。

\chapter{星盘}

说着,三叔又从他的破包里,掏出两张皱巴巴的照片递给我。

我知道这两张照片拍的肯定就是那只丹炉,这些照片,应该是那个老外给他的。这事情比较复杂,没有这些照片,恐怕没法说得明白。现在他都用到我身上了。

接过来再次一看,我就看到了第一张照片上拍的,是一只陈列在博物馆中的巨大丹炉,三叔说的时候我还不知道这东西这么大,简直有一个人高了,想把这种东西走私出国,确实是一个不可能完成的任务。

第二张,则是丹炉底部的情形,我看到了布满花纹的青铜炉底,在炉底的中心,铸着一只拳头大小的望天铜兽,头仰向天,十分的威武,就造型上来说,属于上上之品。

“这是在博物馆中复原后的丹炉,第二张是丹炉的内部。”三叔给我解释,“解连环发现的炉底机关,是一个十分巧妙的加水口,用来在炼丹的时候,往丹炉里加水,炉壁是空心的,里面有水,只要转动丹炉的盖子,把上面的图形转到一定的位置,就能打开这望天兽下面的机栝,炉壁中的水就会从望天兽的嘴巴里喷出来,这样,在炼丹的时候,就不需要打开炉盖。”

我点头称奇,不过这样的机关巧术,在中国其实并不算特别,为何说这个机关是解开战国帛书的关键?

三叔说问题不是这个机关的功能,而是这个机关的运作方式,说着就拿出一个放大镜,就让我仔细看这丹炉底部的花纹。

照片很小,我仔细去看,就看到这炉底上面,以望天兽为中心的四周,有很多细小的浮雕点,非常多,密密麻麻的,不仔细看,会以为是铜锈。

“这是?”我还是不了解,就问道。

“你不知道也情有可原,这炉底上的浮雕,是一张古星图。”

“古星图?”我愣了一下,“就是标示天上星星位置的图?”

三叔点头,然后拿了一张战国帛书翻译出来的图形照片给我对比:“这是这个机关最巧妙的地方,炉底是一张古星图,当炉盖转动到正确的角度时,炉盖上这个图形上的曲线就会和炉底下的星图中的六颗星重合,机关就能打开。”

我一听,立即就想到了什么,随即一想就恍然大悟:“两个图形可以重合,这么说,这战国帛书上的奇怪线条,其实是一个‘星盘’?”

三叔点头:“没错。”

星盘是一种观星的工具,因为天上繁星数以万计,而且根据时间季节的变化而移动,每次观星要从如此多的星星中找出特定的那几颗十分困难,所以便有了星盘这种东西。一般都是根据星与星排列而连起的线条,只要将星盘上的北斗星对好,就能凭借罗盘和季节的刻度,转动星盘,那些特定的线条会和自己寻找的那几颗星星重合。

我不由拍案叫绝,哎呀,这不是很难想嘛,刚才怎么没想到呢。这也很合乎逻辑,战国时期的观星术已经非常发达了,而那个时代的人认为,天象运行代表着事间万物的运动,能够从中洞悉到一些天机。这些天机往往预示着国家的变更、重大的战争和灾变,一般是不能随意泄露的,铁面生将自己观察到的星图藏入帛书之内,也是可以说得通的。

这星图同时又出现在丹炉上,也许是这种天象代表着什么特殊的含义,使得当时很多的人都注意到了,这也是非常有可能的事情。

三叔就点头:“你小子有长进,说得很对。这些东西运到美国之后,裘德考也立即发现了这个秘密,他和你一样,就想到观星术。”

这是一个很令人振奋的发现,可以说在考古历史上,还是第一次,裘德考又一次出了大名。然而,这时候他已经不在乎了,他已经完全沉迷到这考古的过程中去了:星盘圈出的星象是什么含义呢?从它被隐藏得这么严密来看,这星象显然预示什么非同小可的事情,不能被别人知道。

他将这星图和星盘重叠之后,就从整个星图中找出了特定的那六颗星,合成了星象图,然后去查了古籍资料,想知道这星象图在观星术中代表的是什么意思。

可是,中国古代的星象学,几乎是和风水同宗,复杂无比,甚至比风水还要深奥,几乎没有系统的资料。战国帛书上所隐藏的这份星图,预示着什么样的天机,完全无法查找。

当时唯一解开这个秘密的方法,还是去找那些所谓的高人,但是这一次在美国就找不到了,于是,裘德考再次拜托解连环,去中国的民间寻访。

然而这一次解连环没能完成任务,那个时代懂点周易风水的,都给打到牛棚里去,漏网的都战战兢兢,谁也不开口,打听起来也是偷偷摸摸,十分的不方便。

这一找就找了两年时间,没有任何结果,同时在美国的其他研究也都没有任何进展。

万般无奈之下,裘德考又有突发奇想,他的注意力再次集中到了战国帛书上。他推测,既然帛书上有这星图,那么也许在其他的篇幅中,会有星图秘密的记载。

于是,他一边开始在中国收购鲁黄帛,一边就打起了当年出售战国帛书的爷爷的主意。按照他的经验,土夫子一般都贼不走空,这帛书不可能只有一卷,爷爷要盗出来,肯定是整份拿出来,那剩下的部分,也许还在爷爷的手里。

当时解连环和裘德考的关系已经非常好,狼心狗肺的,就帮裘德考到了爷爷那里打听消息。可惜我爷爷口风很紧,什么也问不出来,无奈之下,解连环又来问三叔。当时三叔正对爷爷笔记里的记载感兴趣,酒一喝,话一多,就把爷爷当时盗出战国帛书的经过当故事全说了出去。

听到这里,我就忍不住道:“三叔,敢情那老外知道血尸古墓的事情,是你自己说出去的?”

三叔就苦笑,摇头道:“当时喝得确实多了,酒一过,我也想不起来和他说过这个,后来那老外和我说起我才想起来,我这肠子都悔青了。”

我也陪他苦笑,这真是太有戏剧性了,不过话说来,当时裘德考选择解连环,也许早就知道了吴家和解家的关系,早就有了这一层的打算。这个老外行事之诡秘,实在是让人恐惧。

当时裘德考得到消息之后,就有了重新盗掘血尸墓的打算,可惜解连环不会倒斗,而找其他人,他也找不到。当时中美关系开始回暖,他感觉局势会发生变化,就耐心等待了一段时间,果然让他等到了一个机会。他于是带着一批搞考古的人迫不及待地回到了中国,开始策划这个行动。于是便有了之前三叔经历的事情。

之后的事情,猜猜也能猜到了,那一晚三叔逃出古墓之后,裘德考在第二天的下午也进入了古墓,不用说,这件事情最后变成了一场灾难。在他们打开棺底暗格的时候,飞出的王几乎杀光了当时在墓里的所有人。

也亏得当时解连环找来的一个伙计相当机灵,就是他在最危险的时候,拉爆了炸药,将内室完全炸塌了,当时在外室中的裘德考和解连环才得以保命。可惜他自己和一干人,就全部被埋死在了古墓里。

当时景象极度恐怖,亲眼目睹的裘德考受到了极大的打击,几乎精神失常,他对于中国几十年来的理解完全崩溃了。回到长沙之后,他立即返回了美国,大病了一场,几乎疯了过去。对于战国帛书的研究,也立即终止了。

然而,我们知道这只是暂时的,一年之后,第二次海洋考古时代来临,命运的车轮,开始在西沙的海面下,越转越快。

\chapter{西沙的真相}

裘德考的叙述到了这里,就告一段落,接下来的事情,就是解连环去找三叔之后发生的了。

他的叙述,可以说很清晰地让我了解了这件事情的起因,我实在没有想到,三叔这么早就牵涉进了这件事情,而且,阿宁公司和我们吴家的渊源竟然这么深。

三叔一口气说完之后,休息了一下,让我有什么问题、什么不相信的,可以现在问他。

我知道这是他的气话,显然刚才我不信他,他还耿耿于怀。

我想了想,不信是不能说了,不过,确实有几个地方我还不清楚。

刚才我们已经知道,裘德考和解连环早就有联系,当时的见面只不过是一次重逢,而且根据之后我知道的事情,我推测裘德考来找解连环的目的,很可能就是要他混入到文锦的西沙考古队中去,潜入海底的汪藏海墓,为他取出一样东西,而这样东西很可能就是汪藏海隐藏着东夏国秘密的蛇眉铜鱼。

那么,裘德考知道血尸墓的情况,是三叔自己透露的,这毫无疑问,但是海底墓穴,如此隐秘的地方,裘德考又是怎么知道的呢?难道也是三叔告诉他的?这不可能啊。

还有,显然按照三叔的说辞,这一切的起源就是战国帛书,然而,西沙的汪藏海和战国帛书又有什么关系呢,为何裘德考会把目光转向西沙?

我把这些问题提出来,三叔就点头,道:“你想到关键了,确实让解连环混入考古队的,就是裘德考,不过你的推测只对了一半。他自己的说法,让解连环进入古墓,并不是为了蛇眉铜鱼,而只是让他拍下棺椁中的尸体。”

至于为什么要这么做,那个老外不肯说,同时,他是从哪里得到汪葬海墓的信息,他也不肯透露,三叔问他的时候,他就用了中国的一句老话,故作神秘:“天机不可泄露。”

“不过,”三叔凑过来道,“后来的一些事情,让我或多或少能猜得一些什么,你可以听听是不是有道理。”

我点头说好,他就在床上,用手指画了几个点。“我曾经想了一下,那老外回到中国盯上了西沙,是在长沙那件事情一年之后,从时间上来推断,他知道海底墓穴存在,应该也是在这一年里。那么,这一年里必然发生了一些事情,让解连环得到了这些信息。

“但是我们知道,那段时间,裘德考受了很大的刺激,显然不太可能只是因为知道了海底有个古墓,就立即振作起来,重新全身心地投入另外一件事情中去,当时能吸引他注意力的事情,应该只有和战国帛书有关系的事。那么,我们可以推断,那件事情,必然也和战国帛书有关。裘德考应该是先被战国帛书的信息而吸引,然后才注意到与之联系的西沙的事情。

“这里无法推断这个事情到底是什么,但是根据之后发生的事情,我感觉很有可能这个老外遇到了一个人,这个人应该进过海底古墓之内,很有可能,是他帮裘德考揭开了那帛书之中星图所代表的意义,这个意义和汪藏海的古墓之间,必然有着联系,使得裘德考的兴趣转向了西沙。所以,裘德考才会再次来中国,找到解连环企图混入考古队里。”

“你为什么能肯定是遇到了一个人,而不是其他什么事情?”我问道。

三叔道:“那是因为资料,裘德考对于古墓的资料太详细精确了,这肯定是有人进去过,然后整理出来的,不可能有其他任何的情况能够让他知道得这么详细。”

我点头,这有点道理,不过,战国帛书上的星图,为何会与明朝古墓产生关系呢,这实在有点不可思议。难道铁面生看这个星象,预知了千年之后有一个同行会在那个地方修坟?

如果星象能预知到这种琐碎的事情,恐怕现在就不会失传了。这一点,还需要考证。

之后就是西沙事件,那次事件之后,整个事情就进入到了一片混沌之中,整个考古队在西沙海底的古墓里消失了,只有三叔一个人回来。裘德考一度认为是三叔杀掉了所有人,然而,从三叔之后的表现来看,三叔也完全不知道内情,整件事情变成了一个巨大的谜团。事情的真相如何,就要看三叔怎么说了。

休息了片刻,三叔做了一个手势,准备继续讲下去,我也打起了精神,坐了坐正。

他先吸了一口气,显然要转换一种心情。刚才说的都是裘德考的事情,不痛不痒,现要接下去要说的,就是他的亲身经历了。

吸完气后,他的脸色就沉了下来,语调也变得很慢,有点犹豫。

想了想,就先对我道:“话说在前头,关于西沙,有一些事情,当时在济南的医院,你三叔我确实骗了你。不过,我也是万不得已,这事情,一直是一块心病,我实在是不想重提,你要理解我。”

我点了点头,并没有回答。三叔骗了我,我早就知道了,我也不想去怪他,我只想知道真相。

三叔喝了一口水,就继续道:“其实,那次发现海底墓穴,只是老子演的一场戏,早在那天凌晨,我已经和解连环进去过一次。不过,我进去的地方,应该和你们后来进去的地方不同,因为解连环有十分详细的资料,我们当时直接进入了古墓的核心部分,因为那老外的委托,目标就是放置汪藏海棺椁的椁室。”

“你是指那三个墓室中间的那一个吗?”我回忆着海底墓穴的机构。

三叔就苦笑摇头:“不,你说的那个地方,只是古墓的第一层,这个沉船墓之大,超过你的想象。汪藏海的棺椁,深埋在古墓的最底部,而且处在一个十分古怪的境况中……用语言很难形容。”

当时解连环从裘德考手里获得的资料相当详尽,可以看得出裘德考手里的原始资料应该极富权威性。同时裘德考提供了解连环一部美国的照相机和闪光灯。据说是当年世界上最先进的型号,十分小巧并且有防水的功能。

资料告诉解连环,在考古队考察的礁盘向左大约半里,有一处地方,当地人称呼为“沙头礁”,是一处暗礁林,由数十块主礁和无数星罗棋布的水下暗礁组成。这一片礁石,在水下连成一体,是一块巨型珊瑚礁盘的一部分。在其中一片礁石上,有一处水溶洞,位于海平面下,就算落潮时候,也只会露出一丝,这便是当时沉船时工匠破船进水封墓时候的一个操作口。由此进入,便可进入到珊瑚礁盘之内,那海底的巨大沉船,就嵌在这礁盘之内,海沙之中。

只要进入珊瑚礁洞,就能一路下去,进入到沉船的内部,之后如何走,需要小心哪些东西,资料里都有详尽的说明。简直犹如这一座古墓,便是那裘德考设计的一般。

如此详尽的资料,就是普通的古法文献,也不见得能达到这种程度。所以三叔才会认为,这海底古墓,怕是早有人进去过了,可能是此人虽然进去,但是并未得手,所以裘德考不得不再次找人帮忙。

原本,解连环是有自知之明,他知道自己的斤两,不会再答应任何下地的请求,但是裘德考的身份不同,一来解连环觉得自己亏欠他,二来,这一年来,解连环也参与了家族中很多的活动,总算也下了几次地,胆识以及身手都不同以前,再加上裘德考开的条件很高,自己又是盲目信心的年纪,所以最后还是鬼使神差地答应了。

三叔当时得知了老外和解连环有奸情之后,本来是想竭力反对解连环加入考古队的,然而,之后发生了很多的事情,让三叔感觉事情非常不对。为了知道那老外和解连环的真实目的,三叔冒了一次险,他说服了文锦,故意让解连环进入了考古队,表面上不动声色,其实是暗中监视,看他会有什么举动。

事情就是这么鬼使神差地展开了,这要说还有很多的隐情,但是都不重要,这里话休繁说,只说解连环在西沙,他出事的前一晚发生的事情。

当天是考古队工作进入结束阶段的第一天,打捞工作已经接近尾声,工作轻松,所以睡前所有人都喝了点酒,都睡得很熟。

解连环一直在等待这个机会,此时离工作结束也没剩几天,他知道机不可失,时不再来,于是在确定所有人都睡熟的时候,便假装起来放尿,实则探听虚实,伺机下海。

他并不知道,那个小时候的玩伴,叫做吴三省的老婆奴,现在早已经是心思缜密的老江湖,自己从上船起的一举一动,都被这个人牢牢地看在了眼里。

话说三叔当时,也是相当郁闷。他早已经对解连环有万般的不爽,他并不知道解连环的目的,于是解连环在船上,对于三叔来说就是一颗定时炸弹,不知道威力,不知道什么时候爆炸,本来挺好的和文锦谈情说爱的时间,却变得要防备他。

还有个原因就比较隐讳,三叔没有正面提过,但是我从三叔的叙述中听得出来,显然,文锦很欣赏解连环。一方面的确公子哥懂得讨女人欢心,秉性和三叔差得太多;二来,解连环的相貌和很多方面不比三叔差,三叔这种感情方面的新手,难免会吃醋。

所以解连环一有行动,三叔欣喜若狂,在解连环刚放下皮筏艇,想划离渔船的时候,三叔就突然出现,一把将他按在了甲板上。

三叔的突然出现,是解连环始料不及的,然而他一见是三叔,倒不害怕了,因为如果是其他人,当时就可能落个叛逃越南这样的罪名,但是三叔,大家互相清楚底细,他也不可能拿自己怎么样。于是便轻声让三叔放手。

然而三叔对他是早有积怨,而且已有芥蒂,如何会轻易放他,咬牙就几乎把他的手拧折,问他千方百计进考古队,又这么晚出海,到底想干什么?

这有点借题发挥,发泄自己郁闷的意思,解连环一开始还嘴硬,心里也暗火起来,他在长沙,除了长辈,谁也不敢这么对他,于是就压低了声音破口大骂。

三叔根本不吃他那一套,一听他骂人,直接就把他的脑袋按到了水里,直按到他翻白眼才提起来,如此反复,一来二去,解连环就蔫了,只好讨饶。

三叔再问刚才的问题,他就把这事情的经过一五一十地说了出来。

听完之后,三叔就眼里发光,几乎不敢相信自己的耳朵。原来这海底之下,竟然有着一座沉船葬的海底墓!这真是始料未及的事情。老头子的笔记中,也曾经记载过前人讲过的海底船葬,只是这种海斗极其稀少,老头子本人也只是听说,并未亲身一探。而这茫茫海底,沙行万里,要寻得一方线索,要比在陆地上难上万倍。如今这老外竟然知道得如此详细,到底是何方神圣?

想着三叔便心痒难耐,恨不得立即下到海里去察看一番,便放开解连环轻声说:“只是这样?那你他娘的早说便是,我与你是什么关系,说出来有何关系?难道我还会抢了你的不成?”

解连环已经蔫了,道:“这事情我瞒着我家老爷子,当然也不想你们知道,而且我和你也不算熟络,说了我也怕多生事端。你凭良心说,我要是直说,你会让我进考古队吗?”

三叔心里一想倒也是,已经放宽松了很多,便对他说:“算你有理,不过我提醒你,这裘德考在长沙人称‘白头翁’,此人并不是简单货色,你老表我看这斗并不好倒,你要么暂且放下,咱们回去找些人从长计议,要么这一次就让老表我陪你去,怎么说,老表不是吹牛,经验也比你丰富吧。”

解连环呸了一声,就道:“都说你吴三省比猴子还精,真不是奉承你,你想搭点香火就直说,咱们是同一绳上的蚱蜢,到这个时候了,你说什么我还能说不行?”

三叔听了心里冷笑,心说这二世祖也算看得明白。于是两个人就临时搭伙,说好进去之后,各取所需,谁也别拖累谁,出来之后拿的不好也别后悔。

三叔当时的举动,不可说是利益驱使,说来也并不光彩,甚至让我感觉怎么像胖子的所作所为,可见三叔的秉性,也不是一时半会儿成熟的。

发了毒誓,打点了装备,两人放下橡皮筏,乘夜就下了海,一路摸黑划船,靠着指南针,不久,便行到了那老外说的“沙头礁”。三叔抬头一看,正当乌云盖月,整个礁盘灰蒙蒙一片,便心头一惊,对解连环道:“你真个选了个好时辰,连个毛月亮都没了,乌云盖斗,瞎子进洞,逢二折一,你我恐怕要留一个在里面,招子放亮,你我好自为之吧。”

\chapter{深海}

这话是真亦是假,三叔说来,一是确实当日日子不佳,其次,他也想吓解连环一吓,这也是游戏的心态。如果有家中做兄长的,恐怕能明白三叔当时的想法,大的总想吓唬小的,来突出自己的地位。

然而解连环也不是傻瓜,并不为所动,只是冷笑一声便不再搭话,三叔自讨了个没趣。

礁盘不大,几块露出水面的礁石十分显眼,虽不知道洞口开在何处,但是想必也不会过于难找。解连环划船,三叔打起风灯,进入礁群便一座一座开始探照。不久就在礁盘西面一块臼齿形的礁石下面,寻得了洞口。

洞口大约二人见宽,深不见底,好比是长在礁石上的,岩石边缘隐约可见前人打磨的痕迹,显然此洞经过人工的修凿。洞口隐于水下,内凹于礁石的根部,如果不是事先知道,在水面上根本无法看到。

三叔穿戴上装备,就想进入,却给解连环拦住,说下面水路复杂,他知道路线走法,还是他在前面比较好。

此话有理,三叔也不好勉强,于是解连环先入得洞内,三叔尾随其后。

入洞三十米,便可知道这是礁盘中天然生成的空洞,里面礁骨横生,错起的珊瑚礁岩,犹如一块块巨人的板骨,嵌在洞穴的两壁。不过“板骨”的末端,都和四周的岩石融合成了一体,所以看来更像是无数的怪异海盘车,吸附在岩壁上。

海底洞穴潜水,相当危险,然而两人毫无经验,根本没有意识到自己在做什么,未做一点措施,就一直往内游去。

大约在礁洞中,匍匐游行了十几分钟,三叔便看到了岔口。礁洞在礁盘里面犹如章鱼的触角一样四处发育,到处都是可以通行的洞口,有些很浅,用手电照就可以看到了头,有些则大得吓人,犹如解放卡车一样大的洞口里深不见底。因为照不到阳光,这里的海葵和珊瑚很少,但是很多五彩斑斓的小鱼,以及海盘车和海参,让这个洞穴并不寂寞。

在解连环的带领下,三叔穿行于这个极端复杂的巨大礁洞体系中,好比穿行于鼠洞中的老鼠,为了留一手,他用潜水刀在各个路口都刻下了痕迹,以免在里面生变数。

半个小时后,他们游出礁洞,三叔打起水下探灯四处照时,却发现自己并没有进入到什么古墓之内,出现在他面前的,是一个莫名其妙的地方。

那好像是一个产生于礁盘内的巨大深坑,四周一片漆黑,他抬头便看见了头顶垂落的珊瑚礁,然而他打开探灯去照脚下的时候,却发现自己什么也照不到,脚下是一片深渊。

时隔多年,就算当年的情形再惊悚,三叔也记不太清楚所有的细节,所以他罗唆了半天,我也听不懂他们最后到底到了怎么样一个地方。最后只好找了一张纸来,让他勉为其难,大概地画下来。

三叔的画相当糟糕,比涂鸦还涂鸦,不过,意思倒是言简意赅,凭借我的想象力和三叔的解说,我连猜带蒙,逐渐还真有了点了解。

按照我的理解,那应该是礁盘内一个隐蔽的大型洞穴,具体处于哪里,根本无法考证。三叔行进的礁洞的出口,位于这个洞穴的最顶端,从他的脚下一片漆黑,好似进入了一片黑色的虚无来看,此洞穴的大小应该相当厉害。

三叔他们到了这里,已经没有继续前进的通道,前方左右都是一片的虚无。探灯照射下,海水里有大量的白色海屑,下方又是深渊,手电照出去除了背后的礁石,没有任何的参照物了。用三叔自己的形容,是好比飘在外太空里。

这种感觉其实相当糟糕,因为你无论在什么地方,你的手电光亮还能照到什么东西,你至少有一种存在感,但是在那里,你的手电发射出去,没有任何的反射,除了黑还是黑。你不知道前方有着什么在等待你。

此时氧气的消耗量也很巨大,洞穴潜水不同于一般的探险,它对于活动的时间必须严格控制,因为你必须留一部分氧气,用来返回到洞外,这样就要求潜水人必须时不时地查看氧气表,这对于三叔来说,是相当大的心理压力。

然而解连环却似乎胸有成竹,他在水中转了几个圈后,竟然示意三叔关上水下探灯。

没有探灯,那就是绝对的黑暗,三叔心中奇怪,这小子想干什么呢?现在已经找不到路了,他还要把照明的东西关掉。

不过看他坚持的样子,显然这样的做法也是老外示意的。三叔知道自己也没有其他选择,于是顺着解连环的意思,拧灭了探灯。

两只探灯都熄灭之后,黑暗像墨汁一样地侵袭了过来,同时,他们腰里的防水手电柄部的一圈夜光涂料(那是为了防止夜间潜水的时候,手电掉落到水底无法找到而设计的)缓缓亮了起来,指示出他们各自的位置。

边上的解连环,似乎摘下了手电,用来当指示棒用,三叔看见那光圈挥动起来,指示一个方向。

他朝那方向看去,隐约的,果然看到脚下黑暗的深处,很远的地方,有一大团非常微弱的绿色光点,似乎是一群什么生物的眼睛,正在缓缓地移动。

三叔心里咯噔了一下,顿时紧张起来,因为他听很多渔民说过,海里什么东西都可能有,这绿色的眼睛,该不是什么潜伏在黑暗深处的生物吧。

想着手就不由自主去摸刀,这时候,边上的解连环却挥了几下手电,那手电的指示光圈开始移动,竟然是朝那群绿色的光斑去了。

三叔心里暗骂,别看他平时大大咧咧的,下地之后三叔的处事风格其实很小心,解连环这样横冲直撞,实在是不妥当。但是解连环这样的动作,显然是知道那些光斑是什么,是在示意他跟过去。

同样的,老生常谈,三叔还是不得不跟过去,他心里懊恼也没有办法。

没有灯光照明,只跟着一个冷光环潜水,人就好比少了眼睛,这种融化在冰冷黑暗中的感觉,三叔在以前下地时候尝到过苦头,如今又一次遇到,而且还是在水下,三叔就越发感觉到不安。

绿色的光斑群一点一点靠近,但是因为光线太弱,一直看不清楚是什么,随着靠近,三叔惊恐地发现那斑点的确是在移动,而且速度还不慢,那是一群海洋怪物的念头就越发强烈起来。

但是解连环却好像一点也没有意识到,追得极快,很快,两个人就游到了那光点的上方三十几米处。三叔的恐惧到达了极限,他一下冲过去,拽住了解连环不让他继续靠近。

解连环不知道出了什么事情,也吓了一跳,停了下来。

三叔用手电做着动作,解连环也挥动着回复,但是两个人都无法理解对方想表达的意思。

三叔懊恼极了,真想马上打开探灯说个明白,但是又怕这么近的距离,万一照出来下面真是鲨鱼之类的东西,真的连逃命的机会都没有。

正在焦虑地琢磨到底怎么让解连环明白自己的意思的时候,突然一道白光亮起,解连环竟然打亮了探灯,显然他也郁闷得够戗,实在忍不住想问问三叔为什么要拉住他。

三叔吓了一跳,一边去捂灯,一边低头向下看去。

白光的尽头,下面的黑暗中,朦朦胧胧的,照出了个白色的、裹在破败纱衣中的人状物体。随着三叔越来越适应探灯的光线,他看得越来越清晰,浑身的毛孔都收缩了起来。

那竟然是一具悬浮在水中的古尸,摆着一个诡异的姿势,面目模糊不清,庞大的白色纱衣犹如巨大的水母裙摆,漂散在水中,好像一朵来自幽冥的巨大花朵。

\chapter{浮尸}

幽暗的水深处,那具白纱围裹的古尸,不知道在水中泡了多少年,白纱早已经破败,分不清是男是女,因为距离尚远,尸体的样貌也是一片模糊,看不出保存的情况。

三叔冷汗直冒,不过立即镇定了下来,显然既然是沉船墓,有一具悬浮的尸体在这里,也不算奇怪。

然而等三叔逐渐放开了遮住探灯的手,就看到在冰冷的白光下,那古尸边上的黑暗中,又出现了另外一具古尸,同样的装扮,阴沉沉地隐藏在阴沉的海水中。

三叔就有了不祥的预感,他继续移动探灯,果然发现下面的黑暗中,竟然漂浮着大量的白纱古尸,足有三四十具之多。无数朵白色飘舞的纱衣,真的让人有一股冰彻心肺的寒意。

因为探灯光的关系,现在已经无从知道那微弱的绿色荧光,到底是从这些古尸的什么地方发射出来的,而最让人感觉到毛骨悚然的是,古尸群并不是静止的,僵硬的尸体悬浮在水里,竟然还在缓缓地移动。

三叔的心都要从喉咙里跳了出来,在不透气的头盔里,他的脑袋上全是冷汗,心说幸好他拉住了解连环,要是刚才直游过去,贴到这群古尸边上才开灯,自己不吓死才怪。这些尸体肯定在这里泡了近千年,普通的早就泡化了,怎么可能还悬浮在水中,难不成已经成了粽子?

自己下来的时候一点准备也没有,根本没想过会面对如此险恶的局面,连驴蹄子都不曾带上一个,说来也是冤枉至极,跟着这狗日的解连环,三叔也早已忘记这一切是自找的。

再看解连环,也是一脸的惊恐,可见刚才毫不在乎靠近的行为,应该是不知真实情况造成的,看样子老外并没有告诉他会看到什么。

三叔思绪如电,闪电间已经预见了好几个情况,此时远处的古尸群却渐渐漂近,不紧不慢,白纱缓慢地漂动,要不是四周的黑暗,和那模糊不清的五官,如此情景真如天宫之中仙人踩云而行的场景。

三叔看着看着,突然就灵光一闪,意识到什么了。

他压低身形,潜水几米,使得自己靠得更近,仔细去看。

古尸似乎没有完全腐烂,五官虽然模糊,但是还能看出人的样貌来。一具具呈现各种姿态,有的如托盘,有的如吹箫,有的如弹琴鼓瑟,洋洋十几具,虽然僵硬如铁,但是姿势之优美,无与伦比,三叔就明白了他看到了什么。

在很多古墓的壁画上,都会描绘这么一幅画面,那就是墓主人尸解升天,天上天门大开,群仙集会相迎,祥云缭绕,神鸟飞扬,天光普照。在这样的壁画中,必然会在墓主人踏的云梯之旁的上方,画着“天师舞乐图”,画中必有无数的天乐老仙,鼓瑟齐鸣。

但是这里的墓主显然是感觉一幅画的“天师舞乐”不过瘾,这几十具古尸所形成的景象,正是真实化的天师舞乐,鼓瑟齐鸣,这简直太不可思议了。

他顿时就明白为什么解连环会寻找这几具古尸,因为天师舞乐的路线,就是墓主人尸解升天的仙路,跟着古尸,就肯定能找到墓主人的所在。

一边的解连环缓过劲来,示意三叔跟上去,因为紧张,他的动作都变形了。

三叔努力安抚自己的心跳,他知道自己肯定进了了不得的地方,此时反倒不慌了,因为既然知道了这个地方,古墓又不会跑,现在这样的准备,显然是不充分的,他有了十足的借口,可以说服自己退出去。

现在想来,他们所处的地方,根本就是一片无尽的深渊,那几具古尸往哪里漂去,要漂多久,根本无法猜测,如果贸然去跟,不知道还需要浪费多少时间,氧气也不充裕了,的确是相当不明智。

三叔完全醒悟了过来,他阻止了解连环,示意他回去,不要再进行下去了,现在的情况再继续深入太危险,老命还是重要的。

然而解连环此时却又突然下定了决心(神经质是二世祖的通病,貌似我也有这样的问题),不等三叔阻止,径直就往女尸去的地方追去。

三叔在后面打了几个探灯信号,想让他再等等,解连环却一点也没有在意,三叔一看,心说糟糕,这小子大约是想甩开自己了。

刚才胁迫解连环,两个人一起下来,解连环肯定也是心不甘情不愿的,如今应该是快到尾声了,解连环干脆就甩掉他了。

纵使和他再没感情,解连环仍旧是自己的亲戚,而且自己是所谓的哥哥,中国人在这个问题上,始终有着血缘情结和护幼的情结,三叔此时不可能丢下解连环不管,他只能压住满肚子的火,急追上去。

(说到这里,已经不知道多少次听到三叔提起自己的“不得已”以及“没办法”,重复得我都能听出来异样,似乎他在潜意识里,非常强调自己跟着解连环去的不情愿。事实上,以我了解三叔的个性,三叔在那个时候,还不是那种能够控制住自己好奇心的人。我在这里已经感觉到,必然,解连环之后的死,三叔可能会负上主要的责任。

我之所以这样认为,是因为三叔在我小时候,带过我一段时间。那个时候,他就因为别人叫他去下地,而又无法顾及我,就把我用绳子拴在路边上整整晒了一天,晒得我差点中暑。事后他用很多盐水棒冰贿赂我隐瞒了这件事情,我那时候不懂事,也就没说出去。但是对于这件事情,可知道他年轻时候性格是相当顽劣,自控能力很差。

但是想起解连环在古墓上留下的血字,我却始终无法相信,三叔会特意去害他。所以听到这里,我不由自主地,开始紧张起来。)

接下来事情,节奏十分之快。

三叔一边权衡着氧气的消耗,一边奋力追赶解连环,他是越想越不对,像这样的海底古墓,他到底不曾到过,实在是没有把握。

但是解连环这个时候已经根本是在逃了,在前面潜得飞快,加上三叔并不是太擅长潜水,很明显跟不上他。

跟着前面的灯光,在黑暗中一直往前游了十几分钟,不知不觉的,许多的悬浮物出现在了三叔的四周。三叔草草一看,都是残破的木头构建,雕窗、木梁,成千上万,全部都高度腐败,上面结满了白色的海锈。

紧接着,在这些漂浮物的中间,三叔就看到了一个倾斜的巨大的犹如怪兽一般的黑影。

在水中漂浮的“舞乐古尸”们,径直朝这个东西漂了过去,而前面的解连环已经超过了它们,贴近了那个巨大的黑影,三叔借着他的灯光,一点一点看清了那东西的真面目。

那是一艘卡在礁石中的巨型古船的船头,这里所谓的巨型只是滥用的一个词汇,三叔已经感觉无法用来形容他看到的这艘船头的大小。

船头残骸从礁石中延伸了出来,两边延伸二百多米。残骸已经完全变形了,扭曲的船首上全是白色的海尘和结痂的珊瑚礁,如果不是那怪异的形状,恐怕别人会认为那是一只巨大海洋生物的头骨。

“舞乐古尸”朝着残骸飘然而下,很快就消失到了黑暗的海水中,三叔和解连环紧跟其后,在两只探灯的照射下,残骸的情形越来越清晰。

在船首的甲板之上,三叔看到一座半嵌在礁石中的木制雕花楼台,似乎是巨大木船的主体建筑,现在已经倾斜了,几乎要倒塌了。楼台之上,有一扇变形开裂的汉白玉石门,洞开着,好像一张大嘴,在等待他们自投罗网。

\chapter{沉船}

如果船头和那座楼台没有破损到这种程度,这水下的情形,必然壮观犹如水晶宫一般,然而现在,整个残骸上都覆盖着厚厚的海锈与海尘,死气沉沉,特别是那楼台,已经倾斜成四十度,看上去只要再踹一脚,就会彻底崩塌。

就算如此,三叔他们当时也震惊得几乎窒息了,这样的情形,不说是在海中,就算是陆地之上,也没有多少机会能看到,这究竟是谁的沉船墓,竟然沉在这种地方?

靠近看的时候,三叔注意到嵌入礁石的那扇玉门实在巨大,两人多高,四个臂展宽,玉门左右两壁外侧的海垢下,可以隐约看出浮雕着两个门神,手中各执一虎,模样凶猛可怖,三叔认得它们,但叫不出名字。楼台没有嵌入礁石的那部分,有飞檐瓦顶,瓦片都落得差不多了,只剩下檐骨。

玉门半开,中间有一条两人宽的缝隙,里面幽深无比,不知道通向哪里。

一边的“舞乐古尸”已经沉入了深渊之内,完全看不到了。

解连环没有停留,游进了玉门之内,三叔咬牙用力甩动双脚,加快了速度,很快也尾随了进去。

进去之后,是很长的一条可以并排走六七个人的走廊,一下子四周的空间变得局促,但是探灯的光线反而变得更加充足起来。

刚才在外面,那种幽深冰冷、绝望恐惧的感觉,到这里稍微减轻了一点,到底看到了自己熟悉的东西,三叔稍微有所镇静。

顺着走廊一路向前潜去,因为职业习惯,三叔粗略地观察了四周的装饰,发现每寸地方,包括地面上,都雕刻着连绵群仙图。

走廊的尽头出现了一道阶梯,一直向上,三叔翻转身体,仰卧而上,游着游着,他突然大吃了一惊,因为他发现自己的脑袋露出了水面。

当时他吓了一跳,这的确是十分令人吃惊的事情,在水里泡了快四十分钟,三叔压根没想到这古墓之内会有空气。他忙翻身趴到台阶之下,四肢并用地爬了上去。

一个在水里潜得太久的人,一旦上岸,猛地就会发现自己身子重得犹如背了铁块,更何况身上的确有负重的铅块和氧气瓶。三叔上去之后,几乎软倒,用力咬着牙,才没摔回到水里去。

跌跌撞撞走上阶梯,看到解连环已经把潜水器械脱了下来,一边大口地喘气,一边在用手电照四周的墓室。

三叔心说真是个菜头,要是碰上个闷坑,你早就挂了。不过现在看他没立即死在一旁,就说明空气应该没问题。于是坐到台阶上,也脱掉潜水的装备,一边放松肌肉,解下手电向四周照去。

台阶的尽头,他所处的地方,是一处砖砌墓室,典型的明代风格,高度不高,只能低头而行,宝顶上耸,呈现拱形,估计也是七辐七券的厚度,墓顶砖缝现铁色,灌了铁浆,砖头铺得极其精巧,宝顶的弧度没有任何的棱角越位,好像打磨过一样。

墓室的中间,青花瓷长明灯排成两列,直通到墓室的深处,那里一片漆黑,手电照去,发现墓室的中间,放置着一个巨大的黑色铁缸,不知道做什么用处,挡住了视线。

三叔一看就有些骇然,他盗过墓多了,知道这墓室虽然巨大,但却只是平民的规格,最多是一个财主,这就非常的奇怪,看外面古墓之规格如此巨大,没有几万徭役十年的努力恐怕建设不成,如果不是皇亲国戚,哪一个平民百姓能够有此大手笔?

三叔马上就如我们一样,想到了那个时候的巨富沈万三了。

如此说来,这一次跟着解连环,竟然给他碰到个油斗儿,这可是几世都修不得的福分。

他心中也兴奋起来,又转动手电,照射四周的墓墙。

墓室的墙壁上描绘着大量的壁画,同样相当壮观,三叔照了一圈,发现壁画连绵,几乎没有断裂的痕迹,且褪色也不厉害。

这里水汽弥漫,壁画能够保存得这样,实属难得,不过北宋的时候,已经有壁画上涂油蜡或者蛋清的保护技术,工艺相当先进,这里应该用了这样的技术,所以现在看来,壁画的颜色少许有些浑浊。

壁画之上画的东西,三叔从来不看,此时看了几眼,也不得要领,只觉得和普通的古墓壁画也无两样,就把手电的光线收了回来,去照一边的解连环,想问他刚才吃错什么药了。

解连环累得够戗,一边新奇地看着四周,一边气喘如牛,显然刚才用了死力气,三叔叫了一声,他也不理,被这个墓室吸引了全部的注意力。

本来刚才他甩掉三叔,三叔心中有暗火,但是,既然已经来到了这个地方,再发作并不合适,三叔就忍了下来。

两个人无话,三叔休息了一会儿便完全镇定了下来,心跳也趋于平缓,他随手开始准备进墓室的工具,同时,他留了个心眼,偷偷检查了自己和解连环的氧气瓶。

一看他就知道不太妙了,他自己的倒还好,但是解连环的氧气消耗量太大了,已经少了一半还多。

潜水员越是经验老到,在水下可活动时间就越长,而刚刚潜水的人,往往控制不了自己的吸气量,一发现自己在水里,就拼命地呼气,和老潜水员的消耗量比起来,可能会相差一倍多。三叔虽然也潜得不好,但是因为时常估计氧气瓶,所以比解连环节省得多。此时他一下就明白,解连环已经出不去了。

不过随即想了想,三叔倒是释然了,反正他出不去了,自己必然还要再进来一次带他走,那就没有必要急着出去了。

此时解连环就往墓室的深处走去,他也起来跟了上去,两个人走到巨大的铁缸面前。

三叔停下来走近铁缸察看,而解连环似乎没有兴趣,径直绕了过去。

铁缸重量有五吨以上,上面浮雕着大量的铭文,应该是一种祭器,缸足已经压入地下的青砖,缸中空空如也,但是缸的底部有一突起的鱼身样子的雕刻,不知道何用。

三叔想仔细看看上面的铭文,有没有自己认识的字,忽然就听到解连环惊呼了一声。

他转头一看,原来解连环已经走到了墓室的尽头,解连环的手电光照出了一座三阶棺床,上面有一只巨大的黑色雕花棺椁。

那棺椁几乎高到解连环的胸口,黑得非常刺眼。棺椁表面似乎打过光上了清漆,亮得很不自然,上面的雕花浅但是非常鲜明,大约是大量的鸟篆文字。而解连环可能突然看到棺材,有点害怕,正在朝后退。

这棺椁气势非凡,霸气十足,应该就是墓主的棺椁了,不知道里面葬的是谁。

三叔阅棺无数,不说普通的红木稗子木,整块沉香木做的棺椁,都有幸见过一回,但是像这里这具黑棺椁,他却看不出是什么材质的。他顿时好奇心起,绕过铁缸便走了过去。

走到解连环身后,他看得更加清楚,棺床用的是黄浆砖,垒成莲花圆盘形。棺床之后是一块照壁,上面写满了文字,估计是墓志,写的应该是墓主人生平,不过三叔扫了一眼后,就感觉后背发凉,注意力给那只黑色棺材吸引住了,同时他也知道了为什么解连环会吓得后退。

因为这只巨大的黑棺上,竟然躺着一个“人”。

\chapter{哨子棺}

三叔的手电照向棺材,看到那“人”的一瞬间,他几乎起了一身的褶子,头皮都麻了起来,自己也下意识地就往后退了回来,把手里的刀翻了出来。

不是三叔胆子小,而是这情形实在古怪。在这么隐秘的古墓之中,竟然有“人”躺在棺材的上面,突然看到,任谁也得抖几下。

这一吓的工夫,解连环也退到了三叔的身边了,他想必从来没在斗里出过事,吓得连脸色都变了,退得也急,一脚就踩到了三叔的脚上。

三叔给他踩得差点摔倒,不过这个时候,他就着手电光,看清楚了那棺材上的情形,原来是一场虚惊,棺材上面的人,是一具铜铸的人俑,紧紧贴在黑棺之上。

这铜人浮雕的造型很怪,行云留鬓,面貌夸张,有点像秦时的百戏俑,四肢犹如虫足一般粗肥极短,最诡异的是那张嘴,不笑不怒,竟然是竭力张开的,好似在惨叫一般。

三叔看着,心中立即就感觉到一股异样,一般人都讲究祥和安宁,而这铁棺和铜人配在一起,说不出的阴邪古怪,很不对劲。这确实是墓主的棺椁吗?

他用手电往四周照了照,这墓室一目了然,再没有其他的棺椁了,显然如果这里是主墓室的话,这确实就应该是墓主的棺椁无疑了。

三叔很相信自己的直觉,心中有点不安。

为了看得仔细,他推开解连环走了过去,走近一看,更加的惊讶,发现这巨大黑棺居然是一只雕花的铁棺,这个铜人似乎是后来加上去的装饰品。更奇特的是,那铜人嘴巴的位置竟然从棺盖上凹陷下去,使得棺盖上出现了一个深孔,不知道有没有穿透棺盖,通到棺材的里面。

不对!三叔看着就吸了口冷气,接着他一下就记起了端倪,心里哎呀了一声,心道糟糕。

生铁封棺,棺身带孔,这一具棺材莫不是老底子老人们讲的“哨子棺”?

“哨子棺”还是解放前传下来的说法,扯不到百代之前,三叔也是听老头子讲的。据说那时候湘西一带,有一路军阀,手下有一批发斗的能人,为首的名叫张盐城,此人据说是曹操发丘将军的后人,有神通,他的左手五个手指奇长无比,且几乎等齐,能平地起丘,尝土寻陵,盗墓功夫煞是了得。此军阀跟孙中山北伐,张盐城受命筹集军饷,便以古法盗墓,一路北上,也不知道多少隐秘的古墓被他翻出来,名声很大。当时湘西有“盐城到,小鬼跳,阎王来了也改道”的说法,一方面人被神话,一方面也可知道张盐城盗墓活动的猖獗。

此人盗墓,有一套特别的套路,就是如遇到血煞阴邪之地启出的棺椁,都会用牛血淋棺,观察棺椁的反应,如果棺中有异响,则棺主可能尸变,士兵会将棺材拖出古墓暴晒后启棺;如果棺中无异动,就要看棺材的表面,大部分情况,牛血不会凝结,顺棺身流至棺底,这说明没事情,开棺无恙。

但是还有一种相当特别的情况,就是牛血淋上之后,犹如淋于沙石上一般,血液渗入棺身之内,这是比尸变还要不吉利的大凶之兆,这说明棺中的东西,可能不是人尸。

棺中不是人尸,那是什么东西?答案是,无法言明的尸体。在中国,这种东西被统称为妖。

此时张盐城便会命人就地掘坑,将妖棺沉于坑中,涂上泥浆后烧熔兵器,铁水封棺,只在棺材的顶部,留下只容一只手通过的孔洞,等铁水凝结,他就以单手入棺,探取棺中之物,相传这就是他祖传的发丘中郎将双指探洞的绝技。

而探洞之时,他会命人用三尺琵琶剪卡住自己的手臂,一边将“叩把”拴于马尾上,以便感觉不对,旁人可立即抽马,马受惊一跑,拉动机栝,锋利无比的琵琶剪就会立即旋切,断手保命。

这样处理的棺材,因为上面有一个孔,最后会变成个类似于巨大铁哨子的东西,所以被人们称为“哨子棺”。

张盐城一生用到这双指探洞的功夫,据说也只有三次,全部都全臂而退,最好的一次,他从棺中取出的是一颗二十四香的金葡萄,只有臼齿大小,据说是藏于尸体口中的。张盐城后来随着军阀混战,下落不明,有人说他是投靠了革命,最后“文革”时候死在了收容所里,也有人说,他死在了皇姑屯。总之是个神秘人物。

关于他的传说,老头子们一般有两种说法,一种认为他真的有发丘绝技,双指探洞是名不虚传;另一种就认为张盐城是一个骗子,利用了普通士兵对于棺材的迷信恐惧,将普通的棺材说成是妖棺,然后作秀,使得自己的地位得到抬高。

事实如何,无人知晓。

我爷爷倒是相信张盐城是高人的,那是因为张盐城铁水封妖棺的做派,有一些侧面的证据。据说解放前黄河改道的淤泥中就发现过一只和张盐城所说类似的青铜棺,棺材的顶上确实有一个手臂粗细的孔,只是无人敢伸手进去,胆大的用火钳也只从里面夹出很多黄色的淤泥。后来这棺材在“大跃进”的时候直接给扔进炼钢炉炼了,也不知道有没有出事。

这只铁棺,虽然精致无比,和用铁浆胡乱浇铸的棺材完全不同,但是棺材之上那一个深孔,像极了传说中的“哨子棺”。

这就奇怪了,这解连环带路的墓室,应该就是墓主之地,为何棺床上的主棺椁会是这个样子的?难道那墓主不是人,是个妖怪?

三叔想着就感觉到一股毛骨悚然,想想这古墓深陷海底深渊之中,如此诡异神秘,说不定真不是人的墓,也许是海龙王的也说不定。又想起裘德考让解连环做的事情,不由心虚,难道裘德考知道这墓主不是人,所以才让解连环拍照片上去研究?

不过,三叔当时年少,并不会把老人说的话太当真。虽然有点心慌,但是并不害怕。反而他好奇心起来了,心说那这里面会是什么东西呢?

此时解连环也发现了是虚惊,又走了过来,心有余悸地看着这只铁棺。看了一圈,他便试着去推动棺盖。

三叔看他的脚都在抖,就知道他还在害怕,这个行为可能是为了在三叔面前表现一下,挽回他刚才被吓到的面子。

三叔感觉好笑,就用手电照射他的面孔,让他不要白费力气了。如果这是“哨子棺”,显然此棺材的加工者和张盐城是属于同宗的派系,这铁棺里面的东西绝对不是善类,而且这铁棺修筑起来根本就没有打算让别人打开,要从里面拿到东西,只有像张盐城一样,把手伸进那个棺材孔里。

说着,他就爬了上去,用手电去照那棺材上的孔,看看能看到什么。

由孔洞看下去,棺材内黑幽幽的,不甚分明,手电探孔并不是很好的办法,发散光到了一半就射不下去了,只感觉这“铸人”的喉咙之下,透出一股阴气,看一看就脖子发硬。要把手伸下去摸,真不是平常人能做到的。

三叔想起解连环从老外那里拿来的资料,就感觉自己的推测没错:那老外这么熟悉这里的结构,肯定是在他们之前已经找人进来探查过了,但是进来的那人为何没有完成任务?估计那人也和他们一样是这一行里的老手,进来发现里面竟然是这样一具铁棺椁,知道铁棺封尸非同小可,才临时放弃的。所以这老外才找了个半吊子的解连环。

如此说来,他们必然也不能碰这棺材,否则不就当了这裘德考的炮灰了嘛。

不过,如果不碰棺材的话,好像又有点太窝囊了,他和解连环下来,解连环空手出去还好说,自己也这么出去了,那解连环这么一说自己还有脸在?况且,这棺材看着,也实在是有点诱人。

三叔拿不定主意,不过他转念一想,还是理智占了上风,心说老祖宗的经验,棺材放在最后碰,他现在应该先看看这里其他地方有什么好东西,棺材今天他就暂且不碰,这古墓又不会跑,明儿晚上他们带着火筷和黑驴蹄子再下来,会比现在保险得多,那也不算胆小。

一想他便释然了,就让解连环在这里待着,要拍照就拍这个棺椁,那老外能理解他,自己开始搜索墓室的角落,寻找其他的陪葬品。

这墓室没有耳室,通体一条到底,格局十分的古怪。古人讲究事死如事生,这墓室的格局一般都是按照墓主人生前的布局仿制的,也就是说这墓主生前住的地方也是这么个情况,想不出会是什么一种状况。里面并没有普通的那种陪葬品,只有那些价值连城的巨大瓷器。

(这些东西,放在现在大概价值三十多个亿。)

三叔绕着墓室看了一圈,没看到能搬出去的东西,就绕了回来,棺床后面是照壁,他绕到照壁之后去看,还有一些空间,不过地面上仍旧空空如也。

他不由暗骂了一声,心说也真是抠门,怎么什么都没有,难道那棺材这么大,还是铁的,那家伙把陪葬品全塞里面了?这棺材给当成保险柜用了?

想想还真有可能,不由有些郁闷,这时候,他忽然看见照壁的背面,浮雕着很复杂的雕刻。

壁画不值钱,但是古墓的石雕价值连城,虽然这照壁很大,不太可能运出去,但是三叔看到了,还是忍不住看了一眼。

手电照过去,就很让他意外,照壁后面的浮雕,雕刻的不是一般的瑞兽云佛,或者礼乐升仙的图样,而是好几座宫殿,飞檐凤顶,雕梁画栋,雕刻得非常的精细,甚至连瓦片都一片一片地浮雕了上去。而且每座宫殿的外观都不相同,有的是两层的,有的是一层的,视觉上也有远有近,错落有致。三叔数了一下,一共有七座,列成北斗七星的排列,每座宫殿之间,能看到有无数的亭台楼阁半隐半现,而其他的细节,都被雕刻的云雾遮住了。这幅浮雕的背景,是巨大的山岩,显然一座大山,而宫殿的构图是在整个浮雕的下部分,意思很明白,这是七座修建在一个巨大山谷里的宫殿,山谷里云雾弥漫,把宫殿之外的东西遮掩得朦胧而神秘。

这浮雕是什么意思?三叔错愕了一下,所有古墓中的壁画都有着意义,不是有象征作用,就是歌颂墓主人生前的丰功伟绩。这浮雕是代表着神话中的仙国,还是在歌颂墓主人什么?

三叔当时不知道这里的墓主人是汪藏海,所以也无从联想,不过这精致的浮雕,给他留下了非常深刻的印象。他告诉我,就是在当时,这照壁也是无价之宝,要是能带出去,他就把它放在卧室里,天天看着。

不过,这照壁过于巨大,当时想要运出去是不可能的,三叔虽然心痒难耐,但是也没有办法。他仔细看了几遍,便想让解连环过来,将这东西拍下来,以后也好在同行间吹牛。

正想开口,他却忽然闻到了一股奇怪的味道,好像是什么东西烧焦了。

他愣了一下,心说怎么回事,这里是墓室,怎么会有这种味道出现?忙跑出照壁,向外观看。接着,他就看到了让他瞠目结舌的一幕。

只见解连环站在铁棺之上,手足无措,而那铁棺上的铜人嘴巴里,竟然冒出来滚滚的黑烟。

\chapter{尿}

三叔顿时就冷汗直冒,这棺材怎么就冒出烟来,看解连环的样子,他就感觉到不妙,难道这小子干了什么?

一把就将解连环拉下铁棺材,问他娘的怎么回事?

解连环结结巴巴,做着古怪的动作,但是显然太紧张了,什么也说不清楚,说了半天才说出两个字:“我……我……火……火。”

三叔看着他的动作,就看到他手里拿的东西,那是火折子的盖子——火折子是一碰就着的东西,所以一般都用芦苇的秆子包起来——一下他就明白了是怎么回事。

解连环肯定是好奇这棺材里的情况,点燃了一只火折子,将其丢入了棺孔之内,然后把自己的眼睛贴到了棺孔上,往下去看。

这叫做凿壁偷光,是从北派模仿来的功夫,也是土夫子常用的伎俩,特别是新手开棺,前走三后走四,要谨慎再谨慎。北派的摸金贼甚至可以使用凿壁偷光,不进古墓就从棺材里拿走东西,相当的了得。但这算是掏沙这一行里的旁门左道,实际用起来有很多的限制,而且有很大的风险,所以一般老手是不用的。这解连环不知道是自己琢磨出来的,还是和那些半调子学的。

凿壁偷光最大的风险,就是可能会烧坏棺材里面的明器,特别是尸体干燥的情况下,尸体上腐烂的丝绸干片,几乎是一点就着的东西,一旦烧起来,像古简、斗珠之类的东西一下就没了,连灭火都来不及。所以要求做的人十分的小心才行,这解连环竟然想也不想就用了。

三叔懊恼地骂了一声,心说不看着这小子真是失策,这棺椁他很感兴趣,不说其中肯定有好东西,就是里面的尸体,三叔也想看看,要是棺材里的东西被烧了,那实在太可惜了,说出去也得给人笑死。

想着三叔一下就推开解连环,冲到棺材边上,附身对着那棺孔用力吹气,想把棺孔里的火吹熄掉。没想到一吹之下,黑烟更加猛烈地从棺孔里直冒出来,呛得三叔几乎呕吐出来。他忙闪开脸,又摸出腰间的水囊,就往那棺孔里浇去。

一路过来被海风吹得口渴,水囊中已经没有多少水了,倒了一下就没了。这点水根本没用。

“这狗日的,”三叔急得直冒汗,转向解连环,就看到他腰间的水囊还有点鼓,看他还在那里发愣,气得大骂,“你愣着干什么,他娘的把水囊给我!”

“水?哦!水囊!”解连环这才反应过来,忙解下水囊,三叔一把抢过来打开,一下倒了进去一半,只见那黑烟一晃,不但没有把烟压下去,反而有火苗从棺孔里蹿了出来。

三叔一看不对,怎么是这个动静,一闻那水囊,不由大骂,里面竟然是烧酒。再一看那棺材,铁棺的棺孔口都烧了起来,浓烟几乎弥漫了整个墓室。

当时他一下子也蒙了,也不知道怎么办。这火在铁棺之内已经烧得很大,伸手进去灭火也不可能了,况且要着了什么道,连命都可能没有,用水,少量的水根本不起作用,然而要是不去管,这棺材算是完了。这种烧法,连玉石都能烧裂了,这墓主人一看就知道不俗,要是东西烧了,棺材里面真有夜明珠什么的,自己不得郁闷死?

(其实当时只要拿什么东西塞住那棺椁的孔就行了,但是情急之下,三叔他们根本没想到。)

看着火越来越大,棺材孔里噗噗地冒出黑烟,他和解连环心急如焚。

就在三叔心里绝望,心说油斗成焦斗的时候,突然一边的解连环做出了一个让人目瞪口呆的举动。他一下跳上棺椁,就半跪下来,解开裤腰带,运气走尿,往那棺孔里灌了一泡黄汤,一时间尿骚尸臭火燥混在了一起,极度的难闻。

那完全是急疯了的想法,因为他动作太出乎意料,三叔根本来不及阻止,等反应过来的时候,已经晚了。

三叔一下就蒙了,自古下斗,南派虽然豪放不羁,有着一死万事消,开棺随自在的随意性,但是基于这种活动的危险性,在实际的做派上,南派也是十分小心的。像这样往棺材里灌尿的作孽事情,解连环恐怕是第一个,也亏得解老爷子不在场,否则非气死不可。

不过,解连环的这泡老尿,还是有点威力的,很快,里面的烟一下就小了下来。

尿完之后,解连环自己也蒙了,一下坐倒在棺材上。

三叔眼泪都下来了,看着铁棺上的铸人,擦了擦头上的冷汗,只觉得背脊发冷,心里有几分不祥的预感。

“哨子棺”里鬼吹哨,大凶之物,如今给烧了一把又被灌了一口黄尿,这一次这梁子结大了。不说是粽子,就是一活人,你用火烧他嘴然后再浇他一嘴尿,他也得和你拼命啊。

他冷汗淋漓地看着这铁棺材,就琢磨着会发生什么,有什么东西会从那个洞里出来吗?

烟越来越小,逐渐几乎看不到了,看来火确实是灭了,两个人都死死看着那棺材,一直到一点烟儿也看不到。

然而,棺材里却一点动静也没有,好似刚才的事情从来没发生过。

三叔擦了擦头上的汗,松了口气。他心说黄王保佑,看来解连环命不错,这棺材虽然是哨子棺,却也是一具死棺。

死棺,也就是这棺材里面的粽子早就化了,只剩下一些没有威胁的腐骨,古墓中大部分的棺椁都是死棺材,要不然,盗墓这一行恐怕就没人干了。

死棺是没有危险的,刚才烧了一把火,又灌了一通尿,如果不是死棺材,肯定就出事情了,这么久没动静,应该可以确定了。

又等了一会儿,还是如此,三叔才最终泄了劲。他一下坐倒在地上,解连环看他放松了,知道没事了,也一下坐了下来,哭了起来。

三叔摇头苦笑,心说真是作孽啊,自己竟然和这种货色一起下地,命都短了几年,以后千万不要了,也亏得没有危险,不然这一次真的可能被他害死。

想着,三叔忽然心中一动,心说既然没危险了,那岂不是不用等到明天,今天就可以摸东西了?

来回这里一次,还是要冒点风险,到底是文锦的队伍,不太方便。而且棺材这洞的位置,摸进去,如果对着脑袋,那摸脑门和脑袋两边,还有胸口,肯定能摸到。要是对脚,也有脚底,那是放玉器的地方,都可能会有好东西,但是不会太多,一次就能带走。现在如果把东西摸出来,那明天就不用下来了。

虽然洞里全是尿,但是盗墓的,什么恶心的东西没见过,况且还是自己的,就算拉屎进去,他照样也敢伸进去摸。

一方面,是盗墓贼特有的贪欲;一方面,却是对这个洞的恐惧。三叔在那里天人交战。但是很快,贪欲就赢了,胆子不大也不敢来干这一行,三叔对自己说,他娘的就赌上一把再说。

想着他站起来爬到了棺材上,对棺材拜了拜,撸起袖子一咬牙,一闭眼,先就将手伸进了那个棺孔之内,向下摸去。

可手一入棺材孔一寸,里面的温度传上来,三叔就后悔了。当年传说的张盐城,那不是靠运气的,那靠的是手指上的真功夫,如今自己就这么贸贸然地将手伸进去,这他娘的实在是太莽撞的事情。

他想缩回来,但回头一看,就见解连环在下面目瞪口呆地看着他,这时候回不得脸来,只好硬着头皮继续往下摸。

单手探洞,有一种无法形容的感觉,手越往里伸他的心跳就越快,然后手指越麻,表面上他的脸上什么表情都没有,其实最后他的手碰到尸体的时候,后背都湿透了,伸在里面的手指抖得一点力气也没有。

这种经历可以想象,我听的时候,都感觉到浑身发抖,就算是找一只普通的箱子,挖个洞让人把手伸进去,都会有一种莫名的恐惧,何况是一具棺材。

三叔摸到尸体之后,按了几下,发现手指黏糊糊的,头皮就越发发麻。凭手感那应该是古尸的嘴,摸了几下,他只感觉那应该是一具发黑发肿的尸体,怪异地张着嘴,姿态似乎和棺材上的铜人一模一样,不过摸不清楚细节,让他感觉到十分不安的是,他摸到火折子正掉在古尸的嘴巴里,还烫得很。

他心说这也真是作孽,随即咬牙把手指往里探,他先是把火折子拨到一边后,又摸到一块坚硬的圆环状东西。

丫的,是压舌头的玉饼,三叔心里窃喜,说道:“有了,这东西烧不坏!”一下捏住,就想把那东西从洞里夹出来。

可是才钩了一下,三叔就感觉不对,这玉饼的重量惊人,提起了半分就提不动了,再用力,就感觉整个铁棺轻微一震,却有一阵“咯咯咯咯”沉重的发锈的金属拖动声从脚底传了上来。

三叔的脸色顿时大变,心说,糟糕了,是个机关!

\chapter{机关}

转瞬之间,那机关搅动声就从棺材的底部一路传动了出去,没等三叔反应过来,就听他们出水方向的黑暗处“咣当”一声,似乎有什么沉重的东西,掉进了来时他们看到的巨大铁缸之内。在封闭的空间内,这一下声音极响,震得他心都缩起来了。

三叔忙将手抽了出来,也顾不得污秽,往身上一擦,就转动手电向铁缸的方向照去,心说糟糕了。

这里不比地上,如果有落石压住墓道,或者墓顶灌下黄沙,还有时间可以反打盗洞出去,这里是海底的珊瑚礁盘内,一旦有任何机关将他们困住,那是必死无生,连办法都不用想,直接挑自己陈尸的位置就可以了。

一边的解连环也吓了一跳,因为震动是从棺材里传递出去的,他以为棺材会有异变,一下子退出去很远。

虽然三叔已经见解连环恨之入骨,恨不得一刀切了他,但到底是自己的亲戚,也不能放着任由他乱跑,就喝住解连环,让他别动,自己跳下铁棺,小心翼翼地往那铁缸处靠去,想看看到底这机关触动了什么,什么东西掉到缸里去了。

铁缸离他不到二十步,很快他就来到铁缸的一边,此时已经听不到机关运行的机械声,似乎机关运行已经停止。三叔咽着唾沫往铁缸的上方一照,发现铁缸之上的墓室顶上,设有翻板,这在我们叫“鬼踏空”。墓室顶上这样的机关内往往放置着极其重的乱石,一旦触发,重物砸下来,一下就能把人砸成肉饼子。如今从上面掉下来的,却不是巨石,而是两条巨大的铁链,一直垂到铁缸之内。

三叔看了一愣,心说这是什么机关,好像并不是用来防范盗墓贼的那种陷阱。那触动机关的地方在棺椁里,东西掉入这铁缸内,砸不死人啊,这掉下来的会是什么?

想到这墓室中不符合常理的地方,三叔心中就更加的疑惑。他定了定神,掏出匕首咬住,趴到了铁缸之上,小心翼翼地顺着铁链往下看去。

一看,他就看到了一个奇怪的东西,躺在铁缸底部,仔细一看,发现那是两只黝黑的鬼爪一样的琵琶锁,锁着一具骨骸,肢体和铁链条纠结在一起,手脚都断了,看上去似乎是一个殉葬的奴隶。

骨骸极其魁梧,身着破烂不堪的青铜鳞甲,头骨奇异,那琵琶锁正锁着骸骨的锁骨,一条锁骨已经断裂,另一条却还牢牢地挂在上面。

三叔大是惊讶,心里琢磨,用琵琶锁穿着锁骨,是古代的一种酷刑,用来限制犯人的自由。古代武功高强之人,一般的锁具困他不住,就会使用锁骨的方式紧固,锁骨穿孔之后极其脆弱,一旦过度用力就会骨折,锁骨之所以称为锁骨,也是因为这个原因。

骨骸已经腐烂殆尽,连骨头都起了死鳞,似乎一碰就会碎裂,三叔用手电仔细照去,看到这骨骸头骨的形状异于常人,不说头骨的大小,其长度就比普通人长了一倍,三叔说不出像什么,直觉是一只大个的香蕉。

这古墓之内,竟然困有这样一副奇怪的骸骨,当真是离奇到了极点。看墓室的结构,显然骨骸早就吊在墓顶之上,一碰棺材内的机关,这具骸骨就会陡然掉下,当真巧妙。

可,这到底是为了什么?如果是防盗的机关,可怖虽然可怖,却没杀伤力,能够来到这海底墓室之人,难道会给死人骨头吓走吗?而这吊着的骨骸,显然不是普通尸体,又到底是什么呢?

三叔想象力极度匮乏,心中骇然之际更是没有什么头绪,不过脑子却转得很快,刹那间想到,这骨骸如此骇人,难不成是尸变了的粽子?铁链有碗口粗细,且带着琵琶锁,显然锁着的东西生前力大无穷。早就听闻苗疆有能人在阴地养小鬼和走尸,难不成这里的墓主用琵琶锁锁了一具已经尸变的尸体,用来当看门狗?

骨骸已经腐烂殆尽,就算确实是粽子,也已经挫骨扬灰,不足为惧。三叔心中好奇,他胆子也算大,为了仔细观瞧,就爬上铁缸,一边还招呼解连环,过来把这东西也拍下来,他回去好问问。

解连环却没有回应他,三叔自然不在意,就往缸内爬去,不料铁缸的内部粘着一层从墓顶上飘下来的灰尘,他湿的脚踩去,突然滑了一下,整个人在缸壁上打了个圈儿,一下就摔到骨头堆里去了。

那骨头本来就已经粉脆,刚才掉下来的时候又散了架,如今一撞更是几乎变成了碎片。三叔赶紧手忙脚乱地坐正,端好手电去照,就看到自己正摔在骨骸的怀里,畸形的头骨就垂在他的脑袋边上,被他撞得碎裂了,露出了里面的颅腔。一大团好比蜂巢一样的东西,就粘在颅腔的内部,上面全是一颗一颗好比珍珠一样的虫卵。

\chapter{虫脑}

这些虫卵粘在颅腔的内侧,颜色是灰色的,一颗一颗密密麻麻,细看之下非常的恶心,犹如蜂巢中的蜂卵一般。

三叔不若常人,此时也不害怕,反而更起了兴趣,就爬起身来,仔细去看。

虫卵在手电的照射下,呈现出一种模糊的半透明状,三叔用匕首碰了碰,硬如甲壳,似乎已经干透了。

这是什么东西?他心说,这东西的颅腔里,竟然有这么多的虫卵,难道这些是寄生虫?

古尸体内有寄生虫,那倒也说得过去,楼兰古尸身上经常发现,不过一般寄生虫都是在五脏六腑里的,怎么会在颅腔里出现虫卵?而且把卵产满了颅腔,这是什么虫子啊,也太厉害了吧……

三叔当时科学知识方面是十分匮乏的,在文化方面,大多也是和文锦学的一些用来撑面子的东西,说到虫子或者古代的虫子,他的脑海里出现的同样是毛虫之类的形象。他想着,还推断着,这种虫子寄生的时候,肯定寄主已经死了,否则脑袋里长虫,恐怕会痛死掉,这也许是食腐昆虫的卵。

这可是个大发现,三叔心说,他记起文锦和他讲的,对于考古发现的非物质价值。在考古中,如果发现了前人没有发现的古籍或者风俗以及墓葬痕迹,都属于重大发现,这种发现对于三叔来讲当然狗屁不是,但是对于整个考古界来说,意味着巨大的名声和地位,是名留史册的东西。

他自己对这个东西一点兴趣也没有,不过当时他在热恋之中,一下就想到了文锦,心说要是这东西给了文锦,这丫头也许会觉得有用,且又不值钱,放在这里也没用。

想着他就掏出了一个牛皮袋,那是潜水时打捞东西用的袋子,底下有可以塞住的孔,出水的时候水会流出去,三叔将孔关闭,就将那头骨摘了下来,连同其他一些碎骨头,都塞了进去,鼓鼓囊囊,就背到了自己身上。

做完后爬出了铁缸,去找解连环,此时经过两次惊吓,他已经毛了,贪欲也吓没了,这棺材他也不敢碰了,这个墓室他娘的太邪门,他一刻也不想再待下去。如果解连环东西拍完了,他们应该立即退出去。

此时他也忘记了解连环氧气瓶里氧气不够的事情,如果他还记得,他就应该知道出去这事情已经不好办了。

然而等他爬出铁缸,回到铜人铁棺面前的时候,一下子就发现了不对,第一他看不到解连环,他不在原来的位置,手电扫了一圈也没有;第二,解连环的手电掉在地上,照着一边的壁画,正在忽明忽暗地闪烁。

三叔只愣了不到一秒,汗就出来了,因为这种场景他看得太多了,在古墓里,只要一有人出事,手电肯定掉到地上,以往夹喇嘛的时候,栽的人多了,所以他一看到手电掉在地上,心里就一下绷紧了。

难道解连环在自己到缸里去的时候,出了什么事情,触动了什么机关?

刚才没听到什么声响啊,不过自己在缸里,的确也没有注意外面发生了什么。

什么叫经验?这就是经验了。如果是我,肯定会跑过去把手电捡起来,然后叫几声。然而三叔已经确定出事了,虽然不知道是什么事情。他再次翻出匕首,整个人开始进入一种状态,往铁棺的方向走去,去找解连环在什么地方。

他进铁缸的时间不长,解连环如果中招,也应该倒在铁棺附近。

小心翼翼地但是迅速地绕到铁棺之后,果然他一下就看到解连环倒在了铁棺的后面,蜷缩成了一团,一动也不动。三叔用手电照了照他的脸,没有反应,又扫了一圈,没有发现边上有什么异物。

奇怪,好像没有机关触动的痕迹,他怎么就倒下了?三叔有点诧异,看了看四周确实没动静,他就快步上前,将解连环扶了起来。

解连环已经失去了知觉,死沉死沉的,身上都瘫了,三叔一搭脖子,发现他没死,再一摸他的几个要害,就发现他的后脑勺滚烫,翻手一看,全是血。

操!三叔一下就蒙了,怎么可能?这家伙看上去竟然像是给人打晕了。

可是,这里是古墓之内啊,没有任何古墓的机关设计是将人打晕的,粽子也不可能这么好心只是打晕你,能打晕人的,只有另外一个人啊。

想着,三叔忽然感觉一股极度的寒意,他忙转头看向四周的黑暗,心说,不会吧,难道这里还有其他人?

\chapter{黑暗中的第三人}

三叔一想到这点,虽然不敢相信,但还是出了一身冷汗。他放下解连环,迅速地看了一圈四周。

扫过一圈之后,什么都没有看到,安静的墓室里什么都没有,而手电昏黄的光线扫过墓室的墙壁,一股莫名的寒意就侵入到了三叔的五脏六腑之内。

三叔和我们是不一样的,作为从小就在地下玩耍的人来说,死人并不可怕,因为死人只是物体,虽然有危害,但是它不会来暗算别人。然而,活人就不同,三叔一想到这墓室里可能有第三个人在,一下子就害怕起来。

解连环这一下后脑的重击,可大可小,现在我们看无论什么电视剧电影,想要一个人晕倒,只要拿什么东西在他后脑上敲一下就好了。实际上三叔这种人知道,你把一个人敲昏的力度,和把人敲死的力度是相同的,你一下敲下去对方是死是活完全是看运气,而你稍微敲轻一点,最多把人敲迷糊了几秒,真正不把人敲死而敲晕的方法,是敲人的后脖子,会功夫的人连敲也不用,只要用手捏一下人就晕了。

所以解连环这一下挨的,情况到底怎么样,他也不是很清楚,只知道如果是人打的,对方这一下下去,显然是下了杀手的。摔跤是绝对摔不到这么重的,摔死也是内出血,头皮绝不会破成这样。

但是,怎么可能会有第三个人在这里呢?

如果说这里是陆地上的古墓,那碰到个把熟人虽然不常见,但也是说得通的事情,可是这里是海底,难道正巧有另外的人也知道这里,潜了进来?

不可能啊,这样的可能性也太低了,三叔脑子转得很快,一下他就想到了一个可能性。

他娘的,难道是自己和解连环下水的时候,给船上的人看到了?有人跟着他们下来了?

现在一想这倒是绝对有可能的,这附近不太可能有别的船了,而自己抓住解连环的时候,确实闹腾了一下,难道当时有人给吵醒了?没叫他们,反而一路尾随过来了?

一路过去海上漆黑一片,海黑海黑,那就是一片混沌,什么也看不清楚的黑暗,如果有人跟踪,决计是发现不了的。况且两个人只顾赶路,根本没有想过这些事。

说实话,三叔当时对于那一批考古队是不当一回事的,他想的就是给发现了,文锦也能给他瞒过去,那批人就算再怀疑,也不能怎么样,所以他和解连环下水的时候,并没有太过在意会不会有人知道。但是实在没想到,会有人偷偷跟下来。

会是哪个呢?考古队里的人大部分他都认识,虽然说有几个陌生面孔,但是他平日里看人也颇准,除了解连环之外应该无人可疑啊,如果是船夫的话呢?倒也有可能,难道说自己下水给船夫看到了,有船夫好奇跟了出来?

不过到这里来必然要有潜水器械,那几个船夫游泳厉害,但是潜水器械这种东西,应该不会操作啊?

这么说来,应该还是考古队里面的人,是哪个呢?

三叔也想不出来,心里就说:不管如何,他要是偶然跟来,此时应该就叫出声来交涉,如此不出声,还下了这么重的手打晕了解连环,刚才没有听到任何的大叫,应该是偷袭,那肯定是有问题。等我先制住他再看看他到底是什么人物。

这些思绪是如闪电一般从三叔脑子里闪过的,一想到这一点,他就把手电关了,四周一下暗了下来,光线只剩下解连环那盏摇摆不定的手电,然后他就矮身趴到地上,向边上滚去。

这是不让对方知道自己的位置,敌明我暗是最有机会的,而趴下来,是三叔特有的动作,那是怕对方听到声音扔东西过来。比如陈皮阿四那种人,你如果站着,就是光听心跳,他就能打中你。

滚了十几步后,他大约感觉已经远离了铁棺,就凝神静气,努力去听周围的声音。

墓室里原本就极端的安静,可以说是掉根针都听得见,三叔一下安静下来就更静了,他都听到了自己的心跳声,好比打雷一样。

在心跳声之外,他果然听到了一些莫名的声音,十分的轻,听不出方向,但是确实就在四周,好像是呼吸声,又好像是极端轻微的摩擦声,让他一下出了冷汗。

果然有人。

三叔暗骂了一声,闭上了眼睛,努力去听听到的声音,想辨别声音的方向。

然而,只听了一下,那声音就消失了,好像对方知道他发现自己,屏住了呼吸。

三叔心跳加快,一边慢慢地爬了起来,如果那人在附近,要是不小心给踩到,那自己趴着就处于劣势了。

刚刚爬到一半的时候,他忽然听到了就在自己的左后方,有一声骨骼的关节声,贴得极其近,三叔一下就有点慌了,把身子转了过去,想往后退一点,远离那个声音。

就在那一刹那,他突然感觉到脸边上闪过一丝微风,他心说不好,忙想低头已经来不及了,黑暗中忽然传来一阵劲风,一个人猛地扑了过来,一下将三叔扑倒在地上。随即,三叔感觉到自己腰间插的手电被人拔了出来,接着那人力道却松了,三叔猛地躬起想挣脱,突然下颌一麻,被人用手电狠狠地砸了一下,顿时满口都是血。

他娘的,对方看得见我!三叔在那一刹那就闪过这个念头。

在一片漆黑中能够准确地扑杀过来,而且一下就能抽出自己的手电,显然他看得很清楚。

这是怎么回事,难道对方有一对猫眼?

惊骇之余,他用力把头摆向另一侧,然后对方第二下还是准确无误地砸了下来,一下砸在三叔的鼻子上。这一下被砸得极重,他的头都抬不起来了,嘴巴里一股咸味涌了上来。

这次三叔真毛了,他自小就是孩子王,除了被爷爷打,什么时候吃过这样的亏?马上就起了杀心,一抬头,匕首就划了过去。

然而什么也没有划中,反而下巴上又给狠狠打了一下,那都是杀手,三叔的下巴连痛都感觉不到了,接着他拿着匕首的手就被人死死地抓住了。

这样躺着力气用不出来,手就被他按倒在地上,三叔大骂了一声,心说你他娘的还想强奸我怎么的,猛地抬头就是一口口水,连着嘴巴里大量的血就喷了出去。

凭着身上的感觉,他知道对方闪了一下,就是这一刹那的工夫,三叔整个人扭了起来,一下挣了出来。对方没有想到三叔能挣脱,忙俯身再用膝盖去压,就中了三叔的圈套了。

普通人打架,一人被另一人压住,如果一旦对方用力松了,第一个念头肯定是挣脱出去,然而别人在你上面,想再次制住你非常容易。所以三叔佯装挣脱,等那人再次压下来的时候,三叔另一只手已经抓住了自己那个装着人头骨的隔水袋,轮起来就砸了出去。

那一下也不知道砸在什么地方,就听对方一声闷哼,翻了出去。三叔哈哈一声,一个翻身就爬了起来,抄起隔水袋,就往对方闷哼的地方砸了过去。

可惜那里面骨头肯定已经碎得不成样子,隔水袋甩过去也没有什么威力,三叔也不管有没有砸中,跌跌撞撞地就往解连环手电的地方冲了过去,抓起手电就朝身后照去。

之前考虑的在黑暗中对峙已经没用了,对方竟然能够看到他,那他娘的自己刚才那种关手电然后趴倒翻滚的动作就他娘的是搞笑了,现在要制住对方,只有把对方逼出来。

然而手电闪电一般扫过一个半径之后,他却什么人也没有看到,袭击他的人不见了。

他当时已经是火头上的状态,也没有什么冷静了,一看人躲起来,破口大骂,端着匕首就去找,才绕了棺材一圈,就听到他出水的地方,传来了一声入水声。

他娘的跑了?三叔跳了起来,急追过去。冲到入水口,看到那人已经下水了,水面上还荡着波纹,三叔怒起来想一头跳下去,然而一看水在手电照射下是黑的,下去万一对方埋伏在那里,吃不了兜着走,只得硬生生忍住,指着水大骂了一通。

因为不知道是谁,他索性把船上除了文锦之外的所有人全部都骂了个遍。

然而骂着骂着,他就觉得不太对劲,身边似乎有什么奇怪的咝咝声,听得耳朵发痒。

三叔把手电照向发出声音的地方一看,顿时浑身冰凉,几乎没晕过去。

原来自己的氧气瓶栓,不知道什么时候,已经被人拧了开来,氧气正在咝咝地往外冒。

\chapter{抉择}

说到这里,三叔就长长地叹了一口气,捏了捏自己的眉心,似乎下面的事情并不想说起。

而我听到了这里,也是一身的虚汗,三叔停下来,我也正好可以喘口气。

这事情真是惊心动魄,一路听来我都有点窒息的感觉,特别是听到发现了第三人的时候,我都感觉自己在听评书一样,原来事情竟然是这么发展的。

这个人是谁呢?我心说道,从行为来看,此人相当决绝,氧气瓶栓是不可能给碰开或者自己松开的,现在被拧开,肯定是这个人干的。而且,非常有可能是尾随三叔进来的时候就打开了,里面的氧气必然所剩不多了。

这海底墓室离海面有着相当长的距离,没有氧气,三叔和解连环必然会活活困死在这里。这个人回到船上,也不会把三叔的事情说出来,这个古墓是绝对不会被发现的,船上的人想找也找不到,自然不可能指望船上的人过来接他们。这是非常恶毒的杀招,显然他一定是要三叔和解连环死在里面。

这样一说起来,三叔当时所处的情况其实比我们还要糟糕,他只有一个人,而且深入海底的距离比我们厉害得多。

不过三叔现在坐在我面前大咧咧地抠脚喝茶,显然他最后还是找到办法出来了,这我倒不需要太紧张。

两个人都定了定神,三叔缓了一下,就继续说了下去。

当时,看到那情形,他的脑子立即就炸了,忙上去拧上了气栓,拧好后,浑身已经吓得冰凉了。

那一瞬间,他就以为自己完了,肯定死定了,而且还是他最害怕的死法,在封闭的古墓里,活活困死。他为自己的大意后悔,又是满心的憎恨。对于三叔来说,死在古墓里就死在古墓,如果是中机关而死,那是命没有办法,但是给人害死,他是大大地不甘心,实在是懊恼。

他立即去看氧气表,看了之后牙就咬到牙龈里去,他自己的氧气瓶,可能是因为气栓的防漏作用,没有漏光,还剩下十分之一的氧气,解连环的氧气瓶里也剩下一些,那几乎就是一点点,估计呼吸个三四十口就没了。

这可能还是因为放气的时间比较短的缘故,要是晚几分钟,就可能是几个空瓶子了。

这点氧气,几乎就和没有差不多了,他们进来的时候,三叔用了一半,而解连环用了一半还多,这点氧气是远远不够出去的。

想到这个,三叔就绝望了。他看着四周漆黑一片的墓室,一股极度的恐惧侵袭了过来,心说难道自己真的会在这里活活地困死吗?

越想三叔就越害怕,而且是真的害怕,不是紧张或者焦虑,他当时立即有了一个念头,就是他不能死在这里,要死也应该死在别的地方,那一刹那他几乎想一头跳进那个入水口淹死自己。

不过三叔到底是枭雄,他的这种恐惧很快就被压了下来,他拍了自己一个巴掌,骂了声没出息,就冷静了下来,开始思考应该怎么办。

我、胖子和闷油瓶被困住的时候,因为一点氧气也没有,所以只能把希望寄托在寻找氧气瓶上,然而三叔当时还有氧气,而且氧气的量也不多不少,非常尴尬,所以他的所有思维,很快就被这些氧气的量吸引了。他首先开始考虑,这点氧气有没有一点可能,能撑到外面去。

算来算去,其实都不可能有结果,因为氧气太少了,虽然,刚才进来的时候,一直是很小心谨慎的,速度并不快,如果出去的时候快一点,能够缩短很长的时间,但是,进来的时候用了五份氧气,现在出去要用一份,也就是说,出去的速度必须是进来的五倍。

进来的时候,大概是三十分钟,那出去要六分钟?他又不是鱼,怎么可能做到。

这下三叔又有点难受,他马上又拍了自己一个巴掌,把自己的恐惧拍掉,逼着自己继续往下想。

那六分钟能到达哪里呢?从这里出去大概就要三分钟,六分钟,只能到达那片巨大深渊的出口,这已经是最快的速度了。

一旦进入到深渊的出口,那么大概只需要十分钟,就一定能出去,也就是半个小时路程,如果运气好,则可以在十六分钟内走完,而且,他看了看表,马上就要退潮,到时候,那洞口会露出海面一些,一点空气会进入洞的上方,这样,也许不用到洞口就能呼吸到空气了。

那么自己还能憋气一分钟,则只要能够得到再呼吸十分钟的氧气就行了。

可是,这十分钟的氧气去哪里找呢?这里可是一点都没有办法,三叔抓耳挠腮,就条件反射地到处去看,希望能看到什么给他启发的东西来。

可是,古墓之中会有什么启发,难道会发现一个明清时候的陶瓷氧气瓶不成?

这想了还是等于白想,三叔就懊恼地用力拍了一下入口的水面。这时候,他就看到下面黑黑的海水里,映出了自己的倒影,他把手电偏了偏,倒影清晰起来,他一下就发现了能提供给他十分钟氧气的东西了。

三叔也真是突发奇想,他当时看到的,就是他身上的潜水服。

那么潜水服怎么当氧气瓶呢?三叔想得十分的巧妙,他把潜水服的袖子和裤管子都扎起来,然后用力一兜,把里面的气充满,之后把领口也扎起来,那潜水服就变成了一个气囊。他跳入水里,就解开一个袖子,当成氧气管吸。

一下去,他就发现还真管用,他娘的,他吸了大概三四分钟,才觉得空气浑浊起来。

有门有门!他大喜,立即上来,跑去把解连环的衣服也扒了下来,做成了另外一个气囊,然后把两只水囊也充满气。心说十分钟有了!

想着他一刻也等不下去,立即就拖着所有的东西,准备下水出去了。

三叔的性格不像我会犹豫,也不会选择保守的方式,所以他当时没有一点的犹豫。

不过,就算这些氧气能够撑到外面,那也只有一个人能勉强出去,这个人一定要拿走两只氧气瓶,另外一个人必须在这里等那个人回来接他,如果那个人死在半路上,那就没人会回来了,这个心理压力是巨大的。

三叔当时并没有感觉这是一件多么严重的事情,他心说反正解连环的氧气本来就不够,这下子只不过更严重了而已。而且,此时他也根本就没心思管解连环,他自己已经进入到了一种极度亢奋的状态下。

他将解连环摆到棺台上,然后拿刚才用来砸人装着人头骨的隔水袋给他当了枕头,让他的姿势舒适一点,就回到入水口,想也没想地下了水。

事实如三叔所料,六分钟过后,他已经进入了那深渊之内。氧气竟然还有一点。

三叔此时的心已经安定了下来,心里还真佩服自己,心说这样都困不死我,我回到船上,那个暗算我的王八蛋不给我吓死。

他吃力地拖动着身后两个巨大的气囊,就不由自主地往上浮去,也给他省了不少力气。他凭借着记忆,往这个深渊的出口游了过去。然而,让他没有想到的是,等他游到他认为的那个入口位置的时候,他却愣了。

那里什么都没有,只有一片凹凸不平的珊瑚礁石。

嗯?他就纳闷,再往边上照,一路照过去很多,都没有看到出口。

一下他就凉了,他娘的事情没他想得这么顺利,看样子自己好像记错了入口的位置!一紧张,一出冷汗,他就去看氧气表,只见氧气表的指数,已经在零以下了。

\chapter{上帝的十分钟}

三叔慌了,但是他知道这个时候绝对不能紧张,他把身上的氧气瓶解开,踢了开去,然后接上了解连环的那只,继续去寻找入口。

其实此时,事情已经十分的糟糕了,三叔用手电往四周照的时候,就发现四周全部一片幽深的黑暗,他连来时候的方向都搞不清楚了。

看来自己想得太天真了,三叔暗骂了一声,一股比困死在古墓里的恐惧还要剧烈的心跳开始出现。那就是他意识到,自己可能死定了。

不过这一次极度的恐惧之后,三叔反而平静了下来,心说自己还有十分钟的时间,希望也许就在这十分钟里,如果找不到,也好,不过是早死完死的问题。

他凭借着直觉,再次开始搜索,很快,解连环的氧气瓶也空了,他将气囊解开,开始吸气囊里的空气。然而,四周还是一片漆黑,这种感觉让人非常的无奈,特别是你想一个东西,却怎么找也找不到的时候,三叔开始绝望起来。就在这时,祸不单行,忽然,解连环的手电闪了闪,竟然熄灭了,一下子四周竟然一片漆黑。

三叔一看,心说看来上天要我死,我也没有办法了。就在这个时候,他忽然看到,自己前方的黑暗里,出现了绿色的光点。

哎呀,是舞乐古尸!三叔打开腰间的探灯,朝那里照去,果然看到是那群古尸又漂了回来,而且离他非常近,只有五六米。

三叔心里出现一丝希望,心说对了,这群古尸的运行轨迹经过那个入口,跟着这些尸体,就能找到那个入口了。

于是他游了过去,游入了那群古尸之内,跟着它们前进。

一靠近他就发现,古尸好像是在跟着一股水流走,他也冲入这股水流,开始自动往前漂去。同时用探灯照上面的情况。

然而,让他焦虑万分的是,这尸体漂得极慢,很快,他几乎把第一个气囊全部吸光,还是没有找到那个入口。

三叔对我说,当时他的状态已经快疯了,但是毫无办法,只能继续跟着,他只有寄希望于奇迹了。或者说,他当时的心里根本已经没有心情来害怕,也无法去想氧气的事情了,只希望自己能立即看到那个入口。

不过,等他终于看到了那个入口出现在头顶的时候,第二个气囊也几乎空了,两个水囊里的空气,最多能撑两分钟,这要是进入就等于自杀,如果顺着水流下去,倒是还有希望能回那个墓室。

三叔看着入口,又看了看下面的黑暗,当时就作出一个决定,他怎么样也要搏一下,下去,只不过是死得晚一点,两分钟,虽然不可能,但是也要去试,他不想等死。

他深吸了一口空气,就往上游去,可是游出水流的一刹那,因为外面水速度慢,他被卷了一个跟头,一下就撞到了一具古尸的身上。

这水流的力量是相当大的,三叔控制不住姿势,忙抱住了那具古尸,用力稳定身体。

这时候,他忽然灵光一闪,看到那古尸的嘴巴里,竟然有气体喷出来。嗯?他愣了一下,一按那古尸,立即发现,这不是真人,而是一个用竹子之类的东西编的,外面糊了石胶和泥浆油的人俑,而且,很明显是空心的,里面有空气!

不会吧,三叔想着,立即拔出匕首,一刀捅了进去,马上气泡就从破口喷了出来。

三叔像吸血一样扑上去,吸里面的空气,只吸了一口,他就知道有门了,虽然里面的空气极度的难闻,但不是毒气,能呼吸。

想着,他扯起两具古尸,就推离了那道水流,进到了入口之内。

说起来匪夷所思谁也不相信,然而三叔真的就这样成功地捡了一条命回来。

他回到了船上,当时天已经白了,太阳快升起来了。他一回到船上,将器具放好,就看到了第三具湿的装备放在角落里,这下子他马上就确认了,要置他于死地的人,肯定就是在考古队里的。

然而他回到卧舱,发现所有人都睡得死死的,一个一个看了一遍,他根本就无法看出哪个人有异样。

如果是在平时,他肯定一个一个绑起来问了,现在碍于文锦的面子,他不可能这么干,只得忍了下来,也佯装睡觉。一直到两个小时后天亮,才佯装发现解连环不见了,于是他们便开始寻找。他本想引他们发现那个礁洞,没想到的是,却在那附近找到了解连环溺毙的尸体。

三叔对我道:“我不知道他是怎么出来的,看当时的情况,有可能是他醒了之后,发现氧气瓶不在,只剩下自己一个人,在恐慌下强行出来然后溺死的。我实在是没有想到,他竟然会那么蠢,不过现在想想,说起来也算是我害了他的性命。”

我听了长叹一声,对三叔说:“你上来的时候,应该马上下去救他的,那样就不会出这种事情了,你竟然还能睡觉。”

三叔点头,也叹气道:“当时我是感觉马上下去救人太危险了,我不知道是船上哪个人想要我的命,再进去恐怕还是会着了别人道儿,反正他们醒来之后,马上就会发现解连环不在,肯定会去找。我已经将来时候的充气艇留在当时的礁石处,只要到时候将他们引到那里去,然后趁乱进洞,来去最多也只要半个小时,否则我一个人带着两套器具连夜出海,不仅会给人怀疑,而且救出解连环之后,事情也不好交代。”三叔摇头,“现在你知道为什么这事情我不想提了吧,这是你三叔我最后悔的事情。”

说起这个,我想起了那血书,这下就清楚了为什么解连环会认为是三叔害了他,妈的后脑挨了偷袭,解连环肯定不知道是谁干的,他不可能想到古墓里还有第三个人跟了进来,那醒来之后第一个想到的就是三叔了,然后一看自己的潜水设备没了,那还不以为是三叔要杀他。

千古奇冤,我一下就想到了金庸小说那些解也解不开的误会,还以为是文学夸张,没想到竟然真的会发生。

最后解连环从哪里拿到的蛇眉铜鱼,尸体又怎么出现在礁石下,已经无从考证。想必他在绝望之中,找到了什么出路,但是水下古墓,就算能出来,也逃不过那一段海水,解连环终究没有逃过他的宿命。

解连环误会这事情还是不要对三叔讲的好,免得他听了之后不舒服,我心里暗自打算。

三叔接着道:“接下来的事情,我在济南已经和你说过了,当然,当时我并不想让你知道解连环的死和我有关,所以我和文锦他们第二次进海底墓穴,后面的事情,我没有说。其实我当时进去,确实是装睡,因为我怕他们会到达那间墓室,我不知道解连环会留些什么在里面,所以想在他们到达之前,去看看。另外,我知道下来之后,那个攻击我的人肯定会露出马脚,我想靠这个把他找出来,给解连环报仇。”

此时,我就想起了闷油瓶和我说过的事情了,一想之下,似乎提出探索古墓的,是闷油瓶自己,心里豁然,问三叔道:“那你有没有看出来到底是谁,是不是就是那个张起灵?”

他的身手、他的背景都十分的神秘,如果是他的话,事情也比较好解释。

三叔就皱起了眉头:“他们出去之后,我跟在他们后面,此人确实相当可疑,但是,却有更加可疑之人。总之,看到后来,我也弄不清楚了,我是看谁都可疑,不过我个人认为,以那小哥的身手,我这点三脚猫的功夫,恐怕当时就直接给打死了,不太可能是他。”

我也意识到了,于是点头,闷油瓶平时看上去柔柔弱弱的,睡不醒的样子,他要发起狠来,就是直接去拧别人的脖子,那说起来是最快的杀人方法,三叔肯定不是他的对手。于是又问:“那接下来呢?”

“接下来……那小哥儿带着那帮人出去之后,我就偷偷跟在后面。这古墓之内,他们进入到那个水池的墓室之后,我当时并不知道那水池底下还有通道,我以为他们兜了一圈儿之后会出来,就待在甬道的黑暗中,等了一会儿,他们竟然没出来,我心中一动,怕他们遇到了危险,就跟了进去。后面的事情,那小哥应该和你说过了,我只是跟在后面,他说的应该比我更清楚一点。”

我这时候就想起了一个细节,问道:“那他说你装娘儿们照镜子来引导他们过奇门遁甲,也是真的?”

三叔“嗯”了一声:“什么娘儿们?”

我把闷油瓶当时说的情况,重新说了一遍,三叔顿时睁大了眼睛,“有这种事情?”

我咧嘴,心说别说你不知道。然而三叔却真的倒吸了一口冷气,站了起来来回踱了几步:“他真的这么说?”

“当时的环境决定我肯定不会听错。”

三叔眯起眼睛,让我详细地再说一遍,我就努力回忆闷油瓶和我说的事情,仔细地说了一遍。

三叔听完,摸着下巴,连连摇头:“不对不对!他骗人!”

“骗人?”

“我在石阶上,雾气太浓,当时的情况并没有看到,我可以用文锦保证我绝对没有下到下面去,也压根不知道这里面有什么机关。那小哥一面之词,不能就这么信他。”

我皱起眉头:“但是他当时的情况,我不认为他有必要骗我们啊。他甚至可以不和我们提这事情,我们也拿他没办法。”

三叔拍着脑袋,想了想,就道:“说得也是,那如果假设他说的是真的,也有问题,你看这小子说的:‘我’蹲在那里,他看的只是‘我’的背影,他们所有的判断完全是靠那个背影,整个过程中,除了那个霍玲有可能看到了‘我’的脸,其他人完全就只是凭借一件潜水服就判断了那是我……”

我“哎呀”了一声,心里回忆当时的话,发现的确如此,“这么说,这个引他们通过暗阵的人,不是你,是另一个和你背影甚至相貌都有点类似的人?”

三叔点了点头,脸色变得非常严肃:“如果那小哥说的是真话,绝对是这样。而且,你没发现吗?那小哥没有看到我的脸,他本来是有机会看到的,为什么没有看到?”

我回忆了一下闷油瓶说的情节,一下就一个激灵:“霍玲,他给霍玲拦了一下!”

三叔点头道:“对,就是这个细节,我一直不知道这些,真没想到,竟然在那极短的几分钟里,还发生了这样的事情……”

我感觉到头疼起来,确实,当时的情况如此混乱,能见度也极其低,闷油瓶的确有可能会看错。而且,这样看的话,那个人是三叔的这个结论,自始至终都是霍玲提出来的,只有她一个人看到过那人的脸啊,如果她和那个人是同党的话,这就可能是一个巧妙的骗局。那闷油瓶和其他人可能都错怪三叔了。

我一下又想到闷油瓶当时说过,“如果这个真的是你三叔”这句话,他是否也是在怀疑,那个人不是三叔?

不过一想又不对,闷油瓶看到三叔,不仅只有这一次,在他昏迷前也看到过三叔,而且看到了三叔的脸。这靠背影是骗不过去的了。这又怎么解释呢?

我把这个问题提了出来,三叔就叹气道:“这我就不知道了,也许是那小哥在意识弥留之际看错了。你想,他一路进来都是以为在追我,那个时候迷迷糊糊的,可能出现了幻觉也不一定。”

我摇头,对他说:“这太牵强了,小哥那样的人,不太可能会朦朦胧胧看错吧。”

三叔正色道:“如果是这样的话,那么,他肯定是在说谎了,因为我没有骗你。”

听到这句话,我心中就长叹,我最害怕的事情来了。一直以来,听到三叔和闷油瓶经历重叠的部分我就非常紧张,怕出现那种牛头不对马嘴的事情,那样就说明他们两个中肯定有一个在说谎。

不过一路听过来,我却发现两个人的话大体能对上,我已经有点安心,心想就算不是百分百的真相,也应该是靠近事实了。可是,这事情一路下来,眼看就要通了,却在最后遇到了这么一个卡,真是让人难受,而且这个卡非常的关键,如果三叔不在里面的话,那迷倒他们就另有他人,三叔就完全清白了;如果三叔在里面的话,那就完全相反,三叔就是心怀叵测的大奸角。就这么一点,就代表着完全两种结果。

两人之中,我还是比较相信闷油瓶,因为他是在完全没有必要和我们说的情况下叙述的,他骗不骗我们对他一点意义也没有。不过,三叔这次的叙述,和以往都不同,非常的清晰,而且找不到破绽,如果他是骗人,是没法把谎话编到这种程度,我感觉他这次也不太可能会骗我。而且,只剩这么一点矛盾了,他如要骗我,可以轻松地瞒过去,不需要说出和闷油瓶相反的事实啊,他可以说自己跟进去了,然后也晕了,醒来的时候他们都不在了,这我也根本找不出破绽来。

这似乎是一个罗生门,完全没法解开其中的奥妙。似乎两个人说的都是真的。

想到这里,我突然有了一个奇怪的念头,表面证据优先,那么既然我认为三叔没有骗我,闷油瓶子也没有骗我,会不会有这么一种情况,他们两个说的事情都能成立呢?

这是有点胖子的思维方式,简单明了,把事情分成三条,确定了前两条,那最后一条再不可能,也只有成立。

我把我的想法说了,三叔也正在思考,一想就摇头,道:“怎么可能?如果要这两种说法都成立,那当时的墓里,必须要有两个我才行。”

“两个三叔?”我心中琢磨,心说这好像绝对不可能,三叔又没有孪生兄弟,也不会分身,这个假设没有逻辑性。但是,如果要按照胖子的思维考虑的话,就不需要考虑逻辑性,而是要把所有可能的都列出来,枚举法。

我拿出一张纸,就开始写可能性,然而想了想,却发现,在他们两个都没有说谎的前提下,只有一个结果,就是三叔是在奇门遁甲阵的外面,而闷油瓶在里面看到的,是一个和三叔相貌相似的人。

那么问题其实不是如何产生两个三叔,而是这个相貌相似的人,是从哪里来的?用枚举,也就是几个,一个是这个人是从海上来的陌生人,一个是这个人一直藏在古墓里,这两个就很勉强了,那么有可能的就是,这个人应该是那十个人中的一个。

这倒有根据,回忆闷油瓶的叙述就可以发现,在当时他们发现三叔的两个情况都很奇特,完全有可能是他们一起下海底中的某个人干的。

可是从来没有听三叔提过队伍中有人和他很像,现在再谈论这个话题,如果有的话,怎么样他也应该想到了,而且照片我也看过,不过那照片这么模糊,看上去每个人都差不多不好作数。

那么,会不会是易容呢?我想起那小哥的手段,然而一想,就知道不可能,一次易容要三到四天的准备,五到六个小时的化妆,当时这种情况,他怎么可能来得及。

想到这里又到了死胡同,我不由沮丧,长叹了口气。

三叔看我的表情变化,就问我在琢磨什么,我把自己的推论过程说了一遍。三叔听了就笑,说我怎么学那胖子的思维,那胖子脑子是歪的。

可是才笑了几声,他好像就想到了什么,脸色一变,然后吸了一口冷气道:“哎,也不是,他娘的,难道这事情是这样的?”

我忙问他:“怎么了?”

三叔脸色苍白道:“你别说,这胖子有两下子,给你这么一分析,我好像明白这事情是怎么回事了,但……如果真的是这样的话,这事情就非常的不对劲了,甚至有点诡异了。”

我忙让他快说,三叔就道:“你说那古墓之中还有一个人,和我长得相似,很有道理,但是我感觉这个人也不需要太过相似,你想那小哥中毒了,必然神志不清,而且昏迷前就这么几秒,只要有几分相似,就可以看错了。”

我点头:“对,可是,你们那队伍中,会有这种人吗?要是有这种人,你可能早就注意到了吧,毕竟世界上有两个人相似是很奇特的事情。”

三叔的表情很古怪,他吸了口气,摇头道:“你想错了,其实世界上有一种情况下,有两个人相似是不奇怪的,而当年的考古队里,确实就有这么一个人,和我有七分的相似,但是,所有人都不觉得奇怪。”

我“啊”了一声,心说不会吧,忙问道:“是谁?”

三叔瞪着我回答道:“当然就是解连环。”

\chapter{死而复生的人}

一下子我就起了一身的鸡皮疙瘩,几乎缩在了那里,实在没想到三叔会说出这个人的名字来。

花了好久我才反应过来,结巴道:“怎么可能?”

“怎么不可能?我们他娘的是表兄弟,当时很多方面都很相似,特别是那个年代,大家穿的、发型,几乎都一样,要说这个事情能成立的话,只有他符合条件。”

“可是,当时他不是已经死了吗?”我咋舌道。

三叔很有深意地吸了口气,往后躺了一下,皱眉道:“确实,他当时肯定死了,尸体在发现的时候,已经僵硬了,都泡得涨了起来,那个样子绝对不可能救活,但是,除了这个解释,我想不出其他的办法,可以证明我和那小哥都是清白的。话说回来,运解连环尸体的船,后来也没有回码头,连同那些渔夫一起,这批人就这么消失在海上了,他也算是失踪了。”他顿了顿,又道,“其实,有时候我也想过,自己是不是太小看解连环了。”

“什么意思?”我感觉到有点心寒,“你是说,他诈死?”

三叔点头:“我调查过所有人的背景,都没有可疑,我就想到过这一层,会不会解连环当时没死,他潜了回来,和霍玲搭档,完成了这个阴谋。那样,所有的事情都有解释了,不过,当时检查他尸体的人是我,我也记得很清楚,那尸体,绝对不可能是诈死的。所以我后来把这个可能性排除了。不过,现在听你这么一说,我又感觉如果他没死,倒是能解释所有的事情了。”

我摇头道:“既然你确定他死了,我们就不要去想这个可能性了,这解连环总不是僵尸,那肯定是有别的原因。”

三叔叹了口气,对我道这事情还是暂且不去想了,现在我们的资料太少,那小哥也不在身边,讨论这个不会有结果的,还是待会儿再说,等说完之后,我们从头分析一下,说不定会有什么收获。

我也感觉是这样,一面是三叔的说辞,一面是闷油瓶的说辞,全部都是说辞,没有第三方的东西,要琢磨也只有干想。于是就让三叔说下去。

这之后的事情,三叔就说得很简短,他从海底墓穴出来之后,就开始调查整个事情。因为在解连环那里得知了裘德考的计划,所以他把解开谜题的关键放在调查这个人身上,同时寻找失踪那些人的下落。之后他与裘德考有了数次接触,然而裘德考始终没有透露给他什么消息,直到七星鲁王宫,裘德考再次失败之后。

当时裘德考发现自己全军覆没的地方,有三叔的这一伙人竟然能够全身而退,没有受到多大的损失,他开始意识到也许自己的方法根本就是错误的,于是他和三叔见面,两个人有了一次长谈,就是刚才三叔和我说的那些内容。

然而三叔确实是裘德考的煞星,他和裘德考约好合作,再次进入海底墓穴,这一次,目的是为了拍摄壁画。然而和当年在长沙裘德考背叛爷爷时候的想法一样,三叔也只是利用了裘德考的资源,他已经知道裘德考的目的。他进入了古墓,逼迫陪同的人说出了很多的机密,利用这些信息,他知道了他们的下一个目标就是云顶天宫,于是就开始与他们斗快。

这期间还有着相当多的奇遇,但是写出来未免烦琐,只要略提就可以了。

而之后阿宁他们来找我,并不是三叔安排的。他说我其实只要想想就能发现根本不可能是他让他们过来的,以我的水平,如果做他的后备肯定是死路一条,他怎么会害我?我是被阿宁骗了,当时他们认为我能从鲁王宫出来,也是一个高手,所以用了这个方法骗我。

三叔说,他当时不想告诉我这么多事情的原因,就是怕我牵扯到这件事情里来,可惜在鲁王宫的事情,裘德考肯定非常了解,所以之后,鲁王宫里其他几个能动的人他们都联系过了,我是骗来的,胖子是买来的,那小哥可能也是知道了这个事情之后,才决定混进你们的队伍的。

之后的事情,我就很清楚了,他拿到壁画之后,为了比阿宁他们早点到达云顶天宫,就直接出发了,但是一个人盗这么大的斗总是心虚的,就留了口信给潘子。他并没有准备让我也去,但是显然那个楚哥泄露了消息,将事情告诉了陈皮阿四,这老头就硬插进来,还让楚哥将我也拉了进来,准备到时候用我来胁迫三叔,当时那一批人都很厉害,他们特地找我这个软脚虾来当备用轮胎。

三叔说到这里摇头,说:“合作这么多年的人,一看自己的生意不行了,马上投靠了陈皮阿四,他娘的真不是个东西。现在坐牢,也是报应。”

裘德考背叛了爷爷,三叔背叛了裘德考,楚哥背叛了三叔,然后阿宁背叛了我们,人,真是可怕的动物。

云顶天宫中他的经历,也十分的恐怖,到底他是一个人,他也是顺着那些壁画提供的线索一路过来,但是最后中了招,被我们救了,要说起细节来也十分的精彩,但是,这里也没有必要细说,三叔也就草草地说了过去。当时因为之前的那些叙述听得已经浑身冷汗了,所以我也没有多想,很久以后我才感觉到,也许三叔在这里还隐瞒了什么,但那是很久以后的事情了。

\chapter{重启}

三叔说到这里,他所知道的来龙去脉,都已经叙述了出来。

说完之后,两个人都松了一口气,三叔大概是感觉放下了一桩心事,而我则是好像看完了一部电影一样。

我们两个都安静了下来,三叔出去上厕所了,我则闭上了眼睛,将刚才说的事情从头到尾想了一遍。几分钟后,我已经把事情理得十分清晰了。

虽然整件事情并不是百分之百的明朗,但是,裘德考、三叔的前因后果,大部分都清楚了,不知道的,也就是两三件事情。

三叔方面,在海底墓穴中的经历,是三叔噩梦的开始,也是他从一个草寇逐渐成熟起来的契机,为了寻找消失在古墓中的考古队,可以说他投入了自己所有的人生,那些钱和时间就不说了,就是一个云顶天宫,为了拖延阿宁他们的进度,他竟毅然舍弃了自己的事业,除了少数几个特别忠心的,在长沙的伙计全部都散了。三叔应该说是老九门的后裔里一个数一数二的人物,现在一切都烟消云散了。

如今自己也落得个半死不活的境地,他这个年纪其实早就该退休了。当然最倒霉的就是我,受着精神和肉体的双重折磨,然而听到后来,就发现这事情似乎和我一点关系也没有。现在想想,感觉三叔当初骗我也许真的是善意的,如果我当初知道这里面的水这么深,恐怕自己都不肯踏进来。

三叔给我的最重要的信息就是:当时在他们的船上,除了他和解连环之外,似乎有第三个知道海底古墓存在的人,这个人显然和霍玲有关系,而且这个人显然想干掉他和解连环。

而这个人肯定是在那十人之内,因为最后进海底墓的时候,海面上已经没有船了,而下去的就只有那几个人。

那么,他们一共十个人,除去三叔、文锦、闷油瓶、霍玲、解连环(死了),和一个送他回去的人,那就只剩下李四地等四个人,如果闷油瓶说的是真的,那这个人应该就是四个人之一,这四个人中应该还有一个是女人,那其实只有三个人可以选择。

如果不是解连环的僵尸归来的话,这个神秘人必然就是在这三个人当中了,当然,这里还有一个疑问,就是闷油瓶在昏迷前,看到的到底是谁。这个问题十分的诡异,如果勉强用看错了解释,虽然说得通,但是总归感觉有点问题,我回去还要好好地想想。

裘德考方面,就是裘德考在西沙考古那一年的事情,裘德考不肯说,显然这事情十分的关键,涉及了核心的秘密。而他之所以肯将之前的事情说出来,现在看来,这些事情都无关紧要,当时他追求的,只是战国帛书的含义,是学术上的事情。

但是显然,现在他的目标已经变了,我在这里就发现了一个三叔没有想到的地方,这裘德考的目的是什么?现在也是一团迷雾,拍摄死人,拍摄壁画,进鲁王宫、云顶天宫,这肯定不是学术研究了,他到底想干什么呢?

裘德考已经是一个九十多岁的老人了,他还在做这件事情,显然不为钱或者名誉地位这些事情了,这真是有点离奇。

三叔上厕所回来,我就把自己想到的事情和他说了。他点头,对我道:“这我其实想过,但是这件事情实在太复杂了,我没法来说,你看,这裘德考开始西沙计划之后的事情,我就完全看不懂了,不过,你要是仔细感觉,还是能感觉出一点线索来。鲁王宫、海底墓、云顶天宫,都是汪藏海到过的地方,表面上看,很明显,他们好像是顺着汪藏海的足迹来走,我就感觉,他们是不是在找什么东西,一件汪藏海可能留在这些古墓中的东西。”

“留在古墓中的东西?”我想了想,“难道是蛇眉铜鱼吗?”

当年汪藏海为了将东夏的秘密流传下来,通过这种方式,将隐藏着秘闻的蛇眉铜鱼藏在大风水的宝眼中,希望日后能够被盗墓贼发现。所以那几个古墓中,都藏有蛇眉铜鱼。

三叔摇头说不清楚,感觉不太像,好像是别的什么,他们反复地进海底古墓,似乎就是为了拿到汪藏海到过哪里的线索,然后去找。

“其实你三叔我才不在乎他们想干什么呢。你三叔我只想知道,西沙的海底他们失踪,到底是出了什么事情,文锦他们到哪里去了?我盯着裘德考,就是因为这西沙的事情,肯定和他的目的有关系,可惜,这事情越查越复杂。”三叔说着就叹口气,“到了后来,我都不知道自己在查什么,我只能尽量比他们快,想早一步找到他们要找的东西,这样就能威胁那个老鬼把事情说出来了,可惜,你三叔我到底老了,很多事情已经力不从心了。”

我拍了拍他安慰他,道:“那大风水的线头已经完结了,到了云顶天宫已经是终点了,那一次显然阿宁他们的目的是九龙抬尸棺,但是当时局势混乱,他们没有得手,我想他们可能会再次进去。不管怎样云顶天宫应该是最后一站了,他们进去,无论找到找不到,这事情也应该到了尾声。三叔你也别太执著了,有些事情,你已经尽力了,就别太多想。”

三叔苦笑:“尾声?我一开始也是这么想的,不过,现在看来,这么说还太早了。”说着就拿起闷油瓶寄来的录像带,拍了拍,“这事情肯定还没完,看看里面是什么东西再说吧。”

\chapter{出院}

和三叔的聊天持续了将近两个小时,开水都喝掉了两壶,讲完之后,两个人都感觉十分的疲惫,不论是精神还是身体。三叔的身体还没有完全恢复,说完就感觉到头晕。我也不想打扰他,给他处理一下贴身的东西,换了热水和茶叶,自行离开。

三叔出去买录像机的伙计还没有回来,我估计着买那东西确实够戗,停产太久了,就算能买到也不一定能放。

刚才听的时候已经忘记录像带这回事了,现在又想了起来,不由感觉到一股恐惧,之前听三叔叹气,说这事情还得接着折腾,他的语气疲惫而又无奈,就感觉到很不舒服。

关于闷油瓶的事情,我们了解的几乎是零,他当时是偶然在船上,还是有目的同样混在考古队里,连这一点我们都不知道。而且闷油瓶这个人不比三叔,他不想说的事情,怎么逼他都没反应。三叔虽然告诉了我点他的事情,但是从这个层面上来看,三叔说的那些远远不能说是事情的真相,他其实知道的比我多不了多少。

一想到这个,刚刚感觉到轻松的心情,又有点压抑起来。

处理完事情,三叔那个伙计才回来,并没有买到东西,现在市场都关门了,也只有明天再想办法。

很久没和三叔说话,又解开了心结,我的心情好转起来,晚上我就和三叔他们偷跑了出去,找了一家大排档,好好地喝了一通。吃病号饭吃了这么长时间,总算是吃到有味道的菜了,三叔很高兴,一手烟一手酒,也总算舒坦了一回。

回去的时候,他就去办理出院手续,说再也不在医院里待了,让我帮他订好宾馆的房间。

我喝得有点上头,回到了宾馆,帮三叔订了个套房,就好好地洗了一个澡,给自己泡了一杯浓茶,准备睡觉。

不过洗了之后一下也睡不着,就打开了电脑,调出了三叔在西沙出发前的那张老照片来看。

我看过很多次这张照片了,然而黑白的照片,除了能认出几个熟悉的之外,其他人很难分辨清楚,而且三叔也没有和我说过谁是谁。照片上,三叔清瘦而内敛,一点也看不出他是一个土夫子,而闷油瓶也像极了一个普通的学生。我尝试找了一下解连环,确实发现了一个和三叔有点相似的人,不知道是不是他,不由感慨,谁能想到这张普通的照片下面,藏了这么多的事情。

看了半天,发现根本没办法在照片上看到什么,我就用酒店的电话拨号,上了闷油瓶寄快递那个公司的网站,输入了单号,查询这份快件的信息。

很快查询结果就出来了,我拉到发信地点这一栏,不是空白的,有三个字的城市名称:格尔木。这录像带是从一个叫格尔木的地方寄出来的。

我愣了一下,心说那是什么地方?随即“google”了一下,就更吃惊了,那竟然是一个西部城市,位于青海省。

青海?闷油瓶什么时候去了那里?我疑惑起来,这家伙动作也够快的,一下子就跑到大西部去了,难道去支援西部的盗斗事业了?不过青海不属于土夫子的范围了,那地方是少数民族的聚居地,只有倒卖干尸的和国际文物走私犯才去那儿。他能去干吗,去帮人打井吗?

而且还寄了录像带给我,这好像八竿子也打不到一块儿。

我查了格尔木的一些资料,了解了一下它的历史,就更加的惊奇,发现格尔木是一个新城市,解放军修路修出来的城市,四周全是戈壁。闷油瓶在那里,我真的想不出他能干什么,而且他还从那里寄回来录像带,到底是什么内容呢?

妈的,我有点烦躁起来,一下子我对那录像带的兴趣就更浓烈了。

喝了几口浓茶,压了压酒之后,我把今天听到的信息汇总了一下,发给了几个阿宁那边的人。我和这些人混得熟,希望他们也帮我看看,也许能得到什么有用的反馈。虽然三叔让我不能对别人说,但是我想说给裘德考的人听,总问题不大,而且其中比较敏感但是不重要的内容,我都删除掉了。我还问了他们,是否最近公司有计划再次进云顶天宫。

做完这些事,酒精就开始发挥作用了,我很快就软倒,眼前模糊地睡着了。这一觉睡得格外的安心,也没有做梦,一直睡到大天亮,我被电话吵醒。

我接了电话,是三叔的伙计打来的,他说他们已经出院了,三叔已经在我隔壁套房了,录像机也已经买到了,让我过去一起看。

\chapter{画面}

录像机是那个伙计从船营区的旧货市场淘来的一松下,我到三叔房里的时候,那伙计正在安装,我看到沙发上还摆着两只一模一样的备用,是怕万一中途坏掉耽误时间。不过幸好,那个年代的进口货,质量还不错,三只测试了都能用,我掂量了一下备用的一只,死沉死沉的,那年代的东西就是实在,不像现在的DVD,抡起来能当狗叼飞碟玩儿。

安装录像机的这段时间里,三叔一直都没有开口,就让我坐着,自己一支接一支地抽烟,心里不知道在琢磨些什么。

我宿醉的头疼也逐渐好转,人也有点紧张,不时有乱七八糟的猜测,猜测这带子里到底录的是些什么画面。我想到过西沙,但是他们去西沙时候,不可能带录像设备(那个时候这种设备相当珍贵,国内还是普遍用胶片摄像机,那胶片还是手动的),所以录像带里的内容肯定不是西沙那时候拍摄的东西。同样,也不可能是青铜门后的内容。排除了这两个地方,录像带中会有什么?真的是毫无头绪。

电视机和录像机接好,电源被打开,我就挑出了其中一盒,打算放进去,不过放到录像机的口子之前,我又犹豫了,心里不知道为什么慌了一下,看了一眼三叔。

三叔对我摆摆手,道:“放进去啊?看我干什么,你他娘的还怕他从电视里爬出来?”

我这才推了进去,录像机“咯嗒”开始运转,我坐回到床上,很快,屏幕上闪出了雪花。三叔停止了抽烟,把烟头扔进痰盂里,我们两个加上他的伙计都有些紧张地坐了坐正。

雪花闪了十几秒,电视上才开始出现画面,电视机是彩色的,但是画面是黑白的,应该是录像带本身的问题,画面一开始很模糊,后来逐渐清晰起来。

那是一间老式的木结构的房间,我们看到了木制的地板,镜头在不停地晃动,显然放置摄像机的人或者物体并不是太稳定,我们看到一扇窗户开在后面的墙上,外面很模糊,似乎是白天,有点逆光。

三叔和我面面相觑,这好像是民居的画面,真是没想到会看到这个。难道会是自拍秀?等一下闷油瓶一边吃面一边出来,对着镜头说好久不见,你们过得如何云云。

在窗户下面,有一张相当老式的写字桌,看着有点像革命电影里的老家具了,上面堆满了东西,文件、台灯,还有一部电话。

电话的款式比较老旧,但不是老到掉牙的那种,这段录像拍摄的时间,应该是在20世纪90年代以后,当然现在仍旧有很多的家庭还是使用这种老样式的电话,所以到底是什么时候也不好判断,只是肯定不会比90年代更早。

接着画面就一直保持着这房间里的情景,就好像静物描写一样,我们等了一段时间,就意识到摄像机是固定在一个位置拍摄的,类似于电影中的固定镜头,并不会移动。

这样的话,这静止的画面就不知道会持续多久,我们也不能傻看着,三叔就按了快进。进过去大概二十分钟的时候,一下子,一个黑色的影子从房间里闪了过去。

我和三叔都吓了一跳。

三叔赶紧回倒慢放,原来是一个人从镜头外走进了镜头,我们还听到有开门关门的声音,应该是有人从屋外回来。仔细一看,走进来的那人,是个女人,年纪看不清楚,模糊地看看,长得倒有几分姿色,扎着个马尾。

三叔一下子紧张起来,他走上前去,几乎贴到电视屏幕上了。

可是那女的走得飞快,一下子就从屏幕穿了过去,跑到了另外一边,消失在屏幕外了。

我看三叔的脸色突然不对,想问他怎么回事,他却朝我摆了摆手,让我别说话。

时间继续推进,五分钟后,那女的又出现在了屏幕上,已经换了睡衣,接着她径直走到屏幕面前,屏幕开始晃动,显然在调整摄像机的角度。

这样一来相当于一个特写,那女人的面目就直接贴近了电视机,我看到那女人相当年轻,长相很乖巧,眼睛很大,总体看上去有点甜的那种女孩子。

三叔也正贴近电视,一下子就和电视里的那女孩子对上眼了,我没想到的是,一瞬间,三叔先是愣了一下,然后突然浑身一抖,一声大叫就后退了十几步,几乎把电视机从柜子上踢下来。

他的伙计赶紧扶住电视,我去扶他,只见三叔指着电视里那张脸,发着抖大叫:“是她!霍玲!是霍玲!”

我们给三叔这突如其来的反应吓得够戗,他的伙计赶紧丢下电视去扶他,我则先摆正电视机,唯恐摔下来坏掉。

然而他的伙计根本扶不住他,三叔一边叫一边直往后退,一下就撞到沙发上,撞得整个沙发都差点翻了,自己一滑就摔倒在地。这一下显然撞得极疼,他捂住自己的后腰,脸都白了。虽然如此,他的眼睛却还是牢牢地看着电视屏,眼珠几乎要瞪出来。

这下我也有点惊讶。这个女人竟然是霍玲?

按照闷油瓶的叙述,霍玲是一个干部子女,当年西沙考古的时候,同时下到海底墓穴中几个人的其中一个。关于她的资料极少,我不知道她在那张黑白合照中是哪一个,自然也认不出来。这样一个人,竟然会出现在闷油瓶子寄来的录像带中……真有点不可思议……

而且,让我感觉到异样的是,这录像带是怎么来的?从她调整镜头来看,显然她知道录像机的存在,自拍也不是这样拍的,这应该是一种自发的监视,这无疑是监控录像。她为什么要拍这样的录像,而这带子又是怎么到闷油瓶的手上的?闷油瓶又为什么把这带子寄给我呢?

这里面有戏了,我心里嘀咕起来,三叔说得对,看来整件事情还远远没有完。

此时屏幕上那女人已经调整好了摄像机,屏幕已经不抖了,她也重新远离镜头,坐到了写字台边上,支起一面镜子梳头,因为是黑白的画面,加上刚才的晃动,屏幕上变得有点模糊。

三叔逐渐冷静了下来,但是脸色已经铁青,神情和刚才已经判若两人。他手死抓着沙发的扶手,浑身轻微地发抖,显然十分的紧张。

我为了确定,就问三叔道:“这女的就是你们一起下到海底里去的那个霍玲?”

三叔一点反应也没有。我没有办法,和他的伙计对看了一眼,他伙计也不知道怎么说。

录像中的霍玲不停地梳头,她的马尾解开了后,头发颇长,我都不知道她到底要梳到什么程度,大概有二十分钟,她才停下手来,重新扎起马尾。

梳完头后,她站起来,有点迷茫地看了看窗外,然后突突突跑到了摄像机照不到的地方,接着又跑了回来,可是等她跑回来,我发现她的衣服竟然变了。

也就是说,她到了里屋,换了一身衣服。

接着,让我感觉到匪夷所思的画面就出现了。

她出来之后,又跑到了摄像机前,似乎是不满意角度,又调整了镜头,屏幕开始晃动,她那白色的脸充斥着整个屏幕。

三叔发出了一声很古怪的呻吟,似乎她的脸十分可怕。

我以为她换衣服是要出去,或者做饭之类的,屋里肯定又会很长时间看不到人,于是拿起遥控器,准备快进,这时候,却看见她却又坐回到了写字台边上,拿起梳子,解开头绳,又开始梳头!

“这女的有神经病!”一边的伙计忍不住叫了起来。

三叔马上做了个手势让他别出声,眉头紧紧地皱了起来。

她是背对着我们梳头,也看不到她的表情,镜子中只有一个模糊的影子,动作也几乎一致,频率都似乎一样,我看着看着,简直怀疑她的头是铁头,要是我给这么梳,脑袋早就梳成核桃了。

这样的画面使我感觉气氛变得有点诡异,我忍耐着,又是大概二十分钟的时间,她才重新扎起头绳,站了起来,噔噔噔跑到镜头外面去了。

我和那伙计都松了口气,心说总算完了,要再梳下去,我的头也要开始疼起来了。

然而没等我们舒展筋骨,她又换了一身衣服跑了出来,凑到摄像机面前,第三次开始调试角度了。

我一下就迷糊了,简直丈二和尚摸不着头脑,这个霍玲究竟是干什么的,这也太夸张了,难道她爱好这个……或者,难道她要自杀了?所以不厌其烦地换衣服调角度,接着难道她又要去梳头了?再这样梳下去,梳子都要磨成毛刷了。

就在这时候,突然画面一停,回头一看,原来三叔按了暂停,黑白的屏幕上,顿时定格了那张特写的面孔。

三叔脸色铁青,嘴唇还有点发抖,他凑近仔细看了看,哑声道:“天,她也没有老!”

\chapter{第十一个人}

三叔说的,我也早已经观察到了,只是没有说出来,一方面录像带并不清晰,我不知道自己有没有看错,另一方面,我相信他很快就会意识到。

果不其然,三叔暂停了画面凑过去看,我也凑了过去,想看个仔细,确定一下。

看了几眼,我就断定,毋庸置疑,霍玲在拍摄带子时候的年纪,不会超过三十岁,倒不是说她长得年轻,而是那种少女的体态,不是装嫩的女人能够装出来的,而且,我不得不说这霍玲实在长得很乖巧,难怪迷得考古队里的几个男的神魂颠倒。黑白屏幕的表现力比彩色的要差很多,但是她那种有点迷茫的眼神和精致的五官,还是能给人怦然心动的感觉。这样的相貌,想来必定是十分的自信,自幼在众星捧月中长大,遇到闷油瓶这样的闷王不理睬她,她的反应倒也合乎逻辑。不过现在看来,这些反应也可能是装出来的,如果真是那样,这个女人想必也是厉害角色。

三叔的脸色很难看,窝进沙发里啧了一声:“一个是这样,两个也是这样,他娘的,难道失踪的这帮人全部都会这样?他们之后到底遇到了什么事情?”

我想了想就摇头,对三叔说也不能这么武断,这里我们并不知道录像拍摄的具体时间,看电话的款式也许是20年纪90年代前后,那离她在海底墓穴失踪也没有多少时间,我们不知道霍玲当时几岁,如果她当时只有十七八岁,那就算过了十年也只有二十七八,不能断定说她没有变老。

三叔沉吟了一声,显然没有太在意我的话,而是将录像继续放了下去,我们继续往下看。

然而,让我们想不到的是,继续放了才没几分钟,突然画面上就跳起了雪花。

我们以为是带子的问题,等了一会儿,可是雪花继续,三叔快进过去,一直到底,全部都是雪花。

“怎么回事?”三叔有点愠怒,他不擅长和电器相处,以为机器坏了,就想去拍。

我阻止住他,将带子拿出,扯出来看了看,发现带子没有任何的霉变,就知道了怎么回事:“被洗掉了。”

从刚才画面的连续性来看,后面应该是有内容的,如今突然间变雪花,显然是被洗掉了。

带子拿来一直就没人动过,录像机也刚刚买来,不可能是误操作,那带子应该是在寄出来之前就被洗掉的,然而如果是故意的话,为什么不把前面的也洗掉,非要留下那么匪夷所思的一段?难道后面的内容我们不能看吗?

我和三叔面面相觑,都完全摸不着头脑了,闷油瓶是什么意思?难道是耍我们?这也不太可能啊,这小哥不像是那么无聊的人啊。

三叔想了想,又让我把带子放了进去,倒回去重新看,想仔细看看是否其中有刚才没有发现的东西。因为前面有一段是快进的,不仔细看看终归有点心虚。

这一次我们是实打实一秒一秒地看了下来,房间里鸦雀无声,如果眼神有力量的话,那电视机可能会给我们瞪爆了。然而,一路看下来,眼睛都瞪得血红,仍旧没有发现任何能够让我们产生兴趣的线索。

之后我们又播放了另一盘录像带,然而,这一次更离谱,那完全就是一盘空白的带子,里面的东西全部是雪花。我们来回看了两次雪花,只觉得人都晕了起来。

刚开始看带子的时候十分兴奋,看完之后却是万般的沮丧以及迷惑。我刚开始甚至以为可以看到青铜门里的情形了,然而,没有想到的是,里面竟然是这么莫名其妙的画面。

关掉机器,我和三叔就琢磨这究竟是怎么回事。然而两个人想了半天,发现这事情完全没有入手的地方。

我告诉三叔昨天我查到的信息,这带子是来自青海的格尔木,那么,可以这么认为,闷油瓶在青海给我们寄出了这一份包裹。那么,他现在人一定是在格尔木这个城市里。那是否可以认为,这两盘带子是他在格尔木找到的?然后,寄给了我们。

这也完全无法肯定,不过,从这个带子里,倒是能知道一个问题,就是,那批人在海底墓穴中失踪,显然并不是死亡了,他们在20世纪90年代还活着,但是,行为有一些反常。这批人中的大多数应该死在了云顶天宫里,我这个没和三叔说,怕他崩溃,因为里面可能会有文锦。

之后又逼着自己看了几遍,实在是看不出问题来,三叔还要继续看录像带,我就先回去补回笼觉了。后来三叔将带子翻录了一盘,将母带还给了我,说自己去研究之后几天,潘子听说三叔醒了过来,就到了吉林,将他接走。

这一次三叔的生意损失巨大,伙计抓的抓,逃的逃,三叔在长沙的地位也一落千丈,不过三叔自己并不在乎,对于他来说,钱这种东西也只是个符号而已。临走三叔对我说,这事情如果还有下文,让我也不要去管了,我之前完全是命大,而且身边有贵人在保我,事不过三,老天不会照顾我这么久,好好做好自己的铺子是真,以后他的那些产业,说不定还要我去打理。

我表面点头,心说得了吧,你那种生活我恐怕无命去消受,还是干我的老本行比较实在。

说话休繁,三叔走了之后,我也预备着回杭州,只是也没在吉林好好待待,于是时间拖后了几日,联系了几个附近的朋友,一来是放松一下,二来是叙叙旧。

我有几个大学同学在长春,于是他们赶了过来,几个人到处走走,聊聊以前的事情,我的心情才逐渐地积极起来。后来又去周边的城市走了走,逛了逛古玩市场,帮他们挑点古董,一来二去,又是两个星期。

经历了这么多事情,我变得有点不拘小节,以前花钱还还个价儿,现在只觉得一手交钱一手交货的简单,不过这样着,身边的钱就日渐少了下去。

几个朋友都奇怪我的变化,铁公鸡也会拔毛,实在想不到,都问我受了什么刺激了。

一次吃饭的时候,我就挑着精彩的,和那几个人说了我经历的事情,也算是吹个牛,说完之后,竟然没一个信的,其中一人就笑道:“你说下到海底的那几人,是否就是你让我查的那张照片?”

我听得他说,这才想起来,以前我在网络上找到过一张照片,下面有“鱼在我这里”,当时我就是托这个人去帮我查过,后来只查出是在吉林发在网上的,后面就不了了之。

现在想来,倒也奇怪,网络这个东西真正发达起来,也就是这几年,到底是谁发的呢?

既然想起来了,我就问那人后来还有没有查到更多的东西。那人摇头,显然并未把我的事情放在心上,只是说道:“这样的照片太普通了,而且年代太过久远,那个年代的资料也一般不会上网,我只能通过技术手段,那个IP地址是唯一能查的东西。我感觉,你如果真的要查,不如去国家档案局,查查哪一支十一人的考古队伍在二十年前失踪了,可能会知道更多的东西。”

我沉吟了一声,这倒也有道理,一旁就有个人更正道:“你记错了,我也看过那照片,是十个人。”

那人摇头道:“不对,我感觉是十一个人。”

我心里一跳,问他道:“为什么?”

那人笑道:“照片里拍好的是十个人,但是,不是还有一个拍照片的人吗?你们难道没想到?”

\chapter{尾声}

说话的那个朋友,是我的学长,我和他也不是很熟悉,只是这一批人经常在一起玩,比较聊得来,属于君子之交的那种,互相有需要就帮帮忙,不是非要好到黏在一起的那种朋友。我当时找他帮忙,是因为他似乎是干技术工作的,当然我这个做古董的和他一点交集也没有,他具体是干什么的,我也不清楚。

如今他一语惊醒梦中人,听到这“十一个人”的理论,我当即就是一身的冷汗,连脸色都白了。

是啊,我他娘的怎么没有想到?

那个年代,没有傻瓜相机的,在海南的渔村也绝对不会有照相馆,能够使用相机的人,的确应该是考古队里的一员。我只稍微想了想,就发现他说得非常有道理,我看过很多西沙考古的资料,里面都有照片,一般这样的情况,都有宣传方面的人跟着记录。

可是为什么三叔的叙述中,却始终只提到十个人,从来没有提到过这第十一人,是否这个宣传的人没有跟他们出海,还是三叔另有隐瞒?

看我的样子,那几个人哄堂大笑,那人道:“算了,别想了,到底几个人,去他们老单位查查不就知道了,考古研究所一般隶属于文化系统,当时他们是哪个研究所派出去的,档案应该还在,我们国家很多的档案都是永久保存的。”

我也不言语,反正这也只是个推测,倘若有时间,倒是可以去查查。不过查来如果是十一人,我如何面对三叔的解释?是不是要全盘推翻他?这样的痛苦未免太大了点,想到这里,还是不去查算了。

◆ 第七卷 蛇沼鬼城(中) ◆

\chapter{稀客}

回到杭州之后,天气还是非常的寒冷。

铺子里一如既往地冷清,王盟看到我回来,一脸的疲惫,竟然没有在第一时间认出我来,以为我是顾客,我也只能苦笑。

我那些朋友和我讨论的结果,对我的打击非常大,搞得我心神不宁,又不能再次去问三叔,免得他老人家说我三心二意,心中的苦闷也没地方发泄,只得天天待在铺子里,和临铺的老板下棋,话说今年事情多,各铺的生意都不好,大家都吃老本,过着很悠闲的生活。

说来也奇怪,烦人的事情,到了杭州之后,想得也少了,大概是这个城市本身就非常的让人心宽。

我有很长一段时间没有见过三叔,胖子来找过我几次,托我处理东西。这小子也是闲不住的人,家财万贯,挥霍得也快,很快竟然又说没钱,一问才知道,在北京置了铺子,就花得七七八八了,这年头确实不像以前,有个万把块一辈子就不愁了。不过他好几次带着几个一嘴京腔儿的主顾来,倒也是匀了不少货,想必局面打开了,也是赚了不少。

这一天,我正给隔壁的老板杀得剩下一对马,还咬牙不认输准备坚持到晚饭赖掉,就听到有人一路骂着人过来,抬头一看,竟然又是胖子,这家伙生意也太好了。

隔壁老板和胖子做过生意,敲诈了他不少,看到胖子过来就开溜了,我一边庆幸不用输钱了,一边就问他发什么火。

胖子骂骂咧咧,原来带着两只瓷瓶过来杭州,半路在火车上碎了一只,又没法找人赔,只能生闷气。

我和他熟络了不少,也多少知道了点他的底细,就笑着奚落他,放着飞机不坐,挤什么火车,这不是脑子进水吗。

胖子骂道:“你懂个什么,现在上飞机严着呢,咱在潘家园也算是个人物,人家雷子都重点照顾。这几年北京国际盛会太多,现在几天一扫荡,老子有个铺子还照样天天来磨叽,生意没法做,这不,不得已,才南下发展,江南重商,钱放得住。不过你们杭州的女人太凶了,你胖爷我在火车上难得挑个话头解解闷儿,就给摔了嘴巴子,他娘的老子的货都给砸碎了,他娘的谁说江南女子是水做的,这不坑我吗,我看是镪水。”

这事儿胖子念叨很多次了,我知道是怎么回事,火车上一女孩子人长得瘦,胖子看那女的瘦不拉叽的,还化着浓妆,一边还嘴巴不是很干净地埋怨车里味道难闻。当然胖子的脚丫是太臭了,听着就窝火,也是太无聊了,嘴里就磕碜她,说大妹子,您看您长得太漂亮,怎么就这么瘦呢,您看您那两裤管儿,风吹裤裆吊灯笼,里面装两螺旋桨,他娘的放个屁都能风力发电了。

这不说完就给人扇了一个嘴巴。我听着就乐,对他说人家不拉你去派出所算不错了,你知道不,这世界上有一种叫做流氓罪,你已经涉嫌了。

胖子还咧嘴,说就那长相,哎呀,说我流氓她,雷子绝对不能信,我绝对是受害者。

我给他出了个主意,说以后你也不用亲自来,你不知道这世界上有种东西叫快递吗?你呢,自己投点儿小钱,开个快递公司,多多打点,这物流一跑起来,一站一站,一车上送几件明器还不是小菜一碟儿。

胖子经营方面脑子死,听不得复杂的东西,就不和我扯这个了,他欷嘘道:“说起赚钱,不是你胖爷我贱,这几个月我也真待得腻烦起来了,你说他娘的钱赚过来,就这么花多没意思,咱们这帮人,还得干那事儿,对吧,这才是人生的真谛。对了,你那三爷最近还夹不夹喇嘛,怎么没什么消息?”

我说我也没怎么联系,总觉得那件事情之后,和三叔之间有了隔阂,他不敢见我,我也不敢见他,偶然见一次也没什么话说。

胖子也不在意,只道:“要还有好玩的事儿,匀我一个,这几个月骨头都痒了。”

我心道你说来说去,不还是为了钱嘛,心中好笑,说:“你这胖子秉性还真是怪,要说大钱你也见过,怎么就这么不知足呢。”他道:“一山还有一山高,潘家园豪客海了去了,一个个隐形富豪,好东西都在家里压着砖头呢,这人比人气死人啊,都说人活一口气,有钱了这不想着更有钱嘛!”

我哈哈大笑,说这是大实话。

正说着,打铺子外突然探头进来一个人,抬脸就笑,问道:“老板,做不做生意——”

胖子正挖脚丫子呢,抬眼看了看来人,哎呀了一声,冷笑道:“是你?”

我回头一看,来人竟然是阿宁,如今身着一件露脐的T恤,穿着牛仔裤,感觉和海上大不相同,我倒有点认不出来了。

阿宁和我几乎没有联系过,我也算是打听过这人的事情,不过没有消息,如今她突然来找我,让我感觉到非常意外。

阿宁没理会胖子,瞪了他一眼,然后风情万种地在我的铺子里转了一圈儿,对我道:“不错嘛,布置得挺古色古香的。”

我心道我是古董店,难道用超现实的装修吗?戒备道:“你真是稀客了,找我什么事情?”

她略有失望地看了我一眼,大概是感觉到了我的态度,顿了顿道:“你还真是直接,那我也不客气了,我来找你请我吃饭,你请不请?”

\chapter{新的线索}

杭州楼外楼里,我看着阿宁吃完最后一块醋鱼,心满意足地抹了抹小嘴,露出一个很陶醉的表情,对我们道:“杭州的东西真不错,就是甜了点儿。”

我心中的不耐烦已经到了极点,但是又不好发作,只得咧了咧嘴,算是笑了笑,就挥手埋单。

说实话,作为一个相识,请她吃一顿饭也不是什么太过分的事,我也不是没有和陌生人吃过饭的那种人,但是一顿饭如涓涓细流,吃了两个小时,且一句话也不说,一边吃一边看着我们只是笑,真的让我无法忍受。

同样郁闷的还有胖子,胖子对她的意见很大,原本是打算拍拍屁股就走的,但我实在不愿意和这个女人单独吃饭,所以我死拖着他进了酒店,现在他肠子都悔青了。

我们两个人也没吃多少口,胖子就一直在那里喝闷酒,两个人都紧绷着脸。我心里琢磨她到底来找我干什么,一边想着应对的方法,甚至都想到了怎么提防那女人突然跳起来扔袖箭过来。

服务员过来结了账,看着我们的眼神也是纳闷和警惕的。

两个小时没有对话,脸色铁青,闷头吃喝的客人在“楼外楼”实在是少见,从她的眼神看,她可能以为我们是高利贷聚会,这个好身材的女人吃完就要被我和胖子卖到妓院去了。

而我自己感觉,却是考试没复习的学生突然发现老师家访,也不知道是福是祸,等着老师进入正题的那种忐忑不安的感觉。总之,这是我一辈子吃的最郁闷的一顿饭。

服务员走远之后,胖子看着桌子上的菜,冷笑了一声:“看不出你吃饭也是狠角色,怎么?你为你们公司这么拼命,你们公司连个饱饭也不给你们吃?”

“我们一年到头都在野外,带着金条也吃不到好东西。”阿宁扬起眉毛,“和压缩饼干比起来,什么吃的都是好东西。”

胖子冷笑了一声,朝我看了看,使了个眼色,让我接他的话头。

我咳了一声,也不知道怎么说,不过阿宁显然是来找我的,让胖子来帮我问,肯定是不合适,于是硬着头皮问阿宁道:“我已经请你吃过饭了,我们有话直接说吧,你这次来找我,到底有什么事?”

阿宁翘起嘴角:“干吗老问这个,没事情就不能来找你?”

这一翘之下,倒也是风情万种,我感觉她看我的眼睛里都要流出水来了,胸口马上堵了一下,感觉要吐血,下意识地就去看胖子。胖子却假装没听见,把脸转向一边。

我只好把头又转回来,也不知道怎么接下去问,“嗯”了一声,半天说不出话来,一下子脸都憋红了。

阿宁看着我这个样子,一开始还很挑衅地想看我如何应付,结果等了半天我竟然不说话,她突然就笑了出来,好笑地摇头说道:“真拿你这个人没办法,也不知道你这样子是不是装的,算了,不耍你了,我找你确实有事。”

说着她从自己的包里掏出一包四四方方的东西,递给我:“这是我们公司刚收到的,和你有关系,你看看。”

我看了一下,是一份包裹,我一掂量,心里就咯噔了一声,大概知道了那是什么东西。这样的大小,这样的形状,加上前几天的经历,实在是不难猜,于是我不由自主地,冷汗就冒了出来。

胖子不明就里,见我呆了一下,就抢过去,展开一看,果然是两盘黑色录像带,而且和我们在吉林收到的那两盘一样,也是老旧的制式。

我虽然猜到,但是一确认,心里还是吊了起来,心说怎么回事,难道闷油瓶不止寄了两盘?寄给我们的同时,还有另一份寄到阿宁的公司?那这两盘带子,是否和我收到的两盘内容相同?

这小子到底想干什么?

“这是前几天寄到我们公司上海总部的,因为发件人比较特殊,所以很快就到了我的手上。”阿宁看着我,“我看了之后,就知道必须来找你一趟。”

胖子听我说过录像带的事情,如今脸上已经藏不住秘密了,直向我打眼色。我又咳了一声,让他别这么激动,对阿宁道:“发件人有什么特别的?带子里是什么内容?”

阿宁看了一眼胖子,又似笑非笑转向我,道:“发件人的确非常特别,这份快递的寄件人——”她从包里掏出了一张快递的面单,“你自己看看是谁。”

我看她说得神秘兮兮的,心说发件人应该是张起灵啊,这个人的确十分特殊,我现在都感觉这个人到底是不是存在于这个世界上的,但是阿宁又怎么知道他特殊呢?

于是我接过来,胖子又探头过来,一看,我却愣住了,面单上写的,寄出这份快递的人的名字,竟然是——吴邪——我的名字。

“你?”一边的胖子莫名其妙地叫了起来。

我马上摇头,对阿宁说:“我没有寄过!这不是我寄的。”

阿宁点头:“我们也知道,你怎么可能给我们寄东西。寄东西的人写这个名字,显然是为了确保东西到我的手里。”

胖子的兴趣已经被勾了起来,问阿宁道:“里面拍的是啥?”

阿宁道:“里面的东西相当古怪,我想,你们应该看一下,自己去感觉。”

我心里的疑惑已经非常厉害,此时也忘记了防备,脱口就问阿宁道:“是不是一个女人一直在梳头?”

阿宁显然有点莫名其妙,看了一眼我,摇头道:“不是,里面的东西,不知道算不算是人。”

\chapter{录像带里的老宅}

在吉林买的几台录像机,我寄了回来,就放在家里,不想阿宁知道我实际的住址——虽然她可能早已经知道——所以差遣了王盟去我家取了过来,在铺子的内堂接驳好,我们就在那小电视上,播放那盘新的带子。

带子一如既往是黑白的,雪花过后,出现了一间老式房屋的内堂。我刚开始心里还震了一下,随即发现,那房子的布置,已经不是我们在吉林看的那一盘里的样子,显然是换了个地方,空间大了很多,摆设也不同了,不知道又是哪里。

当时在吉林的时候,和三叔看完了那两盘带子,后面全是雪花,看了很多遍也没有发现任何的蛛丝马迹,此时有新的带子,心想也许里面会有线索,倒是可以谨慎点再看一遍。

王盟给几个人都泡了茶,胖子不客气地就躺到我的躺椅上,我只好坐到一边,然后打发王盟到外面去看铺子,一边拘谨地尽量和一旁的阿宁保持距离。不过此时阿宁也严肃了起来,面无表情,和刚才的俏皮完全就是两个人。

内堂中很暗,一边有斑驳的光照进来,看着透光的样子,有点像明清时候老宅用的那种木头花窗,但是黑白的也看不清楚,可以看到,此时的内堂中并没有人。

胖子向我打眼色,问我和闷油瓶给我的录像带里的内容是否一样。我略微摇了摇头表示不是,他就露出了很意外的表情,转头仔细看起来。

不过,后面大概有十五分钟的时间,画面一直没有改变,只是偶尔抖一个雪花,让我们心里跳一下。

我有过经验,还算能忍,胖子就沉不住气了,转向阿宁:“我说宁小姐,您拿错带子了吧?”

阿宁不理他,只是看了看我。我却屏着呼吸,因为我知道这一盘应该同样也是监视的带子,有着空无一人内堂的画面是十分正常,阿宁既然要放这盘带子,必然在一段时间后,会有不寻常的事件发生。

见我和阿宁不说话,胖子也讨了个没趣,喝了一口茶,就想出去,我按了他一下,让他别走开,他才坐下,东挠挠西抓抓,显得极度的不耐烦。

我心中有点暗火,也不好发作,只好凝神静气,继续往下看,看着上面的内堂,自己也有点不耐烦起来,真想用快进往前进一点儿。

就在这个时候,阿宁突然正了正身子,做了一手势,我和胖子马上也坐直了身子,仔细去看屏幕。

屏幕上,内堂之中出现了一个灰色的影子,正从黑暗中挪出来,动作非常奇怪,走得也非常慢,好像喝醉了一样。

我咽了口唾沫,心里有几个猜测,但是不知道对不对,此时也紧张起来。

很快,那白色的影子明显了起来,等他挪到了窗边上,才知道为什么这人的动作如此奇怪,因为他根本不是在走路,而是在地上爬。

这个人不知道是男是女,只知道他蓬头垢面,身上穿着犹如殓服一样的衣服,缓慢地、艰难地在地上爬动。

让我感觉到奇怪的是,看他爬动的姿势,十分的古怪,要不就是这个人有残疾,要不就是这个人受过极度的虐待。我就看到一个新闻,有些偏远农村里,有村汉把精神出了问题的老婆关在地窖里,等那老婆放出来的时候,已经无法走路了,只能蹲着走,这个人的动作给我的就是这种感觉。

我们都不出声,看着他爬过了屏幕,无声息地消失在了另一边。接着,我们面前又恢复了一个静止的、安静的内堂。

整个过程有七分钟多一点,让人比较抓狂的是,没有声音,看着一个这样的人无声息地爬过去,非常的不舒服。

阿宁按着遥控器,把带子又倒了过去,然后重新放了一遍,接着定格住,对我们道:“后面的不用看了,问题就在这里。”

“到底是什么意思?”胖子摸不着头脑,问我道,“天真无邪同志,这人是谁?”

“我怎么知道!”我郁闷道,原本以为会看到霍玲再次出现,没想到竟然不是,这就更加让我疑惑了,看着那伛偻的样子,如果确实是同一个人寄出的东西,那录像带应该还是霍玲录的,难道,霍玲到了这一盘录像带里,已经老得连站也站不起来了?

胖子又去问阿宁,到底是怎么回事,这拍的是什么东西?

“你们感觉你们自己看到了什么?”阿宁问我们道。

“这还用问,这不就是个人,在一幢房子的地板上爬过去?”胖子道。

阿宁不理他,很有深意地看着我,问道:“你说呢?”似乎想从我身上看出什么东西来。

我看着阿宁的表情,奇怪道:“难道不是?”

她有点疑惑又有点意外地眯起了眼睛:“你……就没有其他什么特别的感觉?”

我莫名其妙,看了眼胖子,胖子则盯着那录像带,在那里发出“嗯嗯”的声音,摇头:“没有。”

阿宁盯着我好久,才叹了口气,道:“那好吧,那我们看第二卷,我希望你能做好心理准备。”

说着第二卷带子也放了进去,这一次阿宁没有让我们从头开始看,而是开始快进带子,直到进到十五分钟的时候,她看向我,道:“你……最好深呼吸一下。”

我给她说得还真的有点慌了,胖子则不耐烦,道:“小看人是不?你也不去打听打听,咱们小吴同志也算是场面上跑过的,上过雪山下过怒海,我就不信还有啥东西能吓到他,你别在这里煽动你们小女人情绪,小吴你倒是说句话,是不是这个理儿?”

我不去理他,让阿宁就开始吧,在自己铺子的内室里,我也不信我能害怕到哪里去。

阿宁瞪了胖子一眼,录像又开始播放,场景还是那个内堂,不过摄像机的镜头好像有点儿震动,似乎有人在调节它。震动了有两分钟,镜头才扶正,接着,一张脸从镜头的下面探了上来。

刚开始对焦不好,靠得太近看不清楚,但是我已经看出那人不是霍玲。接着,那人的脸就往后移了移,一个穿着灰色殓衣一样的人出现在镜头里,他发着抖坐在地上,头发蓬乱,但是几个转动之下我还是看到了他的脸。

与此同时,胖子就惊讶地大叫了一声,猛地转头看我,而我也顿时感觉到一股寒意从我的背脊直上到脑门,同时张大了嘴巴,几乎要窒息。

屏幕上,那转头四处看,犹如疯子一样的人的脸非常熟悉,我足花了几秒才认出来——那竟然是我自己!

\chapter{完全混乱}

我们三个人安静了足足有十几分钟,一片寂静,其间胖子还一直看着我,但是谁也没说话。

电视的画面给阿宁暂停了,黑白画面上,定格的是那张熟悉到了极点的脸,蓬头垢面之下,那张我每天都会见到的脸——我自己的脸,第一次让我感觉如此的恐怖和诡异,以至于我看都不敢看。

良久,阿宁才出了声音,她轻声道:“这就是我为什么一定要来找你的原因。”

我不说话,也不知道怎么说,脑子一片空白,根本不知道如何反应。

胖子张了张嘴巴,发出了几声无法言语的声音,话才吐了出来:“小吴,这个人是你吗?”

我摇头,感觉到了一阵一阵的晕眩,脑子根本无法思考,用力捏了捏鼻子,对他们摆手,让他们都别问我,让我先冷静一下。

他们果然都不说话,我真的深呼吸了几口,努力让心里平静下来,才问阿宁道:“是从哪里寄过来的?”

“从记录上看,应该是从青海的格尔木寄出来的。”

我深吸了一口气,果然是从同一个地方发出的,看带子的年代,和拍霍玲的那两盘也是一样,不会离现在很近。那这两盘和我收到的两盘,应该有着什么关系。可以排除不会是单独的两件事情。

但我脑子里绝对没有穿过那样的衣服,在一座古宅里爬行的经历,这实在太不可思议,我心里很难相信屏幕上的人就是我。我一时间就感觉这是个阴谋。

“除了这个,还有没有其他什么线索?”我又问她。她摇头,“唯一的线索就是你,所以我才来找你。”

我拿起遥控器,倒了回去,又看了一遍过程,遥控器被我捏得都发出了“啪啪”的声音。看到那一瞬间特写的时候,我虽然有了心理准备,但是心里还是猛地沉了一下。

黑白的屏幕虽然模糊不清,但是里面的人,绝对是我不会错。

胖子还想问,给阿宁制止了,她走出去对王盟说了句什么,后者应了一声,不久就拿了瓶酒回来,阿宁把我的茶水倒了,给我倒了一杯酒。

我感激地苦笑了一下,接过来,大口喝了一口,辛辣的味道充入气管,马上就咳嗽起来,一边的胖子轻声对我道:“你先冷静点儿,别急,这事儿也不难解释,你先确定,这人真的不是你吗?”

我摇头:“这人肯定不是我。”

“那你有没有什么兄弟,和你长得很像?”胖子咧嘴问我道,“你老爹别在外面会不会有那个啥——”

我自己都感觉到好笑,这不是某些武侠小说中的情节吗?怎么可能会发生在现实中,苦笑摇头,又大口喝了一口。

阿宁看着我,又看了很久,才对我道:“如果不是你,你能解释这是怎么回事吗?”

我心道你问我我问谁去,心里已经混乱得不想回答她了,事情已经完全脱离了我能理解的范围,我一时间无法理性地思考。最主要的是,我摸不着头脑的同时,心里同时有一种奇怪的感觉,但是我又抓不住这种感觉的任何线头。这又让我非常抓狂。

一边的胖子又道:“既然都不是,那这个人只可能是带着你样貌的面具……看来难得有人非常满意你的长相,你应该感到欣慰了,你想会不会有人拍了这个带子来耍你玩儿?”

我暗骂了一声,人皮面具,这倒是一个很好的解释,但是所谓人皮面具,要伪装成另外一个人容易,但是要伪装成一个特定的人,就相当难,可以说几乎是不可能的。如果有人要做一张我相貌的人皮面具,必须非常熟悉我脸部的结构才行,而且了解我的各种表情,否则就算做出面具来,只要佩戴者一笑或者一张嘴巴,马上就会露馅。

这录像带里的画面,肯定隐藏着什么东西。就算真的是有人带着我相貌的面具,也会出现大量的问题:比如这个人到底是谁呢?他从哪里知道了我的相貌?他用我的“脸”又做过什么事情呢?怎么会出现在录像中?录像中的地方是哪里?又是什么时候拍摄的?和霍玲的录像带又有什么联系呢?

事情不是那么简单的。

我甚至有错觉,心说又或者这个人不是戴着人皮面具的,我才是戴着人皮面具的?

我摸了摸自己的脸,竟然想看看自己是不是吴邪,然而捏上去生疼,显然我脸是真的,自己也失笑。

霍玲的录像带,以及有“我”的录像带,以张起灵的名义和吴邪的名义分别寄到了我和阿宁的手里,这样的行为,总得有什么意义。一切的匪夷所思,一下子又笼罩了过来,那种我终于摆脱掉的,对于三叔谎言背后真相的执念,又突然在我心里蹦了出来。

晚上,还是楼外楼,我请胖子吃饭,还是中午的桌子。

整个下午我一直沉默,阿宁后来等不下去了,就留了一个电话和地址,回自己的宾馆去了。让我如果有什么想法,通知她,她明天再过来。

我估计就一个晚上,我也不会有什么想法,也只是应付了几声,就把她打发走了。胖子本来打算今天晚上回去,但是出了这个事情,他也有兴趣,准备再待几天,看看事情的发展。他住的地方是我安排的,而且中午没怎么吃饭,就留下来继续吃我的贱饭。

那服务员看着我和胖子又来了,但是那女人不在,可能真以为被我们卖掉了,一直的脸色就是怪怪的。要是平时我肯定要开她的玩笑,可是现在实在是没心情。

当时阿宁刚走,胖子就问我道:“小吴,那娘儿们不在了,到底怎么回事,你可以说了吧?”

我朝他也是苦笑,说我的确是不知道,并不是因为阿宁在所以装糊涂。

胖子是一脸的不相信,在他看来,我三叔是大大的不老实,我至少也是只小狐狸,那录像带里的人肯定就是我,我肯定有什么苦衷不能说。

我实在不想解释,随口发了毒誓,他才勉强半信半疑。此时酒菜上来,胖子喝了口酒,就又问我道:“我说小吴,我看这事儿不简单,你一个下午没说话,到底想到啥没有?你可不许瞒着胖爷。”

我摇头,皱起眉头对他道:“想是真没想到什么,这事儿我怎么可能想得明白,我就连从哪里开始想,我他娘的都不知道,现在唯一能想的,就是这带子到底是谁寄的。”

下午我想了很久,让我很在意的是,第一,从带子上的内容来看,“我”与霍玲一样,也知道那摄像机的存在,显然,“我”并不抗拒那东西。

第二,霍玲的那盘带子,拍摄的时间显然很早,20世纪90年代的时候应该就拍了,如果两盘带子拍摄于同一年代,那阿宁带子里的“我”也应该是生活在90年代。而那个时候,我清清楚楚地记得,我还在读中学,不要说没有拍片子的记忆了,就算样貌也是很不相同的。我是个阴谋论者,但如果我的童年也有假的话,我家里从小到大的照片怎么解释呢?我的那些同学、朋友,又怎么解释呢?

现在看来,我最想不通的,是谁寄出了这个带子给阿宁的,他的目的是什么。难道他只是想吓我一跳?实在是不太可能。

胖子拍了拍我,算是安慰,又自言自语道:“冒充你寄东西给阿宁的,会不会也是那小哥?”

我叹了口气,心说这谁也不知道,想起阿宁对包裹署名的解释,心里又有疑问,如果阿宁的包裹是用化名寄出的话,会否我手上的这两盘带子也是用的化名?使用张起灵的署名,也是为了带子能到达我的手上?寄出带子的,不是他而另有其人?

毕竟我感觉他实在没理由会寄这种东西过来。录像带和他实在格格不入啊。

不过不是他又会是谁呢?内容和西沙那批人有关,难道是西沙的那批人中的一个?他们的目的是什么呢?

我问胖子道:“对了,胖子你脑子和别人不一样,你帮我思考一下,这事情可能是怎么回事,就靠你的直觉。”

“直觉?”胖子挠了挠头,“你这他妈不是难为胖爷我吗?胖爷我一向连错觉都没有,还会有什么直觉。”

我心说也是,要胖子想这个的确有点不靠谱,毕竟他和闷油瓶不太熟,对西沙的事情也不了解,至少没有我熟悉。

说起闷油瓶,那我又算不算了解这个人呢?我喝了口酒一边就琢磨。

闷油瓶给我整体的感觉,就是这个人不像是个人,他更像是一个很简单的符号。在我的脑海里,除了他救我的那几次,似乎其他的时候,我看到的他都是在睡觉。甚至,我都没有一丝一毫的线索,去推断他的性格。

如果是普通人,总是可以从他说话的腔调,或者一些小动作来判断出此人的品性,但是偏偏他的话又少得可怜,也没有什么小动作,简直就是一个一点多余的事情都不做的人,只要他有动作,就必然有事情发生,这也是为什么好几次他的脸色一变,所有人头上就开始冒汗的原因。

想了想,我又对胖子道:“那就不用直觉,你就说说,你对这事情有什么感觉,有什么不对劲的地方?哪怕一点也好,给点支持。”

胖子就叹了口气,对我道:“他娘的,你真给我们无产阶级丢脸,我感觉是没有,不过,不对劲的地方倒是真有一个,你刚才说的时候,我注意到有个细节,不知道你注意过没有?”

“什么细节?”我问他道。

“你不是说,那小哥寄给你的录像带,有两盘吗?其中一盘有那个女人在梳头,另一盘是空白的,什么都没有。”

我点头,确实是这样。

胖子就道:“这他娘的就不对了,要是空白的,他寄给你干什么?这不是没有道理吗?他干吗不直接寄第一盘得了,何必要凑齐两盘?”

我叹了口气,当初我也考虑过这个问题,但是因为整件事情非常的匪夷所思,所以这些小方面的不合情理的地方,我也没有精力细细去想,当时感觉,应该是对方别有用意,只是我并不知道他的用意而已。

胖子听了就摇头,说不对:“这事情如果照你这么想,那也太没有头绪了,咱们生活在真实的世界里,这不是悬疑小说,不应该有这么没头没脑的事情发生,我看咱们可能有点把事情想得太复杂了,也许对方寄这录像带来,有着十分简单的理由。”

我脑子有点抗拒思考,不想去想,就让他说说他的想法。

胖子道:“倒也不是想法,只是感觉到你想问题的方式不对,似乎是给人绕糊涂了,咱们直接点想,对方寄了两盘带子给你,一盘有内容,一盘没内容,也就是说,其中一盘完全可以不需要寄,而对方却还是寄出了,对不对?”

我点头,胖子道:“那不就是了,这在这件事情中很正常,因为寄带子的人让人感觉到匪夷所思,我们主观就认为他做任何事情可能都有着深意。但是他娘的,如果不这么想,假设寄东西的那小子是个普通人,你认为普通人在这种情况下,会不会这么做?我想总不会吧,要是我寄带子给你,我干吗还搭一盘空白的寄过来?这不是有毛病吗?我感觉这里肯定有文章,你再想想看,是不是有道理。”

我点了点头,胖子永远会给人惊喜,确实这个问题我没想到这么深,我靠到坐椅上,想着胖子的话,陷入了沉思。

一个普通人,在什么情况下,会用这种方式寄东西过来?一盘有内容的录像带加上一盘没有内容的录像带,这样的组合,是什么用意呢?

不要把问题复杂化,我告诫自己,用直觉去想,想想自己以前借录像带的时候,什么情况下会做这种事情呢?

一想还真想到点以前的事情,心里一跳,感觉到好像确实有一段时候,自己也做过同样的事情。

一边的胖子正在吃东坡肉,看我的样子,就问道:“怎么?想到什么了?”

我歪了歪头,让他别说话,自己心里品味着刚才想到的东西,想着想着,以前的回忆就出现了,我沉吟了一声,突然一下就意识到是怎么回事了,猛地站起来,对胖子道:“我操,原来这么简单!别吃了!我们马上回去!”说着就往外跑去。

胖子肉吃了一半,几乎喷了出来,大叫:“又不吃?中午都没吃!有你他娘的这么请客的吗?”

我急着回去验证我的想法,回头对他说:“那你吃完再过来。”

胖子原地转了个圈儿,也是拿我没办法,只好跟了过来,临走对服务员大叫:“这桌菜不许收!胖爷我回来还得接着吃,他娘的给我看好了,要是少根葱我回来就拆你们招牌!”说着跟着我就出了门。

\chapter{录像带的真正秘密}

楼外楼离我的铺子不远,我急匆匆地跑回去,王盟是五点一刻下班,绝对不多留半分钟的人,早就锁了。我开了锁进去,来到内堂之内,阿宁带来的带子给她带回去了,我就翻出了我自己那几盘带子。胖子紧跟着我进来,帮我接驳电源。

但是我却没打算再看一遍,而是翻了几个抽屉,找出了一把螺丝起子。

胖子看不懂了,问我干什么,我心里翻腾着,也顾不得回答他,就开始拆卸那带子。

如果我想到的不错的话,这事情他娘的还真的是十分十分的简单,甚至我都做过很多回了。

两盘带子,其中一盘录像带竟然是空白的,那就是说,里面的内容根本就不重要,对方要寄给我的,是录像带本身,而不是让我们看里面的内容,所以里面是空白,或者有影像,一点关系也没有。那他寄来这盘带子,只有一个理由,一个简单到不能再简单的理由。而我的推测也非常容易验证。

以前中学的时候,捣鼓过不少这东西,拆起来也不难,三下五除二,就把带子分离了开来,然后我小心翼翼地拿起来一边,一抖,一边看着的胖子就惊叫了一声。

录像带的里面,一面的塑料壳内面,果然贴着一片东西。

“你奶奶的熊,你怎么想到的?”胖子惊讶道。

我咧嘴,也顾不得笑,拍他道:“那是你想到的。”撕下那东西,一看之下,我“哎呀”一声,只觉得心都扭了起来。

那是一张便笺纸,上面非常潦草地写了十几个字。

青海省格尔木市昆仑路德儿参巷349-5号。

识字的人一看就知道了,那是一个格尔木市的地址。

“丫的。”我不由自主地就冒京腔,我擦了擦头上的汗,心中有一种喜悦,总算给我料中了一样东西,原来真的是我自己想得太多了。

这是一石二鸟,一来可以保护这张东西不受长途运输的破坏,二来,如果这东西给人截获了,一时间对方也想不到它里面藏了东西,特别是,如果录像带的内容足够吸引那个截获者的注意力。

我心里明了,可以肯定对方要防范的那个截获者,就是我的三叔,因为里面的内容,只有三叔看了之后才会吃惊,事实也是,他的确被录像带里的内容吸引了所有的注意力。

这事情只要推断一下就很明显,因为如果他直接寄这地址过来,按照当时的情况,这东西必然会落到三叔手里,和最开始的那份战国帛书复印件一样。

想通了这些,我就非常的神清气爽,马上又拆掉了另一盘带子,这一盘带子里,却不是纸片,而是一把老旧的黄铜钥匙,而且是20世纪80年代最流行的四八零锁的那种钥匙。

拿起来展开,可以发现钥匙有点年头了,铜皮都发黑了。钥匙柄的后面,贴着胶布,上面写着一串模糊的数字:306。

“看来对方是想邀请你过去。”胖子在边上道,“连房间都给你开好了。”

\chapter{来自地狱的请柬}

我看着那地址和钥匙,就在那里发愣。胖子说得对,我刚才也在想这个事情,看样子寄录像带的人真的是想让我找过去,这钥匙应该就是纸上地址所在的门钥匙。那这样看来,我过去对方可能也不会在家,他是想让我自己参观?

我突然有了一个奇怪的念头,难道那房子是那小哥的家?他知道自己可能回不来,所以托人把他家的钥匙寄给我?算是留遗产给我?

如果真是这样,那也许到他家里去,还能知道他的过去呢,不过,这怎么想也不太可能……

另外,这样的话,阿宁那两盘带子里,难道也有东西?

当天晚上,我辗转难眠,靠在床沿上,一根一根地抽烟,我平时只有郁闷的时候才会抽一根儿,但是现在怎么抽都是没用,心里还是难受。

回想这整件事情,从我最初收到录像带开始,到现在发现录像带里的东西,不过几个月时间,然而每多一次的发现,就让事情变得更加扑朔迷离,更加复杂。

事实上,录像带的秘密虽然被我发现了,但是,真正让我心烦意乱的,还是录像带的内容,不管对方是想其中的内容来作掩护,还是只不过随手拿了两盘,其里面的内容,绝对会吸引观看者的所有注意力。而这些内容是无法伪造的,他这样的人也不可能会熟悉录像带的录制方式,那么,他是从哪里搞到的带子?

这样的录像带,我可以肯定不止这几盘,按照录像带的记录时间,记录满一天就需要八盘左右,寄给我一盘是空的,一盘是有内容的,这说明对方在拿录像带的时候,有很多的选择,那至少说明那个地方可能还有其他录像带。

里面“霍玲”和“我”,监视着自己的行动,显然有不得已的目的,不会是为了好玩。

当然,最让我在意的还是阿宁的那两盘。我一直自诩为一个局外人,一直自认为自己是一个添头,自己跟着三叔,第一次是自己率性而为,第二次是为形势所逼,第三次是莫名其妙地听从安排,每一次,只要说一个“不”字,就没有我的事,所以事情突然一下子发展到似乎连我也牵涉了进去,就有点找不着北了。

不过,胖子这一次的提示,让我犹如醍醐灌顶,我已经感觉到自己考虑问题的方式似乎太过复杂了,也许正是因为有这样自己困扰自己的习惯,真的使得原本十分简单的事情变得很复杂。或许事情本身就如这件事情一样,一点曲折都没有。

我想了很多,此时又想到当日李沉舟和我说的,这件事情也许和我有莫大的关系,想想三叔处心积虑地骗我,他既然不想让我参与这件事,又为什么要让我跟着上雪山?李沉舟的话其实非常的有道理。

我又回忆了我的过去,我记忆中任何有可能使得自己和这件事情沾上关系的,真的是一件都没有。小时候,我的父亲平平淡淡,凡事都以家庭为己任;我的爷爷叱咤风云,是家里的主心骨;二叔吝啬言语,一本正经;三叔游戏人间,顽劣不化。所有的所有,构成了我童年的记忆。他们虽然秉性都不同,但是都对我很好,连二叔也只有看着我的时候,会和我笑笑。

可以说我的童年虽然不是非常的幸福,但是,应该和我这个年纪的人的童年一样,毫无特别之处。

再到这几年,所谓的大学,更是平淡到了极点,记忆也更加清晰,实在是没有在一个黑暗的屋子里,穿得像个死人一样爬来爬去的经历。

我一个晚上没睡着,一直看天花板看到了天亮,胡思乱想,越想就越郁闷。整件事情,仿佛是一张天罗地网,将我罩在里面,我无论从哪里走,都只能看到无数的窟窿,却给网绳挡着过不去。

造成这样的局面,也是我的性格决定的,我那种犹豫不决又不死心的性格,导致事情越搞越复杂。或许我考虑问题不应该如此的被动,有时候不要等别人给你线索了,你再去琢磨,这样别人给你的线索一来不知道是真是假,二来,总是不太及时且有很多干扰的。

想到这里,我忽然皱了皱眉头,想起我那几个朋友在临走的时候给我的建议,他说:“事情变得如此错综复杂的原因,就是因为你老是执著于从你三叔那里得到答案。你想既然三叔骗过你了,就肯定不希望你知道一些事情,那么你三叔就不可能和你说实话,谎言生谎言,你再问只会让自己觉得世界上任何的东西都变得不可信,乱七八糟的信息越来越多,你要了解事情的真相,不如自己去寻找答案,比如你说探险队是十个人还是十一个人,你去查查当年相关的资料,总比分辨你三叔说的是真是假要可行得多吧。”

现在想想,确实他说得没错。

好吧!我心里对自己说,他妈的,既然这事情和我还有了关系,那我就真谁也不信了,这次我就谁也不告诉,自己一个人去格尔木查查看,这到底是怎么回事。

\chapter{鬼楼}

要么不做,要么就别磨蹭,第二天,我就确定了去格尔木的行程。

我从来没有去过那一带,找了我在旅行社的朋友询问了路线。那朋友告诉我,因为去格尔木没有直达的航班,所以我只有先飞到成都的双流,然后再转机。机票让他去搞,连当地的酒店都可以搞定。我就让他帮我处理,因为这里也不能说走就走,我订了两天后的航班。

这一次不是去盗斗,只是去格尔木的市区逛一逛,而且时间也不会很长,所以只带了几件贴身的衣服和一些现金,总共就一个背包还是扁扁的。

胖子当天就回北京了,我也没和他说起这个事情,既然决定谁也不说,那么胖子也不例外。

这两天时间里,我跟王盟打了招呼,让他处理铺子里的事情,家里含糊地交代了一下,又把一些关系理了理,两天后,我就上了飞机。

一路睡觉,到了成都双流之后已经睡得很舒服了,飞格尔木的几个小时,就在飞机上想事情。当天晚上八点多,我就到达了被誉为“高原客栈”的格尔木市。

这是一座传奇的城市,格尔木在藏语意思是“河流密集的地方”,虽然一路飞过来全是戈壁,但是也可以想象当时城市命名时候的样貌。我在飞机上看的资料是说,这座城市是当年“青藏公路之父”慕生忠将军把青藏公路修路兵的帐篷扎在了这里,扎出来的一个城市。城市只有五十多年的历史,早年繁华无比,现在,地位逐渐给拉萨代替了,整个城市处在一个比较尴尬的位置上。

下了飞机之后,非常丢脸的我发作高原反应,在机场出口的地方就直接晕了两三秒,那种感觉不像以前在秦岭的时候是那种力竭的昏迷,而是一种世界离你远去的感觉,一下子所有的景色全部都从边上变黑,接着我就趴下了。好在两三秒后我马上醒了过来,此时我已经躺在了地上。更丢脸的是,我在买药的时候,才知道自己现在已经在青藏高原上了,对中国的地理不熟悉,竟然不知道格尔木是在青藏高原上!搞得卖药的还以为我是坐错飞机了。

在路边的藏茶摊上喝五毛一碗的藏茶把药吃了,我就到了朋友给我安排的宾馆安顿了下来,顾不得头痛脑热的,又马不停蹄地出发,直接上了出租车,拿出那个地址,就让司机将我带过去。

然而司机看了地址之后,马上摇头说那地方是个很小的巷子,车开不进去,那一带全是老房子,路都很窄,他能带我去那一代附近,然后再往里去,就得我自己进去问人。

我一听那也成,就让他开车,一会儿工夫,我就来到城市的老城区。

那司机告诉我,格尔木市是一个新建的城市,路一般都很宽,当年的老城区都扩建了无数次,但是到处都有这样的小片地方,因为位置尴尬,一直遗留下来。这些平房大部分都是20世纪60、70年代盖起来的,里面到处是违章建筑,我的那个地址,就是其中的一条小巷。

我下了车,天已经是黄昏的末端了,昏黑昏黑,夹着一点点的夕阳。我抬头看去,背光中只看到一长排黑色瓦房的影子,这里都是20世纪60、70年代建的筒子楼,这个时间看过去,老城区显得格外的神秘。

走进去,四处看了看,我就发现这里其实也不能叫做区了,只不过是城市扩张后残存的几段老街,这些建筑一没有文物价值,二没有定期检修,看上去都有点摇摇欲坠,想必也不久于人间了。而老城区里也没有多少人,只见少有几个发廊,穿行于房屋之间,老房子老电线,黑黝黝的和发廊的彩灯混在一起,感觉相当怪。

我在里面穿行了大概有两个小时,走来走去,搞得发廊里的小姐以为我是有贼心没贼胆,都开门朝我笑。然而确实如那个出租车司机所说的,里面的格局太混乱了,很多巷子是给违章建筑隔出来的,连路牌都没有,问人也没有用,几个路过的外来务工人员都笑着善意地摇头,大概意思是他们也不知道这地方是哪里。

有地址也找不到地方,这种事情我还是第一次碰到,一边走一边苦笑,感觉世事的多变。就在绕得晕头转向的时候,后面骑上来一辆黄顶的三轮车,那车夫问我要不要上车?我走得也累了,就坐上让他带着我逛。

车夫是汉族的,大约也是早年从南方过来的,听我是南方口音,话就多了,和我说了他是苏北的,姓杨,名扬,人家都叫他二杨。在这里踩三轮十二年了,问我想到什么地方去玩儿,高档的、低级的,汉的、藏的、维吾尔的妞儿他都认识,全套还给我打个八折,要是不好这口,旅游他也成,格尔木没啥名胜古迹,但是周边戈壁有大风景,他都熟悉。

我心里好笑,心说你老爹要是再给你取个三字名儿,你就能改名叫恒源祥了,不过他说到这个,我就心中一动,心道这些个车夫在这里混迹多年,大街小巷大部分都烂熟于胸,我何不多问几句,也许能从他嘴巴里知道些什么来。

于是便把地址给他看了,问他知不知道这个地方。

我本来没抱多少希望,但是我话一说完,恒源祥就点头说知道,说着就踩开了,不一会儿,他骑到了一条非常偏僻的小路上。

路两边都是老房子,昏黄的路灯下几乎没有行人,他停车的时候我真的很恐慌,似乎要被劫持了。他见我的样子也直笑,对我说,我要找的地方到了。

我抬头一看,那是一栋三层的楼房,有一个天井,路灯下,楼房一片漆黑,只能看到外墙,里面似乎一个人也没有。整幢房子鬼气森森的。

我哑然,问车夫这里到底是个什么地方?他道:这里是20世纪60年代的解放军疗养院,已经荒废了很长时间了。

\chapter{306}

我下了车付了钱,在门口对了对已经模糊不清的门牌,发现纸条上的地址确实是这里。心里就有点发毛。心说这不是我们小时候经常去探险的那种没人住的鬼楼吗,怎么会有人让我到这种地方来?里面还有人住?

那车夫还在数我给他的零钱,我就转头问他,这里面住的是什么人?

那车夫就摇头,说他也不清楚,他只知道这个疗养院是20世纪60年代盖起来的。格尔木是个兵城,军官很多,很多国家领导人经常来视察,这个疗养院是给当时的领导住的,在80年代中期的时候,疗养院撤掉了,这里改成了戏楼,所以他也来过。当时的河东河西就这么几片儿地方,我还比较走运碰上了他,要是其他那些北方来的三轮车夫,保管也找不到这地方。

我听得半信半疑,车夫走了之后,整条街道上就剩下我一个人,我左右看看,一片漆黑,只有这栋楼的门前有一盏昏暗的路灯,有点害怕,不过一想自己连古墓都大半夜下去过了,这一老房子怕什么,随即推了推楼门。

楼外有围墙,墙门是拱形的红木板门,没有门环,推了几下,发现门背后有铁链锁着,门开不开,不过这点障碍是难不倒我的。我四处看了看,来到路灯杆下,几下就爬了上去,翻过了围墙。这是小时候捣蛋的身手,看来还没落下。

里面的院子里全是杂草,跳下去的,可以知道下面铺的青砖,但是缝隙里全是草,院子里还有一棵树,已经死了,靠在一边的院墙上。

走到小楼跟前,我打开打火机照了照,才得以了解它的破败,是雕花的窗门,不过都已经耷拉了下来,到处是纵横的蜘蛛网,大门处用铁锁链锁着,贴着封条。

我扯开一扇窗,小心翼翼地爬了进去,里面是青砖铺的地,厚厚的一层灰,门后直接就是一个大堂,什么东西也没有,似乎是空空荡荡的。我举高了打火机,仔细转了转,发现有点熟悉,再一想冷汗就下来了。

这个大堂,就是阿宁的录像带中,“我”在地上爬行的地方。

来对地方了,我对自己说。我站到了录像带中,录像机拍摄的角度去看,那些青砖,那些雕花的窗,角度一模一样,我越来越确定了我的想法。一种恐惧和兴奋同时从我心里生了出来。

继续往里走,就在大堂的左边有一道旋转的木楼梯,很简易的那种,但好歹是旋转的,通往二楼。我蹑手蹑脚地走过去,朝楼上望去,只见楼梯的上方,一片漆黑,并没有光。

我掏出了口袋里的钥匙,306,那就应该是三楼的。

这多少有些异样,我低头照了照楼梯的踏板,发现踏板上盖着厚厚的尘土,但是在尘土中,能看到一些脚印,显然这里还是有人走动的。

我轻轻地把脚放在踏板上踩了踩,发出咯吱的声音,但是应该能承受我的体重,我咬紧牙小心翼翼地往上走去。

楼上黑黑的,加上那种木头摩擦的“咯吱”声,让我感觉有点慌慌的,但是这里毕竟不如古墓,我的神经还顶得住。

一直往上,到了二楼,就发现二楼的走道口给人用水泥封了起来,没有门,是整个儿封死掉了,按照楼下的空间,水泥墙后面应该还有好几个房间,似乎给隔离了起来,水泥工做得很粗糙。

我摸着墙壁,感觉到有点奇怪,难道这房子的结构出现过问题,这里做了加固?

不过奇怪也没用,我此时也没有多余的精力考虑这些问题,继续往上进入到三楼,我看到的是一条漆黑的走廊,走廊的两边都是房间。但是所有的房门下面都没有透出光来,应该是没人,而空气中是一股很难闻的霉变的味道。

我凝神静气,小心翼翼地走进走廊,绕过那些蜘蛛网,看到那些房间的门上有被尘埃覆盖的油漆的门牌号,我一路读下去,有点感觉自己好像那些欧美悬疑片里的主角。不久,便来到了走廊的倒数第二间房门外,我举起发烫的打火机,照了照门上,只见门楣上有很浅的门号:306。

那一刹那我开始想敲门,一想又觉得好笑,于是在门口犹豫了一下,就掏出了钥匙。往门口的钥匙孔里一插,随即一旋转,“咯嗒”一声,门随着门轴尖锐的摩擦声,很轻松地被我推了进去。

房间不大,里面很黑,进去霉变的味道更重了,先是从门缝里探头进去看看,发现房间的一边可能有窗户,外边路灯的光透了进来,照出了房间里大概的轮廓。房间里贴墙似乎摆着很多的家具,在外面路灯光形成的阴影里看不分明,不过,一看就知道没有人。

我深吸了口气,小心翼翼地走进去,举起已经发烫的打火机,在微弱的火光下,四周的一切都清晰起来。

这是一个人的卧室,我看到了一张小床放在角落里,霉变的气味就是从这床上来的,走近看发现床上的被子都已经腐烂成黑色了,味道极其难闻,被子鼓鼓囊囊的,乍一看还以为里面裹着个死人,不过仔细看看就发现只是被子的形状而已。

在床的边上,有一张写字台,古老的类似于小学时候的木头课桌,上面是一些垃圾、布、几张废纸和一些从房顶上掉下来的白石灰块,都覆盖着厚厚的灰。

在写字台的边上是一只大柜子,有三四米宽,比我还高,上面的木头大概是因为受潮膨胀,门板都裂了开来,抬头往上看,就可以看到柜子上面的房顶和墙壁的连接处,有大量的煤斑和水渍,显然这里在雨天会有漏水。

这地方看来已经荒废很久了,这种破烂的程度,应该有五年以上了,不过房子虽然老旧,却也是普通的老旧而已,寄录像带的人把我勾过来干什么呢?他想我在这房子里得到什么信息呢?

此时忐忑不安的心情,也随着我对环境的适应而逐渐平静了下来,我将打火机放到桌子上,先是开始翻找那张木头写字桌的抽屉,把抽屉一只一只地拉出来,不过里面基本上都是空的,有两只抽屉垫着老报纸,都发霉了,我碰都不敢去碰。

抽屉里没有,难道是床上?我走到床边上,先看了看床底下,全是蜘蛛网,什么都没有,然后到边上拿出一只抽屉,用来当工具,把粘成一团的被子从床褥上拨了开去,想看看里面是不是裹着什么东西,然而拨了几下,被子里直冒黑色的黏水,竟然还有虫子在里面,霉味冲天,我几乎恶心得要吐了。

好不容易把被子全拨弄到地上,却也没发现什么东西,其实我拨了几下也意识到里面不会有东西,谁会把东西藏在这么恶心的地方。

这两个地方都没有,那么只剩下这大柜子了,不过这柜子都有锁,虽然柜子的门开裂了,但是要打开这柜子,还是需要点力气的,而且没有工具是不行的。

我手头什么都没带,只好就地去找,最后在窗台找到了个东西。那是老式窗的插销,能拔出来,虽然都锈了,但是老式插销是实心的,很结实。我拔出了一个,就用来当撬杆,插进那些开裂的柜门板缝里,把缝撬大到能让我伸手指进去,然后一只脚抵住一面,把手伸进缝里,用力往外掰。门板发出恐怖的摩擦声,给我扯得弯了起来,接着就发出断裂的爆裂声,整块板就这样硬生生地掰断了,门上的灰尘都溅了起来,迷得我睁不开眼睛。

楼里相当安静,我这些动静听上去就格外的吓人,门板断裂的那一刹那,那刺耳的声音把我也吓得一身冷汗,好久才缓过来,然后拿起打火机,往柜子里照去。

我对柜子有什么东西,一点预判也没有,感觉最大的可能还是什么都没有,所以也没有太过作心理准备,然而一照之下,我就吃了一惊。

柜子里确实什么都没有,空空荡荡,但柜子靠墙那面的底板已经不翼而飞,露出了柜子遮住的水泥墙,而在水泥墙上,竟然有一个黑幽幽半人高的门洞,连着一道往下的水泥阶梯,不知道通向哪里。

\chapter{线索}

我感觉越来越古怪,显然,这里竟然有一道暗门,有人用一只去掉了底板的柜子,当成掩护挡住了它。只要打开了这只柜子,就能看到后边的暗门,这种方法不算是高明,但是好处在于设置方便,而且便于出入。

可是这里怎么会有这样的构造?看来这疗养院不简单啊,这里以前到底是用来干什么的?这水泥阶梯下又是什么地方呢?

看着手里的钥匙,显然对方寄了这个号码房间的钥匙给我,就是想我发现这道暗门,那么,下面应该有答案。

我擦了擦头上的冷汗,走进柜子里,探进暗门,顿时一股奇怪的味道从下面传了上来。

我转过头把最浓烈的味道让了过去,然后适应了一下,用打火机往下照。

阶梯深不见底,而且有曲折,显然长度颇长,不知道是通向二楼,还是一楼的。

看着楼梯,想到现在已经是半夜,我身在一幢鬼屋里面,又发现这不知道什么时候安置的暗道,心中不免有些害怕,然而毕竟我是下过斗的人,在这种地方,知道外面就是大街和发廊,心中自然会稍微坦然一些。

我只犹豫了一下,就定了定神,一只手小心翼翼地举着打火机,矮身进到这个门洞里面,顺着阶梯向下走去。

既然已经到了这一步,对方指引我寻找的东西,必然就在这楼梯下面,我也不好退缩,来到了格尔木,自然要看看对方的目的到底是什么。

才走了几步,我就感觉到一股难言的阴冷从阶梯前方的黑暗中传了过来,冷得有点让人不寒而栗。我哈了一下,就发现有白气从我嘴巴里呼出来,这下边的温度看来确实很低。

从打火机的光线看去,楼梯两边都是毛坯的水泥墙壁,水泥是黄水泥,20世纪60年代的那种军用品种,上面隐约还能看见一些红油漆刷的标语,都褪色得只有几个轮廓能分辨了。在阶梯的顶上,还能看到垂下的电线,被蜘蛛网包着,看上去就像蛇一样。

比起古墓里的青砖墓瓦,这些东西要亲切得多了,我一边暗示自己,一边尽量放松心情。虽然如此,我还是觉得下面黑暗处的楼梯转角,会有什么东西探出脸来,毛骨悚然的感觉竟然一点没有比古墓里差。

很快就走下了第一段,阶梯转了一个弯,继续向下,脚步出现了回声,听起来毛瑟瑟的。我感觉了一下高度,这里已经是二楼了,就是被水泥封闭的那一个楼层,然而,这里并没有任何的门洞,四周还是封闭的水泥,显然,出口并不在这里。

看来和那二楼没有关系,我心道,深吸了一口污浊的空气,又往下走了一层。

还是同样的情况,出口也不在一楼。阶梯继续转了一个弯儿往下,仍旧黑漆漆的看不到底。

下面就是地下了啊,我心说。这时候心里出现了一个念头:难道这楼梯是通到地下室去的?

难道,这里是以前的地下军事掩体?

我心里记得在杭州有一个著名的704公馆,也是以疗养院的名义修建的,其实里面机构纵深,神秘异常,据说地下面也有巨大的建筑,用来应对紧急情况。

不过,看这暗门的样子,又感觉不像。那暗门就是一个简陋的门洞,如果是特地设置的军事掩体的入口,至少应该会有铁门吧。

我边走边胡思乱想,继续往下走去,不知道是温度继续下降,还是我的冷汗给我的感觉,我忽然感觉到极度的寒冷,牙齿都打起牙花来了,咬牙又下了一层。阶梯到这里就中止了,阶梯的出口就在面前,我小心翼翼地走出去,发现外面似乎有一个很大的空间。

我举起打火机,照了照出口两边,发现这是一个水泥加固过的地下室,非常的简陋,潮气冲天,地上还铺着青砖,四周空空荡荡。

这肯定不是军事掩体,我心里确定了,看这水泥的样子和地上的青砖,像是农村里生产大队自己胡乱盖起来的那种地窖。这里的手工太简陋了,不会是专业的军工部队盖出来的。

这是什么地方?难道真的是个地窖?闷油瓶让我过来是看他的腌白菜入味了没有?

我给自己的念头逗乐了,一边往这个地窖的中心走去。走了没几步,我就隐约看到,地下室的中间,有一个巨大的影子,横倒在地上,看上去非常的怪异。

我朝那个影子走过去,用打火机一照,人就僵住了,只见地窖的中央,停着一只巨大的纯黑色的古棺。

\chapter{计划}

打火机的光线十分的微弱,能照出两三米外的情形已经很不错了,在这种光线下,赫然看到一只棺材,我还真是吓了一跳。

反应过来之后,就感觉到非常的奇怪,这真是闻所未闻的事情,他娘的这里怎么会有一具棺材,而且还是古棺?

一座20世纪六七十年代建造的、给领导休息用的疗养院,有地下的隐秘设施,这说起来已经有点不可思议了,现在在这个地方,还出现了一只棺材,这太匪夷所思了。这里面装的是什么人?难道是当年死在这里的军官?

我看了看身后,来时候的楼梯口就在身后,不至于找不到,就靠过去看那只棺材。

远远看过去就知道这不是现代人的棺材,棺材是纯黑色的,横在地下室的中央好比一只巨大号的长条石墩,这样大小形状的应该是棺椁,民国以后的棺材就没有棺椁了。这棺椁看式样应该有相当的历史,至少在五六百年以上,而且看大小,恐怕不是普通人家用的,至少也是士大夫用的。

我上前摸了一把,上面有细细的花纹,冰凉刺骨,像是石棺,不知道是什么石料。一摸之下,石棺上厚厚的灰尘被我划了几个印子,露出了一些细小的花纹。

拿打火机靠近仔细地看,棺椁的盖子上,有敲凿损坏过的痕迹,盖子和椁身的缝隙里也有撬杆插入的迹象,显然我不可能是第一个发现这只巨大棺椁的人,有人曾经想撬开它,我有过经验,所以对这个特别的敏感。

古棺不可能平白无故地出现在现代建筑的地下室里,那肯定就是有人将这棺椁搬到这里来的,不晓得原因。

地下室里的温度十分低,我喘着气逐渐冷静了下来,用力舒缓我的心跳,一路下来都是在极度的紧张中度过的,虽然自己压抑了恐惧,但是心中还是相当的不舒服。一边深呼吸,我就开始琢磨。

有人寄了录像带、地址和钥匙将我引到这座破旧疗养院里来,指引我发现了这一个暗门,通过暗门后的楼梯我发现了这个地下室,地下室里还放着一具石棺。

这已经超出了任何恶作剧的范畴,对方是不是想告诉我,这疗养院里发生过的一些匪夷所思的事情?

看来,这封闭的楼层和地下室,以及这石棺的背后,肯定有着相当复杂的故事。

我推动了一下石棺的盖子,当然没有用大力气,只是想试验一下能不能推开,好在和我的判断一样,石棺纹丝不动,显然没有工具我打不开它。

我松了口气,在这种场合下开棺,而且是一个人,我从来没有经历过,打不开,也不用硬着头皮逼自己上了。

再仔细地看了一遍石棺的细节,发现没有什么值得注意的,我就绕过石棺继续往前走,一直走到地下室的尽头,就看到一扇小铁门,很矮。我推门进去,后面是一条走廊。

我只走了几步,就发现了这里的结构和楼上是一样的,一条走廊,两边都是房间,只不过这条走廊一路延伸,没有尽头,似乎通到其他地方去,而走廊两边的房间都没有门,十分的简陋。

我拿起打火机走进第一个房间,照了照,就看到了两张写字台靠墙摆在一边,四周有几个档案柜,墙上贴满了东西,地下、桌子上,全是散落的纸。

这里似乎是一个办公室。我心中越加的奇怪,办公室怎么会设置在地下?这也太怪了。地下室里,一边是只棺材,一边是间办公室,难道当年格尔木的丧葬办是设在这儿的?

我边纳闷边走到写字台边,想看看上面有什么线索。

走近一看,我忽然就愣了一下,不知道为何,看到这写字台摆放的样子,我心里有一种异样的感觉,好像这房间在什么地方看到过。

举高打火机我回忆了一下,顿时倒吸了一口冷气,立即就认了出来,这间房间,竟然就是霍玲录像里照出的那一间。

写字台的摆设,地面和墙上的感觉,一模一样,我走到写字台边上,甚至看到了那面她梳头的镜子,还放在录像带里的那个位置上。

我的心一下就狂跳起来,忙深吸了一口气,按捺住自己的情绪,心中的诡异已经到达了顶点。

看霍玲录像带的时候,还只是以为她是在什么民居里,没有想到,竟然会是在这种疗养院的地下室里,而且竟然我还找到了这个地方。那显然这都是真的,录像带里记录的内容是真的。

当年霍玲就在这里,用录像机拍摄过自己,她在这里不停地梳头,而“我”,也很有可能真的爬过头顶的大堂。

一刹那,我的眼里甚至出现了她的虚影,我和她的世界好像重合在了一起。录像带的情景在我面前闪动了一下。

可这到底是怎么回事呢?一个女人在一间疗养院的隐秘地下室里,不停地梳头,而一个和我相似的人,在疗养院的大堂里如残疾人一般地爬行。这些事情都真实地发生,并且被记录下来了,这到底是为了什么?镜头之外的这个疗养院里,到底发生过什么事情?

我脑子有点发木,晕了起来,显然寄录像带给我的人,目的就是引我看到这个房间,可是我看到了之后,反而更加的疑惑了,感觉自己好像在拼一幅空白的拼图一样,完全没有着手的地方。

再一次深吸了几口气,我镇定了一下,接着,就拿起打火机开始观察四周,我必须查看一下这里,看看有什么线索。

\chapter{盗墓笔记}

这是一个神秘疗养院的神秘地下室,一个神秘的女人在这里做过一些匪夷所思的行为。那么,既然她在这里生活过,总会留下蛛丝马迹,如果能找出一点,也许就能明白一些事情的真相。就算都是没有用的资料,我也能知道她当时的生活和精神状态是怎么样的。

我对于这个疗养院里发生的一切,几乎一无所知,所有的线索对于我都是重要的。

我开始搜索,只要是能看的东西,我都要去看一看。

这里的楼很低,我的身体在这里相当压抑,但是打火机的照明却因此比较管用,能照出很远,我大概看了四周,决定从哪里查起。

在录像带模糊的黑白影像里,无法自由地观看房间的全貌和细节,但现在可以了,看到的东西就更加直观一点。我先想象了真实的霍玲梳头的样子,相当的恐怖,忙摇头转移注意力。

我手里的这一款zippo能够持续燃烧照明,但是已经烫得我只要往上再捏一点就捏不住,从桌子上找了块破布,包住继续使用。

在微弱的火光下,我先是看了墙壁,这个房间四面墙壁上都刷着白浆,现在都被灰尘覆盖了,在门边的墙上钉着一条插着衣钩的木棍,那是用来挂衣服的地方。木棍的下面贴着报纸,防止挂着的衣服碰到墙壁上的白灰。木棍过来,就是一只已经没有门的柜子,这应该就是霍玲换衣服的地方,现在里面什么都没有。我走近看时,就发现柜子好像被什么东西抓过一样,满是刻痕。

再边上的墙,就什么也没有了,只有挂在上面的电线,已经全是灰色的了,一边还有一道连通隔壁房间的门洞,不知道是修筑的时候没有封起来,还是后来给人砸出来的,对面的房间里空空如也。

在柜子的对面,摆着写字台,有两张并排放着,上面堆满了东西,似乎都是一些报纸和我看不清楚的垃圾。在写字台边上的墙壁上贴着大量的纸,都布满了灰尘。

我吹掉灰尘,一张一张地看过来。发现墙上贴的内容非常的琐碎,我看到了20世纪90年代的电费单,一些顺手写下去的、毫无意义的号码。这些已经几乎和墙壁成为一个整体的纸,应该都是当时顺手当电话记录本的,因为我记得电话就放在这个位置。不过现在已经没了,只剩下一根断截的电话线。

这些东西无法给我任何的信息,我只能知道她在这里生活的时候用电。我叹了口气,接着开始翻找桌子上的文件。

那些纸都是在灰尘里,一动漫天的烟雾,我也管不了这么多,一张一张地翻开了,纸的里面已经烂了,有很小的蚰蜒被我惊扰出来,不过这些东西和长白山的雪毛子比就是小弟弟,我很快就把纸翻了出来,从里面抽出了几个本子。

拿出来抖了一下,我就发现这好像是大本的稿纸簿,以前没电脑的时候用来写稿的,上面写了什么东西。

我翻了开来,看到第一页上,就三行字:

后室2-3

编号012~053

类:20、939、45

这是什么意思?我心说,好像是什么档案的编号,难道是什么手写的文件或者典籍?

翻过去第一页一看,却发现不是。第二页上,竟然是一幅图画,还是圆珠笔画的,而且画得相当的潦草,一下子竟然没法看出画的是什么。

我定了定神,仔细地去辨认,看了五六分钟才看出来,这竟然是一幅古代人物画,只不过此人显然并不会画画,这人物画得几乎走形,看上去异常诡异,那古代人物,不像人,反倒像只长嘴的狐狸。

人物的四周还画着很多匪夷所思的线条,我看出那鬼东西是个人后,这些线条的意义也显现了出来,应该是人物画的背景,大约是山水庙宇树木之类的东西。

我不由失笑,心说这是什么,难道是霍玲的素描?她的爱好倒也挺广泛。

翻过去,一连又翻了三四十页,全部都是这样的图画。没有文字的内容,我便放下,又看了另外一本,也是同样,除了第一页上的内容不同之外,里面都是差不多的图画。也不知道是什么东西,就堆在一边,继续翻那些纸头。结果下面就没什么,只发现里面有几团类似于抹布的东西,连一张有内容的纸都找不到。

我又骂了一声,心说看来他们离开的时候,可能将那些有信息的东西都带走了。

不过我不死心,我就不信能带的什么都不剩下。我坐到霍玲梳头的那个位置上去,休息了一下,就拉开面前的抽屉,想看抽屉里是什么。

那是那种写字台中部,台面下最大的那个抽屉,我拉了一下,就感觉到有门,他娘的抽屉竟然是锁着的,而且感觉沉甸甸的。

一般搬家之后不会把废弃的家具锁起来,而且这手感表明里面可能有东西了,我兴奋起来。这种锁可难不住我,我站起来,拆了一个门后的挂衣钩过来,插进抽屉缝里用力往下压,一下就把抽屉的缝隙给压大了,锁齿脱了下来,我一拉,就把抽屉拉了出来。

拿起打火机一照,我就YES了一声,抽屉里果然放满了东西,我将打火机搁在抽屉边上,开始翻找。

这肯定是一个女人的抽屉,里面有很多琐碎的杂物,很乱,显然离开的时候已经把有用的东西带走了,剩下了木梳,小的20世纪90年代那种饼一样的化妆盒,一叠厚厚的《当代电影》杂志。这些老杂志历史很悠久了,记得我小时候是当黄色书刊来看的,还有那种黑色的铁发夹,和很多的空信封和一本空的相册。

信封非常多,但都是没有使用过的,我很耐心地一封一封展开口子看,里面什么都没有,相册里也没有照片,可以发现原本肯定是放过的,但是都被抽走了。

接着,我又翻了那些旧杂志,一页一页地翻,格外的仔细,然而仍旧没有发现。

我倒到坐椅上,也不顾上面的灰尘就靠了下去,有点疲惫地透过昏暗的打火机光看向桌子的对面,四周一片漆黑,安静得要命,我的心也失望得要命。显然,如果这个座位属于霍玲的话,这个女人相当的仔细,而且是故意不留下线索的。

四周的寒冷已经在和我打招呼,我咬了咬牙,不能放弃,他娘的,罗杰定律,不可能什么都没有留下,我肯定能发现什么!我再次鼓励自己,虽然心里已经有点绝望了,就把抽屉一只一只地推进去,起身去看对面的写字台。

对面没有椅子坐,我就弯下腰来,发现中间最大的抽屉还是锁着的,这有点奇怪,我故技重演,将抽屉撬了开来。

我满以为看到的景象会和刚才一样,自己还是得在垃圾堆里翻线索。然而出乎我的意料,这一次抽出来一看,抽屉里却十分的干净,空空荡荡,什么都没有,只有在抽屉的正中,放着一个黄皮的大信封,鼓鼓囊囊的,有A4纸这么大,正正地摆在那里,好像是故意摆上去,等着我来看一样。

“咦?”我就心中一动,意识到了什么,马上拿起来看。

这是20世纪80年代末期的那种劳保信封,材料是牛皮纸的,上面有褪了色的毛泽东头像,摸了一下,就发现里面有很厚的东西,不过已经受潮了,摸上去毛刺刺的,很酥软的感觉。信封上没有任何的文字。

我感觉着这就有门了,忙翻过来打了信封,往里面一掏,就掏出了一本大开杂志一样的老旧工作笔记。

我愣了一下,翻开了封面,发现笔记本的第一页上,有一段娟秀无比的钢笔行书:

我不知道你会是三个人中的哪一个人,无论你是谁,当你来到这里发现这信封的时候,相信已经牵涉到事情之中。

录像带是我们设置的最后一个保险程序,录像带寄出,代表着保管录像带的人已经无法联系到我,那么,这就代表着我已经死亡,或者“它”已经发现了我,我已经离开了这个城市。

无论是哪种情况,都意味着我可能将在不久离开人世,所以,录像带会指引你们到这里来,让你们看到这本笔记。

这本笔记里,记录着我们这十几年的研究心血和经历,我将它留给你们,你们可以从中知道那些你们想知道的东西。

不过,我要提醒你的是,里面的内容,牵涉着一些巨大的秘密,我曾发誓要把这些带入到坟墓之中,然而最后还是不能遵守我的诺言。这些秘密,看过之后,祸福难料,你们要好自为之。

陈文锦

1995年9月

\chapter{文锦的笔记}

看到这一行字,我深深地吸了一口气,心中的惊骇简直无法形容。

这段文字的内容,倒还不是让我最惊讶的,说实话,我在看到那本笔记的一刹那,也想到过也许会看到这样的内容。让我一瞬间窒息的,是那个签名。

“陈文锦!”

天哪,我实在是没有想到,这东西竟然会是她留下来的,这么说,给我寄录像带、把我引到这里的,就是她?

这实在是峰回路转,又让人摸不着头脑。虽然三叔并没有说过她的任何信息,但是在我的概念里,她肯定已经在某个地方死了,怎么会在这种时候突然出现,而且,还把我引到了这里来?

而且这短短一段话里,包含的信息太多了,什么三个人?是哪三个?它是什么东西?我们是指谁,难道是西沙的那批人?什么研究?什么秘密?

无数的念头从我脑子里闪过,我却一个都来不及思考,我定了定神,就立即把笔记翻开了,往后面看了下去。

这是一本很厚的笔记,写满字的足有二十六七页,全是密密麻麻的字,写得极其工整,还有很多的图画,好像是一本工作笔记。

我将打火机放到拉出的抽屉沿上,然后自己坐到地上,马上凝神静气看了起来。

刚翻开第一页,扉页后的那一页,我立即被震了一下,我看到了一张奇怪的图画在上面,画得十分的精细。

这张画只有七条线条组成,六条弯曲的线条和一个不规则的圆,我只是稍微回忆了一下,就立即发现这就是三叔给我描述的,战国帛书中翻译出来的那个图形。

我心中诧异,看来文锦他们相当厉害,能得到这个图形非常的困难啊,这么说,她也对这个图形感兴趣过。

然而,和三叔给我画的草图不同的是,这一次这幅画上就有了标注,我一看就冒了一身的冷汗。只见这六条曲线上,各有一个黑点,感觉似乎就是三叔和我说的,那星盘和直线对齐而选择出的六颗星星,然而在其中四个黑点上,我却都看到了几个小字。

从上往下,就是:

长白山——云顶天宫

瓜子庙——七星鲁王宫

卧佛岭——天观寺佛塔

沙头礁——海底沉船墓

我看着就吸冷气,心里乱了几秒之后,一下却如醍醐灌顶一样,立即就明白了我看到了什么。

晕,太晕了,难道这图形的曲线,竟然是汪藏海定的那条大龙脉中,每一条山脉的走势脉络图?

仔细去看曲线,就发现果然是这样。因为不是在地图上看,所以这六条线根本就没法让我联想起这一点,只感觉像是叶子的经脉或者是河流的分布图,然而现在一看,我立即就看了出来这其实就是一条“龙”。六条线条,就是龙头、龙尾巴、龙的四肢!每条线都是一条山脉,而线条上的点,就是山脉上的宝眼。

那这根本不是裘德考说的什么星图嘛!

一下我就浑身冰凉,意识到了怎么回事。我靠,要不就是裘德考给人误导了,要不就是这老妖精骗了三叔!

再看那两条没有写字的线条。我立即就发现上面也有黑点,不过边上写的都是问号,显然,这些也应该是大风水所属的龙脉,不过上面的龙眼的情况,并不清楚。

这突如其来的冲击让我几乎有点不知所措,我实在没想到,一翻开笔记本就会受到这种颠覆。我立即合上笔记深深地吸了口气,然而手还是发起抖来,想起扉页上的那句话:里面的内容,牵涉着一些巨大的秘密。心说你也不用在第一页就这么刺激我啊!

然而,这种震惊很快就被狂喜代替了,我咬牙拍了拍胸口,把那种窒息的感觉去掉,就再次翻开笔记。

仔细地看那幅图形,这一次,我看到了更为关键的地方。

只见在六条线条之外,被六条弯曲的线条围绕的空白处,那个圆圈的内侧,也有一个黑点。这个黑点不在任何一条线条上,独立而孤单地处在整个图形大概正中的位置上。

而在这个黑点的边上,也有一行小字:柴达木——塔木陀。

这个我就看不懂了,但是这一行小字的下面,被画了好几道很深的线,还有两三个问号,显然,这张图上,这个点才是最重要的。而且,画图的时候,文锦有着什么疑问,所以一边想一边画了这些问号。

按照边上的经验来看,这一点应该也代表是一个地方,柴达木?塔木陀?难道也是一个古墓吗?我心里说道,为什么这一点会在线条的外面呢?

一下子,我忽然就意识到,文锦知道的,要比我们多得多。看来这本笔记能够让我知道相当多的疑问了。想着,我立即将笔记本翻了过去,开始看后面的内容。

后面的内容,都是文字和图画混杂的东西,上面的字迹十分的工整,写得也十分有条理,然而,字体很小,在打火机有点暗淡的火光下,看起来十分的吃力。

我定了定神,聚集起精神,用心看了下去,一边看,一边就越来越感觉到疑惑,同时也感觉到失望起来。等到看完之后,我的疑惑和失望到达了顶点。我呆在了那里,心中的感觉很难形容。

整本笔记上的内容大概可以分为三个部分,超过十万字,都是类似日记一样的工作记录,记录的内容非常的烦琐,但是按照里面记录的内容来区分,大概可以分为三个部分。

第一部分是1990年的4月2日-1991年3月6日记录的,这里无法把整本笔记都抄写下来,我只能将其缩写并选出最关键的章节,以求看得明白。第一部分的内容如下:

1990年的4月2日

我们将海底墓穴中大部分的瓷器都进行了编号整理,临摹了几乎所有的瓷器,同时比对壁画,希望能够找出汪藏海的人生轨迹。通过这样的比对,我们确实发现了一些规律,在壁画中记录的东西,是他人生的经历,而瓷画中的内容,是他建筑工程的过程。这从我们整理出来的几个系列就可以证明,比如说进入东夏国——建筑云顶天宫,还有受到朱元璋的封赏——设计明皇宫,都找到了体现,并且按照墓室的顺序,可以很容易地区分这些时间的先后顺序,而且一一对应。

按照这样的方式推断,这些壁画,都是记录着汪藏海显赫的风水大家的功绩,记录下的内容,都和他的作品相关,那些对其他人来说比较重要的,比如说婚娶、狩猎,都没有任何的记录。我称呼这个为“汪氏相对论”。

1990年的9月6日

今天,“汪氏相对论”遇到了一个难题,在汪藏海最后的壁画中,我们发现了这么一段内容:

(下面是一张草图,大约是壁画的临摹,我看到这里,就想起刚才翻桌子时候看到的那些类似于小孩子素描的东西,原来都是他们临摹下来的壁画。

草图的内容很难描写,因为画得很糟糕,我只能大约看出,那好像是一个达官贵人,送别另一个人的景象,背景是一座很大的宫门,四周整齐地横列着“骆驼马匹”之类的动物,当然画得完全像狗和老鼠。我熟悉古代山水画和走兽画,这方面的知识我受过严格的训练,所以我从笔触和形态上,可以猜测出这些奇形怪状的动物,其实应该是马匹或者骆驼。在宫门之后侍者成群,排成仪仗的队伍,可见画中画的是一个相当浩大的场面。

之后还有两三页都是画,我没有兴趣,全部跳了过去,直接看后面的内容。)

这些壁画上,应该是汪藏海六十八岁以后的事情,当时他已经完成了他最后一个工程,而这壁画上的内容,大概是说他接受了皇帝的命令,出发前往一个地方,类似于出使他国这样的活动。这张壁画的构图,明显是模仿唐玄奘西去西域的那些唐代壁画,非常奇特。然而,我们翻查了所有的瓷器,却始终没有发现任何能和这相对的瓷画。

有人说可能是最后的这一次经历,他没有任何的建筑作品相对,然而也有人却坚持地认为,像汪藏海这样的人,不会有这种例外。没有瓷画对应,可能有什么特别的含义或者原因,也许,他的作品被他刻在了其他什么地方。

确实,后来继续研究,就发现在汪藏海最后的那几年十分的神秘,完全没有任何的史料留下来,他的人生,可以说最后的一段时间是空白的。

他在那几年里到底在什么地方,干了什么呢?这是一个大问题。

1990年的12月6日

这几个月,我们一直调查汪藏海最后几年行踪,终于有了线索。我们发现在最后的工程之后,汪藏海陪同皇帝在长白山有一次祭山的活动,之后就开始没有任何的文字记录。

长白山,难道说他进山里去了吗?我们非常怀疑。

1990年的12月7日

这里无从查起,我们调转了方向,开始从皇帝那边入手,在明志中有详细的出使往来和大典的记录,我们想从其中寻找汪藏海壁画上描绘的那次大典,或者他出使别国的记录。

结果非常让人惊讶,我们发现皇帝死之前两年,一共有七次大典,其中六次都很正常,但是,有一次却很奇怪,记录十分的简单,没有任何的旁注:

“洪武二十九年,卫四十六人,士十二人,马匹一百二十六,珍珠十斗,黄金三十斤等,使塔木陀。”

大典和出使,这是唯一两个条件齐备的记录了,然而这个记录没有记录当时出使的官员,但是最让人奇怪的是,塔木陀是什么地方?

是一个国家吗?正史中没有任何的记载,不过很有可能,在明朝的周边,东南亚,西域这两块地方,有着无数的小国,也许这是小国中的一个。不过,汪藏海去和一个小国通使节,这有点奇怪。他的年纪不太适合干这种长途跋涉的事情啊。

1991年2月11日

调查继续进行,期间我们进行了两次讨论。

(中间是十页的废话,都是讨论和猜测,但是后面都证实错误,所以都删除了。)

因为明史在清朝经历过一次浩劫,所以这一次调查起来很困难,很久没有结果,后来还是转换调查角度解决了一个问题。我们对出使“塔木陀”所携带的东西作了比对,就发现礼品的种类表示这应该是一个西域的国家,而且礼品的数量作为礼品看来,很少。然而马匹非常多,这看起来,倒像是一只商队,而不是使节队伍。

1991年3月6日

完全没有线索,突破口也找不到了,研究停滞不前,大家心情都不好。

这就是第一段,明显的,这一段应该是前面还有内容,但是前面并没有发现被撕页的痕迹,看样子,这不应该是一本单独的笔记,这是一本系列笔记中的一本。

第一段里面描写的内容,他们是在作关于汪藏海的研究,发现“塔木陀”,然后研究“塔木陀”,最后研究停滞这么一个过程。从这一段内容里可以看出很多的东西,他们在研究海底墓穴里的壁画和瓷器,而且,看似研究非常的正统和系统,是经典的考古流程(那种查资料的过程看似十分的枯燥,却是考古工作者日常研究的主要方式,考古,就是挖——修——查)。但是当年三叔他们去的时候,根本不可能有这种条件,汪藏海墓那么大,就这么几个人,要工作多久才能把墓穴里的东西全部记录下来啊?那么他们是在什么时候干的这个工作呢?

这是一个很大的线索,不过我没有工夫细想,就继续往下看去。当时我以为后面会继续这样的过程,然而,在1991年3月6日这一段之后,就出现了让我疑惑的一个现象。

从这一段之后,大概有六页的内容,都是收集资料的陈述,这略过去。一直翻过去后,下一段的工作日记,时间却跳到了1993年的1月19日。

然后,再看其中记录的内容,你就会发现,和前面的有了相当大的不同。这一段的内容,是1993年1月19日,一直到1995年2月8日,时间跨度比上一段长,然而,能记录下来的东西并不多。内容如下:

1993年的1月19日,经过了上次的讨论,汪藏海的事情清晰了起来。看来,他前往塔木陀,确实和皇帝祭长白山有关系,他应该重返了云顶天宫,之后,就起程前往塔木陀。这个塔木陀必然和长白山里的情况有关。

1993年4月18日,从壁画中我们整理出了前往云顶天宫的三条路线,我们决定前往长白山,一探究竟。

1993年5月30日,进入长白山的范围内,天气很糟糕。

(之后,大约有十几页都是探险小说一样的行进记录。和我们进入云顶的内容类似。一直到进入之后。)

1993年6月15日,和他们失去了联系,我们两个人继续前进。

1993年6月17日,我们到达了天宫的底部,情况十分的糟糕,其他人可能凶多吉少,我们也没有时间犹豫了,我们决定进入青铜门,看看里面到底是什么地方。

1993年6月18日,看来,我看到了终极!

(到这里之后就中断了,没有任何的内容,下一次日记就是最后一段。显然她将近一年半的时间没有记录东西。)

最后一条记录就是,1995年2月8日,我们开始策划寻找塔木陀,这一切到底是怎么回事,我一定要弄清楚。

这就是第二部分,到了这里大概一共是三十页的内容,非常明显的,第一部分和第二部分之间好几年的内容空缺了。到这第二部分,直接他们就开始了云顶天宫的旅程,看到这里我心里的一个疑问就清楚了,看来云顶天宫里,死在黄金堆里的人,应该就是他们这一批人,而且看他们携带的东西,这里文锦说的我们,应该就是西沙的那一批人了。

如此说来,他们好像没有什么其他特别的窘况,而且活得似乎还很舒坦?不过这些倒是次要的了,最让我震惊是,显然文锦也发现了那道青铜门,并且她也进去过了。

“我看到了终极!”我看到这里就出了一身的冷汗,心说这是什么意思呢,这个终极是代表着什么?

看记录的时间,她进去之后,几乎一年半没有任何的日记,这和她的性格不符合。我感觉她非常有可能是在青铜门里看到了什么,以至于太过震惊,无暇再去想什么笔记。

纵观第二部分,给我最大的感觉就是,那个塔木陀,应该和青铜门有着莫大的关系,文锦进入了青铜门之后,才萌发了前往寻找这个“塔木陀”的想法。

再之后,是第三部分,这部分十分的长,但是时间跨度很短,是1995年2月8日到1995年6月8日,其中,值得提出来看的只有一段。

1995年2月8日

根据那张龙脉图,我们已经可以确定塔木陀的位置,我们将要进行一次勘探,希望在这次勘探中,能够发现那一系列谜题的答案。说实话,我实在没有想到,这背后有这么多的事情,如果我在青铜门里看到的东西是真的,那这整件事情就太可怕了。

之后,就完全都是他们前往那个叫“塔木陀”的内容,看上面的描写,这个“塔木陀”应该是戈壁中的一个绿洲,文锦跟着一只驼队,在1995年的年初自敦煌出发,深入了柴达木盆地,进行了这一次的旅程。

他们由一个叫做定主卓玛的女向导带领进入了戈壁,然后在一处岩山,他们和她分手,进入了这个叫做“塔木陀”的地方。那个绿洲之中似乎非常的危险,一路上有不少人死去,还看到笔记的路线图上,标着很多危险的记号。最后他们到达了塔木陀,不过,她和另一个人产生了分歧,最后她没有继续前进就回来了。

我是非常快速地看了这一遍,并没有细看,这些内容之后,就是一片空白,没有内容了。这部分内容大概有三十页,非常的详细,有大量的路线图以及关于装备的缺损、天气之类的描述。

整本笔记里,根本没有写他们是怎么得到信息,或者如何调查的,也没有提到任何关于西沙他们失踪的内容,也没有提到这个疗养院里的事情。里面所有的信息,都是和这个“塔木陀”有关系,几乎有一半的篇幅,都是对于汪藏海铁面生留下的东西的分析,并且从中发现了指向“塔木陀”的关联。而且,让我感觉很在意的是,这三段内容,中间都有明显的断裂,感觉上,笔记好像是被人装订过的或者重抄过的。

我用力扯起页与页之间的缝隙,发现没有任何重新装订和撕页的痕迹,这是一本完整的笔记。那即是说,这本笔记可能是文锦重抄的一本。她似乎是挑选了几本笔记中关于塔木陀的内容,抄了下来,将其会聚在一本笔记里。

她为什么要这么做呢?这又是一件匪夷所思的事情,这帮人做事情,为什么总是这么神神秘秘的?难道,笔记的其他内容中,有她不想让别人知道的东西?

而且,看着这笔记,很明显的一个感觉,好像就是想让我知道“塔木陀”这个地方很关键,似乎是想让我去那个地方一样。

心里的疑问多不胜数,一下子也理不出个头绪来,我揉了揉太阳穴,把笔记翻到开头,准备仔细地从头看起,看看仔细地推敲,是否还能得到一些什么线索。然而这时候,眼前的打火机已经暗淡了下来。火苗已经萎缩了下去,光线相当的昏暗。

我想起打火机已经用了相当长的时间,可能马上就要断气了,于是就想将那些报纸连同抽屉来点燃,做一个篝火堆,这样不至于一会儿打火机打不起来,自己要摸黑。于是拿着打火机站了起来,舒展了一下筋骨。

就在这时候,我就感觉哪里有点不太对劲,这里好像有什么地方不一样了,我干脆举高打火机,想看看是不是错觉。这不看还好,一看几乎没把我吓死,只见桌子的对面,不知道什么时候竟然出现了一个“人”,这个人坐着我刚才坐的椅子上,看着那面镜子,正在梳头。

\chapter{黑暗}

这个“人”身材怪异,虽然打火机的光线很暗淡,只能照出一个灰色的轮廓,样貌看不完整,但我还是能看到它的脖子长得有点奇怪,那种感觉,说夸张点,让我觉得它不用站起来,就能把脸探到我面前来。

它坐在我刚才坐的那张椅子上,两只细长的手臂在头侧滑动,动作诡异异常。我愣了一下,才意识到它是在梳头,当即整个人就凉了,浑身的毛孔都发了起来。

在这样一间荒废了十几年的地下室里,突然看到一个人在黑暗里梳头,这种举动,加上这种场合,普通人恐怕能当场被吓死。

我一边冒冷汗,一边就奇怪,这是什么人?什么时候出现的?从我发现笔记本,到坐下来看,最多也只有二十分钟时间,狗日的它是什么时候坐到我对面去的?我怎么一点也没有察觉到……而且这里是一座废弃建筑隐秘的地下室,怎么可能会有其他人在这里?

加上这诡异的动作,坐在那张椅子上,看着霍玲的那面镜子,竟然在梳头,不能让我不想,难道霍玲没和其他人一起走……这个“人”是霍玲?

我的冷汗像瀑布一样下来,好在我的神经已经今非昔比了,虽然无法理解眼前发生的事情,我的身体还是不由自主地做出了应对反应。我条件反射地退后了好几步,眼睛盯住对方,进行全神戒备。

如果在电视剧里,看我这样惊慌的样子,这个躲在黑暗里的人肯定会哈哈哈笑三声,然后导演给一个特写,或者掏出一把小手枪,说一句:“没想到吧,邦德邪先生。”可是这不是电视剧,随着我的后退,那人纹丝不动,还是照样做着机械的梳头动作,随着我几步的远离,摇摆不定的打火机越发暗淡了,距离也远了,那人就隐入了黑暗里,几乎看不见了。

直退了五六步,我感觉到有了点安全感,就停住了脚步,鼓起勇气问了一声:“你是谁?”

我到了地下室之后,几乎没有说过话,如今这话说出来,声音嘶哑,几乎都不像是我的声音,听着自己都吓了一跳,不过在这安静得连针掉在地上都听到的地下室里,这嘶哑的声音十分的通透。

然而,我问了之后,对方没有反应,从那写字台后面没有传来任何的声音。好像我在和空气说话一样。

狗日的,想吓唬我吗?我暗骂了一声,真的有点害怕起来,想想刚才看到那人奇怪的体态,心说这东西该不会不是人吧?

不可能不可能,我否定自己,要说在古墓里还有可能,但是这里是现代建筑啊,不会有这种东西出来,这里又没有棺材……等等,等等,不对啊!我操,这里有棺材啊。

我的脑子嗡的一声,心说难道这东西是那时棺材里的粽子?

我忙摇头,努力喘了几口气,让自己平静了下来。

这也是不可能的,哪有碰到棺材就出粽子的道理,要真这样,殡仪馆里的人都得去茅山考个本科回来才行。

这时候,我脑子里就突然闪过一个念头,该不是这人就是寄录像带给我的人?在这里等我?

从刚才看到的笔记本来看,安排寄录像带的人就是文锦。但是,事实上也不能确定寄录像带的就是她本人,有可能是她安排的其他人。

想着我就感觉到很有可能,这种地下室里不可能会有普通人知道,能进来的肯定是知情人,可能是一直在附近等我的寄信人,看我爬进来就跟我进来了。这样想着我稍微平静了一点。我鼓起勇气,心说要是活人就不怕了,于是皱起眉头,把打火机往前伸过去,看看到底是谁。

小心翼翼地往前探了两三步,写字台对面的情形我又可以隐约看见了,可我一看,又吓了一跳。坐在那里的“人”,不见了。

我眯起眼睛,仔细去看,确实不见了,座位上没人,我心里疑惑起来,心说难道刚才自己看错了?错觉?

不可能,那冷汗出的,绝对不可能看错,我顿时就紧张起来,忙举高打火机,朝四周照去。

可就在举起的时候,动作太大,打火机突地亮了一下,然后就熄灭了。

四周立即漆黑一片,伸手不见五指,这里一点光线也没有,是属于绝对的黑暗,顿时我心就揪了起来,也不顾烫得要命的打火机头,忙甩了几下就再去打火。

然而打了摇,摇了继续打,这东西就是不争气,怎么摇也打不起来,只看到火星四溅,在绝对黑暗的地下室分外的耀眼,我意识到可能没气了。

我心说要命了,看了看四周伸手不见五指的黑暗,极度不祥的预感涌了上来。我将笔记放入口袋,正准备往后退几步去摸进来的门口,突然就听到头顶上“咕叽”了一声,好像有一个女人在笑。

\chapter{惊变}

平时我并不抽烟,只有在十分郁闷的时候才会抽几口,所以这打火机买来我也没加过气,这时候突然熄灭,让我大惊失色,在这种地方,没有照明,那是太恐怖的事情了。

正琢磨着该怎么办,这时候就听头顶上“咕叽”了一声,好像有一个女人在笑。

一下我后脖子就凉了,这地下室极矮,房顶我抬手跳起来就能摸到,虽然什么都看不见,我还是条件反射地把头抬了起来往上看。

这一抬,什么也没看见,却感觉到一团毛茸茸的东西垂到了我的脸上。我随手一抓,心里一愣,发现那竟然是一团头发,而且还是湿的,黏糊糊的。

自从海底墓之后,我对湿头发极度地抗拒,这一下我就觉得喉咙里发毛,好比吞了只耗子,赶紧矮下身子,挥动袖子把脸上那种东西全擦掉。同时人就直往边上退去。抬头死命地瞪着那黑暗的房顶。

太黑了,我完全想象不到这种黑,我心里的恐惧一下子就涌了上来,心说这是怎么回事,房顶上有个女人?难道是刚才那人现在吊在房顶上?我靠,这怎么可能,难道它是四脚蛇?

事情越来越不对劲了,摸着手里黏黏的东西,闻了一下,就闻到一股奇怪的味道,一下想不起在哪里闻到过,但是条件反射般,我心中出现一个相当不祥的感觉。

这时候,那“叽咕”的笑声又响了一声,听着感觉就是在房顶上朝我过来了。我马上又退后了几步,“哐当”一下就撞到那写字台上,在安静的地下室里听起来像打雷一样,把我自己吓得一身冷汗。

我站稳身子,再听那声音就没了。我越来越紧张,那不是普通的紧张,不知道为什么,我浑身竟然开始发起抖来,好像是潜意识已经预感到要发生什么极端可怕的事情,接着,突然我就感觉到后脖子发痒,好像有什么东西在我的脑后垂了下来。

我捏着打火机,再也忍不住了,几乎是战栗地转过头,用力滑动了火石。

啪一声火星飞起,极短的时间内,那白光就照出我背后的情形,只见一大团头发从房顶上垂在我的身后。我抬头再滑动火石,就看到头发的里面,一张惨白狰狞的脸孔,正冷冷地对着我。

火星的光芒稍纵即逝,眼前又是一片黑暗,然而那情形已经清晰地印在我的脑海里。

禁婆!顿时我就知道我的身体为什么会有这种反应了。狗日的,这里有一只禁婆!

我脑子突然一片空白,什么冷静全没了。我怪叫了一声,就往后狂奔,什么也不管了,直朝黑暗里冲去,脑子里只剩下一个念头,就想逃离这个地方。

没跑多少步,实实在在的,我就整个儿撞在了墙上,那一下撞的,就是撞墙自杀的那种撞法。“砰”一声,我就翻倒在地,爬起来就听到头顶上一连串“叮当叮当”的声音,直奔我就来了,也不管自己满鼻子的血,爬起来感觉着刚才进来的那个门洞,再次冲了过去。

这次学乖了,我把手伸在前面,一路摸着冲了出去,凭着记忆冲进了走廊,然后扶着墙冲到出口撞出门,回头就把门死死地关上,然后冲进黑暗里,胡乱摸着,想找到下来的楼梯口。

但是在如此黑暗的地方,想找到那个门洞实在太困难了,我摸了半天,连墙壁都没有摸到。摸着摸着,我突然撞在什么东西上,几乎摔倒,我往前扑了一下,趴了上去,一下就知道我踢在那个石棺上了。

撑着石棺我想重新站起来,然而手在石棺上乱摸,我突然就感觉到不对,石棺的形状好像变了。我再摸了一下,马上意识到,原来石棺椁的盖子,竟然被人挪开了一条缝。我的手就摸在缝口子上。

石棺怎么开了?那一刹那我脑子里闪过这个疑问,可是此时脑子里已经混乱得一塌糊涂了,只觉得一阵晕眩,也无暇顾及这个问题了,只闪了一下我就站起来,继续往前摸去。

就在这个时候,突然边上有什么东西动了一下,我的神经已经到了极限,几乎被吓死,刚想拉开架势,就有一只手伸了过来,顿时我嘴巴就被人捂住了,身子也被人夹了起来,动弹不得。

我用力挣扎了几下,制住我的东西力气极大,我连一点都动不了,同时我就听到耳边有一个人轻声喝道:“别动!”

我一听,整个人一惊,立即停止了挣扎,心里几乎炸了起来。

虽然只有两个字,但我还是马上听了出来他是谁!

这竟然啊是闷油瓶的声音。

\chapter{重逢}

我认出声音的那一刹那,我本该有无数的反应,疑惑、愤怒、惊讶、怀疑、恐惧,等等,但是事实上我的大脑就一片空白。

在这里听到他的声音,实在是出乎了我的意料,在我的想法中,闷油瓶现在可能在世界上的任何一个地方,甚至不在这个世界上,但是他万万没有理由出现在这里。

的确!他怎么会在这里?他在这里干什么?

难道寄录像带的人,真的是他?他躲在这里?

还是和我一样,他也是因为什么线索追查而来的?

大脑空白之后,无数的疑问犹如潮水一般涌了上来,我一下子就无法思考了,我的脑海里同时又浮现出了他走入青铜门的情景。一股冲动顿时上来,我真想马上揪住他,掐住他的脖子问个清楚,这小子到底在搞什么鬼。

然而现实却是他捂着我的嘴,黑暗中,我一点呻吟也发不出来,动也不能动,而且我明显感觉到他的力气一直在持续着,他根本就没打算放手,而是想一直这么制着我。这让我很不舒服,我又用力挣扎了一下,他压得更紧,我几乎喘不过气来。

这时候我就听到,刚才被我关上的那道木门,发出了十分刺耳的吱呀一声,给什么东西顶开了。

那东西出来了,我深吸了一口气,立即就安静了下来,屏住呼吸,不再挣扎,用力去感觉黑暗中的异动。

一下子,整个房间安静到了极点,没有了我自己声音的干扰,我马上就听到了更多的声音,那是极度轻微的呼吸声,几乎是在我的脑袋边上。

这是闷油瓶的呼吸声,他娘的他是活的,当时看到他走进门里去,我还以为他死定了,走进地狱里去了。

闷油瓶大概感觉到了我的安静,按着我的手稍微松了松,但是仍旧没有放手的意思。四周很快就安静得连我自己的心跳都能听到了。

就这样好比石膏一样,也不知道僵持了多久,我就听到了一声非常古怪的“噗噗”声,从门的方向传了过来。

又隔了一会儿,什么声音也听不到了,捂住我的嘴的手才完全松了开来,突然间我的眼睛一花,一只火折子被点燃了。

我花了很长时间才适应过来,眯起眼睛一看,那张熟悉的脸孔终于清晰地出现在了我的面前。

闷油瓶和他在几个月前消失的时候几乎没有区别,唯一的不同就是脸上竟然长了胡楂,我感觉到十分意外,再仔细一看才发现那不是胡楂,那些都是黏在脸上的灰尘。

我脑子完全僵掉了,此时就傻傻看着他,之前想过的那些问题全忘记了,一时之间没话讲。而他似乎对我毫不在意,只是淡谈地看了我一眼,什么也没问,就小心翼翼地毛腰到了那门边,用火折子照了照门的里面,接着竟然把门关上了。

关上门之后,他直接站了起来,举起火折子照着天花板,开始寻找什么东西。我心里火大,几次想冲出几句话来,都被他用手势阻止了。

他那种动作的力度,十分的迅速,让我感觉时间紧迫,而他的行为又把我搞得莫名其妙,视线也跟着他的火光一路看了过去。

火折子的光线不大,但是在这样的黑暗中,加上自己的联想很快就能明白这屋子的状况。

进来时候没有注意地下室的顶,抬头看就发现上面全是管道,这和现在的车库一样,这些管道都涂着一层发白的漆灰,可以看得出这里翻新过好几次了,漆里还有着老漆。房顶是白浆刷的,砖外的浆面已经剥落得差不多了,露出了一段一段的砖面,看样子,那禁婆就是顺着这东西在爬。

可是,这里怎么可能会有这种东西,这他娘的唱的是哪出啊。

闷游瓶看了一圈,看得很仔细,但是动作很快,中途火折子就熄灭了,他又迅速点燃了一个,确实没有什么东西藏着了,接着他就回到了我的面前。

“没跟出来。”他看着那门轻声道。

我所有的问题几乎要从我的嘴巴里爆炸出来了,然而没想到的是,他一转头看向我,就做了个尽量小声的动作,接着轻描淡写地问了一句:“你来这里干什么?”

我一下子脑子就充血了,顿时想跳起来掐死他,心说你爷爷的龟毛棒槌,你问我,老子还没问你呢!是我自己想来吗?要不是那些录像带,老子打死都不会来这里!

我咬牙很想爆粗,但是看着他的面孔,我又没法像和胖子在一起一样那么放得开,这粗话爆不出来,几乎搞得我内伤。我咬牙忍了很久,才回答道:“说来话长了,你……怎么在这里?这到底是什么地方?你你你……那个时候,不是进那个门了吗?这里他娘的是怎么回事?”

这些问题实在是很难提出来,我脑子里已经乱成一团,也不知道怎么说才能把这些问题理顺。

“说来话长。”闷油瓶不知道是根本不想回答,还是逃避,我问问题的时候,他的注意力投向了那只巨大的石棺椁。我看了一下,确实石棺椁的盖子被推开了,露出了一个很大的缝隙,但是里面漆黑一片,不知道有什么。

我最怕他这个样子,记得以前所有的关键问题,我只要问出来,他几乎都是这个样子,我马上就想再问一遍。可是我嘴巴还没张,闷油瓶就对我摆了一下手,又让我不要说话,头往棺椁里看去。

这个动作我太熟悉了,虽然不知道发生了什么,我马上就条件反射地闭上了嘴巴,也凑过去看那棺里面。因为闷油瓶把火折子伸了过去,我一下看到了里面,棺椁里竟然是空的,我看到了干干净净的一个石棺底,似乎什么都没放过,而让人奇怪的是,那棺材的底下,竟然有一个洞口。

我正好奇,就听到了从那个洞里,传来一些轻微的声音,仔细一听,也听不出是什么。只等了一会儿,突然一只手就从洞里伸了出来,一个人犹如泥鳅一样从那个狭窄的洞口爬出来,然后一个翻身从棺材盖的缝隙中翻出,轻盈地落到我们面前。

我被吓了一跳,只见那人落地之后,擦了一下头上的冷汗,看了一眼闷油瓶,接着扬了扬手里的东西,轻声道:“到手。”

后者似乎就是在等这个时候,一把拍了一下我,轻声道:“我们走!”

我跟着他们,小心翼翼地踮起脚尖,蹑手蹑脚地顺着原路上去,然而才跨上两三级阶级,就听到身后走廊的门吱呀一声开了。

前面的那人就骂了一声,开始跑起来,我立即跟了上去,一路狂奔,连滚带爬地冲了出去,一直冲回院子翻过围墙,我们才松了口气。

我累得气喘吁吁,可那两个人根本没有停下来的意思,翻出去之后,就往外跑,竟然不管我。我心说这一次可不能让你跑了,忙追了上去。

又是没命地跑,一直跑出老城区,突然一辆依维柯从黑暗里冲了出来,车门马上打开,那两个人冲过去就跳了上去,那车根本就没打算等我,车门马上就要关,不知道是谁阻了一下,我才勉强也跳了上去。

上气不接下气,这跑得简直是天昏地暗,上车我就瘫了,在那里闭眼吸了好几口气,才缓过来。

立即我就四处看,一看就傻了,这车里竟然全是人,而且全部都用一种似笑非笑的表情看着我。而且最让我想不到的是,很多人我都认识。我一眼就看到了几张特别熟悉的面孔。

天,全是从天宫里幸存出来的那一批阿宁的队伍,这帮中外混合的人,我们在吉林一起混了很久。

看到我惊讶的表情,其中几个和我混得特别熟悉的人就笑了,一个高加索人用蹩脚的中文对我道:“超级吴(Super Wu,阿宁给我起的外号),有缘千里来相见。”接着,我就看到了阿宁的脑袋从一张坐椅后面探了出来,非常惊讶地看了我一眼。

我看着闷油瓶,又看了看刚才从石棺材里爬出来的人,那是一个带着墨镜的陌生青年,他们两个人气都没喘,也都看着我。突然我感觉到很乱,问他们道:“你们这帮驴蛋,谁能告诉我这究竟是怎么回事?”

阿宁就道:“这该我问你才对吧,你怎么会在地下室里面?”

依维柯一路飞奔,直接驶出了格尔木的市区,一下子就冲进了戈壁,而我在车内,车窗外一片黑暗,对此一无所知。

一路上,我和阿宁进行了一次长聊,把两边的事情都说了一下。

原来,阿宁也在录像带里发现了地址和钥匙,显然文锦的笔记上写的“三个人”中,有一个竟然是她。她发现了这个秘密之后,立即就分了两方面的工作,一方面让人到这里来寻找地址,一方面亲自到杭州来试探我。她想知道我到底知道不知道这录像带里的情况。

然而,她没有想到的是,我其实也收到了这样的带子,而且在她来找我之后,我就最快速度出发去了格尔木,甚至几乎和他们同时找到了那鬼楼。

(也亏得我这一次行动实在是快速和精准,没有过多的犹豫,否则,肯定我就看不到那本笔记了。想想我就后怕,不过同时我也有点开心,摸了摸在我口袋里的笔记,这是我第一次自己单独活动就取得如此大的成果,看来果然爷爷说的是对的,做事情真的是主动为好。)

之后,我又问阿宁闷油瓶是怎么回事,他们怎么会在一起。

阿宁就笑道:“怎么?你三叔请得起,我们就请不起了?这两位可是明码标价的,现在,他们是我们的顾问。”

说着那黑眼镜就咧开嘴笑,朝我摆了摆手。

“顾问?”说起顾问我就想起了胖子,心说阿宁这次学乖了,请了个靠谱的了,不过闷油瓶竟然会成阿宁的顾问,感觉很怪,我有点被背叛的感觉。

这时候,一边的高加索人说道:“你别听她胡说,这两位现在是我们的合作伙伴,是我们老板直接委派下来的,宁只是个副手了。现在主要行动都是由他们负责的,我们只负责情报和接应,这比较安全,老板说了,以后专业的事情就让专业人士去做。”

这应该是云顶死的人太多了,我想起当时的情形,就问道:“那这整件事情是怎么回事?录像带的内容,还有里面的禁婆,你们有眉目吗?”

这几个人都摇头,而且目光都投向了闷油瓶和黑眼镜,阿宁就瞪了他们一眼,之后朝我使了个眼色,道:“具体情况我们也不清楚,应该和你知道的差不多,我们现在都是按他们说的在行动,这两位朋友很难沟通。”

听完这些之后,我转向闷油瓶,此时已经按捺不住,我一定要找他问个清楚,让他告诉我这究竟是怎么一回事。

可是,还没等我做好准备,车里突然骚动了起来,藏族的司机叫了一声,所有人都开始拿自己的行李。

接着车子就慢慢地停了下来,车门被猛地打开,门外已经能看到晨曦的一缕阳光了,一股戈壁滩上寒冷的风猛地刮了进来。

我给挤下车,接着就看到了一幕让我目瞪口呆的情形,十几辆Land Rover一字排开停在戈壁上,大量的物资堆积在地上,篝火一个接一个,满眼全是穿着风衣的人,还有很多人躺在睡袋里,一边立着巨大的卫星天线和照明汽灯。

这里竟然好像是一个自驾游的车友集散地,但是仔细一看就知道不对,这里所有的车都是统一的涂装,车门上面都有一个旋转柔化的鹿角珊瑚标志,一看就知道是阿宁公司的产业。

看到我们下来,很多人都围了过来,阿宁不知道和他们说了一句什么,很多人欢呼了起来。

这个场面让我非常惊骇,我抓住一旁在和别人击掌庆贺的高加索人,问他这是干什么?

高加索人拍了拍我:“朋友,我们要去‘塔木陀’了。”

\chapter{营地}

我听了目瞪口呆,刚刚才看到文锦的笔记里提到这个地方,怎么他们也要去了。一下子我有点反应不过来,而且他们应该没有看过文锦的笔记啊,他们怎么知道这个地方的存在呢?

“怎么了?”那高加索人看我表情奇怪,就问我道,“脸色突然就白了。”

“没什么,刚才给吓的。”我马上掩饰了一下,装作很奇怪,一边跟着他走,一边就问他,“塔木陀是什么地方?你们去干什么?”

“塔木陀?这就说来话长了,”高加索人看了看前面走的阿宁,轻声对我道,“我待会儿和你说,我们先看看那两个小哥从里面带回来是什么东西。”

我看他给我打的眼神,似乎这些事情阿宁不让他说,于是也心领神会,不再出声。

营地里的人奔走相告,睡在睡袋里的人都被吵醒了,我们只能小心地在挪动的睡袋中穿行,跟着阿宁他们一路走。

整个营地很大,绕过路边的“路虎”集中地,后面还有一片帐篷,其中最大的一顶圆顶帐篷有四五米的直径,应该是当地人搭的,上面有藏文的标识,似乎是住的收费标准。阿宁带着我们走了进去,里面很暖和,我看到边上燃着带小烟囱的炭炉,地上有很厚的五颜六色的牛毛毯子,后来我知道这叫做“粗氆氇”,现在是相当昂贵的东西。此外还有很多的老式藏式木制家具,以及一些打包好没拆分的无纺布包。

整个帐篷非常的舒适,阿宁坐到了地毯上,进来一个藏人,似乎是帐篷的主人,给我们每人倒酥油茶,我也坐了下来,打量了一下这些人。

最让我恼火的就是闷油瓶,他坐在我的对面,看也不看我,靠在一大堆毛毡上,马上开始闭目养神。车上的人没有全来,而是来了一些我不认识的,这也让我相当的不自在。这些人里,我只认识一个乌老四和高加索人,其他都是陌生面孔。

这些人陆续坐定,阿宁就把刚才黑眼镜从鬼屋里带出来的东西放到了我们面前的矮脚桌上。

那是一只红木的扁平盒子,打开之后,里面是一只破损的青花瓷盘,瓷盘的左边,少了巴掌大的一块。

那只石头的棺材下面,肯定有一个空间,看样子这瓷盘本来是放在那个空间里的。这是什么东西,为什么闷油瓶他们会去偷这个?我不由也有点好奇。

我正要调整自己脖子的方向去看盘子,突然帐篷外又进来了两个人,那是一个满头白发的藏族老太婆和一个藏族的中年妇女。老太太犹如陈皮阿四一样干瘦干瘦的,大约也有七十多了,不过相当的精神,眼神犀利,那中年妇女倒是普通的藏族人样貌。她们两人一进来整个帐篷就突然气氛一变,除了黑眼镜和闷油瓶,其他人都不由自主地坐了坐正把身体转向她们,特别是老太太。有两个人还向她行了个礼,似乎这个藏族老太婆在这里有比较高的地位。

老太婆也回了个礼,并打量了一下我们,特别是我,可能是因为陌生,所以多看了几眼,便径直坐了下来。阿宁便恭敬地拿起了那只瓷盘递给她,问道:“嘛奶,您看看,您当年看到的是不是这个东西?”

说完后马上有人翻译成藏语,老太婆听着便接过了瓷盘看了起来,看了几眼她就不住地点头,并用藏语不停地说了什么。翻译的人开始把她的话翻译回来,几个人开始交谈了起来。

他们对话断断续续,而翻译的人不仅藏语的水平不是很高,更要命的是中文似乎也不行,磕磕巴巴的,我努力去听但是听不明白,就轻声问边上的乌老四,这老太婆是谁?

乌老四没有回答我,但是边上的黑眼镜却说话了。他低声对我说道:“她叫做定主卓玛,是文锦当年的向导。”

我听到这个名字,就“啊”了一声,一下子心里清楚了不少,心中也为阿宁公司的神通广大而惊讶,他们不仅知道塔木陀,而且还知道有这个向导,这么说,阿宁应该知道文锦的事情了?

我在文锦的笔记中了解过他们自敦煌出发,进入到柴达木腹地的经过,她的确提到过他们请了一个藏族女向导。我不由摸了摸口袋里的笔记本,心说怎么回事,难道还有人看过这本笔记吗?

不过,我记得笔记里文锦也说了,这个女向导并没有将他们带入到盆地很深,在过大柴旦进入到察尔汗区域之后,女向导也找不到路了,事实上也没有任何的路可以去找,最后他们在一座盐山的山口和向导分手,自己朝着更深的地方出发。柴达木盆地面积二十四万多平方公里,他们最后的旅程走了三个星期,最后走到哪里,谁也说不清楚。

看来,如果他们想去塔木陀,光是这个老太婆并不能给阿宁他们带来什么特别有用的帮助。最多能带他们到达和文锦队伍当年分手的地方。

我正想着,阿宁和定主卓玛的对话就结束了,行礼后中年妇女将老太太扶了出去,有几个听不懂的人就问怎么样,阿宁已经掩饰不住脸上的笑意,兴奋道:“没错了!她说就是这只盘子,陈文锦当年给她看的就是这一只,她说有了这只盘子,她可以带我们找到当年的山口。”

几个人都骚动起来,黑眼镜就问道:“什么时候出发?”

阿宁已经站了起来,对他们道:“今天,中午十二点,全部人出发。”说着其他人都站了起来,就要走出去。

这时候那个黑眼镜又道:“那他怎么办?”

说着就指着我。

阿宁他们转头看向我,似乎刚才忘了我在这里,几个人都错愕了一下,我就盯着阿宁,想看她会怎么说。

没想到阿宁并没有太过在意,想了想就指着一边闷油瓶,对黑眼镜道:“他带回来的,让他自己照顾他。”说着就带着人出去了。帐篷里只剩下了黑眼镜和闷油瓶两个人。

黑眼镜干笑了两声,也靠到了毛毡上,点起了烟,然后就在那里看着闷油瓶道:“我说你是自找麻烦吧。刚才不让他上车不就行了,你说现在怎么办?”

闷油瓶抬起了头,淡淡地看了我一眼,似乎也是很无奈地叹了口气,对我道:“你回去吧,这里没你的事了,不要再进那疗养院了,里面的东西太危险了。”

我看着他,心里十分的不悦。

说实话,我压根儿不想去那狗屁的地方,我也不知道阿宁他们为什么要去那个地方,我现在只想知道,闷油瓶在云顶到底做了什么,我看到的那恐怖的景象到底是怎么一回事。

于是我回答道:“要我回去也可以,我只想问你几个问题。”

闷油瓶还是淡淡地看着我,摇头道:“我的事情不是你能理解的,而且,有些事情,我也正在寻找答案。”说着也站了起来,头也不回地走出了帐篷。

我气得浑身发抖,几乎要吐血,看着他的背影真想冲上去掐死他。

那黑眼镜也叹了口气,就在边上拍了拍我,道:“这里有巴士,三个小时就到城里了,一路顺风。”

说完黑眼镜也走出了帐篷,帐篷中只剩下我一个人。场面一下子冷清了下来。

这让我很尴尬,有一种被小看,甚至被抛弃的感觉,十分的不舒服,刚才阿宁他们,闷油瓶和黑眼镜的态度,简直就是认为我是一个可有可无的人。这比辱骂或者恨意更加伤人。

但是黑眼镜的问题却是实实在在的。

想想也是,阿宁的队伍要出发了,我是他们从鬼楼中救出来的,这是一个突发事件,所以他们根本没准备什么措施安排我,也没有任何责任给我解释什么,我当然就应该自己回去。

但是,我实在是不甘心,看着帐篷外人来人往,准备工作热火朝天,我就感觉到血气在上涌。我想着我回去之后能干什么?寄东西的文锦早我一步走了,此人可以在二十年间躲藏得三叔用尽手段都找不到,我又如何去找?难道我要像三叔那样,为了一个谜题再找她三十年吗?不可能。

疗养院里发生的事情,扑朔迷离,却完全没有任何线索,文锦留下的笔记,却是一直在说着这个“塔木陀”。而现在,外面这批人就要出发去了,可是我却准备买票坐巴士回家。

整件事情唯一的线索,现在只剩下了我口袋里的笔记,而笔记中的内容,似乎一直在暗示我,要到塔木陀去,才能知道一些什么。

我应该怎么办呢?回到格尔木,我又能做什么呢,我什么都不能做了。

“做事情要主动。”

忽然我耳边响起了我爷爷的这句话,接着我就摸到了口袋里的笔记本,想着这一次在格尔木的经历,完全是因为我的快速而果断才占了先机。

好吧,我一下就打定了主意,他娘的闷油瓶,别嚣张,你能去得我吴邪也能去,这一次我也跟着去!我站了起来,走到外面正在准备行李的阿宁边上,问她:“你有没有多余的装备?”

阿宁正在点数自己的压缩饼干,听到我突然问她,露出了很诧异的表情:“多余的装备?你想干什么?”

我耸了耸肩,有点不知道怎么说出口:“我要加入,我要加入,我也要去塔木陀!”

“加你个头。”阿宁笑了,转过头不理我。然而我继续看着她,对她道:“我能帮到你们,想想在云顶天宫里。”

阿宁就抬起头,脸色变了,她看着我的眼睛,朝我微笑了一下:“你是认真的?”

我点头,她就指了指一边的装备车:“随便拿,十二点准时出发,过时不候。”

\chapter{出发}

吉普车队飞驰在一望无际的苍茫戈壁上,气候干燥,车子与车子离得很远,用以逃避上一辆车扬起的漫天黄尘。

我坐在车里,看着窗外,想着之前的决定,也不知道是不是正确,这时候感觉好像有点过于莽撞了。不过,现在上了贼船,也没有脸去反悔了。

阿宁的计划在出发前和我说了,我发现是完全按照当年文锦的路线,由敦煌出发,过大柴旦进入到察尔汗湖的区域,由那个地方离开公路,进入柴达木盆地的无人区。然后由定主卓玛带路,将队伍带到她和当年那支探险队分手的地方。

这条路线几乎和文锦在笔记中写的一模一样,我就十分的纳闷,她到底是哪里得来这些信息的?显然,她知道塔木陀,知道定主卓玛,知道路线,看上去好像她看过笔记一样,可是笔记在我的口袋里啊。

车队一路补充物资,很快便按照计划到达了敦煌。有人告诉我,到达察尔汗区域之前的路线,还是相当于自驾游的路线,相对安全。

一路上两边的雅丹地貌让我领略了戈壁的荒凉,这种一望无际天地尽头的感觉让人有强烈的被遗弃感。这种感觉刚开始还可以由路边很多已经是废墟的居民点缓解一下,但是到了离开敦煌,我们开上察尔汗公路,直接驶入戈壁滩之中后,就根本无法驱除。因为连续行驶十几个小时,而四周的景色几乎没有分别,这种感觉是令人窒息的。也亏得阿宁队伍庞大,扎营时的喧嚣多少让我们心里舒服一点。

我是和高加索人一个车,他和另外一个藏人司机轮番开车,在路上,我就问他这些问题,看他能不能回答。

高加索人却很轻松地回答了出来,一听我才发现原来我想得太复杂了。我总是认为应该是看了笔记,然后知道塔木陀、定主卓玛和路线,其实完全不是这样。阿宁收到录像带采取的第一个措施,就是去调查了寄快递的快递公司,通过快递公司人的回忆,他们就找到了这个快递的寄出者,那个人就是定主卓玛。

之后一探访,拿着快递一问,这些塔木陀、向导、路线就都被问出来了。现在的计划,都是按照定主卓玛的信息来做的。

听了我才释然,这样说起来,文锦的笔记第三部分前半段的内容是不重要的,重要的是他们和定主卓玛分手到进入塔木陀的那一段,可惜那一段我没仔细看,一定要找个机会偷偷再看一遍。

接着,高加索人又和我讲了他知道的塔木陀的事情。

高加索人告诉我,塔木陀这个概念是找到定主卓玛才知道的,根据定主卓玛听当时文锦他们对话的记忆,似乎是汪藏海的最后一站,至于是什么地方,文锦他们自己也不知道,只是去寻找。

不过,定主卓玛后来根据旅途里见闻和经历,就有了自己的判断。她发现文锦他们在寻找的这个塔木陀,就是他们这一带传说里的西王母国。在当地人的说法里,那个应该叫做塔耳木斯多,意思是雨中的鬼城,当时她发现了这一点之后,就很害怕,于是假装找不到路,和他们分手了。

“西王母国?”我听了很吃惊,“那不是神话里的东西吗?”

“其实不是,西王母国是真实存在的,而且是历史很悠久的古国,黄帝时期就有传说了,西王母就是国家的女王,青海湖在羌语里叫做‘赤雪甲姆’,甲姆就是王母的意思,我们认为它就是王母的瑶池,而塔耳木斯多,就是王母之国的都城。西王母在西域传说中代表着神圣的力量,在定主卓玛小时候听的传说中,这座城市只有在大雨的时候才会出现,一旦看见就会被夺取眼睛,变成瞎子,所以她非常的害怕。”

“那你的意思就是说,我们现在要找的,其实就是西王母国的古都?”

“可以这么说,根据现在的考古资料分析,特别是近几年的,西王母的存在已经被证实。”高加索人说,“事实上,如果塔木陀是在柴达木盆地里,那它肯定就是西王母国的一部分。这一次说是去寻找塔木陀,其实就是去寻找西王母国的遗存,你要知道的就是,不是我们去寻找西王母国,而是我们找到的东西,自动就会成为西王母国,这就是考古探险。”

我听了就苦笑,西王母?我记得那玩意不是什么好惹的货色啊。汪藏海最后出使的是西王母?这说得通吗?

想了想,就想到后羿求不死药的传说了,心说难不成汪藏海那次也是去求药?感觉非常离谱,就摇头甩掉这个念头,不去思考。

之后我就在车上点算从阿宁那里拿来的装备,他们公司有特制的衣服,我的衣服在戈壁里行进白天会晒死晚上会冻死,所以我在车上换了沙漠服。我穿的时候就很意外,发现这衣服的皮带上,竟然也有02200059的号码。

我问高加索人这是什么号码,他说是他们公司的条形码号,他们老板很着迷这个数字,据说也是一份战国帛书上翻译出来的。

我心中十分的诧异,想起七星鲁王盒子上的密码,心说这数字是不是有什么特别的意义?

之后的两天,我们向戈壁深处渗入,“路虎”的速度非常快,这两天时间,我们就进入了柴达木的腹地。

阿宁的人很不见外,几次扎营,当初一起在吉林的几个人和我都相处得很好,其他人也和我熟悉了起来,我这样的性格,和别人相处是相当容易的。这样一来,至少有一个好处,我不用整天面对着面无表情的闷油瓶。而他也似乎根本不想理会我。

这其实有点反常,因为在之前的接触中,闷油瓶虽然同样不好相处,但是并没有这一次这么疏远的感觉,我总感觉他是在避讳什么。反倒是那个黑眼镜,似乎对我很有兴趣,老是来找我说话。

车子进入到戈壁后,很快离开了公路,定主卓玛就开始带路,她是由她的媳妇和一个孙子陪同的,和阿宁在一辆车子里,在车队的最前方。我并不知道他们的情况,只知道那老太婆开始带路之后,车子走的地方就开始难走起来,不是碎石滩就是河川峡谷的干旱河床,很快队伍就怨声载道。

定主卓玛解释说,要找到她当年看到的山口,必须先要找到一个村子,他们当年的旅行,是从那个村子开始的,文锦的马匹和骆驼都是在村中买的。现在这个村子可能已经荒废了,但是遗址应该还在,找到它才能进行下一步。

老太婆的记忆力还是相当的好,果然在傍晚的时候,我们来到了那个叫做“兰错”的小村,村里竟然还有人住,有四户人家三十几号人。

这个发现让我们欣喜若狂,一是证明了老太婆的能力,二是事情发展顺利,而且长期在戈壁中行进,看到人类集聚的地方,总是特别开心的。当时天色已晚,我们就决定在村里扎营地。

可惜的是,进村的时候出了一起事故,一辆车翻进了一道风蚀沟里,人没事,但是车报废了,此时我们离最近的公路已经有相当远的距离,不可能得到任何的援助。这就意味着必须有另一辆车也留下来照应。

这件事情出了之后,阿宁就开始显得心事重重。当天晚上我们在报废的车子边上休息,阿宁就对我们说出了她的担心。她有点顾虑,虽然配备的是一流的越野车,但是四周的条件实在是太恶劣了,如果无法在短期内找到山口,这些车子肯定会一辆一辆地报废在这里,有时候可能是在修车厂里非常小的问题,但是在这里都会让车子瘫痪。

而他们进入盆地的深处越远,被遗弃的车子和随车的人可能无法及时地得到救援而在戈壁遇到危险。

车子和骆驼马匹到底是不一样的,骆驼受了伤会自己痊愈,小伤也不影响行进,但是高科技下的车子,只要出了事故,就脆弱得让人伤心,这些到底是民用车,没有军用的结实。

但是这也不是阿宁的失策,因为现在这种时代下,不可能让这一支近五十人的队伍骑着骆驼进入柴达木,一是无法在一时间找到这么多的骆驼,五十人,加上驼运行李的和备用的骆驼,可能需要将近一百峰,如此巨大的驼队实在是太显眼了,肯定会被政府注意到。

随队的机械师对她说其实也不用这么杞人忧天,柴达木盆地在“路虎”的速度下并不是什么太大的地方,在二十年前柴达木可能还是和塔克拉玛干沙漠一样是人见人畏的死亡之海,现在却是随便花十几个小时就能穿越半个开发区域,其中有大量的勘探基地、工业基地,所以并不需要这么担心。

不过这话立即就被定主卓玛的孙子否决了,这个叫做扎西的小伙子说我们太信任机器的力量了,柴达木虽然已经被征服,但是安全的地方只限于公路网辐射得到的地方,大约只占整个盆地的百分之二,其他百分之九十八的区域全是沙漠、沼泽、盐盖,我们这十几辆车五十号不到的人,对于这片几千万年前就在吞噬生命的土地来说是微不足道的。

他说,就算是沿着设计好的最不危险的旅游线路,每年也都有人走失和遇到事故死亡,不要说我们现在准备深入无人区。

他还说,他以前见到的人,都是以穿越盆地为目的的旅行者,这些人在盆地中不会逗留超过两天时间,而我们的目的是在盆地中搜索。那就是说,我们的旅途是没有尽头的,这样在戈壁中绕圈子,是以前这里牧人最大的忌讳,所以,宁小姐的担心不无道理,凡事还是小心一点好。

扎西的话让我们陷入了沉默,阿宁想了很久,问扎西道:那你有什么建议给我们?

扎西摇头说:你们既然要进入柴达木,那么,人头肯定是要别在裤腰带上的,自古以来就是这样。

扎西的说法,总归有点危言耸听的感觉,在之前我听别人说过,扎西对于祖母答应给我们带路十分的愤怒,他认为这件事情太过危险了,阿宁他们还用金钱来说服他的祖母,是一种业障,我们给他的祖母带来危险和罪孽。但是定主卓玛那老太太却很坚决,藏族家庭中祖母的地位十分的高,扎西也没有办法,只好跟来照顾。所以他一路上基本上没给我们什么好脸色,也没说什么好话。

虽然如此,但在这戈壁上只有几间土坯矮房的村落,吹着夜晚戈壁凛冽的冷风,看着搐动的篝火,再想想我们现在离文明世界的距离,我还是感觉到一股不寒而栗。

他说完之后我们就没兴致再说话了,几个人沉默着在篝火边上坐了很久,就各自进自己的睡袋休息。我们明天一早就要出发,阿宁没有支起帐篷,都是露天睡袋,这里晚上的气温有时候会达到零下,所以我们都躲在高起的地垄后面,靠近篝火取暖。

躺在那里,我却感觉到很多人都睡不着,四周是风声带过来的窃窃私语声。也难怪,这里可能是进入柴达木之前地图上有标示的最后一个地方,这种活动的老手自然不在乎,但是队伍中有很大一部分都是在当地请的人,在这种时候当然会兴奋一点。

我也不知道自己是老手还是新手,只是抬眼看天,发现这里的天空离地面近得多,群星也清晰得多,我在南方,成年后就很久没有看到过漫天繁星的场面。现在看到天空中璀璨的银河如此清晰,不由得也没有了睡意。

不过,长途的奔波总是起作用的,闹腾了一阵子,四周的声音便逐渐地安静了下来。

阿宁他们是安排了人守夜的,因为人多,这些疲劳的活主要是由当地雇来的人担任,所以不会轮换到我们。不过因为这里还是村落,所以不需要太过警戒,扎西也说了,只有在靠近可可西里的地方可能会出现大型的野兽,这里的草少得连老鼠都不来,不要说食肉野兽了,所以我也没有听到守夜人聊天的声音,估计也可能是睡着了。我在风声中隐约听到几声动物的叫声,不过也没有太在意,我们睡在整个营地的最中间,要被吃掉,也轮不到我们。

我一边想着事情,一边看着夜空,也不知道过了多久,就在我也昏昏欲睡的时候,蒙蒙笼笼的,忽然感觉有人走到了我的面前,我打了个哆嗦,清醒了一看,竟然是扎西。

我被他吓了一跳,忙坐了起来,想说话,他蹲下来压住了我的嘴巴,轻声道:“别说话,跟我来,我奶奶要见你。”

\chapter{文锦的口信}

定主卓玛要见我?

我看着扎西,有点莫名其妙,因为我和那个老太太从来没有说过话,也没有任何的交流,甚至我都不是经常见到她,她怎么突然要见我?

但是扎西的表情很严肃,有一种不容辩驳的气势,似乎是他奶奶要见的人不见就是死罪一样,他见我有点奇怪,就又轻声说道:“请务必跟我来,是一件很重要的事情。”

我楞了一下,看着他的表情,感觉无法拒绝,只好点了点头爬了起来。他马上转身,让我跟着他走。

定主卓玛的休息地离我们的地方很远,中间隔了停放的车子,大概是嫌我们太喧嚣了。我走了大概两百米,才来达他们的篝火边上,我看到定主卓玛和她的儿媳都没有睡觉,她们坐在篝火边上,地上铺着厚厚的毛毡,篝火烧的很旺,除了她们两个之外,在篝火边的毛毡上还坐着一个人。我走近看时候,更吃了一惊,原来那一个人不是别人,正是闷油瓶。

闷油瓶背对着我,我看不到表情,但是闪烁的火光下我发现定主卓玛的表情有点阴鹜。我一头雾水的走到篝火边上,心说这真是奇了怪了,这个老太太大半夜的,偷偷找我们来做什么呢?

扎西摆手请我坐下,那老太婆的儿媳便送上酥油茶给我,我道谢接了过来,看了一眼边上的闷油瓶,发现他也看了我一眼,眼神中似乎也有一丝意外。

随后扎西看了看我们身后营地的方向,用藏语和定主卓玛轻声说了什么,老太婆点了点头,突然开口就用口音十分重的普通话对我们道:“我这里有一封口信,给你们两个。”

我和闷油瓶都不说话,其实我有点莫名其妙,心说会是谁的口信?不过闷油瓶一点表情也没有的低头喝茶,我感觉不好去问,听着就是了。

定主卓玛看了我们一眼,又道:“让我传这个口信的人,叫做陈文锦,相信你们都应该认识,她让我给你们传一句话。”

我一听,人就愣住了,刚开始还以为自己听错了,刚想发问,定主卓玛就接下去道:“陈文锦在让我寄录像的时候,就已经预料到了,会有这种情况发生,如果你们按照笔记上的内容进来找塔木陀了,那么,她让我告诉你们,她会在目的地等你们一段时间,不过,”扎西把手表移到定主卓玛的面前,她看了一眼,“你们的时间不多了,从现在算起,如果十天内她等不到你们,她就会自己进去了,你们抓紧吧。”

我就蒙了,心说这是怎么回事?目的地?文锦在塔木陀等我们?这……一下脑子就僵了,看向闷油瓶,这一看不得了,闷油瓶也是一脸惊讶的神色。

不过只有几秒钟的工夫,他就恢复了正常。他抬起头看向定主卓玛,问道:“她是在什么时候和你说这些的?”

定主卓玛冷冷道:“我只传口信,其他的,一概不知道,你们也不要问,这里,人多耳杂。”说着,我们全部条件反射的看了看营地的方向。

闷油瓶微微皱了皱眉头,又问道:“她还好吗?”

定主卓玛就怪笑了一下:“如果你赶得及,你就会知道了。”说着,挥了挥手,她边上的媳妇就扶着她站了起来,往她的帐篷走去,看样子,竟然就是要回去了。

我站起来想拦住她,却被扎西拦住了,他摇了摇头,表示没用了。

不过这时候,定主卓玛却自己转过头来,对我们道:“对了,还有一句话,我忘记转达了。”

我们都抬起头看着她,她就道:“她还让我告诉你们,它,就在你们中间,你们要小心。”

说完,她继续转身,进了自己的帐篷里,留下我和闷油瓶两个人,傻傻的坐在篝火前面。

我看向闷油瓶,他却看着火,不知道在想什么。我就问他道:“这究竟是怎么回事?为什么这口信会传给我们两个?”

他却不回答,闭了闭眼睛,就想站起来。

我看他这种态度,一下子无数的问题冲上脑子,人就有点失控,一下把他按住,对他道:“你不准走!”

他转头淡淡的看了我一眼,还真的就没有走,坐了下来,看着我。

他这行为很反常,我还以为他会扬长而去,一下我自己也愣了,不知道说什么好。他看着我,问我道:“你有什么事情?”

我一听就心中火大,道:“我有事情要问你,你不能再逃避,你一定要告诉我。”

他把脸转回去,看了看火,说道:“我不会回答的。”

我一下就怒了,叫道:“他娘的!为什么!你有什么不能说的?你耍得我们团团转,连个理由都不给我们,你当我们是什么?”

他猛地把脸转了过来,看着我,脸色变得很冷:“你不觉得你很奇怪吗?我自己的事情,为什么要告诉你?”

一下我就为之语塞,支吾了一声,一想,是啊,这的确是他的事情,他完全没必要告诉我。

气氛变得很尴尬,我也不知道说什么好了。

静了很久,闷油瓶喝了一口已经凉掉的酥油茶,忽然对我道:“吴邪,你跟来干什么?其实你不应该卷进来,你三叔已经为了你做了不少事情,这里面的水,不是你蹚的。”

我忽然愣了一下,下意思就数了一下,四十一个字,他竟然说了这么长的一个橘子,这太难得了,看了看他的表情,却又看不出什么来。

“我也不想,其实我的要求很简单,只要知道了这是怎么一回事,我就满足了,可是,偏偏所有的人都不让我知道,我想不蹚浑水也不可能。”我对他道。

闷油瓶看着我道:“你有没有想过,他们不让你知道这个真想的原因呢?”

我看着闷油瓶的眼神,忽然发现他在很认真的和我说话,不由吃惊,心说这家伙吃错药了。

不过这么说来,也许这一次他能和我说点什么出来。我立即就正色了起来,摇头:“我没想过,也不知道往什么地方想。”

他淡淡道:“其实,有时候对一个人说谎,是为了保护他,有些真相,也许是他无法承受的。”

“能不能承受应该由他自己来判断。”我道,“也许别人不想你保护呢,别人只想死个痛快呢?你了解那种什么都不知道的痛苦吗?”

闷油瓶沉默了,两个人安静的待了一会儿,他就对我道:“我了解。”然后看向我,“而且比你要了解。对于我来说,我想知道的事情,远比你要多,但是,我没有任何一个人可以像你一样,抓住去问。”

我一下想起来,他失去过记忆,就想抽自己一个巴掌,心说什么不和他去比,却和他比这个。

他继续道:“我是一个没有过去和未来的人,我做的所有的事情,就是想找到我和这个世界的联系,我从哪里来,我为什么会在这里?”他看着自己的手,淡淡道,“你能想象,会有我这样的人,如果在这个世界上消失,没有人会发现,就好比这个世界上从来就没有我存在过一样,一点痕迹都不会留下吗?我有时候看着镜子,常常怀疑我自己是不是真的存在,还是只是一个人的幻影。”

我说不出化,想了想才道:“没有你说得这么夸张,你要是消失,至少我会发现。”

他摇头,不知道是什么意思,说着就站了起来,对我道:“我的事情,也许等我知道了答案的那一天,我会告诉你,但是你自己的事情,抓住我,是得不到答案的。现在,这一切对于我来说,同样是一个谜,我想你的谜已经够多了,不需要更多了。”说着就往回走去。

“你能不能至少告诉我一件事情?”我叫了起来。

他停住,转过头,看着我。

“你为什么要混进那青铜门里去?”我问他。

他听完,想了想,就道:“我只是在做汪藏海当年做过的事情。”

“那你在里面看到了什么?”我问道,“那巨门后面,到底是什么地方?”

他转头拍了拍身上的沙子,对我道:“在里面,我看到了终极,一切万物的终极。”

“终极?”我摸不着头脑,还想问他。他就朝我淡淡笑了一下,摆手让我别问了,对我道:“另外,我是站在你这一边的。”说着慢悠悠的走远了,只剩下我一个人。我一下就倒在沙地上,感觉头痛无比。

\chapter{再次出发}

第二天的清晨,车队再次出发。

离开了这个叫作兰错的小村,再往戈壁的深处,就是地图上什么都没有的无人区,也就是说,连基本的被车轧出的道路也没有,车轮的底下,是几十年甚至上百年都没有人到达的土地、路况,或者说地况更加的糟糕,所谓的越野车,在这样的道路上也行驶的战战兢兢,因为你不知道戈壁的沙尘下是否会有石头或者深坑。而定主卓玛的又必须依靠风蚀的岩石和河谷才能够找到前行的标志,这使得车队不得不靠近那些山岩附近的陡坡。

烈日当空,加上极度的颠簸,刚开始兴致很高的那些人几乎立即被打垮了,人一个接一个给太阳晒蔫,刚开始还有人飙车,后来全部都乖乖的排队。

在所谓的探险和地质勘探活动中,沙漠戈壁中的活动其实和丛林或者海洋探险是完全不同的,海洋和丛林中都有着大量的可利用资源,也就是说,只要你有生存的技能,在这两个地方你可以存活很长的时间。但是沙漠戈壁就完全相反,在这里,有的只有沙子,纵使你有三头六臂,你也无法靠自己在沙漠中寻找到任何一点可以延续生命的东西,这就是几乎所有的戈壁沙漠都被称呼“死亡之地”的原因。而阿宁他们都是第一次进这种地方,经验不足,此时这种挫折是可以预见的。

我也被太阳晒得发昏,看着外面滚滚的黄尘,已经萌生了退意,但是昨天定主卓玛给我和闷油瓶的口信,让我逼迫自己下定了决心。想到了昨天晚上的事情,我又感觉一股无法言明的压力。

它就在你们当中。

它是谁呢?

在文锦的笔记中,好多次提到了自己这二十年来一直在逃避“它”的寻找,这个它到底是什么东西?而让我在意的是,为什么要用“它”而不是“他/她”?难道这个在我们当中的“它”,不是人?真是让人感觉不舒服的推测。

刚进入无人区的路线,我们是顺着一条枯竭的河道走。柴达木盆地原来是河流聚集的地方,大部分的河流都发源于唐古拉和昆仑的雪峰,但是近十年来气候变化,很多大河都转入地下,更不要说小河道,我们在河床的底部开过,发现到处都是半人高的蒿草,这里估计有两三年没有水通过了,再过几年,这条河道也将会消失。

等三天后到达河道的尽头,戈壁就会变成沙漠,不过柴达木盆地中的沙漠并不大,它们犹如一个一个的斑点,点缀在盆地的中心,一般的牧民不会进入沙漠,因为里面住着魔鬼,而且没有牛羊吃的牧草。定主卓玛说绕过那片沙漠,就是当年她和文锦的队伍分开的盐山山口,那里有一大片奇怪的石头,犹如一个巨大的城门,所以很容易找到。再往里,就是沙漠,海子,盐沼交汇的地方,这些东西互相吞食,地貌一天一变,最有经验的向导也不敢进去。

不过阿宁他们带着GPS,这点他们倒是不担心,虽然扎西一直在提醒他们,机器是会坏掉的。特别是在昼夜温差五十多度的戈壁上。

顺着河道开了两天后,起了大风,如果是在沙漠中,这风绝对是杀人的信风,幸好在戈壁上,它只能扬起一大团黄沙,我们车与车之间的距离不得不拉大一百米以上,能见度几乎为零,车速也满到了最低标准,又顶着风开了半天后,车和驾驶同时就到达了极限,什么也看不到,什么也听不到,无线电也无法联络,已经无法再开下去了。

高加索人并不死心,然而到了后来,我们根本无法知道车子是不是在动,或者往哪里动,他只好停了下来,转了方向侧面迎风防止沙尘进入发动机,等待大风过去。

车被风吹的几乎在晃动,车窗被沙子打的哗啦啦作响,而我们又不知道其他车的情况,这种感觉真是让人恐惧。我看着窗外,那是涌动的黑色,你能够知道外面是浓烈的沙尘,而不是天黑了,但是毫无办法。

在车里等了十几分钟后,风突然又大了起来,我感觉整个车子震动了起来,似乎就要飞起来一样。

高加索人露出了恐惧的神色,他看向我说:“你以前碰到过这种事情没有?”

我心说怎么可能,看他惊慌的样子,就安慰他说放心,路虎的重量绝对能保护我们,可是才刚说完,突然“咣当”一声巨响,好像有什么东西撞到路虎上,我们的车整个震了一下,警报器都给撞响了。

我以为有后面的车看不到路撞到我们了,忙把眼睛贴到窗户上,高加索人也凑过来看。

外面的黑色比刚才更加的浓郁,但是因为沙尘是固体,所以刮过东西的时候会留下一个轮廓,如果有车,也可能能看到车的大灯。

然后却外面看不到任何车的灯光,我正在奇怪,高加索人却突然怪叫了起来,抓住我往后看,我转过头,就看到我们的另一面的车窗外的沙尘里,不知道什么时候,出现出一个奇怪的影子。

车窗外的黑色影子模糊不清,但是显然贴的车窗很近,勉强看去,似乎是一个人影,但是这样的狂风下,怎么会有人走在外面,这不是寻死吗?

我们还没有来得及惊讶,那影子就移动了,他似乎在摸索着车窗,想找打开的办法,但是路虎的密封性极好,他摸了半天没有找到缝隙,接着,我们就看到一张脸贴到了车窗上。车里的灯光照亮了他的风镜。

我一下就发现,那是阿宁他们配备的那种风镜,当即松了口气,心说这王八蛋是谁,这么大的风他下车干什么?难道刚才撞我们的是他的车。

窗外的人也看到了车里的我们,开始敲车窗,指着车门,好像是急着要我们下去,我看了看外面的天气,心说老子才不干呢!

还没想完,突然另一边的车窗上也出现了一个带着风镜的人的影子,那个人打着灯,也在敲车窗,两边都敲的和很急促。

我感觉到不妙,似乎是出了什么事情,也许他们是想叫我们下去帮忙,于是也找出斗篷和风镜带起来,高加索人拿出两只矿灯,拧亮了递给我。

我们两个深吸了口气,就用力的打开车门,一瞬间一团沙尘就涌了进来。我虽然已经做好了准备,但是还被一头吹回了车里,用脚抵住车门才没有让门关上,第二次用尽了吃奶的力气,低着头才钻了出去,被外面的扶住拖了出来。而另一边下车的高加索人直接就给刮倒在地,他的叫骂声一下给吹到十几米外。四周全是鼓动耳膜的风声和风中灰尘摩擦的声音,这声音听来不是很响,却盖过其他所有的声音,包括我们的呼吸声。

脚一落到外面的戈壁上,我就感觉到了不对劲,地面的位置怎么抬高了?用力弓着身子以防被风吹倒,我用矿灯照向自己的车,这一看我就傻眼了,我操,车的轮子一半已经不见了,车身斜成三十度,到脚蹬的部分已经没到了河床下沙子里,而且车还在缓慢往下陷,这里好像是一个流沙床。难怪车子怎么开都开不动了。

没有车子,我们就完蛋了。我一下慌了,忙上去抬车,但是发现一踩入车子的边缘,就有一股力量拽着我的脚往下带,好像水中的旋涡一样,我赶紧跳着退开去。这时候一旁刚才敲我们窗的人就拉住我,艰难的给我做手势,说车子没办法了,我们离开这里,不然也会陷下去。

他包的严严实实的,嘴巴裹在斗篷里,我知道他同时也在说话,但是我什么都听不见,我不知道他是谁,不过他手势表达的东西是事实。于是我点了点头,用手势问他去哪里?他指了指我们的后车盖,让我拿好东西,然后做了个两手一齐向前的动作。

这是潜水的手语,意思是搜索,看样子在车里的很多人如果不下车,肯定还不知道车已经开进了流沙床,我们必须一路过去通知他们,不然这些路虎会变成他们价值一百多万的铁棺材。

我朝那个人点了点头,做了个OK的手势,就打开车后盖取出了自己的装备,几乎是弓着身子,驼背一样的完成这简单的事情。此时,其实我的耳朵已经被轰麻了,四周好像没了声音,一片的寂静,这有点看默片的感觉,一部立体的默片。

关上车盖的时候,我就看到我们的车后盖已经凹陷了下去,好像给什么庞然大物擦了一下一样,我想起了车里的震动,就用矿灯朝四周照了照,然而什么都看不到,只有高加索人催促我快走的影子。

我收敛心神,心说也许是刮过来的石头砸的,就跟着那几个影子蜷缩着往后面走去。

走了八十几米,我感觉中的八十几米,也许远远不止,我们就看到下一辆车的车灯。这辆车已经翘起了车头,我们上去,跳到车头上,发现里面的人已经跑了出来。我们在车后十几米的地方找到了他们,有一个人风镜掉了,满眼全是沙子,疼的大叫,我们围成风墙,用毛巾把他的眼睛包起来。

我们扶着他起来,继续往前,很快又叫出了一辆车,车里三个家伙正在打牌,我们在车顶上跳了半天他们都没反应,最后我用石头砸裂了他们的玻璃,此时半辆车已经在河床下面了。

把他们拖出来后,风已经大到连地上的石头都给刮了起来,子弹一样的硬块不时的从我们眼前掠过去,给打中一下就完蛋了。有一个人风镜给一块飞石打了一下,鼻梁上全是血,有人做手势说不行了,再走有危险,我们只好暂时停止搜索,伏下来躲避这一阵石头。

几个人都从装备中拿出坚硬的东西,我拿出一只不锈钢的饭盒挡在脸上,高加索人拿出了他的圣经,但是还没摆好位置,风就卷开了书页,一下子所有的纸都碎成了纸絮卷的没影了,他手里只剩下一片黑色的封面残片。

我对他大笑,扯起嗓子大喊:“你这本肯定是盗版的!”还没说完,一块石头就打在了我的饭盒上,火星四溅。饭盒本来就吃着风的力道,一下我就抓不住,消失得没影了。

我吓了个半死,这要是打到脑袋上,那就是血花四溅了,只能报紧头部,用力贴近地面。

这个时候,突然就是四周一亮,一道灼热闪光的东西就从我们的一边飞了过去,我们都给吓了一大跳,我心说我操,什么东西这么快。还没等我反应过来,前面又是三道亮光闪起,朝我们飞速过来,又是在我们身边一掠而过。接着我就闻到一股熟悉的气味,那是镁高温燃烧的气味。心里立即知道了闪光是什么东西——那是给裹进风里的信号弹。

我不禁大怒,心说是哪个王八蛋,是哪只猪在这中天气下,在上风口放信号弹,怕风吹不死我们想烧死我们吗?时速一百六十公里以上千度高温火球,打中了恐怕会立毙。

但是转念一想,就知道不对了,这批人都训练有数,怎么可能会乱来。在探险中,发射信号弹是一种只有在紧急的时候才会使用的通信方式,因为它的传播范围太广,弹药消耗大,一般只有在遇到巨大的危险,或者通信对象过于远的时候才会使用。现在在这么恶劣的条件,他们竟然也使用了信号弹,那应该是前面出了什么状况。

我看一眼四周的人,他们都和我有一样的想法,我就做了个手势,让三个没受伤的人站了起来,我们要往那里去看看。如果他们需要帮忙,或者有人受伤,不至于没有帮手。

这不是一项说做就做,或者是个人英雄主义的差事,我刚站起来就被一块石头打中肩膀,我们都把包背到前面当成盾牌,调整了指北针,往信号弹飞来的方向走去。同时提防这还有信号弹突然出现。

走了一段时间后,我们也不知道自己的方向有没有走歪掉,不过在一百多米开外,我们看到了三辆围在一起的车,但是车的中心并没有人,已经离开了。我们在车子的周围搜索,也没有发现人,但是车里的装备没有被拿走。

车子正在下陷,我们打开了车子的后盖,心说至少应该把东西抢救出来,就在刚想爬入车子里的时候,又有信号弹闪了起来,在我们很远的地方掠了过去。这一闪,我们发现发射信号弹的地方变成了在我们的左边,离我们并不是很远。看样子我们的方向确实歪了。或者是发射的人自己在移动。

我们背起装备,虽然非常的累,这样一来风却吹不太动我们了,我们得以稳定了步伐,向信号弹发射的地方走去。走着,走着,我们忽然就惊讶的看到,前方的滚滚沙尘中,出现了一个庞然大物的轮廓。

狂风中,我们弓着身子,互相搀扶着透过沙雾,看着那巨大的轮廓,都十分的意外,一下子也忘了是否应该继续前进。

边上的高加索人打着手势,问我那是什么东西。这个家伙有一个惯性思维,就是他现在在中国,那么我是中国人,在中国碰上什么东西都应该问我。

我摇头让他别傻,我心里也没有底。

平常来讲,毫无疑问,在我们前面的不到两百米的地方,如果不是一只中年发福的奥特曼,那应该就是一座巨大的山岩,这是谁都能马上想到的,但是我们来这里的路上是一马平川,并没有看到有这么高大的山岩。

这山岩是从哪里冒出来的?难道是我们集体失神了,都没看到?我心里说,又知道不可能,首先最重要的是我们一路过来都在寻找这种山岩,因为我们需要阴凉的地方休息,这种山岩的背阴面是任何探险队必选的休息地。而平时的戈壁上,这样的孤立的山岩并不多,所以如果有我们肯定会注意。

不过现在也管不了这么多了,这么大的山岩,是一个避风的好场所,那些发信号弹,也许是通知我们找到了避风的地方。

我开始带头往山岩跑去,很快我就明显的感觉到,越靠近岩石,风就越下,力气也就越用的上,跑到一半的路程的时候,我已经看到了前面有五六盏矿灯的灯光在闪烁。

我欣喜若狂,向灯光狂奔,迎着狂风,一脚深一脚浅的冲了过去。然而跑了很久,那灯光似乎一点也没有朝我靠近,他妈的竟然有这么远,我心里想着,一边已经精疲力竭,慢了下来,招呼边上的人等等,我感觉事情有点不对。

可我回头一看,不由得傻了眼,我身边哪里还有人,前后左右只有滚动的狂沙和无尽的黑暗。

\chapter{迷路}

这里的风已经不像刚才那么霸道,风打着卷儿在四周甩,前面肯定是有挡风的东西没错的,可是刚才跟着我那两家伙哪儿去了,我走的也不快啊,这样也能掉队,他娘的是不是给飞石砸中了,摔在后面了?

我举高矿灯往四周照,并没有看到任何的影子,不由有点后悔,刚才注意力太集中了,我没有太过注意四周的情况。不过,在这样狂风中行进,其实四周也根本就没有什么情况可以注意,风声响的人都听不到,而所有的精力都必须放在眼前的目的地和身体的平衡上。

一下子落单,我还是在一瞬间感觉到一种恐惧,不过我很快就将恐惧驱散了,我休息了一下喘了几口气,就开始继续往前走,此时我不能后退去找他们,我已经失去了方向的感觉,如果往回走不知道会走到哪里,最好的办法就是往前。

我甩掉了一包装备,这东西实在是太重了,老外的探险装备很个性化,有一次我还看到有人带着他老婆的盾牌一样大的像框和电话本一样的资料书,我懒的给他们背了,自己轻装就往灯光的地方跑去。

可是,无论我怎么跑,那灯光却还是遥不可及,好像一点也没有靠近一样,我喘的厉害,心里想放弃,但是又不甘心。跑着跑着,前方的灯光就迷离了起来。

就在我快要失去知觉,扑倒在地上的时候,忽然间,有人一下子把我架住了。我已经没有体力了,给他们一拉就跪倒在地上。抬头去看,透过风镜,我认出了这两个人的眼睛,一个是闷油瓶,一个是黑眼镜,他的风镜也是黑色的。这两个人亟亟将我拉起来,就将我拖向另外一个方向。

我挣脱他们,指着前方,想告诉他们那里有避风的地方。

然而我再一看,却呆住了,什么都没有看到,前方的灯光竟然消失了,那里是一片的黑暗,连那个巨大的轮廓也不见了。

闷油瓶和黑眼镜没有理会我,一路拖着我,这时候我看到黑眼睛的手里拿着信号枪。两个人的力气极大,我近一百八十斤的体重被他们提的飞快。很快我也清醒了过来,开始用脚蹬地,表示我可以自己跑。

他们放开了我,我一下就后悔了,这两个人跑的太快,跟着他们简直要用尽全身的力气,我咬牙狂奔,一路跟着,足跑了二十分钟,眼睛里最后只剩下前面跑的两个影子。恍惚中我知道我们已经冲上河岸,绕过了一团土丘,接着前面两个黑影就不见了。

我大骂了一声等等我,脚下就突然一绊,摔了好几个跟头,一下滚到了什么斜坡下。我挣扎的爬起来吐出嘴巴里的泥,向四周一看,斜坡下竟然是一道深沟,里面全是人,都缩在沟里躲避狂风。看到我摔下来,都抬起头看着我。

我们缩在沟的底部,沙尘从我们头上卷过去,戈壁滩并不总是平坦的,特别是在曾经有河流淌过的地方,河道的两边有很多潮汛时候冲出来的支渠,这些戈壁上的伤疤不会很深,但是也有两三米,已经足够我们避风了。

我已经精疲力竭,几个人过来,将我扯到了沟渠的底部。原来在沟渠的底部的一侧有一处很大的凹陷,好像是一棵巨大的胡杨树给刮倒后,根部断裂形成的坑被水冲刷后形成的,胡杨的树干已经埋在沟渠的底部,只能看到一小部分,他们都缩在这个凹陷里面,里面点着无烟炉取暖,一点风也没有。

我给人拖了进去,凹陷很浅,也不高,里面已经很局促了,他们给我让开了一个位置,一边有一个人递给我水。这里是风的死角,已经可以说话,可是我的耳朵还没有适应,一时听不到他们在说什么。

喝了几口水后,我感觉好多了,拿掉了自己的风镜,就感慨他娘的,中国有这么多的好地方,为什么偏偏我要来这里?

不过,这样的风在柴达木应该不算罕见,这还不是最可怕的风,我早年看过关于柴达木盆地地质勘探的纪录片,当时勘探队在搭帐篷的时候来了信风,结果人就给风筝一样给吹了起来,物资一瞬间全给吹出去十几里外。只不过我感觉到奇怪的是,定主卓玛为什么没有警告我们?戈壁上的信风是很明显的,不要说老人,只要是在这里生活上一段时间都能摸到规律。

同样,不知道这风什么时候才能刮完,经常听戈壁上的人说,这种地方一年只刮两次风,每一次刮半年,一旦刮起来就没完没了。要是长时间不停,我们就完蛋了。

闷油瓶子和那个黑眼镜很快又出去了,肯定又是去找其他的人,这里的人显然都受到了惊吓,没有几个人说话,都蜷缩在一起。我心里感觉好笑,心说还以为这些人都像印地安那琼斯一样,原来也是这样的不济,不过我随即就发现自己的脚不停的在抖,也根本没法站起来。

递给我水的人问我没事吧?身上有没有地方挫伤?我摇头说我没事。

说实在的,在长白山冒着暴风雪的经历我还记忆犹新,现在比起那时候,已经算是舒服了,至少我们可以躲着,也不用担心冻死。

倒了一点水给自己洗脸,眼睛给风镜勒的生疼,这个时候也逐渐舒缓了。

放松了之后,我才得以观察这坑里的人,我没有看到阿宁。定主卓玛、他儿媳妇扎和西三个人,在凹陷的最里面,乌老四也在,人数不多,看来大部分的人还在外面,没有看到高加索人。

这支队伍的人数太多了,我心想,阿宁他们肯定还在外面寻找,这么多的人,纵使闷油瓶他们三头六臂,也照顾不过来了,幸好不是在沙漠中,不然,恐怕我们这些人都死定了。

三个小时后,风才有点减缓,闷油瓶他们刚开始偶尔还能带几个人回来,后来他们的体力也吃不消了,也就不再出去。我们全部缩在了里面,昏昏沉沉的,一直等到天色真真的黑下来,那是真的漆黑一片了。外面的风声好比恶鬼在叫,一开始还让人烦躁,到后来就直感觉想睡觉。

我早就做好了过夜的准备,也就没有什么惊讶的,很多人其实早就睡着了。有人冒着风出去,翻出了在外面堆着的很多行李里的食物,我们分了草草的吃了一点,我就靠着黄沙上睡着了。

也没有睡多久,醒来的时候风已经小了很多,这是个好迹象,我看到大部分人都睡觉了,扎西坐在凹陷的口子上,似乎在守夜。这里并不安稳,在我们头顶上的不是石头,就是干裂的泥土和沙石,所以不时的有沙子从上面掉下来,我睡着的时候吃了满口的沙子,感觉很不舒服,一边呸出来,一边就走到扎西身边去。

我并不想找扎西去说话,扎西不是一个很好相处的人,或者说他对我们有着戒备,而我也不是那种能用热脸去贴冷屁股的人,所以他的态度我并不在乎。我走到他的身边,只是想吸几口新鲜的空气,换个地方睡觉。

不过我走过去的时候,就听到外面有声音,然后看到外面有矿灯的光线,似乎有人在外面。

我心中奇怪,问扎西怎么了?扎西递给我一支土烟,说阿宁回来了,风小了,他们叫了人出去找其他人去了,顺便看看车子怎么样了。

我想到陷在沙子里的车子,心里也有一些担心,这么大的风沙,不知道这些车子挖出来还能不能开,而且我比较担心高加索人,不知道他回来了没有,于是戴上了风镜,批上斗篷也走了出去,想去问问情况。

一走到外面,我心里就松了口气,外面的风比我想象的还要小,看来风头已经过去了,空气中基本上没有了沙子,我扯掉斗篷,大口的呼吸了几下戈壁上的清凉空气,然后朝矿灯的方向走去。

那是河床的方向,我走了下去,来到了他们身边。

他们正在查看一辆车,这辆车斜着陷在了沙子里,只剩下一个车头,阿宁拿着无线电,正在边上焦急的调拨着频率。

我问他们:“怎么样?”

一个人摇头,只说了一句:“妻离子散。”

我莫名其妙,并不是很能理解他的意思,于是看向阿宁。

她看到我,很勉强的笑了笑,就走过来解释道:“刚才定主卓玛说,可能还要起风,我们必须尽快找到更好的避风点。不过我们的车都困住了,有几辆肯定报废了,其他的恐怕也不能开动,需要整修。”她顿了顿,“最麻烦的是,有四个人不见了,有可能在刚刚风起的时候就迷失了方向,我们刚才找了一圈也找不到。”

我问是哪几个人,阿宁就说是那个高加索人,还有三个人我不熟悉。

高加索人在失踪的时候是和我在一起的,我就给他们指了方向,问他们有没有去那一带找过。阿宁就点头,说附近都找了,这些人肯定走得比她想的更远。

我叹了口气,安慰了她几句,让她不要着急。这些人都有GPS,而且风这么大,肯定走不远,现在还有风,视野不是很清晰,等到天亮,找起来就方便点了。

她咬着下嘴唇点了点头,但是表情并没有变化,让我感觉似乎有些不妙。我对于戈壁也不熟悉,此时不知道会发生什么事,只好闭嘴了。

我们强行打开了两辆车的车门,拿出了里面的装备,然后他们还要去找下一辆,我只好跟着过去。

此时我发现把车子陷入到河床当中的,似乎不是小说中经常提到的流沙,而是在河床的底部,地面被压塌了,车子给整个陷了下去,又没有没顶。有个人告诉我,是盐壳给压碎了,这里的戈壁下面很多地方都有大量的盐壳,这里是河床,之前有水的时候,河底的情况非常复杂,有着大量的沉淀物,干旱之后,盐壳结晶的时候就留下了很多的空隙,所以这种河床中有些地方其实像干奶酪一样,并不经压,我们停车停错了地方。

我奇怪道:“但是我们一路过来都是在河床上走的,一直没出事情啊。”

那人道:“那是因为之前我们走的河道已经干旱了很久了,但是现在我们脚下的河道,最多干了半年时间。你没有发现这里几乎没有草和灌木吗?”

我吃惊的看了看四周,果然如此,四周光秃秃的,连梭梭都不长。

那人朝我道:“我们现在肯定是朝着这条河的上游走,这条河的尽头肯定是一座高山,如果河流没有改过道的话,在这种河的附近肯定会有古城或者遗迹,这说明那个藏族老太婆并不是瞎带路的。我老早还以为这老太婆是个骗子。”

我看着他指的河道上游,在平坦的戈壁上,好像真有点什么。想起在风里看到的那巨大的黑影,我总感觉那不是我的错觉。

当天晚上,我们将所有的车都找了出来,然后把行李都集中了起来。天亮的时候,其他人陆续的醒了,阿宁开始组织他们忙活,修车的修车,找人的找人。

我和另外几个晚上找车的人就吃了点东西,到睡袋里去补觉,非常疲倦,一睡就睡到了夕阳西下。

醒来之后,风已经完全听了,沙尘都没了,那批人的效率很高,好几辆车都修好,整装待发,各种物资也都重新分配好了,正在重新装车。

阿宁一天一夜没睡,在不停的听着无线电,闷油瓶和那个黑眼镜都不在,一问,两个人还在外面找那四个失踪的人。

我听了感觉到不太妙,已经一天时间了,那四个人竟然还没有找到,不是有GPS吗?难道真的如扎西说的,这东西在戈壁里不管用?

我从包里拿了干粮出来,边吃边到阿宁身边,问具体的情况。

阿宁眉头紧锁,黑眼圈都出来了,感觉很憔悴,问她她也没什么心思回答我,对讲机一直是在外面找人的对话,用的是英文,我草草听了,都不是好消息。

我问她要不要我也出去找一下,她就摇头说不用了,已经分了三组出去,都在找第三遍了,我去了也不见得有用,让我收拾一下,扎西他们在前面二十公里的地方发现了一个魔鬼城,等一下我们出发到那里去休整,晚上还有起风。

我看她的样子已经焦头烂额,也不想烦她,就去看另外一批人修车,帮忙递工具。

大概看了半个多小时,扎西从远处的河床里回来,对我们道又要起风了,前面的地平线已经起沙线了,我们要快走,不然车子就白修了。

我们马上准备,很快就把东西准备好,因为车子少了,没修好的车子就给拖在后头,我和几个藏人一辆车,起程朝太阳落山的地方出发。

在浩瀚戈壁上大概开了二十分钟,夕阳下前方就出现了雅丹地貌的影子,一座座石头山平地而起,对讲机里传来扎西的声音,指引我们调整方向,很快便看到一座巨大的“城堡”,出现在视野里。

那就是扎西选择的避风的地方,我们直开过去,开近看时,发现那是一座馒头一样的大石山,后面就是逐渐密集的大片雅丹地貌,好比城堡后面的防御工事。

魔鬼城又叫风城,是大片岩石被大风雕琢出来的奇特地形,一大片区域内,分布着大量奇形怪状的岩山,可以给人想象成各种诡异的事物,而且风刮过这些岩石的时候,因为分布的关系,会发出鬼哭狼嚎的声音,所以叫做魔鬼城。在戈壁上,这样的地貌非常常见。我以前在新疆参观过,这一次也并不好奇。

我们在那“城堡”外面,一座底部平坦的岩山停了下来,扎西先跳下来吆喝,我们都下来开始扎营,两个小时后,果然开始起风,一下又是遮天蔽日的风沙,一直刮到半夜,才像昨天一样慢慢小了下来。

风太大,魔鬼城里鬼哭狼嚎的,谁也睡不着,风小了,才逐渐一个一个睡了过去。那两个白天睡觉的守夜,这两个人都对魔鬼城很感兴趣,看我和扎西也没有睡,都到外面去拍照。扎西就让他们小心点,不要走进去,里面很容易迷路。

我白天睡了觉,非常精神,阿宁则是琢磨明天的搜索办法,手还一直抓着对讲机,看来不找到那几个人,阿宁是不会休息了。

我过去劝她睡一会儿,还没说了几句话,忽然就有人在远处的戈壁上大叫:“队医!队医!”

阿宁的队医是个胖子,也没睡在看书,一听就醒了,我们也朝那边望去,就听到那边在喊:“快过来!找到阿K了!”

阿K就是失踪的四个人中的一个,我们一听全部跳起来,三步并成两步的跑过去,一下就看到是那两个拍魔鬼城的人,在一个土丘上朝我们招手,冲过去一看,只见在土丘上有一个大坑,坑底就躺着一个人,正是那个阿K。

队医跑得气喘吁吁,跳了下去,摸了一下,就大叫:“还活着。”

几个人手忙脚乱的冲下去抬人,队医大叫让他们把他抬到帐篷里去。

现场一片混乱,扎西背起那人跑了回去,我就给挤到了一边,看了看那个坑,又看了看一边我们来的方向,心说天哪,这人怎么会倒在这里,这和我们昨天停车的地方还有二十公里还多啊,而且当时这方向还是逆风。他是顶着风过来的?

回到队医的帐篷里,看着队医抢救,很快那个阿K就被救了过来,队医松了口气就说只是因为疲劳过度晕倒了。队医给他打了一针,很快他就醒了。

他醒了以后,我们就问他是怎么回事,他就说他也不知道是怎么回事,一路走,走着走着,就看到前面有影子,他以为有石头山,就靠过去,结果走啊走啊,也不知道走了多久,就摔坑里去了。说着他就问:“哎,那个老高和另外两个人回来了没有?”

老高就是高加索人,我一听他说那影子的事情,心中就一个激灵,想问他详细。但是阿宁一听到他问老高,马上就问他为什么这么问,是不是见过他们。

他道:“当时他们就在我前面,我怎么叫他们,他们都不回头,想想是逆风走,他们听不到,后来我就摔晕了,怎么,他们没回来?”

阿宁惊讶道:“你是说你在摔晕前还看到他们?”

阿K就点头,阿宁转过头,对我道:“听到了没有?发现老K的地方是魔鬼城外面,前面就是魔鬼城,这么说,他们进城里去了!难怪我们怎么找也找不到。”

她一下眼睛都有了神采,马上拍手让我们出去,我们走出队医的帐篷,一商议,阿宁就坚持马上进魔鬼城去搜索。

这些人也不知道是怎么回事,逆风走了二十多公里,老K在外面摔昏了,里面的人可能也已经精疲力竭了,必须马上把他们找出来,这样我们也可以安心一点。

我精神很好,就点头答应。我们马上分配了一下队伍,很多人就睡了,没有叫醒他们,就是队医和我,准备三个人先进去探一圈看看,其他人等两个小时,再叫醒跟进来。

说完我们马上开始准备,刚把包拿起来,一边的扎西走了过来,拦住了我们,道:“等一下,我奶奶说,你们不能进去。”

\chapter{魔鬼城}

阿宁很奇怪,问道:“为什么?”

扎西对我们道:“我奶奶说,你们眼前的这一片魔鬼城,不是旅游景点,这片雅丹地貌大概有八十七平方公里,十分广袤,里面还是最原始的状态,没有任何的路标,晚上在里面行进,如果不熟悉环境,非常容易迷路。而且据说这里面有很多的流沙井,在1997年的时候就有一队地质考察队员在里面失踪了,当时出动了很多人找都没找到,后来在1999年的时候起大风,几个摄影师在这里拍照片的时候就在一个沙坑里发现了两具干尸,其他的人到现在还没找到。”

阿宁听了摇头,道:“这你不用担心,我们带着GPS,如果如你说的,这里面地形这么复杂,我们更要进去,如果等到天亮去找,他们说不定已经出事了。”

说着就不听扎西的劝告,招呼几个人,拧亮了手电,打算继续深入。

我想想她说得也有道理,扎西一直以来都扮演着危言耸听的角色,现在他的话阿宁自然不会全信,而且老外的做派是以人为本,把那三个人放掉不管,在他们心里相当于是亲手杀了他们,这些人没法作出这种决定。

我自然是要跟着去的,因为那三个人是和我一起的时候失踪的,或多或少,我也得尽点力气,否则要是真有个什么意外,我心里也不会安宁。而且坐在这里也完全不可能睡着。

扎西还要说话,这时候一边的定主卓玛发话了,她摇了摇头,让扎西不要说了,接着用藏语很快对扎西说了几句什么。

扎西马上露出了很不理解的表情,然而定主卓玛的表情很坚决,扎西还要抗议一下,定主卓玛就呵斥了一声,扎西就不敢再继续说话了。他对定主卓玛点了点头,退了回来,一脸郁闷的对我们道:“你们走运,我奶奶让我带你们进去。”说着拧起手电就走到自己的行李边上,开始清理装备。

我听不懂藏语,问阿宁那老太婆说了什么。阿宁也摇头,说太轻了听不奥,大约是收人钱财、替人消灾这样的话吧。

我心里好笑,就看了一眼定主卓玛,这老太婆已经回帐篷去了,看来倒是一点也不担心这些事情。

扎西把自己的装备清理了一遍,让我们把不必要的东西都放掉,带上足够的水和干粮,还有信号枪,然后叫醒了一个司机,告诉他我们的打算,让他在外面待着,准备接应,如果看到我们在里面打信号弹,不要进来,就在外面打信号弹给我们指方向。如果还不出来,等天亮了再让其他人进来找我们,他会沿途留下记号。

那司机迷迷糊糊的答应,我们四个人整顿了一下,扎西拉长个脸带头,就往身后魔鬼城城口出发。

我们避风的地方在魔鬼城的边缘,扎营的高大岩山之后便是一个陡坡,向下一直延伸,尽头时沙暴时看到的那座城堡一样的岩山,这应该是魔鬼城里比较高的一块岩山了。

扎西在陡坡上用碎石头堆了一个阿拉伯石堆,为后来人标志方向,他说,一路过去只要有转弯他就会堆一个,而一旦在前进过程中看到自己堆的石堆,我们就不能再前进了,再前进就会开始绕圈子。这是他的底线。

我们感觉有道理,就说没问题。

很快就走入城口,我们进入到了魔鬼城的里面,四周的情景开始诡异起来,举目看去,月光下全是突出于戈壁沙砾之上黑色的岩山,因为光线的关系看不分明,手电照去就可以看到岩山之上被风割出的风化沟壑十分的明显。在这种黑色下,少数月光能照到的地方就显得格外的惨白,这种感觉,有点像走在月球表面。

我一路看着,想象着当年的地质力学里的内容,已经忘记得一干二净了,只知道这地方的雅丹风蚀岩群还未成年,大概是地势比较低,岩山和土丘暴露出地表的时间不长,并没有被风化得十分厉害,所以大部分的岩山土丘还十分的高大。

这种情况下,我们只能在岩石土丘之间穿行,无法像其他魔鬼城一样随意的爬上土丘,不过,这种地貌下的山谷也并不平坦,高的地方突出在沙砾之上,低的地方则被戈壁覆盖。在地质学里,这种岩山其实都被认为是地下山脉的山顶,别看只有十几米高,但是我们脚下几公里深的地方有着巨大的岩石山基,这些藏在沙砾下的大山都是昆仑山的支脉。理论上说,我们现在也是行走在昆仑山上。

不过我没空多想这些学术问题,一进到两三公里的地方,阿宁开始用对讲机呼叫,我们则大声的喊起来,希望那三个人能听到我们的声音,给我们回应。

在寂静的魔鬼城,我们的声音一下就被反弹成无数种回升,重叠在一起,能传播出去很远。远远的听去非常的诡异,好像来自幽冥的鬼声。

就这样一边喊一边走,足找了两三个小时,深入到了魔鬼城的深处,手电扫着四周的岩石,眼睛也花了,嘴巴也喊麻了。可是根本没有发现一点高加索人他们的影子,我们的喊声也没有任何的回音,回答我们的只有我们自己的回音和轻微的呜吟风声。

我们停下来休息,阿宁几问扎西,按照他的经验,怎么找会比较好?

扎西摇头:“也只有你们这种办法,我们现在大概走了七公里,按照直线距离我们已经走了很长一段路了,但是其实我们早就不知不觉的转了方向,看指南针现在我们几乎在往回走,人在这里好比蚂蚁一样,会不知不觉走S形路线,所以说我现在只能保证带你们出去,找人我没法提供建议……他们不动还好,如果他们也在找出路,那你说你在八十平方公里的迷宫里两队人相遇的概率是多少?”

阿宁对这个回答不满意,皱眉道:“你们之前就没有人走失过?”

扎西堆着石头堆,头也不抬的摇头:“这种地方我们晚上从不进来。”

说完他就叹了口气,不知道是什么意思。

阿宁看我们的表情,鼓舞了我们几句,让我们不要灰心。不过显然作用不大,我们抽了好几根烟,稍微恢复了一下精神,就继续前进。

可是,事情还是没有向我们期望的发展。又一边喊一边走,也不知道走了多少时间,期间休息了四次,扎西堆了不下三十个石堆,却还是连个人影也没有看到,没有任何的回应,寂静的魔鬼城里好像吞吃掉了任何给我们的声音。

而让我真切感觉到可怕的是,我们没有看到任何一个扎西的石堆出现,说明我们现在还在前进,这魔鬼城真好像深不可测一样,不知道里面还有多少的路程。

继续往前,我们走进了一道岩石夹成的峡谷,在一块大石头下,实在是走不动了,只能第六次停下来休息。

这时候我们嗓子都哑了,再也喊不动了。我们大口的喝着水,所有人都进入到一种失语状态,脑子都有点空白起来。

沉默了一段时间,那个队医突然道:“该不是这魔鬼城真的有魔鬼?他们被魔鬼带走了?”

这话说得很突兀,我们都愣了一下,扎西瞪了他一眼,让他别胡说,藏人比较传统,这种话听着不舒服。

“魔鬼是肯定没有,人也是肯定在这里。”隔了半晌,扎西含着一口水,边润喉咙边慢慢的说道:“只不过不知道现在是什么状况。”

几个人又沉默了下来,各自琢磨自己的心思。事实上我知道现在我们几个人心里的希望已经非常小了,刚开始进来,我还认为找到他们的概率很大,至少能发现点痕迹,现在,则完全没了想法。

又休息了一段时间,阿宁看了看表站了起来,招呼我们准备继续出发,我们都条件反射的站起来,深呼吸,准备振奋一下,继续呐喊。

就在这个时候,我们几个人都听到阿宁的对讲机里突然传出来一声人的大叫声。静电声音很大,非常的刺耳,听不出是什么话。

四周安静得要命,突然这一下声音把我们吓了个半死,马上看向阿宁的对讲机。

阿宁也愣住了,花了好几秒才反应过来,忙拿起对讲机仔细去听。

那声音又响了一次,静电极其刺耳,但是很明显能听出是一个人在呼叫。

“他们在附近!”我们惊叫起来。阿宁几乎跳了起来。

魔鬼城这样的地形,对讲机几乎没有作用,只有在非常短的距离内,才能收到信号。阿宁一路调试就是想收到这样的信号,然而都没有结果,现在信号突然响起来,显然对方的对讲机就在非产近的地方。

我们心里长出了一口气,阿宁马上开始调频率,那声音就清晰了起来,但是仍旧听不出他在说什么。接着她对着对讲机大叫:“我是领队,我们在搜救你们,你们在什么方位?”

回答是一连串难以言喻的声音,干扰非常眼中,但是语调变了,显然对方能听见我们的声音。

刚才的沮丧一扫而光,队医大叫了一声“YES”。我也掏出了自己的对讲机,拍了拍,调了一下,看看是不是机器的问题,很快我也调出了声音,同样是嘈杂的。

阿宁又呼叫了一次,这一次声音又稍微清晰了,我们几个人努力去听,希望能听清楚对方在说什么。

听着听着,我就发现不对,对讲机那头的人好像不是在说话,那种说话的语调,十分的古怪,很难形容,仔细听起来,竟然好像是一个人在怨毒的冷笑。

\chapter{魔鬼的呼叫}

我“嗯”了一声,就感觉到不妙,再听了听,越听感觉越像,这绝对不是说话,不能肯定是笑声,但是十分的相像。

其他几个人也意识到了,阿宁停止了呼叫,我们互相看了看,都有点诧异。

队医道:“怎么回事?他们怎么在……笑?是不是听到我们的声音太开心了?”

扎西就反问道:“你开心的时候是这么笑的?”

阿宁也是一脸的疑惑,她不再呼叫,而是继续调试了一下对讲机,想让里面的声音更加的清晰一点。

调试没有作用,不过那声音倒是又响了几分,我们再次贴上去听,又听得更加分明了一点,真的非常像冷笑声,听上去如此的怨毒,根本不是正常人发出的,倒像是疯人院疯子发出的。不过仔细去听,又感觉这笑声之后,还有一些别的声音,非常的轻微。两种声音混杂在一起,在这带着恐怖色彩的魔鬼城里听上去相当的诡异。

听着这不怀好意的冷笑,我感觉很不舒服。就连一路过来一脸臭屁的扎西现在都害怕了,脸色惨白,咽了口唾沫:“怎么回事,这笑得真他妈的难听。”

阿宁做了个手势让他别说话,把对讲机贴住自己的耳朵,又听了一会儿,就道:“这好像不是人的声音!”

“你别乱说!”队医叫起来:“不是人难道是鬼?”

“你们仔细听。”阿宁让我们凑近,“这声音的频率很快,而且,语调几乎是平的,已经响了五分钟了,你尝试这么笑五分钟给我听听?”

我一听,感觉有点道理,就问道:“那这是什么声音?”

“这种频率,应该是机械声,比如说手表贴在对讲机上了,不过听频率又不固定,也有可能是有人在不停的用指甲抓对讲机的对讲口。”阿宁示范了一下,“加上静电的声音,就成了这个样子。”

“用指甲抓对讲机口,为什么他们要这么做呢?”队医道,“为什么不大叫,这样也许我们不用对讲机就能听见。”

他话一说,扎西和阿宁的脸色都变了,我也突然意识到了什么:“他们可能处在不能大叫,也不能说话,只能用这种方式和我们联络的处境中。”

“流沙坑!他们陷在流沙坑里了!”扎西叫了起来:“可能已经沉得只剩下个头了,那种情况下,放个屁都会沉下去!”

“狗日的!”我们一下就紧张起来,马上都站了起来,看向四周的黑暗。心说到底在哪里。

阿宁此时保持了相当的镇定,她拍了拍手让我们不要慌乱:“冷静冷静,他们能发出信号表示他们现在暂时安全,我们能收到信号,说明他们的对讲机肯定就在附近,我们应该能很快到达。”

“但是说是附近,这附近也非常大啊。怎么找?”

阿宁让我们跟着,开始拿着对讲机四处走,判断信号传来的方向。

我一看对啊,我怎么没想到,枉我也算是个博学的人,在这种地形中,能够收到无线电信号,必然在四周有无线电波衍射的缺口形地形,而且无线电衰弱程度的大小,和距离密切相关,所以通过对讲机对无线电波的接受程度就能判断我们是否在靠近。

我们马上跟上去,走了一圈,就发现峡谷的深处信号最响,显然发出信号的源头在峡谷里面。阿宁招呼了一声,我们就快速往里面跑去,同时手电已经甩开了来照,扎西大叫:“当心脚下,别光顾找!”

我们也管不了这么多了,一边跑一边找,很快峡谷就到头了,在我们面前,出现了一座巨大的半月形土丘堵住了去路,足有五十米高,好像一面巨大的风帆,非常陡峭,看上去没法爬过去。

懂对讲机的人一看就知道情况了,这样的地形,无线电信号是最弱的,这和在大山的山谷中信号差是一样。然而我们看向对讲机,那声音现在已经十分的清晰,丝毫没有减弱。那就是说,发出信号的东西绝对就在这个半月形土丘围成的大概一百一十米长宽的区域内。

“就在这里?”我们都冒出了冷汗,感觉到不对,因为手电一扫,这片地方就一目了然,连个鬼影也没有。

“难道已经沉下去了?”我心理出现了一个不好的念头。

阿宁摇头,因为对讲机中的声音仍旧在响,就叫了一声让我们分开去找。

我们分散开去,仔细的搜索地面的痕迹,很快扎西就叫了起来,有了发现,我们冲过去,发现了地上有非常杂乱的脚印。不是我们的。

“他们就在这里。”扎西道,“这半月形的土丘好比是一个避风港,他们肯定是被狂风逼进来躲避的,而这里面几乎没有风,脚印才会留下来。”

我们马上顺着脚印往前找去,沙质的地面脚印非常的清晰,可以看出是三个人,我们跟着脚印走了十几米远,就来到了那土丘的根部,脚印竟然戛然而止。没有拐弯的脚印,也没有流沙坑。

“靠。走到土丘里面去了?”扎西咂舌道。

“不是!”阿宁露出了一个匪夷所思的表情,她抬头看向土丘,上面一片漆黑,什么也看不见,“他们爬上去了。”

这就怪了,我们都愣了,抬头往上看去,只见背光的土丘是一片漆黑,犹如一团纯黑色的巨大黑幕,我们的手电扫射上去,因为实在是太高了,小小的手电光根本照不出个全貌。

他们上去干什么?难道这土丘上有什么东西?

阿宁这时候让我们退后,然后掏出信号枪,朝天打了一枪。

灼热的信号弹飞上半空,爆炸后把整片局域照得犹如白昼一样,那一瞬间,四周隐藏在影子里的景象全部都显现了出来。

我们全部将目光投向四周,一下这么亮眼睛有点不适应,还没有看清楚,就听到阿宁惊叫了起来:“天哪!”

我们忙眯起眼睛抬头将目光投向半空,在信号弹闪烁的光芒下,我们看到在半月形巨大的山丘的半山腰下,竟然镶嵌着一个巨大的物体,一半埋在土丘的里面,一半则突兀的横在半空。

\chapter{沙海沉船}

在信号弹然后的几十秒里,我们全部都惊呆了。大家都看着那巨大的东西,脑子一片空白。一直到信号弹熄灭,我们才反应过来,随即所有的手电都朝那个方向照了过去。

零碎的光线无法照出那个东西的全貌,在手电的光线下,我们只能知道那里有个东西,然而看上去也是模糊不清的。如果刚才没有信号弹照出的印象,手电扫过我们肯定不会注意到异样。而我们从下往上看,也实在看不分明。

“这是什么东西?”扎西自言自语了一声。

没有人能说出这是什么,我只能肯定这是一块古老的木头物体残骸,只是不知道是什么东西的残骸。这乍一看像一只巨大的棺材,然而仔细看又发现形状不对,似乎是建筑的残骸。然而,我却从来没有见过这样古怪形状的建筑。

“爬上去看看!”不知道谁说了一声,我们才反应过来。他们几个就想往斜坡上爬,我忙把他们拦住,说道:“别乱来,冷静一点,这么高,而且是土丘,不是随便爬爬就能爬上去的,要是除了意外就糟糕了。”

阿宁也点头道:“对,那三个人还没找到,这下面我们都找过了,没有发现任何线索,那么很可能他们在上面,现在一点动静也没有,肯定有问题。说不定这上面有什么危险,我们要小心。还是我先上去看看,如果比较好爬,你们再上来。”

说着她把手电往腰带里一插就让我们给她照明,自己准备往上爬。

这时候扎西拦住了她,道:“别动,我来,这种事情没道理让女人去做。这种土丘我以前爬过很多,绝对比你有经验。”说这也不等阿宁回应,就咬住匕首,跳上土丘,然后用匕首做登山镐,开始向上爬去。

他动作很快,姿态犹如猴子一样敏捷。我们用手电给他照着,几乎没费什么力气。我们就看他“腾腾”爬到了那个巨大物体的下方。他找了一个地方站稳,就对我们做了个手势,意思是不算难爬,接着他就用手电去照那个东西。

在下面我们只能看到他的动作,也看不到他照出了什么,心里很急,那队医问道:“那是什么东西?”

“我不知道。”扎西的声音从上面传下来。我看他在上面挠了挠头,冒了一句藏语,然后说道:“天,这……好像是艘船啊。”

“船?”我们互相看了看。扎西就又叫了起来:“真的是船!你们自己爬上来看看。”

他刚说完阿宁就爬了上去,我动作笨拙,跟着阿宁。而队医太胖了,爬了几下就滑了下去。我们让他在下面待着,别乱来,等一下摔死就完了,然后朝扎西靠拢过去。

这土坡确实不难爬,有点坡度,虽然土很松软,但是上面十分不平整,很多地方都可以落脚。我们学着扎西用匕首当登山镐,三下五除二就靠了过去。

我手脚并用的爬到扎西的边上,这上面很冷。我踩着几处突出的土包,滑了一下后站稳脚跟,就朝那东西看去。不过我离得远,视线又给扎西遮住了,也看不清楚那船是不是真的船。

我挪了一下,给自己挤出一个位置,这才看清楚。在扎西的手电下,一块古老的残骸镶嵌在土丘里,只露出一半,另一半深深的插入土丘,看形状,确实是一艘古代的沉船。

阿宁点起一个冷焰火,就往沉船上扔。此时四周亮了起来,我发现这沉船的解体程度已经非常眼中,几乎和那些泥融成了一体,木头的船身完全破碎了,已经炭化。在木船的一边还有一条巨大的裂缝,里面似乎是空的,我能看到里面的泥,但是最深的地方却漆黑一片看不清楚。

我转头看了看四周的地貌,心想这可是大发现。这里以前应该是古河道,这条沉船沉没在古河道里,被裹在了淤泥里。没想到沧海桑田,古时候的河道竟然变成了戈壁,而且这包裹这沉船的土丘,竟然高出了地面这么多。

阿宁爬到那古船的边上,用手电照那个裂缝,就照出里面大量的泥巴和裹在泥巴里的东西。在泥巴里,还能看到很多类似陶罐一样的东西。

阿宁道:“这似乎是艘去往西域通商的货船,这些是他们的货品,着简直是惊世的发现,现在还有很多人认定西域没有水路运输。”

古时候这里是十七条丝绸之路中比较险恶的一条,而西域各国就分布在这片荒芜的土地上,这里是阿拉伯文明和中国文明交易的中间地带。以前这里无数的河流上非常的繁闹,不知道有多少布匹和丝绸通过这些河道到达了西方,据说西域各国的皇室还能吃到中原的西瓜。当时这里的河道千变万化,也有不少的商旅因为古河改道而搁浅沉没,这里的沙漠深处起码被掩埋着上千艘沙漠沉船,然而因为沙漠变化太频繁,几乎无法寻找,没想到这里竟然有一艘。

队医在下面什么都看不到,很心急就大叫:“看到什么?那三个人在不在上面?”

扎西对下面叫了几声回答他,队医又说了什么就听不清楚了。

这时候我突然想到高加索人,可能他们也是因为看到这艘沉船,然后才爬上来查看的。下面全找过了,没有发现什么人,他们应该就在上面。可是四周的崖壁上刚才看过,什么人也没有,这三个人到哪里去了。

这里的岩壁除了这沉船,没有其他地方能藏人,难道那三个人在这沉船里面?

这时候月亮被乌云遮住了,一下子四周变得更加给俺,我们几个人都找了个位置站稳。我让阿宁打开对讲机,再找找信号的位置。

阿宁拿出对讲机,一打开,那声音就响了起来,非常清晰。她挥动了一下,信号都差不多。接着扎西指了指船,让她对准古船试试。阿宁伸了过去,一靠近那古船的裂缝,我们真的就听到了无比清晰的声音从对讲机里传了出来。

我们互相看了看,都感觉到很不可思议,看样子,信号真是从这古沉船里面发出来的。

扎西看了看那裂缝,说道:“真见鬼,难道那三个白痴爬到里面去了?”

那裂缝很宽,确实可以爬进人去,只是这里面的空间不知道能不能容纳下他们是那个。我们用手电去照,发现这船里面非常深,最里面很黑。我喊了好几声,但是没人回应。

“怎么办?”

“可能是他们进去过了,但是又出来了,然后把对讲机掉在里面了。”阿宁说,“也有可能他们在里面出了意外。”

“那这声音是怎么发出的?”我问道。

“这个没人能回答你,不过进去看看就知道了。”阿宁给我使了个眼色,说着就放下背包,意思好像是让我和她钻进去看看。

扎西是向导,要保存实力。这里就我和阿宁的体型比较正常,我也没法说不行。她脱掉外套,咬住匕首就猫腰先爬进了裂缝里。

一进去,船身上的泥巴就不停的往下掉,还好船身比较结实。她进去后停了几秒,稳了一下,扎西就把手电递给了她。然后我也脱掉外套爬了进去。

这裂口正好能让我爬进去,不过里面比我想的要宽大。我笨手笨脚的进去,发现里面完全是个泥土的世界,头顶上全是干泥,人没法坐起来,只能匍匐前进。本来这船舱内的空间应该很大,然而现在基本上全塞满了泥土,其实我们就在一个泥洞里。

阿宁开着对讲机,此时正在清晰的发出那犹如冷笑一般的声音。那声音在这里格外响亮。看着船舱内部漆黑一片,我的心提到了嗓子眼。到底是什么在发出那种声音呢?

阿宁在里面用了一个侧爬的姿势,就是士兵拖枪匍匐前进的那种动作。她用单手前进,另一只手打着手电开始四处照射。我喘着粗气学她的样子,也开始用手电去照四周的泥巴,真的全是泥,除了零星能看到镶嵌在泥里的一些木片,我感觉好像在地道战的场景里。

这些肯定是沉船之后从破口涌进来的泥土。当时的船应该没有完全沉没,所以泥没有充满整个船舱。这些泥巴下面应该都是当时的货物,不知道里面运的是什么。

往里面爬了七八米,我们就能够直接听到那种奇怪的声音了。没有对讲机的过滤,这声音听上去稍微有些不同,是从船舱的最里面发出来的,很轻。阿宁停了停,关掉了对讲机,就向着那个声音的方向爬去。

我稍微和她保持了距离,给她能够退后的空间。没等爬几步,阿宁惊叫了一声,停住了。我也赶紧爬过去,从她侧面探头过去,就看到船舱尽头给泥土覆盖的“甲板”上有一个圆桌大小的洞,好像是坍出来的。下面竟然还有空间,用手电往下照去,下面一片狼藉,全是从上面塌落下来的土块,一个人就埋里面,只露出了上半身。

我用手电一照,发现那就是失踪的人中的一个,脸上全是泥,脸色发青,不知道是死是活。那冷笑一般的声音,就是从下面的土堆里发出来的。

“真的在里面!”我大叫起来,心说这帮人也太能玩了。我边叫喊着边往前挤,想赶紧下去把他挖出来。

没想到我突然一叫,那种冷笑一般的声音一下就消失了,整个船舱突然安静了下来。

这一静把我吓了一跳,手脚不由自主的停了停。

随即我就想到,刚才我们讨论这声音是他们的求救信号,现在我大喊了一声,这声音就停了,显然有人听到了我的叫声,于是停止发出信号。这有两个可能,一个是他认为救援已经在身边,没有必要再发出这种声音来吸引我们;另一个是,他听到我们到来,信念一松,失去了意识。

无论是哪种,我们都必须马上把他就出来,特别是后一种,我知道很多求救的人就是在得救前一刻失去求生意志而功亏一篑的。

阿宁和我想法相同,她让我给她照明,爬了过去,然后小心翼翼的翻身滑进了那个洞里。我跟着过去,阿宁让我别下来了,在上面接应。

扎西在外面听见了我的叫声,对我们大叫,问里面情况。我让他等等,我看清楚再说。

在这个位置上,看得更加清楚。那洞口下面,应该是古船的第二层货仓,或者叫底舱。一般是用来放置一些容易破损的东西,因为底部的晃动不会很激烈。底舱的空间不大,里面也全部是泥土,但是被侵蚀的程度远远小于我待的地方。我基本还能想象出这是一艘船的内部,可以看到那些泥土里混杂着很多的陶罐,应该是货物,不知道里面装的是什么东西。

阿宁下去了之后,马上就拨开那人身上的土块,然后把放到他的脖子上,感受脉搏。

我忙问:“怎么样?”

阿宁明显颤抖了一下,回头对我摇头,示意已经不行了。

我叹了口气。阿宁开始挖土块,很快把那个人挖了出来,然后用力的拖到一边。这时候我就发现挖出的土块里面,出现了另外一个人。我看到了头发和一只手,阿宁继续挖掘,然而这个人就埋得比较结实。她挖了一会儿也没有起色。

我实在看不下去,自己也跳下塌口帮忙。我一摸到那人的手,心里就一沉,知道也没戏了,那人的手冰凉冰凉的,已经死了。

我们花了九牛二虎之力才把他挖出来,也拖到一边。在这个人的下面,我看到了高加索人苍白的脸庞,他蜷缩着身子,瞪着眼睛,手往前伸着,握着一只对讲机,保持着一个僵硬的手势,好像是想要从里面爬出来。

看来发出信号的就是他,我看到那只对讲机,心想。

我将他拉出来,阿宁又摸了摸他的脖子,脸色一变,“还活着!”就马上解开了高加索人的衣服,然后给他做心肺复苏,同时对我大叫:“告诉扎西,让队医准备抢救,有人本掩埋窒息。”说着就去给高加索人做人工呼吸。

我忙爬起来对外面大叫,扎西听到之后,马上也对土丘下的队医叫了起来。我转头,就看到高加索人抽搐了一下,人缩了起来,同时开始呕吐,但是显然恢复了呼吸。

“你上去接手!”阿宁用一种不容置疑的口吻对我道,语气很平,但是充满了威严。

我愣了一下,突然被她这种神态电了一下,像条件反射一样按照她的说法做了。接着阿宁迅速脱掉自己的衣服,绑在高加索人身上,做了一个简易的担架,把衣服的袖子扔给我,然后叫我用力。

我在上面咬紧牙关用力往上拉,她在下面抬脚,把高加索人运了上来。然后,我一路往后,用力将他拖出沉船的裂缝。

外面的扎西已经在准备了。高加索人刚一被拖出来,扎西就把高大的高加索人整个儿背到了身上,用皮带扣住,然后往下爬去。我累得够戗,一边把阿宁从里面扶出来,一边喘着气跟着,护住扎西,之后一点一点爬了下去。

费了九牛二虎之力,好几次看到扎西差点摔下去,幸亏他反应够快,每次都能用匕首定住身形。好不容易爬到了土丘下,队医已经准备好了一切,我们把高加索人放到地上,队医马上准备抢救。

可是刚撕开高加索人的衣服,他突然就抽搐了起来,一下扯住了队医的衣服。我们赶紧过去把他按住。队医揭开他的外衣,我就一阵作呕,只见他保暖外衣的里面,已经全部是血,竟然好像有外伤。

队医又用剪刀剪开他里面的内衣,当掀起带血的布片时,他叫了一声:“天哪。”这时我几乎要呕吐出来。只见在高加索人的肚子上,全是一个一个细小的血洞口,没流多少血,洞口十分的细小,但是密密麻麻,足有二三十个。

“这是什么伤口?”扎西问道。

队医摇头:“不知道,好像是……被什么东西扎的,类似于螺丝刀这样口径的东西。不过衣服怎么没破?你们在现场没注意到?”

我们都摇头,其实当时这么混乱,我们真没有注意到他的肚子,但是他的衣服没有破洞我们可以确定。应该不是坍塌造成的外伤。

现在也管不了这么多了,队医让我们帮忙按住,先给他爆炸,然后简单的检查了一下,就给他注射了什么东西,最后拿出一个小氧气包给他吸。大概是那一针的作用,高加索人慢慢安静了下来。

做完这些我们已经全身是汗,队医擦了擦汗就让我们想办法。这人现在十分虚弱,我们不能把他带出去,但是那些比较大的设备都在外面的车上,需要搬进来,另外还需要帐篷和睡袋给他保暖,等他稳定下来才能把他带出去。

这里只有扎西知道该怎么看他的石头堆,他就说他去拿,顺便叫些人进来帮忙。我们一路走进来花了很长时间,不过出去就快很多,我说跟他一起,他说不用了,他一个人更快,我在这里多个照应。

说完他就跑开了。队医解开高加索人身上阿宁的衣服,还给她,然后拿出背包里的保暖布,给高加索人的几个重要部位保暖。

我点起无烟炉子,加大火焰,放到一边,给几个人取暖,同时拿出烧酒,这些东西都是为了驱寒用的。我们刚才出了一身的汗,戈壁的夜晚相当的冷,很容易生病。

大火起来,照亮了四周,一下就暖和起来。队医继续处理高加索人的伤口,我和阿宁退到一边,几个小时的疲劳一下子全部涌了出来。我坐到一块大石头上喝水,阿宁披上了衣服,我们两个都是一脸的泥土,十分狼狈。我朝她苦笑了一声,却看到她一脸的疲惫靠到了土丘上,摆弄着对讲机,似乎相当的沮丧。

我想起刚才她那种气势,心说真是不容易,她一个女人能在那种场合干练到那种样子,想来估计也是逼出来的,想想一个女人要强悍到这样,真是有点心酸。

不过说来也奇怪,看她也不像是缺钱的样子,干这种事情也不见她开心。她到底干什么非要为裘德考卖命不可?而且还拼命到这种程度,真是想不通,以后有机会要好好问问她。

喝了几口水就想方便,于是绕了个圈子到了土丘下面放水,在沙漠里这批人都是这个样子,我也习惯了。

尿着尿着,忽然我就听到一边的石头后面,突然传来一声怪异的冷笑,那声音和刚才在对讲机里听到的如出一辙,顿时让我浑身一凉。我转头往那块石头看去,心说难道一直听这个声音,出现幻听了不成?

\chapter{西王母罐}

刚才那一个多小时都是听着那怨毒的冷笑般的信号一路过来,脑子里几乎习惯了这种声音,现在船里突然安静了下来,我已经感觉到有点不适应。不知道为何,现在我又听到了同样的声音出现在四周的黑暗里,那声音我一直感觉到不妥当,这时候听到,心里觉得十分异样。

虽然感觉也有可能是幻听,但是在这种地方还是不要想当然的好,我拉上拉链,打起手电,朝那块石头后面走去查看。

石头很不规则,不知道是什么种类的岩石。这里都是土丘,不知道这些乱石是从哪里来的,总不会是地里长出来的。

石头后面漆黑一片,有一个手电没法照到的死角。绕过去一照,却什么也没有看到,石头后面的缝隙很小,不太可能藏什么东西。我踢了一脚这个石头,发现不太稳,在四周又照了照,也没看到什么,一切都很平静,就心说我也许真的听错了。摇摇头,我就走了回去,阿宁问我怎么了,我告诉她说可能是有点神经过敏,以为那里有什么东西。

坐回到篝火边取暖,两相无话,我靠到了石头上,本来只想闭目养神,怕还有什么事情会需要我们帮忙。然而疲倦袭来,我很快就有点迷迷糊糊,也不知道什么时候睡了过去。

醒过来的时候,天已经亮了,但还不是很亮,好像是清晨。这时风已经完全停了,我听到了扎西的声音,爬起来一看,只见他们都进来了,好像外面的营地给搬了进来,四周搭起了帐篷和篝火。高加索人已经被挪到了帐篷里面,阿宁还在一边的睡袋里休息,有人在四周忙碌着。

我身上多了条毯子,不知道是谁给我盖的,我挣扎着爬起来,打着哈欠,往四周看去。第一眼,我就被四周那些风蚀岩石的景色吸引了注意力,不由愣了一下。

白天的魔鬼城视野极度的宽阔,四周风蚀岩比晚上看上去要壮观的多,拔地而起的巨大山岩犹如金字塔一般矗立在我们的四周。那些晚上看上去黑漆漆的岩石,现在显现出了各种奇异的形态,配上戈壁的无限苍茫,这种壮观的感觉,不是语言可以形容出来的。

这里不是成年的雅丹地貌,要是再经过一百万年的风沙磨砺,这里的景色该壮观到什么程度?

我看着发呆发了一会儿,才回过神,注意到四周的人,他们正在从土丘上的沉船里运出东西来。昨晚的土丘比我看的还要高大的多,在上面打上了钉子和绳子,便于攀爬,还做了一个吊篮,有人在上面发掘,乌老四则在下面接应和整理,东西直接从吊篮上吊下来。

定主卓玛和她的儿媳妇煮了早饭和酥油茶,她看到我醒来,就做了个手势让我去吃。我过去喝了碗茶,拿了一个面包,边吃边走到乌老四身边问他们在干什么。

乌老四听说是行内人,给裘德考招安的,对我有点喜欢,看到我过来就点点头,对我说高加索人的伤势比较严重,队医还在检查他腹部的伤口,有感染的迹象,所以可能队伍要退回去整顿再做打算。他们不想空手回去,这沉船也算是个大发现,他们想记录一下,带点东西出去通报给公司。

我坐到他边上,看了看头顶的沉船,真大!晚上感觉不到有这么大,看上去这船是正规的商船,头部大概是以前土丘坍塌过才露了出来,架在半空,下面已经给上了支撑的支架。

又低头看他们从里面清理出来的东西。那些陶罐一个个都有抽水马桶这么大,出奇的是一个都没有破损,看来沉船的过程十分缓慢。罐子上面有着西域特有的花纹,有些是黑色的图样,有些则是类似于文字的东西,都不是汉人的东西。我问这是什么,乌老四就摇头说没人知道。西域的文化非常特别,非常神秘,而且留存又相当稀少。西域五千多年的历史,这么多城池古城,都给戈壁黄沙掩埋了。在过去的可可西里和塔克拉玛干,古时候都叫做西荒,人口分布十分稀少,现在要研究实在太难了。

“不过这些古陶的历史相当久了,一般我们西域交易都是瓷器。这些陶罐是陶发展到顶峰时候的产物,应该是唐朝以前的。不知道是中原运出到西域,还是西域运出到阿拉伯世界的。这片区域应该已经是西王母国的疆域,不知道是否和西王母国有关系。”旁边另一个戴眼镜的人说。

乌老四就点头赞同,说:“我也感觉很有可能,你看。”他指着一个陶罐上的花纹,那是一只鸟的图案,“这是传说中西王母的图腾致以,三青鸟。当然,也不排除其他国家的人也会使用。因为当时西王母国还是西域的精神重新,因为其诡异的神秘,即使它已经没有周时期的强大,其他国家仍旧敬畏西王母传说中的魔力,而都要来朝奉,或者在形式上表现崇拜。”

我对此完全没什么兴趣,这些属于考古的范畴了,于是就打断他们,问道:“那这罐子里有什么东西?该不会是空的吧,那多浪费。”

罐口都被封着,是用一种特别的泥封上的,绿绿的,黑黑的,有点像酒坛子上的那种泥封口。我闻了闻,有点辛辣的味道,感觉很熟悉,搬了搬,罐子有点分量,肯定里面是有东西,不过不是液体。

我问他们为什么不打开?乌老四说他们尽量不破坏这些完好的,等会儿看看有没有破损的,就不用开了,万一里面的东西比较珍贵,经不起氧化,这样可以节省一下,防止考古浪费。

我就笑了,心说三叔他们可没有这一套,要是胖子在肯定不由分说就砸开。

不过我们得尊重别人的做事方法,我吃完最后一口面包,就和他说那你们自己先搞,到时候找到罐子,打开的时候叫我一声。说着我就走到高加索人的帐篷里,去看他的情况。

走进帐篷就发现很局促,仔细一看,才发现另外两具尸体也搬了下来,躺在一边盖着保温布。队医一个晚上没睡,眼皮明显黑了一圈,正在给高加索人测体温。

我问他情况,他就跟我说了一遍,说人很迷糊,说胡话,但比之前有起色,窒息和缺氧应该没关系了,只是这肚子上的古怪伤口……他让我看两具尸体,也有同样的伤口,一个在胸口,一个在大腿内侧,都出了少量的血,但是外衣上都没有洞,不知道是怎么产生的。

我走到高加索人身边,他的脸色发白,满头是汗,但呼吸器不用了,显然确实是稳定了。我看到他嘴唇一动一动的,好像在说什么,我贴近听了听,不是中文,好像是英文。

“他在说什么?”我问队医。我的英文到底不是怎么样,谈生意还可以,听说胡话就不行了。

队医也摇头,说他也听不清楚,他的英语也不好。不过意识有点恢复之后,高加索人就一直在念叨这个。

我俯下身子,想凑近了听,还是不行,就只好放弃了。走出帐篷,想回去再睡个回笼觉,反正这里也没我的事情。

到了睡觉的地方,躺下琢磨着昨天晚上的事情,很快就眯了过去,不知道睡了多长时间,突然听到有人叫我的名字。我迷迷糊糊的坐起来看,看到乌老四那里围起很多人,他在朝我招手,好像有什么事情。

我爬起来走过去。一走近他们,我就闻到一股极其古怪的味道,说臭不臭,但是闻了就感觉喉咙发辣,好像吸了硫酸气一样,十分难受。我捂住鼻子凑过去看,看到原来是他们找到了几个破损的罐子,正在砸罐子,乌老四让我来看。

有十几个罐子已经给砸碎了,乌老四正在一个一个往外倒里面的东西。我首先看到的就是泥屑,里面全是黑色的干泥屑,在这些泥屑中有一种土球,上面全是泥,非常恶心。奇怪的是,我看到这些球的表面粘着很多的黑毛,看着非常不对劲。

一边已经堆了十几个土球,不知道是什么东西,我心说难道是当年的西瓜,现在都变成石头了?

走近了再仔细一看,我就感觉一阵窒息。我发现,那些泥球竟然都是一个个裹在干泥里的人头,那些黑毛,竟然是人头的头发。

\chapter{鬼头}

我感觉到有点恶心,乌老四他们显然也没有想到这些陶罐里竟然装的是这种东西,都带着既厌恶又诧异的神情。

其他人看人群积聚,也逐渐聚拢了过来,几个藏人司机从来没见过这事情,都很好奇,凑过来看。

我捂住鼻子看着乌老四戴上手套,就捧起人头,清理上面的泥土。这东西年代十分的久远,但是头发还是很坚韧,皮肉都腐烂掉了,掰掉上面的泥土,能看到干瘪的皮肤和空洞的眼洞。这是一个古人的骷髅。

边上那个戴眼镜的人对比了一下人头和罐口的直径:头骨大,陶罐口小,显然人头是放不进陶罐的。

这是怎么回事,我就问他。

“这就是西王母部落的诡异传统,这个肯定是西域其他部落的奴隶,可能在两三岁的时候他脑袋就给装进了这陶罐里,然后一直长到成年,脖子和陶罐的缝隙里塞不进食物为止,那时候他脑袋早就出不来了,接着就砍掉他的头,把这陶罐封起来,献给西王母做供品,这是人头祭祀的传统。”四眼说道。

“我靠,这也太邪了,咱们西游记里的西王母挺和蔼的,不像这么阴毒的啊。”一个人咂舌道。

“那个西王母是中原人化的西王母,真实的古代传说只能够,西王母是个厉鬼一样的东西,根本就不是个人。”有人就给他扫盲,“当时的那个年代,靠和蔼统治不了人,统治者都是靠这些神秘主义的诡异残忍的仪式,渲染自己的超自然力量进行统治的。”

我就问乌老四,那这人头为什么要放在这个罐子里?砍了就砍了,何必这么麻烦。

乌老四就道:“有很多的西域部落,都认为人死之后灵魂是从眼睛或者耳朵里飞出去的,放在陶罐里杀头,就是为了把这个人的灵魂困在这个陶罐里,这样献祭祀才有意义。祭祀完成,这些人头一般都会堆在一起,喂食乌鸦这种东西,或者抛进海水里喂鱼。这在中原也一样,我们叫做鬼头坑,河北易县燕下都有一个‘人头墩’,和这种类似。”

我听这就觉得脖子很不舒服起来,这样的事情也只有在蒙昧时期才有,然而我有时候真的怀疑这到底是谁第一个先发明的?古人是什么时候开始信奉起这种血腥的东西?

“可是把他的头从小塞进这种陶罐里,他平时怎么生活啊?”有人问。

“生活?你不要说,祭品的生活相当的优越,被选择为祭品的人一般吃的都是给神的食物,是整个部落最好的食物,平时根本什么都不需要干,性成熟之后马上就有最美丽的少女和他交配,以便怀上下一代的祭品。为了让他的脖子尽快长到足够粗,他们会限制祭品的活动,有些人吃得太胖,还没到年龄就被陶罐口勒死了。”有一个人道:“比起来,那些在外面累死累活的干活,可能连三十岁都活不到的其他奴隶,舒舒服服活上十几年然后痛痛快快的死掉,也许是个不错的选择。”

那人就摸着下巴:“这听上去倒不错,俺对吃没兴趣,不过最美丽的少女俺有兴趣,要是俺当祭品,俺就不吃东西,让脖子长不粗,然后就可以……”

话没说完,那些藏人司机都笑起来,我拍了一下他脑袋,骂道你他娘脑子里全是什么东西。

大家笑了一会儿,乌老四就开始用一种溶液来洗涤头骨,这是考古作业,几个人围着看也没意思,有人就在一边拍手,让他们都回去干活,作撤退的准备,修车的好好去修车。准备好我们就出发了。

人还没走开,突然,所有人都听到了一声诡异的冷笑,清晰无比的从人群里传了出来。

一下我就一身的冷汗,几个人都停了下来,互相看了看,我看到他们的表情就知道自己不会听错了,心都吊了起来,心说到底是怎么回事?谁在笑?

由不得我多想,那种冷笑声又响了起来,这次有了准备,我们全部顺着冷笑声望去,就发现,那声音,竟然是从一边堆着的人头堆里发出来的。

乌老四吓得把手里的那人头丢到了地上。我头皮就一麻,心说怎么可能有这种事情。就在这个时候,几个人突然跳了起来,然后尖叫,有人就大叫:“看,人头在动!”

我赶紧去看,只见那头骨堆里的一颗人头上,泥土正在裂了开来,人头在晃动,好像活了一样。我几乎窒息,心说怎么可能?这时候,在泥土开裂的地方,突然破了,两只血红色小虫子爬了出来,每一只都只有指甲盖大,十分的眼熟。

我一看,脑子就嗡了一声,简直不敢相信自己的眼睛,还不信,再仔细一看,顿时魂飞魄散,那竟然是几只蟞王!

我脚都软了,几乎是连滚带爬的退后了几步。就看着,两只,三只,四只,然后是一团红色的虫子从里面喷了出来,和我当时在鲁王宫里看到的那种一模一样!一下就爬得到处都是。

“我靠,这是什么虫子,我从来没见过。”这时候有人还奇怪,就看到一个藏人司机走了过去想仔细看。我大叫了一声:“你他娘的别白痴!有毒,快退后,不能碰!”

那人就回头看我,才一回头,突然一只蟞王一下飞了起来,停在了他的肩膀上,我大叫不要!已经来不及了,他条件反射就一抓,“啊”一声惨叫,他就像被烫了一样,马上把手缩了回来,一看,只见犹如一片潮水一般的红疹在他手上蔓延了开来。

四周的人都尖叫起来,纷纷后退。他看着自己的手迅速的好像融化一般的变成红色,惊恐万分,就大叫:“队医!队医!”一边摔倒在地上。

有人上去扶他,有人就往队医的帐篷跑去,我知道那人已经完了,暗骂了一声,冲上去拉住那些上前的人,对其他人大叫:“不要碰他,碰他就死!别发呆,快想办法弄死这些虫子,等它们全飞起来我们就死定了!”

那些人这才反应过来,开始后退操家伙,几个司机脱下衣服就去拍那些虫子。然而没用,那些虫子迅速的分散了开来,拍死的没几只,爬出来的更多。很快又有两个人惨叫了起来。

混乱中乌老四拿起边上一个工具盒就朝那颗人头砸了过去,那人头早就酥化了,一砸就全碎了。我一看,天哪,整颗人头的颅腔里几乎像蜂巢一样了,全是灰色的卵和虫子,恶心的要命。

我的后背全是冷汗,心说看来那眼镜说的事情完全不可信,这人头肯定不是用来祭祀这么简单,倒像是用来养虫子的培养基啊,难道这种蟞王是在人的大脑里产卵了?我靠,要这虫子飞到城市里去,传统四害的地位要不保了。

“糟糕了,其他的人头也动了!”这时候又有人大叫起来,我也没空去顾及了,所有人飞快的后退,接着我就开始听到嗡嗡嗡的声音,有红光飞了起来。一下子几道就从我耳朵边飞了过去,吓得我一缩脖子。

那一刹那,我脑子里第一个念头,就是晚了,这一次要死不少人了!刚想完,果然又有人惨叫起来,我转头一看,就看见乌老四倒在地上,痛苦的翻滚起来。再往陶罐的地方一看,只见血红一片,整片沙地上都是红色的斑点。无数的蟞王已经飞了起来,四周充斥着翅膀的声音。

这已经根本没法去处理,一只蟞王弄不好就能杀光我们这里所有的人,不要说一万只。我心说这他娘的哪里是祭品,明明是武器,这东西就是当时的原子弹啊,谁要是不服气,往他城池里扔进一个,他娘的全城都可能死绝!

现在只能放弃营地,逃命再说了,我冲到帐篷里,那边休息的人已经听到动静走了出来,看到我跑过来,问我怎么回事,我也说不清楚,就大叫别问了,快逃命,到外面车子的地方再说!

几个藏人司机从帐篷里把高加索人背了出来,扎西背起了定主卓玛已经一路跑得没影了。

看着陆续有人跑出来,我心里稍微安了安,跑去叫阿宁。阿宁已经被惊醒,刚站起来,我冲过去拉起她来就跑,她还一下挣脱我,问我出了什么事情。

我大叫你跑就是了,问个鸟事情!话没说完,突然一只蟞王就嗡一声从我额头飞了过去,一下撞倒了阿宁的肩膀,翻了一下停住了。

阿宁低头一看,吓了一跳,想用手去拍。我一看,忙抓住她的手,然后用力一吹将那只蟞王吹飞掉,拉起她往外跑去。

闷油瓶和黑眼镜在外面看车,我们得先跑到那个地方再说。一路就狂奔,也不管三七二十一了,跑出去三四百米,就看到了一个石头记号,我脑子一僵,突然意识到我根本不知道怎么出去,这里的石头记号,只有扎西看得懂。

\chapter{启示录}

我们只得停下来,往左右看看,这里是一个十字路口,这阿拉伯石堆就在最中央,也不知道是什么意思。

我回头看看,远处那让人窒息的“嗡嗡”声,以及乱成一团的那种类似于冷笑的声音——也不知道是它们的叫声还是其他的原因发出的——我还是觉得头皮发麻。

一边跑得气喘吁吁,几乎上气不接下气的阿宁就问我到底是怎么回事,她显然已经知道了事情的严重性,但是还没有反应过来。

我把发生的事情,以及蟞王的毒性说了一遍,一听到乌老四已经中招了,阿宁的脸色就白了。

刚说完,就听到“嗡嗡”声靠近了不少,抬头去看,就见远处这些蟞王正在四散开来,更多的已经飞了起来,天空中出现了一大片红色的雾气一般的虫群,好像集团起飞的马蜂一样,全部朝我们这里来了。

我一看心说我操,没时间琢磨了,拉起阿宁,站起来拔腿就跑。

那时没命的跑,我从来没想过我这么能跑,也不管什么阿拉伯石堆了,一下就冲出去了,足跑了一千多米,在山岩间绕了十几个方向,实在跑不动了,才慢了下来。

回头一看,半空中全是虫子,那红雾一般的虫群竟然跟着我们来了,铺天盖地,速度非常快,直压在后面。

狗日的,我大骂了一声,努力压住晕眩继续往前跑,阿宁体力比我好,这时候跑得比我快,她叫了一声:“不要光跑,找地方躲!”

话音刚落,我们面前就出现了一个缓坡,我没有准备,一下踢到了什么,一个趔趄就滚了下去。

一路滚到底,阿宁把我扶起来,我已经晕头转向,她拖着我继续狂奔,一连冲出去几百米,前面突然出现了一大段犹如城墙一样的山岩挡住去路。我们马上转弯,顺着山岩狂奔,想绕过去,可跑到了一般,就看到山岩的另一头竟然是封闭的,这里是一个封闭的半圆形,是死路。

我看到这个情景,大骂了一声,又回头看后面,只见后面的红雾盘旋着就来了,直接从山岩的顶上铺天盖地的罩了下来。

我一看完了,逃不掉了,看这些蟞王的行为,竟然像是在捕猎我们!

但是我也不想坐以待毙,就到处看是否有藏身的地方。然而这里都是石头,根本藏不下人。

正叹气的时候,忽然一边的阿宁大叫:“到这里来!”

我回头一看,原来那岩山上有一个凹陷,根本躲不进人,不过那是唯一能躲避的地方了,只有看运气了。

马上冲了过去,和阿宁蹲着缩进那个凹陷里,我脱掉T恤挡在面前。

接着,透过衣服我就看到一大片虫子降了下来,空气中突然炸起了一股嗡嗡声,辛辣的味道充斥着鼻孔,很快,无数红色的轨迹把我们包围了。很多虫子撞到了凹陷边的山岩上,发出吱吱的声音,好像子弹在朝我们扫射。

我感觉一阵窒息,人就不由自主的往那凹陷里面退,然而凹陷就这么点空间,再退也没办法把身子完全缩进去。

我几乎是闭着眼睛等死了,这么多虫子,只要有一只碰巧撞进来,后果都不堪设想。我内心深处不认为我们会这么走运,几乎是在等待那一刻的到来。

令我惊奇的是,那种紧张之下,我反倒没有一丝恐惧,脑子里几乎是一片空白。

然而我没有想到的是,慢慢的,外面的声音竟然减小了,一点一点,那种虫子撞击岩山的声音也稀疏起来,很快,外面就恢复了平静。

我咬牙咬了很久,直到阿宁拍我才反应过来,探出头一看,蟞王群竟然已经飞走了,外面零星的几只蟞王,撞在第上晕了,我看的工夫,也一只一只的飞了起来。

我和阿宁面面相觑,不知道是怎么回事,不过都松了口气。我往身后的石头上一靠,就怪笑起来,这他娘的太刺激了,我神经吃不消啊。

笑了几声,就给阿宁捂住嘴巴,轻声道:“看来它们不是在追我们,可能是想飞出去,我们碰巧和它们同一个方向,你也别得意忘形,待会儿把它们再招来。”

我一想也是,忙点头,阿宁才放开手,我不再说话,又在凹陷里待了一会儿,才小心翼翼的探头出去。

外面的魔鬼城一片寂静,好像刚才的惊心动魄完全没有发生过,只是我们的想象一样。

我深吸了几口气,才最后镇定下来。这时候,刚才狂奔的疲劳显现出来,一下腿就抽筋了,趔趄了几下,绷直了才站住。

一瘸一拐的,我们找了几块石头,检查了没有虫子才坐下来,我摸着腰间的皮囊,想喝水,摸了一把,发现自己什么都没有带出来。

随即想起来,出事的时候我是刚起来,甚至连外衣也没有带,好在是白天,晚上就可能会冻死。

回头一看阿宁,发现她连我都不如,穿着短背心,刚从睡袋里出来,头发蓬乱,再仔细一看,似乎连胸罩都没戴。

我一下有点尴尬,想着当时拉她逃命实在是太急了,只好把目光移开。

“这些到底是什么虫子?你了解多少?”阿宁问我道。

我心说我怎么对你说呢,我虽然听说过很多次,但是实际看到这也是第二次,之前就是在鲁王宫里,虫子是在血尸体内爬出来的,当时只有一只,就差点让我们全部死在那里。而今天这么多,铺天盖地一起出现,我也是第一次看到。

把自己知道的一些情况和阿宁说了,阿宁显然十分的不能理解,这一切发生得太突然了。她对我的话半信半疑。

我自己也感觉这有点难接受,也没有心思去和她详细的解释。我心里觉得这应该和我们要找的西王母古国有关系,这些人头罐也许是当时培养蟞王的容器。我三叔也说过在海底墓穴里看到过这样的人头,看来这种蟞肯定是在人的颅腔里繁殖的,而且能保存活力相当长的时间,非常的可怕。不知道西王母古国要这种可怕的虫子来干什么呢?是当成武器吗?

如果当时西王母真的能够运用这么可怕的生物武器,那这个野蛮而落后的古国却能够统治西域这么久,原因可能就在这里。

一边想,一边往四周打量,我们逃到了什么地方,看了一圈,这块封闭的城墙内的区域完全的陌生,一点印象也没有,刚才跑的时候也不知道绕了几个弯了,我们彻底的走乱了。

我们是一路往东北偏北的方向跑,根据扎西的说法,这里有八十多平方公里宽,我们现在在哪个位置不知道,不过不会是魔鬼城的边缘地带,前面还是看不到广阔的戈壁滩。

魔鬼城里的“街道”,也就是风蚀岩山只见的距离非常宽阔,虽然这些岩山形态各异,但是只要角度一变,看出来的东西就完全不同,我也无法在这么短的时间去记忆这些,加上宽阔的视野,视觉纵深非常深远,很干扰人的方向感。相信走回去也不太可能了,我们只能看准一个方向先走到戈壁上,然后顺着魔鬼城的边缘,绕一个圈子回到车子抛锚的地方,和闷油瓶他们回合。

那些虫子不知道生存能力怎么样,现在天上全是积压云,阴天没有太阳,如果它们乘风飞上马路,后果不堪设想。不过,这里离公路线已经相当远,又没有水源,我想只要太阳出来一晒,这批虫子应该活不了多少时间。

把我的打算一说,阿宁也觉得可行,现在我们身上什么都没有,必须在天黑前赶到,不过现在才中午,时间还充足,而且没有太阳,这对我们来说是万幸。

确定了走法,我们又休息了一下,就开始上路。我看了一圈四周,记住了四周几块岩山的样子,都是好像城堡的炮楼一样,如果我们不幸走了回头路,那么如过走回到这个地方就能察觉。

当时,我以为最多为费点腿脚。谁也没想到,这一走,会走得这么痛苦,几乎走到阴曹地府去。

我们迷路了。

穿行在魔鬼城里,我们并没有放松警惕,那些毒虫子不知道现在飞到什么地方了,如果走着走着又碰上,那刚才的死里逃生就是个笑话。

于是一边前进就一边注意着四周的声音,不知道什么时候,风又起来,魔鬼城里出现了各种各样诡异的动静。好在风不是非常大,这么听着也是轻轻的,若隐若现,不至于干扰人的神经。

我和阿宁没什么话说,而且她衣衫不整,和她并排走在一起,我的眼睛总是要忍不住看她,所以我干脆就走在前面。两个人都不说话,就是偶尔停下来交流几句。

她也没什么表情,显然也是心力交瘁,没有心思考虑更多的事情。

说实话,如果是在旅游,和一个美女两个人行走在这片诡异的魔鬼城里,看神妙莫测的风蚀岩山,听魔鬼的哭号,虽然不是什么靠谱的事情,但是也不失为一件美事。偏偏这个世界就是如此的奇异,看着我们两个人简单在这里行走,其实,就在刚才我们经历了死里逃生,这种情况下,我就是再有闲心也不会觉得这情景是美好的。

就这么走着,最开始的三个半小时,还真有点像旅游,看着奇形怪状的山岩,我有时候还会产生错觉,想去摸照相机。

半个小时之后,口渴就开始折磨我们,水分从汗水里流失掉了,我和她的嘴唇都干肿了起来。说起来我早上还喝了一杯酥油茶,阿宁什么都没喝,但是实际上我们两个的感觉都是一样。

这种口渴是十分难受的,我们舔着嘴唇,努力不去想这个事情,才能继续往前走。也亏得没太阳,否则这时候,我可能已经中暑了。

又走了个把小时,在我最初的概念里,这个时候应该已经到达魔鬼城的边缘了。

我们停了下来,喘口气,然而四周看去,仍旧是不变的景色,都是那种高大的风蚀岩山,没有戈壁的影子。

我多少有点异样,这距离有点太长了,假设我和阿宁每小时只能走五公里,这也有十五公里的路了,这片魔鬼城绝对没这么长,显然我们在走弯路。

然而,一路过来,我很用心的记忆了很多特征明显的岩山,以防走回头路,但是都没有看到,显然我们确实还在往前,并没有绕圈。

这多少有点让我放心,我自己安慰自己,也许是我们的脚程不知不觉放慢了,或者走的路线曲折得比较厉害,不用担心,只是顺着一个方向,就能走出去。

这时候不能休息,因为天色渐晚,我估摸着这里虽然不是戈壁,但是离戈壁也不远了,应该用不了多少时间就能出去,出去之后还得花点时间回到魔鬼城外的营地,着也需要相当长的时间。

于是,我们继续赶路,还特意加快了脚程。然而,越走我就逐渐感觉到不对劲,时间一个小时一个小时过去,四周的景色还是如常,好比这魔鬼城在跟随我们移动一样。

硬着头皮坚持,一直走到天色抹黑,还是不见戈壁滩的影子。我已经意识到了问题的严重性,这绝对不是什么脚程慢可以解释的了,这样走,不说八十平方公里,就是再大一倍,我们也应该到边了。

一股寒意涌上背脊,看来这魔鬼城里的情况比我想象的要复杂得多,不单单是有很多岩山而已,我们迷路迷得非常彻底。

天色逐渐暗淡,夜晚又要来临了,这个时候,我就感受到了当时高加索人和另外两个牺牲者在这里迷路的感觉。正琢磨着该怎么办,后面的阿宁已经把我叫住了。

一停下来,两个人精疲力竭,谁也走不动了,空气中的温度陡然降了下来,我们的汗水开始冰凉起来,这里的昼夜温差太大了。

“不能再走了。”阿宁往地上一坐,对我道,“天黑前肯定走不出去了,我们没有手电,这里全是石头,也没法生火。只得趁天没有完全黑下来,找过夜的地方。今天晚上连月亮都不会有,这里肯定一片漆黑。”

我也软倒在第,抬头看天,只见天上一片黑云,云压得更低了,夕阳的金色光芒从云的缝隙里如剑一般刺下来,形成了一个巨大的金色十字,十分的壮观,这么厚的云,如果风不大起来,是吹不走的。

当夜我们就用石头搭了一个石头槽,在里面窝了一个晚上。我和阿宁身上就只有单衣,我还有点不好意思,但是阿宁直接就缩进了我的怀里,两个人抱在一起,互相取暖。夜晚的魔鬼城里一点光线都没有,你简直就无法想象那种恐惧,整个空间你什么都看不到,只能听到各种各样的声音从四周传来,甚至还能听到有些声音从你身边经过,好像有东西在魔鬼城穿行一般。

这种情况下几乎是完全睡不着的,我们只好聊天消磨时间。

期间,我们就讨论为什么会走不出去,想了很多的可能性,就是扎西给我们的信息是错的,也许这里的魔鬼城远远不止八十平方公里。阿宁说,如果明天再走不出去,就找座高点的山崖,爬上去看看。

想来也奇怪,我和阿宁并不熟悉,如果是平时这么亲昵的举动,我可能会觉得非常的尴尬,然而这时候我却觉得无比的自然。

这也算是温香软玉,可是我一点想法也没有,突然就想起了柳下惠,突然很理解他。他当年也是在严寒之夜拥抱着一个女子,没有任何越轨之事,我也是一样。想想,要是一个男人在沙漠里走上一天,然后半夜在近零下的温度里去抱一个女人,就算是个绝世美女恐怕也不会有任何越轨的举动,因为实在没力气了。

我几乎是一个晚上没睡,只眯了几下,也都是十几分钟就醒,一个晚上我都在想乱七八糟的事情,想得最多的还是睡袋和帐篷,想着那些藏人的呼噜,当时怎么睡也睡不着,还埋怨睡帐篷对颈椎不好,现在显然想到那睡袋就是感觉浑身的向往。

早上天一蒙蒙亮,我们就爬起来,那状态很糟糕,我从来没有这么累过,感觉身上所有的肌肉都不受控制,眼睛看出去都是迷糊的。特别是口渴,已经到了非常难以忍受的地步,连嘴巴里的唾沫都没了。

我自己知道自己的身体,心里有些慌乱,就和阿宁揉搓着自己的双臂开始赶路。

继续走,这一次是阿宁走在前面,因为她晚上还睡了一点,比我有精神,我们继续按照昨天的走法,一路下去。很快,又是三个小时,无尽的魔鬼城,这时候比无尽的戈壁还要让我们绝望,我看着远处望不到头的岩山的重重黑影,实在想不通这到底是怎么回事。感觉我们就像被关在一个巨大沙盘里的蚂蚁,被一种莫名的力量玩弄于股掌之中。

熬过了一个小时又一个小时,很快就到了中午,这时候我才开始有饥饿感,但是这和口渴比起来,简直可以忽略不计。我的喉咙都烧了起来,感觉一咳嗽就会裂开来。

走到后来,我们实在忍不住了,阿宁就开始物色岩山。但是一路过来岩山都不好爬,最后我们找到了一座比较高大的土丘,咬紧牙关爬了上去,站到顶上往四周眺望。

然而也没有作用,这里的岩山都差不多高,我们目力能及的范围内,全是大大小小的石头山,根本看不到头,再往外就看不到了,但是能肯定的一点是,我们绝对不在魔鬼城的边缘。

我和阿宁愣在那里,心说这到底是怎么回事,为什么我们怎么走,都好像是在这魔鬼城的中心?难道,有什么力量,不想我们走出这个地方?

我们爬回到山丘下,找了一个有凉气的地方休息,我和阿宁商量怎么办,这好像已经到了绝境。我们走不出去,身边没有任何的食物和水,再过一段时间,我们连走路的力气都不会有了。可能会死在这里。

我心中琢磨着,冒出股股的凉意,已经在考虑人不喝水能活几天。

在阴凉舒适的环境下,据说是三天时间,但是现在我们一路走过来,已经走了整整一天一夜,体液的消耗非常大,我估计能够撑到三天已经是极限了,据说喝尿能多活一天,可是狗日的我哪里来的尿。

想着一阵绝望,也就是说,就算我在这里不动,也最多只能活两天时间,如果没有人来救我们,而我们又走不出去的话。

阿宁显然也作着同样的打算,她低着头。

接下去怎么做,这是一个很简单的选择题,继续走,也许能够走出去,然而如果失败,则明天就可能是我们的死期,我们会在这里脱水而死;而不走,等待别人的救援,希望十分的渺茫,也最多能活两天时间,还是会死。

阿宁是性格很强悍的人,我虽然有放弃的念头,但是在生死关头,倒也不算糊涂,我和她最后合计,就是继续走,走到死为止。

不过阿宁此时比我要冷静,她开始做一些石头的记号,并且拆下了她手链上的铜钱,她有一条铜钱穿起来的手链,压在石头记号下。她说如果有人在找我们,那这是一个希望,最起码,他们能发现我们的尸体。

这些铜钱相当的值钱,放在这里当记号,相当于放了一块金砖在这里,我想着这可能是世界上最昂贵的记号,可惜,它指引的是我们的葬身之地。

接下来的两天,我们继续在这魔鬼城里穿行,我都不知道自己是怎么度过这段时间的。

三天三夜滴水未进,到了最后,连意志力也没有了,好比一个行尸走肉。

从第二天的夜里起,我的一切直觉都不再清醒,我看见的东西,都是沙砾的戈壁和四周高耸的岩山,这些景色有时候甚至在旋转,我不知道是自己在转,还是真的天在转,我已经分不清楚,到底哪些事情可能发生,哪些事情是不可能发生的。有时候我就感觉自己已经死了,自己是在飞,然后下一秒,我就看到阿宁在我面前蹒跚的前进,煎熬还在继续。

此时我还在期望,期望着能突然看到广阔无垠的戈壁,或者前面的岩山一过,我们就能看到戈壁了。然而,除了岩山还是岩山,好像怎么都走不完似的。

最后终于,阿宁先倒了下去,我看道她一下就消失在了我的视野里,那一瞬间,我有了瞬间的清醒,接着我就绊到了东西,也滚到了地上。

我不知道自己到底是绊到了什么,也不知道自己是摔在石头上还是沙地上,那一刹那,我就看到了天,那不是蓝天,是黑沉沉的乌云。

我心里苦笑,如果不是没有太阳,我想我现在已经开始腐烂了,可是,就算给我多活了几个小时,时间也到了。

看着乌云,我想站起来,可是根本没处用力气,眼皮越来越重,在完全合上的那一刹那,我忽然看到天空闪了一下,好像是闪电,接着,一切都安静了下来,一切都远去了。我缓缓的沉入了深渊之中。

\chapter{第一场雨}

那一刻,我迷迷糊糊的以为自己就要死了,心理也已经认命,心说死就是这种感觉,那还不错。

就这么意识混沌着,这种迷离的状态也不知道持续了多久,慢慢的,我感觉到好像有什么东西在拍打我的脸,这种感觉非常的遥远,但是,一点一点的清晰起来。

接着知觉开始复苏,我逐渐的恢复意识。一开始还只是朦胧的感觉身体回来了,到后来意识开始清醒,我才逐渐对四周有了感觉。

首先感觉到的是凉,非常的凉,一路走在魔鬼城,精神上的压抑和低矮的云层让人非常气闷,这四周的凉就特别的舒服,好像给浸入到了冰水的浴缸里面。

接着我就发现那种嘴唇干裂的感觉没有了,嘴唇上凉凉的,好像有一股冰凉的东西在往我嘴巴里钻。我舔了一口,又舔了一口,再舔了一口,就发现那竟然是水!

难道有人在救我!我心中狂喜悦,此时身体已经做出了反应,我拼命的吮吸,用我最大的力量动着嘴唇,一点一点,就感觉一股冰凉开始进入我的五脏六腑。

喝完水,我又沉沉睡了过去,在失去意识的一刹那,我好像听到了几个熟悉的声音在说话,听不分明,也没有力气去注意,瞬间就又失去了知觉。

再一次醒来,感觉睡了很久很久,各种各样的知觉就一起回来了,听觉、触觉,我的力气开始恢复,意识也越来越清醒,最后我终于睁开了眼睛。

首先映入眼帘的是一张粗犷的大脸,十分的熟悉,在对着我傻笑。

我看到这张脸,立即就觉得有点不对劲,又想不出为什么有这种感觉。这是谁呢?我闭上眼睛想了一下,搜索着那些藏人司机的脸,是那个开876的?不是。那个开取水车的?也不是。

想来想去想不出这个人是队伍里的哪个,随即我就一个激灵,马上意识到为什么,不对,这不是队伍里的人,这是……嗯?这脸不是王胖子吗?

我脑子紧了一下,啊?王胖子?他怎么会在这里出现?不可能啊?他已经回北京了啊。

难道我在做梦?出现幻觉了?

又睁开眼睛,还是那张熟悉的胖脸,满脸的胡楂,比在北京的时候老了点儿,就这么瞪着我,凑得更近了。

我又闭上眼睛,感觉不正常,不对不对,不可能是王胖子,我就算做梦也不会梦到他啊。

我用力的咬了咬牙,第三次睁开眼睛,这时候,我的脑子已经非常清晰了,一看,确实就是王胖子,他点起了烟,正转头对着身后说着什么。我的耳朵还不清晰,听不清楚他在说什么,接着,我就看到另外一个人头探了过来,也是十分的熟悉,那竟然是潘子。

怎么回事,我皱起眉头,心说难道自己根本没没进戈壁,还是在杭州?之前的一切,都是我的一个梦?

回忆遇到的事情,大量的记忆涌了上来,我们遭遇沙暴,车抛锚,人失踪,镶嵌在土丘内的沉船……一切都非常的真实,绝对不可能是做梦啊。

这时候我的耳朵恢复了听觉,我听到潘子说了一句:“小三爷,你感觉怎么样?”

我用力弓了一下背,就想坐起来,潘子上来扶我。我坐起来长出了一口气,就看到四周的情况,这里好像是一个山洞,里面生着篝火,我看到几个睡袋和装备丢在四周,洞外一片漆黑,显然已经是晚上了。

同时我看到闷油瓶坐在篝火的边上,正在煮什么东西,而阿宁躺在另一边的一个睡袋里,还没有醒过来。

我逐渐意识到自己不是在做梦了。“这是怎么回事?”我按摩了一下太阳穴,问潘子:“你们怎么在这里,我不是在做梦吧?我不是死了吗?”

“不是死了,是差点死了。”胖子在边上道,“要不是你胖爷我眼尖,就看不到这东西,那时候你们已经在发臭了。”

我看着胖子玩弄着几枚铜钱,就知道是阿宁的记号,不过我还是搞不清楚。

“那你们怎么在这里?”我奇怪道。

“我们一直跟在你们队伍的后面。”潘子道,指了指闷油瓶,“你不知道,其实你们进戈壁之后,三爷的队伍马上跟了上去,你们每一个宿营地,这小哥都有留下记号指引我们,我们就保持着和你一站的差距,一直在后面。”

“什么?”我一下没听懂潘子的话,“记号?在我们后面……他……”

潘子道:“这是三爷的计策,这小哥和黑瞎子都是三爷安排和那个老外合作的,目的是为了混进队伍里。因为三爷说事情到了这一步,想自己弄明白裘德考的真正目的已经不可能了,他只有通过这种方式,像当年裘德考的做法一样,打入内部去了解情况。实在没想到,你也混进去了。早知道这样三爷直接请你就得了。”

我还是有点搞不清楚,花了好半天理解潘子的话:“等等等等,什么,我三叔?你是说这些我三叔都计划好了?那……你们?”

“我们早在格尔木准备好了,在敦煌我们的人准备了近半个月了。你们的队伍刚出发,我们就跟在后面出发了,当时这小哥留下信息,告诉我们你在队伍里,三爷还吓了一跳。小三爷你也真是的,三爷不是让你别再蹚这浑水了吗?你怎么还来?”

我用力吸了一口气,突然感觉到很无力,我靠,心说这次我真的就没想到,那……那个黑眼镜一路过来这么照顾我……看来还是我三叔的面子……

潘子继续道:“你在里面,三爷不得不顾虑你的安全,所以让黑瞎子提点着你点。这次排场很大,裘德考还是棋差一着,以为这一次把三爷摆脱了,没想到咱们早就计划好了。”

“那我三叔呢?”我看着四周,没看到三叔的影子。

“三爷在我们后面,差了点路,这一次我们来了不少人,人多不好跟踪。我和王胖子打先锋,在前面开路,一直跟着你们,然后沿途留下记号给三爷,就是没想到,你们到了这里就出事了。”

这时候我的思维才清晰起来,一下就想起来,那天晚上和闷油瓶长谈的时候,他就说自己是站在我这边的,让我不用担心,原来是这么个意思。原来这是三叔的计划。

这,我实在是没有想到这一层,看来老狐狸真的是老狐狸,和三叔斗,我还真的不够格。

“也算你们命大,我们一直跟这你们,要不然你们现在已经晒干了。”边上的胖子道,“就你这体质还想干这一行,我看你回去真的就该好好倒腾你的小生意。”

我问潘子:“他怎么也来了?”

潘子就说长沙的伙计、好手都跑到别人加去,现在三爷重新带了批新人,经验都不够,所以请了他来撑场面,也是老价格。

胖子道:“怎么?你还看不上我了?告诉你,你可是老子背回来的。”

我忙摆手,心忽然就安了下来,三叔的人到底像是家人,是我自己人,我不用凡事都戒备了。而且和这些人合作惯了,知道他们的本事,最开心的是闷油瓶真的是站在我们这边的,那就万事大吉了。

刚才是胖子在给我喂水,我逐渐恢复了力气,就自己喝了几口,他们不让我多喝,说是要缓慢的补充水分。

我看着阿宁没有反应,不知道什么情况,就问潘子她有没有事。

潘子道:“你放心吧,你的相好体质比你好,已经醒过一回了,现在吃了点东西又睡了。这里不是沙漠,你们只是脱水昏迷了过去,不是晒伤,补充点盐水,多睡睡就好了。”

潘子调侃我,大概是看到我和阿宁都衣衫不整,我也没有力气去反驳他,也就不去理会。此时身体虽然有点虚弱,但是人的精神已经相当好,我爬起来吃了点东西,问这是什么洞,当时他们是怎么找到我们的。

潘子告诉我,这里还是在魔鬼城,是在一个岩山的洞里,这洞是胖子发现的。当时出了事之后,扎西他们逃到了外面车子抛锚的地方,等我们等了很久都没出来,扎西就想到我和阿宁都不会看阿拉伯石堆,现在也不知道我们是遇难了,还是迷路了。

当即闷油瓶就用镜子给他们发了信号,他们赶了上,黑眼镜留下照顾剩下的人,闷油瓶就带着潘子进来找我们。

我问这么大的地方他们是怎么找到我们的,潘子说这地方有点邪门,这些石山的顺序好像是设计好的,他们也就是跟着感觉,其实走的路线完全和我们一样,最后看到了阿宁的标记,就一路找到我们倒在沙地上。

说起这个我就心有余悸,忙点头:“确实,这狗日的地方,好像怎么走都到不了头,却又不是走回头路,不知道是怎么回事。”一下我心里又紧张起来,心说那现在我们还在魔鬼城里,不还是走不出去?

“我们可没你们这么蠢,我们是一路留着记号的,你就放心吧。”潘子道。

胖子也道:“老子搭的记号,全是这么大的石头,离一公里都看得见,而且这走不出去的原因老子也看出来了。”

“哦。”我松了口气,问道,“那是为什么?”

潘子就说,一开始我们也不知道,还是胖子厉害,确实是他看出来的,我实话告诉你,我们现在待的已经不是原来的那个魔鬼城了,这里离原来的魔鬼城起码有一百五十公里。这是一片巨大的雅丹地貌群,由十几个小型的魔鬼城构成,中间是戈壁,而所有的魔鬼城都有岩山群相连,首尾相接,形成了一条巨大的魔鬼城链环。你们就是顺着这链子走,那就是三千六百平方公里,你们走得出去吗?

我摇头:“不可能啊,哪有这么巧?我随便找个方向一直走,就一点都没有偏移?”

胖子就道:“说你笨你还不承认,你顺着哪个方向走,是别人设计好的。那是因为这魔鬼城里有很多的石头,这些石头的摆放非常的讲究,经常是绕过一座岩山,一边的石头多,一边的石头少,但是因为石头杂乱无章,你在瞬间意识判断不出哪边好走哪边难走,感觉差不多,但是潜意识里,你却能分辨出石头少的方向,而条件反射的选择那个方向,结果你在这魔鬼城就一直在走别人给你设计好的方向。而且,几乎每一个路口都是这样的情况,就算有一个路口判断错了,你接下来还是有无数个机会被纠正。这种招数在古代很普通,有一个非常朴素的劳动人民取的名字,就叫做奇门遁甲。”说着,就看向闷油瓶,“小哥,我说得没错吧?”

闷油瓶抬头看了看我们,没理他,看着火,好像有心事。

我失笑,说:“你啥时候懂奇门遁甲了?”

胖子道你不知道的事情多了,看那表情还挺得以。我心说估计这家伙又是现学现卖,收回话题,就问胖子道:“那你是说,这魔鬼城里,有人用这些碎石头,摆了一个障眼法?”

胖子点头:“就是这么回事,不过不算高深的阵法,遇上个缺心眼的,或者观察能力特别仔细的,肯定能发现。看这些石头在这里也有年头了,估计这里当时是战场,西王母应该是个术数高手,这些石头是用来防御的。”

说到这里,一边睡袋里就传来了阿宁的声音,她轻声说道:“你说得不对,西王母根本就是奇门遁甲的创造者,当年黄帝得到的天授神书,就是西王母给他的,论起奇门遁甲,她是祖宗。”

原来这女人没睡,我们都给吓了一跳,我随即想起九天玄女的传说,心下骇然。确实是这样,当年的传说和一些历史记载,都说当时黄帝统一中原是得到了西王母国的鼎力相助。

再一想那古船,心说当年这里肯定是浅湖,这些岩山露出在水面上,水下的岩石会搁浅船只,那么在水里船夫更加的会选择暗礁少的地方行进,更加容易迷路,这可能也是西王母国这么多年下来,未给人发现的原因。

想到这里,我忽然眼皮一跳,对潘子道:“你是说这里的魔鬼城是一个环?”

潘子点头,我问道:“你是怎么知道的?”

潘子就摇头:“这是我们的向导说的。怎么了?”

我兴奋的在沙地里画了一圈:“你不知道,我看过文锦的笔记,她说西王母是在无形的城墙的保护下,这城墙别人看不到,但是碰到了,必然就会回头。在这里,几千年前,应该都是水,也就是说,这里有一条水带,类似护城河一样,围成了一圈。如果我们假设这条保护带就是别人无法进入,掩护了西王母古城这么多年的‘无形的城墙’,那么,西王母国应该就在这个圈之内,也就是在这个魔鬼城圈的中间。”

我说完后,所有人都无动于衷的看着我,好像在看一个傻瓜。

我被看得莫名其妙,摊手道:“我说得不对?”

潘子拍了拍我的肩膀道:“小三爷,你说的,不用看文锦的笔记我们都猜到了,只是,情况如果真的是这么简单,那么西王母古城早就被发现了。这里是柴达木盆地,不是塔克拉玛干,这里虽然人迹罕至,但是经过了无数的地质考察,所以,如果鬼城就在这里的话,情况也一定十分的特殊,很可能就整个儿被埋在戈壁下面了,或者处在一种别人很难发现的境地里。你认为到那里就能看到,很傻很天真。”

我一想也是,就问他们,那他们的打算是什么?

潘子说他们本来是打算跟着阿宁的队伍,到达塔木陀再说,现在到了这里就出了这么大的意外。不过,按照定主卓玛的说法,在这个魔鬼城西边,跟着古河床再走两天就是她当年和文锦的队伍分别的那个岩山口,接下去的路,定主卓玛也不认识了。他准备在我们恢复后,就到那里去休整,等三叔的队伍。之后,就打算顺着河道往下游走,因为古城肯定是在河道附近,当年的队伍肯定也是这么走的,我们也可以这么碰碰运气。

我对潘子道:“可是古河道到了这一段已经基本上和戈壁混在一起了,根本看不清楚。”

“那个不用担心。”胖子道,说着指了指一片漆黑的外面。

我们在洞的底部,不知道他是什么意思,就走了出去,一到洞口,忽然一股冰凉潮湿的气息扑面而来,接着我就听到了一种非常熟悉的声音。

外面一片漆黑,也看不清楚到底是什么状况,但是这情形我十分的熟悉,然而一刹那我却有点不敢相信我想到的。

等我走出洞口,脸上瞬间被水珠打到,我才反应过来,心说,我操,外面竟然在下雨。

可是这怎么可能?这里是戈壁滩啊,这里一年有可能只下一场雨,而且绝对不是这个季节。

我走回,就说:“这到底是怎么回事,为什么会下雨?”

潘子道:“小三爷,你得谢谢这场雨,要不然你等不到我们过来,就成咸鱼了。我们找到你们的时候,这雨已经开始下了,现在外面全是水,走也走不出去,不染我们背你就出去了,在这里待着也不舒服。这雨下了之后,老河道肯定会满水,往下游走,就算河道我们看不见,但是水能知道,所以你放心吧。”

这个时候我想到了定主卓玛和我说的:时间快到了,错过了就只能再等五年,心说难道是指这场雨?

越想越不靠谱,不过看闷油瓶没有说话,应该是没有什么问题,我安心了不少。

之后,我就去休息,这一次睡得不好,第二天醒过来的时候,我又一次以为自己是在做梦。

在这个山洞里,我们休整了两天时间,我和阿宁的身体都痊愈了。阿宁和胖子熟悉,到底是潘子他们救了她,她也没有说什么,不过对我一下子变得很冷淡,可能是认为我也是三叔安排进来的,骗了她。

我也不在乎,心说差点就挂了,还会计较这事情。第三天我们就出发了,顺着记号,我们蹚着到脚踝的水,冒雨走了两天,先走回到了外面,和黑眼镜回合。外面的人已经绝望了,看到我们平安出来,都不敢相信自己的眼睛。

在外面潘子又休整了一天,他建议我留在这里,等三叔到来,然后再决定要不要进去。

然而这时候我却感觉没什么脸见三叔,而且定主卓玛给我的口信,让我已经下定决心,要找到文锦,算起来我们已经没有多少时间了,就执意要和潘子一起打先锋。

一边的阿宁也安排了自己的队伍,大部分人都想要回去,高加索人的状况非常不妙,队医说等三叔的队伍到了,借了车必须马上回去。阿宁安排了一下,就告诉潘子,她也要参加我们,怎么说她的队伍也是打了先头。

阿宁的加入没有问题,潘子也拗不过我,况且这段路我们有车,也不是什么危险的路段,就答应了。我和胖子、潘子、闷油瓶、阿宁正好一辆车,黑眼镜在这里等三叔。

之后的两天时间,我们顺着水位逐渐见涨的河床,在戈壁中越走越深。因为雨水的冲刷,河床中出现了很多的支流,我们一条一条去找,然而,怎么看,我们都没有看到那座岩山。我的望远镜都看裂了,最后开始怀疑,是不是那座山已经变成沧海桑田了。

雨在出发前就停了,我们最后在河床边上休息,车的轮胎磨损得非常厉害,后来一个还破了,我们只能开一段就下来打气,然后继续开,苦不堪言。

胖子就说:“会不会那老太婆是胡扯的,根本就没有那座山?或者那根本不是山?也许是土丘,这十几年给风吹没了?”

我感觉不是,定主卓玛那样子怎么看怎么不像是骗子,或许这古河道的走向已经改变了。

“那怎么办?再走下去,汽油都没了,我们要走回去可够戗。”

“这喝水能会聚的地方就是整块平原最低的地方,那里应该有个湖,我们要不先找到那个湖?然后从湖开始寻找河道的痕迹,这样至少能缩小范围。”

想想也只能这样了,我们继续赶路,开上一个斜坡的时候,忽然,潘子大骂了一声,一脚刹车。

我们全部都撞到了前面的坐垫上,胖子大骂,还没骂完,几个人一下都愣住了。

这斜坡的另一边,竟然是一块断崖,我们的车头已经冲了出去,两只轮胎已经腾空了。

我们心惊胆战的下了车,走到悬崖边上,发现面前出现了一个巨大的盆地,烟雾缭绕,一片凹陷在戈壁中的巨大绿洲!

\chapter{向绿洲进发(上)}

眼前的情形之壮观,言语根本无法表达,我们都看得呆了,虽然文锦的笔记中提过这么一个绿洲,但是,我的印象里应该不是这个样子。

盆地非常大,而且看上去很工整,胖子说起来,竟然好像一个陨石坑。从悬崖上往下看去,只看到下面烟雾缓绕,几乎全是密集的树冠,看不到具体的情况。

这应该就是塔木陀了,没想到,我们竟然是以这样的方式发现它的,好像有点太简单了。

潘子把车倒了回来,我们就一边用望远镜看盆地,一边琢磨这是怎么回事。

潘子道:“看来定主卓玛和文锦他们分开时候的岩山,确实己经消失了,这里是盐盖地区,可能那是一座岩山,几十年,几次雨就剩个土包了,不过,顺着河水的方向,还是能够找到这里。”

这些文锦的笔记上没写,我也不可能知道,不过如今这么就发现了这绿洲,我们也有点不知所措,我就问潘子,有什么打算。

潘子就道肯定要先下去看看,他听我说了笔记和定主卓玛的口信,知道文锦肯定就在下面,说现在不能等三爷会合,要直接先进去看看情况,文锦就是师母,要是因为等三爷,把师母漏过去,他这伙计也不用再当了,时间已经不多了。

我心说你真是个二十四孝的手下,不过我也是这么想的,时间已经不多了,算起来,十天几乎就在眼前,问了几个人,都没有意见,他们就让我看看,这盆地应该怎么进去。

文锦的笔记上有详细的路线描述,他们当年是通过一条峡谷进入盆地的。不过这里的地貌己经完全变了,通过她的路线描述看来是找不到那条峡谷,我们只能开车绕着盆地寻找,几经曲折,终于发现了一条宽大的峡谷。

潘子绕了一个大圈子,在盆地大概四公里的地方,找到了峡谷的路口,最开始的一段可以开车,我们一路进去,一直到乱石挡住去路为止。

然后几个人下车,背起装备就步行前进。一直走到看到树木,才停下来休息。我拿起文锦的笔记,仔细看里面的记载。

看了笔记之后,我不由有点心虚,从文锦的笔记中记载的事情推断,这条峡谷十分的危险。峡谷再往前去,因为海拔降低,热带植被丛生,瘴气弥漫,我们的防毒面具有可能应付不了这么潮湿的环境,而且这里是通往西王母宫的唯一入口,一路过来遇到的事情,让我感觉到西王母宫诡异非常,料想这路也不会这么好走。

不过相比之下,我最担心的还是过了峡谷后的事情。峡谷的尽头就是绿洲的核心地带,这里是河流会聚的地方,坑谷下茂密的树冠之下全是潮湿的沼泽,这里的奇特地貌几乎形成了一个戈壁中的热带雨林。虽然我们知道西王母的古王城就在沼泽之内的某处,但是在里面搜索几乎就是玩命。

我们在峡谷的树荫下详细的看了文锦在笔记中描绘出的大概行进路线,因为没有进入沼泽实地,很多的地方都看得一头雾水,而且文锦在很多地方都画着问号,我们不知道这些问号代表着什么,这让我们非常的为难,最后只能决定走一步是一步。

之后我们各自做准备工作,搜索的时候,知道前路漫漫,我们必须控制着自己物资消耗,如今要进入到西王母的后院了,自然也就顾不了这么多,照明弹、冷烟火、火柴、药物,所有能带的东西我们都装了进来。

潘子在越南打过仗,现在成了我们的顾问,他说从在悬崖上看下面的情况,这里的情况应该和越南的热带雨林差不多。这种湿润地带的沼泽最危险,上头是原始雨林的阔叶冠,几乎覆盖了整个谷底,这么茂密的植被,下面肯定透不过阳光,树冠下面一片漆黑,瘴气弥漫,是蚊子蚂蟥毒虫的天下。尽管这里的气温超过三十度,我们也必须穿长袖长裤,不然没一个小时你身上绝对一块好肉都没有。

阿宁说我有驱蚊水,行不行?

潘子说你驱走了蚊子,但是那东西会引来其他东西,在雨林里不要用太浓烈的气味。否则就算你当时没碰到野兽,它们也会一路尾随过来,咱们这一次只有我带了枪,就算碰上野猪也够戗。

他最后说,一旦进入了沼泽之后,不到万不得已不要去蹚水,或者去碰那些污泥。他有一个战友,在打伏击的时候脚陷在沼泽里面,才一分钟不到,拨出来的时候,整个腿上全是洞,给蛀空了,也不知道是给什么咬的。在现在这样的环境下,如果出现这种事情就等于送命,也许还不如送命。

我从潘子的眼神中感觉到他不是在危言耸听,心里也多了几分异样,于是将裤管扎得更紧了点。

花了两个小时,我们把所有的东西都整理打包完毕,在潘子的吆喝下我们就出发了。闷油瓶打头,潘子殿后,砍着树枝阔叶,就往峡谷的深处走去。我们前脚刚动,天又阴了下来,似乎是要下雨。我在心里感慨,大自然的奥妙真是无法穷尽,在干旱的柴达木戈壁的深处,竟然有这么一块潮湿多雨的绿洲,真是天公造物,不拘一格。

这条峡谷不像是在魔鬼城看到的那种雅丹峡谷,不是由风力雕琢而成的,好像是由地质运动产生的裂谷,谷底不平坦,怪石嶙峋、层层叠叠,岩壁仿佛被利刀雕琢而成。不过,要让我说,我却同意胖子的说法,这里的地形实在是像一个陨石坑,裂谷好比是陨石坠落的时候砸裂的地壳裂缝,产生的时候可能比现在深得多,逐渐风化,给填平了。这样的峡谷在这个坑谷的四周应该不是唯一的。

峡谷很宽,进入密林之后,四周变得非常的闷热,我们的身上一下就汗透了。石头和树上到处都有青苔,无法立足,我们的脚下已经到处是潮湿的烂泥和盘根错节的树根,在怪物触须一样的树根网里行走,一脚一个陷坑,头顶上的树冠也密集得看不到阳光了。一下子我就产生了非常严重的错觉,我现在真的是在青藏高原上而不是在亚马逊的原始丛林里吗?

本来以为这种情况只有在峡谷的尽头才会碰到,没想到在峡谷中已经如此了,那坑谷里的情况估计更加的糟糕。

胖子走得气喘连连,看着前面的情形,就说不知道这绿洲里面有没有什么动物,他娘的打几只来吃吃,也算是种福利,要不然这路走得就冤枉了。

潘子说这片封闭环境中的雨林说小不小,说大也不大,恐怕不会有大型的野兽,最多的恐怕还是虫子和长虫。在很多这样的沼泽中,蛇是最常见的。

胖子说蛇也不错,在广东还吃过烤蝎子,反正只要是新鲜的东西,老子都不在话下。

我想起文锦在笔记中写的:“泥沼多蛇,遇人不惧。”想必潘子说得不错,不知道这些蛇的大小,在很多好莱坞的电影里,有些蟒蛇可以长到老树这么粗,压路机都压不死,不过这里应该没有这样的条件。

而且这里的生态环境十分特殊,是一个封闭的陆上孤岛,我想除了飞鸟和人类,其他东西根本不可能进入到这里来,这里的生物是在这个绿洲形成时开始就在这里繁衍的。当时柴达木还是一片富庶的河流密集之地,物种丰富,也许我们能够在这片绿洲中发现很多己经灭绝的动、植物,这有可能比西王母宫里的东西更加的有价值。

转念一想,又心说不要了,在山海经的西王母传说中,西壬母宫是被一群人面的青鸟守护着,这肯定是一种我们所不了解的巨大猛禽,保不准就是在长白山攻击我们的那种怪鸟,这种东西还是灭绝了好。

由于树木太过密集,而我们又是在峡谷中,没有迂回的条件,我们只能一边砍掉老藤阔叶一边前进。这很消耗体力,胖子和闷油瓶轮流开道也没有多少起色。好在峡谷边上的磷响山崖夹着一道蓝天,好比一道天蓝的锦带,景色十分的绚丽,不时还有前天大雨形成的瀑布倾泻下来,我们一路过去,也并不无聊。

走了不久,我们就发现前面的峭壁上,出现了很多的石窟,密密麻麻,足有百来个,上面覆满了青苔,不知道里面雕着什么东西。

我们一下子紧张起来,看景色的心情也没了。一路过来没有看到任何关于西王母国的遗迹,一直有一种不真实的感觉,现在突然看到了,我们真的开始靠近这个神秘古国的核心地带了。这想来是件兴奋的事情,但是实际看到,又觉得有点恐怖。

收拾起嬉闹的心情,我们上去查看。这些石窟有大有小,大的能并排开进去两辆解放卡车,小的只有半人多高,和敦煌的有很大的不同,石窟都很浅,在外面就能看到里面的雕像,只是被厚厚的青苔整个盖住了。

我爬上去拿出匕首,开始刮其中一座上的覆盖物,在青苔中,逐渐露出了一座怪异的石雕。

\chapter{向绿洲进发(下)}

青苔中,是一座石刻的人面鸟身的神像,和我们在古沉船里发现的陶罐上的雕刻风格一样,是真正的西王母国的雕刻。经过千年的腐蚀,石雕表面布满了石斛,显得模糊不清。

我把上面的石斛也全部去掉之后,雕刻的整体浮现了出来。那是一尊立像,是在山崖上直接凿出来的,鸟的头部是一张似人非人的女性怪脸,长着两对眼睛,面无表情,冷酷异常。两足下雕琢着五个骷髅头,鸟立于其中两个的天灵盖上,似乎这些骷髅都是它吃剩的骨骸。

胖子在下面看着,就惊呼了一声:“天哪,小吴,这他娘的不就是……”

我跳下来看到石雕的整体之后,也倒吸了一口冷气。

原来这崖壁石窟里的人面鸟身的石像,竟然和在长白山地下裂隙中看到的怪鸟几乎一模一样。

雕刻的形态极其生动,看山石的表面,修凿之时应该还涂有颜料,如果不是青苔覆盖,在这阴暗的丛林里看到,准会以为那种怪鸟从长白山飞到这里来了。

众人都露出了惊异的神色,连闷油瓶都显的很意外。这里所有的人都到过长白山,看到这些石雕,难免回想起当时可怕的情形。

我和胖子又忙动手,将其它几座石窟的雕刻也一一刮开,发现里面都是一样的人面鸟的石雕,有大有小,形态各异。

阿宁吸了口气道:“看来我们之前推断得没错,长白山中的人面猛禽便是西王母的图腾——三青鸟的原形。西王母手上可能掌握着一些我们所不了解的古老技术,可以驯养这种诡异的猛禽。长白山中的地下陵墓应该和西王母国的消失和遗民的神秘东迁有关系。那些怪鸟可能原本是栖息在这片绿洲之中,后来给那些分裂出来的遗民带到东方,充当了陵墓的守护者。”

我道:“不错,我一直感觉,这里的地形和长白山地下皇陵的地形是如此的相似,都是在一个巨大的陨石坑状盆地里,看来那里可能是西王母宫的一个翻版,咱们在长白山里的经历只能算是一个演习,这里是人家真正的老窝。”

胖子听了就擦了擦汗道:“他奶奶的,照你们这么说,这是那些鬼鸟的老家?那咱们这么进去不是送死吗?”

这还真不好说,我回头苦笑,阿宁道:“那到不至于,事隔了这么多年了,这里的气候剧烈的变化,大片的草原浓缩成了这一片绿洲,食物太少,这种鸟在这里可能已经绝迹了,在长白山看到的那些可能是硕果仅存的一些。不过,不管怎么样,西王母国以青鸟为守护神,这里有这样的图腾,说明我们已经进入到西王母宫的界内,这种石窟图腾刻在这里,既是对外来人的一种标示,也是一种警告,这后面我们得加倍小心。”

我们都点了点头,胖子道:“妈的,承你贵言,这些鬼鸟真的灭绝了才好,要不然连累到了我,摸金校尉就要灭绝了。”

胖子的担忧也是我们的担忧,我们相顾一下,都没有话说,神情都很复杂。

又耽搁了片刻,阿宁给这些石像拍了照片,四处看了一圈,除了石头再无发现。闷油瓶就让我们出发。

我们最后看了一眼那些石窟,抖擞了精神,离开了这块崖壁,向峡谷的深处继续走去。大概是因为那些石窟雕像的影响,那一刻,我就感觉到一种不安开始笼罩进丛林里,我们似乎正在走进一个无人理解的诡异世界之中。

\chapter{第二场雨(上)}

离开石壁上的石窟之后,我们各自调整心情,继续往峡谷的深处前进。

因为石窟中石雕的影响,我们走得非常小心,注意着丛林中的每一个动静,生怕会遇到西王母千年之前设下的埋伏。

然而随着我们的深入,却并没有什么诡异的事情发生,一路无事,甚至连西王母国的其它遗迹都没有看到。只有雨林越来越密集,盘根纠错,铺天盖地,仿佛我们是在远离西王母的王宫,而不是在靠近。走到后来,眼睛就花了,只感觉到处是绿色的绞结的腾蔓,好像穿行在一碗发着绿霉的龙须面里。

我这才领悟到“丛林”是什么概念,我在山东和秦岭穿过的树林和这里比起来简直就是在旅游,在那边走上一公里,在这里一百米都可能前进不到,简直是步履维艰。看着潘子满头是汗的坚毅脸庞,也不知道他们当年打仗是怎么挺过来的。

就这样一直闷头往前,一直都到林子黑下来,两边的峡谷变成了剪影画,我们也并没有前进多少距离。

队伍中也没有了人说话,只剩下喘息的声音和拍打蚊子的声音。

胖子走的蒙了,犯了臆症,就在前面哼山歌给自己提神,唱花儿为什么这样红。

“花儿为什么这样红?为什么这样红?哎红得好像,红得好像燃烧的火。”他是开路手,在队伍的最前面,他唱歌也同时能给我们提神,这事情你无法指望闷油瓶来做。

不过胖子唱歌实在是难听,加上也不是正经的唱,听起来像是在招魂一样。

潘子后来听不下去了,就骂道他娘的这里这么热,你就不能唱点凉快点儿的?

胖子说你懂什么,这是冰山上的来客的歌曲,我唱起来,就想起长白山的冰川,多少能凉快点儿。

潘子说那你唱白毛女不行吗?多直接的,还省的联想。

胖子说我操你还点歌了,你还真以为我是电台,想听什么唱什么,老子唱给你听是给面子。少他娘的这么多意见。

正骂着,天上就打起了雷,云层里电光闪动,风也吹了起来,空里里出现了雨星子。

我们都安静下来,抬头看天,透过树冠,乌云亮了起来,似乎有闪电在云里攒动,云都压到了峡谷的顶上。阿宁叹了口气,说:“行夜路偏又遇风雨,看来西王母并不欢迎我们,咱们今天晚上有的罪受了。”

胖子道:“下吧下吧,最好它下雨,下了雨凉快,这么闷着,你胖爷我裤裆里的蛋都要孵出小鸡来了。”

我们听了就忍不住笑了出来,潘子骂道:“那你把你的小鸡看好了,别等一下给雷劈了。”

话音未落,雨就真下来了。起初是几滴雨弹打在了我们脸上,还没等我们反应过来,磅礴大雨就来了,一下子好像整个森林都安静了下来,万木无声,接着“轰”一声,整个峡谷瞬间轰鸣了起来,雨水像鞭子一样从树冠的缝隙里抽了进来,几乎没把我们砸趴下。

我们没想到雨会这么猛,一下子猝不及防,全部都抱头鼠窜。幸好我们是在密林的底部,四周有很多的大树,树冠密集,有一棵树上有一块由藤蔓纠结起来的遮盖,在阿宁的大叫下,我们爬了上去躲雨。

所有人挤在一起,都好像从汤里捞出来一样。胖子说我操这他娘的哪里是下雨,这干脆就是龙王爷在我们头顶上滋尿。

此时一道闪电亮起,照亮了整个峡谷。借着闪电往前看去,一边的崖壁上雨水已经汇聚成大量的瀑布倾泻下来,黑夜中雨林翻滚,两边是冲下的巨大水幕,好比摩西分开大海的情形,壮观异常。

\chapter{第二场雨(下)}

而峡谷之下,冲下的雨水形成的无数条小溪开始汇集,很快,它们就会聚成河流,向下游的沼泽涌去。

看到这幅景象,我忽然就意识到了这片绿洲形成的原因:这里是柴达木盆地的最低点,所有的地下水和雨水,都会会聚到这里来。可以说这里是整个柴达木地下水系的中心,柴达木干涸的河床也许并不真正的断流了,而是转入地下流到了这里。所以无论这几千来年来气候如何变化,盆地的周边如何由森林变成沙漠,这里仍旧保持着五千年前树木繁茂的样子。

“藏风聚水而不动”,所谓风水宝地,不就是经千年而不变的地方吗?这西王母宫所在的地方,果然应该是昆仑山系龙脉之祖的宝眼所在。这样的奇景,也只有在这种地方才能出现啊。

正在感慨,胖子却不安份了起来,大屁股挤来挤去。这树上的空间本来就不大,他一动所有人都不自在,潘子就骂道:“你小子他娘的干什么,皮痒还是怎么的?”

胖子皱着眉头,说:“不知道怎么回事,老子屁股突然痒的要命。”说完又挪了挪屁股,在树上蹭了起来。

我心说就他事情最多,刚想说他几句,突然自己的屁股和背也痒了起来,一下子奇痒难忍,好像有什么东西在爬一样。我忙弓起了腿想用手去抓,一抓之下就感觉不对,一下跳了起来:“虫子!”

所有人全站了起来,我挠着屁股往我们靠的树干上看,一看之下脸都绿了。只见满树干都是密密麻麻的花虫子,大概都只有半个小拇指指甲盖大,好像都是从树杆的缝隙里爬出来的,我们的腿上和屁股也全都是了,拍都拍不掉。

“我靠!”胖子大骂了一声,几个人都跺起脚来。但是跺脚并没有什么作用,这些虫子根本不怕人,似乎当我们是树木,毫不犹豫的朝我们身上爬,幸亏我们的裤管是紧的,它们爬不进来。但是我和胖子的屁股已经遭殃了,我们只好跑到雨里,让雨水冲自己的臀部。冰凉的雨水渗入到裤子里,我才感觉到奇痒消退了点儿,只是痒完了之后,屁股上原来痒的地方又疼了起来,我心里大骂,心说该不是有毒吧。这时候其他人也都逃了出来,一下子雨水朝我们身上猛冲,我们也说不了话。

我们爬上另外一条枝丫,朝树的上面爬去,那里还有一块雨水稍微少一点的几条枝丫密集的死角,但是并不够我们五个人全部进去,最后阿宁和我被他们推了进去,其它人用防水布遮着头,算是勉强不用给雨水冲头。

潘子道:“妈的,刚才他妈的是什么虫子?”

阿宁甩掉头发上的水,又拍了拍暗淡下去的矿灯,总算把它打亮了,然后她照了照自己的裤腿,把粘在她腿上的死掉的虫子用小拇指的指甲挑到矿灯的前面。

那是一只好像蜘蛛一样的小虫子,又有点像没有尾巴的小蝎子,阿宁的手在抖,所以我也看不清楚,我屁股又疼了起来,就又问了一声:这是什么?有没有毒。却看到阿宁的眉头皱了起来。我心里咯噔一声,还没来得及说糟糕,阿宁就顺手拔出了边上潘子腰里的刀,对我道:“转过去,快把裤子脱了!”

\chapter{青苔下的秘密(上)}

阿宁说着就要来拽我的皮带,我一下急了,也不知道她想干什么,忙捂住裤子,缩了一下:“你想干什么?”

阿宁道:“那些虫是一种草蜱子,给它们咬了很麻烦。你和胖子给咬了,如果不想以后趴着睡的话就赶紧把裤子脱了,等一下它爬到你的裤裆里你这辈子就完了!”

我一听,还真觉得敏感部位有点搔痒,但是怎么样也不能让阿宁给我处理啊,还是死死抓着裤子,对阿宁道:“那你把刀给我,我自己去处理!”

“你自己怎么看自己的屁股?”阿宁道。

我心说就算这样也不能给你看啊,这时候边上的胖子一边挠屁股一边就说话了:“别吵了,”说着从阿宁手里拿过刀,对我道:“这婆娘说的没错,草蜱是很麻烦,咱们两个到那边去,互相处理一下。”

“你会不会处理?”阿宁问。

“不就是把刀烧烫了去烫嘛,老子少说也插过队,放过牛羊,这点还不知道。你们也自己检查一下,你细皮嫩肉的,最招这种虫子了。”

说着指了指另一边的树枝后面让我走过去,那里雨也不大,但是树枝似乎不太牢固,但此时也管不了这么多了。

爬到那里,往后看看阿宁他们似乎看不到了,胖子的脸就变形了,抖起来一下就脱了自己的裤子,对我道:“快快快,老子要给咬残了!”

我把矿灯往树枝上一架,一看就傻了眼,我操,只见他满大腿满屁股都是豌豆大的血包子,有的都大的像蚕豆一样,再仔细一看,就看到那些血包子全是刚才那些小虫吸饱了血的肚子,都涨得透明了。

“你怎么搞的!”我突然想吐,捂住自己的嘴巴:“这也太夸张了,这么会爬进去这么多?”

“这裤子太小了,老子过魔鬼城搬石头的时候档崩裂了!”他抖了抖他的裤子:“裂了条大缝,他娘的当时我还说裂着凉快,一直没处理,进林子的时候就给忘记了,真是作孽——你快点!这虫子能一直吸血两三天,能吸到自己体积的六七倍,三十只就能把一只兔子的血吸光,老子已经贫血了,可经不起这折腾。”

我拿起刀,只觉得胃里翻腾,也不知道怎么割,比画了半天就想用手去摘,那胖子忙缩起屁股躲开道:“千万别拽,它是咬在肉里,脑袋钻进皮里去吸的,你一拽头就断在里面,和雪毛子一样,得照我刚才说的,用火烧匕首去烫!”

我点了点头,一下竟然连自己的搔痒都忘记了,发着抖拿出打火机,将匕首的尖头烧红了,然后把一只一只吸的犹如气球一样的虫子烫了下来,那虫子爬烫,一靠近就马上把头拔了出来,我一下就倒下来,用刀柄拍死,一拍就是一大包血。每烫一只,胖子就疼的要命,到了后来,我看他的腿都软了,我的手也软了。

足足搞了半个小时,雨都小了下去,我才把胖子的大腿和屁股上弄干净了,潘子检查完自己之后也想过来帮忙,但是他一过来树枝就开始颤动,所以只好作罢,他让我们弄完后一定要消毒,不然很容易得冷热病。

搞完之后,给胖子涂上消毒的水,我又勉为其难的脱掉裤子让胖子处理。说实话在那种场合蹲马步给人观察屁股实在是难堪的事情,但是没有办法。不过我被咬的情况还好,十几分钟就处理好了,最后检查了确实一只都没漏下,才算松了口气。

穿上裤子,我们爬回到众人那里,两人尴尬的笑笑,潘子就问我们怎么样,我点头说还好,总算没给咬漏了。又问他们有没有被咬。

潘子和阿宁只有手臂上被咬了几口,闷油瓶则一点事情也没有。“草蜱的嗅觉很敏感,能闻出你们的血型,看来你们两个比较可口。”阿宁解释道。

我想起刚才的事情,比较尴尬,就转移话题问她道:“这里怎么会有这么多的蜱子。这种东西不是潜伏在草里的吗?怎么在聚集在这棵树上,难道它们也吸树汁?”

吸血的东西一般都在草里,因为动物经过的几率大,在树上的几乎没有。

阿宁摇头,表示也不理解:“不过,这里有这种虫子,我们以后一定要小心,这些虫子是最讨厌的吸血昆虫,其他的比如蚊子,水蛭这些东西很少会杀掉宿主,唯独这种虫子,能把宿主的血吸干。我上次在非洲做一个项目,就看到一头长颈鹿死在这种东西手里,尸体上挂满了血瘤子,恐怖异常。我们一靠近所有的草蜱子都朝我们涌过来,黑压压一片,像地上的影子在动一样,吓的当时的向导用车上的灭火器阻挡,然后开车狂逃而去。”

我想起胖子的屁股,再想想阿宁说的场面,不由不寒而栗起来。

正说着,我忽然发现少了一个人,一辨认,闷油瓶不见了。

问他去了哪里?阿宁用下巴指了指下面,我就看到闷油瓶不知道什么时候爬到了我们下边刚才避雨的植物遮盖那里,打着矿灯,不知道在看什么。

\chapter{青苔下的秘密(下)}

我看着就好奇,问阿宁道:“他下去干什么?”

“不知道。”阿宁表情的复杂的看着下面的矿灯光,“一声不吭就下去了,问他他也不理人,我是搞不懂你这个朋友。”

我叹了口气,自从魔鬼城里那次交谈之后,闷油瓶的话就更少了,甚至最近他的脸都凝固了起来,一点表情也没有出现过,也不知道这人的脑子里到底在想什么东西,也许他真的像定主卓玛说的:他自己的世界里,一直只有他一个人,所以他根本没有必要表露任何的东西。

看着那下面的灯光,应该是架在树枝上,给风吹的晃来晃去,我有点担心他会不会掉下去,随即又想到这小子是职业失踪人员,会不会趁这个机会,又自己一个人溜掉了?

阿宁他们没经验,这还真有点玄……我看着下面晃动的灯光,也看不清楚他到底是不是在那里。

想到这里,我就放心不下了,于是打开矿灯,对阿宁说我下去看看。接着顶着大雨,抱着树干小心翼翼的一段一段下来。

爬到下面矿灯的边上,我四处看了看,心里顿时一凉。

真的没人!

刚才我们躲雨的那块植物遮盖下,空空荡荡,哪里有闷油瓶的影子!

“狗日的!”我暗骂了一声:“难道真的跑了!”一下子气的不行。这人怎么这样,比起胖子做坏事还和你打个招呼,这人根本就当我们不存在,实在是太过分了。

怒火中烧,正想喊胖子他们下来商量对策,突然树枝整个一动,闷油瓶却从那植物遮盖上面的黑暗处探了出来,把我吓了一跳。我抬头一看,原来他是站在这片遮盖的顶上,不知道在看些什么。

虚惊一场,我不由长长的出了口气,他看到我也下来,略微楞了一下,就招手让我上去。

我爬了上去,看到由树枝、寄生藤蔓、蕨类植物互相纠结,长满了绿苔的植物覆盖物表面,已经给他用刀割了开来,青苔被刮开,里面大量的藤蔓给切断,露出了里面裹着的什么东西。雨水中可以看到大量细小的草蜱子在这些藤蔓里给水冲下去。

我不知道闷油瓶想在这堆东西里找什么,只闻到一股很难闻的味道,正想凑近看,闷油瓶又用力扯开一大片已经枯死的藤蔓,一瞬间,我只觉得眼睛一辣,从那个破口里涌出一大团虫子。

我吓的赶紧后退,差点从树上摔下去,幸亏下着大雨,这些草蜱子一下就给磅礴的雨水冲走了。我扶住一边的树枝,捂着鼻子再次凑过去,就看到了这团遮盖里面缠绕着的东西。

那是一团腐烂的皮毛裹住的动物残骸,皮已经烂成了黑色,不知道是什么动物。闷油瓶用匕首插入到毛皮上,搅了一下,发现残骸已经腐烂光了,皮里面就是骨头,那些藤蔓长入它的体内,纠结在它的骨头里,将残骸和树紧紧缠绕在了一起。上面又覆盖满了青苔,所以我们才当它是普通的树上缠绕的植物混生体,进到下面去遮雨。

“不知道是什么动物,很大,可能是给这些虫子吸血之后染病死的,临死之前趴在树上,结果把四周的虫子全引来了,活活给吸干了,之后虫子就歇伏在尸体上,等下一个牺牲品。”闷油瓶皱着眉头对我道。

我听着想起刚才我们在下面躲雨,就感觉到一股反胃,对闷油瓶:“这里的草蜱子这么厉害?这尸体都烂光了,它们还没死?”

闷油瓶摇了摇头,大概是表示不知道,又低头看了看那堆骨骸,不知道又想到了什么,突然拔出了他的黑金古刀,在自己的手掌上划了一道,用力一挤伤口,血从他的掌间流出,然后他握了一下我的袖子,将血沾了上去。

我愣了一下,还没意识到他是什么意思,他突然就猛地一俯身,奇长的手指伸出,将满是血的手伸进了藤蔓下的骸骨里。

顿时无数的草蜱子有如潮水一样从里面蜂拥而出,我吓得大叫起来,闪电一般,同时他的手就从骨骸里扯出了什么东西。

\chapter{蛇骨(上)}

如果他动手的时候稍微有一丝的迟疑,那么我也能做点心理准备,至少不会叫出来,但是这家伙做事情太凌厉了,如此恶心的骨骸,这么多的虫子,他也能面不改色的伸手下去,换了谁也措手不及。还好这家伙总算有良心,在我袖口上抹了血,不然这一次真给他害死了。

镇定了一下,发现转瞬之间,四周的虫子已经一只也看不到了,一边惊叹他的威力,一边又郁闷起来。

在秦岭和雪山上,长久以来我一直感觉自己的血也有了这种能力,不知道为什么在这里好像对这些虫子不管用,难道闷油瓶的血和我的血还有区别?我的血火候还不够?

闷油瓶把从骨骸中夹出来的东西放到了矿灯的灯光下,仔细的看起来。我凑过去,就发现那是一件青绿色的大概拳头大小的物件。闷油瓶把手伸到雨水大的地方,冲洗了一下,再拿回来,我就惊讶的发现,这东西我还见过,那竟然是一只扭曲了的老式铜手电。

稍微看了一下,我就知道这东西是八九十年代改革开放之后的东西了,铜的外壳都锈满了绿色,拧开后盖一看,里面的电池烂的让人好比一团发霉的八宝粥。

我心里疑惑到了极点,这种东西怎么会出现在这里——这具动物骨骸里?难道这是具人的骨骸?

正琢磨着,闷油瓶又把手伸进了骨骸里,这一次已经没有虫子爬出来了。他闭上眼睛在里面摸着,很快他就抓到了东西,而且似乎是什么大家伙,另一只手也用上力了,才把它挖了出来。

我一看喉咙里就紧了一下,那竟然是一段人的手骨,已经腐朽得满是孔洞,里面填满了黑色的不知道什么东西腐烂的污垢。

“这……”我一下子不知道该说什么。

“这是条大树蟒,吃了一个人。这手电是那个人身上的。”闷油瓶面无表情的说道,“而且,是个女人。”

我看到手骨上粘着一串似乎是装饰品的东西,知道闷油瓶说的没错,心里涌起一股异样的感觉。人一下就兴奋起来,想到了很多的事情。

这片绿洲的地形奇特,只有在大暴雨之后,地下暗河安卡拉扎浮出水面的时候,才能够被人发现。而柴达木盆地下雨是和摸奖差不多的事情,如果是有石油工人或者是探险队正巧在大雨的时候发现这里,然后闯进来给巨蟒吃掉,这种事情虽然有可能发生,但是机率不大。另一种可能性则让我感觉到毛骨悚然,这巨蟒里的尸体,会不会是当年文锦驼队里的一员。

毕竟,当年的文锦在最后关头放弃了进入西王母宫的机会自己回来了,然而进入西王母宫遗址的霍玲他们,最后如何,连她也不知道。

闷油瓶肯定也想到了这一点,看了看上面的阿宁他们,就对我道:“上去叫他们下来帮忙,把这条蛇骨挖出来,看看里面到底是谁?”

\chapter{蛇骨(中)}

我应了一声,就转身往上爬了几步,一边就朝上面大叫。这时候就看到胖子已经在往下爬了,听到我叫,加快了步伐,跳到我的身边,问我怎么了?

我说有大发现,又对着潘子和阿宁叫了两声,把他们两个也叫了下来。

几个人来到那团蛇骨的边上,我就把我们发现的事情和他们说了一遍,一下子众人也大奇。阿宁一下就紧张起来,马上走过去看,胖子则道:“难怪我觉得刚才有人在召唤我,原来我们还有革命前辈牺牲在这里,那可太巧了,赶快挖出来瞻仰瞻仰。”

此时的雨已经趋向平和,虽然不小,但是已经不是刚才时的那种霸道的水鞭子,我们身上其实本来就是全湿的,此时也没有什么顾忌了。倒是我,小心的把闷油瓶的血沾染的袖口保护起来,这下面的路,这东西可能会救我的命。

我们爬到那片巨大的植物身体的上面,刚才两个人的时候还可以,现在人多了,这东西就有点支撑不住,胖子和我就只好把另外一只脚踩到一边的树枝上,以防这东西塌掉。我们用匕首割掉里面的枯死的藤蔓,将裹在其中的蛇尸暴露出来。

如果是在晴天,可能挖起来更方便,但是现在是在大雨里,头一低雨水就顺着刘海往下滴,眼睛就不是很管用,我们不时的甩掉头发的水,才能看清下面的东西。

不过人多总是好的,特别是胖子,大刀阔斧,丝毫也不考虑一刀刀下去会不会砍伤他革命前辈的遗骨。

藤蔓很快被挖出一个更大的缺口,一截巨大的蛇骨暴露了出来,胖子骂了一声,我也有点惊讶,因为刚才说蛇的时候,我并没有意识到这蛇会这么大,看蛇骨的直径,这条蛇可能有一个人这么粗,这么大的蛇,吃一个人可能一分钟都不用。

扯动了一下,盘绕着的蛇骨中,我们就看到了扭曲的人的骸骨剩余部分,这条巨蟒死的时候应该是刚刚吞下这个人不久,否则骨头会给吐掉。骨骸的身上还有没有腐蚀完全的衣服,但是已经完全看不出当初是什么样子了。潘子学闷油瓶子一样俯身从里面也夹出了一样东西,那是皮带的扣,只有少许的锈斑,似乎是不锈钢的。

他拿了出来,用刀刮了刮,然后递给我,我们凑过去,我就看到上面刻了几个数字:“02200059”。

我吃了一惊,马上看向阿宁:“是你们公司的注册号,这是你们的人!”

\chapter{蛇骨(下)}

02200059(零贰贰零零零伍玖),这一串号码,按照阿宁的说法,是最后一份战国帛书上隐含的一组神秘的数字,汪藏海将其解出之后,百思不得其解,于是称其为天数,乃用作自己的密码。铁面生为何在要最候一份帛书中隐藏这一组奇怪的数字,背后又有什么样的奇遇?这件事情或许更有隐情,但是与现在我们经历的事情无关,这里也就不作表述。而阿宁的传教士老板裘德考对汪藏海十分的着迷,于是通过关系,将此数字用作了自己资源公司的标示码。阿宁队伍的装备,车上都有这组号码,这种公司的标示在国际探险活动中确定第一发现人非常重要,现在我的皮带上也有这一组号码。可以这么说,这皮带扣就是确定死亡者所属队伍的证据。

阿宁一开始不理解我说的是什么意识,接过来仔细看,一看之下,脸都白了,“这……”

“是你们公司的标示码没错吧?”我问道。

阿宁点了点头,这再明白不过了,就去不顾这里已经摇摇欲坠,跳到我们挖出来的缝隙里,蹲下去用矿灯去照那具骨骸。别人都不了解我在说什么,胖子问我什么标示码,我就她告诉我的东西转述了一遍。

胖子听完就看了看自己的皮带,但是他和潘子的皮带是他们自己的,我的装备是阿宁的,所以只有我的上面才有标识。胖子看了之后就露出了很不快的表情,转头问阿宁:“喂,我说宁小姐,你他娘的该不是又在晃点我们?你们的人早就到过这里!”

阿宁摇头:“不可能,公司里完全没有记录,要是我们到过这里,以我们的实力,绝对轮不到你们来和我合作。”

“那这你怎么解释?”胖子举着皮带扣质问道。

阿宁转头冷冷的看了他一眼,显然心里也不舒服,道,“我不知道!你安静一下,让我先看看这个死人,再来给你解释!”

胖子一下给阿宁呛的说不出话来,就有点愠火,潘子对阿宁也一直不信任,此时就看了看我,想看我的反应。

我倒是相信她确实不知道,虽然阿宁有着前科,但是现在并不是危机时刻,她应该不至于骗我们,而且,如果她们真的来过这里,确实如她所说,她的队伍就不会在到达这里之前就瓦解了。于是给潘子打了个眼色让他别作声,我还是比较理想主义的,既然大家走在同一条路上,人际关系还是不要搞的太紧张的好。

我又看了一眼闷油瓶,想看他的反应,他并没有什么表示。

此时,不知道为什么,我突然想到奶奶在我爷爷的笔记上写过这么一句话:“在危难中和你并肩的人,并不一定能和你共富贵,而在危难中背叛你的人,也并不一定不能相交,世事无常,夫妇共勉之。”

这是写在笔记本里面的一句话,大约是劝解爷爷少和他以前的草莽兄弟来往。

后来也证明了我奶奶看人的透彻,虽然这些人一起上山下海,倒斗淘沙,和爷爷是生死之交,但是后来富贵了之后,大部分就真的散了,这个和那个有矛盾,这个玩了那个的老婆,打杀的都有,弄得爷爷两边不知道怎么帮好。他最后感叹说,在社会上,没有生死之忧,背靠背保护你的兄弟一下也变的不那么重要了。

阿宁和闷油瓶,这两个人还真是应了奶奶的话。

胖子还要说话,我就出来打了圆场,让他们不要问了,让他们再去看那具骨骸。

蛇骨中藤蔓纠结,人尸被纽成了麻花样,很难再发现什么,阿宁把手伸到骨骸里面去,在她脖子处搜索着什么,但是显然没有。

“没有名牌!”阿宁再没有发现,爬了上来,从自己脖子里拿出一条项链,给我们看,“我是1997年进公司的,从那年起我们下项目都要带上这种东西,学美国的军队,好知道尸体的身份,这具尸体没有,应该是1997年之前的队伍,看来应该是我们公司的人没错……”她的表情很严肃,顿了顿又道:“我确实没有在公司里得到任何这一只队伍的资料,我不知道为什么她会在这里!这不符合逻辑。”

“小姐,可是尸体是不会说谎的,你不要说是这条蟒蛇游到你们公司吃了一个人然后再回来。”胖子悻然道。

我看着骨骸,心里也疑惑到了极点,这确实不太可能,看阿宁的所作所为就知道,他们为了得到这里的确切线索,做了多少事情,如果在1997年之前他们公司就有人到达了这里了,那么他们怎么会需要这么多的精力才能再次到达这里。

正想着,一直没有听我们争论,一直在看尸体的闷油瓶就“嗯”了一声。

他突然说话,我们都愣了一下,随即都看向他。他正死死的看着那具蛇骨,脸上不知道什么时候露出了一个惊讶的表情。

我一下就脑袋一炸,要知道要他露出这种表情,是多不容易的事情,他肯定是发现了什么极度奇怪的事情了,我们都忙凑过去看发生了什么。

然而顺着他的目光看去,我们并没有看到什么异样的地方能让我们感到奇怪。看了一会儿,胖子抬头就问他怎么了,大半夜的你别吓人。

闷油瓶没有理胖子,而是转过头看着阿宁,对她说道:“太奇怪了,这好像是你的尸体……”

\chapter{沼泽魔域(上)}

闷油瓶说完,我们一时间都没有明白他是什么意思,几个人就楞了一下,反应过来,我就感觉莫名其妙:都说这尸体死了很久了,怎么一下子就变成阿宁的尸体了,而且阿宁这不好好的站在这里的嘛。

几个人都很疑惑,而阿宁就皱起眉头,不知道闷油瓶这么说是什么意思。

闷油瓶并没有理会我们的眼神,而是将我刚才看到的尸体手骨上的手链小心翼翼的取了下来,递给阿宁,对她做了一个看看的眼神。

阿宁莫名其妙的接过来,看了看闷油瓶,然后去看手链。一开始,她的表情是很疑惑的,但是等她的目光投到这手链上,几秒钟后,她的脸色就变了,刷的惨白。

我们在边上看着,一看她的表情冷汗就下来了,心说这不对啊,这是什么表情,胖子没头没脑的就问了一句:“怎么?这尸体真是你的?”

阿宁没有说话,但她转头看着我们的时候,脸色已经有点发青了,一边就把闷油瓶子给她的手链递给我们,然后伸出她的右手,伸到我们面前。

阿宁的左手上,带着一串铜钱组成的装饰品,这我在海南的时候就注意到过,在魔鬼城里落单迷路的时候,这串铜钱被当成记号压在那些石头下,一共七枚,全部都是安徽安庆铜元局铸造的当十铜币,当时我和她开玩笑说这可能是世界上最值钱的记号了。她和我说,她之所以选择用这种铜钱做手链,就是因为这样的手链世界上绝对不可能有第二条了。

因为有了这样的对话,所以当她把她的手和女尸上的手链一起放到我面前的时候,我就知道了她的用意。

我忙就仔细去看女尸身上取下的手链,刚才粗看的时候,并没有仔细端详,现在仔细一看,就发现手链被铜锈结成了一个整体,拨开表面的铜泥,里面果然就是几枚腐烂的铜钱,上面都有模糊的“光绪元宝”四个魏书。

我一开始还不相信,又掰开了一点,就看到了里面的满文,顿时感到骇然,抬头看向阿宁。

“不用看了,就是当十铜钱。”阿宁对我道,“一共七枚。”

“这……”我哑口无言,心说这怎么可能呢?

这具女尸的手上,戴的也是七枚当十铜钱……可是,当十铜币非常的稀少。阿宁手上的七枚,是她在十年时间里一点一点收集起来的。不说这种想法上巧合的可能性,就是光铜钱的珍稀程度,也不太可能解释这件事情……碰巧有一个女人也有将当十铜币做手链这样的想法,并且也有这样的财力和渠道能够买到七枚铜币,并且也是一个野外工作者,又并且也来到了这里给我们发现尸体,这样的概率是多少……

这样的事情不是扑朔迷离,而是根本不可能发生……

其他人还不明白是怎么回事,我就把这铜钱的珍贵之处,和他们说了一遍,说完之后,他们还是弄不懂,潘子就道:“那就是两串一样的铜钱链子嘛,也许是一个巧合,这种铜钱的赝品很多的。”

闷油瓶看着阿宁,就摇头。

“那这是怎么回事情?”潘子苦笑了起来:“这没天理啊,难道站在我们面前的这位大妹子是个鬼?她在十几年前就死在了这里?”

潘子说着看着阿宁就笑,但是只笑了两声,他就笑不出来了。接着,他的脸色变了,一下就站了起来,去摸手里的刀。

我心里奇怪,心说怎么了,也转头去看阿宁,一看之下,我差点吓晕过去。

只见在雨水中的阿宁的脸,不知道什么竟然变了,她的脸好像融化一样扭曲了起来,眼睛诡异的瞪了出来,嘴角以不可能的角度咧着,露出满口细小的獠牙。

我的脑子“嗡”的一声,心里大叫:“我操!”闪电一般就去摸自己腰里的匕首,同时就往一边退去,想尽量和她保持距离。

慌乱间就忘记了自己是在树上,往后一退,人就踩空了。只是一瞬间,我就栽了下去。

我整个人猛地一缩,心说完了,这一次不摔死也重伤了,忙用手乱抓四周的树枝,但是什么也没抓住。这时候有人一把揪住了我的皮带,我只觉得腰里一疼,几乎给勒断了,不好好歹算是没摔下去。

那人提着我就往上拉,我稳住身体回头看是哪个好汉救的我,一看之下,屁滚尿流,抓着我皮带的竟然是阿宁,一张大嘴口水横流,直滴到我的脸上。

这真是要了命了,情急之下,我意识到给她提上去老子可能就小命不保了,要是摔下去可能还有一线生机,忙去解自己的皮带,可是那皮带勒在我的肚子上,怎么解也不开。我头皮都奓了起来,用力去扯,扯着扯着,我就听到有个人在道:“醒醒,醒醒,你他娘的做什么梦呢?”

\chapter{沼泽魔域(中)}

一下我就醒了,猛地坐起来,头就撞到了一个人的胸口,哎呀一声,一边的阿宁差点给我撞到树下去。

条件的反射的拉住她,我一下子清醒了过来,就发现自己靠在树上,手扯着皮带,已经扯开一半了,边上就是蛇骨的挖掘地,雨还在下,四周的矿灯刺得我的眼睛睁也睁不开。

所有人都莫名其妙的看着我,蛇骨头上已经搭起了防水的布,矿灯架在四周的树枝上,闷油瓶和潘子坐在那里,而胖子睡在我的边上,鼾声如雷。阿宁捂着胸口,显然给我撞的很疼。

我这才明白刚才是在做梦,顿时长出了一口气,一摸脑门,上面也还是湿的,也不知道是冷汗还是昨天雨水。

我是什么时候睡过去的,一想就想了起来,之前把他们叫下来挖蛇骨,但是蛇的骨骸缠入藤蔓最起码有十几年了,里面结实的一塌糊涂,挖了半天没挖出什么来,就轮番休息,没想到一路过来太疲倦了,躺下去就睡着了。脸上还全是雨水,刚才阿宁的口水,就是这些东西。

我尴尬的笑了笑,就站起来,抹了把脸就过去继续帮忙。潘子就在那边不怀好意的问我:“小三爷,你刚才做什么梦呢?还要脱裤子?”

我拍了他一下,心说这次有理也说不清了,不由想到建筑师与火车的故事,心说原来这样的事情并不只是笑话里才有。

看了看表,睡去也没有多少时间,浑身都是湿的,也就是浅浅的眯了一会儿,浅睡容易做噩梦,不过总算是睡了,精神好了很多。话说这梦也有点奇怪,真实得要命,都说梦是人潜意识的反应,我想起老痒以前和我讲过的一些心理上东西,心说难道在我的潜意识里,对阿宁这个女人有着无比的恐惧吗?在梦里竟然是这样的情节。

回头看阿宁,她已经靠到树干上,接替我继续休息了,闭着眼睛闭目养神,人显的有些憔悴,不过这样反倒使得她那种咄咄逼人的气势减淡了不少,看上去更有女人味了,梦境中阿宁扭曲的脸和现在的景象重叠在一起,一下子我又感觉有点后怕。

转头看他们的进度,却发现似乎并没有太多的进展,藤蔓缠绕进骨骸里,经过一番折腾,都碎掉了,腐烂并且已经矿物化的巴掌打的鳞片散落在藤蔓堆里,看起来像是古时候的纸钱。

我就自嘲的笑了笑了,长出了口气,就问潘子他们有什么发现?为什么不挖了。

潘子拿起一边的矿灯,往骨骸里面照去,说没法把这具尸体弄出来,一来骨头都烂的差不多了,一碰就碎,再挖就没了;二来,他们发现了这个东西。

我顺着矿灯的光往下看去,就看到蛇骨的深处,藤蔓纠结的地方,有一捆类似于鸡腿的东西,只不过是黑色的,而且上面结了一层锈壳,我趴下去仔细去看,就发现那竟然是三颗绑起来的老式手榴弹。已经锈成了一个整体。

弹体的四周,有一条发黑的武装带,显然这三颗东西是插在武装袋上的,背在这具尸体身上的。

我看着不由就倒吸了一口冷气,一下子走动都不敢用力了,小心翼翼的退回来。潘子就对我道:“这是胖子先发现的,要不是胖子眼睛毒,我们几个现在都可能被炸上天了。”

我惊讶道:“这具尸体到底是什么人,怎么会带着这种东西?”就算是文锦他们的队伍,要带着装备,也应该带炸药而不是手榴弹啊。这种木柄老式手榴弹完全是实战用的武器,是以杀伤人为目的的,用来做工程爆破基本上没用。

“你还记得不记得定主卓玛那个老太婆和我们说过,在1993年的时候这里有一批搞民族主义分裂的反动武装逃进了柴达木后,民兵追了到戈壁深处,这只队伍却失踪了?”潘子问我道,“我看这具尸骨就是当时那批人之一的,也许是女匪,也许是家眷,他们当时失踪,我看他娘的就是因为误入了这片沼泽了。十几年了,这批人没有再出现,应该是全部死在这里了。”

\chapter{沼泽魔域(下)}

潘子提起这茬,我才想起来,觉得有道理,应该就是这么回事儿,不过我并不同意潘子最后的看法,那时候逃进戈壁的是武装份子,可都是带着好枪的,虽然人数不多,但是装备精良,如果他们真的进入到沼泽之中,不一定就死了,也许在里面待了一段时间离开了也说不定。这里了无人烟,很多偷猎人都是从这里进可可西里,打了动物后直接进走私小道,去尼泊尔,要逮他们一点辙也没有。

甚至,这帮人也有可能在这里定居下来了,当然这可能性很小,这里的条件不适合外面的人生活。我也心说最好不要,这种人太极端了,见了面非打起来不可,我们没枪没炮,要是有个死伤就对不起之前遭的罪了,虽然隔了这么多年,他们的武器也应该都报废了。

胡思乱想着,胖子就醒了,我让潘子去睡一会儿,他说不睡了,这么潮湿,他一把年级了,睡了肯定出问题,这里有那几颗东西,这死人咱们也不能再琢磨了,你们多休息一下,我们就离开这里,反正雨也小了。再往前走走,天也就该亮了,到时候找个好点的地方生上火再慢慢休息。

话虽然这么说,但是这样的条件下,主观想去睡觉确实也睡不着,我们缩在一起,一边抽烟,一边就看着外面黑暗,听雨声和风吹过雨林的声音。潘子就擦他的枪,这里太潮湿,他对他枪的状况很担心。其他人就聊天,聊着聊着,闷油瓶却睡着了。

潘子和我讲了他打仗时候的事情,当时他是进炊事班的,年纪很小,有一次,他们的后勤部队和越南的特种兵遭遇了,厨师和搬运工怎么打的过那些从小就和美国人打仗的越南人?他们后来被逼进了一片沼泽里,因为越南人虐待俘虏,所以他们最后决定同归于尽,当时保护他们的警卫连每人发了他们一颗手榴弹,准备用作最后关头的牺牲。

越南人很聪明,他们并不露头,分散着在丛林里潜伏向他们靠拢,这边放一枪,那边放一枪,让他们不知道到底他们要从哪里进来。他们且战且退,就退到沼泽的中心泥沼里,一脚下去泥都裹到大腿根,走也走不动,这时候连长就下命令让他们准备。

所有人拿着手榴弹,就缩进了泥沼里,脸上涂上泥只露出两个鼻孔。这一下子,倒是那些越南人慌了,他们不知道为什么,不敢进入沼泽,就用枪在沼泽里扫射,后来子弹打得差不多了,就撤退了。

潘子他们在泥沼里不敢动,怕这是越南人的诡计,一直忍了一个晚上,见越南人真的走了,才小心翼翼的出来,可是一清点人数,却发现少了两个人,他们以为是陷到泥里面去了,就用竹竿在泥沼里找,结果钩出了他们的尸体,发现这两个人已经给吃空了,只剩下一张透明的皮,胸腔里不知道什么东西在鼓动。

这样的经历之后,潘子开始害怕沼泽,后来调到尖刀排到越南后方去作战,全排被伏击死得就剩下他和通信兵的时候,他们又逃到一个沼泽边上,潘子就宁可豁出去杀光追兵,也不肯再踏进这种地方一步。

潘子说着说着,就不停的打哈切,我也听的朦朦胧胧的,眼皮只打架,又睡了过去。

半睡半醒,也不知道过了多久,感觉又开始要做梦了,却感觉有人摇我。那是我最难受的时候,就想退开他继续睡,没推到他人,一下子我的嘴巴却给捂住了。

这一下我就睁开了眼睛,就看到是阿宁在捂我的嘴巴,一边的潘子轻轻在摇胖子,几个人都好像是刚醒的样子,在看一边。

我也转过去看,就看到大风刮着我们头顶上的一条树枝,巨大的树冠都在抖动,似乎风又起来了,但是等我仔细一感觉,却感觉不到四周有风。再一看头顶上,一条褐色的巨蟒,正在从相邻的另一颗树上蛇行盘绕过来。

\chapter{狂蟒之灾}

说是头顶上的树冠,其实离我们的距离很近,几乎也就是两三米,蛇的鳞片都能看的清清楚楚。这是条树蟒,最粗的地方有水桶粗细,树冠茂密,大部分身体隐在里面也不知道有多长,让我感觉到惊异的是,蛇的鳞片在矿灯的光线下反射着褐金色的色泽,好像这条蛇好像被镏过金一样。

刚才爬上来的时候,四周肯定没有蟒蛇,这蛇应该是在我们休息的时候顺着这些纠结在一起的树冠爬过来的。蟒蛇在捕食之外的动作都很慢,行动很隐蔽,而外面还有少许的风,丛林里到处都是树叶的声音,几个人都迷糊了,一点也没有感觉到。守夜的潘子也没发现它的靠近。

不过这里出现蟒蛇倒也不奇怪,热带雨林本来就是蟒蛇的故乡,而古怪的事情看多了,区区一条大蛇似乎还不能绷紧我们的神经。

潘子他们都见过大世面,几个人都出奇的冷静,谁也没有移动或者惊叫。这种蛇的攻击距离很长,现在不知道它对我们有没有兴趣,如果贸然移动,把蛇惊了,一瞬间就会发动攻击,我们在树上总是吃亏。

我们这边僵持着,树蟒则缓缓的盘下来,巨大的蛇头挂到树枝的下面,看了看我们,黄色怨毒的蛇眼在黑夜里让人极端的不舒服。

潘子已经举起了枪,一边还在推胖子,这王八蛋也真是能睡,怎么推也推不醒。闷油瓶的黑金刀也横在了腰后面,另一只手上匕首反握着。所有人都下意识的往后面缩去,尽量和这蛇保持距离。

我在最后,心里暗想要攻击也不会先攻击我,就看了看树下,琢磨着如果跳着下去行不行,这里毕竟是树上,而且颇有点高度,活动不开,硬拼恐怕会吃亏。

大雨之后,两边崖壁上的瀑布在峡谷的底部会聚成了大量的小溪,现在这些小溪已经汇合了起来,树下的烂泥地已经成了一片黑泽,下面应该是树根和烂泥,不晓得跑不跑的开。

想着又转头去看前面的雨林,这时候四周又传来了树冠抖动的声音,窸窸窣窣,这一次好像是从我的身后传了过来。

回头一看,我的冷汗就像瀑布一样下来了。就在我的脖子后面又挂下来一条小了一点的树蟒,也是褐金色的,这一条大概只有大腿粗细,离我的脸只有一臂远,一股腥臭味扑鼻而来。

我吓得又往前缩去,前面的人缩后,我缩前去,几个人就挤在了一起,再无退路。

这下子真的一动也不敢动了,所有人都僵在那里。人瞪蛇,蛇瞪着人,连呼吸都是收紧的。

我心里就感觉奇怪,蟒蛇是独居动物,有很强的领地观念,很少会协同狩猎,除非是交配期间,难道这里的雨季是它们的交配期?那真是进来得不是时候。这两条蟒蛇一前一后,似乎是有意识的要夹攻我们,很可能是一对刚交配完的公母,想起蛇骨里面的人尸,我就觉得一阵恶心,心说他娘的我可不想成为你们HAPPY完的点心。

两相僵持了很久,谁也没动,蟒蛇可能很少见人,一时间也搞不清楚状况,所以不敢发动攻击,而且闷油瓶和潘子的气势很凌厉,两个人犹如石雕一样死死盯着蛇的眼睛,蟒蛇似乎能感觉到潜在的危险,犹豫不前。

十几分钟后,果然两条蟒蛇找不到我们的破绽,就慢慢的缩回到了树冠里,似乎想要放弃。

看着两边的蛇都卷了上去,我不由缓缓的松下一口气,潘子紧绷的身子也缓缓的松下来,枪头也慢慢的放了下来。我心中庆幸,说实话,在这种地方和蛇打架,还是不打的好,不说这蛇的攻击力,就是从这里失足摔下去也够戗。

可就在我想轻声舒口气压压神的时候,一边的胖子突然翻了个身,打了一个很含糊的呼噜,而且还拉了一长鼻音。

那是极度安静下突然发出的一个声音,所有人一下都惊翻了,阿宁忙去按他的嘴巴,可已经来不及。整棵树猛地一抖,一边腥风一卷,前面的树蟒又把头探了回来,这一次蛇身已经是弓成了U形,一看就知道是要攻击了。

潘子立即举枪还是慢了一步,蟒头犹如闪电一般咬了过来。刹那间,潘子勉强低头,蛇头从他头侧咬了过去,他身后的闷油瓶视线不好,躲闪不及就给咬住了肩膀。接着肌肉发达的蟒身犹如狂风一样卷进来,在极短的时间内它好比蟠龙一样的上半身猛的拍在我们脚下的蛇骨上,已经摇摇欲坠的骨骸堆顿时就散架了,我们被蛇身撞翻出去,接着脚下就塌了,所有人裹在蛇骨里摔了下去。

幸好蛇骨之中缠绕着大量的藤蔓,骨断筋连,塌到一半各部分都给藤蔓扯住了。我手脚乱抓,抓住藤蔓往下滑了几米也挂住了,抬头一看,就看到闷油瓶给蟒蛇死死的缠了起来,卷到了半空,黑金古刀不知道给撞到什么地方去了,蛇身蜷缩,越盘越紧,闷油瓶用力挣扎但是毫无办法。

我急火攻心,就大叫潘子快开枪,转头却看不到潘子,不知道摔到哪里去了。就在这个时候,我就看到半空中的闷油瓶突然一耸肩膀,整个人突然缩了起来,一下就从蟒身的缠绕中褪下来,落到一根树枝上,翻身就跳到纠结的藤蔓上往下滑,一下就滑到我的边上,对我大叫:“把刀给我!”

我赶紧去拔刀,可是太紧张了,拔了几下竟然没拔出去来。这时候那蟒蛇发现自己盘了个空,不由大怒,猛然盘回树上,转瞬之间就到了我们跟前,蛇头一翻又猛咬了过来。

“我靠”,我大骂了一声,眼看着血盆大口朝着自己的面门就来了,那种视觉冲击力恐怕很少人能见识,闷油瓶抓着藤蔓一下就从藤蔓中扯出一块骨头扔了过去,蟒蛇凌空一躲,给我们争取了少许时间,闷油瓶就对我大叫:“快跳下去!”

可那时候我已经蒙了,也不知道自己是怎么想的,条件反射就蜷缩起了身子,一下子反应不过来,那一刹那蛇头又弓了起来,闷油瓶“啧”了一声,飞起一脚就把我踹翻了出去。

这一脚极其用力,我拉的藤蔓就断了,慌乱间又是乱抓,但是连抓了几下却什么都抓不住,就自由落体直落而下,连撞了好几根树枝,然后就重重摔到了地上。幸亏下面是水和烂泥,我翻了几下趴在里面,一嘴巴的泥,却不是很疼。

恍惚中就给人扶了起来,就往外拖,拖了几步才开始感觉浑身都火辣辣的疼,抹掉脸上的泥就看到扶着我的是阿宁和胖子,再看四周,矿灯全掉在泥里熄灭了,什么都是模模糊糊的。潘子端着枪瞄着树上,但是从树下看上去,树冠里面一片漆黑,什么可看不到。

“你怎么样?”阿宁就问我。

我摇头说没事,他们就拖着我往外走,我就说不行,那小子还在树上,不能扔下他不管!

刚说完整颗树狂抖,闷油瓶像只猴子一样踩着树干就跳了下来,同时树叶树皮卷着一个巨大的黑影一阵风一样也跟了下来。两个影子几乎是裹在一起摔在泥水里,水花还没落下,就看到蟒蛇一个扑咬就朝他冲了过去,闷油瓶矮身一闪就裹进水花里看不见了。

我一看心说我操,他竟然在和这条蛇肉搏,忙大叫了一声潘子,快去帮忙!

潘子不等我说早就骂着冲过去了,歪头躲过水花,举枪瞄准,终于开了第一枪。他的枪法极其好,一枪就打在蛇头上,凌空把蛇打的扭了起来,一下闷油瓶就从蛇身下翻了出来,拔腿就往外跑。那蛇竟然没死,猛的一翻,犹如弹簧一样又反身扑咬了过来,但同时潘子又是一枪,又将它打的缩了回去。他同时后退,然后对我们大喊:“我掩护!你们快出——!”

话音未落,突然就从树上猛的就射下来另一条树蟒,一下就咬住了潘子的肩膀,接着一闪间蛇身一弓就将他整个提了起来。

那攻击太快了,谁也没有反应过来,我们大惊失色,他已经给卷到了半空中。我看着他手脚乱抓,顿时心里一沉,心说完了!

说时迟那时快,就见潘子临危不惧,单手连转了几下,就把自己的折叠军刀翻了出来,然后往上一刺,猛地就扎进了蛇的眼睛里。那巨蟒疼的整个身子都弯了,一下就松口了,潘子给甩了一下,撞在树上翻着跟头摔下来,满脸都是血。接着阿宁就从背包里打起两个冷烟火,双手往膝盖上猛一敲点燃,就冲到蟒蛇和潘子中间,用冷火焰挡住蟒蛇同时对我们大叫:“把他拖走,跑!”

我大叫不要!冷烟火的温度不够!阿宁就道,你知道蛇不知道!

我和胖子猛地冲过去,扶起了潘子就往树林里跑,但是还没有走几步,突然水花伴着烂泥浪一样的打了过来。转头一看,闷油瓶身后的巨蟒竟然仍旧没死,蛇头上都是血,巨大的身躯狂怒着追着闷油瓶,而后者正朝我冲了过来,巨大的蟒蛇在身后狂舞,看上去竟然像飞了起来一样。

蟒蛇很生气!后果很严重!我脑子突然出现了这么一句话,看着那情形竟然脚软了,闷油瓶大叫“趴下”,胖子一把抓住我往前跑了几步,猛就卧倒在水里。蟒蛇瞬间就到了,闷油瓶和阿宁一翻身也滚进泥里,蟒蛇巨大的身躯贴着我的后背卷了过去,一个刹车不住,就撞到一边的大树上,树几乎给撞折,树叶和树上的附着物下雨一样的掉下来。

我们爬起来,也分不清楚东南西北了,胖子的杀心大起,大骂了一声:“我操你奶奶的,跟它拼了!”说着竟然一下抽出我腰里的刀,朝着那撞蒙的蟒蛇冲了过去。我赶紧冲上去,拦腰抱住他,不让他过去,闷油瓶也爬起来,我看到他肩膀上全是血,显然受了很重的伤。他气喘着指着一边的丛林,就对我们叫道:“快跑,这两条蛇不对劲!”

一看闷油瓶伤成这样,胖子也犯了嘀咕,忙将潘子背起来,将潘子的枪扔给我,我抬枪殿后,一行人就直往丛林里逃去。刚冲进灌木里,后面水花溅起,那蛇竟然又来了。

谁也没工夫看后头了,树木之下是丛极其茂盛的灌木和蕨类植物,我们一下冲进去,枝条都带着刺,滑过我裸露的皮肤,拉出了无数血条,疼得我直咧嘴,但是也管不了这么多了,咬紧牙关就狂跑。

谁也想不到我们可以在丛林中达到那种速度,要是一直按照这个速度,我们早在今天中午就过峡谷了。我们很快就冲到了峡谷的边缘,山壁上全是瀑布,水一下就深到了膝盖,这下再也跑不快了。

我们回头一看,我靠,那条蛇几乎就没给我们落下多少,蟠龙一样身子在灌木里闪电一般跟了过来。我们想要再跑,再往前就是瀑布,没路了,胖子就大骂:“我操,谁带的路!”

几个人都慌了,这里水这么深,动又动不了,而树蟒在水里十分的灵活,这一下真的凶多吉少了。这时候阿宁看到什么,对我们叫道:“那里!”

我们顺着她的矿灯看去,只见一边山岩的瀑布后面,有一道裂缝,似乎可以藏身,胖子就急叫:“快快!”

我们冲过去,冲进瀑布,裂缝的口子很窄,蟒蛇肯定进不来,我们人进去都很勉强,几个人都侧身往里面挤,里面全是水,我们几个勉强挤了进去,胖子却打死也进不来了。

我们拼命的拽他,他也拼命的往里面挤,也只是进来一条腿,在里面的阿宁就把矿灯照向缝隙外,巨大的蛇头已经在瀑布的水帘外,那是一个巨大的影子。胖子也慌了,大叫你照什么!关灯关灯!

我就上去捂住他的嘴巴,轻声喝他闭嘴。但是所有人都知道,躲肯定没用了,都抄起家伙,准备拼命了。

可是奇怪的是,那条蟒蛇竟然在瀑布外面徘徊,没有把头探进瀑布里来,徘徊了几下,竟然扭头走了。

这一来,我们面面相觑,都莫名其妙。只要这蛇稍微把头在往里一探,胖子肯定就完蛋了,我们不可能袖手旁观,那就是一场死战,不死一半也够戗,怎么突然它就走了,难道它害怕这瀑布?

这时候,我们都听到缝隙的深处就传来一连串“咯咯咯咯”的声音,好像是鸡叫一般,外面水声隆隆,也并不响亮,但是这里听到鸡叫,特别的醒耳,我们一下就全部听到了。

所有人转头,此时才有精力来观察这条缝隙,发现里面水都没到我们的腰部了,再看缝隙的里面,再进去就没有了,而在尽头的石头缝里,站着什么东西。这东西完全是隐在黑暗里的,利用矿灯的余光,根本发现不了。

我的眼神恍惚了一下,也看不清楚,但是我一看到这东西站着的姿态,就感觉不秒。我也说不出到底奇怪在什么地方,于是让阿宁把矿灯转过来。

灯光探过去,那东西露出了真面目,我看了一眼,足有两三秒,没有意识到那是什么,那是极度惊讶的两三秒,随即我就反应了过来,简直不敢相信自己的眼镜。

我看到在缝隙的最里面,有一条大概手腕粗细的蛇,这条蛇不是蟒蛇,浑身火红,蛇头是非常尖锐的三角形,上面竟然长着一只大大的鸡冠。而让我不敢相信的是,这条蛇竟然是直直的站在那里,蛇头低垂,目露凶光的看着我,整个姿态好似一个没有手脚的人一样。

我看着那蛇的眼睛,一下就几乎不能动了,就这样给它瞪着,直到阿宁拉了我一下,那一瞬间我才意识到我看到了什么东西,一下就知道为什么那条巨蛇要放弃我们了。童年时候的恐惧一下就传遍全身。

\chapter{蛇王}

这竟然是一条“野鸡脖子”。

这里怎么会有这种蛇!

我再仔细去看,火红的鸡冠和蛇身,以及那种直立的骇人的姿势,就是“野鸡脖子”没错。

一下我的冷汗就滋滋的冒出来。这种蛇十分的罕见,在我们老家,它被叫做“雷王红(音译)”,我小时侯在山上见过一次。据老人说,这蛇就是蛇里的帝王,所有的蛇都怕它,它贴地而飞,行迹如电,而且其毒无比,爬过的地方,植物杂草甚至会自己分开。而且这种蛇不能打,打死了会有同类来报复。

我后来看过一本清人笔记小说,云这种蛇乃是小龙,沿着山川龙脉而栖,又说是盘踞在龙脉上的蛇精,有的地方有天雷杀妖的传说,大多是有雷劈在山上,炸出这种蛇的事情。不过这种蛇近几十年就几乎绝迹了,竟然在这里还有,真是出乎我的意料。

胖子他们没见过这种蛇,都啧啧称奇,几个人里面只有闷油瓶也和我一样脸色有了变化。不过那火红的蛇身和凶狠的姿势,就表明了这剧毒蛇的身份,几个人也都不敢轻举妄动。

这真是刚逃离蟒口,又遇到毒蛇,我心里一边懊恼,一边提醒自己,看来在这个地方,真的要加倍小心,不能什么地方都乱钻了。

和蟒蛇硬拼还有一线生机,和毒蛇搏斗,一般不是全胜就是全输,这个险没人肯冒,而且“野鸡脖子”一般也不会招惹人,现在它做出这种威胁的姿态,是一种警告,可能这缝隙是它的巢穴。

那这里绝对不能呆了,我就挥手让他们不要做出攻击的姿态,慢慢出去。阿宁扯出冷烟火,递给我,让我当武器。

我把冷烟火横在自己面前,这样不至于在“野鸡脖子”突然发动攻击的时候只能用手去挡。我们小心翼翼的退出缝隙,一个一个,都很顺利。轮到我的时候,我总算松了口气,转头看了一眼缝隙里面,黑黑的已经看不到蛇了,心说幸好没出事。

从缝隙里下来,踩进水里,胖子就用矿灯探到瀑布外面,照了几圈,说:“大蛇也不在了,安全了……”

几个人都吁了一口气,我们去看被胖子扶着的潘子,他有气无力的摆了摆手,说没事情,就是摔得有内伤了,不过还死不了。我们互相看了看,都发出苦笑,几个人衣衫不整浑身是泥,阿宁的胸口都几乎露了出来,她若无其事扯了扯自己的衣服遮住,我们也没有力气去看。装备包只剩下两个,闷油瓶的黑金古刀丢了,胖子手里是我的匕首,他自己的匕首也没有了。闷油瓶和潘子的肩膀上全是密密麻麻的血孔,给蟒蛇的牙齿咬的,特别是闷油瓶,他可能是硬挣脱出来的,很多伤口都豁开了。

真是没有想到一条蟒蛇就能把我们搞的如此狼狈。

我看了看天,雨已经停了,天光已经亮起,峡谷的边缘树木稀疏一点,能够看到黎明即将到来的那种晨曦,一边是瀑布,一边是丛林,四周传来鸟叫,如果不是亲身经历了刚才的恶战,这将是多么美好的情形。

众人安静的看了一会儿风景,胖子就问道,“现在怎么办?”

阿宁走到瀑布边上,接了点冲下来的雨水,洗了洗脸,就说:“等天亮了,我们回去把装备捡回来,然后找个地方休息一下,这里太危险了,我们还是得快点出去。”

胖子道:“他娘的,你说的容易,刚才我们跑的时候,完全是乱跑,也不知道那颗树是在什么地方,我们怎么去找?”

“那也得去找,现在不回去,等需要的时候想去找就更不可能了。”阿宁疲惫的按了按脸,又卷起自己的袖子,把头伸到瀑布里面草草冲洗了一下,洗完之后短发一甩,泥砂退去,俏脸总算恢复到以前的样子。就招呼我们出发。

我想到还要回到那个地方,心里就长叹了一声,但是这个女人说的没错,这个时候确实必须这么干,就是不让人喘气。感觉还没有缓过来。

几个人背起自己的东西,阿宁到底是个女人还是比较爱干净的,看我们走的远了,就拉开了自己的衣服,用水去冲自己的胸口,这个时候,我的眼角一闪,就看到瀑布里面有一团红色闪了一下,同时我们隐约听到了“咯咯”的一声。

我突然感觉到不妙,对阿宁道:“小心一点,离瀑布远点!”

“怎么了?”阿宁转过头看了我一眼,不知道为什么,露出了一个很淡的笑容,和她以前的那种笑容不同,我看着惊艳了一下。

就在那一刹那,一下子,一条火红的蛇就猛地从瀑布里钻了出来,一下就盘到了阿宁的脖子上,高高的昂起了它的头,发出了一连串凄厉而高亢的“咯咯咯”声。我一看完了!丢掉手里的东西就冲过去,才迈出去第一步,就看着那“野鸡脖子”闪电一般的咬了下去。阿宁用手去挡却没有挡住,蛇头一下就咬住了她的脖子。她尖叫了一声,一把把蛇拽了下来,扔到一边,捂住脖子就倒在水里。

我们冲了过去,那蛇竟然不逃,一下又从水里蹿起起来,犹如一支箭一样朝我们飞了过来。胖子叫了一声,用刀去劈没劈到,眼看又要中招,一边的闷油瓶凌空一捏,一下就把蛇头给捏住了。蛇的身子一下盘绕到他的手臂上,想要把蛇头拔出来,就见闷油瓶用另一只手卡到蛇的脖子上,两只手反方向一拧,咔嚓一声,蛇头给他拧了三百六十度,然后就往水里一扔,那“野鸡脖子”扭动了几下,就不动了,漫漫浮了起来。

我们忙去看阿宁,我上去抱起她,却见她脸上的表情已经凝固了,喉咙动着想说话,眼里流着眼泪,似乎有一万个不甘心。我头皮一下就麻了起来,不知道怎么办了,整个人发起抖来。接着,只是几秒的工夫,她的眼神就涣散了,整个人软了下来,然后头也垂了下来。

\chapter{蛇沼鬼城(上)}

两分钟后,阿宁停止了呼吸,在我怀里死去了。凌乱的短发中俏丽的让人捉摸不透的脸庞凝固着一个惊讶的表情,我们围着她,直到她最后断气,静下来,时间好像凝固了一样。

突然间我感觉一切都停止了,心中悲切,想哭又哭不出来,胸口像是被什么堵住了。

这一切发生的太快了。

一路上过来虽然危险重重,我也预料到了有人会出事,但是我从来没有想过这个女人会死,而且死的这么容易,这么突然。事情毫无征兆,就这么发生了,然后刚才还在说话的人,一下就这么死了。而且是真的死了,我们连救的机会都没有。

我一开始还不相信我眼前的情形,以为自己在做梦,这个女人怎么可能会死呢?她是如此强悍,艳丽而狡猾,外表柔弱却有坚强如铁的内心,虽然我并不喜欢她,但是我由衷的佩服她。如果要死的话,这里所有人都比我强,最容易死的应该是我才对。

可是她确实是死了,就在我的面前,这么容易的,真真切切的,随随便便的死去了。

我一下子有了一种被打回原形的感觉,一次次的事情,虽然都是危险重重,但是我们几个人都闯了过来,就连在秦岭我一个人出去,也勉强活着回来了。我一度认为在这些事情之后,我们这样的人已经非常厉害了,有着相当的经验,只要我们几个人在一起,虽然会遇到危险,但是大部分都能应付,就算要死,也应该是死在古墓里最危险的地方。但是现在,阿宁就这样轻易的死在了一条蛇上。我突然就意识到,不对,人本来就是脆弱的动物,不管是闷油瓶、潘子,还是我,在这种地方,要死照样是死,身手再好,经验再丰富也没有用。

这就是现实的法则,不是小说或者电影里的情节,只要碰上这种事情,我们都会死,就算是闷油瓶,如果站在瀑布边上,刚才肯定也死了!

我抬起头看前面茂密的丛林,一下子就感觉到无比的恐惧和绝望。那一瞬间我简直想拔腿而逃,什么都不管,逃离这个地方。

这个时候天终于亮了,阳光从峡谷的一边照了下来,四周都亮了起来,前面水气腾腾,瀑布溅起的水幕在阳光的照射下,形成了一团笼罩在茂密雨林上空的白色薄雾。

美景依旧,美人却不在了。

潘子是个看破生死的人,此时虽然也是一脸可惜之色,但是比我们从容多了,只是受了重伤,也说不出太多话来,就对我们道,这是意外,虽然很突然,我们也必须接受,这里不知道还有没有那种蛇的同类,不宜久留,我们还是走吧,找个干净点的地方再想办法。

我想起闷油瓶刚才杀了那条鸡冠蛇,心中也多了些恻然,转头去看浮在水面上的蛇尸,却发现尸体不见了。这种蛇据说会对杀死同类的东西报仇,然而不死不休,诡异异常,待在这里确实有危险,想起阿宁的惨状,也待不下去了。

一时之间也不忍心将阿宁的尸体丢在这里,我就背了起来,胖子扶起潘子,几个人不敢再往丛林里去,就沿着峡谷的边缘,蹚水前进。

谁都不可能聊天了,胖子也没法唱山歌了,我甚至都不知道为什么还要往前走。脑子里一片空白。

深一脚,浅一脚,恍惚的往前走了十几分钟,却一直无法找到干燥的地方让我们休息。日头越来越高,昨夜大雨的凉爽一下就没了,所有人都到达了极限,太累了,一个晚上的奔袭,搏斗,爬树,死亡,逃生,就是铁人也没力气了,更要命的是,随着温度的升高,这里的湿度变的很大,胖子最受不了这个,喘的要命,最后都变成潘子在扶他。

正在想着要不要提出来就地休息算了的时候,突然前面的峡谷出现了一个向下的坡度,地上的雨水溪流变得很急,朝着坡下流去,我们小心翼翼的蹚着溪流而下,只下到坡度的最下面,就看到峡谷的出口出现在我们面前。

外面树木稀疏起来,全是一片黑沼,足有两百多米,然后又慢慢的开始茂密起来,后面就是一大片泡在沼泽中的水生雨林,都是不高但是长势极度茂盛的水生树类,盘根错节,深不可测。

\chapter{蛇沼鬼城(中)}

我们都面面相觑,一种宿命的感觉传来,原来到所谓峡谷的出口,昨天晚上我们只剩下这十几分钟的路程了,而我们竟然选择了停下来,如果当时坚持走下去,可能结果就完全不同了。

再往前走了几步,来到了沼泽的边缘,从这里看沼泽,视野有限,并不像我们在外面山谷的顶端看到的那么辽阔。如果不是沿着山壁在走,也不知道已经出了山谷了,前方还是一片密林,感觉只不过是峡谷的延续。当然区别还是有的,脚下越走就觉得不对,水越来越深,而且地下的污泥也越来越站不住。

好在沼泽的浅处,有一块很大的平坦石头,很突兀的突起在沼泽上,没有给水淹没。我们很奇怪怎么会有这么大的一块石头在这里,小心翼翼的蹚水过去,爬了上去,才发现这块巨大的石头上雕刻着复杂的装饰纹路,而且在水下有一个非常巨大的影子,似乎是好几座并排的大型的雕像的一部分。

这里是西王母城的一个入口,西王母是西域之王,在很长一段时间里都是西域的绝对精神领袖,那么西王母之城的入口自然不会太寒酸,也许这是一座当时的石雕,或是是这里城防建筑上的雕像,用来给往来的使节以精神上的威慑,当然这么多年后,这种雕像在雨水的冲刷下自然不可能保存。

我乍一看石头上的古老纹路,就感觉和吴哥窟的那种很像,仔细看才发现并不是高棉佛教的纹路,而是因为这块石头也给风吹雨打得发黑发灰,看起来特别的古老和神秘。

正想着如果这里有一座倒塌的雕像,那么是否沼泽下面还有其他的遗迹,就听到胖子招呼了一声,让我们看他那边。

我们转头看去,只见在阳光下,前方的黑沼比较深的地方,现出了密密麻麻的巨大的黑影,似乎沉着什么东西,看上去似乎是石头,有些就完全在水下。我和闷油瓶用望远镜一看,才惊讶的发现,在沼泽水下的影子,似乎全部都是一座座残垣断壁,一直连绵到沼泽的中心去。

西王母的古城的废墟,竟然是被埋在了这沼泽之下的。

“这座山谷之中应该有一座十分繁茂的古城,西王母国瓦解之后,古城荒废了,排水系统崩溃,地下水上涌,加上带着泥沙污泥的雨水几千年的倒灌,把整座城市淹在了水下。看来西王母城的规模很大,我们现在看到的只是凤毛麟角。”闷油瓶淡淡道。

我也有一些骇然,古城给水淹没这种事情倒是比较常见,这片沼泽其实绝对面积不大,当时的古城竟然已经发展到这座盆地的边缘,说明当时的文明已经到了鼎盛时期。但是这么说来的话,西王母宫,启不是也在水下的污泥里了,我们如何进入呢。

不过,想起文锦的笔记,这篇沼泽形成了也不是一年两年了,在二十世纪九十年代她的队伍中,霍玲就进入了西王母宫,也是在大雨之后,那么应该是有办法进去的,只是我们还没有到达那种境况而已。

\chapter{蛇沼鬼城(下)}

石头上相对干燥,我将阿宁的尸体放下,几个人都筋疲力尽,坐下来休息。

把衣服脱掉,铺在石头上晒,胖子想打起无烟炉,可是翻遍了行李却一只也找不到,看样子昨天晚上混乱的时候掉光了,没法生火,就用燃料罐头上的灯棉凑合。意料之外的是,这里的沼泽竟然是咸水,看样子有附近的大型盐沼的水系联通,万幸雨水从峡谷冲刷下来,口子上基本上没有味道,不然我们可能连喝水都成问题。我先放了几片消毒片煮了点茶水喝。然后打水清洗自己的身体。

浑身在水里泡了一个晚上,身上的皮都起皱了,鞋子脱掉,脚全泡白了,一扣就掉皮,就算我扣紧了鞋帮,脱了袜子之后脚上还是能看到小小的类似于蚂蟥的东西吸在脚上,拿匕首烫死。挑到眼前来看,也看不出是什么虫子。

不过,如果沼泽里是咸水的话,昆虫的数量应该相对少一点,至少这里不太可能有咸水蚂蟥,这对于我们进入沼泽深处来说,是一个大好消息。

潘子递给我他的烟,说这是土烟,他分别的时候问扎西要的,能怯湿。这里这种潮湿法,一个星期人就泡坏了,抽几口顶着,免得老了连路也走不了。

我接过来吸起来,烟是包在塑料袋里的,不过经过昨天晚上这样的折腾,也潮了,吸了几口呛的要命,眼泪直流,不过确实挺有感觉,也不知道是生理上的还是心理上,抽起来感觉脑子清醒了不少,疲劳一下子不这么明显了。

胖子也问他要,潘子掐了半根给他。他点起来几口就没了,又要潘子就不给了。这时候我们看到闷油瓶不吭声,看着一边的沼泽若有所思,潘子大概感觉少他一个不好意思,就也递了半根给他。我本以为他不会接,没想到他也接了过来,只不过没点上,而是放进嘴巴里嚼了起来。

“我靠,小哥你不会抽就别糟蹋东西。”胖子抗议,“这东西不是用来吃的。”

“你懂个屁,吃烟草比吸带劲多了,在云南和缅甸多的是人嚼。”潘子道,不过说完也觉得纳闷,就看向闷油瓶:“不过看小哥你不像老烟枪啊?怎么知道嚼烟叶子?你跑过船?”

闷油瓶摇头,嚼了几口就把烟草吐在自己的手上涂抹手心的伤口。我瞄了一眼,只见他手心的皮肉发白翻起,虽然没有流血,但是显然这里的高温也使得伤口很难愈合,涂抹完后他看了眼潘子,潘子用怀疑和不信任的眼光盯着他,但是他还是没有任何表示,又转头去看一边的沼泽,不再理会我们。

这样的局面我们也习惯了,闷油瓶对于自己的情况,似乎讳莫如深,但是我明白,这些问题有很大的一部分连他自己也不知道答案,“凭空出现的一个人,没有过去,没有将来,似乎和这个世界没有任何的联系”,这是他对他自己的评价,偶尔想想真的十分的贴切。

脱的光溜溜的,加上身上水份的蒸发,感觉到一丝的舒适,觉得缓了一点过来,胖子就拿出压缩的肉干给我们吃,我们就着茶水一顿大嚼,也不知道是什么味道,总之把肚子填满了。肚子一饱就犯困,于是潘子用背包和里面的东西搭起一个遮挡阳光的地方,他放哨,我们几个缩了进去。大家都心知肚明,进入沼泽之后可能再也没有机会休息了,现在有囫囵觉睡就是种福利了,也没有什么多余的想法,一躺下,眼睛几乎是一黑,就睡了过去。

这一觉天昏地暗,也不知道睡了多久。迷迷糊糊的醒了过来,却发现四周一片漆黑,浑身黏糊糊的,揉了揉眼睛一看,发现竟然天黑了,而且又下雨了。潘子在一边倒在行李上,也睡着了,胖子在我边上,打着呼噜,闷油瓶脸朝内也睡的很深。

远处的燃料罐头还燃烧着,不过给雨水打的发蓝,也照不出多远。我拿出风灯把火苗点上,然后想把其他几个人都叫醒,这个时候却发现了有点不对劲。

原来一边裹着阿宁尸体的睡袋,不知道什么时候给人打开了,阿宁的上半身露了出来。

\chapter{尸体的脚印}

这在平时是很普通的一件事情,在戈壁中行进,进入到绿洲之前,我们上半身一般都不脱衣服,就下半身捂进睡袋里取暖,这样能够在有突发事件的时候迅速起身。阿宁这样躺在睡袋里的样子,这一路过来也不知道看了多少眼了,十分的熟悉,然而想想,又想起她已经死去了,感觉就很凄凉。

不过我睡着的时候尸体明显是完全裹在睡袋里的,是谁把她翻出来的呢?难道是潘子?他把她翻出来干什么呢?

站起来走到尸体边上看了下,我就发现了似乎有点不对劲。尸体确实给人动过了,双手不知道为什么,不自然的蜷缩着,整具尸体的样子有点奇怪。

我下意识的看了看四周,天色灰暗,沼泽里不同在峡谷,四周的树木比较稀疏,没有什么东西能照出来,那燃料罐头的火苗又小,四周完全是一片沉黑,什么也看不到。

转身叫醒了潘子,潘子睡不深,一拍就醒了过来,我就问他是不是他干的?

潘子莫名其妙,凑过来看了看,就摇头,反而用怀疑的眼神看着我,我看他的表情也不像是装的,就更纳闷了。

一下又想到了胖子,心说难道胖子看上阿宁身上的遗物了?这王八蛋连自己人身上的东西也不放过吗?但我印象里胖子虽然贪财,但是这种事情他也不太可能干。

潘子用一边的沼泽水洗了把脸,就走到阿宁尸体的边上,打起矿灯照了下去,想看看到底是怎么回事。

阿宁的脸上还凝固着死亡那一刹那的表情,现在看来有点骇人。尸体给雨水打湿了,潘子蹲下去,把她脸上的头发理得整齐了一些,我们就看到阿宁被咬的地方的伤口,已经发黑发紫,开始腐烂了,身上的皮肤也出现了斑驳的暗紫色,这里的高温已经开始腐蚀这具美艳的尸体了。

照着,我们就发现尸体的衣服上有好几条泥痕,潘子摸了一把,似乎是沾上去不长时间,顺着泥的痕迹照下去,我们就陡然发现在尸体的边上,有几个小小的类似泥脚印的东西。

潘子看了我一眼,就顺着这些泥印子照去,发现脚印一直是从沼泽里蔓延上来的,因为下雨,已经很不明显,只有尸体边上的还十分的清晰。

沼泽里有东西!我们的神经绷了一下,喉咙都紧了紧,互相看了一眼,我就转身去叫醒胖子他们。潘子站起来拿起枪,就顺着脚印走到了沼泽的边上,蹲了下去,往水里照去。

胖子叫不醒,闷油瓶一碰就睁开了眼睛,也不知道是不是在睡觉,我把情况一说,他就皱起了眉头。

我们两个走到潘子身边,水下混浊不堪,什么也照不清楚,潘子又把那几个泥脚印照给胖子和闷油瓶看,说:“妈的,好像在我们睡觉的时候,有东西爬上来过了,看来以后打死也不能睡着了。”

照了一下脚印,闷油瓶的脸色就变了,他接过矿灯,快速的扫了一下尸体的四周,就挡住我们不让我们再走进尸体。

“怎么了?”我问道。

“只有一排脚印,那东西还没走。”他轻声道。

\chapter{蛇的阴谋}

我们刚才根本没有注意有几排痕迹,听闷油瓶一说,探头往脚印处一看,果然如此,这下我们就更加戒备起来。潘子立即端起了自己的短枪,瞄准了阿宁的尸体。

我们后退了几步,另一边的闷油瓶举着矿灯照着尸体,一边示意我立即去把胖子弄醒。

之前经历了一场生死搏斗,之后又遇到了阿宁突然死亡的变故,我的神经早已经承受不住了。现在没消停几分钟神经又绷紧了,让我感觉到十分郁闷,不过我也没有害怕,而是退后到胖子身边,先从胖子身上摸出了匕首,然后拍了他几巴掌。

可胖子睡得太死了,我拍了他几下,他只是眉头稍微动了一下,就是醒不过来。而我一下打下去,却感觉到他脸上全是汗。

我就感觉有点不对劲,怎么有人会睡成这样,难道是生病了?然而摸胖子的额头却感觉不到高温,我心说难道在做梦?正想用力去掐他,忽然我就看见,在胖子躺的地方的边上,竟然也有那种细小的泥印子。而且比阿宁身边的更加多和凌乱。

我心说不好,赶紧站了起来退后,叫唤了一下潘子。

“怎么了?”潘子回头,我指着那泥痕迹的地方,让他看,“这里也有!”

潘子一边瞄着阿宁的尸体,一边退到我身边,低头一看,就骂了一声娘,并把枪头移了过来。一边的闷油瓶回头也看到了,退了过来。

三个人看了看尸体,又看了看胖子。我心说这情形就复杂了,尸体还好办,也容不得我多考虑什么。潘子看了一眼闷油瓶,两个人就做了一个手势,显然是交换了什么意见。潘子举起枪退到脚下岩石的边缘,远离了尸体和胖子,这样可以同时监视两个方向。而闷油瓶把灯递给我,让我照着胖子,同时把我手里的匕首拿了过去,猫腰以一种很吃力的姿势走到胖子身边。

这是一种半蹲的姿势,双脚弯曲,人俯下身子,但是却不完全蹲下,这样可以在发生变故的时候保持最大的灵活。他靠近胖子,头也不回就向我做了一个手势,让我把灯光移动一下,照向胖子身边的脚印处。

气氛真糟糕,我心里暗骂了一声,心说这种事情什么时候才能到头?我把灯光移过去,就在那一瞬间,忽然有两三个不明物体以飞快的速度,从胖子的肩膀下冲了出来,一下子就掠过了灯光能照到的范围。

那速度太快了,只是一闪我眼睛就花了。但是我的手还是条件反射一般直接向着那几个东西冲出来的方向划了过去。可惜什么都没照到,只听到一连串不知道是什么东西跳进沼泽的声音。同时阿宁的尸体那边也突然有了动静,同样的一连串入水声,好像是在田埂边惊动了很多青蛙的那种感觉。

闷油瓶反应惊人,但是显然对于这么快的速度,他也没辙。他只是飞速转身,连第一步都没追出就放弃了。他忙挥手让我过去,去照水里。

我冲过去举起矿灯朝水中照去,一下就看到水中的涟漪和几条水痕迅速的远去,潜入沼泽里。

“是什么东西?水老鼠?”我问道,第一感觉就是这个。以前九十年代城市建设还没这完善的时候,见过不少这种老鼠。

闷油瓶却摇头,脸色阴沉:“是蛇!是那种鸡冠蛇。”

我咋舌,看着地上刚刚留下的一连串印记,忽然意识到没错,那就是蛇形的痕迹,难怪有点像脚印却又不是。心里顿时冲起了不祥的念头,传说这种蛇报复性极强,而且行事诡异,现在果然找上门来了。

我这时候发现胖子还是没醒,不由心里咯噔了一声,心说难道胖子已经被咬了?

我立刻过去看胖子,因为不知道是不是所有的蛇都走了,所以我小心翼翼的靠过去,先推了他一下。没想到这一推他就醒了,而且一下就坐了起来,脸色苍白,但是人还是迷迷糊糊的。他看着我们几个,又看了看天,有点莫名其妙。看我们如临大敌似的看着他,隔了半天才道:“你们他娘的干吗?胖爷我卖艺不卖身的,看我也没用。”

看他这样子应该是没事,我们松了口气。而我还是不放心,让胖子转过来,给他检查了一下,确实没有被咬。胖子看我让他脱衣服,更觉得莫名其妙,问我怎么回事,我就把刚才的事情说了。

胖子将信将疑,我们也没空和他解释了,又起身走到阿宁尸体的边上。我照了一下附近的沼泽,完全是黑色的,什么也看不见,尸体边的石头上全是刚才那些蛇离开的痕迹。

“真他娘的邪门,难道这睡袋是这些蛇打开的?”潘子轻声自言自语了一句,一边用枪拍了拍尸体的上下,看还有没有蛇在里面。

没有蛇窜出来,但是我感觉到非常不安,一种梦魇一样的恐慌在我的心底蔓延开来。我们睡觉的时候,有几条鸡冠蛇从沼泽中爬了上来,爬进了胖子和阿宁的身下,还不知道怎么样打开阿宁的睡袋。这实在太诡异了,它们到底想干什么?我看着漆黑一片的沼泽,总感觉,肯定要有什么不祥的事情发生了。

其他人都有这种感觉。闷油瓶蹲了下来,检查了一下阿宁的尸体,也没有发现什么异样,做了个手势,让我们都把矿灯打开,他要仔细看看四周水下的情况。

我们照闷油瓶说的办,一边的胖子也来帮忙。我们打开矿灯分四个方向,就开始扫射水里,才扫了没几下,忽然身后的胖子惊呼了一声。

我们以为蛇又出现了,马上转身,顺着他的灯光看去,就看到我们面前的沼泽中大概二十几米处,竟然有一个人影,好像是从沼泽的淤泥里钻出来的。

一只矿灯的光芒无法照清楚,立即所有的灯都汇聚了过去,只见一个浑身污泥的人,站在齐腰深的水里,犹如一个水鬼直勾勾的看着我们。

“狗日的,这是什么东西?”胖子喊道。

闷油瓶仔细一看,惊叫了一声:“天哪,是陈文锦!”说着一下冲入了沼泽,向那个人蹚去。

◆ 第八卷 蛇沼鬼城(下) ◆

\chapter{追击}

那一刹那,我也不知道他怎么能肯定那人就是文锦,我看过去那人的脸上全是淤泥,连是男是女的都分不清楚,但是这时候也没有时间过多的考虑什么,潘子叫了一声去帮忙!几个人一下全跟在闷油瓶后面冲下了水去。

冲下去没几步就是淤泥,沼泽的底下有一层水草,我没有穿鞋子,那油腻淤泥和水草刮脚的感觉好比是无数的头发缠绕在脚上,实在令人头皮发麻,几步扑腾到水深处,我们撒开膀子游了起来。

闷油瓶游的飞快,一转眼就冲到了那个人的附近,那地方似乎水位不高,他挣扎着从水里站起。随即潘子也爬了上去,接着是我和胖子。我的脚再次碰到水底,就发现那地方是个浅滩,感觉不出水下是什么情况,好像是一些突出于沼泽淤泥的巨大石头。

这时候离那个人只有六七米,我近距离看着那个人,心突突直跳,异常的紧张。

文锦算是一个关键人物,一直一来她好像都是存在在传说和照片里的一个概念,如今出现在我面前,也不知道是不是真的是她。然而这里只有胖子拿着矿灯,他刚站定没缓过来,灯光晃来晃去,我根本看不清楚前面的情况。

闷油瓶已经冲了过去,显的格外的急切,一点也不像他平时的作风,我看着他几乎能够到那人了,就在这个时候,那人忽然就一个转身缩进了水里,向一边的沼泽深处逃了。

我们一下都急了,纷纷大叫,可是那人游的极快,扑腾了几下,就进入了沼泽之后的黑暗里,一下竟然就没影了。闷油瓶向前猛的一冲想拉住,但是还是慢了一拍。

这看着只有一只手的距离,但是沼泽之中人的行动十分的不便,有时候明明感觉能碰到的东西,就是碰不到。

不过闷油瓶到底不是省油的灯,一看一抓落空,立即就一个纵身也跳进了水里,顺着那人在水面上还没有平复的波纹就追了过去,一下也进入了黑暗里。

我一看这怎么行,拔脚也想跟过去,但是一下就给前面的潘子给扯住了,水底高低错落,我被一扯就摔倒,喝了好几口水,站起来潘子立即对我道:“别追了,我们追不上了。”

我呛了几声之后冷静了下来,站稳了看去,只见这后面的沼泽一片漆黑,我们慢了半拍,进去之后必然是什么也看不到,根本无从追起,在很多时候,慢了半拍就等于失去了所有的机会。现在只有希望闷油瓶能追到她。

我们筋疲力尽,气喘嘘嘘又面面相觑,胖子就奇怪的问。“我操,怎么跑了,你们不是认得吗?难道被我们吓着了?”

我想起那人的样子,心说不知道谁吓谁,潘子问我道:“那人真的是文锦?”

我哪里看的清楚,摇头说不知道,那种情况下,也不知道闷油瓶是怎么判断的,刚才从我们看到那个人到他叫起来也有只一瞬间,他的眼睛也太快了。不过,说起来,在这种地方应该没有其它人了,出现一个人,很容易就让人想到是文锦,可是如果真是她她又为什么要跑呢?不是她引我们到这里来的吗?

“现在怎么办?”胖子就问我们道,“那小哥连矿灯也没拿,在那从林里几乎是绝对黑暗,他这么追过去会不会出事情?要不咱们回去拿装备进去支援?”

我心说那真是谁也说不准了,一边的潘子就道:“应该不会,那小哥不是我们,我相信他有分寸的,况且我们现在进去也不见得有帮助,到时候还要他来救我们。”

我想起刚才闷油瓶朝那人冲去的样子,那样子不像有分寸的样子,说起来,我总觉得进入到这个雨林之后,闷油瓶好像发生了一些变化,但是我又实在说不出到底哪里有区别。

我们在那里等了一会儿,也不见闷油瓶回来,身上好不容易干了,这一来又全泡起了皱子,一路进来我们就几乎没干过,这时又感觉到浑身难受。

胖子就说我们不要在水里等了,还是到旱地上去,这里的水里有蛇,虽然在水中蛇不太会攻击人,但是那种蛇太诡异了,呆在这里还是会危险。

他不说我还真忘了那蛇的事情,我们下半身都在水里,水都是黑的,完全看不到水下的情况,听到这个还是毛毛的,于是便应声,转身想朝出发地游过去。

上了岸,胖子抖着自己的胸部,一边搓掉上面的泥,一边就去看刚才我们背包四周那些蛇的印迹,我坐到无烟炉边上,稍微缓了点儿过来,此时脑子里有点乱了起来,一方面有点担心闷油瓶,他就这么追进沼泽,想想真是乱来,也不知道能不能出来,另一方面,这一系列的事情让我很不安。

阿宁的死其实是一个开始,但是当时更多的是震惊,现在想想,野鸡脖子在我们睡觉的时候偷偷爬上来干嘛呢,几乎就是在同时,沼泽里还出现了一个人,还没有进沼泽就一下子冒出这么多的事情出来,这是在是不吉利,这地方还没进去,就给人一种极度的危险感,甚至这种感觉,和我以前遇到危险时候的感觉还不同,我总感觉这一次,可能要出大事。

这也可能和闷油瓶的反常有关系,虽然我不愿意这么想,但是不知道为什么,这一次在闷油瓶的身边,我没有以前那种安定的感觉,反而更加的觉得心神不宁。

这时候再回想起之前下决定来这里时候的情形,真是后悔的要命了。

潘子处理完了衣服就来提醒我,我也把衣服脱了去烤,一边我们就加大了火苗,能让闷油瓶回来的时候看到我们的位置,胖子口出恶言说这点儿孤火小火苗有点像招魂灯,别等下把沼泽里的孤魂野鬼招来,潘子张嘴就骂。

不过胖子说的也有道理,这确实有点像,我心里不舒服,就又打起矿灯,在石头上一字排开,这样看着也清楚一点,我拿着矿灯就走到阿宁的尸体边上,就想放在她的头边。可走过去一看,我忽然就意识到哪里有点不对。再一看,我脑子就嗡了一声。

阿宁的尸体竟然不见了,只留下了一个空空的睡袋。

\chapter{消失了}

我心说坏了,忙向四周查看,然而四处都不在,一下便慌了手脚,心说这是怎么回事情,这荒郊野外的,难道诈尸了不成,忙唤来胖子和潘子看。

两人一看也傻了,胖子就大骂了一声:“狗日的,谁干的?”都条件发射的往四周去找,这动作我们也不知道做了多少次,都是懵了。

然而四周一片寂静,即没有人,也没有听到任何野兽的声音。我立即就感觉到一股恐惧袭来,这西王母古城里必然没有其它人,这睡袋附近又没有野兽的脚印,我们都清楚不可能有什么搬动这具尸体,难道真的是诈尸了?

想起之前那个诡秘的梦,我不由喉咙干涩,心说难不成要噩梦成真。

胖子和潘子到底是见过大世面,此时没有慌乱,而是立即蹲了下来,翻找睡袋,想看看到底是什么情况。

睡袋一翻开,潘子又倒吸了一口冷气。

就看到睡袋里面,竟然全是蛇爬过的那种泥痕,睡袋下面也全部都是,痕迹之杂乱,显然这里爬过的蛇数量极其多。摸了一把,黏糊糊的,痕迹非常新,显然就是刚才留下的。

胖子脸色大变,就惊讶道:“我操,难道是那种蛇把尸体搬走了?”

潘子显然不信,“这不可能,蛇怎么能把这么重一具尸体带走?”但是他的脸色也变了,显然这里的痕迹表明胖子说的是对的。

我背脊发凉说不出话来,如果这是真的,这事情太邪门了,一直以来我对野鸡脖子都有一种特别的恐惧,一方面是因为它的毒性,另一方面则是关于这种蛇那些神乎其神的传说,很多很多的传说里,这种蛇的行为都是十分乖张的,让我印象最深的就是这种蛇的报复手段十分的诡异,但是它们竟然把阿宁的尸体搬走了,这是实在太匪夷所思了。

“一条当然不行,可是你不看看现在有多少条,大象都抬的走。”胖子翻开整个睡袋,只见下面全是蛇印,睡袋一边到水中的区域更是多的变成一片烂泥,刚才光线问题才没有注意。

“可这些蛇要尸体干什么?”潘子又道,看着胖子。确实,阿宁的尸体显然不能当食物,蛇也不是有爪子的动物,要打开睡袋,运走一具尸体,非常困难。蛇又不是蚂蚁,要尸体来干嘛?

“那你他娘的就要问蛇去了。”胖子顿了顿就道:“不过蛇这种东西很功利的,总不会是为了好玩,肯定有原因。没想到这娘们死了也不得安稳,倒是合她的性格。”

我想着,心情就压抑了起来,刚才那这一系列的事情,每一件都没头没尾,而且全部都让人摸不着头绪,这感觉实在太糟了,想着有点失控,心说怎么可以被蛇欺负,想着就拿起矿灯,对他们说:“我们一来一回也就几分钟,这尸体肯定还在周围,我们去找一下。”

还没站起来,就给潘子拉住了:“找个鸡巴,几百条蛇,你找死。”

“可是!她总不能葬在蛇窝里。”

胖子把我的矿灯抢了回来,潘子就拍了拍我的肩膀,“小三爷,你得想开,人活着才是人,死了就是个东西,臭皮囊而已,我们已经不可能把这女人带回去了,这也算是她自己选择的归宿,范不着为具尸体拼命。”

胖子也道:“就是,死了就是死了,死在哪里不是死,不过改日要是胖爷我也挂了,你们就把我烧了,别给这些蛇绕去,鬼知道它们要尸体干嘛。”

我听了一下也泄了气了,坐倒在地上,抓了抓头皮,心里很不舒服。

胖子看着那些痕迹,又道:“这里的蛇果然邪门,你想搬一具尸体要多少蛇?少说也要百来条吧,你想就光这里就有这么多了,这整个林子里到底会有多少这种蛇?咱们在这里呆着,恐怕不太明智,要是它们再回来,咱们三个恐怕也抗了不了几分钟,到时候挂了碰上阿宁,又要被那臭娘们笑话了。”

“其实我感觉不用那么害怕,刚才我们睡着的时候都没咬我们。”潘子道:“老子在越南也碰到过不少蛇,被咬过也有两三次了,对蛇也算熟悉,一般蛇不太会主动攻击人的,阿宁当时算是个意外,可能是阿宁弄瀑布的水,惊扰到那条蛇了。”

这一听就知道是安慰的话,心说谁信,看潘子的脸色就知道他自己都不信。普通蛇还好说,那种蛇看着就邪门,不是善类。

我将矿灯放到原本想放的位置上,看着空空的睡袋,心中非常的酸楚,胖子却把我的几个矿灯全部调整了方向,照着四周的水面,说是要警惕一下。

胖子行为让我立即又担心起闷油瓶,这家伙不会出事情吧,如果是在古墓之内,我必然不会担心,因为那是他轻车熟路的地方,但是像胖子说的,蛇这种东西是不讲道理的,咬一口就死,你拿他没辙。

我们又合计了一下,也不知道该怎么办了,只好继续等闷油瓶,这晚上必然是不敢睡了,三个人背靠背在一起,看着四个方向挨夜。

此时其实时间也不早了,只过了一会儿天就亮了,随着晨曦的放光,持续一个晚上的压抑减轻了不少,我们也少许放松了下来,不过闷油瓶却没有回来。

我们重新审视沼泽,没有晚上那么恐怖,不过雨停了,没有雨声,四周只剩下流水的声音,还是安静的异样。远处的雨林之中漆黑一片,天亮不天亮似乎和雨林深处的世界一点关系也没有。

见闷油瓶没有消息,我又开始焦虑起来,我很少有这种随时会失去一个人的感觉,现在却感觉这里的人随时有可能会死,这大概是因为阿宁的死亡,打破了我的一些先入为主的感觉。

潘子和胖子虽然也有点担心,但比我好的多,胖子说起来,最差也不过就是挂了,让我无言。

我们吃了点东西,潘子淌水回到峡谷口,减了些树枝回来晒干,烧了个篝火做了个火炭堆。

我问他想干嘛,他说我们已经过了峡谷了,基本的情况都知道了,时间也过了几天了,三叔他们如果没有意外,应该马上就会到达峡谷口,这里昨天虽然还有小雨,但是外面的戈壁已经给太阳晒了好几天了,现在地表的地上河还不知道在不在,他要在这里做一个信号烟,一方面标示我们的位置,让三叔知道我们已经进去了和我们进去的路线,二来,也可以警告三叔这里的情况,让他们提高警惕。

潘子说完就从包里掏出一种黄色的类似于药丸的东西,丢入了炭堆中,很快一股浓烟就升了起来,他告诉我这是海难时候求救的信号烟,他这种是托他一个还在部队的战友弄来的伞兵用军货,就这么几个球能发烟三四个小时。

我道能不能告诉三叔这峡谷里有毒蛇?

潘子就摇头,说不同颜色的烟代表着不同的意思,但是都是简单的意思,这黄色代表的是前路有危险,要小心前进,更复杂的交流,要等到三叔看到了烟,给了我们回音后他才能想办法传达过去,三叔他们所处的地势比我们高,应该很容易就看到,我们要时刻注意峡谷的出口方向,或者四壁上有没有信号烟响应。

这倒是一个非常有效的远距离的沟通方法,我看着烟升上半空,心里忽然有了一丝安全感,如果三叔到了和我们会和了,那事情就好办多了,他们人强马壮,我想最起码晚上能睡个囫囵觉。

潘子没隔两个小时添一次烟球,第一次烟球熄灭后,没有任何的回音,闷油瓶也没有回来,我们也没有在意,一直等到下午,第二次烟球烧了大概一半的时候,忽然胖子就叫了起来:“有了,有了!有回音!”

我正在无聊的45度角看天,立即就跳了起来,和潘子一起朝悬崖上看去,一开始还没找到。胖子大叫:“那边那边!”

我转了几个圈,才看到了有一股烟从远处升了起来,冉冉飘上天空,烟竟然是红色,乍一看,犹如一条巨大的鸡冠蛇,从很远处的树冠底下冒了出来。

我欢呼了一声,条件反射就想笑,然而笑容才到一半,忽然就凝固了,几乎是欢呼的同时,我立即就发现不对劲。

因为那烟升起的地方,根本就不是峡谷外,而是在我们所处的盆地的中央,这片沼泽的深处。

\chapter{信号烟}

三叔他们一直潜伏在阿宁的队伍之后,按照潘子的说法,应该是有一天到两天的路程差距,此时按照计划,他们的位置应该是在这片盆地的外延,即使发现了这片绿洲,他们也不会立即进入,而必须等待潘子给他们的信号。

然而,让我们目瞪口呆的是,三叔回应我们的信号烟,竟然是从相反的方向,从我们身后,沼泽的中央升了起来,这就是说他们现在竟然已经身在沼泽之中了。

潘子简直不敢相信自己的眼睛:“我操,这是怎么回事情?他们怎么在里面?”

我怕是误会,马上拿起望远镜去看,一看正确无误,那烟绝对不会是起火产生的,因为烟的颜色红的不正常。

“大潘,看样子你家三爷比你动作快多了。”胖子喃喃道。

“不可能啊,难道三爷他们从其它的峡谷先进去了?但是,按照计划不是这么来的,他们应该等我的信号啊,而且他娘的他们也太快了……”潘子想不通。

“会不会不是你们三爷的队伍,是那小哥放的?”胖子道。

“昨天晚上他什么都没带,不可能是他。”潘子道:“就我带了烟球,都在这儿呢。”

“那就奇怪了,看来你三爷和你的交接有错误。”

“这烟是什么意思?”我忽然想起颜色可以代表信息,就问道。

潘子从我手里接过望远镜,往烟的方向看去看,看着想了想,他忽然脸色就变了,凛然道:“不好,他们出事了。”

“出事?”我看潘子脸色有变,但是又不知道他说的是什么意思,就让他说详细。

他道烟的颜色有简单的意思,黄色的烟代表前路有危险,要小心前进,橙色的烟表示停止前进,等待确认,而红色烟则更加的严重,表示绝对不能靠近,一般是在极度危险的情况,警告后来者发出的。一般的活动中,几乎不会用到红色的烟。

不过他也有点犹豫,因为毕竟他们不是搞考察的,这种东西也是临时想出来的法子,那烟的用法他有没有记错尚且不说,也许对方记错了也不一定。

不过这毕竟不是好消息,我问潘子能不能再发个烟,问问到底怎么回事情。

潘子摇头,顿了顿显然有点急起来,就对我道:“不行,小三爷你留在这里,我得过去看看,三爷别出什么事情。”

我心里也担心着三叔,不过知道轻重,赶紧抓住他,心说这怎么行,那小哥已经没回来了,你再去我们这里不是只剩下两个人了。况且你一个人进去也实在太危险了。还是等闷油瓶回来再说。

潘子摇头道:“三爷他们有三十多人,人强马壮,一般情况下不会发出红烟,那边肯定出了状况,他娘的那黑瞎子果然还是太嫩,小三爷你放心,这种林子我在越南的时候钻的多了,我能穿过去,你们在这里等那小哥回来再做打算。”说着就收拾自己的装备。我一看拦不住他,就急向胖子打眼色。

没想到胖子也立即收拾起了装备,我一下就头大了,心说怎么胖子也怎么关心我三叔了。刚想说话,胖子就对我道:“你别向我抛眉眼,不仅大潘得去,我告诉你,这次咱们也得进去了。咱们身上的装备根本不够过戈壁的,所以必须得和你们三叔汇合,至少得拿到他们的东西,否则,出的了峡谷,咱们也会渴死在路上。”

我一想,心哎呀一下,心说他娘的对啊,顿时就有点不知所措。胖子又道:“大潘一个人进去也不是不可以,但是万一他有一个什么意外,我们两个再进去就麻烦了,不如现在三个人一起进去,齐进齐退,成功的几率也好大一点。否则我们留在这里,也只是等死。”

“可是那小哥,怎么办?”我问道。“如果我们走了,他回来不就找不到我们了,要不我在这里等你们。”

“那你不是找死,就你那小体格还不得给那些蛇轮了,得了,进去之后抗东西出来也需要人手,我和大潘肯定不够,我们留下记号,给他指明方向,到时候最多再起个烟给他当信号,不过,”胖子看了丛林一眼,“我想那小哥恐怕不会回来了。”

这事情虽然非常的糟糕,但是却明朗化了,我虽然觉得很不妥当,但是也知道胖子说的对。想了想,只好点头同意。

进峡谷的时候是五个人,现在只剩下了三个,一个死了一个跑了,原本的物资显然要重新分配,不过胖子说闷油瓶的那一份就不要带走了,用防水布包好之后,用大石头压住,接着用麦克笔在防水布的里层写了我们的去向。然后在那包裹边上,把无烟炉调到最暗,这样能烧三天,如果闷油瓶晚上回来不至于找不到。

搞完之后我们身上的物资反而减轻了不少,潘子说信号烟最多只能烧三个消失,这一次进去,我们不能休息,所以一次要尽量轻装,反正我们如果要回来,必然也会经过这里,所以能不带的东西就不带。

之后我们过了一遍装备,将防毒面具,洛阳铲等一些重的东西留下了。接着潘子又将我背的一些比较沉的东西换到他的背包里,他的行军负重是专业的,背的多一点不影响速度,我就不行了,他说丛林行军非常消耗体力,这样主要是要保证我能撑到目的地。

他这么说我很没面子,我很想反驳说这半年我也练出了点肌肉来了,不过他根本不给我机会,说完就只顾自己收拾,显然心思已经不在我这里。

一下就整理妥当,刚要动身,忽然胖子又抓住了我们,让我们抬头看远处的烟。

\chapter{无声的山谷}

我们抬头一看,原来那远处的信号烟已经日渐稀薄,不知道是那边发生了什么变故,还是烟球放的不够。看这样子,这烟必然坚持不到我们到达。

在丛林中,如果没有信号烟的指引,我们在没有导航的前提下是肯定无法到达那个地方的。我们问潘子有何办法?潘子就爬上树冠,以信号烟的位置为中轴,用远处的盆地边缘的峭壁上怪石为参照物在指北针上做了标记,道只要在往这两块峭壁怪石的之间重点的位置走,必然能经过信号烟的燃烧点。不过,这丛林密集,就算误差十来米都有可能错过,所以咱们得在烟熄灭前尽量靠近。

这就不能再耽搁了,我们立即整顿装备,和从潘子那里对了指北针,淌水走入沼泽往信号烟的方向进发。

在白天通过沼泽边缘那一片水域非常容易,因为雨水汇聚的沼泽水水位很高而且清澈,我们可以寻着水底可以落脚的石头前进,没有落脚的地方就游泳,半只烟的功夫我们就通了过去,来到沼泽真正的边缘。

那是一片比较稀疏的雨林带,这里明显地势较高,很多的连接在一起的“树群”突出了水面,好像一些巨大的岛屿,可以看到有大量的乱石混在这片区域下的淤泥里,看上去似乎水位不深。

但是往里走就会发现,树木在这片区域里非常迅速的密集,大概只有两百米后,树冠就密集的偷不过天光了。树根盘根接错在一起,我之前其实有一个想法,就是做一条独木舟,这样就不用这么小心翼翼的淌水前进,但是一看这种水下环境,就知道独木舟在这里也是寸步难行,非的人自己走不可。

深入林中,光线就非常的暗淡,很快四周就都是骇人的树根,树根上绕满了藤蔓,藤蔓上又覆盖着绿色的青苔,潮气逼人,那种绕法,铺天盖地,大部分地方我们全部匍匐下来才能勉强通过,让人感觉是进入了一个巨大的长满树的山洞之中。

潘子砍着拦路的藤蔓,因为几乎所有的树之间都有大量的树根和藤蔓相连,所以我们反而几乎不用淌水,架空走在大腿粗的藤蔓上非常的稳当。

然而让我们奇怪的是,这么密集的树林里,却出奇的安静,除了我们行进的声音,听不到其它的动静,静的有点让人不舒服。

“西王母的地盘果然邪门,”胖子边走就道:“他娘的连个鸟叫都没有。”

“何止,他娘的好像这里什么都没有。”我心里道,静的实在不正常,让我有一种错觉:我们可能是这片雨林里,除了这些树外唯一的生物。

“也许这里的蛇太多了,鸟全给吃光了。”潘子道。

“不可能,那这些蛇现在吃什么?”

想起那种蛇,几个人又是一阵紧张,不过一路过来,却丝毫不见任何蛇的踪影,这让我们有点意外。

绷紧神经继续前进,不久我们便看到前方出现了一些裹在树木中,突出水面的古建筑遗迹,因为时代过于久远,这些残圭断璧都已经成为不同形状的石块,大量藤蔓和青苔在这些建筑的缝隙里生根,然后包裹全身,混在在雨林中很难辨别,非到跟前了才能发现。

这些建筑必然在当时属于建筑顶部的部件了,所以还能突出于水面,因为看不到水下的部分,不知道整体的形状如何,但是看顶部,都是一些简单的塔楼的样子。数量很多,高低错落,大小不一,看上去像埋和尚的那种塔林。

一路过来基本没有见到西王母的遗存,现在终于看到了,倒是松了口气,之前我还有一个臆想就是我们几个别走错了,毕竟峡谷口上没有牌子写“西王母城往里2公里,移动信号已经覆盖”,呆会儿进去发现里面啥也没有那玩笑开大了。

我们没有时间停下来查看这些遗址,很快深入其中,不过虽然主观上不想去研究,但是前进的路线蜿蜒曲折,总有绕到这些遗址之上的时候,我就发现,这些遗迹虽然经历千年,却坚实无比,十分的坚固,而另人奇怪的是所有的这种“塔”上,都有很多的方孔,显然是当时建造时候打磨而成的。

方孔说大不大,说小不小,大约人是通不过,但是比人小的东西都不成问题。

胖子看着奇怪,路过的时候就下意思的用矿灯顺手对内观瞧,然而却什么都看不到,只听得下面有水声。不知道是通往何处。

潘子没空理会这些,就催促快走,胖子知道急人所急,也只好草草看一下就跟了上来。

这个山谷的绝对面积并不大,越往里走,水下的淤泥明显的减少,水下的各种的古迹遗骸就露了出来,非常的清晰,形成了一副非常诡异但是壮观的景象,水深大概只有两三米,无数的残圭断璧和水下的繁盛的树根混在一起,让我感觉只隔着一层薄薄的水面,就恍如隔世一般。

直到这时候我才有进入到一座古城的感觉,看着这些残迹,依稀可以想象当年这里繁盛的样子,然而时过境迁,就算是女神的城市,也终于尘归尘,土归土了。

感慨间,忽然脚下水流的速度发生了变化,前面似乎有向下的陡坡。我们小心起来,这里树木太多,滑倒踩空就是重伤。

再走几步绕过一棵大树,胖子就惊呼了一声,我们看到左前方的密林中突然出现了一张巨大的怪脸,离我们不到十米,足有卡车头大小,脸上绿斑斑斓,大目高鼻,和我们在峡谷口看到的人面怪鸟石窟一模一样。那是被包在青苔和藤本植物中一座巨大石雕。

胖子打开矿灯照射过去,石雕的身体部分沉入了沼泽中,只剩下了头颅,与密林融为了一体,在水中鸟身的呈现一种非常奇怪的蹲势,好像要突然展翅而起的感觉,犹如猫科动物攻击前的蓄势。还可以看到石雕的下方的水下,有一些形状奇怪的黑影,不知道沉了些什么。

我们面面相觑,想起之前的想法,如果峡谷外的人面鸟雕像,是告诉外来者已经进入了西王母国的领地了,那么,这里出现了巨大的人面鸟石雕,又代表着什么呢?难道这是一种更加严重的警告。

我下意识的看了看雕像之后的树海,心说该不会在这石雕之后的区域里,有什么巨大的危险正在等待我们这些不速之客。

\chapter{石像}

思索间我们已经来到了雕像的下方,水流越来越急,我们看到在树根下的沼泽水流絮乱,下面不知道是什么情况,潘子让我们小心,说可能淤泥下的遗迹中有什么空隙通往地下。好比下水井口。

胖子根本就没听进去,所有的注意力都被一边的石雕所吸引,矿灯在上面滑来滑去。

在焦距灯光下,我看到了更多的细节,石雕似乎是整块巨石雕刻而成的,很多地方已经残缺开裂,因为大量覆盖着的青苔,使得其看上去更加的诡异丑陋,这么近看,反倒感觉不出雕刻的整体是什么。

看了几眼,胖子就把灯光朝水下照去,石像几乎是被包裹在两颗巨大的龙脑香树中间,沼泽之内的部分完全被树根残绕住了,还能看到,在水底比较深的地方,同时被包裹住的还有一些奇怪的影子,形状很不规则,缩在树根里面。不知道是不是石雕上的一部分。

胖子看了半天,都无法看清楚那是什么,而且我还发现奇怪,为什么四周的树根都能被矿灯照的这么清楚,那东西怎么照却都是个影子,再照我们才恍然大悟,原来那不是什么黑影,而是一个个空洞。

而且看树根上附作物的飘动方向,看样子这里的水正在往这个黑洞里流下去。果然如潘子所说,这雕像下面有空隙通往地下。

本来以为能看到什么稀奇古怪的东西,现在不由大失所望,潘子于是继续催促,我们只有继续出发。

胖子不是很甘心,边划动矿灯往回照,边自言自语:“这水流到哪儿去?难道这古城下面是空的?”

我道不是,这可能是以前城市下水工程的一部分,某些地下水渠井道还能使用,就会有这样的现象。

胖子道:“那这下水渠道通到什么地方去呢?这儿的可是低洼地带,再低就没有可以流去的地方了。”

我想了想,一般城市的排水系统,出口都是附近的大江大河,最后冲进海里。像这种西域古城,附近没有大型的地上湖泊或者江河,但是肯定会有暗河经过四周,那么按照道理这种排水系统应该是通往附近的暗河,不过,事实上西域雨量极少,这里的水格外珍贵,怎么也不可能需要“排水”这么奢侈的系统,一般楼兰和附近遗址的考察,所谓的排水都是地上排水,然后引入井中,这里出现地下排水实在有点奇怪。

所以我感觉,这里的排水系统要么是引入底下的暗河,要么就是在古城的地下四处都有蓄水的井或者水池,这些水都在涌向那个些深井之内,被储藏了起来,而这些井可能和吐鲁番的坎儿井一样,在地下井井相连,一井满了自动把水往另一口井送,直到所有的井口都蓄满水为止。

这座雕像下面的空洞,也许就是当时的井口,这倒也是相当有可能。刚才我们看到的石塔,胖子说下面有水声,可能也是地下的引水地道的声音。

“这他娘的就是深挖洞,广积粮,看来毛主席的思想也是来自古人嘛,咱们的西王母真不含糊。”胖子道。

潘子道:“但是这里雨量这么少,几年才下一场大雨,这种这么大的工程可能要画上几百年的时间,他娘的管用吗?”

“如果从短时间来看可能得不偿失,不过西域国家,有水便可以称王,楼兰号称西域大国也才几千号士兵,这里地形奇异,如果有大量屯水,就算国家规模不大也可以固守,你看这里的情况,这片绿洲肯定就是因为这样而形成的,树又可以固水,水有可以养树,当时的西王母显然是一个深谋远虑的人。”

本来西王母古城的地域位置就极其的低洼,这样的设置甚至可以引入有限的戈壁地下水,不过,如果我想的是对的,那我们到这里来已经有几天了,这么长的时间,这些井道还在排水,说明那些井道到现在还没有满,这底下的井和通衢到底有深?

潘子想了想,点头道:“有道理,不过凡事有利必有弊,如果打仗起来,有人潜入城里投病疫毒药,那不是全城的人都倒霉?”

我道:“井口必然不会很多,我看可能西王母宫和权臣家里才可能会有井口,百姓可能就是用刚才看到的那种公用井口,这些地方肯定都是把守森严,咱们也看过古装片的,投毒这种事情看起来容易,做起来还是有一定难度的,毕竟井口深,再毒的毒药一稀释,恐怕连大肠杆菌都毒不死。”

说到这里胖子愣了一下,不知道想到了什么,想了想忽然道:“我靠,这么说,这些井口必然都是通的,那么咱们从井口可以通道西王母宫里去。”

我道:“确实如此,不过我们又不是鱼,而且下面的井道必然纵横交错,犹如迷宫,就算给你最完善的潜水设备,你也不一定能活着出来。说不定,那下面的井口通衢只有碗口粗细,那更麻烦。”

胖子骂道:“你又讽刺我吧?胖爷我胖点碍着你什么事了。”

我道:“我靠,我这哪里是讽刺你。我自己都没瘦到碗口粗细。”

“我觉得不会。”胖子道:“我们以前支边的时候学基础课,挖田埂引水渠,宽度也要根据水量定宽度,如果是这么大的雨,碗口粗细的通衢够用吗?小吴你不是能算这些吗?”

我学建筑的时候,有讲过这方面的问题,不过现在临时要用,已经完全不行了,琢磨了几秒只好放弃。对他道:“现在想不起来,还是等我休息的时候仔细想想。”

潘子说:“得,小三爷,你们两别琢磨这些了,赶紧吧,算出来就算是地铁这么宽咱们也下不去,而且,现在咱们最重要的是尽快赶到三爷那里。”

我一想也是,立即点头,就收敛心神不在琢磨这些,就在这时候,忽听身后的林子里,忽然想起了一声树枝折断的声音,同时似乎有树冠抖动,树叶抖动声连绵不绝,不知道是什么东西在密集的灌木中移动了一下。

我们一路过来,林子里几乎什么声音也没有,一下子出现出现这种动静,把我们都吓了一跳,全部都停了下来,转头望回去。

树冠密集,除了那座巨大的人面鸟身石雕,什么也看不清楚,那声音随即也慢慢停歇了下来。树林很快就恢复了那种让人窒息的安静。

我们互相看了看,这种动静肯定不是小个的东西能发出来的。看样子这林子里并不是什么都没有。

潘子就把抢端了起来,示意我们准备武器,不要说话了,快点离开这里。我们点头,不敢再怠慢,凝起精神开始观察四周的动静起来就加快了脚步。

走了没两步,突然胖子就咦了一声,道:“等等!”

我们问他干嘛,他转回头去,指了指身后的人面鸟石像,问我们道:“刚才它的脸是朝哪儿的?”

我们朝石雕看去,就发现那石雕的脸不知道何时竟然转了过来,面无表情长满青苔的狰狞巨脸朝着我们。因为被树木遮挡了一半,犹如躲在树后偷窥的不明生物。

\chapter{石像的朝向}

一下我的头皮就麻了一下,心跳陡然加速,紧张起来。

潘子就咽了口唾沫,说:“我没注意……不过肯定不是这一面。”

胖子道:“他娘的,有鬼了,那难道它自己转过来了?还是咱们触动了什么机关了?”

我说不可能,刚才走近的时候我看的清清楚楚的,明显是石头的。而且是一整块的,不太可能有机关陷阱。

潘子盯着那石雕,道:“刚才没看仔细,也许这雕像是两面的。”

“两面你的头,刚才离开的时候我回头看了好几眼,石像的背面绝对没有这张脸。”胖子道:“而且,这张脸也有点不对劲。”

的确,和从正面比起来,这张石像的脸让人感觉很怪异,同样是面无表情,但是那脸上的表情就透着一股阴郁和怨毒。让人看了就心惊。

“他娘的,肯定是自己转过来的,这东西难道是活的。”潘子道。“咱们碰上石头精了。”

我道:“我们走的不是直线,也许是角度的问题,不要吓唬自己。”

胖子骂道:“狗屁角度,这肯定有问题,你这么琢磨是自欺欺人。”

我有点尴尬,胖子道:“要不要回去看看?”

潘子摇头,忽然就掏出了枪,上膛,对准了那巨脸,就想开枪,我们给这举动吓了一跳,差点来不及反应,胖子立即把枪抬了一下,呯一声子弹呼啸而过,打到石像边的龙脑香木上,打的整棵树都整了一下,我们立即就看着那石像,心说这也太横了,要是真一活的,你不直接就把东西给招惹了。

胖子已经做好的战斗的准备,手都摸到了腰上。几个人看着那石雕,随时准备它有什么异动。

然而看着那雕像,却一点反应也没有,那诡异的脸还是冷冷的面无表情,丝毫没有什么改变,似乎只是普通的石像。等了半响,潘子就把枪退弹,对我们道:“你看吧,没事,是石头的,可能真是看错了,这里的路七拐八拐的,咱们快走,别磨蹭了。”

我也送了口气,说真是自己吓唬自己,在这种地方真是让我神经紧张。连正确判断的能力都没了。

胖子皱着眉头,还是不相信:“老子支边的时候,干过车床,眼睛毒的狠,这怎么可能看错。”

“车床是车床,这里是森林,参照物复杂,看错了不奇怪。”我道。

潘子就催促快走,胖子却死命不肯,要过潘子的枪,放下自己的装备,就对我们道:“你们别动,我去看看,就两分钟。”说着就往雕像的方向走。

我们知道胖子的脾气,也没办法,只好让他去。我坐下休息,潘子骂了一声麻烦。

就看胖子把枪背到身上,小心翼翼的往回走,走到一半的距离,他忽然就停了下来,退了一步,不知道看到了什么。

潘子很不耐烦,大叫着问他:“你搞什么鬼?快点!”

话还没说完,胖子突然回头。转身狂奔,对我们大叫:“是活的!快跑!”同时就见远处人面怪鸟的“脸”,竟然起了变化,眼睛吊了起来,嘴角不可思议的上扬,从那种面无表情,变成了极度狰狞的笑。

\chapter{破裂}

我头皮一炸,心说还遇到鬼了真是,这东西还真是活的?

胖子已经冲到我们面前,并不停留,拉住我们就跑,大叫:“发什么呆啊!”

我们给胖子带出去好几步,此时还是没反应过来,回头去看,却看到更加离奇的场面,那石雕的脸,竟然碎了开来,五官挪位,好像是石头里面裹着什么东西,要从中出来。

“狗日的!”我大骂了一声,心说自己的预感果然没错,立即撒腿狂奔。

我们在大片的废墟里,下面是乱石和藤蔓,实在难以加速,只得顺着废墟的山势,哪里方便朝哪里跑,摔了好几下,膝盖都磕破了,一直跑到筋疲力尽,才回头去看,才发现自己并没有跑多远,不过那石雕还在原地,并没有追过来,这个距离已经无法看清。

狂奔的时候,体力已经把我们拉出了距离,胖子和潘子都跑出比我还远,还在往钱跑,我感觉叫住他们。他们冲回来就来拉我,我扯住他们,让他们先躲起来,然后看那远处的石雕。

发现石雕并没有追过来,他们颇感意外,我们喘着粗气,又看了一会儿,远处的石雕纹丝不动。

我们这才松下劲来,胖子喘的和风箱似的,吃力道:“怎么回事?小吴,它不动,这会不会是机关?”

“我们根本就没碰那东西,怎么可能是机关?而且机关也做不到那种程度。”

这绝对不可能是机关,整体的石雕雕刻,加上它被两颗巨树夹在中间,如果它要转动头部,那么会产生巨大的动静,那两颗树甚至可能会被扭断,所以就算真的有机关,它也不能转动,这一点是毋庸置疑的。但是,无论我怎么想,显然它转了过来了,这实在太诡异了。

我对西王母国里可能遇到的事情其实是有着心里准备的,但是这样的事情还是超出了我的想象。

这时候潘子从装备中拿出了望远镜,朝雕像的方向看了看,我忙问怎么样?到底是什么东西?却见潘子露出一个非常惊讶的表情,道:“我操?没了?”

“什么?”我立即抢过望远镜,朝那里看去,一看果然,那石雕的背部呈现在我们面前,然而,那张狰狞的脸孔竟然消失了。

我还没放下望远镜,就被胖子抢去了,我脑子一片混乱,难道我们刚才看到的是幻觉?不可能,我们三个人都吓的差点尿裤子,那这是怎么回事情?我们刚才看到的脸是怎么一回事?难道是鬼魂?

“他娘的,难道有人在玩我们?”胖子站了起来。

我们怕他莽撞,立即又把他拉坐下,这里石头不稳,胖子一下就一个趔趄滑了一下,我们又赶紧去拉他。无意间就看到,身后大概十几米外的巨石上,有一张巨大的人面浮雕,和刚才看到的如出一辙,同样面无表情的看着前方,犹如尸体的表情。

刚才跑的时候,一路狂奔并没有注意四周的遗迹,所以不知道是否这浮雕原本就在。

胖子和潘子看到,也立即觉得不妥,纷纷站定。胖子道:“我靠,这总不是活的。”

“不止一个!”潘子就道,指着一边,我们看去,就发现四周的巨石上,隔三岔五就有一片人面浮雕,有大有小。但是大部分都被藤蔓掩藏着,不仔细看看不分明,仔细一辨认,就发现规模惊人。几乎到处都是。我们趴的地方不到十米,就是巨大的人面,奇怪的是,这里的浮雕全部都是人面,而没有鸟身的图案。

胖子看到这么多呆滞的石眼看着他,不由一慌,就端起了潘子的枪,我立即按住,让他别轻举妄动。我已经感觉到四周有点不对劲了,这些好像不是浮雕。

可还没等我清晰的想明白,这些到底是什么东西,忽然,其中一块浮雕竟然裂了开来,接着我就看到了一副奇景,碎裂的石头,竟然全部都飘了起来。

我目瞪口呆的看着,心说难道我终于神经了?开始大白天也产生幻觉了?就听潘子大叫了一声:“他娘的,是蛾子!”

我顿时恍然大悟,仔细一看,果然,飘起来的石头都是一只只黑色的飞蛾,这些人面是这些蛾子排列成的,难怪会突然出现又突然消失。随即就看到四周的人面浮雕都开始扭曲开裂,大量的飞到空中,向四周散去。

这些飞蛾显然都是趴在这里的遗迹上,被我们惊扰之后,不知道为何排列出了人脸的样子。

很快天空中几乎布满了黑色的碎片,这些飞蛾也不知道有没有毒,我们都下意识的用衣服蒙住口鼻,不过,使用保护色的东西一般都是无毒的,看着飞蛾逐渐飞散,犹如漫天的黑色花瓣,颇有感觉。

胖子抓了几只说要看看仔细,这些蛾子不知道是什么品种,不过抓了几只没有抓住。我们的心逐渐放下,这也算是一场虚惊。不过,这倒也怪不得我们,这情形实在是骇人。

我们在原地呆着,一时间也不敢轻举妄动,飞蛾陆续飞走,只剩下了零星的一些,这时候,我们就看到,原来的遗迹发生了变化,在飞蛾刚才遮盖的地方,露出了大片的白色,仔细一看,就发现全是一团团的白花花的蛇蜕,被缠在植物的藤蔓中,看着好比什么动物的白色肠子。

胖子跳下去,看到藤蔓,挑起一条就骂了一声,大部分的蛇蜕已经腐烂的千疮百孔,极其恶心,大量的藤蔓从其中穿插缠绕,往四周看去,蛇蜕到处都是,遗迹的缝隙里,树根间隙,足有成百上千,刚才这些蛾子,全部都是停在蛇蜕上面,可能是被上面的腥味吸引,这里可能是这些蛇蜕皮时的藏匿地。

我们看着就浑身发凉,这片遗迹规模巨大,要多少蛇在这里生存,才能蜕皮成这样的规模?

胖子爬了上来,把他挑上来的蛇皮给我们看,蛇皮的头部部分膨胀,可以看到鸡冠的形状,确实就是那种毒蛇褪下来的皮,这以条蛇皮足有小腿粗细,比我们之前看到的蛇都要粗,看来这里的蛇的体形我们没法估计。

胖子显然觉得恶心,皱着眉头,连看也不要看。

蛇蜕是一种非常贵重的重要,一斤能卖到百元以上,这里的规模,起码有几吨的蛇兑,价值惊人,要是胖子知道估计就不会觉得怎么恶心。不过,我就是知道,也感觉到浑身的鸡皮疙瘩。

潘子摸了摸蛇皮,就道:“这皮还很坚韧,好像是刚退下不久,这里是它们褪皮的地方,蛇一般都在他们认为安全的地方蜕皮,如果在这里碰上以两条,它们会认为自己的地盘收到了最严重的侵犯,肯定袭击我们,我看此地不宜久留。”

我向后看看,要向往后走,必须走过这些蛇蜕的区域,那是极不愉快的事情,不过潘子的担心是正确的。这里的隐蔽处可能就有哪些毒蛇。

我们立即出发,急急的走出这片区域,我原以为至少会碰到一两条蛇,不过过程出奇的顺利,我们什么都没发现。想起来,似乎在白天很少见到蛇,看来这些蛇是夜行动物,这也说明,这个林子的晚上绝对会非常的热闹。

我深入其中,闻到了令人作呕的腥味,那种味道非常古怪,走出遗迹,顺着地势回到林子的时候,胃力的东西已经卡在喉咙口子上了。

之后重新进入雨林里,遮天蔽日的感觉又扑面而来,不过经历刚才的一段时间,感觉雨林中的空气简直是享受,带着沼泽味道的湿润的空气比蛇腥要好上很多,很快,我呕吐的感觉就消失了。

在遗迹中耽搁了一段时间,潘子走的格外快,不过体力已经到达了极限,我们也不在说话,如此走了四五个小时。我们明显感觉到地势降低,沼泽中水流湍急起来,四周随处可以听到瀑布激流的声音,但是就是不知道是在何处。

潘子拿出了干粮,我们边吃边继续前进,不久之后,终于遇到了一处瀑布,是一处地势突然降低的断层,不知道是什么古代遗迹。

一路走来,我几乎可以肯定这个山谷是一个凹底的地势,山谷的中心部分应该是最低的,这样所有的水都会流向那里,我感觉西王母宫应该就在那里,但是此时它已经一点也不重要了。

我们过了瀑布之后整个人都湿透了,到了瀑布下面又是以个洞天,水似乎渗入了地下,植被更加的密集了,几乎没有可以通行的间隙,而且在下面根本看不见天,我们几乎是挤着前进了一段距离,就失去了方向感觉。

三叔他们的烟稀薄的很快,纵使我们调了指北针的也担心会走偏太远,潘子只好停下来,爬上树去辨认方向。

我此时已经完全走蒙了,潘子一翻了上去,我和胖子就往树上一靠趁机喘口气。不过没多少时间潘子就指明了方向,道已经靠近三叔他们,催命似的让我们继续前进。

此时看表,已经马不停蹄走了一天了,在这种环境下如此强度的跋涉,我还真是没有经历过,现在我竟然还能站着,向来确实体质强悍了不少。不过现在已经超过我的体力极限了,我感觉只要一坐下,就能睡过去。

胖子和潘子商量了一下强行军又开始了,胖子看我脸色煞白,就知道我体力透支了,不过现在的情况他也不可能来帮我什么。只能不停的和我说话,让我转移注意力。

四周的景色单调,没什么话题,胖子就看着水中的东西,就问我道:“小吴,你说这些水淹着破屋子里,还有没有明器?”

我说按照楼兰古城的勘探经验来看,自然是一些东西,但是因为这座古城被水掩埋了,所以像丝绸竹简这些你就不用想了,锅碗瓢盆可能还能剩一些。你想干嘛?该不是又手痒。

胖子忙说不痒不痒,你怎么可以用不发展的眼光看你胖爷我,这一次咱们的目标就是来一票大的,东西到手我就退休了,这写瓶瓶罐罐值几个钱,咱们怎么样也得摸到能放到北京饭店去拍卖的东西。

我听着直叹气,心说烦人的事情这么多你还有心思惦记这个。

边走边说,刚开始还有点作用,后来我越来越觉得眼前模糊起来,远处的东西逐渐看不清楚了,树都变的迷迷糊糊。心说难道要晕倒了,这可真丢脸了。却听胖子道:“我靠,怎么起雾了?”

用力定了定神,揉了揉眼睛往四周看,发现果然是雾气,不是我的眼睛糊了,这雾气不知道是什么时候起来的,灰蒙蒙一片,远处的林子已经完全看不见了,眼前几米外的树木,也变成了一个一个的怪影。一股阴冷的气息开始笼罩四周的森林。

不知道是过度疲劳,还是温度降低的原因。我开始产生极度不安的心悸,犹如梦魇一般纠结感压迫在我的心口。

昨天晚上是在树海之外,树海之内有没有起雾我们并不知道,也不知道这雾气有没有毒性,不过我们没法理会这么多,防毒面具都没带进来。

我们扯了点衣服,弄湿了蒙住口鼻,又走了一段距离,并没有感觉什么不适应,才放下来。不过这时候,我们就发现,雾气已经浓的我们什么都看不见了。

\chapter{第一夜:大雾}

本来,按照潘子的估计,我们如果连夜赶路,再走五六个小时,没有太大的意外发生的话,我们可以在今天的午夜前就到达信号烟的位置,但是人算不如天算,没有计算到的是,日落之前气温变化,大雨过后的树海中竟然会起雾。

这样一来,我们就根本无法前进了,我们靠着指北针在林中又坚持行进了二十分钟,潘子虽然心急如焚气急败坏,但是也不敢再前进了。

虽然我们的方向可以保持正确,但是在林中无法直线行进,现在能见度更低,很可能路过了三叔的营地都不会发觉,甚至可能一直在走s形的路线。

加上能见度降低之后,在这样的雨林中行进体力消耗极其大,已经到了人无法忍受的程度,走不得几米,就必须停下来喘气,四周灰蒙蒙的也让人极度的不安。

雾气越来越浓,到我们停下来,能见度几乎降到了0点,离开一米之外,就只能见到一个黑影,本来树冠下就暗的离谱,现在简直如黑夜一般,我们不得不打起矿灯照明,感觉自己不是在丛林里,而是在一个长满了树山洞中。

潘子说按照原来的计划到达三叔那里已经是不可能了,现在只能先找一个安全的地方暂时休息,等到雾气稍微消退一点,再开始行进,一般来说,这种雾气会在入夜之后就逐渐消散。来的快去的也快。

潘子有丛林经验,说的不容反驳,我真的是如释重负,感觉从鬼门关上回来了,要再走下去,我可能会过劳暴毙,活活累死。

我们找了一棵倒塌在淤泥中的枯萎朽木,这巨木倒塌的时候压倒了附近的树,四周空间稍微大一点,我们在上面休息,一开始潘子说不能生火,但是最后浑身实在是难受的不行了,才收集了一些附近的干枝枯藤,浇上油做了一个篝火。

这些干枝枯藤说是引火,其实都是湿的,一开始起了黑烟,烤干之后,篝火才旺起来,胖子不失时机的就把更多的枯藤放到一边烘烤,烤干一条就丢进里面。

实在太疲劳了,连最闲不住的胖子也沉默了起来,我们各自休息。

我脱掉鞋,就发现袜子全磨穿了,像个网兜似的,脚底全是水泡。从长白山回来之后,我的脚底结了一层厚厚的老茧,我当时觉得永远不可能再磨起水泡了,没想到这路没有最难走,只有更难走。

按摩着脚底和小腿上的肌肉,潘子回忆着刚才我们行进的路线,说晚上看不见烟,明天早上烟也肯定熄灭了,我们现在基本还能明确自己的位置,要做好记号。胖子重新分配装备,将我背包里的东西继续往他们背包里挪。

我有点不好意思,但是此时也不可能要面子了,体力实在跟不上了,胖子让我睡一会儿,说这样绷紧着休息,越休息越累,我不想逞强,闭上了眼睛。

不过,此时已经累过头了,四周的环境又实在很难让人平静,眯了几分钟,浑浑噩噩的睡不照,就闭目养神。

才有点睡意,就听到胖子轻声问潘子:“大潘,说实话,要是咱们到了那个地方,你那三爷人不在那里,你有什么打算?”

潘子道:“活要见人,死要见尸,我当然要去找,你琢磨这些干什么?”

胖子道:“老子是来发财的,不是给你三爷来擦屁股的,你三爷现在没按计划行动,把事情给整砸了,小吴醒着胖爷我照顾他的心情没说,但是现在不说不行了,我丑话可要说在前头,要是你三爷不在了,我拿了我那份装备,我可就单干我的正事了,这林子这么大,我不会跟着你去找他们的。”

潘子冷笑道:“散伙?这林子诡秘异常,我们还没遇到状况,要是遇到状况你一个人应付的了,况且这外面大戈壁几百公里,你就算摸到东西活着出去,一个人能穿出戈壁?”

胖子笑了一声,没接话,道:“你胖爷我是什么人物,这些老子都自有计划,提前和你说说,就不劳你担心了。”不过,听他的语气,似乎对这个事情胸有成竹。

潘子摇头,叹气道:“这事情老子不勉强你,拿到装备,你要走随你,不过,可不要指望遭难的时候我们来救你,我们摸到的东西你也别指望拿一份。”

“你还唬我,你也不打听打听,唬人胖爷我是祖宗。”胖子道:“胖爷我早想明白了,你三爷这次进来,根本就不是来摸冥器的,要摸到好东西,老子只能单干,得和那小哥一样,玩失踪,前两次那小哥都把我们甩了,指不定摸了个脑满肠肥,咱们都不知道。”

我听着就实在忍不住笑了出来,心说这我倒可以肯定,闷油瓶甩了我们不是为了钱。

胖子一看我没睡,就不说了,只道:“大人说话小孩子听什么,去去去,睡你的觉去。”

我心里感觉胖子是知道我在假寐,这些话话里有话,应该是说给我听的,但是我不知道他想表达什么意思,好像是在提醒我闷油瓶每次都消失的事情,难道是他注意到了什么,想单独和我说吗?

不过这种场合下,我也不可能避开潘子,只能不做任何的表示,等待时机,而且我实在太疲倦了,根本没法去琢磨这些复杂的事情。

之后大家又陷入了沉默,我靠在一边一根枝桠上,逐渐就平静了下来,睡死了过去,连怎么睡着的都不知道。

期间应该有做了一些梦,但是睡的太沉,梦都是迷迷糊糊的。也不知道睡了多久,我醒了过来,发现四周的雾气淡了很多,看了看表,才睡了不到三个小时。

睡的相当好,精神一下子恢复了不少,但是身体犹如铁锈般的酸痛,看样子比刚才还要糟糕,我同样也有想过以后不可能再有这种肌肉酸痛的情况发生,没想到还是没办法逃脱。

我活动了一下,舒缓了一下筋骨,感觉好多了,就看到胖子正坐在那里,头朝上看着一颗树。四周没有看到潘子。

我心中奇怪,问他道:“潘子呢?”

胖子立即朝我做了一个不要说话的手势,指了指树上。

我按着腰,忍着浑身的酸痛站起来,走到他的身边,抬头看去,只见雾气间已经能看到月亮模糊的影子,树上似乎有人,潘子好像爬到树上去了。

我问怎么回事情?这小子现在学猴了,喜欢在树上休息。胖子就轻声道:“刚才有点什么动静,他爬上去看看。”

话没说完,树上传来嘘的一声,让我们不要说话。

我们赶紧凝神静气,看着他,又等了一会儿,就看到潘子朝我们做手势,让我们马上上树。

\chapter{第一夜:手链}

我们两个马上活动手脚,开始爬树。

这里的树木比较容易攀爬,落脚点很多,但是需要格外小心,树干之上都是苔藓之类的植物泥,落脚不稳就容易滑脚。一但滑了第一下就可能会一路摔下去。

我们小心翼翼,一步一口气,好比在爬一颗埋着地雷的树,好不容易爬到了潘子的身边。

潘子所在的地方是树冠的顶部之下,枝桠相对稀疏的地方,雾气更淡,这棵树很高,头顶上是雾气中透出的毛月亮,大概是因为这里是高原,月亮特别的明亮,竟然能透过薄雾照下来这么多的光线,不过月光和雾气融合,还是给人一种毛呼呼的感觉。在晦涩的白光下,能看到四周的树木,但是绝看不清楚,雾气中一切都暧昧不清。

我们上去,轻声问潘子怎么回事,他压着极其底的声音道:“那边的树上,好象有个人。”

“哪边?”胖子轻声问。潘子指了指一个方向,做了一个手势:“大概20米左右,在枝桠上。”

“这么黑你怎么看的见?是不是那小哥?”

“本来也看不见,刚才它动了我才发现。”潘子皱着眉头,又做了个手势让胖子小点声。“有树叶挡着,看上去不太清楚,但应该不是那小哥。”

“你没看错吧,是不是急着想见你三爷晕了?”

潘子没空理会胖子的挤兑,招手:“我不敢肯定,你自己看!”说着拨开密集的枝桠,便指着远处的树冠让我们去看。

我第一眼只看到一大片茂密的树冠,我的眼睛有少许近视,在普通的时候还好,在这么暧昧的光线下很容易花眼,所以找了半天也看不出有什么,胖子的眼睛尖,一下便看到了,轻声道:“我操,真有个人。”

潘子递过望远镜给我,我顺着胖子的方向看去,果然就看到了树冠的缝隙中有一类似于人影的形状,似乎也是在窥视什么,身体缩在树冠之内,看不清楚,但是能清楚看到那人的手,满是污泥,迷蒙的毛月光下看着好像是动物的爪子。

是谁呢?

我问道:“会不会是昨天晚上咱们在沼泽里看到的那个‘文锦’,小哥昨天没追到她?”

潘子点头:“有可能,所以才让你们小声点,要是真是她,听到声音等下又跑了。”

我把望远镜递给吵着要看的胖子,对潘子道:“怎么办,如果她真是文锦,我们得逮住她。”

潘子看了看四周的地形,点头:“不过有点困难,从这里到那里有20多米,如果她和昨天晚上那样听到声音就跑,我们在这种环境下怎么也追不上,她跑几下就看不到人了,最好的办法就是能偷偷摸到树下,把她堵在树上。而且,咱们得尽快了——”他看了看一边的树海。“现在雾快散了,我们也不能耽误太多时间,抓住他之后,要赶紧赶到三爷那里。”

我想了想说行,没时间犹豫了,只有先做了再说。想着拍了一下胖子想拉他下树。

胖子忙摆手:“等等等等。”

“别看了,抓到她让你看个够。”潘子轻声喝道。

胖子还是看,一边看还一边移动,潘子心急就火了,上去抢胖子的望远镜,被他推开。“等一下!不对劲!”

我们愣了一下,胖子眼尖我们都知道,他忽然这么说,我们不能不当回事。我和潘子交换一下颜色。这时候就听到胖子倒吸了一口冷气,放下望远镜骂了一声,立即就把望远镜给我:“果然,仔细看,看那手。”

我急拿过来,仔细去看,胖子就在边上道:“看手腕,在树叶后面,仔细看。”

我眯起眼睛,往那人手腕看去,穷尽了目力,果然看到了什么东西,看到了的那一刹那,我心里咯噔了一声似乎意识到了什么,下一秒我一下就明白了。

这是阿宁的那串铜钱手链!

因为之前在魔鬼城里的经历,以及那个怪梦,我对那条铜钱手链印象极其深刻,所以即使是在这样的光线,我也能肯定自己绝对不可能会看错。

“狗日的。”我也吸了一口凉气。

如此说来,远处树上的这个“人”,竟然是阿宁的尸体,那些蛇把她的尸体运到这里来了?

潘子看我的脸色有变,立即将望远镜拿过去,他对阿宁的印象不深,我提醒了他之后,他才皱起眉头,歪头若有所思。

“从入口的地方拖过密林沼泽,又搬到这么高的树上,这简直是蛇拉松比赛,这些蛇还真是有力气。”胖子往边上的枝桠上一靠,嚼了嚼嘴巴,沉思道:“这些蛇怎么好像和蚂蚁一样,你们说会不会它们和蚂蚁一样是群居性动物,它们的蛇巢里藏有一条蛇后,这些尸体是运给蛇后吃的。”

“什么蛇厚?”我一下子没听懂。

胖子道:“你没掏过蚂蚁窝吗?蚂蚁里的蚁后负责产卵,蚂蚁负责养活蚁后,我看没错了,肯定是这样,这里的鸡冠蛇可能和蚂蚁和蜜蜂有着一样的社会解构。这林子里肯定有一条蛇后,这些小蛇都是它生的。”

我越加的疑惑:“确实,这些蛇的行为无法理解,但是你这么猜肯定是没道理的,蛇和昆虫完全不同种类,这种可能性非常少。”

“我觉得这应该算是个不错的推测。”胖子道。

我不置可否,不想继续讨论这个问题,再次看到阿宁的尸体,又是这样的场面,让人很不舒服。我都不敢想象,隐藏在树冠内的部分,现在是什么样子了,虽然胖子表过自己对于生死的态度,但是他这时候说的话还是让我感觉有点郁闷。

三个人沉默了一会儿,胖子就道:“他娘的不管它们要来干嘛,显然尸体在这里,附近肯定有很多蛇,我们最好马上离开这里。”

“这就不管她了?”我心里有点不舒服:“既然找到了尸体,要么——”

胖子摇头,我想想也不说下去了,这确实不是什么好想法,这里的蛇我们一条也惹不起,况且也许阿宁也不想我们看到她现在的样子。于是叹气,不再去看那个方向,轻念道了一声:“阿弥陀佛,得,我闭嘴。”

这时候我就发现潘子一直没有把望远镜放下来,心说奇怪,看这么久还没看清楚。仔细一看却发现潘子的手竟然满是汗,脸的都发青了。

我一惊,凑上去问道:“怎么了?”

潘子放下望远镜,有点异样,摇头对我道:“没什么。”

但是那表情绝对不是“没什么”的表情,我拿过望远镜再次往那放向看去,却发现确实没有什么异样。心中就怀疑了一下,不过胖子已经动身下树。我没功夫再考虑这些,最后看了一眼远处,就跟着胖子爬了下去。

潘子下到树下,脸色已经完全恢复正常了,刚才也不知道是怎么回事情,但是我发现潘子老是往那个方向去看。

他不说,我也不想问,我估计他也可能是不能肯定,与其问出来让自己郁闷不如就这么算了。三个人立即收拾了东西,背上了背包,潘子修正了方向,就立即准备离开。

刚想出发,潘子又看了看那个方向,忽然就停住了,这时候胖子也发现了他的异样,问他怎么了,他抬手指了指那个方向,做了不说话的收拾。

我们都停下脚步,就恍惚间听到四周某个方向的林子里,传来了一声声轻微的人声,西西叔叔,好象是有人在说话。

因为林子十分的安静,所以这一下下的声音显得极为突兀,我三个都莫名其妙。我更是一头冷汗,侧耳去听,就感觉这断断续续的声音,好像是一个女人在低声说话。

我们静静的听,那声音忽高忽低,飘忽不定,又似乎是风声刮过灌木的声音,然而四周枝叶如定,一点风也没有,而让我们遍体生寒的是,声音传来的方向,就是阿宁尸体的方向。

胖子轻声骂道,“狗日的,这演的是哪一出啊,该不是那臭婆娘真的诈尸了,在这儿给我们闹鬼了。”

我说不可能,但看了看四周,妖雾弥漫,黑影从从,这里不闹鬼真是浪费。

胖子道:“不是鬼,那是谁在说话?”

我又想起了昨天晚上看到的“文锦”,心说不一定是闹鬼,也有可能是这个女人在附近,然而昨天晚上,她并没有发出声音来,所以其实也不知道她是男是女。

还有另外一个可能,就是三叔或者他的人就在附近,那就太走运了。不过这情形实在是古怪,三叔的应该不会发出这种声音,之前我碰到过太多离奇的事情,在这关口,我还是自然而然生了不详的预感。

我不喜欢这种感觉,对他们道:“这里月光惨淡,我看肯定有事要发生,咱们还是快走,呆着恐怕要遭殃。”

说罢就问潘子:“你刚才算了这么久,我们现在该往哪里走?”

潘子脸色铁青,就指了指那声音传来的方向:“问题是,我们要前进的方向,就是那棵树的方向。”

一下我就楞了,“那边?你没搞错?”

潘子拉上枪栓,点头道:“搞错是孙子。起雾之前,最后一次看到烟就是在那儿。”

一下我就蔫了,也不知道再说什么好,这时候胖子站了起来,骂道:“他奶奶的是福不是祸,是祸躲不过,人家堵在我们路上,存心不让我们好过,但是咱也不是好惹的,走,就去弄弄清楚,看看到底怎么回事。”说着站了起来就去过去。

我暗骂一声点背,潘子立即拉住了他,摇头道:“千万不可过去,你仔细听听她在说什么?”

\chapter{第一夜:丛林鬼声}

“孤魂野鬼还能说些什么,还不是还我命来这些话。”胖子道。潘子让他别废话,仔细听,他不是在和他开玩笑。

那人声在说什么,我倒真没注意,刚才声音想起,吓的我们三个头皮发麻,哪里还有心思去听具体的内容。

而且这声音并不响,如果不是这林子安静异常,恐怕会被我们忽略掉,现在不仔细去听也根本听不清楚,只感觉是一个女人,用着一种非常奇怪的语调,不知道在自言自语的说些什么。

潘子说起来,我们的注意力才集中到这方面,潘子示意我们屏住呼吸,仔细去听。

距离似乎太远,那声音黏黏糊糊,而是时段时起,就算这么听,感觉在哭,又感觉在念什么东西,也实在听不出个所以然来,唯一最大的感觉,就是语气暧昧。

“难道是在叫春?”胖子皱起眉头道:

潘子拿枪托拍了他一下,让他别乱说,我这时候有了一点感觉,“等等,怎么,这声音……好象在叫我的名字?”

“叫你的名字?我怎么听不出来?”

“不是叫我的本名,是在叫‘小三爷’,你仔细听听。”

胖子听了听,摇头听不出来,我更仔细的听,也反而听不清楚了,不过那声音确实有点怎么回事情,好比鬼魅勾魂一般。“确实是在叫我的名字,就是不是,也是像是在叫我的名字。”我斩钉截铁道。

潘子点头:“没错,你说这里知道你名字的女人有几个人?我看这真是闹鬼了,阿宁那婆娘可能举的自己死的冤,不想一个人烂在这里,想找我们陪葬。”

我摇头,这时候想到了另外一个可能性:“天,难不成她还活着?”

“活着,怎么可能?老大,你不是没看到,你背到峡谷口的时候,她都烂了。”胖子道。

我一想,心里又凉了,的确,阿宁的死非常确定,一点可能迂回的地方都没有。当时检查的非常仔细。

潘子道:“我看是这死女人想引我们过去,我们绝对不能上当,你们跟着我走,我们想办法迂回过去。那边情况不明,可能有很多的毒蛇,而且这情形诡异异常,去了讨不了好。”

我看向胖子,问他的意见。

一边是未定的因素,一边是生死存亡,高低立现,胖子也犯了嘀咕,想了想只得收敛好奇心,一顿,道:“你胖爷我不是反悔,不过大潘说的说对,咱们手里家伙太少了,这一次还是悠着点,打鬼也要看鬼是谁,万一真是阿宁我也吓不去手!”

我如释重负,我本来就不想去看什么女鬼,也不知道胖子是怎么想的,没有什则好,要是有什么,咱拿什么本事脱身啊?想着立即应声。

三个人转身动身,不再理会那诡异的声音,潘子定了个方向,我们小心翼翼的猫着继续赶路,试图从那声音发出的地方绕过去。一边也可以走近,听听,到底是怎么回事。如果真是三叔的人在说话,那我们也有足够的距离补救。

不敢把矿灯大的太亮,我们用布蒙着灯头,靠着黯淡的光芒在树木的缝隙中艰难的穿行。

说是绕过那树,其实距离离的并不远,那诡异的声音一直我们耳边徘徊,我们走的同时捏着把汗,连一句话也不敢说。

随着距离的靠近,我们离声音也越来越近,我越听就越不像说话的声音,那声音非常脆,不停的重复着一个节奏,完全无法感觉到底是什么发出的。

不过能肯定发出声音的地方,就在附近的一个方向,我的心理作用作梗,感觉那个方向看过去都是鬼气森森。

一边走一边注意着这个声音,我就听的入了神,听着听着,我感觉到这声音好像在哪里听到过,我脑子有点印象,而且还很新鲜。

我立即让他们停了停,听了一下,忽然,我就想到了那是什么:“糟糕,难道这是阿宁身上的对讲机在响?”

“对讲机?”

我道阿宁他们的制式装备里包括对讲机,我没看她从她口袋里拿出来过,这种对讲机防水防火防摔,你要不是认真想对付它,它不是那么容易坏掉,而且可以连续使用三个星期不需要充电。阿宁很可能一只开在那里。“把对讲机的话筒口用湿的布蒙上,然后如果有静电噪音,你感觉会不会和这个声音很想?”

胖子没经验,但是潘子显然知道,就猛点头:“小三爷说的对,真的很像。”

“那现在是谁在呼叫她?”胖子问:“丛林中的无线电信号很弱,无法传播太长的距离。”

“但是她在树冠上,如果对方也在树冠上,或者说,在峡谷的外延,那么很可能就可以收到信号。而且你看那声音时段时续,说明对讲机开在自动搜索频率的功能上,它循环收索所有的频率内的声音,显然这里有一道无线电频率正在被人使用,潘子,我三叔这一次有没有带对讲机这种东西?”以为在魔鬼城里对对讲机印象很深,所以这些功能我都倒背如流。

“三爷绝对不用这种东西,因为下地淘沙绝对不会有几个小组分散行动的情况发生,一般斗就一个,能下去不错了,他娘的,不过车上有无线电,难道是在戈壁上留着守车的人在使用这个频率通话?我……”潘子突然就想到了什么:“我明白了,他们也看到红烟了,可能三叔和他们有什么约定,他们在进行调度。”

我就道:“我们得拿到那个对讲机,这样就可以和戈壁上的人对话,我们就能知道他们的行进计划,以及三叔为什么会在我们之前就进入到了沼泽中心,而且我们离开的时候,也可以让他们做接应,说不定我们可以从峭壁直接上去。”

胖子兴奋起来,看来他实在是在林子里走的厌烦了,道:“那还等什么,他娘的既然不是鬼,咱们也不用客气。”

潘子摇头道:“这事情要考虑周详,没有鬼还有蛇,四周全是树枝,冷不丁黑暗里蛇出来钉你一口,那你就真成鬼。”

这蛇其实比鬼还头疼,胖子急的记得抓耳挠腮,恨不得身上能有把喷火器:“要是带了蛇药就好了,看来以后真的得什么都带足了,谁能知道戈壁里的古城是这个样子的。”

“这种蛇会怕蛇药,老子很怀疑。”潘子道。“依我看,这些东西可能根本不是蛇。”

“不是蛇是什么?黄鳝?”

“我们那里说,东西活的久了都能成魅,这些说不定就是蛇魅,蛊惑人心,这座古城就是这些东西建的。”潘子道。“专门引人进来,吃掉。这包不齐就是个陷阱,咱们还是不要过去。”

胖子拍了拍他道:“你封建迷信的书籍看的太多了,被毒害的太深了,蛇就是蛇,就是它智商高点,它也只是蛇,怎么说也只是一种动物,咱们是万物之灵,他娘的还爬这些没手没脚的?”说到这里,他眼珠一转,计上心来,道:“哎,你们看这样如何,动物都怕火,你们把衣服全脱了,我用你们的衣服把我身上所有的地方全部都包住,淋湿了之后然后浇上烧酒,点起来我就冲过去,这些蛇肯定不敢咬一个火人,我拿了对讲机,然后回来跳进沼泽里,最多不会超过2分钟。”

“然后呢?我们是不是要拿着对讲机在这里裸奔?”我怒道:“你用点脑子好不好。而且这也太难控制了,我们用的酒精温度极高,万一你就烧死了怎么办?我们还需要你运装备呢。”

“哪有这么容易烧死。”胖子道。潘子就接道:“我们穿的都是防水透气的纤维衣服,一烤就干,一点就着,你不用浇酒精就能把自己烧成火人。这绝对行不通。”

胖子骂了一声,忽然又想起了什么:“哎,那或者咱们干脆在树下放把火,堆上湿柴,把烟烧起来,把那些蛇全熏走。”

我一听这个办法可行,对于这种东西就不能正面冲突,一定采取这种办法,以前农村里打老鼠也经常用这种烟熏。

于是点头同意,立即就开始要收集湿柴,胖子让潘子帮忙,潘子却一下又抓住了我们,不让我去动,他脸色很不好看,简直就是有点心虚。

我看潘子的脸色,想到他在树上那种表情,忽然意识到了什么,问道:“潘子,你刚才是不是看到了什么?”

潘子点头,有点欲言又止。顿了顿道:“老子本来不想说,怕吓到你们,不过现在还是说了吧。那尸体绝对有问题,我们他娘的打死都不能过去。”

“难怪我怎么感觉你他娘的怯了。”胖子道:“你他娘的到底看到什么?”

“我看到了,妈的我不知道怎么说——就在刚才,我在树上看到,我看到——”

潘子讲话的水平很差,用土话能说出来的话,用普通话就很难表达,说了半天不知道怎么形容。

“你是不是看到阿宁像蛇一样,从树冠里探出来看着我们?”胖子忽然就道。

潘子忙点头,“对,就是这样,嗯?你他娘的怎么知道?”

胖子脸色铁青的指了指我们身后,我看胖子的表情不对,忽然就头皮一麻,立即和潘子回头。

一下就看到我们身边那棵树下阴影中的灌木丛后,站着一个既像蛇,又像人的影子。就静静的蹲在那里,离我们只有五六米的距离,那对讲机的轻微声音,正从这东西的身上发出来。

\chapter{第一夜:逼近}

我们咽了口唾沫,胖子就呻吟了一声:“我操,她什么时候走过来的?”

我下意识的往相反的方向挪了挪身子,压低声音道:“不对,你听这声音,和我们刚才听到的一样,他娘的,刚才我们感觉离这声音越来越近,可能是错觉,不是我们靠近这声音了,而是这声音靠近了我们。”

这时候发现自己腿肚子不知道什么时候起在不停的打哆嗦,要就是个粽子,我也许还不是那么害怕,可这偏偏是阿宁,老天,天知道一个我认识的人现在竟然变成了这个样子,她到底成了什么了?我简直无法面对,想拔腿而逃。

不过,那玩意黑不隆冬的,我们也看不清楚,是不是阿宁也不好肯定。我心中实在有点抗拒这种想法。胖子矮下身子,想用手电去照那个人影,潘子就按住了他的手:“他娘的千万不要轻举妄动,你听四周。”

我们凝神听了一下,就发现四周的树冠上,隐约有极端轻微的悉悉索索的声音传过来,四周都有。

“那些蛇在树冠上,数量非常多,刚才那声音恐怕就是这东西发出来,勾引我们靠近的。”

我们浑身僵硬起来,胖子转头看着四周,四面八方全是声音:“妈的,咱们好象被包饺子了?”一边就举起砍刀。

潘子对他摇头,把我们都按低身形,让我们隐蔽,然后从背包里掏出了酒精炉,迅速拧开了盖子,“你用刀能有个屁用,咱们真的要用你的火人战术了。”

“你不是说这样会烧死自己吗?”我轻声道。“烧死我宁可被蛇咬死。”

“当然不是烧衣服。”潘子道,让我们蹲起来,迅速从背包里扯出了我们的防水布,批在我们头上,把酒精全淋在了上面。

我立即就明白他的意图,心说果然是好招数,这经验果然不是盖的。

潘子道:“手抓稳了,千万别松开,烫掉皮也得忍着,我打个信号,我们就往前冲。”

四周的稀疏声更近了,我们立即点头,潘子翻出打火机立即点上火,一下防水布上头就烧了起来,他立即钻进来,对我们大叫:“跑!”

我们顶着烧起一团火焰的防水布立即朝着一个方向冲去,立即四周的树干上传来蛇群骚动的声音,我们什么也管不了了,用尽最快的力气跑出去二三十米,酒精就烧完了,防水布就烧了起来,潘子大叫扔掉,我们立即甩掉已经开始燃烧的防水布,开始狂奔。

那是完全发疯似的跑,什么都不管,什么也不看,锋利的荆棘划过我的皮我都感觉不到痛,咬牙一路跑出去大概有一两里,我们才停下来,立即蹲入草丛里,喘着气去听后面的声音。出呼我意料的是,后面听不到任何蛇的声音,连那诡异的对讲机的声音也没有了。

我有点不太相信,自己就怎么逃脱了,不过着多少让我们松了口气,虽然寂静如死的森林,也并不是那么正常。我的手被烧伤了,也顾不得看看,现在揉了一下,发现只是烫了一下,当时还以为自己要废掉一根手指了。

“好象没追来,看来这些蛇也怕了我们不要命的。”胖子道:“大潘有你的,知道灵活变通,这一招老子记着了。咱们还有多少防水布?”

潘子喘气,脸都跑黑了,道:“防水布有的是,可他娘的酒精只剩下一灌了,这一招没法常用。快走,这地方太邪门,再也别管什么闲事了,老子可没命再玩第二回了,它们可能就在附近,没发出声音来。”说着看了看指北针。

我知道潘子说的没错,于是一边牛喘一边咬牙站起来,潘子确定了方向,立即推着我们继续往前。

我看了看身后的黑暗,心里想着那似人似蛇的影子,不由毛骨悚然,我们不敢再停下来,走更加急和警惕,几乎一有什么风吹草动就加快速度,这么一来体力消耗就成倍的增加,之前高强度的消耗显然没有办法在这么短的时间完全恢复,休息完之后的轻松感早就在刚才崩溃了,走的极度辛苦。胖子喘的像风箱一样,我几乎就是跟着这声音往前走的。

这时候我心里多少还有点欣慰,因为一路过来,每次有什么动静之后总会有事情发生,这一次竟然能绕过去,显然运气有所好转,这是以前从来没有的事情。

然而,走着走着,我忽然又隐隐约约的听到我们前方的林子里,响起了那种窸窸窣窣的声音,断断续续,犹如鬼魅在窃窃私语一般。

我们全部僵在了那里,胖子立即把我们两个按蹲下隐蔽,我累的实在不行,几乎崩溃,胖子喘着就森然道:“我操,大潘你怎么带的路?怎么我们又绕回来了?”

潘子看了看四周,脸逐渐扭曲,道:“我们没绕回来。”

我们向四周张望,确实看不到一点曾今来过的迹象。四周的林子很陌生。潘子就道:“他娘的,它们没追我们,它们在包抄。”

\chapter{第一夜:偷袭}

“包抄,这些畜牲还会这个?”胖子的冷汗下来。“胖爷我总算长见识了。”

潘子道:“老子早说了这些蛇不正常,这些绝对是蛇魅,都快成精了。”

听得前方的动静,群蛇似乎正在逐渐靠拢,但是树冠都静止着犹如凝固了一样,这声音就好比是一股无形的邪气在朝我们逼过来,我的汗毛都立了起来,问潘子道:“你老家有没有什么土方子对付蛇魅的?”

潘子道:“哪里能对付,在老底子这些都是神仙,听我姥爷说古时候都献过童男童女。”

胖子就道:“有没有靠谱点的,现在这时候我们上哪儿去找童男童女去?”

潘子道:“老子都是说古时候,现在这年头在城里哪里还碰到的这种东西,我看硬拼绝对是不行,你看阿宁一下就死了,我们还是撤吧,打游记他娘的我是祖宗,就和他们玩玩躲猫猫,看谁包抄谁。”说着就指了一个方向,要我们跟着他。

我听着潘子说的话,忽然有什么让我灵光一闪的东西,走了两步,我就想了起来,拉住他道:“等等,我感觉不太对。”

潘子看向我,我对他们道:“这里面有蹊跷,你们想想阿宁中招的时候,几乎没有防御的能力,一下就死了,其实这些蛇要弄死我们太容易了,他们根本不需要搞这么多花样,随便缩在某个草丛里,我们走过的时候叮我们一口,我们有几条命都没了,何必要搞的这么复杂。”

“你是什么意思?说明白点。”胖子问。

“它们在峡谷外面就有无数的机会要我们命,但是我们都安然无恙,蛇不同于人,它们不会犯低级错误,这些蛇没有采用暗算的方式,现在反而在搞这种虚张声势的诡计,可能它们的目的并不想要我们的命。”

潘子摇头道。“这说不通,不想要我们的命,那它们为什么要叮死阿宁呢?也许它们现在是在忌讳我们什么。”

我道:“你想想阿宁和我们有什么地方不一样?”

他们两个互相看看,胖子就惊讶道:“难道因为阿宁是女的?”

我点头,“很有可能就是因为这一点,这些蛇行为太乖张了,我们不能用普通动物的行事方式来推测它们的意图,我看这根本就不是包抄,它们这种行为背后有着其他更加诡秘的目的,我们如果贸然行动可能就会陷入到更加无法理解的境地里去。”

胖子皱眉道:“你这么一说倒也有道理了,那怎么办?难道应该硬拼。”

我摇头道:“我觉得我们应该先别轻举妄动,先搞清楚它们的意图,否则我们实在太被动了。”

胖子咧嘴道:“你真是天真无邪,咱们又不是蛇,怎么可能搞的清蛇的意图?”

我道人的意图我们都可以分析出来,何况动物,人败在动物手里往往是低估了对方的智商,我们应该把这些蛇当人去看,如果是一群人,在我们进来的时候,杀了我们其中的唯一一个女人,然后不杀我们,而是用这种方式,时刻让我们的神经保持紧张,你会觉得他们有什么目的?

三个人沉默了下来,胖子皱起眉头,迟疑道:“按照这么说起来,难道它们都是母蛇,在垂涎我们的美色?”

我心说都什么时候,你还有心思开玩笑,却发现胖子竟然是认真的在思考这个问题。

这时候潘子突然就吸了口冷气道:“哎呀,小三爷,这一次你说的太有道理了,我好像知道是怎么回事了——你们有没有听说过有一种森林,进去之后就出不来?”

胖子道:“你是说东北的‘鬼林子’。”

“我不知道怎么叫,越南那边叫‘akong’,树林本身就是非常容易迷路的地方,但是有种林子,树木的长势会受到某种规律的影响,不知道是巧合还是必然,会特别的容易迷路,而且这种林子有一种诡异的说法,在里面会受到各种声音的干扰,林子会像有生命的一样将你困死在里面。”潘子有点兴奋,砍了一根藤蔓,把里面水挤出来喝了几口道:“当地说起来,森林有他娘的自己的想法。”

我知道这种说法,有人说这是一种进化的体现,所有的森林都是复杂和诡秘的,而且越进化就越复杂,是因为森林希望将所有进入其中的东西困住,为其提供养料,这是森林的一种群体智慧。

但是我并不信,这样的说法太玄乎了,我更相信另一种说法,就是这种现象是某些动物将猎物往包围圈赶。

潘子也道:“现在的情况可能是类似,我感觉这些蛇确实在逼着我们往一个地方走,他们在修正我们的方向。”

听着我就出冷汗,觉得太不可思议了。

我们不敢往有声音的地方,又不可能回头,那么肯定是会选择绕路,那么只要在我们前进的地方发出声音,我们经过若干的绕路,肯定会到达一个地点。这想起来,其实和魔鬼城中的无形的城墙很相似。

潘子指了指那声音传来的方向:“我知道有一种狼就会这样来逼死大型猎物,如果猎物一直避开狼的声音,就被赶到什么绝境,比如说悬崖边上,然后被狼逼的摔下去,所以一旦开始绕路,我们就算是中招了。”

说着他眼睛里冒出凶光,对我们道:“多亏了小三爷多疑,否则咱们真的要倒大霉了。”

我心说你这是夸我还是损我,胖子就问道:“那现在如何是好,咱们难道只能走回头路?”

潘子道:“恐怕连回头路也不会有,它们既然堵了前面,必然也会堵了后面,这叫逼上梁山,咱们只能去会会它们了,既然它们不想杀我们,那么肯定我们或者对它们有好处,我们就赌一把,看看能不能冲过去。”

本来想着能一路避过危险,找到三叔再说,然而此时看来确实不可能了,潘子就提议主动进攻,无论对方是什么,也不能被诱入陷阱中,到时候可能有比死更惨的事情等着我们。

胖子说他早就说这么干了,我们还非得迂回迂回,浪费时间。

于是开始准备,不过在这种环境下,我们的武器几乎没有防身的作用,潘子的枪不能连发,如果第一枪没打中还不如匕首管用,而在这样的能见度下,打中目标几乎只能靠运气。

三个人一琢磨,就做了几个火把,两个短柄的,一个长柄的,一般的动物都怕火,就算是狗熊之类的大型猛兽,看到三团火也不敢贸然靠近。

而只要有这火焰帮我们威慑住对方,那潘子就有从容的时间射击和换弹,遇上危险应该能应付一下,当然,真是的情况到时候才能知道。

潘子说,如果对方是人,他完全可以神不知鬼不觉的摸过去,他在越南摸林子偷袭的本事相当厉害,但是如果是蛇,那就等于送死,况且还有那只不知道到底是什么的怪物。那东西不知道是不是阿宁,不过,既然声音是从这东西身上发出来的,那么它肯定也在前面,所以我们要尽量避免产生正面冲突,以通过为主要目的,实在不行再拼命。

我们准备妥当,点燃火把,就往那声音传来的方向缓缓猫去。

这其实是相当矛盾的事情,在午夜的雨林中,举着火把无以是最大的目标,比开着坦克还要显眼,但是我们三个全部都猫在那里,似乎要去偷袭别人,者有点像举着“我是傻B,我来偷窥”的牌子闯女厕所的感觉。

那窸窸窣窣的声音离我们并不远,大概就只有两三百米,我们所有的注意力全部集中在四周和那声音上,听着声音越来越近,也越来越清晰,那无线电噪音的感觉也越来越明显,我不由咽了口沫。但是即使如此,我们还是听不清楚那声音到底说的是什么。

很快,那声音就近的几乎在我们头顶上,潘子举手让我们停下,抬头去看头顶的犹如鬼怪一般的树影,辨认片刻,无法分辨。

在这边月光照不到树冠下的情形,我们的火把不够长,光线也没法照到上面,只看到树冠之间一片漆黑,声音就是从其中发出来,也无法来描绘树的全貌,反正这里的书,树冠几乎都融为一体,也说不出哪棵到底是哪棵。

\chapter{第一夜:冲突激化}

让我们奇怪的是,就算是到了树下,从树上传下来的,还是那种窸窸窣窣类似电磁噪音的声音,并没有任何其他声响,更没有动静。而且在这里听起来,我总觉的那声音不止一个,难道这不是对讲机的声音?

进入这里之后,一切的判断都无法肯定,我总感觉我没有抓住关键。

“那些长虫真他妈镇定!”胖子在一边用唇语道。

我预想的最好的情况,就是那些蛇对我们的这种举动目瞪口呆,无法做出反应,我们可以无惊无险的过去。不过我感觉这有点太贪心了,虽然树冠纹丝不动,但是我已经感觉到一股难以言喻的噪动在四周蒸腾,不知道是我的心里作用,还是确实能感觉到这种危险的气味。

我们已经相当靠近了,如果这些蛇的智商真的这么高,现在却仍然没有动作,显然这些东西相当的谨慎。

这种谨慎是我们可以利用的,因为我们什么料也没有,如果这些蛇突然改变主意要杀我们,那么我们连一点反抗的能力都没有,这种利用对方小心的性格暗度陈仓的计略叫做偷鸡,我以前以为只有对人类可以玩偷鸡这种把戏,想不到这一次我们还可以偷蛇的鸡,今年黄鼠狼该郁闷了。

我们不动声色,潜伏着慢慢过去,不敢说话,不敢有任何大的动作,更不敢有任何的停留,那声音越来越近,我就汗就如雨一样从我的脸上挂下来,声音越清晰我就越无法集中注意力。

这种感觉让我心慌,胖子发现我不对,立即捏了我一下,让我放心,我转头看他,就发现他也是满头汗。

不过被他这一捏好多了,这时候那声音就在我们的头顶,我们抬头注视上面,怕那些东西直接扑下来,一边迈步继续往前。

这走的不知不觉的就快了一点,我们犹如木偶一样走出去十米左右,就在我心中涌起了一股希望的时候,忽然,那树冠上传来的声音嘎然而止,顿时林子一下安静了下来,我们全部打了个寒战。

那一瞬间三个人都僵住了,但是胖子反应最快,退了我一把就让我跑,我却一下缓不过来摔倒在地,爬起来刚要狂奔,一件让我瞠目结舌的事情就发生了。

我竟然听到四周的树冠有一处抖动了一下,接着上面就有人幽幽的叫了一声:“是谁?”

我们一下全愣了,面面相觑,怎么回事情?怎么有人说话?

“难道是三爷的人?”潘子一下兴奋起来,“我靠,不是蛇,我说怎么就没事情呢,咱们真是自己吓自己。”他立即就对树上叫道:“是我,大潘,你是哪个?”

树上一下没了声音了,静了好久,我们又面面相觑,潘子就又叫了一声:“问你呢,你是哪个?”说着就把火把和矿灯都往上招呼。

火把一上去,树冠就抖了一下,接着那个幽幽的声音又道:“是谁?”这一次语调变了,似乎很痛苦。而且,这是个男人的声音。

我又感觉有点问题,但是这时候已经不可能是一走了之了,潘子道:“我上去看看。”

说着咬住火把,就开始爬树,胖子端枪掩护。我就拿刀警惕四周,掩护胖子。潘子的动作极快,几下边爬了一半,这时候树冠又抖动了一下,他没有犹豫立即加快了速度,几步冲进入了树冠之内,我也无暇去看四周,把脸转了上去。

原本以为立即会听到潘子的叫声,但是一下子动静就没了,我的神经开始崩紧,就看着树叶中潘子的火把移动,发现似乎没有打斗的迹象。

僵了片刻,胖子也很疑惑,就转头看我,我心说你看我也没用,我又没透视眼,又僵持了片刻,潘子还是没有动静。

这就有点不正常,我冷汗就下来了,心说难道这是蛇的陷阱,潘子该不是被秒杀了。

胖子就轻声喝了一声:“大潘!”

上面还是没动静,胖子就暗骂了一声,将枪递给我,就要接着上去,我还没接过来,忽然我和胖子的脸上都一凉,树上有什么东西滴了下来,一摸一看,竟然是血。

“妈个B!”胖子一下就毛了,枪也不给我了,一下将手里的火把就往上一甩,甩进了树冠,端枪就打。

连开了三枪,巨大的声响在无比寂静的森林犹如炸雷一般,一下整个树冠都抖了起来,在晃动的火光中,我竟然看到无数的蛇影,在树干中骚动起来。

我大惊失色,已经晚了,只见无数的红光犹如闪电一般从树上游了下来,上百条血红色的鸡冠蛇如流血一般布满了整个树身。并倾斜而下,朝我们直扑过来。

“我操,这里是蛇窝!”胖子大吼一声,又朝着蛇群连开了两枪,但是这点攻击力对于如此多的蛇来说实在是不值一提。他拉着我就大叫:“跑!”

此时根本没法估计潘子了,我心中一酸心知必然是凶多吉少,只得立即朝后狂奔。只听得身后稀稀疏疏的声音犹如瀑布一般急追而来。

顺着来时候的路线,我们连窜出去十几米,回头一看,在这么密集的丛林中,原本蛇也没法行动的很快,然后这些鸡冠蛇竟然在藤蔓乱草中犹如闪电一样,我们一停几乎就到了,一下自立起来,全部做出了攻击的姿态,就要咬将过来。

鸡冠蛇王贴地而飞果然是真的,我心说这次绝对死定了,胖子看我还拿着火把,立即抢过来,用力一挥,就将最近的几条蛇逼退。同时把枪甩给我,大叫:“装子弹。”

我一下去接,竟然没接住,枪就掉到了地上,弯腰去捡,一条鸡冠蛇一下窜到枪的附近,吓的我立即缩回手去。

胖子几乎吐血,挥动着火把冲过来,一甩将那蛇逼退,然后用后跟钩住枪带甩给我。

这一下我接住了,立即扯开枪膛,往里面填子弹才填了两颗,忽然脖子一凉,还没等我看清是什么,胖子的火把已经挥了过来,火焰从我耳边呼啸而过,将那蛇拍了出去。

同时一下我的头发就着了,烫的我大叫,胖子已经把枪抢了过去,单手对着逼来的蛇连开两枪,把其中两条蛇的脑袋打飞。但是随即后面的蛇一样就把打死的蛇掩盖了过去。

胖子还想开枪,扣了两下扳机没子弹了,大骂:“狗日的,你他娘的才装了两发!”

我回骂:“你自己抢的怎么快,有两发不错了!”

此时我们已经逼到一棵巨树前,后面再无可以退的空隙了,胖子拿着火把,徒劳的挥动着,也只能逼的那些蛇暂时退后,但是我知道只要胖子露出一点破绽,我们就完蛋了。

就在火烧眉毛之际,忽然就从一边的树上,呯一声爆起一团火花,一道火球呼啸着穿过树林,射到了我们面前的蛇群里,接着爆了开来,炙热的强光一下烧的我睁不开眼睛,还好我反应快,否则肯定直接爆盲。

“信号弹!”我纳闷,还没等我眼睛恢复,又是一发从远处飞来,正打在我们脚下。我眯眼睛只看到一片白光,脚下滚烫,一摸原来我和胖子的裤子着了,烫的我们立即拍打。

信号弹不是攻击性武器,但是其燃烧时候的高温竟然被用来在奇袭时候点燃油库,威力巨大,如果直接打在我们身上,我们马上就成半成熟的牛排。

强光烧了五十秒才暗了下来,眼睛很久才能睁开,全是影斑,不知道视网膜有没有烧坏,再看我们面前,鸡冠蛇群已经烧死了大半,高温引燃了我们脚下的灌木和藤蔓,在我们面前形成了一片火海,到处是焦香味。剩余的鸡冠蛇,全部都退了开去。

这一切发生的极快,真是九死一生,我看着眼前的情形,几乎瘫软了下来。

胖子拍灭了裤管上的火,就纳闷是谁救了我们,一边的灌木就抖动起来,潘子捂着肩膀从里面摔了出来,手里拿着信号弹发射枪,看到我们就摔倒在地。

我大喜:“你没死啊!”就见潘子浑身是血,似乎受了极重的伤。

我忙冲过去将他扶起来。他挣扎的爬起来,就对我们奄奄一息道:“快跑!”

我一楞,跑什么?

忽然就冲潘子后的灌木中,突然站起了一个巨大的黑影,一下抓住潘子的腿,在潘子大惨叫中闪电一般将他拖进了灌木中。

\chapter{第一夜:追击}

我被眼前的场景吓蒙了,还没反应过来,一边的胖子端着枪就冲到我身边,大叫:“子弹子弹!!”

我掏出一把,他立即抢过去,一边把抢夹到胳肢窝里,一手举着火把,单手填弹,一边跳进灌木从追了过去。

跑了几步看我不动,大骂了一声:“跟上,你呆这儿,等下我去哪儿找你去?”

我骂了一声,抽了自己一个巴掌,立即扯紧背包立即紧随其上。

撞紧灌木之中,行走万分困难,我咬牙趟着荆棘罗曼,追着胖子的火把,很快衣服全撕破了。追出去几十米,闪烁间就见前方树冠剧烈的抖动,拖着潘子的东西显然上了树。动静极大,显然这玩意是个庞然大物。

胖子冲到树下,我们就看到树上被什么东西刮出道道破痕,树冠上抖动的树叶朝边上的树移去,显然是要到另一棵树上。

我们不是猴子,根本就没有办法在树上追踪,但是在树下实在是跟不上了,胖子喘着气,又追了几步,只好端起枪,朝着树叶抖动的方向就瞄准。

我立即对胖子大叫:“他娘的小心打到潘子!”

胖子咬牙道:“横竖是死!赌一把!”说完抬手就是一枪。

枪声震耳欲聋,胖子的枪法极好,但是在这样的情况下根本就没有瞄准的目标,也不知道有没有打中,远处树冠持续抖动,这东西在树上比在平地上走还快,正在飞快的远去。

“妈拉个X的!狗屁的步枪,口径太小了。”胖子骂了一声,咬牙又往前追了几步,连开了四枪,把子弹全射了出去。

我清晰的看到子弹的火旋射入黑暗,还是没有作用,等胖子再次装填完毕,那东西已经出了我们的视野外,要追上已经不可能了。

“怎么办怎么办?”我急的大叫。

胖子也急的团团转,不过才传了一圈,他就发现了什么,把火把照到树上,我们看到树干上全是血迹。

胖子疾走几步,再照下一棵树,发现同样有。

“有门!”他叫了一声,立即把火把交给我,“妈个B,这下它倒霉了,咱们跟着血迹过去,端了它老窝,就算救不回潘子,也要它偿命。”

这可能是能救回潘子唯一的希望了,我一想也没多考虑,立即就点头。

胖子让我把子弹全部给他,潘子的子弹是放在香烟壳子里的,带的不多,一路过来已经用了不少,我全拿出来,就发现只有一盒半不到了,胖子又骂了一声:“下次如过还有夹喇嘛,没有口径5.54以上的家伙我就不来!”

“得,下次给你门火箭炮,别啰嗦了快追!”

胖子倒出五颗子弹,三颗放到衣服的胸口袋里,两颗咬在嘴里,一甩头:“走!”

我在前面用火把探树,他端抢掩护,我们循着血迹就朝黑暗的深处追去。

血迹一路衍生,树干上没有,则家下的灌木和蕨类植物上就有,我越看越觉得不妙,这血迹肯定是潘子的,这么多的血量,有可能是伤到动脉了,要真是这样,大罗神仙也救不回来了。

但是活要见人,死要见尸,事情绝对没有一个“绝对”。

追出去有五六百米,前面树冠上的动静已经听不到了,我们已经没法去顾及什么方向,迷路,已经刚才诡异的那些声音了。只知道血迹有在,我们就必须跟下去。

血迹断断续续,越来越不明显,我心里越来越不安,不知道是血止住了,还是血被放光了。

胖子警惕的看着树顶一边迅速前进,一边开始大叫:“狗日的,你他娘的有种回来连你胖爷我一起给叼了,看是你的牙口硬,还是你胖爷我的皮糙!”

我感觉阻止他:“你他娘的干什么?”

胖子道:“野兽喜欢在绝对安全的情况下吃东西,它听见我叫就会警觉,不会这么快对潘子下口。”

我道:“警觉个屁啊,你别把其他东西招来!”

他道:“你没看过动物世界?这么大的捕猎动物,有自己的势力范围,这个范围内不会有太多的大型猛兽的,最好能把它引过来,我们少走点冤枉路。”

我还是觉得非常不妥当,胖子却我行我素,继续边跑边大叫:“狗日的,你他娘的叼的那个有艾滋病,吃了肠穿肚——。”话没说完,突然绊倒了什么东西,一下滚倒在地。

我扶他起来用火把一照,只见地上的落叶上是潘子的背包,上面全是血。

胖子立即警惕起来,我想说话,他就对我做了一个安静的手势,让我把火把举高看树冠,我刚直起身子,就看到一个巨大的黑影悄无声息地从他背后的树上挂了下来。

\chapter{第一夜:搏斗}

我立即大叫,胖子一看我脸色有变,反应极快,看也不看立即就一枪托往回砸去,但是已经晚了,那黑影一缩躲了过去然后猛扬了起来,我就看到一团满是鳞片的东西从黑暗中闪电一般弹了出来,一下卷向胖子。

胖子真不是省油的灯,那么胖的身体竟然能反应这么快,顺势一滚就翻了出去,他一让开,火把的光线一下照亮了他的身后,我顿时看清楚了那影子的真面目,那竟然是一条水桶粗的褐金色巨蟒,浑身都是血,巨大的蟒头垂了下来,可以看到上面全是弹伤,血肉模糊。

我看着脑子一闪,一下就认了出来,这竟然就是在峡谷里袭击我们的那两条巨蟒其中的一条,竟然在这里又遇上了。

巨蟒一击落空,几乎没有停顿,缩回头颅张开血盆大口,就朝地方滚着的胖子咬去。

这一次胖子避无可避,一下屁股就给咬了个正着,巨蟒力气极大,身子一卷就将胖子卷了起来,扯到半空准备绞杀。

胖子没有闷油瓶缩骨脱身的功夫,一下就动弹不得,枪也甩在一边,大叫着在空中头朝下转了好几个圈。

我不知道哪来的勇气,立即冲过去用火把去敲蛇,但是这实在是蠢招,我被盘起来的蛇身猛的一幢,就摔了出去,火把砸到自己的裤子上,把本来就没剩多少的裤子又点了起来,我滚了一下把活压熄,胖子已经给卷到树冠里。

我慌起来,这时候手碰到了胖子的步枪,立即捡了起来,躺在地上单手对着蛇头就开了一枪。

很久没有开枪,枪的后座力把我的虎口都震裂了,但是单手开枪实在太勉强,这么近的距离竟然没打中,子弹偏了出去,撞到一边的树杆上。

我爬起来,还要再开枪,突然从树上传来一个咬牙的声音:“小三爷,枪给我!”

我抬头一看,之间潘子竟然还没死,在枝桠间伸下了流满鲜血的手来:“快!!!!”

我立即把枪抛了上去,他一抓抓住,晃晃悠悠的往枝桠上一靠,不去瞄准蛇,反而瞄准了一边的盘着蛇的巨大树枝,咬牙连开了三枪。

近距离就算这种枪的口径威力也极大,那一人粗的枝桠硬生生被打出了一个豁口,巨蟒本身就极重,加上胖子立即就把枝桠往下压折了,枝桠重重砸在地上,几乎像是一棵树倒了下去。

这一下摔的极重,蛇摔的蒙了,猛的就盘起来,一下也不知道是谁袭击了他,胖子趁着蛇盘起身子的一刹那,从蛇身中褪了出来,滚到我的脚边,此时已经被绞的面红耳赤,连站起来的力气都没有了,我将拉住他的腋窝,把他往树后拖,不想他却呕吐起来。

我心说糟糕了,该不是内脏被绞碎了,忙问他怎么样。

他一把推开我,极其艰苦的站起来,又吐了一大口,才道:“晕蛇,狗日的,比云霄飞车还晕——”

话音未落,巨蟒又扑了过来,血盆大口一下绕过树干,咬住胖子的肩膀,将他整个人扯了过去,连同我一起用力一甩,我翻到一边的灌木中,胖子大吼一声撞到树上,滚到地下。巨蟒根本不停,一下又拱起头部,满是倒勾牙的巨嘴张开,准备给胖子来致命的一击。

我心中大叫完了,千钧一发之际,突然有一根小树枝从树上扔了下来,打在了巨蟒头上。

巨蟒一抬头,立即看到了潘子,立即改变了攻击目标,一下就朝树上猛弹过去,就见潘子单手拿枪用力一插,一下把步枪连同他的肩膀一下就插进了巨蟒的喉咙里,接着巨蟒甩头就将他从树上提了起来,还没绞过去,就听一声闷想,突然巨蟒的咽喉部分就炸开了好几个口子,疼的它一下翻了起来。

潘子飞了出去,摔进了黑暗里,那巨蟒狂怒的疯一样的四处乱撞,巨大的力量把四周的灌木全部摔飞,枝桠给拍下来像下雨一样。

我抱头躲在树后,只看到树皮全被拍了下来,吓的不敢动弹,等了十几分钟,那动静逐渐就安静了下来,我探头去看,就看巨蟒翻到在地,扭动了几下不动了。

我完全懵了,直到胖子哀号起来,才立即反应过来,站起来跑过去,胖子已经完全晕了,我将他扶起来,他看着我对我胡话道:“把开蛇的司机拽过来,乘胖爷我没死,让老子捏死他。”

我看他还能说胡话,说明还没事情,将他放倒,立即跑到远处,去找潘子,这家伙恐怕真的是要凶多吉少了。

潘子躺在六七米外的树下,浑身是血,手里还死死的抓着已经炸开了膛的步枪,步枪的头都炸成喇叭花了。

我冲过去,他一张嘴就吐血,看着我说不出话来,我看着这一滩烂泥一样的人呢,急的直抓脑门。拍了自己好几个巴掌才稍微镇定一点。立即开始解潘子的衣服。

衣服一揭开,我就一阵反胃,只见他身上竟然全是口子,都是被巨蟒在灌木中快速拖动照成的,好在他身上本来就全是伤疤,皮肤相当坚硬,伤口都不深。

我掏出水壶,想给他清洗伤口先,他就艰难的举起一只手,往我身上塞,嘴巴艰难的动着。

我拿过来一看,是他的指北针,在这么剧烈的拖动下,他的背包都被甩脱了,这东西竟然他能拿着没有掉。

指北针上全是血,但是还能看到他做的记号,和夹角标尺,他艰难的发出了一声:“找三爷……小心……蛇会……”就浑身痉挛,再也说不出来。

“蛇会什么?”我不知道他的意思,不过没意义了,不由骂了一声,把指北针拿过来放进口袋,让他不要再说话了。他一下吐了好几口血,连呼吸都困难起来。

我心说怎么会有这么执着的人,一边草草的用水冲洗了他的伤口,然后翻起他的背包,从里面拿出抗生素给他注射进去。

一边的胖子已经缓了过来,一瘸一拐的捂住伤口靠过来,问我情况。

我其实根本就不知道情况,我甚至不知道潘子能不能救活,但是我根本没有勇气去求证这些。只能尽力去救他。

胖子也用水壶清洗了伤口,给自己注射了抗生素,我们把潘子搬到蛇尸的边上,我就坐倒在地上,开始给他做全身的检查。

四肢都有脉搏,而且并没有虚弱的趋势,我不由松了口气,但是不敢放松,立即翻找他的全身,一路上流了这么多血,很有可能是动脉出血,我必须找出那个伤口,如果不处理,肯定会失血而死。

最后我在潘子的左大腿后面找到了那个伤口,简直深的可怕,不过竟然已经止血了,结了很大一块血茄,上面全是碎叶子,可能是在被拖动过程中,潘子情急之下做的措施。

这个伤口必须清洗缝合,不然会感染,到时候这脚就不能要了,但是我们身边没有处理伤口的设备,全部轻装掉了。

这一下,我们确实必须和三叔汇合了,而且真的是越快越好。

我拿出潘子给我的指北针,叉掉上面的血迹,想找到方向,可是上面的刻度我完全看不懂,给胖子,胖子也摇头,我拍了一下脑门,骂自己当时干嘛不多点心思学一下。

胖子也筋疲力尽,完全没有力气折腾了,道:“得了,现在只有等天亮了,到你三叔哪儿只不定还需要多少时间,咱们全身是血,很容易招东西来,还是就在这里呆着安全,而且不给大潘缓缓,他恐怕也经不起长途跋涉的折腾了。”

我看了看潘子,意识已经模糊了,要是我受了这么重的伤肯定挂了,这家伙的意志真是没话说。不过确实,这伤实实在在,搬动他可能真的不行。于是整了一片空旷的地方出来,暂时将潘子安顿好,我看了看表也快天亮了,心里祈祷他一定要顶住。

我脱掉衣服给潘子盖上去取暖,一下子我也有点缓不过来,如此疲劳之下又经过了这么剧烈的搏斗,我感觉人有点虚脱。

我就坐下来喘气喝水,胖子把潘子的枪捡了回来,给我看,道:“这家伙是个爷们,他拿东西堵了枪眼,让枪在这蛇喉咙里炸膛了,把这蛇的脊柱给炸断了,否则,还真的不容易的那么弄死它。”

我想着就奇怪,之前在峡谷里,潘子枪枪要害,几乎把它的脑袋都打烂了,本以为它死定了,没想到这蛇竟然还没死,还能袭击我们。

胖子道:“这种大蛇智商很高,恐怕是之前给潘子打了好几枪,记住了潘子,一直在追踪我们,等机会要报复我们。”

我一边把火把甩甩亮,站起来去照蛇的尸体,仔细去看就发现这蛇真是大,简直像龙一般,就是这么看着还是感觉到自己背脊发凉。

蛇全身都是褐金色的大鳞片,一片有巴掌大小,最粗的地方简直有柏油桶那么粗。身上有很多的伤口,有的都腐烂发臭了。

我小心翼翼的走到蛇头的地方,用火把去照,就发现那蛇的舌头竟然还在动,显然还没有死绝,整个蛇头几乎被打开了花,黑色怨毒的眼睛反射出火把的光芒,犹如来自地狱的恶龙。蛇的脖子处,就是枪炸膛的地方,出现了好几个破口,肉全翻了出来,血流不止,已经躺了一地。

这蛇没有这么容易死透,说不定还能活过来,怕它突然再爆起伤人,胖子掏出砍刀,准备将蛇头剁下,但是砍了两下,这蛇身上连个印子都没有。

拿砍刀在蛇的鳞片上划了两下,才发现这些鳞片坚硬的要命,简直好像盔甲一样,胖子凑近蛇的伤口,就发现,这蛇竟然长了两层鳞片,皮糙肉厚,难怪潘子怎么打也打不死。

从伤口附近掰下两三片巨鳞,胖子道这能拿回去吹牛,绝对能干倒一大片,说着就放进兜里。我让他弄干净点,蟒蛇的鳞片下面经常会有寄生虫。还没说完,胖子就哎呦了一下,手腕好像被什么东西咬了。

翻过来一看,我发现一只蜘蛛一样的小虫子咬在小臂上,我们都见过这虫子,是一只草蜱子。我用火把靠砍刀,顺手就把它烫了下来。这时候,自己的裤裆里一疼,用手一摸,一下也摸出一包血。

我顿觉不妙,火把往地下的灌木中一靠,就发现我们站的四周的灌木上,竟然已经爬满了这种恐怖的虫子,有的已经爬到我们裸露在外的小腿上。

\chapter{黎明:血光之灾}

草蜱子嗜血成性,肯定是被这里的蛇血吸引过来的,这林子里草蜱的数量太恐怖,而且显然已经饿昏了,全部朝这里聚集了过来。

我把火把放低,讲四周的灌木上的草蜱烧了一遍,脚上又被咬了好几下,这时候没时间来处理了,只好任由着,想办法突围。

胖子用炸膛的枪临时做了一个火把,我们用火逼开它们,将潘子抬了起来,一看,潘子的背部已经全部吊满了血瘤子,刚才就应该已经被咬了,背部压在草下没发现。

胖子立即用火把去烧,一烧吊下来一大片,接着我们拖起潘子的背包,就急急离开。

幸好潘子的血已经止住了,没有招惹来更多的草蜱,回头看时候,就看到,巨蟒的尸体已经完全被黑点覆盖,很快这东西就会和在峡谷中看到的那具蛇的骸骨一样被吸的只剩下一层皮。

“评四害的时候没把这东西评上,真是委屈了它。”胖子看着就咋舌道。

我们一路抬着潘子,来到一处沼泽边,怕我们身上的血迹再次吸引来那些草蜱,就用水把我们身上的血和潘子的背包全部洗干净。洗着洗着,天就蒙蒙亮起来,黎明终于来了。我看着天上透出来的白光,欲哭无泪,这是我在这里度过的第二个黑夜,如果有可能,我实在不想有第三个。

胖子又问我往哪里走比较好,我掏出指北针,爬到树上,想学潘子的做法。

晨曦的光线昏暗,狱亮不亮的样子,我爬上树后,突然就闻到了一股极度清馨的空气,精神不由为之一振,这个鬼地方,要说还有什么好的话,早晨应该算是唯一能让我心情一荡的东西,这大概也是因为这里的夜晚实在太可怕了。

我深吸了一口气,刚想往四周观瞧,忽然我就惊呆了,我一下发现眼前无比的宽阔,在我的前方,不过五六十米的地方,赫然出现的一座巨大的神庙似的黑色遗迹。

我不知道怎么来形容我的这种感觉,我原本以为我会看到大片的树冠,和以前看到的一样,这突然出现的庞然大物让我一下子无法思考。好半天我才反应过来:如果我不是在这个地方爬上树,我可能会一直前进,从这座神庙的这么近的地方擦肩而过。

和以前看到的遗迹不同,这座神庙完全是一个整体,是一座巨大而完整的多层建筑,在现在的光线下看不到全貌,但是感觉规模可能远不止我们看到的那么大,而且看轮廓,保存的比雨林里的废墟要好很多。整片我能看到的遗址中只有少量的地方有杂草和树木,我看到了久违的大片的干燥巨石。神庙廊柱和墙壁上西域古老的浮雕在这个距离看上去就像巨石上细小的花纹,让人感觉无比的神秘。

我带着胖子往那里走,不到两分钟我们就从林子中穿了出去,走入了遗迹的范围之内,树木逐渐稀疏。

从树下去看,遗迹更是大的惊人,咋一看真的很像吴哥窟的感觉,到处是石头的回廊,不知名的方塔,最后来到一处高处,看到树冠后巨大的神庙,胖子看的都惊呆了,我一边看一边赞叹的对他道:“这地方要是开发出来,就是世界第九大奇迹了,你信不?”

“我信。”胖子忽然看到了什么,给我指了一个方向,“他娘的不是世界九大奇迹,也是我们的一大奇迹,你看那边。”

我朝他的手指方向看去,就看到在神殿之前的平地上,有连绵了一片的十几个大帐篷,竟然是一个野外营地。

帐篷是帆布的,很大,很旧,大大小小分的很散,颜色是石头的灰色所以刚才远看没发现,这不是阿宁他们的帐篷,但也没有旧到在这里立了十几年的地步,我心里就闪过了一个希望,这时候胖子已经叫了起来:

“这是你三叔的帐篷,胖爷我认得!”

我一下心中狂喜,差点就大喊出来,这真是山穷水复疑无路,柳暗花明又一村,看来老天爷玩我玩够了,想让我休息一下了。

我和胖子立即就往营地冲去,也不知道是哪里来的力量,我脑子只想着休息休息,睡觉睡觉。

我们狂奔过遗迹之前的开阔地,这是一片巨石堆砌成的广场,其间有很多的巨大水池,水是活水,非常清澈,能看到水池下面有回廊,回廊深处一片漆黑不知道通向哪里,显然原本这些部分都是在水面上下,现在被淹没了,我们看到的巨大神庙,可能只是当时神庙的房顶,或者最顶层,这建筑到底有多宏伟,实在无法估计了。

还没靠近营地,胖子就开始大叫,叫了半天没有反应,跑着跑着,就发现这个营地有点不对劲。

——整个营地安静的让人发毛,没有人走动,没有人影,没有任何的对话声和活动的声音,一片死寂,好像被荒废了一样。

我们跑到营地的边缘,就停了下来,已经筋疲力尽,当时刚才的兴奋已经没了,我已经意识到休息可能离我还远,胖子喘着气,静了静,仔细听了听,晨曦中的营地一点声音也没有,寂静的犹如雨林,感觉不到一点生气。

胖子就喃喃道:“不妙,咱们可能来的不是时候。”

\chapter{黎明:寂静的营地}

我们兴奋的心情,瞬间被眼前诡异的营地浇熄了,两个人互相看了看,我有点想抱头痛哭,我实在太累了,无法在应付任何的突发事件。我忽然觉得我要疯了,这个森林想把我逼疯掉。

胖子神经比我坚强的多,一边放下潘子,让他靠在一块石头上,一边就让我跟他进去查探。我们身边已经没有了雾气,他捡起一块石头打头,我们两个小心翼翼的警惕着那些帐篷,走进了营区。

一走进去,我才感觉到三叔这一次的准备到底有多充分,我看到了发电机,火灶台,竟然还有一只巨大的遮阳棚。遮阳棚下面是一块平坦的大石头,上面用石块压着很多的文件,我看到有几只刷牙的杯子放在一边的遗迹石块上,另一边两只帐篷之间的牵拉杆被人用藤蔓系了起来,上面挂着衣服。这简直像一个简易的居民居住点。

一切都没有异样,没有打斗过的痕迹,没有血迹,但是也没有人,好比营地里的人只是远足去了。

我们在营地的中间,找到了一个巨大的篝火堆,已经完全成灰了,在篝火堆里找到了烧剩下的发烟球,显然没有错了,发信号烟的就是这里。昨天烟就是从这里升起的。

帐篷的门帘都开着,可以看到里面没人,我们甚至还能闻到里面香港脚的味道。

蹑手蹑脚的转了一圈,什么都没有发现,胖子就和我面面相觑。

我想起了当时看到的信号烟的颜色。潘子说,红色的信号烟代表着“不要靠近”的意思,显然可以肯定这里发生了什么事情。不由又紧张起来,感觉浑身沾着刺茫,这些人到哪里去了?这里发生过什么?

不安的感觉无法压抑,如果我们装备充足,体力充沛,我甚至可能决定立即离开这里,在附近找安全的地方仔细观察,但是我们现在几乎就剩下半条命,我实在不想离开这里,再去跋涉。潘子的情况,也不可能这么做了,他必须立即得到护理。

在遮阳棚下的巨石上,胖子找到了一包烟,他心痒难耐,立即点上抽了一只,不过他实在太疲劳了,抽了两口有点顶不上劲儿,我也抽了几口,烟草在这个时候发挥的是药用价值,我慢慢舒缓下来。

接着,我们立即把潘子抬到其中一只帐篷里,我看到里面有两只背包,这种帐篷很大,一个帐篷起码可以睡四个人,帐篷里的防水布上还有着很多的杂物,手电筒,手表,都没有带走,我甚至还看到一只mp3,却没有看到任何的电灯,我心说难道外面的小型发电机是为了这个充电准备的?这也太浪费了。

在里面终于可以真正的放松下来,我们把潘子身上的衣服全部脱光,把剩余的草蜱弄掉,胖子翻动一人的背包,从里面找到了医药小盒子,用里面的酒精再次给潘子的伤口消毒,接着他就到营地里面的帐篷里逐个的翻找,找到了一盒针线,把潘子身上太深的伤口缝起来。

潘子已经醒了,迷迷糊糊的,不知道神智有没有清醒。胖子一针下去,他的脸明显有扭曲,但是没有过大的挣扎反应。

看胖子缝伤口的利落劲,我就惊讶:“你以前是干什么的,还会这手艺。”

“我和你说过你老忘,上山下乡的,针线活谁不会干,没爹打没娘疼,只好自己照顾自己。”他道:“不过这人皮还真是第一次缝,你说我要不缝点图案上去,否则这家伙会不会觉得太单调。”

我知道他在开玩笑,干笑了几声,表示一点也不好笑。

看着潘子我就感慨,万幸这巨蟒虽然力大无穷,但是牙齿短小,即使这么严重的伤,也没有伤到潘子的要害,只是失血太多,恐怕没那么容易恢复。看着赤身裸体的潘子,和他满身的伤疤,我忽然意识到他这些伤疤的来历了,恐怕没次下地,他都是九死一生,难怪三叔这么倚重他,这家伙做起事情来真的完全不要命。

不过,也许正是这样的做事情风格,虽然他每次都受重伤,却每次都能活下来,我心道。

胖子就对我道:“这叫做自我毁灭倾向。我很了解,我有一死党,以前也上过战场,和他一个班的人都死了,而且死的很惨,他退伍后就缓不过来,老琢磨当时为什么死的不是他,好像他活下来是别人把他开除了一样,和我倒斗的时候,干起事情来拼了命的找死,什么危险干什么,其实就是想找个机会把自己干掉,这种人就是得有个记挂,否则真什么事情都干的出来,所以我感觉你三叔对大潘来说就和救命稻草似的。”

我没有那么深刻的经历,无法理解胖子说的话,不过看他的手有点抖,就让他别说话,专心缝合。

两个人缝了将近一个小时,才把伤口缝好,手上全是血,又给潘子消毒了伤口,胖子才送了口气,此时潘子又昏睡了过去。

我们走出帐篷,都不得不坐下来休息,胖子并没有完全放松,立即看着四周就道:“这里不对劲,我看我们撑现在多收拾一下,也不能在这里久呆。”

我点头,想站起来,可是一动我就发现我实在走不动了,身上没有任何一块肌肉能听我的命令,胖子动了两下,显然也走不动,我两相视苦笑,就一起叹气。

说实在的,我们已经油尽灯枯,就算现在有火烧眉毛的事情,我恐怕也站不起来。无论是精神和肉体,已经超出了疲累的极限,完全就无法用了。

看我不动,胖子就苦笑说,不过现在再回丛林里,恐怕也不安全,与其在潮湿阴冷的地方被干掉,他宁可死在这里,听这mp3给蛇咬死也配的上他这种倒斗界名流了。

这有点阿Q精神了,不过我点头,还是真心的点头,虽然以前也经历过几次这种筋疲力尽的场合,但是这一次特别的严峻,主要是进入这里之前,我们穿越大戈壁已经耗费了太多的精力和体力,本来在进入峡谷之前我们已经非常疲倦了,之后完全是硬撑下来的。这种长途跋涉之后发现旅途才刚开始的感觉,让人极端的绝望,但是更可怕的是,我知道如果我能活下来,那么回去的路途才是真正的考验。现在阿宁的对讲机如果真的存在我们也不可能拿不到。那么这后面的事情完全会是一个噩梦。

这些东西想起来就让人头疼欲裂,我实在不想琢磨这些。

我们休息了片刻,煮了茶水,吃了点干粮,然后把身上的衣服全脱了,那衣服脱下来就穿不上去,随便找个洞都比裤脚大,只好不要,随便找了几件在晒的换上,再看自己的腿,全是荆棘划出的血痕,索性都是皮外伤,碰到水刺痛,但是没有什么感染的危险。

恶心的是那些草蜱子,腿的正面一只都没有,全集中在膝盖后的脚窝里,血都吸饱了,胖子找来专门的杀草蜱的喷雾,碰了一下,草蜱全掉了下来,我想要拍遍,胖子说一拍可能引更多的过来。就全部扫到灶台里,烧的啪啪响。

用自己血煮的茶水格外的香,我喝了一点,又洗了脚和伤口。已经完全麻木的肌肉终于开始有感觉了,酸痛,无力,麻痒什么感觉都有,我连站也站不起来,只能用屁股当脚挪动。

昨天晚上,只有我睡了一会儿,所以虽然困意难忍,我还是先让胖子睡一会儿,自己靠到一边的石头上警戒。

此时阳光普照,整个废墟全部清晰的展现在我们面前,四周无风安静,整个山谷安静的犹如静止一般,我料想胖子必然也睡不着,没想到不到一秒钟他靠在石头上就发出了雷鸣一般的呼噜声,脸都没掐掉,叼着就睡死了。

我把他的烟拿来自己抽,苦笑着摇头,这时候就感觉到自己几乎也要睡去了,立即强打了精神,竭力忍住不让自己睡着,但是不行,只要坐着不动,眼皮就重的很铅一样。

晨曦退去,太阳毒了起来,我深吸几口气,躲到遮阳棚里,一边强迫自己开始整理自己的背包。这时候,就看到塞在最里面的文锦的笔记本。

怕这珍贵的笔记会在这么严苛的跋涉中损坏,我用自己的一双袜子包着它,进入峡谷之后一直是计划赶不上变化,都没有机会再仔细看一下,这时候回忆,就感觉这笔记中的内容基本上帮不上什么忙。

也许是文锦来的时候距离现在也有一些年头了,虽然对于这座古城的历史来说,十几二十年的时间实在是太短的时间,但是对于这里的环境,也足够长了,二十多年,这里的树木恐怕完全是另外一长势。

倒是文锦写的:“此处多蛇。”没有骗我们,不过,我觉得文锦写的太简略了,这些蛇,实在有太多可写的东西,但是她只注意到多,难道是缺心眼不成?

笔记中记载了大量他们穿越雨林的而经过,我倒是可以再仔细看一下,看看有什么可以帮助我们的,这番之后,我脑子已经一片空白,一心想着怎么从这里出去,所以把笔记翻到了最后的部分。

然而实在是太疲倦了,字都发花,只好一边用水浇了浇眼睛,强打精神。翻了几页,我就实在熬不住了,感觉现在看书像催眠似的,就把笔记放下,然后尽量使脑袋一片空白,可是神智不可逆转的一点一点朦胧起来。

就在马上要睡着的时候,恍惚间听到一声幽幽的声音,好像是潘子叫了我一声:“小三爷。”

我一下惊醒,以为潘子有什么需要,立即揉了揉眼睛,痛苦的支起身子,却发现四周安静的很,没有任何声音。

我心说糟糕,累的幻听了,立即按柔太阳穴,却一下又听到了一声很轻的说话声,好像是在笑,又像是在抱怨什么,从营区的深处传了过来。

我一个激灵,心说他们回来了?

立即跑了出去,却见里面没人,我叫了一声“嗨”,在往几个大帐篷中间走,走了一圈,什么都没看到。

奇怪?我拍了拍自己的脑子,四周安静的让人心悸。

在原地站了一会儿,什么都没有发生,我莫名其妙的走了回去,坐回到原来的位置,深吸了几口气,点起了烟感觉可能是脑子精神错乱了。

但是立即我就知道我没有,我看到面前的石头上,有几个泥脚印,从远处一路衍生过来,到我坐的地方。这在刚才是没有的。

我警觉起来,往四周看了看,看到放着文件的大石头上也有很多的泥浆,显然有东西撑在了这上面。接着我就发现,我放在上面的文锦笔记的位置变了,上面沾着泥浆。

一瞬间我的困意全无,立即站了起来。

谁干的?这么多泥脚印,难道是那个文锦?这家伙看到自己的笔记,翻了一下?还是那个好像是阿宁的怪物?

我看了看四周,没有人在,就去看脚印,就看到脚印一路衍生,竟然是进了潘子的帐篷里。我一下紧张起来,立即捡起一块石头,到胖子身边,想叫醒他。

叫醒胖子没有这么容易,我摇了几下没有反应,又不敢发出太大的声音,只好咬紧牙关,自己朝帐篷走去。

帐篷虚掩着,我走到跟前,就看到帐篷的尼龙门帘上有一个泥手印,立即咽了口唾沫。

深吸了一口气,我想象着过程,我一下拨开门帘,然后冲进去,先大叫一声,如果那人朝我扑过来,老子就用石头砸她。

这时候忽然又感觉那石头不是很称手,但是也没时间再去找一块了。我又深吸了一口,咬牙一下钻进帐篷里。果然一下就看到一个浑身是泥的人正蹲在潘子面前。

我大叫一声,正准备扑过去,就看到那人转过了头来,我一下愣住了,我看到满是泥浆的脸上,有一对熟悉无比的眼睛。

\chapter{第二夜:再次重逢}

本以为是文锦尾随我们进入了营地,我拿着石块进去想堵他一下,却发现进入营地的,竟然是满身是泥的闷油瓶。

他的样子让我咋舌:一身的淤泥,几乎把他的全身包括头发全部都遮住了,他肩膀上的伤口全部都被烂泥糊满,也不知道会不会感染,不过倒是没有看到他身上添上新伤,他昨天晚上一定过的比我们舒坦。

我无法来形容当时的感觉,就僵在了那里,他转过来,我才反应过来,把石头放下,解释道:“我以为你是……那个啥……”

他没理我,只问我道:“有没有吃的?”

我一下想起来,他冲进沼泽的时候,什么东西都没带,看他的样子,可能一连二十几个小时都没有吃东西了。

我带他出去,给他倒了茶水,他就着干粮就吃了下去,什么话也没说,脸冷的犹如冰霜一样。

他吃完了,我给他布擦手,就忙问他情况怎么样,当时追出去之后发生了什么事情?又是怎么追上我们的?

他脸色凝重,边将脸上的泥擦掉,边断断续续的说了一遍。他说的极其简略,但是我还是听懂了。

原来前晚他追着那文锦出去之后,一直连续追了六个小时,无奈在丛林中追踪实在太困难了,最后不知道那女人是藏起来,还是跑远了,就追丢了,到他停下来的时候,已经不知道身在何处了。

没有任何的照明设备,失去了目标,连四周的环境都看不到,他算了一下来这里的时间和自己的速度,知道离开我们并不会太远,但是如果继续深入雨林,要回来就更加的困难,他就缩在了树根里,等待天亮之后回去。

这和我们当时的想法是一样,胖子推测他也可能会在早上天亮之后回来,但是天亮之后,事情却出了变化,天亮之后他看到了我们的信号烟,同时,他也看到了三叔他们点起的烟。

他按照距离判断出我们的烟的方向,回到我们给他留纸条的地方,却就发现那里已经被水淹了,他只好立即返回,来追我们,但是和我们一样,追着那烟走,路线并不笔直,一直没和我们碰上,后来在晚上听到枪声,才摸了过来,一直跟到了这里,发现了营地。

我听完心说真是碰巧,如果昨晚没有那场大战,恐怕他不可能找到我们。也亏的他能在这么恶劣的环境下保持这么清醒的判断。不过他能回来,我心里已经放下了一块石头,这本来我是不包任何希望的。

这时候看他抹掉身上的淤泥,我就问他,同样是跑路,我们虽然也很狼狈,但是也没搞成你这副德行,你遇到了什么事情弄成这样?

“这不是搞的,泥是我自己涂上去的。”他道。

我更加奇怪,心说你学何马打滚吗?还是身上长跳蚤了?你这体质,躺在跳蚤堆里跳蚤也只敢给你做马杀鸡啊。

他看了看手臂上的泥解释道:“是因为那些蛇……”

“蛇?”

“文锦在这里呆了很久了,这里这么多的毒蛇,她一个女人能活这么长时间肯定是有原因的,而且那个样子实在不平常,我感觉这两点之间肯定有关系,想了一下,我意识到这些淤泥是关键。”闷油瓶道:“我在身上抹了泥,果然,那些蛇好像看不见我。”

我一想就恍然大悟,原来是这样,我说文锦怎么是那个鬼样子,蛇是靠热量寻找猎物的,用淤泥涂满全身,不仅可以把热量遮住,而且可以把气味掩盖,确实可能有用。

心中不由狂喜,这实在是一个好消息。如此一来,我们在雨林中的生存能力就高多了,至少不再是任人宰割了。

闷油瓶把身上的泥大致的插了一下,就看向四周的营地,问我道:“你们来就这样了?”

我点头,就把我们的经历也和他说了一遍。

我从和他分开说起,说的尽量简略但清楚,一直说到我们到这里的时间比他早不了多少时候,这里已经没有人了,而且这里的情况有点奇怪,所有的贴身物品都没有被带走,也没有暴力的痕迹,好像这些人从容的放弃了营地,什么都没有带就离开了。

他默默的听完,眼睛瞄过四周的帐篷,也没有说什么,只捏了捏眉心,似乎也很迷惑。

我对他道你回来就好了,因为潘子的关系,我们暂时没法离开这里,而且我们也实在太疲倦了,需要休整,否则等于送死。现在多一个人多一个照应。

他不置可否,看了看我道:“在这种地方,多一个少一个都一样。”

我有点意外他会说这种话,不过他说完就站起来,拿起一个提桶,去营地外的水池里打了一筒水,然后脱光衣服背对着我开始擦洗身子,把他身上的淤泥冲洗下来,我看他的样子知道没什么话和我说,心里有点郁闷,不过总算他回来就是一件喜事了。

他洗完之后就回来闭目养神,我也没有去打扰他,不过我也睡不着了,就也洗了个澡,洗完之后感觉稍微有点恢复,就打了水回去,给潘子也擦了一把身,他的身上有点烫,睡的有点不玩问,我擦完之后他才再次沉沉睡去。

出来看到胖子,我想他总不需要我伺候了,一边坐下来按摩着小腿,也没有想再把文锦的笔记拿来看,转头看闷油瓶,他也睡着了,想起来他肯定比我们更累,就算是铁打的罗汉也经不起这么折腾。

我就这么守着,一直到下午三四点的时候,胖子才醒了,朦朦胧胧的起来看到闷油瓶,“嗯”了一声,好久才反应过来,道:“我靠,老子该不是在做梦吧。”

闷油瓶立即就醒了,显然没睡深,看了看他,又看了看天,也坐了起来,胖子就揉眼睛道,“看来不是做梦。工农兵同志,你终于投奔红军来了。”

闷油瓶真是一个神奇的人,虽然他寡言寡语,但是他的出现在好比一针兴奋剂,一下子我看的出胖子一下子是发自内心的高兴。我就道你高兴什么,你不是说要单干嘛。

他站起来坐到我边上,吐了几口血痰,道:“那是之前,小哥回来了,那肯定得跟着小哥干,跟着小哥有肉吃,对吧。”

我看他痰里有血,就知道他也受了内伤了,不过他满不在乎,应该是不是太严重,就让他小心点儿。

闷油瓶也没回答,胖子递我一根烟,自己从水壶里掉了点水出来洗了洗眼睛,就也问闷油瓶之前的情况。我就把刚才闷油瓶和我说的事情,和胖子转述了一遍。

胖子边听边点头,听到淤泥能防蛇那一段,也喜道:“我操,这是个好方子,有这方子,我们在沼泽里能少花点精力,他娘的我刚才睡觉的时候还做梦着有蛇爬在我身上呢,赖在老子裤裆里不肯出来,吓死我了。”

我笑起来,一下感觉只有闷油瓶在的时候,胖子的笑话听起来才好笑,道:“估计是看上你裤裆里的小鸡了,说起来,你到底孵出来没有?”

胖子道:“还没呢,整天泡在水里,都成鱼蛋了,呆会儿老子得拿出来晒晒,别发霉了。”

我大笑起来,胖子也笑,拍了我几下,“你笑个屁,我就不信你的还是干的,要不咱们拿出来拧拧?”

我摇头说不用了,胖子就让我去休息。虽然我有点兴奋,但是身体的疲劳已经无法逆转,我躺下不久也睡着了,大概是因为闷油瓶在的关系,这一下就睡沉过去了,觉得特别的安心,到了傍晚才醒来。

天已经夕阳红了,我起来就闻到了香味,是胖子在煮东西,也不知道煮的是什么,我动了几下,那种感觉好像是躺在坟墓里的僵尸复活了一样,身上的肌肉酸的都“苦”起来,无法形容这种感觉。

双手双脚都没有一点力气,几乎是爬到篝火边上靠在石头上,手都是抖的,就听到胖子在和闷油瓶说话,他正在问闷油瓶有什么打算。

我心说这家伙又开始搞分裂主义了,潘子废了,没人会逼他去找我三叔,他开始拉拢闷油瓶搞他的阴谋诡计了,立即靠了过去,听到他正对闷油瓶说:“我说这事情绝对不能让吴邪知道,否则他非疯了不可……”

\chapter{第二夜:秘密}

我听了心中暗骂,胖子就听到我的动静,一下回头,就面露尴尬之色,立即道:“醒了?来来来,给你留着饭呢,趁热吃。”

我怒目道:“你刚才说什么呢?什么事情不能让我知道?”

我大约是刚起床,脸色不好看,而且我现在最恨别人瞒着我,虽然我知道胖子所谓的不能告诉我的事情可能很不靠谱,但是我还是非常不爽。

胖子给我吓了一跳,还装糊涂:“什么不让你知道,我说不能让你累到,你听岔了吧?”

我呸了一口,坐到他边上道:“得了得了,你别以为你是我三叔,你可糊弄不了我,到底什么事?快说否则我跟你没完。”

胖子看了看我的表情,我就一点也不让步的看着他,催道:“说啊。都露馅了你还想瞒,我就这么不能说事情吗?你要不告诉我,那咱们就分道扬镳,你知道我最恨别人瞒我事情,我说到做到,你要不就看着我死在这里。”

胖子就挠了挠头:“妈的,你他娘的怎么学娘们撒泼,还要死要活的,我不告诉你可是为了你好。”

我骂到:“少来这套,这话我听的多了,好不好我自己会判断,到底怎么回事情?”

当然我只是说说的,不过我知道胖子不像三叔,这样的情况下他一般不会坚持,否则他受不了那种气氛。胖子不是一个特别执着的人,这一点我特别欣赏。

果然,胖子就看了看闷油瓶,闷油瓶没做任何表示,他就叹了口气,道:“你跟我来看样东西。”

我走不了,胖子就搀扶着我,来到遮阳棚的下面,上面的文件已经被整理过了,显然刚才他们看过,胖子把所有的文件叠到一起,露出了下面的石台子,我就看到文件下面,平坦的巨石表面,有黑色的碳写了好几个大字。

晚上黑,这里离篝火又远,看不清楚,胖子就打起矿灯给我照明,我走远几步辨认了一下,就愣住了。

那是一句话:

我们已找到终极入口,入之绝无返途,自此永别,心愿将了,无憾务念。

且此地危险,你们速走务留。

我就呆住了,胖子在我后面道:“我收拾文件的时候看到的,本来遮起来不让你看到,免得你看了钻牛角尖……你三叔这一次似乎是抱着必死的决心,而且,他娘的他选择了永远把你丢下。”

这确实是三叔的笔迹,虽然写的不是很正,但是做了拓本这么多年,我还是能认出其中的比划习惯,字写的相当的草,显然当时是在相当紧急或者激动的情况下。

我有点反应不过来,但心中出奇的心如止水,没有任何的情绪,脑中一片空白。我以为我总会有点什么情绪,比如担心或者愤怒之类的,但是我什么都没感觉到。

胖子以为我情绪低落,拍了拍我,就没说话,我走进几步,看着那些字,还是无法激起一点波澜。

对于三叔安危的担忧,已经在这漫长的过程中被消磨殆尽了,我虽然仍旧不希望他出事,但是在这样的环境下,就是出事,其实也并不奇怪。我都有自己会死的觉悟,那么死亡在这里已经不是我们需要担心的问题。

这和战争一样,在人人都有很大可能会死的时候,人们关心的只是事情的结果,而不是单个人的安危。

我忽然觉得我能够理解三叔,这句话出现在这里,已经三叔对我最大的关爱。如果我们互换一下身份,我追寻的一个无法告诉侄儿的秘密近在眼前,而前路极其危险,他即不希望我跟过去冒险,也无法告诉我事实的真相,那么这样的办法是最好的。

而且,如果是以前的我,我可能会泪流满面,从此三叔不再出现,而我则一直心怀遗憾,直到时间把它抹淡。

问题是我不再是以前的我了,我追寻的东西是这些事情之后的巨大谜题,而已经不是了三叔本身,所以这些文字对我来说只有一个意思,就是三叔还活着,他已经找到了路。事态和之前完全没有区别,这也许就是我心如止水的原因。

这不知道是我的一种进步,还是我的疲累,或许这些都是借口,三叔已经离我很远很远了。

我默默看了一会儿,就转身,胖子上来钩住我的肩膀,安慰我道:“我早说不让你看了,你看不听你胖爷我空添烦恼吧,这事情你也无能为力,不要多想了。”

我不想和他多解释我的心境,就没有回答,他钩住我就把我扶回到篝火边上,给我打了碗东西,让我先吃。

东西还是水煮的压缩饼干糊,我没有什么胃口,吃的很慢,胖子就继续安慰我,道:“你三叔不是凡人,非凡人必有非凡之结局,命中注定的,而且他经验这么丰富,不一定回不来。”

我叹了口气,说我没事,对于这种我已经习惯了,我现在就是在想,那入口在什么地方。

在雨林中的时候我就预见过可能会见不到三叔,因为红色的烟代表着危险,那么发烟者必然不会带在发烟的地方。当时我心里的琢磨,三叔可能发烟之后就离开了这里。

现在显然料对了大部分,只是没有想到三叔会找到了入口,那么意味着他们的位置已经完全不可知。

三叔在这里扎营并发现了入口。接着,他们应该开始整理装备,从容的离开这里,留下这个无人的营地。为了不让我跟来,他点起了红烟并且在这里留下了留言,接着进入了入口,不再回归。

他说此去没有归途,三叔不是那种会认命的人,这入口之内一定极其凶险,以至于他做出了自己必死的判断,或者是,本身有一些原因使得这个地方进入之后,就绝对无法返回。

事情看上去好像是这样。

按照这样的判断,这入口应该就在附近,也许就在这座神庙内,我不知道三叔手里掌握了多少,但是他应该不是瞎找,肯定是遵循了某种线索或者痕迹,这一点我们完全不了解,但是,未必就推测不出来。

胖子道:“那咱们过会儿到四周去找找有什么线索,也许也能发现。对吧,小哥。”

他问了一下闷油瓶,给他打了个眼色,显然也想闷油瓶安慰我一下,闷油瓶却摇头。我看向他,他就道:“吴三省既然这么写,就有把握我们找不到那地方。”

“为什么?”胖子就不服气。

闷油瓶看着篝火,淡淡道:“吴三省心思缜密,知道我们看到留言必然会得知入口就在附近,他不想吴邪涉险,所以如果入口很容易发现,他必然不会留下文字。他之所以会留,说明这个入口必定极难发现,或者即是发现了,我们也无法进入。”

他说的有道理,我叹了口气,想到其实即使有线索,三叔为了保险,也许也会把线索破坏掉。

胖子就郁闷道:“那咱们不白跑一趟?”

闷油瓶摇头:“对于你们来说,这也许是一件好事。”

“你胖爷我他娘的跑了上千公里,穿过戈壁越过沙漠,进入雨林来到这里,然后晒了太阳浴就回去,这叫好事?”胖子往石头上一靠就挠头。“这里什么破烂都没有,这一次真是亏的爷爷都不认识。”

闷油瓶抬头道:“不过,要找到入口,也未必绝对没有办法。”他看了看四周的营地:“而且,这个营地的情况很不对劲,不像是单纯的撤走,吴三省的话未必可信。”

\chapter{第二夜:反推}

闷油瓶看着篝火,静静的给我们解释了一遍疑点。他说这里最大的问题,是有好多的背包,三叔人员众多,即使他们精简装备,也不会多出这么装满东西的背包出来。而且,因为整个营地的状况非常的自然,这些背包都胡乱的放在每个帐篷里,加上各种的细节,一点也不像轻装整理过装备的样子。

这里的人确实是从容的离开的,但是这种从容不是通常意义上的从容,他们离开时候的状况肯定很不平常。

闷油瓶说的疑点,其实我也大概注意到了,只是这个疑点可以用一些比较复杂的理由解释,所以我没有在意。他提出来,我就点头,但是我道:“也许他们并没有全去,那个地方这么危险,说不定有些人留了下来。”

闷油瓶摇头:“如果有人留下来,就没有必要留下留言。这种留言,只有在所有人都会离开的前提下,才会留下,而且吴三省不会把必死之心告诉给手下,这是大忌,一定是在手下全部离开的最后时候,他写上去的,那些人,会陪着他一起去死。”

但是这样又解释不了现在营地的状况,除非那些人发现了入口,一开心什么都没带,就进入那个入口了,但是这是不可能的。

胖子“嗯”了一声,显然觉得很有道理,他喝了口水就皱起了眉头,想了想道:“这事情挺邪门,有点乱,从头上想恐怕想不明白,咱们得从后面反推。”

胖子总是有招,特别是这种时候。我问他怎么反推,他道:“这件事情我们知道很多的结果,但是不知道过程,那么得从结果去想,先从那字开始,按照小哥的说法,那留言在这里,说明他们全部都离开了,不可能有人留了下来,那么这里有这么多的背包在,就说明人比背包少啊。这……”

胖子说到一半就卡了一下,好像自己推出来的东西有点说不出口,但是我已经知道是什么意思了,人比背包少,而且少了很多。

那意味着,有很多人都死了。

而且死亡是在他们在这里扎营后发生的。

沉默了一下,胖子就继续道:“这里,或者附近,肯定发生过巨大的突变,这里没有暴力的迹象,那么突变应该发生在四周,当时应该有什么事情让他们离开了营地,然后再也没有回来,但是你三叔幸存了下来,带着剩余的人找到了入口,然后离开了,应该是这样的过程。”

我听了茅塞顿开,但是也听出了破绽,摇头道:“不对,通常在这种情况下,幸存者必然会离开这里,也不会有心情再去寻找入口,然后回来再留记号。”

“那么,应该他们在出事之前就已经发现了入口了。”胖子修正道。

我点头,闷油瓶也点头,喃喃道:“或许,他们正是因为那次突变,而发现了那个入口。”

“也有可能,不过这个没法证实了,也没有意义。”胖子道:“总之他娘的这事情能成立。”

“那么,突变是什么呢?”我问道,心里有点毛起来:“难道是那种蛇?”

胖子看了看四周的黑暗和沉入虚无的雨林,道:“你放心,在你睡觉的时候,我和小哥已经搞来了几桶淤泥,等一下抹到帐篷上,守夜的人身上也抹上,就不用忌讳那些野鸡脖子。不过,这地方邪气冲天,说不定还有其他邪门的东西,而且变故一定在晚上发生,我们一定要提高警惕。且要记得,一旦有任何的动静,绝对不能离开营地。”

我点头,就道:“那我守第一班。”

闷油瓶摇头:“你们警觉性太低,如果我们判断正确,那么这种变故将极其凶险,恐怕你们无法应付,今天晚上我守全夜,你们好好休息。”

\chapter{第二夜:它}

我感觉有点过意不去,但是我立即明白闷油瓶说的没错,我并不是一个意志坚定的人,在这么疲劳还未完全恢复的情况下,我不可能很好的守夜,一个不小心大家都会在危险之下。这时候让闷油瓶守全夜,其实是形式所逼。

胖子也没反对,只道:“我看一个人还不够,小哥你一人守不了这么大的地方,晚上我陪你半宿,熬过今天晚上,咱们明天换个地方在使劲休息。”

闷油瓶想了想,没做什么表示。胖子就道这么定了。

我心里想着是否也别睡了,但是转念一想,明天闷油瓶肯定得休息,我休息完可以顶他明天的,这样想心里也舒服了一点。

胖子伸了个懒腰,道:“这事儿基本上就这样了,也别琢磨了,咱们再想想明天怎么办?小哥你刚才说你有办法能找到入口,那又是怎么回事?”

闷油瓶看了看他,道:“这个办法很难成功,不提也罢。”

胖子立即道:“别,千万别,你先说来听听,我可不想就这么回去。”

闷油瓶沉默了片刻,就看了看我们:“我们去抓文锦。”

一下我和胖子都楞了,随即我就苦笑了,一边笑就一边摇头。确实,这个办法很难成功,我们到达这个营地已经是十分困难的事情,这里况且目标巨大,还有信号烟,文锦只有一个人,而且还能逃跑,在这么大的树海中寻找一个人,大海捞针。

胖子本来满怀希望,这时候也颓然缩了起来,道:“你还不如说去抓他三叔,难度几乎一样。而且,说不定文锦还不知道那入口呢,小吴找到的那本笔记上不是说她没进入这里就回去了嘛。”

闷油瓶往篝火里丢了几根柴,道:“不会,她一定知道。”

“为什么?”

“我的感觉。”

胖子看了看我耸肩,就没辙了,叹了口气:“感觉,我的感觉就是这一次肯定白跑了。”喝了一口水,一脸郁闷。

几个人都不说话了,我靠在那里想了想,却感觉闷油瓶这么说还是比较有根据的。

按照事情的来龙去脉来推断,一切的源头都在那些录像带上,裘德考和我都收到了录像带,我们都通过不同的方式,得知了文锦若干年前的一次考察,从而促成了这一次考察。所以,文锦寄出录像带的目的,应该就是引我们来这个鬼地方。

我三叔此行的目的,是为了跟踪裘德考的队伍,搞清楚他们到底在追踪什么东西,查探这么多年来他们在华活动的真是目的,但是裘德考的队伍在进入魔鬼城之前就他娘的崩溃了,跟踪就失去了意义,以我三叔的性格,他会在和黑瞎子汇合之后,对着剩下的裘德考的人严刑逼供,问出裘德考此行的目的。

所以三叔可能得到的信息,应该是有限的,这种情况下看来,寄出录像带的文锦肯定是知道最多的人,没有理由三叔能知道的线索,文锦会不知道。

想到那些盘带子里,我心里有点不太舒服,那张和我一模一样的脸到底是怎么回事,如果真的抓到文锦,我一定要问清楚。不过现在不是考虑这些的时候。我对胖子道:“不管怎么说,文锦知道的概率比不知道的大的多,我觉得我们现在已经走投无路的情况下不应该去考虑这些,最困难的,应该是抓到文锦这件事上。”

胖子点起一只烟,抽了一口就道:“这不是困难,这是不可能,她看到我们会跑,就算她身上带着GPS,在这么大的地方我们也不一定能逮住她。”

“也许我们可以做个陷阱诱她过来。”我道。

“你准备怎么诱?色诱吗?”胖子没好奇道:“咱们三个一边跳脱衣舞一边在林子里逛荡?”

我叹了口气,确实麻烦,如果她是向着我们的,那我们一边叫喊,或者用火光什么做信号,总有得到回应的时候,两边互相修正方向,就可能碰上,但是问题是她见到我们竟然会逃,这是为什么呢?

我就郁闷道:“你们说,为什么她在峡谷口看到我们的时候,要跑呢?托定主卓玛传口信给我们的不是她吗?她当时在那里出现,应该是在等我们,为什么没有和我们汇合?难道她真的神智失常了?”

闷油瓶缓缓的摇头,说神智失常的判断是我们在看到她满身泥污的时候下的,现在知道她满身泥污是有原因的,那么显然文锦在当时看到我们的时候是极度冷静的。她逃跑是她根据形式判断的结果。

胖子不解。“这么说她逃跑还有理了,我们又不会害她,她跑什么啊。”

“冷静……逃跑……”我却听懂了他的意思,背脊冷起来。

文锦害怕什么?

在她的笔记中,她的口信中,都反复提到了她在逃避一个东西,这个东西被她称呼为“它”,而且,她告诉我们,那个“它”就在进入柴达木盆地的我们之中。那么,只有一个比较合理的说的通的可能性,我啧了一声道:“难道,文锦逃走,是看到那个‘它’,就在我们几个人之中?”

闷油瓶点头,“恐怕就是这样。”

我一下看向胖子,看向帐篷里的潘子,又看向闷油瓶,心说我靠,不会吧。

“当时在场的是,小哥,小吴,我,大潘四个人,这么说来,咱们四个人里,有一个人把她吓跑了?”胖子也看了看我们,“咱们中有一个坏蛋?”

我和闷油瓶都不做声,胖子立即举手说:“胖爷我可是好人,绝对不是我,我对你们那小娘们一点也不感兴趣。”

“这只是一个想法,也许并不是这样。”我对这样的说法感觉很不舒服,这里的每一个人都出生入死过,我宁愿相信文锦逃开是她疯了。

“关键问题是,那个‘它’到底是什么?”胖子道:“小哥,你也不知道吗?”

闷油瓶抬眼看了看他,摇头。

“会不会有人易容成我们几个样子,我们其中的一个是有人假扮的?”胖子问道,说着用力扯自己的脸皮,表示自己的清白:“你看,胖爷我的脸皮是原装的。”

“我想到过这一点,刚才你睡着的时候,我已经检查过你和潘子了。”闷油瓶道:“没有问题。”

我想起看到他的时候,他正蹲在潘子边上,原来是在搞这个名堂,看来他老早就想到这件事情,但是一直没有说出来。这人还真是城府深。

胖子就看向我:“那小吴呢?”

我立即拉自己的脸:“放心,绝对是原装的。”

“难说,你可是半路加进来的,说不定你就是假扮的。来,让我胖爷我检查一下。”胖子伸手过来,用力拉了一下,疼的我眼泪的出来,才松手,道:“算你过关。”

“所以,应该不是这方面的问题。”闷油瓶指了指我口袋里文锦的笔记,问我道:“这上面有相关的记载吗?”

我拿出来,就摇头,“能肯定的是,在文锦的描述中,这个‘它’是在追踪他们,应该是有智力的,而且我感觉,肯定应该是一个人吧,只是不知道为什么会用这个‘它’。”

胖子站起来,喝了几口水,把水壶递给闷油瓶道:“说起来,追踪他们的,不就是你三叔吗,会不会那个它就是你三叔呢?黑灯瞎火的,文锦看错了也说不定,你不就和你三叔有点像吗?”

我心说我帅多了,闷油瓶接过胖子的水壶,刚要说话,就在这时候,胖子忽然就一下伸手过去,去捏闷油瓶的脸。一下捏住用力一扯。

\chapter{第二夜:盲}

我被胖子的举动给惊呆了,花了好几秒才明白他想干什么。

闷油瓶检查了我们的脸部,但是他自己的脸部没有检查,胖子怕他玩这种心理游戏的手段,也要看看他脸上有没有带人皮面具。

闷油瓶纹丝不动,就坐在那里,看了胖子一眼,胖子就尴尬的笑笑:“以防万一,小哥,你也是四个人之一啊,他娘的小心使得万年船。”

闷油瓶喝了口水,也没生气,但是没理胖子,我就对胖子道:“你也不用偷袭啊。”

胖子怒道:“什么偷袭,我这是动作稍微快了点而已。”

我倒是习惯了胖子的这种举动,无可奈何的笑笑,胖子就坐了回去,大概是感觉挺尴尬的,转移话题道:“这下可以证明咱们四个人都是清白无辜的了,那现在看来,这个‘它’的含义,可能和字面的意思不同了,说不定不是生物。”

“怎么说。”我问道。

“它除了可以称呼动物外,也可以称呼物品,也许文锦逃避的,是一件东西呢?”

胖子总是有突发奇想,不过这个好像有点不靠谱:“东西?”我就道:“你是说,她这十几年来,一直是在逃避的,可能是我们的内裤或者鼻屎吗?”

“他娘的胖爷我说的东西当然不是指这些。”胖子道:“你们身上有什么东西,是和这件事情有关系的,都拿出来看看,说不定咱们能发现些什么。”

我摇头心说拿什么啊,那几枚蛇眉铜鱼我都没带来,闷油瓶突然皱起了眉头,道:“不对,说起物体,我们少算了一样东西。”

“什么?”

“阿宁。”

一下我就一个激灵:“你是说,尸体?”

这倒也有可能,我们陷入了沉思,却感觉好像没有直接的证据,不过阿宁身上发生的事情相当的诡异,也许真的有这层关系。

胖子却拍掌道:“哎呀,小吴,你还记得不记得昨晚我们在林子碰到的事情,该不是就是这样,这阿宁有问题,所以死了就变成那玩意了。”

我张了张嘴巴,心说我怎么说呢,这东西靠猜测根本证明不了,尸体也不在了,要说诡异,这里那件事情不透着邪劲。

想着我就受不了了,立即摆手道:“我看咱们我们不要谈这个了。现在前提都还没有明朗,说不定文锦确实是疯了也说不定,这个时候非要在这几个人当中找出一个来,我看是不太可能的,我们还是想想实际一点的东西,怎么逮到她比较现实。”

胖子就没兴趣了,站了起来,道:“想什么,我说了就是不可能的事情,铁定想不出来,有条狗说不定还能想想。你又没你爷爷那本事。现在实际的东西,是怎么过今天晚上,这些扯淡的事情别聊了。”说着就走去,提起他们挖来淤泥的筒子,就往潘子的帐篷去刷。

我看了看表,已经入夜了,天空中最后一丝天光也早就消失了,为了保险,确实应该先做好防护的措施,于是也过去帮忙。

我们把淤泥涂满帐篷,又在上面盖了防水布,以免晚上下雨。我去检查了一下潘子,他还在熟睡,体温正常,胖子告诉我醒过一次,神智还没恢复,就喂了几口水又睡死过去了。不过低烧压下去了,那几针还是有效果的。

营地中没有任何的火器,胖子捡了很多的石头堆在一边,说实在不行我们就学狼牙山五壮士,我说人家至少还有崖可以跳,我们丢完了石头就只能投降了。

胖子扇起了篝火,将火焰加大,然后把在营地四周的几个火点全点了起来,以作为警戒和干燥之用。红色的火光,照的通亮。做完这一切,已经近晚上10点,我刚稍微感觉有了点安全感,四周又朦胧起来,他娘的又起雾了。不到一个小时的时间,整个营地就没蒙入粘稠的雾气中,什么也看不清楚。

看着四周一片迷蒙,我感觉到冷汗直冒,已经完全没有能见度了,就算是火焰,离开两三米的距离也就看不清楚了,此时要想防范或者警惕,都已经不可能。

鼻子里满是混杂着泥土味的潮湿的味道,我想到这雾气是否有毒?昨天在雨林中,没法太在意这些事情,但是现在需要注意了,我听说雨林之中常有瘴气,到了晚上气温下降就会升起来,特别是沼泽之内,瘴气中含有大量有毒气体甚至重金属的挥发物,吸的多了,会让人慢性中毒,甚至慢慢的腐烂肺部。

想到这里,我就问胖子是否应该去摸那些帐篷的装备,想找几个防毒面具出来备用。

胖子道:“这绝对不是瘴气,瘴气的味道很浓,而且瘴气哪有这么厉害,瘴气吸多了最多得个关节炎,肺痨什么的,西南方山区多瘴气潮湿,那边人爱吃辣子就是防这个,你不如找找这里人有没有带着辣椒,咱们呆会儿可以搞个辣椒拌饭,绝对够味。”

我说:“别大意,这里和其他地方不一样,我看还是找几个带上的保险。”

胖子和闷油瓶开始往身上摸泥,这肯定是极其不舒服的过程,所以他语气很差,摇头:“要带你带,这种天气再带个防毒面具,他娘的撞树上都看不见。还怎么守夜。你要有空琢磨这些,还不如快点睡觉,等会儿说不定就没的睡了。”说完立即呸了几口:“乌鸦嘴,乌鸦嘴,大吉大利。”

我给他说的悻然,心理其实有点挺恨自己的,他们两个人守夜,潘子受了重伤,我却可以睡一个晚上,这简直和重伤员是同一个档次,这时候想是否自己来这里确实是一个累赘。

进帐篷躺下,我心说这怎么睡着啊,脑子里乱七八糟,身上什么地方都疼,因为外面和着泥,篝火光透不进来,用一只矿灯照明,为了省电也不能常用,就关了在黑暗里逼自己睡。听着胖子在外面磨他的砍刀,听着听着,真的就迷糊了起来。

那种状态也不知道是不是真的睡着,蒙蒙的,脑子里还有事情,但是也不清晰,一直持续了很久,就没睡死过去。在半夜的时候,就给尿憋清醒了。

醒来听了一下外面没什么动静,心说应该没事情,就摸黑撩起帐篷口准备出去防水。

一撩开我就惊了一下,我发现外面一片漆黑,所有的篝火都灭了。

这是怎么回事情?我立即就完全清醒了,缩回了帐篷,心说完了,难道出事情了?

可怎么一点动静也没有,刚才我没有睡死啊,我自己都能知道自己是在一种半睡眠的状态中,以闷油瓶的身手,能有什么东西让他一点声音都不发出来就中招吗?

我静下来听,外面什么声音都没有。就有点慌了,这时候不敢叫出来,立即摸回去,摸到我的矿灯,然后打开,但是拨弄了两下,发现不亮了,又摸着自己的口袋,掏出了打火机,打了几下,也没亮,甚至连一点火光都没有。

我暗骂一声,立即深吸了几口气,告诉自己冷静,心说怎么要坏都一起坏。收起来就想去打我的手表荧光。一收我却发现打火机很烫。

我有点奇怪,心说怎么会这么烫,刚才明明连个火星都没有,我又再次打了一下打火机,然后往我自己手心下一放,一下我的手就感觉到一股巨烫,立即缩了回来。

我楞了一下,心说打火机是打着的。

可是我的眼前,还是一片漆黑,一点光亮都没有。

\chapter{第二夜:影动}

打火机的存气苟延残喘,烧了一下肯定是迅速熄灭,但是问题是我看不到任何的火光,眼前就是黑的。

那一刹我完全没有反应过来,下意识就以为有什么东西蒙着我的眼睛,就用手去摸,摸到眼睫毛才发现不是,接着我就纳闷,心说这他娘的怎么了。

是不是这里的雾气太浓了?我打亮我的手表,贴到眼睛前去看。还是一片漆黑,而且我逐渐就发现,这种黑黑的无比均匀。

我还是非常疑惑,因为我脑海里根本没有任何这个概念,所以几乎是丈二和尚莫不着头脑,我用力挥手,想驱散眼前的黑暗,总觉得手一挥就能把那黑暗拨开。但是丝毫没有用处。

蒙了好久,我才冷静下来,仔细去琢磨这是怎么回事,外面一片漆黑,什么声音都没有,难道在我睡觉的时候出了什么事,把所有的光都遮了。

可这说不通啊,就这么近我却看不到光,想着想着,我慢慢的反应了过来,心里出了一个让我出冷汗的念头。

遮住光怎么也不可能啊,这种情形,难道——我瞎了?

我无法相信,我脑子里从来没有过这种概念,这也太突兀了。但是我的内心已经恐惧了起来,那种恐惧不同于以往任何一种恐惧,甚至远远超出对死亡的恐惧,我开始用力揉眼睛,下意识的用力去眨,一直到我眼睛疼的都睁不开才停了下来。

接着我就立即想到了潘子,爬过去推他,想推醒他问问是不是他能不能看到光,推了几下,发现他浑身很烫,显然在低烧又发了起来。摇了半天也没醒。

我坐下来心说糟糕了,深呼吸了几口,立即又想起了闷油瓶和胖子,如果我是真的瞎了,那么这是一种爆盲,爆盲肯定有原因,比如说光线灼伤或者中毒,人不可能无缘无故的就瞎掉。所以,很可能受害的不只我一个人。

假如他们没有瞎,只有我一个人受害了,那么他们可能就在帐篷外,只是没发出声音。我立即爬到帐篷边上,听了听外面的动静,轻轻叫了几声:“胖子!”

等了一会儿,没有任何人回应。

我叫的不算轻了,在这么安静的不可能听不到,除非他们两个都睡着了,但是闷油瓶绝对不可能睡着。

我的冷汗下来了,心说他们肯定也出事了,坐了回去,心里就想到几个小时前我们的推测,一下就毛了,心说难道这就是三叔他们遭遇的突变?

在这里扎营能把人变瞎?

脑子乱的马一样,根本没法理解,我们想到了无数种可能性,但是根本没有想过会这样。

在这种地方,对于一队正常人来说,这种突如其来的失明等于全员死亡,甚至比死亡更可怕。

我浑身发抖,脑子里闪过无数的画面,想到我在雨林中摸索,什么都看不见,又没有盲人对于听觉的适应,死亡只是时间问题,而且死亡之前我恐怕会经历很长一段极端恐怖的经历。

但是,到底是什么东西导致我失明的?吃的?压缩饼干我们一路吃过来都没事情,难道,是这座遗迹?

我还算镇定,这大概是因为我还是无法接受我已经瞎了的事实,就在这时候,忽然在帐篷外面,挺远的地方,传来了一个奇怪的说话声。

一下我打了一个寒战,立即侧耳去听,就听到那竟然是我们在雨林里听到的,那种类似于对讲机静电的人声,忽高忽低,说不出的诡异。

我的脑海里浮现出犹如蛇一样站立着的那个狰狞的人影,不由喉咙发紧。他娘的这玩意怎么阴魂不散。

发出这种声音的到底是什么东西?到底是不是阿宁?要是我的眼睛能看到,我真想偷偷看一眼,他娘的在这种时候我竟然瞎了。

不过这东西即使不是蛇,也必然是和那些蛇一起行动的,显然在这营地的附近,已经出现了那种毒蛇,当即我就脑子发紧立即想到了帐篷的帘子,刚才我有关上帐篷的门吗?我看不见不知道,我必须去摸一下。

想着立即去帐篷的门帘,我发着抖刚摸到,忽然从门口一下就挤进一个人,一下把我撞倒,我刚爬起来,立即就被人按住了,嘴巴给人捂住。

我吓的半死,但是随即就闻到胖子身上的汗臭了,接着一只东西按到了我的脸上。我一摸,是防毒面具。

我立即不再挣扎,带正了面具,就听到胖子压低了声音说道:“别慌,这雾气有毒,你带上面具一会儿就能看见,千万别大声说话,这营地四周全是蛇。”

我听了立即点头,胖子把我松开,我就轻声问道:“刚才你们跑哪儿去了?”

“儿子没娘说来话长。”胖子道:“你以为摸黑摸出几个防毒面具容易嘛。”

我骂道谁叫你不听我的,这时那诡异的静电声又想起了一阵,离我们近了很多,胖子立即紧张的嘘了一声。“别说话。”

我立即禁声,接着我就听到胖子翻动东西的声音,翻了几下不知道翻出了什么,一下塞到了我的手里。我一摸发现是把匕首。我心说你要干嘛,就听到了他似乎在往帐篷口摸。

我立即摸过去抓住他,不让他动,他一下挣开我轻声道:“小哥被咬了,我得马上去救他,你呆在这里千万不要动,到能看见了再说!”

我听了脑子就一炸心说不会吧,还没琢磨明白,胖子就出去了,我整个人就木在了那里。感觉到一股天旋地转。

先惊的是闷油瓶被咬了,胖子什么也没说清楚,但是那些蛇奇毒无比,被咬之后是否能救,我不敢去想。然后惊的是闷油瓶这样的身手和警觉,竟然也会被咬,那外面到底是什么情况。

一下我就心急如焚,真想立即也出去看看,可是他娘的却什么都看不见。这时候就想到一个不详的念头,万一胖子也中了招怎么办,他娘的我一个人在这里,带着潘子,实在是太可怕了。

那种焦虑无法形容,眼前一片漆黑,不知道到底需要多少时间恢复,外面的情形极度的危险。我摸着手里的匕首,浑身都僵硬的好像死了一样,心说不知道胖子给我这个东西是让我自杀还是自卫。

但是毫无办法,我什么都不能干,只能在原地坐着。听着外面的动静,一面缩着身子抑制身上打战的感觉。

就这么听外面还是什么声音都听不到,绝对想象不到外面全是蛇是什么样子,那静电一般的声音没有继续靠拢,但是一直时段时续。听距离,最近的地方在我们营地的边缘,但是它没有再靠近一步。

也不知道过了多久——我完全没有时间的概念,那段时间脑子是完全空白的——我稍微有点缓和下来,人无法持续的维持一种情绪,紧张到了极限之后,反而身子就软了下来。

逐渐的,我的眼前就开始迷蒙起来,黑色开始消退了,但是不是那种潮水一般的,而是黑色淡了起来,眼前的黑色中出现了一层迷蒙的灰雾。

我松了口气,终于能看到光了,我不知道怎么才能让他复原的快一点,于是不停的眨巴眼睛。

慢慢的,那层灰色的东西就越来越白,而且进度很快,在灰色中很快又出现了一些轮廓。

这可能有点感觉像重度近视看出来的东西,我转动了一下头,发现眼前的光亮应该是矿灯没有关闭造成的,我举起来四处照了一下,果然眼前的光影有变化。确实是我的眼睛好转了。

但是现在的模糊程度我还是没有办法分辨出帐篷的出口在什么地方,只能看到一些大概的影子。

我听说过毛泽东白内障手术复明之后老泪纵横,现在我感觉能深刻的体会到这种悲喜交加的感觉,很多东西确实要失去了才能懂得珍贵。就在我打算凭着模糊的视力去看一下潘子的时候,忽然我就看到,在我眼前的黑影中,有一个影子在动。

眼前的情形是非常模糊的,甚至轮廓都是无法分辨的,但是我能知道眼前有一个东西在动。我不是很相信我的视觉,以为是视觉恢复产生的错觉,就没有去理,一点一点朝潘子摸去。很快就摸到了潘子的手,温度正常了,我心里惊讶,竟然自己就退了烧了。也好,现在这个样子也没法给他打针。

去摸水壶想给他喝几口水,一转身忽然又看到眼前有什么东西闪了一下,这一次因为视力的逐渐好转,我发现在我面前掠过的影子的动作,非常的诡异,不像是错觉。

我楞了一下,就把脸转到那个影子的方向,死命去看,就看到一团模糊如雾气的黑影,看上去竟然是个有四肢的东西。

我起了一身的鸡皮疙瘩,心说难道这帐篷里还有其他东西,在我刚才失明的时候有什么进来了?

胖子?闷油瓶?但是他们不会不说话啊,我一下捏紧匕首。

一下那影子又动了,动作非常快,我就忍不住轻声喝了一声:“谁?”

那影子忽的就一停,接着动的就更快了,我看到它跑到一个地方,不停的在抖动,我的视力逐渐的聚拢,那动作越来越形象,我就意识到它在翻动一只背包,它在找什么东西,而且我就问到了一股沼泽淤泥的味道。

我心里立即就哎呀了一声,心说这人一定也抹着淤泥,是谁呢?想着,我慢慢移动身子,就想靠近过去看看。

还没扑呢,那影子又是晃动了,接着就站了起来,迅速移动,我反应不过来脑子转了一下,就发现他不见了。

这一切发生的太快,我有点摸不着头脑,心说难道还是我的错觉,一下想到电视剧中看到的,复明之后开始的时候视觉会延迟,难道我刚才看到的是胖子进来时的情形?

可几乎就在同时,忽然一亮一暗伴随着剧烈的气喘声,我就看到一个很大的重叠影子冲了进来,几乎是摔了进来,听到胖子气急败坏喘道:“关灯!关掉矿灯!”

我反应不过来就给他一下抢了去,灯一下关了,我的四周光线一沉,他立即轻声道:“趴下,安静,不管发生什么,都不要发出任何声音。”

我立即趴下,可以感觉到胖子也趴了下来,一开始还能听到他的喘气,但是能感觉到他在尽量的克制,很快他的气喘就非常微弱了,我正纳闷为什么要趴下,忽然我就听到“嘣”的一声闷响,好像有什么东西撞到了隔壁的帐篷下,撞得极重,紧接着,又是一下,能听到支架折断的脆裂声。接着就听到一声帐篷垮塌的动静,显然隔壁的帐篷被搞烂了。

我脸都青了,还没等我反应过来,我们的帐篷忽然就抖了一下,显然被什么东西差了一下。

我顿时觉得天灵盖一刺,马上抱头,以为下一击肯定就是这个帐篷。

但是没有想到的是,没有攻击打来,我这样抱头隔了几分钟,那剧烈的撞击声出现在比较远的地方。

我心说这到底怎么回事?外面是什么东西?刚想对胖子说我们还是跑吧,没张嘴就被胖子捂住了。

外面几下巨响,又是帐篷垮塌的声音,接着隔了几分钟,又是同样的动静,这样足持续了半个小时,远远近近,我估计足有十几个帐篷被摧毁,我们趴在那里,每砸一下心就停一下,那煎熬简直好比是被轰炸的感觉,不知道那炸弹什么时候会掉到我们头上来。

一直到安静了非常长的时间,我们才逐渐意识到,这波攻击可能结束了,慢慢的,也不知道是谁第一个反应过来,我们都坐了起来,我就发现我的眼睛基本上已经恢复了。虽然还有些糊,但是能看到色彩和人物的轮廓了。

后来摸了一下,才发现剩下的模糊也是因为防毒面具镜片上的雾气,擦掉之后都清晰了。

我就看到胖子和闷油瓶,闷油瓶身上受了伤,捂着腕口,胖子浑身都是血斑,两个人浑身是淤泥,狼狈的犹如刚从猪圈里出来。显然昨晚经历了一场极度严峻的混乱。

我们还是不敢说话,等了一会儿,胖子就偷偷的撩开帘子,一撩开忽然就有光进来,原来是天亮了。

接着他就小心翼翼的走了出去,我问了问闷油瓶,他摆手说没事情,也紧随其后的探了出去,我跟着。

雾气退的差不多了,晨曦的天光很沉但是已经可以看到所有的东西,我出来转头一看,整个人就惊呆了。

我们四周,整个营地全部都垮了,所有的帐篷全部都烂了,好像遭遇了一场威力无比巨大的龙卷风似的,若大一片地方,只剩下我们一个帐篷孤零零屹立在那里。四周什么都没有,没有袭击我们的东西,没有任何的蛇的痕迹。

胖子骂了一声,坐到已经基本熄灭的篝火边上,我目瞪口呆无法做出反应,这时候身后一声肢体摔倒的声音,我回头一看,闷油瓶晕倒在了地上。

\chapter{黎明:转移}

我们将他抬回进帐篷,我立即检查了他的伤势,让我松一口气的是,我发现他被咬的地方是手腕,有两个血洞,但是伤口不深,显然他被咬的一刹那就把蛇甩脱了,这种伤口都会出现,昨天到底是如何的惊心动魄我可能无法想象。

胖子对我说,已经第一时间扎了动脉,又吸了毒血出来,还切了十字口放血,但是毒液肯定有一些已经进入进去,这蛇太毒了,就这么叮了一下手立即就青了,好在小哥动作快,就在那一瞬间就捏住了蛇头,那蛇没完全咬下去,不然估计小哥也报销了。

我给闷油瓶注射了血清,给他按摩了一下太阳穴,他的呼吸舒缓了下来,我捏了他的手,发现整体的浮肿并不厉害,就对胖子道应该没事,这陆地上的东西再毒也没海里的东西毒,只是不能让他再动了。

潘子还是躺着那儿,我们把闷油瓶也放好,看着一下躺了两个就头疼,也亏的是他们,要是我早死了,这地方他娘的真的和我们以往去的地方完全不同,这两个人经验丰富都搞成这样子。

接着,我就问胖子到底昨天发生了什么事情?

胖子说的和我推测的也差不多,道昨天他们守夜的时候,逐渐逐渐的就发现自己看不见了,胖子就想起了我的话,一下意识到可能这雾气真的有毒,立即就去找防毒面具,但是找来找去找不到,眼看就完全看不见了,他急的要命。

他和闷油瓶就先用淤泥弄湿毛巾捂住鼻子,这还真有效果,后来他们在其中一个帐篷中找到了几个,刚想带上,闷油瓶眼睛看不清楚就被躲在背包里的蛇咬了一口,好在他反应极快,立即凌空捏住蛇头,但是还是被叮了一下,立即手就青了。

但是因为注射的毒液量有限,闷油瓶没有立即毙命,他们简单处理了一下,这时候胖子听到我在叫,立即就带了防毒面具先到我这里来,在帐篷外面就发现,不知道什么时候,从四周建筑的缝隙里,出现了大量的鸡冠蛇,这些蛇全部躲在缝隙中,即不出来也不进去,就看到那些缝隙里面全部都是红色的鳞光,似乎是在等待什么。

所以他立即回来给我带上防毒面具,然后再返回照顾闷油瓶,将他扛回来,在中途,他就看到了一副奇景。

无数的鸡冠蛇从缝隙中涌出来,逐渐盘绕在了一起,组成了一陀巨大的“蛇潮”,好像一团软体动物一样,有节奏的行进,动作极其快,好像海里那种巨大的鱼群……

胖子道:“这肯定就是他们运送阿宁尸体的办法,你胖爷我还想不通他们是怎么做到的,那蛇潮简直就感觉是一只整个头的生物。”

我奇怪道:“那它们为什么要把这里破坏成这样?”

胖子道:“它们肯定是能知道我们的存在,但是因为帐篷上有了淤泥,它们找不到我们,这些到底是畜生,最后就采取了这种方式。”

我听了乍舌,胖子就立即道,我们不能再呆在这里了,今天晚上他们肯定还会来,我们必须走。而且离这里越远越好。他问我能看清楚了没有?我点头,他就让我马上去收集这里的食物和物资,点齐之后打包。到中午的时候看他们两个的状况,再决定去哪儿。

我苦笑,但是也知道这是必须要做的,但是现在不知道蛇走干净没有,所以先休息了一下,等太阳了出来了,才开始翻帐篷的废墟,把其中所有可以用的都拖出来,他在这里照顾他们两个。

收集的最主要的东西就是食物,我找到了大量的压缩饼干,都堆在一个袋子里,后来又幸运的在其中一个中发现了罐头。

有车的时候阿宁他们也带着罐头,不过因为要探路,罐头太重都轻装在峡谷外了,进林子以来一路过来都是吃轻便的压缩干粮,吃的嘴唇都起泡了,没想到三叔他们还带了这好东西,真是不辞辛苦,不过,带这么累赘的东西,不像三叔的性格。

野战罐头非常接近正常食品,一般都是高蛋白的牛肉罐头,金枪鱼罐头或者是糯米大豆罐头,这些东西吃了长力气而且管饱,不容易饿。

我忙招呼胖子问他要不要,胖子一看就摇头说怎么带,不过我们可以立即把这些都吃了,看了看罐头的种类就流口水:“圣母玛利亚,你三叔他娘的真是个爷们,够品味。”

我继续搜索,找干粮和其它,还有容器,我们需要东西装水。

翻出一只背包的时候,我就发现里面有一张他家人的照片。这人我没见过,是一个大概三十出头的中年人,他老婆抱着孩子靠在他身边,照片拍的很土,衣着也很朴素,但是看的出他相当的幸福。

我就有点感慨,心说这人也不知道怎么样了,要是死在这里,他老婆孩子怎么办?干这一行的人,生生死死太平常了,何必要去耽误别人。

又想到阿宁死在了路上,还有乌老四,和那些在魔鬼城里死掉的,这些人真的是不知道为什么死的?一想就想到自己,不由自嘲,他娘的,如果当时不跟队伍过来,我现在应该在我的铺子里上网吹空调,有脸说别人也不想想自己。

收集完了,我边清点边心思万千,全部打进包里做好已经是中午了,胖子想立即开路,但是闷油瓶和潘子的情形都不是很好,潘子一直意识模糊,都没吃过东西,我们喂了水从他嘴唇下去,闷油瓶意识清醒但是身体乏力,也站不起来,但是手上的青色已经褪去了。

这下子就非常麻烦,我们不可能背着两个人又带着这么多东西离开这里太远。

闷油瓶就指着一边的神庙,虚弱道:“到里面去,离水源远一点!”

我们一想也是,这些石头的缝隙下全是水,和沼泽相连,难怪这些蛇全从缝隙里出来。

现在也没有别的办法,我们先把东西往里面运。

神庙完全坍塌,只有一个大形,连门都不知道在哪里,我们随便找了一条回廊进去,就发现其内的空间还是相当大。这建筑本来应该有两层,地下的一层破坏严重,但是上面一层还能看到当时的结构,都是黑色的石头累的,不高但是之间有很多非常精致的石柱。两层之间本来不知道靠什么通途,但是现在坍塌下来的东西已经成为了一条陡坡。

我们爬上去,进入到一间基本完好的石室内,能看到下面的营地,放下东西,东西搬完之后,就把潘子和闷油瓶也抬了过来。不过此时他基本上已经能走动了。

太阳犹如催命的魔咒,我感觉时间非常快,昨天的恐惧和梦魇还没有消退,等我回过头来,又是西晒太阳了,黄昏马上就要到来。

白天一天就基本上没有任何的休息和停止,我看着树影狰狞起来,就觉得一股无形的压力逼来。

绝望,这真的有一丝绝望的感觉。有个声音好像在我脑海里问,顶了两晚,今晚能继续熬过去吗?

\chapter{第三夜:浮雕}

我们几乎把所有能用的东西都搬了过来,还准备了几桶淤泥。不敢点大篝火了,做了一个小碳堆,晚饭胖子煮罐头也不敢在里面煮,把灶台搭在废墟外面。

我们估计那些蛇肯定会在雾气弥漫之后开始活动,所以黄昏的时候并不慌,我帮胖子烧饭,闷油瓶在上面看着帮我们望风。

但是胖子动作很快,我其实帮不上什么忙,开完罐头就在边上发呆。

胖子最烦我这个样子,他说我就是个林黛玉,整天不知道在琢磨东西,这人世间的东西哪有这么多好琢磨的,没心没肺的活着也是蹬腿死,你机关算尽也是蹬腿死,反正结局都一样,你管他妈的中间那个羁绊干什么。

我听了有点意外,胖子竟然会用羁绊这个文绉绉的词,一回味才发现他说的“鸡巴蛋”,不由苦笑。

正琢磨着,就听到胖子叫我:“我说天真,你看小哥这是干什么?”

我收回神,抬头看到神庙内的闷油瓶正在用什么东西擦上面的石壁,就叫道:“怎么了?”

闷油瓶没理我道,继续干着,也有可能是没听见。

我这里的事情已经做的差不多了,也来了兴趣,放下罐头刀就爬了上去。从神庙的回廊绕到他的身边,就看到他正在用篝火的里的碳抹墙壁,好像是想拓印什么东西。我问他干嘛,他指了指边上的石头,“我刚发现的。”

墙被涂黑了一大块,我用嘴吹了一下,发现这些石壁上,有着已经几乎被磨平的浮雕。

“在日光下基本上看不见了,只有涂上碳粉,才会有阴影出来,还能分辨一下。”他道,说着又从篝火中拣出一块来涂抹。

黑色的碳黑抹上岩石,光影变化,我晃动了一下,找了一个合适的位置,石头上的浮雕显现了出来。第一眼我便看到了大量的蛇。很难分辨了,光影攒动,蛇影飘忽好像是活的一样。

闷油瓶继续涂抹,我们就看到了一幅幅古老的浮雕出现这里的岩石上,这么多年下来,但是依然形神俱在,在闷油瓶的涂抹下如同魔术一般浮现了出来。

他涂完后就站立不稳,我立即扶住他,看了一遍,他就道:“这里讲的是那些蛇的事情。”

“讲的什么?”我问道。因为我不是能看的很清楚。

“一下子没法看懂。”他道:“得慢慢琢磨。”

对于这些我很有兴趣,而且一路过来也实在没看到多少关于这里的历史遗存。对于这里一无所知就是我们现在这种境况最直接的原因,所以我移动身子寻找着最好的角度,下了功夫仔细去看。

一幅一幅看过来,全部都不知所云,根本不知道是什么意思,浮雕上表达的东西很多,有的似乎是祭祀,有的又似乎是一场仪式,要说还真说不出什么来。

半猜半琢磨的看着,感觉有几幅似乎是在说这里的先民,供奉着这些带着鸡冠的毒蛇,他们将一个一个陶罐丢进一些孔洞里,好像就是路上看到的那种带着方孔的石塔,大量毒蛇开始钻入破碎的陶罐。有祭祀在主持仪式,很多人跪在四周。

原来这些祭品祭祀的就是这里的蛇,难道这里的人把这种毒蛇当成神了吗?不过,这倒不稀奇,毒蛇崇拜非常普遍,古人不知道毒蛇的毒性,只知道被咬一口后就会死去,看着这么小的伤口致死人命,都会认为这是魔力所致。中国少数民族里有很多都崇拜蛇。

这些鸡冠蛇可能喜欢食用尸鳖王的卵,不过尸鳖王的卵应该毒性剧烈,这蛇和尸鳖到底哪个更毒一点?

闷油瓶移动身形,边上的浮雕,是很多拿着长矛的人物,和先民打扮的厮杀在一起,很多人的身体被长矛刺穿了,似乎是一场战争。

战况看来对西王母国这一方不利,因为西王母的人数显然比对方要少的多。而西王母国全部都是步兵,对方的队伍中还能看到骑兵。敌方的统帅在队伍的后面,坐在一辆八批马的拉的车上。浮雕里不见西王母的身影。所有的浮雕造型精致,连五官都有细致的琢磨,惟妙惟肖,显然出自顶级工匠的手艺。

“这是,战争……”闷油瓶喃喃道。

“看来西王母国被侵略了,而且对方是一只比较强大的文明,有可能是楼兰或者北匈奴。”我道。“这些人看不出服装的款式,不过兵器的样子形似中原,应该是楼兰的军队。这个在战车上的,应该是楼兰王。”

说完我感觉很有道理,但是闷油瓶却没有注意我的话,而是用手摸那个战车上的统帅,皱起了眉头。

我心说怎么了?他忽然抬起手指着那敌方首领,对我道:“我认识这个人。”

\chapter{第三夜:似曾相识}

“啊?”我愣了一下。心说你认识他,他是你二大爷?

闷油瓶随后说了一句,我立即意识到自己理解错了,他道:“这八匹马,这个人是周穆王。”

“周穆王?就是写《穆天子传》的那个?”

穆天子的传说我也十分熟悉,在来之前那批人经常提到,穆天子传说主要记载周穆王率领七萃之士,驾上赤骥、盗骊、白义、逾轮、山子、渠黄、骅骝、绿耳等骏马,由造父赶车,伯夭作向导,从宗周出发,越过漳水,经由河宗、阳纡之山、群玉山等地,西至于西王母之邦,和西王母宴饮酬酢的事情。

也就是说,他是坐着八骏马来西王母宫旅游的,当时被西王母盛情款待。

不过从浮雕看来,这穆天子不像是来旅游的,难道传说有误,当年周穆王确实来了西王母国,不过是来打仗的?

我立即继续看浮雕,下一副画就让我倒吸了一口凉气,只见周穆王的军队杀进了一座宫殿,画面上出现了很多的蛇头人身的女人,她们将一种东西倒入了那种塔的孔里,接着无数的鸡冠蛇从塔里爬了出来,和周穆王的军队撕咬在一起。

看到这里,我立即明白了雨林中这些石塔的意义:“看来,当年周穆王确实进攻过这里,但是被这里的毒蛇打败了,可能为了掩盖自己的失败,他编了那个故事,这些毒蛇保护了西王母国,难怪他们会把这种蛇当成神一样来饲养。这好比满族人不杀乌鸦一样。”

想了想又觉得不止这样,看这些石塔下面肯定是相通的,蛇就生活在城市下面的通道里,被人们用人头来喂养,而遇到危险,就用某种东西把蛇引出来迎敌,这是设计好的防御方式,这种蛇这么毒,速度又快,谁也挡不住。

也就是说,人生活在城里,蛇生活在城市的下面,现在人全死了,蛇就到地面上来。这西王母的文明和亚马逊人比较像,那边用食人鱼防御敌人和猛兽,他们也祭祀食人鱼,用活人和活动物,这里用人头。

再往后看去,越看越复合我们的推论。我脑子对于这里的概念也逐渐清晰起来。

正暗爽间,闷油瓶的视线就停了下来,看到了这块石壁最中心的部分,这里的浮雕着一副巨大的原型图案,显然是整片岩石石刻构图的中心部分。上面雕刻着一条巨大的蛇被许多小型鸡冠蛇包围住,互相搏斗的场景。其中那条巨大的蛇缠绕在一根巨大的树木上,鸡冠蛇犹如装饰花纹一样缠绕在它四周。

“这是那种双鳞大蟒和这里的鸡冠蛇在打斗,看来在西王母时期,这里已经有两种蛇了,这种双鳞大蟒可能是这种鸡冠蛇的天敌。”我道。

闷油瓶摸了摸石刻,就摇头:“不对,这是交配。”

“交配?”我愣了一下,有点无法理解,想了想才明白他的意思:“你是说,鸡冠蛇和这条双鳞大蟒在混种交配?可是,这是两种完全不同的蛇啊,而且体型相差这么大,怎么交配啊。”

“你知道什么是老鸨吗?”闷油瓶突然问我。

“老鸨?”我莫名其妙,心说他怎么突然问这个了:“老鸨就是开妓院的啊。”

“那是一种戏称,老鸨其实是一种鸟,古时候有人发现,老鸨这种鸟,只有雌鸟,没有雄鸟,它们要繁衍后代,可以和任何其他品种的鸟类交配,为万鸟之妻,所以人们就用这种来代称人尽可夫的妓女。”闷油瓶淡淡道。“然而,事实上古人对于老鸨的说法是完全错误的,老鸨其实是有雄鸟的,但是,这种鸟类,他们的雌雄个体差异太大了,雄鸟比雌鸟大了好几倍,所以就被误认为是两种不同的鸟。”

我听懂了他的话,立即明白是什么意思,“这么说来,你认为这两种蛇其实就是一种蛇,只是两种性别有两个体型而已,那你说哪一种是雄蛇,那一种是雌蛇?”

“按照数量分析,小的应该是雄蛇,大的是雌蛇,不过,也有可能相反。”他摸着岩石的表面,忽然就伸出了奇长的手指,去摸双鳞巨蟒缠绕着的那刻巨树图案。“奇怪。”

“怎么了?”我也去摸,但是摸不出所以然来。就看着他,他皱起了眉头,忽然后退了几步,拿起碳把浮雕上方下方的没有涂抹的部分也涂了起来。

很块整块石壁上的浮雕整体就呈现了出来,我也退后了一步看,就在看到全景的一瞬间,我就张大了嘴巴,怔住了。

只见浮雕上,那条双鳞巨蛇缠绕着的巨树,拉远来看,更本不是什么树,而是一条盘成了一个圆形的更加巨大的大蛇,这条蛇因为太大了,那条双鳞巨蛇和他比起来简直就像筷子和擀面杖,而那些鸡冠蛇简直就是牙签了,所以看局部的时候,根本看不出那是蛇。

“这……这是什么东西?龙吗?”我咋舌道。那双鳞巨蛇已经极大,这蛇比它还要大这么多,那不是简直和解放卡车一样的直径,这种东西还能算是蛇吗?

闷油瓶怔怔的看着,不说话,一边用手一条一条的去摸那些鸡冠蛇的花纹,摸了好一会儿,才道:“你看,这些小蛇并没有盘绕在这条锦蟒上,它们只是拥簇在锦蟒上,帮助它不滑下去,真正在交配的,是这条锦蟒和这条巨蛇……”

我立即去看,看他顺着那些小蛇盘绕的纹路摸去,果然发现小蛇只是互相缠绕,并没有盘绕在锦蟒身上,而锦蟒则紧紧的盘绕在巨蛇上,我惊讶着,忽然就意识到了什么,吸了口冷气道:“我的天,胖子说对了!”

\chapter{第三夜:蛇母}

“社会性蛇群。”

一瞬间我脑子里闪出了这么一句话,同时就想起了胖子在昨天随口说了一句话,他说这些蛇的举动很想蚂蚁,说这里可能会有一条蛇后。

我当时认为这是不可能的事情,这些蛇这些协作的举动最多只是像秃鹫争食一样的群体本能的体现,没想到,在这里竟然看到了这样的浮雕,这简直就是动物社会习性的一个模型。

这些红色的小型鸡冠蛇,就是社会性昆虫中工兵的角色,数量众多,而那些锦蟒,就是雄蛇,体型大,数量少,而这条巨大的犹如龙一样的蛇,就是胖子说的“蛇后”,这里唯一的蛇母。从这浮雕来看,这条蛇母实在太巨大了,以至于雄蛇没法和它顺利交配,需要这么多的鸡冠蛇来辅助。而且按照自然规律,如此巨大的蛇母恐怕也无法运动,确实需要别人辅助交配,就好像被豢养的一些巨型母猪一样。

难道,这在这片树海的深处,真的有如此巨大的蛇吗?

我对于蛇的历史颇了解,我脑海里的记忆中,关于巨蛇的传说中,最大的是在巴西的雨林里,有人声称看到过一条50米长的巨型森蚺,蛇这种生物和人类不一样,它没有固定的极限寿命,一般的蛇会在体型大到无法捕猎食物的时候自然死去,但是在某些食物充足的情况下,蛇可以一直长下去,那些巨蛇简直就是雨林之神。不过,即使如此,那些蛇的死去时候的年龄也只有100年左右,这浮雕在这多久了,少说有三四千年了,如果这里真的存在过这条蛇母,也应该死去了。

而且如此巨大的身躯,如果它曾今存在,也必须是生活在水里,这里的沼泽显然没有这么大的浮力。

我看着都有点发怔,如果是在博物馆中看到这些浮雕,那么我们可能会以为这是古人的夸张或者神话,但是我们在这里遇到过了这些毒蛇,而且亲眼看到了它们诡异的举动,那么,这浮雕极有可能描绘的是真实的场景。那这可能是生物学,历史学,考古学甚至于社会学方面的巨大发现。

看着,这浮雕的情形实在让我们无法释怀,这种蛇诡异的行为到底是怎么进化出来的?为什么会和其他蛇类完全不同?我感觉到其中肯定还有更深的原因。这些原因肯定和西王母国的历史有关。

之后的浮雕,是一连串膜拜的场景,在一座神庙中,很多人对着一条毒蛇跪拜,看这神庙的轮廓,显然就是我们所处的地方,往下数去,在沼泽没有把这里淹没前,这座神庙有五层这么多,现在淤泥把下面的三层全部埋住了。在神庙的神台上,那蛇挺立着在众人之前,这应该也是祭祀的场景之一,除了蛇的奇怪动作,其他并无诡异的地方,神台是在神庙正门的前方,我们来的时候那里只有乱石,显然完全坍塌了。

我们从整体来看,就发现巨石的表现手法,中心是蛇的生殖场景,四周是对于蛇的祭祀,蛇的饲养,和蛇与人的战争,以及很多其他关于蛇的场面,正如闷油瓶所说,这是一块记述蛇的信息的石壁。

我还想再从其中得到一些信息,然而看了几遍,发现能仔细辨认更细节的部分实在很少,再也没有任何收获。边上的石壁也没有了浮雕。

我们的心神收了回来,这时候才听到胖子声音从远处传来,骂道:“你们两个卿卿我我的干什么呢?有完没完,老子叫了几遍了,你们到底要不要吃饭?”

我们意犹未尽,但是见一下子暂时没有了线索,肚子也叫了起来,食欲一下战胜了求知欲,只好暂停。

我扶着他,爬下去走到灶边,已经闻到了一股久违的肉香。胖子用一只脸盆当锅子,吊在篝火上烧烤。

胖子就问我们在那里到底在干什么呢,真把他当厨子了,也不来帮个手。

我把我们刚才发现的东西和他一说,他也颇为吃惊,不过也甚为洋洋得意,道:“伟大的头脑总是可以做出正确的决定,你们要吸取教训,以后一定要听从我的教导,这样才不会后知后觉……不过,如果那蛇母真的死了,为什么那些蛇还在收集尸体,他们收集尸体给什么东西吃呢?”

“可能是在喂食那些锦蟒一样的雄蛇,你还记得不记得,我们昨天晚上,找到阿宁尸体的时候,那条锦蟒四周有大量的野鸡脖子,显然是在保护这条锦蟒,这种雄蛇也是贵族阶级,会被蛇群供养,这些蛇的体型还可以继续生存,但是蛇母就绝对不可能存活,这里的食物太少了,真有这么大的蛇在近代活动,我们也应该会看到一些痕迹,所以我看在千年钱这条巨蛇已经死了。”我道。

这样说胖子才点头,我对他道现在可以想象,这些蛇并不是居心叵测的蛇魅,它们的行为同样是在按照本能办事,多少能放心一点。

他叹气说,“也只能稍微放心一点而已,这事情的疑点还很多,今天晚上也不知道怎么过,快点吃,吃饱了好打仗。”

我肚子饿坏了,不想再讨论这些,就问他煮了什么吃?

“我把罐头都煮了,只剩下一点,午餐肉炖馒头加沙丁鱼,大杂烩,不过味道没的说。”胖子就道:“得,别说这些蛇了,听了倒胃口,来尝尝胖爷我的手艺,第一口不要钱,第二口开始,一口一个明器。”

“煮这东西要什么手艺,不就是放水煮吗?”我道。

“啧啧,所以说你比你们家三爷档次低多了,只能一辈子当个小贩。”胖子不以为然,我饿的肚子都叫了,马上用空罐头舀了一碗,吃了一大口,烫的我直流眼泪,不过确实好吃,那味道有点像年糕,至少像是顿饭了。

我一搅动香味出来,胖子也没法摆谱了,不和我们废话,三个人一通风卷残云,把底糖都喝了个赶紧。

吃完浑身发汗,身上顿时有了力气,膝盖也不酸了。

“怎么样,不错吧,你们学着点,人活七十古来稀,吃喝嫖赌,只有吃是人一辈子的享受,你胖爷我过的可是刀口上的日子,咱们这种人,能享受的时候就的享受,指不定这就是咱们最后一顿了。”

“我呸!”我怒了:“什么吃喝嫖赌,你他娘才最后一顿,别把我们扯进去。”

这个时候说这个太不吉利了,因为这确实有可能是最后一顿。

“你瞧你瞧,这就是封建阶级的封建遗毒。”胖子做了个很欠扁的表情。不过接着就道:“这些东西有劲道,昨天我们眼睛都被那雾气迷了,吃点补一下,否则容易老下病根。”

我想起昨晚的雾气,就奇怪道:“对了,为什么我们在林子就没事,在这里就瞎了?”

胖子道:“我觉得可能是这里的水的问题,雾气都是水汽凝结的,在林子里的水都是活水,但是这里下面的积水可能是死的,具体的情况,咱们也不知道。”

我点头,又想起复明的时候看到的影子,就问他们是不是也有这种现象,一说胖子就摇头:“我们经历的情况比你复杂多了,哪有心思注意这些,你听谁说的?”

“电视剧里有讲过。”

“那玩意你都信。”他摇头,忽然就看到闷油瓶抬起了头,皱起了眉头,看向我。

闷油瓶从刚才开始就没有在听我们说话,我以为他还在想浮雕的事情,对他道:“别想了,兵来将挡,水来土掩,等一下我们再去仔细看看浮雕,找找其他线索,现在你就安心休息吧。”

我话还没说完,他就突然道:“你看到了一个黑影在翻背包?”

我给他吓了一跳,点头道:“很模糊,没看清楚,也不知道是不是错觉,但肯定不是你们两个。”

闷油瓶忽然就站了起来,对我道:“那是文锦。”

\chapter{第三夜:捕猎}

“啊?为什么?”我反应不过来。

他没回答我,想了一下,忽然对我道:“跟我来!”说着立即就往外跑。

我看了看太阳又下去了一点,心说要给他玩死了,立即跟去,他跑到原来帐篷的地方,从其中一个帐篷里找到一只防水袋,一下又跑过去,顺手拿了放在石上的几个刷牙杯,又直接抄起一只矿灯,就往林子跑去。

我跌跌撞撞的跟在后面,就见他几下就跑到和丛林交接处的沼泽里,立即跳了下去,用那杯子去挖沼泽底下的淤泥,倒进放水袋里,又抹在自己身上,我看的呆了,他对我一招手,我点头立即也跳了下去,还没站稳,一杯子泥就拍在我的脸上。几秒后两个人在淤泥里抹成和当时看到文锦一模一样。

我本想到起雾的时候再抹,因为裹着淤泥实在不舒服,心中不爽问他干嘛,他道:“抓文锦。”

“抓文锦?”

“她在找食物,她的食物耗尽了,所以她今天晚上必定还会来,我们要设一个埋伏。”

“晚上?埋伏?”我立即摇头:“我不干,伏下去就永远站不起来了。”

闷油瓶就看着我,忽然就道:“你为什么要来这里?”

我楞了一下,他冷冷的看了我一眼,爬上了水潭,头也不回的走了。

我楞在水潭里,感觉到心里极度的不舒服,心说你瞪我干什么?我来这里还不是因为你们什么都瞒着我,我为什么要来这里?我他娘的——

想着我就也不舒服不下去了,我知道他的意思,怕死已经晚了!我骂了一声,也爬了起来。

回去和胖子一说,胖子也有点犹豫,昨天的情形太骇人了,他觉得是否会有些冒险,但是仔细一说,胖子就答应了。

这事情的兴致就变了,一下子我们从晚上尽量活下来,变成晚上尽量找死,但是胖子道不会,文锦也不是傻的,她应该在雾没起来,或者刚起来的时候出现,甚至我们不在营地附近,她应该是天一黑就过来,如果真如小哥推测她在找吃的,那么她可能已经饿的不行了。

闷油瓶让胖子再烧半锅子汤,做成是没吃完的汤底的样子。胖子立即动手,让炉灶烧的更旺,很快,又一锅杂烩火锅就烧成了,香气四溢。闷油瓶提着淤泥就到潘子的边上,用泥往他身上抹,把他也用泥覆盖起来。接着是胖子。

全部搞完,闷油瓶提起锅子,让我们两个跟上,我问道潘子怎么办?他道:雾没起来之前我们就会回来,三个人去,抓到的几率大一点。

三个人一路走到原来的帐篷处,闷油瓶就把那锅杂烩放到昨天我们的篝火处。

此时天色还早,我们三个找了个隐蔽处蹲下来,我就只感觉要笑,这事情有点扯淡,拿着锅汤勾引文锦,文锦又不是猫。

我们蹲在那里,一直看着太阳从树线下去,四周的黑暗如鬼魅一样聚拢,什么都没有等到,连汤都凉了,胖子实在忍不住,想问他话,却给都他摆手制止住,然后指了指耳朵,让我们注意声响。

我们凝神静气,听着周围的动静,浑身的泥巴又臭又黏糊,弄的我难受的要命。特别是脸上和腰部的部分,因为热量高干的块,这些地方的皮都扯了起来,痒的要命,但是又没法去抓,抓了更痒而且干的更快。

就这么咬牙一直等着,一直到天蒙黑只剩下一点天光的时候,我都已经进入到恍惚状态,忽然,身边的人就动了,我立即清醒,绷紧了身子,甩了甩头,跟着他们偷偷从石头后面探出头去,在非常黯淡的光线中,就看到一个浑身淤泥的人,从林子里小心翼翼的走了出来,看身材,赫然是一个女人。

“真的是文锦!”我喉咙一紧,心说还真管用。还没来的及细琢磨这来龙去脉,闷油瓶的手已经推在我的肩膀上。把我拉了回来。

我看向他,他就对我和胖子说了一个手势,意思是,只要他一动,我们两个立即从营地的两面包操过去,一定要堵住她。

此时也不知道闷油瓶到底在搞什么鬼,我们点头,耐心的等着,这埋伏的感觉相当刺激,我的心狂跳,一直等到我们听到了那只汤筒的动静。

胖子就想出去,但是闷油瓶没动,他不动我们就也没动,等了大概十分钟,闷油瓶闭了闭眼睛,突然一个翻身就从石头后面窜了出去,几乎就是同时,我们听到一声惊讶的叫声,接着就是转身狂奔的声音。

我和胖子立即撒开腿,从左右两边一下冲出去,然后绕着营地一下围了过去,从几个帐篷中间冲过去,三个人同时到位,一下就把她围了起来。

文锦显然惊慌失措,人不知所措的在我们三个中间转圈,满脸惊恐。

接着火光,这一下我才清晰的看到文锦的脸,在淤泥中看不到真是的情况,但是我却可以肯定,她极其的年轻,简直就好像是十八九岁的小姑娘,即使是在这种情况,我还是能知道,这女人极其的清秀,远远超过那张照片。

这几乎是一次超越时空的见面,如果是在正常情况下,我几乎会感觉她是从那张照片里走出来的,然而现在我根本没有闲心雅致来想这些。

文锦显然被我们吓坏了,有点不知所措,一边到处看,想找空隙逃出去。

“不要怕,陈……阿姨。”我想说话来安抚她,但是说了一句,发现实在很难叫的出口。

文锦一下看向我,突然就朝我冲过来,我张开双臂,想一把抱住她,将她制服住。没想到她突然一矮身子,一下扭住我的手臂,将我整个人扭了过来,我疼的大叫,她一推就把我推的趴到帐篷上,几乎把帐篷压塌,自己狂跑进了浓雾中。

我爬起来,就看到胖子和闷油瓶已经狂追了上去,心中暗骂自己没用,立即也跟了上去。

\chapter{第三夜:暗战}

文锦跑在最前面,我根本已经看不到了,我追的是胖子的背影,在这样的光线下追人,连一步都不能落下,否则,一闪你就看不到了。

这一次绝对不能给她跑了,我心里道,我们有太多的疑问需要问她。

跑到营地外,还没有进丛林的宽阔地带,在这种地方,闷油瓶速度极快,一下将她逼到一快巨石附近,我们三个又将她围了起来,她靠在巨石上,似乎已经无路可逃,只听到她喘气的声音。

“大姐,你到底在怕什么?”胖子就问道:“我们是好人,别逃了,搞的我们和日本人追花姑娘似的。”

文锦突然叫了一句,我没听清楚她叫的是什么,她忽然转身几下就爬上巨石,她的动作极其轻巧,显然是练过功夫的,竟然没有一丝的迟缓。

我们中几个只有闷油瓶能跟上去,他立即翻了过去,一下就从后面抓住了文锦,文锦一挣扎,两个人滚在一起,滚到了巨石的后面,就听一声水声,好像摔进了水里。

我和胖子追过去,就见那巨石之后就是之前看到的那种水潭,底下是这神庙的低洼部分,深不见底,下面有回廊和甬道通到废墟的内部,闷油瓶摔下去之后,不得不放手,以免窒息文锦,他浮上水面,我心说这一次肯定抓着了,和胖子两个人在岸上一人把了一块,如果她爬上来,马上把她按住。

然而,三个人,两个在岸上,一个在水里,等到水面上的水波平下来,文锦也没有上来。

等了几秒我立即心说糟糕了,难道她不会游泳沉下去了,这不是给我们害死了,闷油瓶立即一个猛子扎了下去,潜入水中去找。

水里气泡不断,他翻了半分钟才浮了上来,就对我们道:“这下面通道其他地方,她钻进去了!”

“这怎么办?那她不是死定了?得立即把她救出来!”我道。

这种废墟里的结构极端复杂,回廊够错,四处肯定还有大量的塌方,就算有氧气瓶进去也凶多吉少。

“不会,这里的几个水池好像都是通的。”话刚说完,我们背后一个地方就传来人出水和剧烈喘气的声音。

我们立即转身朝那个地方冲去,跑了没几步就看到果然那里也是一个水池,水潭边上一片潮湿,脚印直朝林子里去了,显然文锦对于这神庙下的水路极其的熟悉。

我们立即尾随脚印狂追,跑不了几步,就听到了前面的急促的喘息声和脚步声,立即加速,就在这时候,我的头顶出现了一片沉重的黑色,我骇然间,发现我们追进了雨林里。

我顿了一下,心说不好,就这么追进去,如果迷路了怎么办?就是这么一顿,闷油瓶和胖子立即就跑远了。我大骂一声,只能跟上去,现在只有希望在最前面的闷油瓶能立即逮到她,否则我感觉会不妙。

虽然胖子分析林子中的雾气是没有毒的,但是谁知道推测是不是正确,要是在里面忽然瞎了,那绝对完蛋。

但是这文锦在雨林之中,简直犹如一条泥鳅,在树木的缝隙间穿梭,如入无人之境,这一通追简直是天昏地暗,最后我是头撞上一棵矮枝,直接被撞翻才停了下来,等我站起来,胖子和闷油瓶早没影了,只有远处传来遥远的穿过灌木的声音,也已经辨别不轻方向。

我眼冒金星,蹲下来大喘了半天才缓过来,感觉到肺都要抽起来了,抬眼看了看四周,却分不清方向,顿时心急如焚。

顺着大概的方向追了几米,我就停下来不敢再追了,开始大叫,让他们别追了,这样太危险了。

叫了几声,却听见一边树叶抖动的声音和传奇声,似乎他们又跑了回来,我立即朝那个声音的方向追了过去。

一连跨过好几道几乎没法通过的藤蔓群,一下却又丢了,我心说这简直是在拍猫和老鼠,永远是在绕圈子。

再次循着声音自己的去辨别方向,这时候,忽然就在我身后,有人叫了一声:“小三爷。”

那声音好像是捏着鼻子叫出来的,奸细的要命,是个女人的声音,听起来让人寒彻心扉。

我吓了一跳,立即转身,用矿灯照去。“文锦?”

身后浓雾弥漫,什么都看不见。但是那声音确实货真价实,我知道自己没有听错,立即就问道:“谁?”

在浓雾的深处,又有人叫了一声:“小三爷?”

我立即把矿灯调整了一下方向,朝那个方向照去,并且走了两步,但是还是什么都看不到。

我心中有点奇怪,那声音离我十分的近,应该是就在咫尺,绝对是手电可以照到的范围,为什么会没有人,难道那人藏着?

“你是谁?”我又问了一声。

没有回答,我感觉有点不对,用手电照了照四周,想找点东西防身,但是太黑了什么也看不见,我又不敢让手电光过久的离开我的前方。

“是不是三爷的人?”我又道。

“小三爷?”那声音又响了起来,而且移到了我的左边,我吓了一跳,立即把矿灯照过去。还是没有人的影子。

这家伙一定藏起来了,我心里毛起来,但是转念一想,不对,能说话的,就肯定是人,而且叫的是小三爷,肯定是认识我的,应该就是三叔的伙计,听这声音他似乎在围着我转圈子,会不会是他也看不清这里,不敢贸然现身?

想着我就立即道:“我就是小三爷,你是三叔哪个堂口的?”

那边没有回音,我心说他到底在忌讳什么,立即划动着矿灯,就朝声音传来的方向走去,一边走一边道:“出来吧,老子是人不是鬼。”

一直往前走了六七米,前方出现了一棵大树,却还是没见到人,我就纳闷起来,犹豫了片刻,忽然从那大树的后面,又传了一声:“小三爷。”

这家伙该不是聋了,我心道,扯起嗓子就大喊了一声:“老子在这里!”

那树后忽然灌木抖动了一下,我心说没时间和你这么耗了,一下冲过去,冲到树后就去照,没想到树后竟然就是一个断崖。我还没站稳忽然我就一脚踩空,人一下往下载去。

\chapter{第三夜:泥潭}

这一下摔倒是完全的猝不及防,比起在丛林中跋涉的摔倒完全不同,我根本连反应的时间都没有就已经滚了下断崖,混乱间我用力往身后抓,想抓到任何的东西可以把我自己停下来,但是手上摸到的全是光秃秃长满青苔的岩面,手直接滑了下去,接着膝盖又装到了石头上,我整个人一下没法保持姿势,翻倒摔到了崖底。

还在这断面并不高,而且下面是水和淤泥,并没有致命伤,但是我发现水流很急,一下就就扯着我往下游卷,我立即扑腾了几下,抓住水下不知道什么东西,咬牙吃力的站起来,就发现矿灯挂在半崖高的地方。已经够不到了。

缓了一下,感觉没有什么地方有骨折,我就观察四周的环境,也不看请清楚,只能感觉自己站在沼泽里,脚陷在淤泥中,而上面矿灯照出的区域,我看到摔下来的岩面应该是一幢遗迹的一部分。

我心中奇怪,怎么那树后竟然会是断崖,那刚才那人在哪里说话,难道是像壁虎一样趴在树上。

于是大叫了一声,但是再没有回音。好像那人就是要勾引我掉下去一样。心里一下又想起白天听到声音,心道完了完了,我真的有点幻听了,难道这里的森林扰乱了我的神经不成。

又扑腾了几下,我游到断崖的边缘,抓住一快突起的石头定住身体,接着矿灯光被石壁反射回来的极端微弱的光线,开始想爬上去,但是无奈青苔实在太滑了,又没有任何东西可以借力,爬了几次都滑下来。

我换了几面都不行,唯一可以前进的地方,就是顺着岩壁往沼泽的下游走,那边一片黑暗。但是这里水流这么急,附近不是有那种井口就是会有陡峭的断层,一旦我失足,很可能给井口的漩涡卷进去,或者冲下小瀑布,那不死也得脱层皮。

犹豫了片刻,我就发现我这样的处境其实就是被困住了,要么就要等到天亮,要么就是有人来救我,等到天亮我是绝对不肯,立即就扯起嗓子,喊了几声救命。

他们也许就在不远的地方,这里这么安静,喊响点他们可能能听见。

可是天不从人愿,喊着喊着,喊了半天,我喉咙都哑了,却连一点回音都没有,四周一片寂静,而且静的离谱,黑暗中连一点能让人遐想的动静都没有。

我实在喊不动了,心里那个郁闷就别提了,心说怎么什么倒霉事情我都碰上了,深吸了口气定了定神,我就看表,想看看雾气大概什么时候会散。雾气散了之后,能见度会加大,这矿灯的光线就能照的更广,这样也许我就有办法爬上去,或者我可以在水底找什么东西,把矿灯砸下来。

看了看表,按照昨天的经验,雾气应该维持不了几个小时,时间还可以忍受,我摸着一边的石头突起,让自己维持着一个舒服一点的姿势,看了看四周,心说这什么都看不见,这几个小时怎么打发。

双脚在淤泥里,让我心里很不舒服,这种感觉绝对不好,潘子和我说的故事,我还记得,此时也感觉淤泥之中的脚正在给虫子钻食,不时抬出来摸一把,却发现只是错觉。

这种错觉让我心绪不宁,我尽量人靠在岩石上往上爬去,让脚出水,但是每次都失败,我鼓起勇气,摸着岩壁往边上靠,脚贴着,想找水下有什么东西也好,能让我踩一下出水。或者能踩到一些树枝杂物什么的,我可以用来砸矿灯。

脚动着动着,我果然就踩到了什么东西,不过那不是树枝,那种感觉让我机灵了一下。

毛细细的,好象是人的头发。

我一下开始出冷汗,我现在对头发有着极端厌恶的记忆,从西沙回来之后的开始几个星期,我几乎碰到自己的头发都会觉得作呕。

立即把脚抽了回来,我不敢再伸过去,但是脚一动,我又踢到了什么,这一次是软软的,我忽然意识到这里的淤泥里,可能沉着什么的大个的东西。

谨慎起见,我打起手表的蓝光,往水下照去,这种蓝光本来设计就只是为了让人能在黑暗中看到电子表的数值,灯光几乎照不进水里,我只好蹲了下来,把手表沉入到水里去。

接着我就惊呆了,幽灵一样的蓝光下,我就看到一个沉在淤泥里的人,被脉在了淤泥里,头发像水草一样顺着水波舞动着。

我的手颤抖着移动,我就发现这是一具尸体,而且是一具新鲜的尸体,虽然完全给裹在淤泥之中,但是可以看出他穿的行军服,和胖子的很像。

接着,我就发现有点不对劲,转动手表的方向,我用力往前探去,就发现这前方底下的淤泥中,竟然全是死人,全部都沉在淤泥之中,肢体交错在一起,犹如屠杀后的乱葬岗一般。而且所有的人都是刚死不久的。

我将我面前的那具尸体从淤泥里拉出来,就发现死沉死沉,犹如灌了铅一般,一下就看到那人腰间的各种装备,都和胖子和潘子的一模一样。

我发着抖,忽然就明白了这是怎么回事情——三叔的队伍竟然在这里!

\chapter{第三夜:藏尸}

再看那句尸体,我就发现这些尸体都已经给水泡的发灰,但是都没有严重的腐烂,显然死了没有多少时间,尸体在泥水中没有被泡的发白,反而有点发青,显得有点不同寻常。

这里有这么多的死人,而且都是刚死了没多久,显然这些肯定是三叔的人。我想起空无一人的营地,不由感觉不寒而傈,这些人必然是给鸡冠蛇咬死后运到这个泥潭中来的。

这批是人最早出事的那批人,还是幸存下来的三叔?三叔在不在他们之中?

我一下又想起了刚才听到的小三爷的叫声,心说难道这不是人在叫我,是这里的伙计的冤魂,想让我发现这里,在指引我?

我脑子就发涨起来,但手表的蓝光再一次熄灭,四周又陷入了黑暗。

我再次打起手表,就开始摸着眼前尸体的口袋,从他裤袋中,摸出了一只皮夹,已经被水泡的死重,我掂起来,就朝一边石壁上的光点扔去,第一下没有扔中,我又把那人的皮带上的手电解了下来,甩了过去,一甩我就发现不对,但是已经晚了,手电已经飞了出去,我正想抽自己一个巴掌,这一次却成功了,卡住矿灯的灌木被打了一下,矿灯就滑了下来,掉进水中,沉了下去。

我一手抓住岩石的突起,一边竭力伸长了手,勉强够到,将矿灯捞了起来,手电很轻,却被水流往下游冲了几米,不知去向。

这一下看个更加清楚,我把手电朝四周照去,就发现这是沼泽的一部分,类似于一个原形的水潭,水流朝一边流去,手电照去,就看到水流流向的下游处是一处雕刻着一个兽头的石头遗迹,水流就是流向遗迹,由张开的兽口流入,和我想的一样,那下面肯定有井口,过去必然危险。

我开始逆流而上,将矿灯系到腰里,开始靠着岩壁移动,一路照去,就看到沼泽之中,横陈着大量的尸体,大部分全部陷入淤泥之内了,只伸出了僵硬的手或者其他部分。整个水潭低部几乎全是。

一边走一边避过尸体,但是尸体太多,实在无法脱身而过,很多尸体身上的淤泥被我激起的水流冲掉。我就发现在他们的脖子上,都有两个发黑的齿孔,整个脖子都是发黑的,到了四周部分就呈现青色。

他们全是被蛇咬死的。整个营地里都没有打斗的痕迹,有可能是在睡梦中直接被咬死的,也有可能是在这里行军的时候受到了大规模的攻击。

我调整矿灯,忐忑不安的一张一张寻找他们的脸,想从中看看有没有三叔。

我并不想看到三叔,但是理智告诉我,我不能逃避,这种心情想是认儿子尸体的父母,必须去确认又实在不想确认,不过在淤泥覆盖下,要想辨认并不容易,我一张一张看过来,都没有发现像三叔的人,同时却也无法肯定这些都不是三叔。

就在我想放弃的时候,我的矿灯就照到了其中一张脸上上,这脸还没有完全给淤泥覆盖,我下意识的停住了脚步,一下发现这脸有点熟悉,随即我就认了出来。

那是阿宁!

她的眼睛闭着,整个人呈现一个非常古怪的肢势,身上只盖着一层薄拨的淤泥。脸上的尸斑已经非常明显了。

我几乎窒息了,看了看四周,心说那些蛇竟然也把她的尸体运到这里来了!

矿灯照去,从尸体的表面来看,似乎这还是一具平常的尸体,并没有什么蛇化的异变,那么,我们当时看到的黑影不是她?那,那具发出类似于无线电噪音的黑影到底是什么东西呢?

我深吸一口气靠过去,心中已经无法形容是什么感觉了,把手伸到她的身上摸了一圈儿,没有对讲机。我想把她抱起来,却发现更本无法着力。她的脸被我搅动的沼泽水冲的干净,头发垂下来,呈现出一股异样的宁静,那一刻我仿佛还觉得她还活着。

但随后我重新将她沉入沼泽,浑浊的水一下隐没了一切,这错觉顿时消失的无影无踪。

我心中无比的酸楚,看着四周的景象,越想越心寒。

这个泥潭是什么地方,难道这里是他们堆积食物的场所。这里可能会出现巨大的蟒蛇来进食?

我感觉到极端的不安起来了,这个地方不安全,我必须立即离开这里。

想着我挥动矿灯,去找四周可以攀爬的地方,很快发现水流的逆方向,有一处树木的腾蔓挂到了水里。我咬住矿灯,就朝那边游去,几步够到之后一把抓住藤蔓。

雾气已经有些稀薄下来,我咬牙爬上藤满,却又想到闷油瓶说的,淤泥防蛇的时候,又下去掬起一手淤泥,抹到身上泥被水冲走的地方,再重新上爬,一直爬到了藤满缠绕的枝桠上,才松了口气。

顺着枝桠,走到树冠的中心,刚想顺着树爬下去,忽然听到一边的水潭中一声水声,什么又有什么东西掉了下去。

我寻着声音去照,就看到果然水滩边果然激起了涟漪,有东西从岸上滚了下来,手电照向那个角落,我看到一团红色的肠子一样的东西,那是缠绕在一起的大量鸡冠蛇。而它们之中,好象裹着什么东西。

我仔细看着,有一瞬间我看到一只手从蛇堆里伸了出来,接着我看到了一个胖胖的人头。

我浑身一凉,发现那是胖子。

\chapter{第三夜:又一个}

胖子并没有反抗,我甚至没有看到他在动,我心里的寒意越发冰冷,难道他已经死了。

蛇群路动着,我曾经想象了相当多的方式,来推测它们怎么运送尸体,但是我没有想到是这个样子,红色的大大小小的蛇盘绕在一起,将尸体裹在中央,然后挪动身体使得尸体前进,胖子体重极重,但是这些蛇还是能把他迅速移动到了这里,显然这样的移动方式效率相当高。

胖子摔入谭中之后,蛇群稀疏开来,开始重新爬上岸,很快就消失在石壁的上面,我看着静静躺在水里的胖子,有点不知所措。不知道他是死是活,如果是死了,我感觉他这样命硬的人都死了,自己在这里早晚也死定了,如果是活着,那我必须去救他,不过去了也有可能只是送死。

想了想,不管怎么样,我必须,去看一下,胖子和我出身入死,我不能连他有没有死都不知道,就把他丢在这里。

我警惕的看了看四周,似乎蛇已经走远,检查了一下身上的淤泥,就顺着藤蔓再次爬了下去,小心翼翼的下到水里,我趴着岩壁,走到胖子的身边。

胖子纹丝不动,大半个头浸没在水中,我心里一凉,心中就有点发颤。

仔细听了听,四周没有声音,我才靠进胖子,将他整个人翻了过来,下半身胖子沉在水里,一摸,我的心才一松,还与微弱的呼吸,但是我也立即看到了他脖子上的血孔。他也被蛇咬了。

这里的蛇真是阴毒的要命,竟然都咬在脖子上,这样除非那人对蛇毒有免疫力,否则基本上无法处理,只能等死。也不知道他们刚才出了什么事情,怎么身上的淤泥被冲掉了。

闷油瓶是因为动作快,注入的毒液量少才没事,胖子肯定就没这么幸运,不知道为什么他现在仍活着。不过,就算不死,他也快死了,我看了看四周,心说必须先把他从这个水潭里拖出去。然后立即采取一些措施,否则保不齐这些蛇会回来补上一口。

这相当困难,好在藤蔓在下游,我一边扶起胖子,借着水的浮力和推力将他往下游推去,没想到两步我就失控了,为了不冲到水流中去,我用力拧转身体,让自己的手浮在上面,冲过藤蔓的时候一把抓住,才重新控制住身体。

我用尽九牛二虎之力在水里站定,接着我把胖子挂到藤蔓上,用他的皮带把他固定住,然后自己先爬了上去,想把他拉上来。但是拉了两下之后我发现是不可能的,虽然藤蔓足够结实,但是胖子实在太重了,我这里小力气,实在不够看。我看了看四周,看到我站的树枝上面还有一根y形的大枝桠,立即就把藤蔓挂了上去,做了一个滑轮,然后用我的体重加上力气,把他提上来。

只一下我就把上面的枝桠压成弓形,整棵树都发出让人毛骨悚然的声音,我忽然就感觉胖子太重了,这简直是重的离谱,我的体重加上我的力气,把他吊起来应该没有这么困难。但是现在显然相当的勉强,我以前还背过他,绝对没有现在这么重。

这次如果能活着回去,我一定要让他减肥了,我心道,继续压下死力气,一点一点,用了整整半个小时,才把他从水里一点一点吊上来,等我把他拖到树枝上的时候,我的虎口全破了,已经连抬手的力气都没有了。这时候我站的树枝干脆就被胖子和我的重量压弯的恐怖起来。

我已经没心思来琢磨这些事情了,缓了一下,心说该怎么处理他的毒,要我吸出来已经晚了,看样子还是要回营地,这就地拖他过树林了,我一个人实在是够戗。不过够呛也得做,不然如果他挂了,我怎么过自己这一关。

休息了一下,我立即又下去,再次掬了一把淤泥上来,涂在胖子身上,就去扯四周的藤蔓过来,把藤蔓草草连接了一下,做一个拖架子,想把胖子从树上放下去。

往胖子身上绑的时候,我发现胖子太胖了,实在很难固定,只好用藤蔓先把他的几个地方绑进,藤蔓很粗,我的手的力气不够,我就站起来用脚帮忙,把结打紧,大概是用力拉的力气太大了,忽然胖子就张开了嘴巴,从他嘴巴里,喷出了一口绿水。

那绿水极其腥臭,我立即捂住了自己的嘴巴,心说他吃了什么了,这时候就看到,那绿水之间,竟然混杂着很多的细小的红色鳞片。

我摸起一片,心说不好,一下扯开胖子的衣服,就发现胖子的肚子极大,用力摸了一下,发现硬的好象吃了一个秤砣。

\chapter{第三夜:宿主}

糟糕了,怎么会这样?难道有蛇钻到他肚子里去了?

我立即把胖子翻过来,用膝盖去顶他的胃,用力碾进去,他就开始剧烈的呕吐,大量的绿水混杂着一些白色的棉絮一样的东西被吐了出来。全部吐到了树枝上,然后滴落下去。

我用力顶了几下,等到他吐完,就发现他的呼吸稍微顺畅了一点,看来这胃里面的东西也非常压迫他的呼吸。

看着吐出来的东西,量极大,简直就像从桶里倒出来的,好在胖子胃大,否则普通人这么多东西撑进去,胃可能已经爆了。

我将他放好,就捂住嘴巴去看他呕出来的东西,一股酸臭扑面而来,发现绿水之间,都是蛋花一样的白色凝胶,我折下一根树枝拨弄了一下,就发现凝胶之中,竟然全是一种类似卵的东西。

一瞬间一股极度的恶心涌上胸口,我差点也吐了出来,看着其中混杂的鳞片,我心说这该不是蛇蛋?我操,这真是太恶心了,这种蛇竟然会在人的胃里产卵,简直像好莱坞电影里的怪物。想着立即把这些蛇卵全都拨弄了下去。

这么说来,下面这些尸体的肚子里,应该也塞满了蛇卵,我操,我都无法想象这些蛇卵孵化出来会是什么样子。

努力忍住自己的恶心,我看了看下面的泥潭,又看了看那些漂浮在水面上,向下游飘去的蛇卵,开始明白了这里是怎么回事。

难道这里的泥潭是一个“孵化室”?这些蛇,靠尸体腐烂产生的热量孵化蛇蛋,所以它们不停的搬运尸体。倒入这个泥潭内,让他们不停的腐烂,和泥土混合产生热量。

我听说过有很多蚂蚁可以通过发酵和腐烂来控制蚁巢内的温度,这些蛇显然做不到,但是它们已经在通过腐烂的热量来孵蛋了。

但是,这里附近的废墟阳光很好,为什么它们不像其他蛇类一样用阳光来孵蛋呢?难道是因为这些蛇蛋孵化对于温度的要求非常精确?

想想不对,我想到一个可能性,如果没有那几场大雨,这个泥潭中不会有水,最多是一片烂泥沼,那么胖子摔入到里面,要很长时间才会死,那么他的体内的温度会维持到他完全死亡,这也许就是胖子现在还没有死的原因,那些蛇只想麻痹我们,不想杀死我们,就是为了用我们体温孵蛋。

我知道有一些进化的非常高级的蛇,它们的蛋在体内已经孵化的差不多了,生出来只要一到两天稳定的温度就会孵化,难道这里的蛇就是这种意思?那好在下了这场大雨,否则,我刚才已经摔进小蛇堆里了。

最让我感觉毛骨悚然的是,这里有蛇卵,那么这不就说明,这里还有一条母蛇?想起那浮雕我就浑身发凉,但再想还是不可能的,这么巨大的母蛇绝对违反了自然生物的规律,这些卵可能是那条巨大蛇母的后代生的。

胖子肚子还是有点胀,不知道里面还有没有这些东西,我觉得保险一点还是让他全部吐出来的好。于是我扶起胖子,扣住他的喉咙,让他继续呕吐,但是他接下来呕出来的,都是发绿的水,最后就成了干呕。

我相信应该是没了,再有就应该过了胃了,那就只能让他拉出来了。

雾气已经散的差不多了,能见度逐渐恢复,我继续刚才的工作,将他身上的藤条拉紧,然后准备慢慢的放下去,这非常困难,如果我稍微有点抓不住,胖子就可能直接从树上摔下,他现在失去了意识,不会运用肌肉和动作去保护自己,那么这一摔可能就会摔死他。所以,必须把藤蔓的长度控制好。

我把一切准备妥当,然后用矿灯照射树下,这颗大树长在泥潭的边上,弄的不好,可能放下去就直接摔回泥潭里,前功尽弃,一定要选一个好地方。

矿灯一照到树下,我就楞住了,之间树下一片迷蒙,竟然看不清楚地面。手电照过去,好像照在一团混沌上。

这真是有鬼了,刚才我没有用矿灯去照,就用矿灯的余光,都能看到地面模糊的影子,怎么现在反而看不到了?难道雾气又浓了起来,可是为何只浓在地面附近的部分?

仔细一辨认,我就发现原来是这泥潭中不知道出了什么变化,从水中蒸腾起了一股黑气,已经笼罩了整片水面,其中的尸体若隐若现,在黑气中竟然好像动了起来。

\chapter{第三夜:沼泽怪影}

仔细去看,就发现是潭底的淤泥不知道被什么东西鼓动着,似乎有一个巨大的东西在淤泥的底下活动,将存在淤泥下的黑气翻上来。整个潭底都在动,淤泥中似乎有一个不规则的漩涡,把那些尸体裹进去又吐出来。

随着淤泥活动的更加剧烈,越来越多的黑气从下面翻了上来,我此时已经没有任何力气感觉到害怕了,只是牙齿发紧,浑身的发条已经上到了最紧,不可能再进一步,一边脑子飞快的转动琢磨怎么办,一边警惕的关注着下面的情况。

这些黑气可能是沼泽下雨林中大量树叶腐烂形成的有毒气体,这种气体经常存在于沼泽和雨林的深处的淤泥之下,如果有大的自然气候变化就会释放出来。

很多热带雨林人力不可涉及,就是因为这种毒气的存在阻断了大片的通路。而有的毒气则是由于特别的矿物或者火山气体挥发,或者和雾气混合而形成的剧毒云雾,这种毒气的毒性就厉害了,世界上有很多的连鸟也飞不过去的“死亡谷”就是这么形成的。

如果是的话这玩意肯定不是好玩意儿,也不知道会不会和昨天在神庙前遇到的雾气一样致盲。

想过是否能立即下去,冲回遗迹,但是算了一下距离和时间,此时已经毫无办法,那黑气已经弥漫在树下,我已经无法下去。而且神庙那边的雾气如果没有退,很可能又会让我中毒失去视力,碰上蛇群我就可能和胖子一样了,那我宁可自己了断自己。

我祈祷着,这黑气只在树下蔓延,不会浮上到树冠,但是显然这是不可能的,缓缓的,我发现黑气犹如有生命的一样,滚动着开始充斥整个空间。

我心中暗骂,知道这一次如果这黑气有毒,恐怕会比致盲更加厉害,情急间,我立即撕下自己的一条衣服,往身上抹下来一大块黑泥,捂住了口鼻,又给胖子也做了一个。

之后想起自己在树上,立即找了藤蔓把自己绑住挂在树上,以防如果等一下中毒神智模糊,从树上摔下去。

刚做完这黑气就到了脚下,蒙上来的时候,蔓延的速度惊人,黑色的影子如鬼魅一般,几乎是一瞬间就裹住了我们坐的枝桠,我甚至听到它经过的时候,这里的树都发出了轻微的噼啪声,接着四周目力能及的地方一下就被黑气所笼罩了。

稀薄的黑气一下就布满了四周,看着黑气腾起来,我感觉自己好像被困在大火中的房子里一样,但同时我立即就闻到一股奇怪的味道,喉咙开始发痒起来。

喉咙发痒显然不是好兆头,本能的屏住了呼吸,尽量少吸几口。

几秒钟后,我没有立即毙命,就松了口气,显然这黑气毒性不烈,这样我们就多了很大的机会,不过,如果吸入太多,但是到底如何,也很难说。

我一边祈祷这黑气会和雾气一样自己退去,一边往上开,想看看是否能爬的更高,到黑气稀薄一点的地方,但是,抬头看整个树冠目力所及的地方,已经完全给这些黑气秒绕了,而且在矿灯的光柱下,我看到这些黑气好像是固体的小颗粒,似乎是烟,而不是气,上去摸了一把又摸不着。

这是什么东西?我忽然感觉我在什么地方看到过这种黑色的烟雾,是在哪儿呢?我想着心里就应约感觉出不安来,有一股极端不吉利的感觉冒了出来。

我忽然就想起闷油瓶,心里只问候他的祖宗,要是刚才听我的,现在就不至于那么狼狈。自己怎么就不坚持一下,要是死在这里不知道找谁去含冤。

可能是之前我实在太信任他了,可是他最近做的决定都有些失常,心里顿时想抽自己一个嘴巴。

不过,就算是不来,今天晚上也不知道能不能过来的,当时没带放毒面具倒是我的失策,不过阿宁他们装备的防毒面具个头很大,儿胖子和潘子用的都是老军用,结实但是太重了,都不方便。

怎么想都不对,想想这也是逃不过的一劫难。

继续看着泥潭,就听脚下的沼泽里传来了一连串水声搅动的声音,很沉,并不吵耳朵,听着好似有什么庞然大物要从里面出来了。

这沼泽之下必定出了什么异变,否者不可能会出现这种动静,我想着会不会尸体肚子里的蛇卵孵化出来的,又或是有大蛇来进食了?

只听得这水声越来越响,好像在朝我们树下靠近一般,我拿矿灯去照,就间黑气中,隐藏着一个足有小牛犊一样大的黑斑,正在不停的移动,体形比我们之前遇到的那条还要大上一圈。但到底是不是蛇真的无法判断。

黑气弥漫影响视野,那黑斑之下到底是什么东西根本无法看见,我感觉这时候也只能听天由命,都凝神静气,看着那黑斑的动向。

这雾气之下全是沼泽,黑斑从沼泽中来,必然不是什么陆地上的生物,看形状也不是之前碰到的那中巨蛇,我心说否则他这样大的体型我刚才不可能没有看见,会不会是一条埋在淤泥里的大鱼。

然而,沼泽里什么鱼能长大小牛犊这么大,难道是鳄鱼吗?想想不太可能,如果是鳄鱼,刚才我已经挂了,在这种泥潭里,如果有小牛犊大的鳄鱼,我肯定会给拖进去,鳄鱼绝对不会放过侵入他地盘的东西。

思索间,黑斑忽然在我矿灯光斑的附近停了下来,似乎注意到了这个光点,我有点感觉不妙,立即把光点移走,转到树冠之内照着胖子。

这一照,我就发现不对劲,胖子头都耷拉了下来,竟然从眼睛里流出了黑血,我心中大骇,探手过去摸,就出了冷汗,只感觉胖子浑身冰冷,只有出的气没进的气了。

我暗骂一声不好,不知道是蛇毒发作了,还是这黑气的毒性,当下也没法管这么多了,我把胖子搬正,就用力掐他的人中,掐了几下根本没用,心里一阵恶心,心说得给他做人工呼吸了。

然而胖子的姿势非常别扭,背后又没有什么树枝靠住,我必须用手扶住他才能让他的头正起来,然而此人极重,我踩着树枝啪啪响,换了好几个位置都不行,单手根本扶不住他的上半身。

最后我干脆就踩到他坐的那枝桠上,趴到他的身上,然而急火攻心,才趴上去,忽然就听得“咔”一下,接着是一声脆响,他坐的枝桠就断了,我忽然感觉身下一空,还没意识到怎么回事呢,抱着胖子就翻下了树下,往水潭里摔去。

\chapter{第三夜:鬼声再现}

一刹那我就吓了个半死,然而没等我反应过来,我们就被身上的藤蔓一扯,两个人在空中打两个转儿,狗啃屎趴进下面的水里。

我摔的七荤八素,入水那一下我几乎是平着拍进水里的,那种感觉就好像被人用灌满水的热水袋狠狠的甩了一巴掌,好在水冰凉,否则这一下我就肯定背过气去了。

扑腾了几下再次浮起来,我忙去找胖子,心里就说要糟,这泥潭里算是黑气最浓的地方了,胖子已经这样了,又摔了个半死,在这里再喝几口水那是死定了,再加上刚才的黑影不知道是什么,要是什么沼泽怪物,连我也会挂。

我身上绑着藤蔓,连顺畅的活动都不行,就算胖子能挺,我也没办法将他重新搬回到树上去,而且,虽然我不知道为什么我在树上黑气似乎没有剧烈的影响,但是在这里浓度这么高的地方,我自己能不顶住还是一个问题。

但等我一探头出水,忽然就发现不对,水面上全是水泡,一是四周的黑气把大部分的光线都遮住了,能见度比起雾的时候还低,二是整个沼泽里全是翻滚起的泥水,一片浑浊,完全看不到水底。胖子在哪里都不知道。

四处去听,全是水泡的声音,听不到一样,而且我明显就感觉到水流竟然急了不少,我稳不住身子。我心中奇怪,仔细一感觉,我就发现不单是水流的问题,我身上的藤蔓原本是缠绕在枝桠上,现在那一人粗的树枝已经给水流冲往下游,一下全部的拉力就扯在了藤蔓上,将我往下游带去。

没有在自然河流中游泳过的人不会明白这种感觉的,水是一种非常重的东西,就算是水流缓慢,你在其中要定住身形也是非常困难的,何况还有如此大的东西在前面拽我,我四处张望的功夫,已经给水流跌跌撞撞往前带去了好几米。

这时候我就更急了,我已经看不清楚四周的情况,前面肯定有一个井口,我不知道有多大,如果这枝桠冲入井中,那种拉力可能一下就把我扯下去,我连一点反抗的力量都没有。而且那兽口一般的遗迹就在不远处,这过程肯定不需要多长时间,这时候不要说找胖子或者小心那黑影,就是能留个全尸就不错了!

想到这里,我立即深吸了一口气,就一下潜入水里,去解我那藤蔓,但是那藤蔓被巨大的拉力拉的极紧,根本没有可能解开,我去摸匕首又发现根本没带。

我心说完了,想起胖子武器不离身,肯定有带着,就去找胖子。就顺流往前扑通,他身上也有着藤蔓,我就去水里摸。

水下全是泥浆,摸来摸去都是横陈的死人,几乎什么也摸不到,不过胖子体型大,绝对不会比我漂的远,我竭力对抗着水的推力,终于摸到了另一根绷紧的藤蔓,我抓住藤蔓靠了过去,忽然就看到前方两三米处,一个黑色的影子漂在水面上,朦朦胧胧,根本看不清楚是什么。

我心里发毛,看着那影子漂着的样子,就知道这是我刚才看到的水下怪影,心里有点不详的预感,藤蔓的尽头就是这个影子,心说难道胖子已经被他吃了。

水深只有两米多,那黑色的影子突出水面的高度很高,显然肯定不是鱼,到底是什么?我扯动藤蔓,正犹豫怎么办,就见那影子一抖间,突然改变了形状,消失在水下,接着我手里的藤蔓一下松了。

我知道糟了,它发现了我,刚想转身,一团巨大的泥水花就从沼泽里炸了起来,我看到一对大鳌闪电般朝我的脖子钳了过来。

“我操你爷爷!”我大骂一声,心说这是什么鬼东西,但是它离我的距离实在太近了,根本避无可避,眼看那巨鳌就要夹到我的脖子,就在这时候,我腰上的力量忽然一紧,我整个人被藤蔓突然扯飞了出去,正好躲了过去,我刚想说上帝保佑,却发现腰上的力量变得极其霸道,回头一看间我已经被扯到废墟附近,那兽面石雕就在我身后,张着巨口,而藤蔓已经掉入口中,口里能听到咆哮的水声。

我知道那牵拉我的树枝已经摔入井中了,心说上帝你是不是在耍我,立即用手抓住一边的岩石,大吼一声定住身体,感觉几乎腰都要被拉断了,就这转念之间,身后水花飞炸,那东西又来了,我心念一慌,手立即脱了,一下通过兽口,眼前一黑,身后一空,也摔了下去。

那一瞬间,四周的声音都消失了,腰间的矿灯随着我打转的身体转动,划过四周的黑暗,我凌空翻了一圈看到了被流水冲的满是沟壑的井壁和四周飞溅的泥水,但是下落并没有持续多少妙,我的后背就撞到了什么东西,整个人一震,几乎吐血,没等我缓过来,背后又是一空,我又翻了个圈,接着肩膀又是一撞。这井下竟然不是垂直的,好像有一个坡度,上面全是被水冲的圆润无比的台阶一样的突起,我一路就翻滚着摔了下去。

三四次之后我就完全晕了,直到我摔进水里,我连喝了十几口泥水,才挣扎着探出水面,就发现自己在一个狭窄的井道中,被裹在一道极其急促的水流中,速度极快的朝某的地方冲去。

四周一片漆黑,狭窄的感觉是水流的剧烈轰鸣告诉我的,四周一摸就能摸到井道壁,但是什么也抓不住,好在我之前把矿灯系在腰间,但是这么急的水流中,只要你稍微一动你的动向就完全混乱,甚至会给从井壁上撞回来的乱流直接翻个头朝下,所以我也不敢轻举妄动,只能尽力维持自己的姿势。

没多少时间我就听到更加剧烈的水声从前方传来,那简直是水龙的怒吼,振聋发聩,我心惊心说我操肯定又是一个下坡,但是转念之间身下已经一空,接着就又摔了十几个跟头,发现自己摔进了一个空洞中,这时水流趋缓,可以控制自己的身形了。

我立即掏出自己的矿灯,朝四周去照,就发现这里是一个地下蓄水池,四周有巨大的水主动水池壁上的井道口冲下来,好像看大坝泄洪口的感觉,四周水花飞溅,声音震耳欲聋,我忽然感觉自己像是一只被冲下抽回马桶的蟑螂,现在从粪道被冲到了化粪池里。

我扑腾了几下,就发现水流还是在缓慢的朝一个方向流动,我游过去,手电照去,我又看到了井壁上有一个兽头,水流还是留向兽口之内,不过这一只兽和上面的一直造型并不一样,显然这里只是一个分流的蓄水池,用来蓄洪防止井壁被冲刷的太厉害,而在那兽口四周,我就看到了巨大的犹如山一样的狰狞枯树枝几乎将其堵塞了,这些应该都是常年累月从沼泽外延冲下来的淤积在这里的。

拉着我的树枝也卡在了上面,上面还挂着一个什么东西,我仔细一照,发现竟然是胖子,他也被冲下来了。

从海南回来之后,我的游泳技术突飞猛进,在水里倒不觉得活动十分困难。一下我就扑腾了几下,往堆起来的枯树枝堆游去,游到边上趴了上去,就看见胖子身上的藤蔓就卡在枝桠外盘根结错的枝节中,使得他没有沉到水下去,这里磅礴的水声已经远了很多,我的耳朵终于可以听见东西了。

我从枝桠下的水下潜水过去,到了胖子那一边,就看到他的脸已经全部青了,气息弱微,脉搏都几乎摸不到,我再次潜下去,抱住他的脚把他的脚也架到枯树枝上,用肩膀去顶他的肚子,顶了几下他就吐了,一团的泥水,然后我用肘部给他按摩胸口,胖子给水一呛,竟然有了反应,一阵咳嗽。

我心中一喜,心说有反应就是有门,立即用力再顶,却几下就没力气了,上来喘了口气,心说这样不行,胖子如果不做人工呼吸就死透了,我必须把他整个人脱出水去。让他平躺在树枝上。

想要让胖子上去,就必须我先上去,想着我开始爬那些枯树枝堆,无奈在边缘的那些树枝并没有足够的支撑力,我只要上去,就把枝桠整个儿压进水里,而且有侧翻的危险,枝桠侧翻,胖子会被压进水里,那等于是我杀了他,而且这里大部分是荆棘枝太多,稍微动作大点就会撞到尖刺,疼的我眼泪都下来了,而里面的树枝都已经腐烂发软,根本无法受力。

在那几分钟里,我也不知道爬了多几下,全部都在两步到三步之间树枝就被才断滑了下来,我最后就绝望的发现,以我个人的力量,在这个位置绝对爬不上去。这树枝堆看上像山一样结实的地方,其实都极度的脆弱,根本没法呆人,其实之上只有半米不到就可以出水,然而这半米却似万丈鸿沟,我怎么也越不过。

这种绝望感实在太强,要是我面前是个峭壁那也就算了,可是偏偏是这种树枝。我突然感觉好像老天在玩我。

我又爬了几下,手全破了,意识到蛮干肯定不行,于是架住胖子,用他的匕首割断藤蔓,就把着树枝堆向边上挪,想找找这里的岩壁上有没有更容易爬的地方,最好是有可以搭手的地方。

这里没法逆流,我用力架着胖子绕过了突出的好比棱刺一样的树枝,忽然就看到另一边的岩壁上,有一个干涸的井道口,可能是哪里被淤塞住了,并没有水冲里面冲出来。仔细一看,这种井道口还不少,但是都是在很高的位置上,只有这一个我能够的着。

我心中大喜,就靠了过来,先把胖子架在一边,然后自己抓住石头的缝隙,就往上爬,才爬到一半我就知道有门,不由就笑出了声来,接着咬牙就想一鼓作气。

就在这个时候,忽然在边上的胖子,突然动了一下,说了一句话:“没时间了!”

我吓了一跳,转头一看,却见胖子丝毫没有动,也没有任何的表情,我心中奇怪,揉了揉太阳穴心说完蛋了,又开始幻听了,忽然,又一声的清晰人声,就从胖子身后发了出来。那声音就道:“没时间了。”

\chapter{第三夜:雾中人}

这里之中除了远处水泄的隆隆声,几乎听不到任何其它的声音,这一声说话声极其突兀,突然一响,我猝不及防,就吓了一声冷汗。

第一个反应就想到了是不是三叔的人,心说难道这里还有幸存者?

刚才的声音,能肯定是人在说话。我知道我不是幻听了,我之前没有期望过还能碰上一个活人,是人就让我心里稍微安了一点,我停止动作,就探头往胖子身后看去,然而后面全是堆起的干枯树枝,交错不清,光线又差,什么也看不清楚。

应该是三叔的人,我有了一个念头,这林子不可能有其它人,如果突然碰上一个人,最有可能的还是三叔的人。也许就是这个刚才在叫我,然后在我跌下泥潭之前就被水冲到这里来了,听刚才的话,似乎他在和别人对话,那可能还不止一个人。

“谁在那里?”我就叫了一声,眯起眼睛使劲的看着那个方向,如果在这里碰闪三叔的人,那真是老天保佑,可以知道三叔的下落和遭遇了。

然而等了一回儿,胖子身后却一片寂静,没有任何回音。那边的树枝遮掩下的兽口犹如凝固,也没有动静。

我立即警觉起来,心里出现了一种不详的预感,一边就摸到边上一根长条的木棒,抄起来端着,然后慢慢往那里靠去。可才走了几步,我就听到从树枝堆的伸出,又传来了一个幽幽的声音:“小三爷?”

那声音非常的怪异,说的极快,不过确是一个人的说话,而且是在叫我的外号,我一下心就一放,那肯定是三叔的人。而且肯定还认识我。

我一下就松了口气:“是我!”立即过去,扒开树枝堆的空隙,边扒边问:“谁在里面?是不是被困住了,别担心,我马上来救你!!”

“小三爷?”深处又问道。

“是我!!!是我!!”我就叫起来,一边就把树枝堆扒出了一个洞,从树枝间中的缝隙中探头了过去,去找深处的人。

扒开了很深一段距离,什么人也没有看到,里面全是腐烂的树枝了,那里边的人却没有说话了,我觉得奇怪,就用长沙话骂了一声,道:“嬲你妈妈别的,到底谁在里面,你搞什么鬼,说句话告诉我你在哪个位置。”

叫了几声,还是没有回音,我又感觉到有点不对了,听那人的声音不像是受了伤或者不能移动的样子,那听到我这么说怎么样也应该过来了怎么会叫了这么久无动于衷?又或,难道他听不清楚我在说什么?还是他也意识模糊?

想着我就忽然意识到,虽然我自己没有受到什么影响,但是刚才沼泽中全是黑气,这里也必然会有一些,这人可能也是被蛇咬了,如果中毒很深,肯定是神志不清的,就是没被咬,也可能因为刚才水流的关系撞坏了脑袋,听不清我说什么。

想着我就不叫了,咬紧牙关,猛往里挖去,想挖到他再说,要是对方确实也中毒了,那麻烦就大了,我一个人照顾两个可不成,不过又不能假装不知道。

这篇树枝堆大约有六七米高,看着不大,但是在里面挖出一个洞找东西也相当的困难,我忍着剧痛,用手趴着那些树枝,花了两三分钟才一下挖通一个空间,立即我趴着探头过去,往那声音传出的地方看去。

我原以为会看到一个人靠在哪里,然而,让我目瞪口呆的是,树枝堆内竟然什么都没有,根本就没有人,后面竟然就是兽口。

“怎么回事?”我就骂了一声,话音未落,忽然就从我挖出的树枝堆洞的边上,又传出了一声幽幽的,犹如鬼魅一样的声音。

“小三爷?”

那声音几乎就是在我耳朵边上叫了起来,我吓的头皮一炸,几乎从树枝堆上摔下去,猛转头一看,就发现我挖出洞的一边,树枝交叉内的黑暗中,竟然和我一样趴着一个人,缝隙中露出了一对血红的眼睛,正死死的盯着我看。

\chapter{第三夜:窥探}

我身边没有照明的东西,树枝之内是封闭的空间,是一个死角,在这种光线下是很难看清里面的情况的,我盯着那血红的眼睛,只感觉到喉咙发紧,一时间也忘了反应,也直直的和他对视。

对视了几秒,我便发现了不对,这眼睛的血红似乎不是一般的血丝弥漫,而是真的被“血”染红了,那血色甚至渗出了眼眶,而且那眼睛根本不眨,好比凝固了一般。

活人可以不动,但是绝对忍不住不眨眼睛,这是一个常识,我立即心中起疑。

摸索身上,就摸出几只火折子,拧掉防水的芦苇杆,打起来就小心翼翼的往那方孔中送。

靠近孔口,里面的情形就照了出来,我一看之下,人整个就炸了起来,从脑门到脚底一下全凉了。

映入我眼帘的是一张狰狞的怪脸,已经有点发肿了,这甚至不能说是一张脸,因为他的下巴已经没了,整个脸的下半部分不知道被什么撕走了,血肉模糊,整条舌头都挂在外面,没有下巴的连接,舌头直接从咽喉里出来,看上去就奇长无比,好比一条腐烂的蛇。

这是一个死人了,我一下就感觉想吐,好不容易忍住,就感觉到一股毛骨悚然。

看此人的发型和装备,显然也是三叔的人,死了也不长时间,应该是被水冲进来卡在这堆树枝内的。但是,如果这是一个死人,那刚才叫我的是谁?

我立即再次看向那尸体,这时候,火折子却烧完了,那狰狞的脸孔重新隐入黑暗,我只看到那血红的眼睛还怨毒的瞪着我。

我身上的鸡皮疙瘩全部都起来了,看了看四周,这是黑漆漆的地下水池,没有任何其它人在四周的样子,而且刚才我也没有听到任何人移动的动静。

冷汗刷刷的下来,我的脖子有点发硬,忽然意识到不妙,这里肯定发生了诡异的事情,我不能再留在这里了,不管怎么样我必须带胖子立即离开。

深吸了一口气我就爬了回去,解开自己腰上剩余的几条结实的藤蔓,套在腰间,就探身下去,抓住胖子的手往上拉。

胖子是在太沉了,加上他的衣服泡了水,简直犹如铅块,我只有一只脚能出力,拖了几下几乎纹丝不动。几乎自己又要滑下去。

我立即知道用手拉是没有办法了,看了看四周,看到胖子身上也还系着我做的建议拖架,就把托架的藤蔓绑在自己身上的藤蔓上,用木棍打了个套节套在胖子的腋下,横过他的腋窝做了个类似担架把手的东西,另一端撑在地上,就用自己的体重加上力气,像黄河纤夫一样咬牙往上拉。

这是建筑学里的三角力学,当时老师教我们怎么用一根棍子和一条绳子配合自己的体重做牵引吊具,工民建里的也有这样的课程。

有我体重的帮助就好的多了,我扯住藤蔓一点一点的往井道里跑,水里的胖子就给我一点一点提起来,最后终于给我把大半个人抬出了水面。但是此时我腰间的藤蔓几乎就把我扣成双截棍了。

我找了一条比较粗的石头缝隙,将我备用的木棍卡进去,将腰间的藤蔓过了过去,固定住胖子,然后再爬回去到水里,将胖子的双脚抬上来,拖过来到达安全区及,然后解开他身上的藤蔓拖架,看树枝堆中暂时没有异状,立即就给他做心肺复苏。

我没有收过专业训练,动作都是连续剧里看来的,只记得如果心脏停跳,极限时间是8分钟,8分钟内救活的可能性很大,现在胖子还有微弱的脉搏,呼吸微弱,这应该是中毒症状,不知道心肺复苏是否有用。

搞了几下不得要领,也不知道对不对,只能硬着头皮做下去,又按了不到两三分钟,忽然胖子一声咳嗽,整个人抽搐了一下,又吐出了一团黄水。接着就深深的吸了一口气,胸部开始起伏起来。但是只吸了一两口,他一下人又翻起了白眼,呼吸又微弱了下去。

我看了看他脖子上的血孔,显然这毒蛇确实厉害,这一口咬的份量精确,胖子形同废人就是不死,只要这体内的毒不去掉,怎么救胖子都没用。我脱掉自己的衣服,在水池里捞了点水,用匕首切开他的伤口,洗了一下放出黑血,接着一边继续给他按胸口,让他能坚持下去,一边琢磨该如何是好。

只按了两下,我忽然听到背后又传来一声阴恻恻的声音,同样是在那树枝堆之内。

情急之下,我没有听清楚说的是什么,但是听着耳熟悉,这一下子把我吓僵了,我猛的再次回过头,用手电去照看那方才我在树枝堆上挖出的洞。

就隐约看到那血红的尸眼还是呆滞的看着我,冰冰凉凉,看着让人万分的不舒服。而让我头皮一炸的是,我看到那尸体的舌头,竟然在动。

\chapter{第三夜:毒舌}

我暗骂了一声,心说他娘的真是倒了血霉了,难道这也诈尸了?

不过这个时候的我已经完全豁出去了,心说就算是诈尸,这新鲜粽子也没有下巴,它也咬不死我,正欲大战一场,忽然就看到在那舌头下,探出了一只火红的蛇头,大约拳头大小,头上有一个巨大的鸡冠,那蛇头一扭动,整条蛇就从舌头下爬了出来,爬到树枝堆上。

我和胖子所在的井口,离那树枝堆也不到两三米的距离,这蛇蜿蜒爬到树枝堆上之后,顺着树枝堆上横生的枝桠就慢慢游了下来,蛇身颇长,足有一米多。比咬死阿宁的那条还要长点。

这蛇显然是躲在那树枝堆之内的尸体里的,被我惊动了。

那蛇很快就顺着树枝堆爬上石壁,石壁很不平滑,它顺着石壁就如同壁虎一样悄无声息的往我们爬了过来,我一看糟糕了,我根本没有时间来避开,情急之下我悄悄从井口上滑了下去,缩进了水里。

本以为它会给我们惊动,然后从水里翻出来,我离树枝堆已经有了两米多了,马上往上看去,就看到那蛇被胖子吸引了注意力,那边上就是胖子所在的井道口,它顺着石壁堆一路往下,就到了井道口,立即它就发现井道里的胖子是个活人了,停了下来,转动了几下头部。

我的心立即吊了起来,心说它该不是要咬胖子,这不太可能啊,胖子像死鱼一样躺着,如果不惊扰蛇,蛇不会主动去咬东西的,毕竟毒液是很宝贵的。

看着那蛇忽然就又动了起来,爬了井道内直笨胖子的头部,竟然盘到了胖子的额头上,好像要往胖子嘴巴里钻。

我一看坏了,它又要进去给胖子补充蛋白质了,立即想找什么东西砸过去将它赶开,却发现在水里什么也摸不到。只好用手甩起水花,去打那蛇。

这真是个愚蠢的决定,如果是别的种类的蛇可能就一下被吓跑了,但是我忘记了这蛇是有邪性的。那蛇被我的水一拍,一下缩了出来。立即就发现了我,它直起蛇身,鸡冠直立,发出了一连串“咯咯咯咯”高亢的声音,似乎在威胁我。

我一看还以为有效果,继续拍水,还没等我拍起第二个水花,忽然那蛇一个收缩,一下就发现了了,接着犹如离弦之箭一样竟然飞了起来。窜出井道口,贴着水面一个非常优美的8字舞动,几乎不到一秒就冲到了我的面前。

我只看到红光一闪,条件反射就用手去挡,那蛇整个就盘上了我的手臂和肩膀,只感觉竟然有手臂粗细,鳞片滑腻非常,那一刹那我几乎看到了它的毒牙,脑子立即嗡的一声,就大骂了一声往外甩去。

那是疯了一样的动作,这一甩应该是用出了我全部的力气,蛇竟然真的给我甩了出去好几米,但是它还粘到水突然就一个回旋,尾巴拍水又弹了起来,贴着水面又来了。

我转头就逃,用起全身的力气扑腾开来,往前一窜就扎进水里改变方向连游了好几下,就钻进了树枝堆下的空隙躲了进去。

一直躲到实在憋不住气了,才从水里探出来,大口的喘气往四周看,我努力压低剧烈的呼吸,往四周看,想看看是否骗过了那蛇。

我心中想的是蛇始终是畜生,总不会人那一套东西,这种简单的小计谋总能起点作用。

看着却让我意外,我看了一圈,水面上没有那蛇的影子,似乎是没有追来。

我心里松了口气,心说小样的小命算是捡回来了,刚苦笑,嘴巴还没裂开,在我脑后,忽然又有人阴侧侧的冷笑了一声。

我已经经不起惊吓,立即遍体生凉,回头一看,立即就看到那条血红色的鸡冠蛇就直立在我的脑后,怨毒的黄色蛇眼居高临下的看着我。

我一下喉咙窒息,立即就想潜入水里,却看它鸡冠一抖,忽然就发出了一个幽幽的声音:“小三爷?”

\chapter{第三夜:蛇声}

听到这蛇说话,我先是愣了一下,接着就懵了,几乎不敢相信自己的耳朵,一下就定在那里,目瞪口呆。

这怎么可能?

鸡冠蛇的邪性我是早就有准备了,但是,它们再聪明,也不可能会说人话啊,当时刚才那话清晰无比,我绝对不可能听错——

我随即就感觉我肯定是幻听了,这是绝对不可能的事情,显然是我的神经太紧张了,出现了错乱,我咬牙就继续往下潜去。

那蛇居高临下的看着我,看我往下沉,忽然扭了一下脖子,好像在打量我,然后一下就俯了下来,挂到了我的面前,鸡冠一抖,又发出一声:“小三爷?”

这一次更加的清晰,而且那动作太像一个人在和我说话了,我的冷汗不停的出来,一下不敢动了,心说他娘的,这次真碰上蛇精了,真的是蛇在说话!

我的脑子几乎是完全混乱,无数的念头在一秒内涌了上来,这是条神蛇?过了人语六级,研究生毕业的蛇?这鸡冠蛇他娘难道真的有人性,或者这干脆已经是有思维的蛇了?

那一刹那间,我忽然想起我们现在是西王母的势力范围,靠那在古代这里就是仙境……蛇说话也不稀奇。

那蛇看着我表情变化,大约也是十分的感兴趣,又转了一下头,抖了一下鸡冠,道:“小三爷?”

这一下我是有心里准备的,所以听的比前两声清楚,一听,我忽然就意识到哪里不对,咦,这蛇说话怎么带着长沙口音?

难道,这是一条祖籍长沙的鸡冠蛇,到西王母国来支援西部建设?

那一刹那我脑子里闪过一个非常离谱的念头,我突然想问它:“你是不是湖南卫视派来的?”但是随即我脑子里灵光一闪,冷汗就下来了,逐渐就意识到了怎么回事情。

如果是这蛇真有过人的灵性,那它会说的也应该是西王母国当时的语言,但是这蛇现在说的竟然是普通话,而且是带长沙口音的,这显然太不寻常,普通话是50年代才开始推广的,长沙味的普通话更是70年代出身的人用的,这完全是现代的东西,这蛇就算有超人的智慧,他也不应该说出这种口音来。

那就只有一个可能性了,如果它不是在“说话”,那它必然是在“学话”,这蛇竟然和鹦鹉一样,学人说话!

我立即就冷静了下来,这肯定是这样,想象一路听到的声音,都只是在叫“小三爷?”,没有第二句了,而且连语气都一样,显然这不是有意识的行为。这长沙口音的普通话,就是潘子的口音,而潘子就是喜欢“小三爷”、“小三爷”的叫我,这三个字他重复的最多,这蛇肯定一直跟着我们,所以就学会了。

不过,鹦鹉学会说话是人的训练,这蛇学我们说话就很怪了,这显然不会是单纯的好玩,它学这声音必然是有理由的。

想到这里我的冷汗就直冒,想到了响尾蛇,这种蛇是通过模仿水流的声音来吸引猎物,这蛇说话,难道也是同样的目的?

一想狗日的,老子正不是给它吸引过来的,他娘的,这一次竟然上了蛇的档,真是丢人丢到家了。

那蛇打量着我,血红色三角的蛇头几乎离我的鼻子就一个巴掌的距离,我几乎能闻到到它身上一种辛辣的腥味,这些念头在我脑子里一闪而过,我就没法继续思考了,心说不管怎样,我面前还是一条剧毒蛇。

我缓缓的向后靠,想尽量远离,至少要远离到能有机会躲过它的攻击,然后想办法潜入水里。

然而,我稍微动作一下,那蛇就又猛的靠近了一点,死死的盯着我的眼睛,似乎知道我的意图,我退了几下,它就靠近几分,又不攻击我,只是和我保持了一个巴掌的距离,那低垂的蛇头让我浑身僵硬,不敢有任何大的动作。

我就感觉到十分的奇怪,它似乎只是想控制住我,然而这种行为本身就是十分古怪的,因为蛇是一种爬行动物,它所有的行为都应该是条件反射,它这么做没有任何的意义,它想干什么呢?

就在我纳闷又无计可施的时候,忽然我就感觉我的脚踝被什么东西碰了下,好像有什么东西,从水底潜了过来。

\chapter{第三夜:获救}

我不敢低头,但很快四周的水里冒起了气泡,我用眼睛去下瞄,就看到水下有一个白色的人状影子。

那影子几乎就是在我的脚边上,飘飘忽忽的我看不清楚到底是人是鬼。不过看那白影的动作,我感觉这确实应该是个人的可能性多一点。

是谁呢?

一边的胖子肯定不可能苏醒,潘子还在神庙中,就算他们两个过来,也不可能这么白啊。

我此时一点办法也没有,只有一边戒备着那蛇,一边静观其变。

那气泡在我四周冒了一圈,我就感觉到那人必然是抓住了水下的树根,我四周的树根晃动了一下,在水面上震起一片涟漪。

一下那蛇就警惕了起来,转头看了看四周,显然弄不清楚四周怎么会震动,它迅速的看了一圈儿,什么都没有看到,立即将头昂起,直立起来,发出了一连串高亢的犹如鸡叫一样的叫声。

瞬时间我感觉那蛇的鸡冠更红了,整个蛇身鼓了起来,简直感觉有血要爆出来,这不知道是一种警告,还是在召集同伴。

与此同时,我就感觉脚踝给人抓了一下,正抓在我扭伤的地方,疼的我一嘶牙,接着那人在我的小腿上划动了起来,似乎在写字。

这是小时候经常玩的玩意儿,我一感觉,就发现他写了一个“准备”,这准字我感觉不清楚,但是备字很明显,我心中一安,知道下面肯定是个喘气的了,立即动了动脚表示知道了,凝神静气,却不知道该准备些什么。

那蛇并不知道这水下的猫腻,叫了几声,看四周没什么反应,就慢慢软了下来,就在这个当口,我看到水下的影子一下浮了上来,还没等我意识到怎么回事情。突然我面前的水就炸开了,一个雪白的人猛的从水里窜了出来,以迅雷不及掩耳的速度一下就捏住了鸡冠蛇的脑袋。

我给那人一挤就脚下一滑摔进了水里,没看到接下来情况,我也不想看到,顺势往外一蹬,扑腾出去就向水池中央的方向游去,直游出三四米远才敢转身往回看。

只见那边水花一片,显然那蛇并不那么好对付,一时之间我不知道该自己逃跑还是旁观还是过去帮忙。还在犹豫,忽然一道红光就从那水花团里炸了出来,一下卷着树枝就绕到树枝堆上,同时发出了一连串极其凄厉的声音。

那白色的人立即对我大叫道:“快走,它在求救,等一下就来不及了!”说着一下就潜入了水里。

话音未落,四周的井道之中已经传来此起彼伏的,咯咯声,似乎有无数的人蛇在我们四周。

我一下慌了,忙追着那人在水里的影子就游,游了两下就想到胖子,心说不能把他丢下,再探出头去看胖子。却发现井道里,胖子已经不见了。

这可要了命了,只听的黑暗之中,大量的咯咯声越来越近,我转头两圈都看不到胖子在哪里,前面又大叫,想了几秒不由只能咬牙心头一叹,说对不住了,蒙头就追了过去。

那人游的极快,很快就在前面爬上另一个干涸的井道,一下就消失在了雾气里,我心中大急,心说这人到底是谁啊,到底是来救我的还是玩我的,跟着我也靠了边,这时候我已经完全不知道方向了,只是被那催命一样的咯咯声逼浑身发毛,直想立即爬上去。

爬了一下才发现我根本够不到那个井道,我简直欲哭无泪,大叫了几声,用尽全身的力气往上跳了几下,还是滑了下来,四周那咯咯之声已经聚集到了我背后,我用脑袋撞了几下树根,心里几乎绝望了,忽然一下我的手被人紧紧的握住了,接着就有人用力将我往上拉去。

我给扯到井道内立即就看到拉我的是一个带防毒面具的人,身后还有十六七个同样装扮的大汉,六七盏强光手电照的四周通亮。我正想问你是谁,那人就扯开了防毒面具,一张熟悉的老脸露了出来。

“三叔。”我一下惊叫起来,可还没叫完,三叔一个巴掌就打了过来,几乎没把我打蒙了,随即就有人递上来一个防毒面具,立即给我按在了脸上。

我给架起来,就看到三叔重新带上防毒面具,一挥手,立即就有人拧开一种黄色的烟雾弹,往水里丢去,其它人架着我,迅速往井道的深处撤了进去。

\chapter{第三夜:入口}

给三叔打的眼冒金星,倒也没什么脾气,自己搞成这个样子也实在不敢说什么,只得乖乖给人架着往深处退,在狭窄的井道中被拖曳绕过几个碗,就到了一处分茶口,我被扯了出去,发现下面也是和刚才同样的干涸井道,但是更加的宽,看来经历过坍塌,有巨石横亘在井道底部,上面有大量枯萎的树根,我抬头看了看上面,心说上面应该就是地面上的废墟,巨石上,我看到还有几个人在等着我们。

我走上去,一下就看到被裹的严严实实的胖子混在里面,还是昏迷不醒,有人正在给他打针,一下心头一放,暗叹一声上帝保佑,看来在那白色的人救我的时候,另外有人救走了胖子。这王八蛋也算是命大了。

同时也看到那个浑身白色的人坐在朽木上,也带上了防毒面具,缩在树根之间。那一身白色的皮肤在水里看着雪白,上面来看却十分的奇怪,好似发黄的一般,我仔细一看,就发现那是一套看上去非常旧的潜水服。

再一看其它人,几乎也都穿着潜水服,不过都是新的,显然三叔的准备相当充分,在这里有潜水服会舒服很多。

那人没注意我,我想到刚才几乎没看到他的样貌,心说这真是大恩人,要好好谢谢他,被人架着到他面前的时候,我就想道谢,结果那人头转过来,我就从防毒面具的镜片里,看到一副十分熟悉的黑眼镜。

我一下目瞪口呆,心说竟然是他!不由哎了一声。

他抬头看到我,好像是笑了,就向我点了点头。

我点头,刚想道谢,一边的三叔就走了过来,我给拖到三叔的面前,他蹲到了我的面前,打量了一下我,就叹了口长气:“你小子他娘的~怎么这么不听话?”

我感觉有点尴尬,事情搞成这个样子,实在是始料不及,也不知道怎么说了,想叫他,又被他做了个手势拦住了,他坐下来,也没责备我,只是立即轻声用长沙话问我道:“你别说话,我问你,潘子和那小哥呢?”

我立即就把刚才我们经历的过的事情说了一遍,三叔听了就“啧”了一声:“想不到这死胖子这么机灵,这一次也中招了。”

“怎么了?”我听他这么一说,心里也不舒服。

“这里的蛇太邪门了,会学人话,它的鸡冠能模仿听到的声音,把你引过去,老子们差点给它们玩死。”一边一个伙计道:“在这鬼地方,你听到什么声音都不能信。”

我看了看胖子,就问三叔:“那家伙怎么样?没事情吧?”

“已经给他打了血清,接下来只能听天由命了。”三叔看了看手表,对我道:“快把衣服脱了。”

“脱衣服?怎么了?”我心说干嘛,他们已经自己动手了,一下我的上衣就给扯掉,我给按在井壁上,衣服一脱下,我立即就听到一声轻声的“我靠,真有!”,不知道是谁发出的。

我一下懵了,冷汗就下来了,这是什么意思?我背上有什么东西?就想转回去看背,却一下给按住了。

“别!别动!”三叔轻声道:“就这么站着!”

我开始起鸡皮疙瘩,就去感觉自己背上,但是仔细感觉,什么也感觉不到,那味道似有非有,难受的要命。

“我的背上是什么?”我问道,才说了一句听到三叔又嘘了一声:“我的祖宗这时候你就别好奇了,你等会就知道了。”接着我就听到了火折子的声音。

“搞什么?”我心叫起来。心说他难道想烧个尽忠报国出来吗?

想着我就感觉背后烫起来。还没来的及做好准备,一下我忽然就感觉到背脊上有东西动了,接着我们都听到一连串叫声从我背后发了出来。让我毛骨悚然的是,那声音听着竟然像是婴儿的声音。

没等我细琢磨,三叔就下了狠手了,我一下就感觉一团巨烫的东西在我背脊上连戳了几下,烫的我几乎跳起来,同时那诡异的叫声也尖锐起来,接着那在我背上动的东西就滑落下来,那感觉就好像一团泥鳅从你背上倒下来。

“下来了,快走开!”不知道是谁轻叫了一声,我忙站起来,但是脚不知道为什么软了,竟然没站成功,踉跄了一下,回头一看,就见好几条铅笔粗细的白色的东西犹如肠子一样挂在我的脚踝上,我往后一缩脚将它们踢掉,然而一刹那那些东西都动了起来,我清晰的看到那小毒牙在它们嘴巴里张了开来,朝我的小腿就咬了过来。

就在那一刹那,边上有人出手,只见黑光一闪,一块石头就砸了下来,把第一只砸死,接着乱石拍下,瞬间这些小蛇的脑袋全部被拍扁了,变成一团浆糊。

我缩起脚来一看,抹掉脸上的冷汗,就看到那是一条扭曲的好比肠子一样的蛇,白花花的,就剩个身子,在不停的翻滚扭动。一下感觉到我背后的粘液顺着脊背低落下来,我坐倒在地上就干呕了起来。

三叔对着蛇又补了几刀,把它们砍成两截,才松了口气,他顺手把衣服递给我:“擦擦干净穿上,把领口和裤管都扎紧了。”

“这……这……这是怎么回事?”我摸着后背道。发现那都是一条条很小的鸡冠蛇,但是这蛇不是红的,而是白色的。体型也非常小。

“这是刚孵出来的小蛇,皮都还没硬呢,你刚才在死人潭里呆过,那里泥下面其实全是这种小蛇,有东西经过肯定会被附上,我们之前几乎每个人身上都有。”一人道:“这蛇用牙齿咬住你的皮,你只会感觉痛一下,接着你的背就麻了,被皮鞭抽你都没感觉,然后他就慢慢往你皮里面钻,吸你的血,等它长大了,毒性大到把你毒死了,才从你皮里出来,这时候浑身都吸饱了血,皮就成红的了。”

我看着那蛇,心有余悸,心说刚才是怎么到我背上去的,我怎么一点感觉也没有。

这么恶心的东西,钻入我的衣服怎么说也应该觉到有点异样,不可能不知道。

一想,我刚才在水里总觉得脚踝在被什么东西咬,难道就是那个时候,这些蛇在偷偷爬上来?想着摸了摸自己的后背,全是黏液,恶心的要命。

我用衣服搽了搽,又有一批人从井道口退了回来,看到三叔就摇头,轻声说:“三爷,那边也根本不通,没法出去,怎么办。”

三叔站了起来,想了想就叹了口气,点了点头,对他道:“没办法了,这里不能再呆下去了,我们得回去,只有明天再出来。”说着又骂了我一声:“让兄弟们出发。”

那人点头应声,就对四周的人打了个呼哨,那些人全部站了起来。立即背好了装备。

我也被人扶了起来,三叔看我似乎有话要说,就对我说:“有什么话回到我们落脚的再说,这里太危险了,在井道里里别说话,知道吗?”

我明白他的顾虑,点头表示知道了,他们立即就出发,往井道深处退去。

一路跋涉,我完全不知道自己是在朝什么地方走,只知道四周的能见度极低,不时能听到四周的岔道深处忽然就传来一声“咯咯咯咯咯”的声音,非常近,非常的高亢。显然,这里是它们的地方,到处都有蛇在我们的周围。

我有点紧张,然而这里到底是人多,有蛇一叫,立即就有人警戒那一个方向,这多少让我安心。看来人果然是需要安全感。

也不知道走了多少时间,期间路过了两条有水的井道,我估计最少也有一个小时,我开始听到寂静的井道里出现了一种声音,很熟悉,而且是一点一点逐渐出现的,我想问,但是其它人一路都不说话,连咳嗽声都没有,也就不好意思发出声音。

随着深入井道,温度逐渐降低,又走了一段距离之后,我们开始经过一些破坏严重的地方,上面还能看到干涸的青苔和藤蔓的痕迹,有些上面还有活的树根,这是上面的树根盘绕在石头的缝隙里长到了下面。我们肯定这一段路是靠近地面,也许随便那块石头一捅就能看到阳光了。

我们从吸附在井壁上的庞杂树根中爬过,依稀可见其中有一些已经腐烂的发黑的蛇蜕,人知道这里应该是蛇活动的活跃区域,我想想也可怕,这如此复杂的下水系统,估计都可以和古罗马比上一比,没想到竟然变成了一个巨大的蛇巢。

这种生物防御的技术,在西域算是高科技了,不知道当时这个国家为什么没有继续称霸下去,我感觉有可能是终于有一个国家发现了对付这些毒蛇的方法。

猫腰走了好久,一直到我有点头晕,我们才到达了目的地,我老远就看到了隐约的火光,逐渐走进,发现那是一个巨大台阶似蓄水池,有六到七个梯田一样的相连的水池组成,四周能看到石柱,石梁,这好像是当时罗马浴场一样的地下建筑,爬了下去后,又发现了四周的整片岩墙上,有大量的石窟,石窟很深很大,好像一个个石头方洞,而且似乎都有通道和石头台阶相连,在石窟与石窟之间形成了一道一道的走廊。

于是又感觉也许是一座用以宗教的神庙场所。不管怎么说,这里就应该不是单纯的蓄水池。因为这里有人类活动的迹象。

火光就是来自石窟之中,我们过去,走上一条台阶,穿过几个石窟之间的通道,进入到了一个比较宽敞的石窟内,足有六七十方大。

我们进去就看到了帐篷,睡袋和大量的装备,凌乱的堆放在里面,里面有两个人坐在篝火边上,应该是看火的,背对着我们似乎没有注意到我们回来。

一行人全部走的筋疲力尽,脚上简直没有一点力气了。

我给人放下来,单脚就跳了几下,托着我那人累的够呛,揉着肩膀就去踢了看火的那两人一脚,道:“还不起来给小三爷让坐,木头似的杵着像什么话。”

我刚想说不用这么客气,那两人忽然就倒了下来,翻倒在地,我们一看,不由倒吸了一口凉气,这两人脸色发黑,双面圆睁,显然已经死了。

\chapter{第三夜:避难所}

长途跋涉,我累得筋疲力尽,看到眼前的情形,都有点反应不过来,只是条件反射地往后退了几步,心力交瘁得似乎要晕过去了。

然而四周的人看到我的样子,却都笑了起来,接着就有人将那两具尸体扶了起来,我这才发现,那两个原来是假人,是往潜水服里不知道塞了什么东西,而那两个的脑袋是两个吹了气的黑色防水袋,上面贴了两片拍扁的口香糖,中间还粘了两粒石头当眼珠,因为防毒面具的镜片模糊,加上神经敏感,乍一看还真是那么回事。

当下我自己也失笑,扶起假人的人就把假人移到石门处,我就问边上的人,这是干什么?

一个人就对我道:“吓唬蛇用的,这里的蛇他娘的太精了,只要人一少就肯定出事情,所以我们不敢留人下来看营地,不过好像它们还分不清楚真人假人,把这个堵在门口,晚上能睡得踏实点。”

听那人说话的语气,显然深受这种蛇的危害,接着有人拿出刚才的那种黄色的烟雾弹,丢进篝火里,一下子浓烟腾起,另外有人就用树枝拍打放在地上的装备。

“这是硫黄,用来驱蛇的。”那人继续道。

拍打了一遍,似乎没有什么动静,这些人才七倒八歪地坐了下来。

有人从一边的装备里又拿出几个用树枝扎起来的,简陋一点的假人,把自己的衣服脱掉,给假人披上,然后都堆到了门口,和坍塌的口子上。

搞完之后,气氛才真正地缓和下来,黑眼镜往篝火里加了柴火,然后分出去几堆,这小小的遗迹之内的空间被照得通红通红,四周的人陆续摘掉了自己的防毒面具。有个人看我不摘,就示意我没关系,说这些蛇非常奇怪,绝对不靠近火,加上我们刚才查过了,基本上没事。

我只好也摘掉防毒面具,戴了六七个小时,脸都快融化了,一下清爽多了,眼前的东西也清爽起来,我也得以看到三叔那些伙计的真面目。

一打量就知道潘子说得是不错,除了两三个老面孔之外,这一次全是新鲜人,看来三叔的老伙计真的不多了。

我们纷纷打招呼,有一个刚才给我解释的人,告诉我他叫做“拖把”,这批人都是他带来跟着三叔混的。

我听着他的语气有点不舒服的样子,不过又听不出来哪里有问题。

黑眼镜还是那副悠然自得的样子,乐呵呵地看看我,拿出东西在那里吃,很多人都脱了鞋烤脚,一下子整个地方全是酸脚气,我心说:他娘的就这味道,不用假人那些蛇也进不来啊。

正想着,三叔坐到了我的边上,递给我吃的东西,我们两相对望,不由都苦笑,他道:“你笑个屁,他娘的,你要不是我侄子,老子真想抽死你。”

我连和他扯皮的力气都没有,不过此时看三叔,却发现他一扫医院里的那种委靡,整个人神采奕奕,似乎又恢复了往日的那种枭雄的本色,不由有些释然,道:“你就是抽死我,我做鬼也会跟来。三叔,咱们明人就不说暗话了,你侄子我知道这事情儿我脱不了干系,要换是你,你能就这么算了?”

三叔应该已经知道我跟来的来龙去脉了,点起一支烟就狠狠吸了一口,还是苦笑道:“得,你三叔我算是认栽,你他娘的和你老爹一个德行,看上去软趴趴的,内底里脾气倔得要命,我就不和你说什么了,反正你也来了,我现在也撵不回去。”

我是粲然一笑,就问他道:“对了,你们是怎么回事?怎么会到我们前面去了,潘子不是说你们会在外面等信号的吗?”

“等不了了,您三叔知道文锦在这里等他,而且只有这么点时间,怎么可能还等你们的信号。”边上的黑眼镜笑道,又拍了拍三叔的肩膀,“三爷,您老爷子太长情了,咱在长沙唱K的事情可看不出来您有这种胸怀。”

三叔拍开他的手,瞪了他一眼,解释道:“我当时听了那老太婆和我说,文锦在前面等我们,就意识到这可能是我这辈子见文锦的最后一个机会了。我无论如何也不能错过,否则,你三叔我这辈子真的算是白活了,所以我一点险都不能冒,说实话,你三叔我只要这一次能见到文锦,就是马上让我死也愿意了。”

我听了一激灵:“等等,听老太婆说?”一下意识到他指的是定主卓玛。心里一晕,心说“不会吧”,“这么说来,她……也……给你传口信啦?”

看着我莫名其妙的脸,黑眼镜就“咯咯咯咯”笑了,也不知道在笑什么,三叔点头,就把他和黑眼镜会合的情形和我说了一遍。

原来,三叔的进度比我们想象的快得多,潘子带着我们刚出发不到十个小时,三叔他们已经赶到了魔鬼城并得知了情况,就在他认为事情一切顺利的时候,在当天晚上,定主卓玛竟然也找到了他,也和我与闷油瓶在当时遇到的一样,传达了文锦的口信。

三叔不像我们那么老实,他立即追问了定主卓玛更详细的信息,定主卓玛还是在和三叔玩神秘,但是三叔岂是那么好脾气的人,加上他一听到文锦还活着的消息立即就抓狂了,立即叫人把扎西和定主卓玛的媳妇放倒,具体过程三叔没和我说,然而显然是来了狠的,威胁了那老太婆。

道上混的做事情的方式真的和我想的很不一样,这事情我是做不出来的,虽然我不赞同三叔的做法,然而这肯定是有效果的,那定主卓玛只好透露了文锦交代他口信的情况,并且把我和闷油瓶也得到口信的事情和三叔讲了。

“她说当年她和探险队分开之后的一个月,她在格尔木重新碰到了文锦,当时的文锦似乎经历了一场大变,整个人非常憔悴,而且似乎在躲避什么人,她把文锦带到家里住了一晚,就在当天晚上,文锦把录像带交给了她,让她代为保管。”三叔道,“之后的十几年,她们之间没有任何联系,一直到几个月前,她忽然收到了文锦的信,让她把三盘录像带分别寄到了三个地址,并告诉她,如果有收信人上门来询问,就传达那个口信。”

知道文锦在它木托后,三叔几乎疯了,立即起程找到了这片绿洲,因为我们的车胎爆了,最后几天进展缓慢,他们就是在这个时候已经超过了我们,进入了绿洲之内,但是他们进的是和我们不同的入口。

之后他们连夜在雨林中行进,在那片废墟上扎了营地,当晚三叔带人出去寻找文锦,回来的时候,剩下的人全不见了,三叔就知道出事了,在第二天早上他们发现了我们的信号烟,三叔就打起红烟让我们不要靠近,自己带人去四处寻找,一路就被那些蛇引诱着,最后也找到了那个泥潭,接着,他们就听到了兽口之下有人惊叫,于是立即进入救人,没想到,那些声音竟然是蛇发出来的。

之后的事情,就不用叙述了。

听完之后,我不禁哑然,这和我想象的情况差不多,我当初看到文锦的笔记前言,就有感觉其中肯定有三叔的份。不过证实了,却反而有点不太相信。

如此说来,定主卓玛对更深的事情也并不知情。她被阿宁他们找到,重新雇用做向导,完全是一个意外,否则,我们听到口信的地点,应该是她的家里。

我脑子里的线越来清楚,一些碎片已经可以拼接起来了:文锦的笔记上所说的三个人,显然应该是我,闷油瓶和三叔。我之前以为阿宁收到了带子,之前也证实是给闷油瓶的,闷油瓶这一次和三叔合作,将带子送到了阿宁的手里,是为了让阿宁他们能够找到定主卓玛,并策划这次行动。

所有事情的矛头,就直指向裘德考这一次行动的目的了。这一次,大家全是最后一搏,几乎用尽了心机。

想着,我忽然想到了什么,问三叔道:“三叔,既然你也收到了口信,那你不是也应该收到了一盘录像带?”

三叔抬眼看了看我,把烟头丢进篝火里,点了点头:“对。”

“果然!”我心道。

“这盘录像带,应该是咱们在吉林的时候寄到杭州的,我不在的这段时间堆了一堆的东西,混在里面,我刚回去没发现,后来整理铺子的时候,才看到。”他看着我说,“并不是我有意瞒着你。”

我点头,这我确实相信,这时候心里一冲动,就问三叔道:“三叔,你不觉得这事情奇怪吗?寄给你,或者寄给那小哥,这都说得过去,可是,文锦姨为什么要寄给我呢?你们谈恋爱的时候,我还很小很小,我实在想不通,这事情难道和我也有关系?”

而且,录像带中还有那样惊悚的内容,那个人真的是我吗?还是只是别人的恶作剧?

三叔看我表情变化,叹了一口气道:“不,其实,你文锦姨把东西寄给你,是有她的理由的。”

\chapter{第三夜:录像带}

“什么理由?”在篝火的温暖下,我的疲惫逐渐的减轻,身上的伤痛袭来,整个人没有一处不疼,然而我并没有在意那些不适,注意力集中到了篝火边的三叔身上。

火光下的三叔显的阴沉,他又吐了一大口烟儿,才继续道:“我说了你能相信吗?”

他看着我,我也哑然,显然,我是不可能信的,之前在医院我发了誓绝对相信他,但是我食言了,然而三叔也没有说实话,我们之间的博弈似乎进入了一个死循环,在这种情况下,三叔任何的解释都是徒劳的。

他沙哑的笑了笑,就道:“如果我要骗你,那是我有非骗你的理由不可,那必然会一直骗到到最后一刻,我料准我说了你也不会相信,与其浪费我的力气,还是等我们找到她,你自己去问她吧。”

我长叹一口气,忽然感觉一下子和眼前的这个以前如此亲密的叔叔产生了莫大的距离,我有点控制不住道:“三叔,我真不想这样,我也想回到咱们以前,您说一我绝不说二的时候,不过,现在我真的看不透你,咱们就不能再扯皮一回?您就让让您的大侄子。”

三叔看着我,又点起了一只烟道:“大侄子,这是最后一回了,我保证,我太累了,这一次,真的是最后一次了。”

我们两个人相视苦笑,两相无话,我心里非常难受,不知道是什么滋味,总感觉一个不可化解的死结在我心里堵着,而且不是麻绳,是钢筋的死结。

静了一会儿,一边三叔又对我道:“其实,我和你说过很多次了,这件事情里面的水太深了,牵扯的秘密太多了,我自己都不清楚这到底是怎么一回事情,所以,你三叔我其实还是挺能理解你的感受的。”

我心说你理解个屁,你就算知道的再少,也肯定比我知道的多,我们两个在这件事情里,所处的位置是完全不同的,你是在事情的中心,而我现在怎么说也只是在外面看着,连进去的门都找不到。

不过多说无益,即使是这样,我也走到了这一步了,我看了看外面黑漆漆的一片的地下水池,不想再去想这些事情,反正我已经跟着他了,除非他把我杀了,否则我一定要跟他到底。

喝了几口辣椒茶去湿,我的扭伤的地方开始发作,我一边揉着,就转移话题道:“对了三叔,文锦姨寄给你的录像带,是什么内容?”

三叔站了起来,让我让开,从他行李里,拿出了他的手提电脑。“我没法来形容,你自己看吧。”

我自然是想看,但也想不到三叔会这么主动,他将手提电脑放在自己的背包上翻开。原来他将录像带里的内容转到了磁盘里面。

“我让一个伙计把录像带转成文件了,花了三百块钱,我自己看了很多遍,根本看不出什么来,你不要抱太大的希望。”说着,已经点开了文件。“快没电了,你将酒着看吧。”

屏幕上跳了出播放器,我看了看四周的环境,忽然感觉这情形有点奇怪,这里是什么地方,我竟然还在看手提电脑,这时代果然探险的性质也不同了。

三叔显然不想再看,电脑给了我就走开了,一边似乎是有人发现了什么,让他去看一下。黑眼镜就凑过了过来,坐到我身后,好像准备看电影的姿态。

这人让我很不自在,我看了他一眼,他根本不在意,我看他他也看看我。

我没办法,暗叹一声这是什么人啊,只得换了个舒服的姿势,点了播放,开始仔细的看屏幕。

播放之后,先是一片黑暗,接着扬声器里传出了非常嘈杂的声音,十分熟悉又感觉不出是什么,听了一会儿,我才听出来,原来那是水的声音。

屏幕是黑色的,看不出哪怕一点的光影变化,但是扬声器里的水声,却告诉我们,里面的内容正在播放当中,夹杂着远远的几声闷雷,可以想象,这卷录像带在拍摄的时候,应该是在湍急的水流旁边,或者附近有着小规模的瀑布,可能是镜头盖没有打开,或者遮了雨篷的关系,屏幕上什么也没有拍到。

水声一直持续,忽远忽近,应该是摄像机在运动当中。

大概播放到了五分钟左右的时候,我听到水声之外的声音,那是几个人喘息声和脚踩在石头堆里那种脚步声,脚步声很凌乱,而且很慢,听的出那是几个人蹒跚的走动,但是这几个声音只出现了一下就又消失了,接下来还是水声。

我有点意外,第一盘带子我在吉林收到,里面是霍玲在格尔木的那座诡秘的疗养院的地下室里梳头的情形。

第二盘带子是阿宁带来的,里面是一个相貌和我极度相似的人,在那座格尔木的疗养院的大堂里爬行。

我以为第三盘带子至少也应该是那疗养院的内容,然而,如今看上去,好像是在室外拍的。

我立即就想起了我们来的时候的那一场大雨之后,丛里里出现湍急溪流的情形,难道这里面录的是当年文锦的队伍进入峡谷时的情形吗?这可是重要信息。

继续听下去,接下来还是水的声音,忽远忽近,似乎是摄像机又开始运动。

我之前看的两盘带子都是这样,非常枯燥,所以我心里有数,并不心急,另我吃惊的是,一边的黑眼镜竟然也看的津津有味。

又耐心的听了大概二十分钟,水声才逐渐舒缓下来,从那种嘈杂的磅礴,慢慢变成了远远的在房屋里听出去的那种水声,同时几个人喘息的声音又再次出现,这一次清晰了很多,而且还夹带着鸣声,感觉是几个人找到了远离水的地方,这个地方还是一个比较封闭的空间。

然后,我们听到了整卷录像带里第一句人的声音,那是一个女人的声音,她似乎精疲力竭,喘着气道:“这里是哪里?我们出去了没有?”

没有人回答她,四周是一片的喘息声和东西放到地方的撞击声,屏幕上一直是黑色的,不免有些郁闷,但是听声音又不能快进,只得忍着集中精神。

那个女人说话之后很长一段时间,都是装备放到地上和咳嗽,叹气的声音,很久后才有另一个男人说话,也不是回答他,而是问另外一个人:“还有烟吗?”

这声音很远,类似于背景音,如果不仔细听是听不懂的,让我印象深刻的是,这个人的声音,带着闽南的口音。

同样没人回答他,我们也不知道他要到烟没有,但是接着我们听到了很响的一声金属落地的声音,然后是那个讨烟的男人骂道:小心点。

之后是沉默,好像是摄像机朝外面挪了挪,还是拿着摄像机的人又回到了湍急的水流附近,水声又大了起来,不过没几分钟,又恢复了回来。那个刚才讨烟的声音道:“我们到底再往哪里走?”

没有人回答他,一切如旧,进度条一点一点的往后跳,屏幕一直是黑色的。

我耐心的看着,时间一分一秒过去,慢慢的,连我自己也感觉不耐烦起来。就在我是在忍不住,想去把进度条往后拉一点的时候,一边的黑眼睛把我的手按住了。

我心中奇怪,心说他干嘛,忽然扬声器里一下传出了比较连贯的话语,那是一个西北口音极重的人说的话,他似乎被吓了一跳,叫道:听,有声音,那些东西又来了!

接着是一片骚动,再接着就是那个闽南口音的人低声喝道:全部别发出声音!

这些人似乎训练有素,那口音一落,整个扬声器里突然一片寂静,所有人的声音瞬间消失在背景的水声中,这一静下来,我就听到那水声中,果然有了异样的声音,只是和水声混在一起,根本听不清楚。

我的神经一下子绷紧了,忙凑到扬声器的边上,只觉得那异样的声音自己肯定在哪里听到过。

果然,那声音由远及近,我越挺越觉得似曾相识,听着听着,我的身体竟然不由自主的发起抖来。一股让我发炸的毛骨悚然从我的毛孔里直发出来。

我想起了这是什么声音了。

这是闷油瓶进那青铜巨门之前,那地下峡谷深处想起的号角声。

(《蛇沼鬼城篇》完)

◆ 第九卷 谜海归巢 ◆

\chapter{集结号}

我听的浑身冰凉:绝对不会错。这就是青铜门打开之前,响起的号角声。

当时的诡异经历,只有我和胖子亲眼看见,如今想起来也是历历在目,又听了几遍就完全想了起来,确信无疑。

早先两盘带子的情形诡异非常,我已做好心理准备,我的神经已经足以能应付了。稍微定了定神,我就从毛骨悚然中摆脱了出来,心中不由长叹。

有可能这卷带子,是文锦他们在长白山底青铜巨门的地方拍的。而且听声音,他们有可能在往那地下峡谷的尽头走,甚至,这可能他们已经在青铜门之内了。

凭借几句对话,我几乎就能想象当时的情形,这号角声响起,那些马脸的怪物肯定出现了,这录像带的人似乎非常忌讳这些东西,马上闭声隐蔽。而且,听语气,他们应该遇到不止一回了。

这又是一片线索的碎片,由此看来我和胖子遇到的事情应该不是一个特例,那时候也绝对不会是我们的幻觉。不过,暂时这片碎片我还不知道应该往哪里拼。

我继续听下去,号角声响了一段便逐渐平息了下去,喇叭中全是水声,我期待着之后会发生什么,但是我发现此时播放器的条栏已经接近尾声了,后面似乎没多少内容了。

我耐着心思听了下去,果不然,几分钟后带子就结束了,屏幕上还是漆黑一片,什么都没有,确实如三叔说的,什么都看不出来。

我重新听了一遍,仔细的寻找其中新的线索,生怕有一丝遗漏,但是没有任何新的收获,我相信三叔的这种性格,必然也研究的相当仔细了,他说没有就肯定不会有了。

合下笔记本我就头痛,看来,从这录像带里想找什么线索是不太可能。想必文锦寄这些带子的时候,也没有想过看带子的人会怎么样,这些内容也许不是主要的。

一边的黑眼镜看我的样子,就很无奈的笑笑,拍了拍我的肩膀,起身坐到我对面。

四周已经传来了鼾声,显然有人已经睡着了,剩下的人也只有偶尔的窃窃私语,篝火的温度,火光和柴火的啪啪声让我心里很放松,之前的那一段跋涉太累了,眼前的景象一时间我还无法习惯。

我本来也非常的困顿,然而给这录像带一搞就精神了,想逼自己休息一下,却发现脑子不受空子的胡思乱想。这时候三叔满头污泥的走了回来,走过身上竟然带过一丝尿味,但是看脸上带着一丝异样,不知道刚才做了什么。

他看我已经合上了电脑,就问我怎么样?

我摇头说没头绪,确实是没头绪,光听声音,可以配上任何的画面,这带子对于了解事情其实基本没帮助。

三叔早就料到,叹了口气也没说什么,我就问他怎么了,怎么搞成这样。

他道:“有一个伙计发现了一些有趣的东西。”指了指其中一个渠口。我一看,那里是他们选中用来撒尿的地方,难怪这么臭。三叔这德性,难道刚才竟然钻进去了?

三叔说那东西就在这渠口的下面,“太脏了。”他指了指身上的污泥和苔藓。说着他就踢了几个睡着的人,让他们爬起来准备绳子。

我走过去就发现这个渠口往下比较深的部分,因为废墟崩塌时候的巨大破坏,里边砖石扭曲了,水渠四壁石块全部移位,渠壁上塌出了很多的豁口,露出了后面的砂土,砂土层同样也裂开着一条非常宽缝隙,因为几乎是垂直往下的,三叔的伙计就临时把那当小便池。

这里的戈壁地质应该砂土,这里有点深度了,土质应该比较坚硬,那条缝隙直接裂进砂土层里,可能是地震的时候照成的,一路过来经常能看到地震的痕迹,显然这几千年来这里已经经历过好几次浩劫,有这样的痕迹在并不奇怪。

三叔说的有意思的东西,应该就在里面,但是我什么都看不清楚,裂缝几乎就是一个人宽,手电光照不进去。

那几个人身体素质显然极好,醒了之后只几秒就清醒了过来,三叔把事情一说,他们二话没有立即准备。我看他们的样子,似乎打算要下去。

我立即就觉得非常不妥当,这缝太窄了。就这么下去前胸贴后背都不行,还得缩起来才能,而且缝隙的内部非常的不光滑,指不定到哪里就卡住了。

“原来这缝外面有一层砂泥,我对着滋尿泥就冲垮了,这缝才露出来。”有一个伙计道。

黑眼镜捂住嘴巴,扇掉尿烧气道:“你最近火气挺大啊。”

“这不折腾这么久了,脑袋别着裤腰带上也不知道能熬到什么时候,火气能不大吗?”那伙计苦着脸。

三叔盯着那缝隙就道:“入这行就别这么多废话,钱好赚还轮得到你?收拾收拾,帮我提着绳子,我和瞎子下去看看。”

我立即拦住三叔道:“这种缝隙之中很可能会有蛇,那么狭窄的环境,遇到了蛇连逃也没办法逃,你干嘛这么急,要么等到天亮?”

“你这书呆子,这里他娘的又照不到太阳,天亮了不还得打手电,一样。”三叔道,一边的伙计已经结好了绳子。三叔显然要自己下,系在了自己身上。

我越发感觉不妥当道:“可以让伙计先下去探探,你一把老骨头,这时候逞什么能?”

三叔就很古怪的笑了,似乎很是无奈,先是拧开那种硫磺烟雾弹,往里面一扔。然后接过矿灯。“你三叔我有分寸,下去马上就看一下,立即回来。”

接着一边的黑眼镜已经穿上了紧身服,他做三叔的策应,拿着硫磺弹,和三叔一根绳子而下。

我在上面看着提心吊胆,这渠井的口子并不狭窄,但是倾斜的角度很大,看着三叔和黑眼镜拉着绳子一点一点溜下去,进入黑暗,越来越远,我总感觉要出事情。

然而显然我多虑了,那距离似乎比我想象的扼要近,才几分钟他们已经到那个地方。缝隙就在边上。

上面的人停止放绳子,这时候几个影子叠在一起,我们已经基本上看不清楚他们在干嘛了。只看到手电曳光晃动,滑过石壁产生了的光影。让我恍如看到海底墓穴天道里的感觉。

他们停顿了一会儿,黑眼镜就往上打了信号,看到信号,那几个拉绳子的伙计都愣了一下。

我问他们是什么信号。一人道:“三爷说,他们还要继续往下。”

三叔在下面,我们不敢大声叫喊,所以也没法问原因和状况,而这批人自然是唯三叔马首是瞻,我也不能阻止,只能暗自骂娘。心里又痒痒起来。

显然三叔在下面有了新的进展,否则不可能做这么武断的决定。

绳子继续往下,就看到他们并没有垂直,而是往砂土裂出的缝隙里爬了进去,两人进去的非常勉强,很快我们就看不到三叔的任何影子了,只看到有光从缝隙的最深处不时的闪出。

连拉绳子的人都开始冒了冷汗,一边没睡着的人全围了过来,气氛自然而然凝重起来。

在上面大概等待了有一个小时,三叔才从下面发来信号,上面的人都等的石化了,马上拉绳子,逐渐的黑眼镜被拉了上来,然而却不见我三叔。

我心里咯噔一声,刚想说话,就听那满身的泥味和尿味的黑眼镜对我道:“小三爷,三爷说,让你马上下去。”

\chapter{深入}

我的身体素质在这里的人中是最差的,本来是打死都不应该动的,三叔知道这一点,但还是让我下去,显然不会是让我做体力活,我想肯定有他的理由。但是闻着这渠井的味道,我实在是不想下去。

不过这是不可能的,所有人都看向我,一方面对这下面的情形非常的好奇,一方面黑眼镜也说得一点余地也没有,我无法拒绝,只好由黑眼镜护着,顺着裂缝降了下去。

大概是心理因素加强了我的错觉,下到下面之后,立即我就闻到了一股浓烈的尿骚味,浓的让我无法呼吸,而且这渠道也没有我想的如此好走,角度非常大,看着三叔这么平稳的降下去原来是用了死力气的,滑了一下,立即我的身上粘上了大量的混这尿液的烂泥和苔藓。不由直皱眉。

在我上面的黑眼镜就笑道:“不好意思,哥们,不过尿对皮肤好。”

“他娘的,还好你没让他们往这里拉屎。”我骂道。

他呵呵地笑起来,上面的人听到,以为出了什么事情,绳子停了一下,他马上往上打了信号,让他们继续放绳。

四周很快就一片漆黑,因为这里太过狭窄,连头都没法抬,所以除了黑眼镜的手电,我什么也看不见。好在是下降,如果爬上来更累。

我看着他还是戴着黑眼镜,就忍不住问他道:“你戴着那玩意能看得见吗?”

他朝我笑笑:“戴比不戴看得清楚。”

我不知道他是什么意思,不过他不想解释,也就不再问什么。

一路往下,很快就到了刚才上面看到的砂土裂缝的口子处,照了一下立即就发现其中别有洞天,里面是一条只能一个人前胸贴后背横过去的缝隙,但一进去就能发现缝隙虽然非常狭窄,但是极深,而且往上下前方都有发育,看上去好像是一座巨大的山被劈成两半,而我爬进了劈出的刀缝里的感觉。

而且让我吃惊的是,缝隙壁上都是石窟上的那种佛龛似的坑,就是把整块砂土的裂缝壁砸出了一个个凹陷来,每个凹陷里都是一团干泥茧,用烂泥黏在凹陷出,和四周的根须残绕在一起。泥巴都开裂了,好像干透的肥皂。

往上下左右看看这种凹陷到处都是,一溜照去,缝隙深处只要有手电光照的地方都有。

我们挤进缝隙中,我摸了一下里面的砂土,发现硬的好比石头,这些应该是砂土沉积下的土质,非常潮湿,富含有水份,再往里挤进去,一下我就下到一个泥茧的边上,我想去摸一下,但是黑眼镜喝了我一声,不让我碰,说:“小心,不要碰这些泥茧。”

“这些茧里面是什么?”我问道。

“死人。”他照了照其中一只,那是一只已经破裂的泥茧。里面露出了白色的骨骼,“曲肢葬,这里可能是当时的先民修建的最原始的井道,没有石头,只有泥修平的一些山体裂缝,后来被当成墓穴使用了。”

“墓穴?这种地方?”我纳闷着。

“修这种工程肯定会死很多人,这些可能是其他国家俘虏来的奴隶,死在这里,不可能运出去埋,就就地掩埋,长城边上就有不少。”黑眼镜就道,“到了。”

我往下看去,这缝隙远没有到底,但是在缝隙一边的石壁上,巨石继续开裂出了一条缝隙,有手电光在闪着,显然三叔就在里面。

黑眼镜往上打了信号,绳子停住,我们小心翼翼地攀爬下去,三叔就伸手出来把我拉了进去。

这一条缝隙十分的狭窄,最要命的是十分的矮,大概只有半人高,我只有毛着腰进去。脚疼得要命,一进去就坐倒在地上。接着黑眼镜也毛着腰进来了。

转目看四周,就发现这里裂缝的两边,全是细小的树根须和干泥包裹的泥茧,缩在凹陷中一直排列在两边,能听到废墟下水流的声音。再往里看,我发现这条缝隙裂在另一条石头井道上的。显然地震使得这的砂土层开裂,裂缝将相距很深的两条井道连接了起来,我们走了一条近路。

井道的里面一片狼藉,也是四处开裂,显然废墟倒塌的时候,形成了无数这种裂缝。

我就问三叔道:“为什么让我下来?”

“我来让你看个东西。”他道,示意我跟他走,我们在矮小的缝隙里蹲着走了几下,他用手电指着一边的树根后的沙土壁。

我一开始看不清楚那里有什么,因为全是粘在沙土壁上的树根,凑近了看,才看到上面,有人刻了一行字,好像是几个英文字母,我心里一惊,抓住三叔的手让他照得准点,仔细辨认,就“哎呀”了一声。

三叔道:“你看看,这和你在长白山里看到的,小哥留下的记号是不是一样的?”

我忙点头,这就是闷油瓶在长白山里刻的记号,心里一下骂开了,他娘的难道闷油瓶刚刚来过这里?

“你是怎么发现的?”我问三叔道。

他抹了抹脸上的泥道:“你别管这些,你能肯定这是小哥的笔迹,不是其他人刻的类似的记号吗?”

我不明白他的意思,点了点头表示可以肯定,他立即招手给黑眼镜:“瞎子,告诉上面的人给老子全部下来。咱们找到入口了。”

黑眼镜应了,退了出去,就给上面打了信号。

我问三叔到底是怎么回事,三叔就道:“你仔细看看这个记号,感觉一下和长白山刻的有什么不同?”

“不同?”我一下子没法理解三叔的意思,凑近去看,忽然发现这个记号颜色发灰。

记号是刻在砂土上的,这种砂土本来是不适合刻任何东西的,因为虽然坚硬但是非常脆,力道用的小了,刻不出痕迹来,力道用的大了,可能正块砂土都裂开来,这记号有点复杂,显然刻的时候十分的小心,而这发灰色颜色,是砂土经年累月氧化的痕迹,记号之中的灰调和周围的砂土几乎一样,这就表示,这记号显然刻在这里有点年头了。

“不对。”我就疑惑道:“这是个老记号?你让我再看看——”

三叔道:“不用看了,既然笔迹是,那就没错了,这就是他刻的,不过不是这几天刻的,而是他上一次来这里留下的。”

\chapter{记号}

我摇头,脑子乱得犹如烧开的泥浆:“我不明白,什么叫他上一次留下的,他来过这里?”

三叔摸着那几个符号,“没错,我在这片废墟里,看到这个记号不止一次了,到处都有,我就是跟着这些记号,以最快的速度穿过了雨林,到达了你找到的那个营地。不过我当时还不敢肯定这记号就是这小哥留下的,现在证实笔迹一样,那就没错了,这小哥以前肯定来过这里,而且还有点年头。”

“可是,这是怎么一回事?”我一时间失语,想问问题,却完全不知道该怎么问。

我是认拓片的,对于笔迹,特别是雕刻的笔迹有着极端敏感的认识,所以我能肯定这符号确实是闷油瓶刻的。但是,这上面的石糜不会骗人,这确实不是最近刻上去的,这么看来,唯一的解释确实是闷油瓶来过这里。

是他失忆之前的事情吗?难道,他也在文锦和霍玲当年的考察队里?

不可能,他在西沙的时候就完全失去记忆了。

“我暂时也不清楚,不过我和你说过了,这个小哥不简单。显然他的过去深不可测,而且他做的每一件事情都有理由。”三叔道,“不过,我猜我们只要跟着这个标记走,我们就能知道,他最后到达了哪里,也可能找到出去的路线。”

我感觉我的脑子无法思考,不过闷油瓶的过去我确实一无所知,他如果真的来过这里,时间上倒也完全可行,这时却看到三叔说这些的时候,眼睛看着黑眼镜出去的方向。

我问他怎么了,他做了让我别说话的手势,看着黑眼镜出去,才压低声音对我道:“我真被你气死了,这一次你实在不应该跟来。”

我看他突然转了话锋,又是这么轻声说话,好像在忌讳着黑眼镜,就愣了一下。

三叔继续急促道:“你他娘的真是不会看风水,你三叔我已经今非昔比了,这一次的伙计都是你三叔我临时从道上叫来,这批人表面上叫我声三爷,其实根本不听我的,只能做个策应,还得防着他们反水。我一个人都应接不暇,你跟来不是找死。”

我一下就明白了刚才三叔的表情为什么这么无奈,潘子和我说过这些情况,没想到事情严重到这种地步,立即也轻声道:“我也没办法,你叫我……”

没说完,三叔立即给我打了个眼色,我回头一看黑眼睛已经回来了,他问黑眼镜道:“怎么样?”

“下来了,我让他们先把装备送下来。”黑眼镜咧嘴笑,“他们问那个死胖子怎么办,要么把那个死胖子留在上面,找个人照顾?带着他走不现实……小三爷,你脸色不太好看啊。”

三叔刚才一说,我有点反应不过来,也许脸上就表现了出来,但我应变能力还是有的,立即道:“这味道太难闻了。”

三叔想了想道:“不能留下来,绝对不能分散,告诉他们先全部下来,然后我们找个地方再想那个胖子的事情。”

“得。”他道,“那小三爷出来帮个手来,这家伙算是个大部件。”

我点头道:“我这边说完就来。”就看着黑眼镜出去了。

我和三叔对视了一眼,见三叔的表情也很异样,心说确实没有想到事情会到这种程度,看来三叔真的很不容易。

说实话我对黑眼镜印象还不错,虽然这人好像有点癫,看来这江湖上的事情我懂得实在太少。

三叔轻声继续道:“你别和我争,你这次跟来我真的没法照顾你了,你要自己小心,我真被你气死了,要是咱们能出去,我肯定到你爹那里狠狠告你一状。”

我看他的表情知道他不是在开玩笑,就点头。他急促道:“我长话短说,你记住,这批人都是长沙地头上的狠角色,也只有这些人才敢夹这种喇嘛。这黑眼镜是个旗人,名字我不清楚,道上都叫他黑瞎子,他是一伙。另外一伙就是那个叫拖把的带的人,这批人以前是散盗,亡命之徒,你要特别小心的就是这批人,不要当成我以前的伙计,也不要什么话都说。”

我继续点头,三叔看了看外面。这时候黑瞎子叫了几声,三叔就拍了我一下,让我自己注意。

我于是不再说话,跟着黑瞎子出去。这时其实我还没完全反应过来,一边帮忙一边想了想才真正意识到事情的麻烦程度,三叔要和我单独说话竟然要这样,显然这伙人已经心生戒备了,有可能是之前发生过一些事情了。

江湖上的事情我完全不懂,此时也不能多考虑,只得尽力装出和刚才无恙的样子,心说只能静观其变了。

胖子是和“拖把”绑在一起下来的,两个不好控制,拉进来之后,两个人身上的尿味浓得离谱,几乎让人作呕。接着,上面的人就一个一个下来。

拖把倒还是很客气,骂了几声长沙话,对我还是点头笑,小三爷长小三爷短。不过我听着一下就感觉和刚才在上面大不相同,看着这些人,觉得表情都有点假,不知道是否是心理作用还是真的就有这一层意思在。

我就装作完全听不出,这就上了心了,也没心思去考虑闷油瓶的事情到底是怎么回事。

四五个小时后,所有人都下到了下层的井道,整理装备,找了两个人抬着胖子,我们开始顺着闷油瓶的记号,往井道的深处前进。

三叔给了我一把短头的双筒虎头猎枪,双管平式,这是我以前打飞碟的枪,型号一样,只是轻了一点,一次两发,用的是铅散弹。这应该是三叔能搞到的最高档的武器了,我们在七星鲁王宫也用这种东西,当时还是我从黑市里买过来的,一把好像要五千多。

这东西打大型动物只能起一个阻碍和威慑的作用,但是要打那种鸡冠蛇应该相当便利,一次可以扫飞一大片。我心说潘子怎么就没带一把,还用他那种短步枪真是落伍了。

想到潘子又很担心,不知道他现在怎么样了,在那个神庙中应该会比在这里安全,但是如果他再发起烧来,恐怕就真的凶多吉少了,如果有他在,三叔应该就不需要这么担心。

我提醒三叔之前看到的浮雕,这些坑道除了蓄水之外的作用,就是侍养那些鸡冠毒蛇,我一路从雨林过来,并没有看到太多的鸡冠蛇,只是集中看到过几次,显然这些蛇的地盘,是在这些坑道里,我们要加倍小心。

三叔道这些蛇防不胜防,加倍小心都没用。

坑道高高低低,这里的环境,让我感觉和鲁王宫相当的类似,难道当时的西周嵌道,根本就不是我们想的嵌道,而是排水的井道吗?

无法推测,因为山东那边雨量充足,不需要如此复杂的地下蓄水系统。否则碰到连月大雨,这些蓄的水可能会淹出来,这里应该只是单纯的相似而已。

行不到五百步,井道就出现了分岔,三叔用矿灯照了照,一道朝上去,一道朝下去,朝上去的应该是上游的井道,水从上面下来,然后和这一条汇合往朝下的那道流去。我们在附近搜索,立刻就在下面井道上看到了闷油瓶的记号。

三叔掩饰不住兴奋的神情,但是我现在能看出他的兴奋有点假,我也不得不装作非常紧张的样子。他毫不犹豫,挥手继续前进。

在这种井道行进,是极度枯燥乏味的事情,四周全是石砖,没有任何浮雕和人文的东西,有的只是简陋的石头,矿灯的光斑晃动的井壁,长时间都没有一点变化。

第一段足足走了三个小时,一个又一个的岔口,看到闷油瓶留下的许多记号,过程很枯燥,不多赘述。途经很多的蓄水池,唯一让我感到有点意思的是,我发现随着我们高度的降低,这些蓄水池一个比一个大,而且,四周没有任何的声音,似乎这里根本就没有蛇。

这多少有些出乎我们的意料,也可以说有一些庆幸,不过,我总觉得不太对劲,这种安静下好像隐藏着什么。

长话短说,一直走到晚上都相安无事,我们紧绷的神经终于开始松弛了下来。我们当天只能在井道中一字排开地休息,点了好几堆火,吃饭的时候,胖子第一次醒了过来。

三叔给他打了针巩固,又给他吃了东西,我就问他到底发生了什么事情,但他还是没力气说话,只说了几句,很快又睡着了。

但是我心已经宽了,这中蛇毒不是重伤,如果他能醒过来,说明他已经没有什么大碍了。果然到了第二天早上,他醒来的时候,脸色已经有所恢复,虽然还不能走动,但是被人搀扶着能站起来了,看着四周,就有气无力地问我怎么回事。

我道这一次你可得谢我了,难得老子不抛弃不放弃,差点把我折腾死,才把你救下来。你这一次新生得怎么感谢我?

胖子这人能折腾,就找人要了烟抽,一脸萎样道:“我靠,胖爷我都救了你多少次了,你就救我一次还来这套。我和你说,这一次扯平都不算。”然后问我这是什么地方。

我把后来的情况大概一说,他听了也没做什么表示,我就问他闷油瓶最后和他怎么了?

他道他们追着追着就跑散了,那小哥是什么速度,他根本撵不上,后来就听到蛇的声音,他和我的想法一样以为,三叔的人还活着,但是没我那么莽撞,偷偷摸了过去,结果撩开一草丛,一下就被蛇咬了。

这和我琢磨的差不离,他道,那小哥恐怕也得中招,娘的那些蛇太邪门了。上帝保佑他比我们两个机灵。

三叔看到胖子还是挺开心的,递给他烟,我想来大概因为胖子总算是个自己人。不过胖子看到三叔就很郁闷,道:“三爷,你看你这个喇嘛夹的,你回去得给我加钱,否则我可不干。”

说完其他几个人也附和他,一通说笑,看上去气氛一点问题也没有,似乎谁也没注意到三叔笑容的苦涩。

胖子复原得很快,我让他多喝水,第一次他的尿都是黑的,慢慢的,尿开始清起来。他的体质确实好,脸色也越来越红润起来,等我们要出发的时候,他已经基本可以站起来自己行动了。

我搀着他继续出发,还是和昨天一样一点一点地深入,一个蓄水池一个蓄水池地下去,我们发现其实这蓄水系统应该是一个网兜状的,越往下越结构简单,但是井道和蓄水池体积越大。

最后我们在第六个蓄水池里停了下来,这个蓄水池已经大到不成样子,在水池的中央竟然立了一根三人合抱的石柱防止倒塌。整个蓄水池都是干涸的,目测距离,足有半个足球场那么大。

胖子已经不需要我搀扶,不过体力还是没完全恢复,坐下就直喘,一身的虚汗。

我们停下来倒不是因为休息,在井道中行进比起雨林行军简直是在风和日丽的沙滩上漫步的感觉,一点也不疲倦。而且到了这个蓄水池,我们发现里面长满了干枯的树根,几乎把整个蓄水池都覆盖了,那些分流的井道口全部被遮盖在树根之中了,上面长满了奇形怪状的菌类,找不到继续前进的道路。

我倒奇怪,我们现在已经深入地面以下了,为什么这些树根会长到这里来,世界上有根系这么长的树吗?

那个“拖把”看了看道,这些不是树根,都是菌丝,这个蓄水池看来是种香菇的好地方。说着,让手下人去砍掉这些菌丝,寻找闷油瓶留下的记号。

我凑近去看,发现这些菌丝和树根很像,但是很软,而且上面长满了黑毛,紧贴在井壁上,看上去好像很难吃。

找着找着,有人就惊叫了一声,翻倒在地,我们立即端枪朝他瞄去,一下就看到他砍掉了一片菌丝之后,菌丝后面的井壁上出现了一张石雕的人脸。

我一看就知道这是什么东西了,立即报以报复性的大笑,来报复他们嘲笑我被假人吓到。他们莫名其妙地看着我,我就捡起地上的碎石丢了过去,当下组成人脸的飞蛾被惊飞了起来。

那人一看,长出了一口气,所有人都笑起来。

这些蛾子可能是偶然飞进井道来的,这里可能也有蛇蜕来吸引它们。我对他们道,小心一点,附近可能有蛇。自己就到飞蛾聚集成脸的地方去翻找,果然在树根密集处,看到了一大片白色麻袋一样的东西。不过让我吃惊的是,这片白花花的蛇蜕不是很多,而好像是一个整体。

我用猎枪把蛇蜕挑了起来,发现那是一条大蛇,足有水桶那么粗,能看到蛇蜕上长着双层的鳞片。

三叔过来一摸,一手的黏液,他的脸就白了,叫道:“他娘的把枪都给老子端起来,这玩意是新鲜的,这皮是刚蜕下来的!”催促寻找井道口的人快点,这地方不能久待。

我马上也过去帮忙,用刀去砍菌丝,把菌丝砍掉后扯掉,然后用矿灯去照井道口子,按照我们的经验,闷油瓶会把记号刻在那个地方附近。

忙活了半天,竟然没有找到,人都有点急躁起来,这稍微矮点的井道口几乎都找了,只剩下蓄水池顶上的一些。我心说这一次该不是开在上面,上面没有坡度,几乎是垂直的,必须攀着井壁的缝隙爬上去。

这里有个瘦瘦的小个子身手最好,义不容辞地爬了上去。我们用手电帮他照明,看他一边单手抓住巨石的缝隙,一边就用砍刀砍掉菌丝,然后像攀岩运动员一样抓住缝隙,扭动身子吊过去。

我心说要我像他这样我可做不到,等一下找到了,我怎么进去啊。

他探了几个井道口,道:“在这里。”我们才松一口气,三叔让他立即结好绳子,我们开始陆续地爬上去。才爬上去三四个,忽然上面那小个子又叫了声:“三爷,不对,这里也有,记号不止一个。”

\chapter{三选一}

就在这时候,我们都看到红光一闪,接着那人整个就不见了,速度极快。一下我们都愣住了,他好像是被什么东西拖进去的。

没等我们反应过来,那道井口里就传来了一声惨叫,接着,他就摔了出来,还没摔到地上,从井坑道中猛地射出一条巨蟒的上半身,凌空一下把他缠绕住。

这是一条刚蜕完皮的巨蟒,我原以为会看到一条褐金色的大蛇,然而我看到的却是血红色的。顿时就明白了,我靠,这果然是同一种蛇!

身边已经开火了,在狭窄的空间中,猎枪的声音几乎把我的耳朵炸聋了。

刚蜕完皮的巨蟒,鳞片还不坚硬,立即被打得皮开肉绽,无奈铅弹的威力太小,剧痛的蟒蛇暴怒,把那人往井壁上一拍,那人就摔了下来。接着它沿着蓄水池壁旋风一样盘绕了下来,巨大的身躯一扫,扫飞了好几个。

三叔的伙计大惊失色,好几个人撒腿就跑,三叔大骂:“稳住!别跑!”

但是这批人真的完全不听他的,好几个人都钻进了坑道里,四散而逃。

三叔气得大骂,我拉着他一边开枪,一边也往坑道里退。

本来如果所有人都齐心,对这蟒蛇来几个齐射,就算是龙王爷也被打烂了,但是人就在这种关头会乱,没法判断形势。

我们退得最慢,巨蟒一下就冲了过来,我连开两枪,无奈巨蟒的头闪得太快,没有打中要害。我最后一次打飞碟是什么时候已经忘记了,要连射这么快速移动的物体我已经生疏了。

一边黑眼镜已经把三叔拖进了坑道,三叔对我大叫,让我快上来。我立即转身,但是人才扑进去一半,忽然我就头皮一麻,我的视线越过三叔的肩膀,看到这个坑道的深处,涌动着一大团黑影,正迅速爬过来。

“后面!”我立即警告。

他们猛回头,手电一照,我们就看到有十几条碗口粗细的鸡冠蛇,犹如血红色的潮水一样涌来。看样子这里的枪响惊动了它们。

黑眼镜立即回头开了一枪,将最前头的一波扫飞,我身后的劲风也到了,三叔大叫“抬手”,我忙抬手,他的枪从我的夹肢窝里伸出去,一声巨响,把身后的巨蟒震飞,背后又传来黑眼镜开枪的声音,他竟然还带着笑:“太多了,顶不住了!”

我心想这人真是个疯子,转身就见很多的井道口中,都开始爬出红色的鸡冠蛇,一坨一坨,我一边装弹一边让开,让三叔爬出来,一边寻找没有鸡冠蛇爬出的井道口,再去找胖子,却发现胖子已经不见了,不由大骂没义气,竟然跑得这么快。

一个一个看过来,好不容易找到一个井口,立即爬了进去,对三叔大叫,三叔和黑眼镜一边开枪一边挪过来。但是已经来不及了,鸡冠蛇速度奇快,几乎是腾空飞了过来,已经从我所在的井口爬了上来,发出高亢的咯咯声,我一枪把它们轰成肉泥,但是井道口瞬间又被蛇围满了。

三叔叫我自己快走,他会想办法,说着和黑眼镜朝另外一个没有蛇的口子退去。我大骂一声,再开一枪,就往后狂跑。

一边跑一边装子弹,就发现只剩下六颗了,这种子弹又大又重,我刚才为了方便就没多带。我这性格真让人头疼,一到关键时候总有事情掉链子。

那些蛇的速度之快,我之前已经领教过了,知道跑的时候完全不能分心,否则根本就没有生还的机会,咬紧牙就开始狂奔,脑子就想着“淤泥!哪里有泥?”

一连冲过好几个岔口,我看到了井道上的裂缝,里面同样是沙土,我停了一秒马上挤了进去,里面空间比之前看到的那条要大,我一眼就看到了大量屯起来的泥茧骸骨。

有救了,我心说,立即掏出水壶,听着外面窸窸窣窣的声音不断靠近,立即将水全倒在一只泥茧上,把人骨身上的泥和稀了,抓起来就往我身上草草涂了一遍,搞完后把那死人往裂缝的口子上一推,大概堵住,自己缩进那个凹陷,闭上眼睛装成是死人。

瞬间那些蛇就到了,一下盘绕着我丢在地上的矿灯和水壶开始咬起来。有一些蛇没有发现我在缝隙里,就继续朝前飞快地爬去,但是有几条停了下来,似乎发现了这裂缝里的异样,朝里面张望。

我心说真邪门,这些蛇果然有智力,却见几条蛇小心翼翼地爬了进来,开始四处盘绕上那些泥茧,似乎在寻找我的去向,一下我身上就爬上来好几条。

我闭上眼睛,屏住呼吸,感觉心都要从喉咙里跳出来了。

那几秒钟,我感觉像一年那么长,忽然我感到后脖子一丝凉意,浑身就出了冷汗——一下想起来,完了,刚才太急了,我的后脖子忘记涂泥了。

我小心翼翼地睁开眼睛,果然看到一条红得发黑的鸡冠蛇盘在我的肩膀上,正饶有兴趣地想盘到我的后面。

完了,我心道,这下子我也得成胖子那样了。

就在那蛇慢慢朝我的后脖子凑过来的时候,忽然我身边的骸骨中,发出了一声奇怪的声音,那蛇立即就扬起头,看向那个方向。

几乎就在同时,一件令我更加惊悚的事情发生了,我身边的那具骸骨忽然动了,手一下就按在了我的后脖子上,把我没有涂泥的地方遮住了。

我头皮麻了起来,用眼睛一瞄,发现不对,那不是骸骨的手,而是一只涂满泥的人手,仔细一看,发现我身边的死人后面,还躲着一个浑身是泥的人。

是谁呢?我看不清楚,我心说原来不止我一个人知道淤泥的事情。

我心里完全不知道是什么滋味,高兴也高兴不起来,只觉得气氛诡异无比。

那鸡冠蛇看向那个方向,看了半天也不得要领,再回来找我的后脖子,却也看不到了。它一下显得十分的疑惑,发出了几声咕咕声,在我后脖子附近一直在找。我就感觉那蛇信好几次碰到我的脖子,但是它就是发现不了。

我一直不敢动,就这么定在那里十几分钟,那些鸡冠蛇才忽然被外面什么动静吸引,全部都迅速追了出去。这一条也游了出去。

它们消失之后很长时间我还是不敢动,怕它们突然回来,直到捂住我后脖子的手动了一下,才好像是一个信号,我简直浑身都软了,一下就瘫倒了下来。

刚想回头看那人是谁,忽然就听到一个女声轻声道:“不准转过来。”

我愣了一下,还是转了过去,身边的人一下就把我的眼睛捂住了。我手下意识地一摸,就摸到一个人的锁骨,竟然发现那人没穿衣服,接着我的手就被拍了一下,听到那女声道:“闭上眼睛,不准看,把上衣脱下来。”

我一顿,还没反应过来,我的上衣已经给剥了下来,窸窸窣窣一阵折腾,那人似乎在穿我的衣服。

等捂住我眼睛的手拿开,我就看到一个女人坐在我的面前,身材很娇小,穿着我的衣服好像穿着大衣一样,再看她的脸,我一下就认了出来。

“陈……文锦……阿姨!”

在我面前,竟然就是文锦!

我看着惊讶得说不出话来,语无伦次地问了一句:“你没被逮住?”

文锦整理着衣服,看着我扑哧一声笑了:“什么逮?你当我是什么?”俨然和之前被我们追捕时候的神情完全不同了。说完,她用涂满泥的骸骨,将这个泥井道口堵住了,然后用水壶挖起泥把缝隙全封上,我就看到,这捆着骸骨的材料,竟然是她的衣服和胸罩。

做完后她才回来看我笑了起来,摸了摸我的头发:“你也长大了。”

我也看着她,几乎无法反应,想说什么,但是脑子里一片空白。

这有点太过梦幻了,以前我只在照片里见过她,她现在竟然在对我笑,而且笑得这么好看。

她看着我,看我这么看着她,就问道:“怎么?你反应不过来吗?”

我点头,心说怎么可能反应得过来,这应该是一个满脸皱纹的中年妇女,二十多年前在一座诡异的海底古墓中失踪,这么多年间一直做着一些极端隐秘的事情,牵动着无数人的神经,制造了无数的谜,现在却就这样站在我的面前,满脸淤泥但是不失俏皮地看着我,那眼睛那皮肤显然比我的还要嫩上几分,叫我如何反应。

她笑着说:“我看到你长这么大了的时候,我也反应不过来,想想已经二十多年了,当时你还尿床,我还给你洗过尿布,你那时候长得好玩,比现在可可爱多了。”

一说到小时候,我立即就朝那缝隙口看去,想想,我忽然觉得无比的奇妙,三叔处心积虑要找文锦,但就在十几米外,我不知道他的生死状况,却在这里看到了文锦,还说上了话。要是三叔再快一步跟着我,他和文锦已经见面了。

“你也可爱多了……”我口不择言,抓了抓头,“文锦……姨,这,好久没见了……我实在不知道该怎么反应,我现在是不是应该大哭一场?对了,我有好多话要问你……我们很想你……到底发生了什么——妈的,我在说什么?”

看着我语无伦次,文锦就做了轻声的手势,听了听外面,轻声笑了,道:“谁说好久没见了?前不久我们不是还一起喝过茶吗?”

“喝茶?”我愣了一下,心说之前见的时候,她在沼泽里啊,当时没见她端着茶杯。

只见文锦把自己的头发,往头上盘绕了一下,做了一个藏族的发型,然后用袖子擦掉脸上的泥,我一看,顿时惊呆了:“你!你!你是定主卓玛的那个媳妇!”

\chapter{真相}

我简直不敢相信自己的眼睛,拍了拍脑袋:“原来你一直跟着我们!那口信,那定主卓玛和我们说的话——难道——”

“不错,那都是我临时让她和你们说的。情急之下,我没有别的办法。那些事情说来话长了。”文锦道,爬到缝隙里头,双手合十做了手势,放到嘴边当成一个口器,发出来了一连串“咯咯咯”声。

我奇怪她在干什么,难道在和那些蛇打招呼?就听到缝隙的深处也传来了咯咯咯咯的回音。不一会儿,就有人从里面挤了出来,我一看,发现那人竟然是闷油瓶。

他挤到我们边上,看了看文锦又看了看我。我就目瞪口呆地看着他们两个,“这是怎么回事?”忽然感觉到一些不妙,“该死,难道这是个局,你们该不是一伙的?”

这两个同样不会衰老,而且同属于一个考古队,同样深陷在这件事情当中,我忽然想到我一个朋友说的,闷油瓶肯定不是一个人,难道被他说准了?

闷油瓶摇头不语,我就看向文锦,文锦道:“没你说的那么恶心,我和他可清白着呢。”

我皱眉,真心真意地想给他们磕头道:“大哥大姐,你们放过我吧,到底是怎么回事情?”

文锦对我道:“在这件事情上没有什么复杂的,其实当时在那村子里卓玛找你们的时候,他已经认出我来了,不过他没有拆穿我。我在峡谷口子上找到你们的时候,他追了过来,当时我们就已经碰面了。这接下来的事情,确实算是合谋,但也是为了谨慎。”

我看向闷油瓶,他就点了点头。

我怒起来,“太过分了,你为什么不说?”

他看着我:“我已经暗示过你了,我以为你已经知道了。”

“胡扯!我那个样子哪里像知道了!”我几乎跳起来,一下就意识到了,为什么闷油瓶一直心神不宁,天,他一直在担心文锦的安危。

一边的闷油瓶立即对我做了一个“轻声”的动作,我才意识过来,立即压低声音:“你丫太不够义气了!”

“不,他这么做是对的,否则,我会落在你们那个女领队手里,她也不是省油的灯。”文锦道。“而且,当时,我也不知道,你们之中哪个有问题,我需要找一个人帮我检查。”

这大概就是为什么闷油瓶回来之后开始检查我们有没有戴面具的原因。妈的,原来事事都是有原由的。

“那些录像带呢?”我问道,“这整件事情,到底是怎么回事?”

话音刚落,外面又传来一声惨叫声和几声枪声。

闷油瓶啧了一声道:“他们这么开枪,会把所有的蛇都引过来。”

文锦听了听外面,转过头来拍了拍我的头,好像一个大姐姐一样对我道:“这是一个计划,说来话长了,长到你无法想象。这些事情我都会告诉你的,但是现在不是时候,我们先离开这里。”说着就指了指一个方向。

我叹了一口气,但是知道她说的是对的,于是点头,几个人都站了起来,迅速往泥道的深处退却。

一边走我就一边问她道:“你们有什么打算?不去和我三叔会合吗?”

“我们没有时间了,”文锦道,“你没有感觉到,四周的水声已经越来越少了?”

这我倒没注意,在这种地方谁还有精力注意这些。文锦道:“这里的地下水路极端复杂,但是在有水的时候,它其实并不是一个迷宫,你至少知道你是不是往地面上走,只要逆着任何一道水流往上,你肯定能找到一个地面上的入水口。而顺着水流走,你也肯定可以找到这个底下水路的终点——最大的那个地下蓄水湖泊。但是,一旦水消失了,你就永远不可能走出去。现在雨已经停了,沼泽的水位会逐渐降低,再过一两天,水就会完全干涸,到时候我们就会被困在这里。这就是我为什么让定主卓玛告诉你们,如果不及时赶到就要再等十几年的原因。不过你们这一次运气好,今年的雨量特别大,把整个沼泽都淹没了,否则现在已经晚了。关于你三叔,吴三省和我们的目的地相同,只要他没有出意外,我们肯定会碰上。”

我一听,在理,立即点头:“那我们现在是往上还是往下?”

文锦指了指下方:“最大的秘密已经近在咫尺了,你打算就这么放弃吗?”

近在咫尺?我心说我才不信呢。文锦看了看表就道:“现在已经快天亮了,那些蛇大部分都会在夜晚到地面上活动,天亮之后会全部下来,到时候我们行走更麻烦。在天亮前,我们得找一个地方躲起来,到时候你有什么就问吧,我都会告诉你,现在还是专心走路。”

文锦说这话的时候,几乎没有什么严厉的言辞,但是她的眼神和她分析问题的语气,却让我感到自然而然的服帖,似乎天生就有一种领袖的气质。难怪当年她是西沙的领队,连三叔都要忌讳。

我不再去烦她,三个人立即加快了脚步,顺着坑道一路往下。很快就到了另一个坑道。

这里已经很深了,坑道显然没有上面那么错综复杂,岔路很少,加上我们身上的淤泥,走得非常顺利,到早上的六七点钟,我们已经走了相当长的距离。这里的井道连淤泥都没有了,只有天然的岩洞,很难看到人工开凿的迹象,显然这里几乎不会有人来。

我们能听到岩石中传来扑腾的水声,显然所有井道的水,都在四周汇集了,整个西王母城的蓄水系统的终点应该非常近了。

此时地面上的晨曦应该已经退去,虽然附近还没有任何蛇的声音,但是我们都知道这些蛇数量惊人,一旦归巢很可能会出现在任何地方,按照文锦的经验,此时还是躲起来的好。

怎么躲就是经验了,她让闷油瓶脱掉衣服,用水壶的水抹上泥,将通道的两端用碎石头堆起来,然后将衣服撕碎了塞缝隙里。

“这样,在蛇看起来,这里的通道就是被封闭的。”文锦道,“我这些天都是这么过来的。”

我喝了几口水,感觉这么薄弱的屏障不会有用,要是碰上那种巨蛇,不是放个屁就倒?

此时点了很小的篝火,也只是稍微暖和一下身子,这里潮气逼人,而且阴冷得厉害,没有火没法休息。

缓了片刻,我逐渐才放松下来,心里有些忐忑。文锦递给我吃的东西,看我的表情就知道我忍不住想问问题,让我想问什么就问什么。

我早就在琢磨了,立即振奋起来,想问她问题,却一下子发现脑子很混乱,要问的问题实在是太多了,反倒问不出来。

“没关系,你可以一个一个问,我早就料到会有这样的情形了。”文锦笑吟吟地看着我。

我理了理脑子里的问题,想想哪一个是最主要的,想了片刻,我发现无论从哪里开始问,无论问什么,都有可能导致混乱,我心里的谜题太多,大的小的,无数无数,必须有一个系统的提问方式,于是道:“我们还是按着时间来问,如何?”

她点头:“没问题。”

我就问她道:“第一个问题,我最想知道的,可能有点贪心,你能告诉我西沙到底是怎么回事吗?”

文锦看了我一下,表情很惊讶:“你这个问题太大了,西沙发生了很多的事情,你到底指的是哪件?”

我对文锦道:“就是你在古墓里失踪之后,到底发生了什么事情?”

文锦静了静,好像没有想到我会一开始就问这个,想了想,忽然叹了口气,道:“你竟然想知道这件事情……看来你确实已经知道了不少,这件事情,很难说清楚,你三叔是怎么告诉你的?”

我把三叔之前在医院里和我说的,大致和她说了一遍,然后对她道:“他说没有跟你们进入那机关内,所以之后的事情他不知道。你们在古墓里失踪之后,他一直在找你们,但是找了这么多年,什么都没有找到。他还说他一定要找到你们。”

文锦听完,怪怪地笑了笑,顿了顿,才道:“这个问题我本来想最后告诉你,因为,这里面有一个很关键的前提你必须明白,但是这个前提,我就这么说出来,你是不会相信的。我不知道你现在有没有做好知道事实真相的准备。”

我道:“早死早超生,你就是告诉我三叔其实是个女的,我是他生的,我也能信,你就说吧,这两年下来,我已经什么都能信了。”

文锦看上去还是有点顾虑,想了想,又问道:“对于这件事情,你自己有什么判断吗?”

我摇头:“我什么判断都没有。”

文锦看着了闷油瓶,似乎在和他做一个交流,但是后者没有什么反应。她定了定神,弄了弄头发,似乎是下了一个什么决心,就从背包里掏出一个笔记本。

这是一个新的笔记本,是现代的款式,应该是在最近才买的,果然她还是保持着写笔记的习惯。她翻开笔记本,从里面掏出了一张发黄的老照片,我一看,这张照片再熟悉不过,就是三叔和他们一起出海前拍的那张合影,这张照片我不知道看了多少遍,里面每一个人的位置,我都能背出来,所以我只看了一眼就递了回去,道:“我已经看过这张照片了。”

文锦道:“其实,所有的秘密都在这张照片里面。但是这个秘密普通人很难发现,西沙所有的事情都起源在里面。秘密其实不复杂,但如果我直接告诉你,你肯定无法接受,我先来告诉你,这张照片中隐藏了什么。”

这时候,我的脑子里突地闪过一个概念,难道之前和那批朋友喝酒的时候,他们说的第十一人的事情是真的,这张照片中还藏着那十人之外的一个神秘人?文锦想告诉我这些?

看她的样子,又不像是这么简单的,我就不知道她是什么用意了。

文锦把照片重新给我,让我把照片上能念出来的人的名字和位置,都对应一下指给她看。

我看了看,道:“我只认识和这件事情比较有关系的几个人,其他人我能知道名字,却不知道是哪一个。”

文锦说:“没关系,你念就可以了。”

我首先看到了最吸引我注意力的闷油瓶,道:“这就是小哥。”文锦点头,然后指了指一边的一个女孩子,“这就是你。”文锦又点头,“然后,这个是三叔。”我指着三叔道。我看了一下文锦,等她点头后继续说下去,但是她这一次却一动也不动,而是直直地看着我。

我愣了一下,她这是什么意思?文锦把照片拿了过去:“你为什么会觉得这个人是你三叔?”

\chapter{颠覆}

我道:“这……这是三叔年轻时候的样子啊,我看过他以前的黑白照片,和这个很像啊。”

文锦就笑道:“这个世界上并不是只有照片才会相似,两个有血缘关系的人,也可能会相似。”

“啊?”我愣了一下,忽然就领悟到什么,“等等,你这是什么意思?你难道想告诉我,这个人不是我三叔?那他是谁?”

说完我忽然一凉,以前的碎片一下在我面前聚拢成了一张脸。

血缘关系!相似容貌!

我突然恍然大悟:“不可能,不可能!”我几乎吼了起来,闷没瓶立即把我按住。我已经没法控制我的声音了,破声道,“我的天,我的天,难道这个人是——谢连环?”

文锦点头,我毛骨悚然,所有的毛孔都竖了起来,无数的线头开始在我的大脑里结合起来,我的天,我好像明白是怎么一回事了。

“照片的解析度不高,看错是正常的,特别是在你三叔那样说的情况下。”文锦道,“谁都会那样认为。”

“那我的三叔呢?”

文锦道:“你三叔当时确实也和我们在一起,但是,他并不在这张照片里,而是在照片之外。”她立起了照片,指了指照片的前方。

我一看文锦的手势,忽然就明白了,感觉所有的血都冲到喉咙,这……这……狗日的,这是怎么回事,你是照相机的位置。

也就是说,当时三叔在给他们拍照,那——那第十一个人不是别人,竟然是三叔自己?

“可是不对啊,说不通,这样的出发合影,为什么会让三叔去拍,你们可以让其他比较不重要的人拍啊,比如说谢连环就是混进来的,他反而站在这么主要的位置上,而三叔只能拍照?”我问道。

文锦长出了一口气:“你还是有悟性的,你应该感觉到这里的问题了。在你三叔跟你说的版本里,有一些东西,出现了根本的问题,而且是在最初的时候。”她顿了顿,“我告诉你,其实当时,来托关系找我加入考古队的,不是解连环,而是你的三叔吴三省。”

“啊?”我一下反应不过来了。

“你仔细考虑一下,你三叔和你说的那些事情,其中虽然非常顺遂,逻辑上却全是一些很小的破绽。裘德考作为一个经验这么丰富的走私大头,怎么会选择一个没有任何下地经验的解连环,来执行他的计划?他当时在长沙,通过关系能找到的最出色的,也是对海外走私最有兴趣的人,就应该是你的三叔,只有你的三叔会有这种魄力和这种背景这么黑的老狐狸合作。所以,当时裘德考合作的人,不是解连环,而是你三叔,而裘德考选择吴三省还有另外一个好处,就是我和他当时是男女朋友的,可以非常方便地打入到考古队里,所以,这才是最符合逻辑的。”

我点头,忽然想到三叔也提过这么一句,我当时以为他是在和我抱怨,原来他是在这上面和我玩圈子。

“而当时的解连环,确实是在我的考古队里工作,他是当时考古大学的学生,因为家族的关系,他的父亲把他安排到了我的学校里。这个人并不像你三叔说的那么没用,虽然有一些少爷脾气,但是解连环天分极高,‘连环’二字是他父亲在他三岁在他三岁的时候给他改的名,因为他当时已经可以靠自己的能力,解开‘九连环’。这个人沉默内向,但是心思非常的缜密,成绩也十分好,他进入大学,完全是自己的意愿。”她顿了顿:“你明白了吧,你的三叔,把一切都说反了。”

我一下无法处理这么复杂的事情,就摆了摆手,心里理了一下:当时裘德考找到了三叔,说了西沙的事情,三叔于是设计加入考古队去西沙寻找古墓,而解连环根本和这件事情没关系。

“可是,他为什么要反着说,这没有任何的理由,他是这样的人我早就知道了,难道他为了保持在我心里的地位,就处心积虑地撒了这么大的谎,这不符合他的性格啊。”

“为什么这么干?你到现在还没明白吗?他把一切都说反了,但是西沙出发之前的事情,并不是一切,他真正想掩饰的,是后面的事情。”

我仔细地回忆三叔说过的整个过程,忽然有如掉入了万丈冰渊,浑身的血都冻了起来:一切都说反了,那么,最可怕的就不是这些旁枝末节,而是出事当晚发生的事情!

那么,就不是解连环下水被三叔发现,而是三叔偷下水,被解连环发现。

解连环可能威胁三叔将他带入古墓,否则就告诉文锦一切,三叔之后将他带入古墓,接着就应该是解连环在古墓中触动机关。

一切都毫无破绽地合理起来。所有的事情开始符合人物的资历和性格。

最后的关头,三叔告诉我的版本是,他将解连环留在古墓中,然后他逃了出来,那么,最让我无法想象的局面就产生了。

如果是完全相反,要这一切继续合理下去,那从古墓中出来的,就应该是解连环,而三叔被打昏,留在了古墓里。那么,死在海底的,竟然是三叔自己!

那我现在的三叔又是谁呢?天,我不敢再想象下去了。

文锦看着我的表情,才道:“你现在终于明白了,你所谓的三叔,根本就不是吴三省,这也是你的三叔绝对不会和你说实话的原因,因为从最开始,一切就已经错了,他在海底已经和别人掉了包。”

“可是,可是这怎么可能呢?为什么我的家里人都没有发现?”

“那是因为你三叔这个人性格乖张,十几岁就离群独居,几乎和你家里人很少见面,只要稍微化装一下,对于你三叔的品性有一些了解,就可以蒙混过去。我想你也感觉到了,你现在的三叔,和你小时候记忆里的三叔,是完全不同的。”

我的衣服全部湿透了,一个人分别了五六年后突然出现,他的性情或者相貌变化,别人都是可以接受的,我也感觉到现在的三叔比起以前的,秉性要平和得多,他年轻时候简直是无法无天的一个人。

文锦说完之后,我整个人已经完全无法思考,或者说,心中如此多的谜题,如此多的推测,一下子必须要重新静想一下,这实在太混乱了。

“可是,三……解连环,他为什么要那么做?他为什么要和我三叔掉换身份?”

“这是一个无比复杂的情况,首先可能是因为档案,他从海底古墓回来之后,我们全部都消失了,如果他好好地出现在单位里,那他的问题就相当严重,别人会查他,他的背景在长沙太特殊了,一查株连太多,可能会形成巨大的麻烦。而吴三省当时是编外的,档案中没有他的名字,也就没有人知道他和这件事情的关系,所以他们解家权衡利弊,可能选择了这样的办法,同时,他也可以拿到吴三省所有的产业,对于当时家道中落的解家也有巨大的好处。可是,这一场戏一旦唱起来,就无法结束了,你知道你家的二叔,小时候在长沙就是出了名的刺头,绝对招惹不得,要是让他发现弟弟被害死掉包了,必然会来对付解家,以吴狗爷和你奶奶家的势力,这将是一场腥风血雨。”文锦道,“我一直在暗中注意这件事情,想通过某种方式把这个事情通知你的家里。但是解连环之后表现出来的能力让我极度害怕,这人心思极其缜密,我感觉如果贸然出来说这件事情,反而可能会被反咬一口。所以我只能一直潜伏。”

我捂住脸,心中开始抗拒,感觉这一切肯定不会是真的,道:“那么,你们在西沙海底最后到底发生了什么呢?为什么你们会突然消失。还有,为什么古墓的顶上有血字说‘吴三省害我’?如果是解连环害了三叔,那么应该是相反的意思才对!不对不对,这说不通,你肯定也在骗我!”

文锦看着我,似乎有点心疼地抓住我的手,柔声道:“小邪,你和他生活了这么多年,我知道你不可能相信这些,所以,我也想过不把这些说出来,但是你对于这个谜实在太执着了,即使我现在不说,我想他也不可能瞒下去太久,因为事情到现在这个地步,漏洞已经太多了,他除了不停地编你,已经没有任何办法来混过关,你现在这个时候再选择不信,已经太晚了。”

我心说我不是不信,而是已经信了,否则心里还会这么不舒服,镇定了一下,就问道:“我知道,你继续说吧,我只是发泄一下,这有点难受。”

文锦把我的手放到她的小手心上,拍了拍,我顿时感到一种温暖传递过来,她继续道:“接下来的事情,你可能更加无法相信。”

三叔忽然溺毙,被发现的时候,手握着蛇眉铜鱼,显然心怀鬼胎最后恶果上身。文锦悲恸欲绝,但是后来情况紧急,她不得不继续主持工作,带着人下到海底。

这之后的过程,和“三叔”,也就是解连环之后和我说的基本符合,他大概是因为害怕真正的三叔在海底古墓中留下什么关于他的线索,于是假装身体不合适,等他们开始勘探古墓之后,偷偷跟在后面,最后确实被阻隔在奇门遁甲之外。

文锦他们对于他来说,就此消失在古墓中,再也没有出现,所以他才会促成了假扮三叔、交换身份的想法,在被人救起之后,别人问他的名字,他对当时救他的渔夫就使用了吴三省的名字。否则之后肯定会露马脚,这显然是经过了深思熟虑的,文锦说解连环心思细腻,确实不假。

而文锦他们一路深入,最后到达了放置云顶天宫烫样的那座殿内,却被一个酷似三叔的人迷晕了,这又是怎么回事呢?

文锦道:“说出来,你可能更加无法相信。”我心说已经到这种地步了,其实已经没有什么所谓信与不信了,让她不用顾及我的感受。

文锦就道:“当时迷晕我们的人,并不是酷似你三叔的人,他恰恰就是你的三叔。”

东一个三叔,西一个三叔的,真假三叔我有点搞不清楚了,就对她道:“我们不如用本名来说,你的意思是,迷昏你们的,确实就是吴三省。但是他的尸体不是被发现了吗?”

“我们弄错了,我们在海里发现的尸体,并不是吴三省,那应该就是裘德考第一批雇用的人中的一个。这批人失败了,但是带出了古墓详细的地图,所以裘德考才能提供如此好的资料,那具尸体的脸已经被礁石撞烂,而且已经泡肿,加上他身上的潜水服,和吴三省从裘德考那里得到的潜水服是一个样子,我们才认定他就是吴三省。其实当时我也有点怀疑,但是我没有认这种尸体的经验,而且那潜水服款式很奇特,这个说服力太大了。”

“那么,按照小哥当时回忆起来的,你们第一次看到他的时候,他先是装了女人,而后又躲着你们,逃进了镜子后的洞里,迷昏了你们,他为什么要这么做呢?”

“因为他以为解连环已经把一切都告诉了我。”文锦道,“他以为我是进来找他兴师问罪的,如果我单是我一个人还好说,可是考古队所有的人都下来了,显然他认为他的事情已经完全暴露了,这在当时是极其严重的犯罪。那么,我作为领队,不可能在这么多人面前偏袒他,他必须自己采取措施又不连累我,于是他决定迷昏我们,然后再作打算。”

“这样,就发生了最后的一幕。”我接着道,“这确实说得通,可是,那些血字是怎么回事?”

“那此血字是你的问题,是你自己理解错了。”文锦道,“你想想,那些字到底是怎么排列的?”

我心说这也可能会理解错?这么明白,就用手蘸了点水壶的水,在一边的石壁上,按照记忆把那些字写了下来。

〖吴害解
三我连
省死环
不
瞑
目〗

一看我就愣了,顿时明白怎么回事了:“天,我把顺序搞反了!”

做拓本做得太久了,拓本上一切是反的,所有的竖立文章我都反着看,都是习惯从左往右读,但这是两边都可以读的,而且意思完全相反。

“我操。”我就骂了一声,心说三叔的文化水平不高,假道学旁门左道精通,文章写起来根本不用脑子,这种血书简直让人吐血。

“现在你不怀疑了吧?”文锦道。

我尴尬地点头:“接着呢?”

她接着脸色就变了变,道:“之后的事情,我到现在还无法理解,因为,等我们醒过来的时候,我们已经不在海底墓穴中了,而是在一间地下室里。一间很古旧的,好像五六十年代三防洞一样的地下室,里面有一只黑色的石棺,我们能看到地下室的出口,但是出口被封死了,我们怎么也打不开,而且看表上的日期,已经是我们昏迷之后一个多星期了。”

“那是在格尔木的那个疗养院?”我道。

她点头,顿了顿:“我们少了几个人,起灵已经不在了,另几个都被困在了那里,而且,我们发现我们被人监视着。”

\chapter{囚禁}

文锦被三叔迷晕之后的记忆,一片空白,他们醒过来的时候,已经在格尔木的疗养院里。

听到这里我已经非常迷糊了。这也太玄了,显然有人在他们昏迷的时候把他们绑架了过来,关在那里。

按照文锦的说法推测下去,三叔迷晕他们之后,会把文锦弄醒,然后解释一下,再商量对策。但是文锦没有醒来,显然当时他们昏迷之后,又出现了变故。

“吴三省不在你们当中?”

文锦摇头,我就道:“那奇怪了,是谁绑架了你们?”

“是‘它’。”她幽幽道。

我一直就对这个很疑惑,于是问文锦道:“‘它’到底是什么?”

文锦说的话多了,喝了一口水,就缓缓摇头道:“我无法来形容,这是我们在研究整件事情的时候发现的,怎么说呢,可以说是一种‘力量’。”

“‘力量’?”我皱起眉头。

“我们生还之后,在那间黑屋子里,对于整件事情进行了从头到尾的推测,但是,有很多的环节,我们都无法连接起来,最后,我们就发现,在整件事情当中,在很多地方,可以发现少了一人。”文锦把头发拢到耳后,“也就是说,这件事若要发生,光这么几个人肯定是不够的,但是这件事情却发生了,好似有一个隐形的人,在填补这些环节。而且,我们越研究就越发现,这个人肯定存在,但是到现在为止,他一点马脚也没有露出来,简直就好像是没有形状的,他只存在于逻辑上。”

她正色道:“我们就把这个人,称呼为‘它’,这是除了裘德考、解连环,以及我们之外,还有一股势力,在插手这件事情,这股势力埋藏得最深,几乎没有露过面,但是它的力量却实实在在地推动着事情的进程,这让我毛骨悚然。”

我听着也有点发凉,就问她道:“你能举个例子吗?”

文锦就道:“战国帛书的解码方式,真的是裘德考揭开的吗?他一个老外能解开这么复杂的东西,可能吗?而且,他是从哪里知道海底古墓的存在的?如果没有人告诉他这些信息,他就不会来中国,不会去收买你三叔,也不会到现在还在执着于一个谁也不知道的目标。这就是第一个逻辑的缺口。还有——”

文锦坐直了身子,挺胸拢起自己的头发,让我看她的瓜子脸:“我们所有人,好像都失去了衰老的能力,这么多年过去了,我们一直没有老。”那姿势真好看,我看得几乎呆住了,她却立即放下来,甩了甩道,“在我们昏迷之后,肯定有人对我们的身体做了什么手脚。”

我道:“那这还是好事,这种事情,很多人都梦想着出现呢!”

文锦凄凉地摇头道:“梦想?你还记不记得你在格尔木地下室里碰到的那东西?”

我心说我怎么可能会忘记,便点了点头。

“那就是我们最终的样子。”文锦道,“你看到的那个,她就是霍玲。”

我一个激灵:“什么?那怪物是霍玲?”突然就感到一阵恶心。

文锦道:“她从塔木陀回来之后,就开始变了,变成了一只妖怪。”

“这……”

“这种保持青春的效果是有副作用的。”她看着我,伸出了她的手,让我去闻,我一下就闻到了一股淡淡的非常熟悉的香味,禁婆的味道,“到了一定的时候,我们就会开始变化,而我的体内,这种变化已经开始了,不久之后,我就会变得和你看到的妖怪一模一样。”

\chapter{会合}

“这怎么可能?”我看着文锦,摇头表示无法理解,文锦身上的香味,确实就是禁婆的味道没错,但是要说她很快就会变成禁婆了,这也太不可思议了。

“你没法接受,我也不怪你。”文锦幽幽地叹了口气,“当初我们发现这一点的时候,也无法相信。”

我还是摇头,这时候完全无法思考,只觉得一切都乱得离谱了,如果之前我所整理出来的东西全部都是事件的碎片,那文锦给我的这些信息好比一只大锤,将这些碎片全部都敲成了粉,现在连任何拼接的可能都没有了。

“那个它对你们做了手脚,使得你们无法变老,但是,却会使你们变成那种……那种……怪物?”

文锦点头:“按照我的经验,从身体内部开始变化,到完全变成那东西,只有半年时间,我们称为‘尸化’。第一个尸化的,是一个女孩,当时我们看着她一点一点变成那种样子,实在太恐怖了,这种感觉就好像,你的身体省略了‘死亡’这个步骤,直接从‘活人’变成了‘尸体’。”

“可这到底是怎么产生的呢?”我问道,“有没有办法可以治?”

文锦摇头:“‘尸化’发生的时间完全没有规律,唯一的信号就是这种气味,我们推测这种奇怪的变化,可能和西沙下的那个古墓有关。当时第一个想法,是否这是一种古老的疾病,一直被封闭在这座古墓中,我们受到了传染,后来研究了之后发现不是,但是,这种现象肯定和汪藏海有关。”

“这就是你们研究汪藏海的原因?”

她默默地点了点头。

他们在格尔木的地下室里被困了相当长的时间,逃出去的过程相当复杂,文锦虽然也对我简要地叙述了,但这是另外一个故事,这里就不长篇赘述了。

逃出之后,一开始他们受到了一群陌生人的追捕,他们无路可去,经过了一番颠沛流离,他们重新潜到了疗养院,却发现人去楼空,疗养院里所有的东西都被搬空了,他们什么资料都没有发现,根本不知道到底是谁囚禁了他们,又是出于什么目的。为了逃避这股莫名的力量,他们决定反思维而行,选择了这个被废弃的疗养院作为藏身之所,一边调查汪藏海的历史,一边躲避那批人的追查。

之后便有了后面的事情。

说到这里,我就问他们道:“那么,你们是认为,在这个鬼地方,有什么办法可以治疗这种‘尸化’?”

“我们根据大量的细节推测,汪藏海追查的是战国锦书中记载的,一种关于成仙的技术,但是显然他从古籍中复活的这种技术并不成熟,我们可能成为这种不成熟的东西的实验品,虽然我们可以永葆青春,但是效果很不稳定,最终都会变成怪物。”文锦道,“汪藏海这一生追求的必然是完善这种技术的方法,我想这里是他的最后一站,战国锦书中的记载来自这里,那么这里是最有可能的地方。但是在这件事情上,我和霍玲发生了分歧,那一次她自己带人进入了这里而我选择了等待。我一开始以为她死了,没想到过了几个月她竟然回来了,但是显然她并没有成功,当时她的尸化已经开始,她开始健忘,开始情绪失控,她的新陈代谢越来越快,最后还是变成那个样子,整个考察队只剩下了我一个人,等待着未知的命运。”

“我本来想一直隐藏下去,但是在一个月前,我终于闻到了我身上发出的味道,知道最后的宿命到来了,我必须把这一切做一个了结。你的三叔,裘德考背后的那个‘它’。”

“可是,这些和我有什么关系?”我想起来,问道,“为什么你要寄录像带给我?”

“寄录像带给你的,不是我。”文锦正色道,“这又是一个缺失的环节,我看到你出现在队伍中的时候,相当的惊讶,所以让定主卓玛把你也叫上了,从你的出现,我就断推出‘它’已经渗入了我的计划中,所以我向你们提出了警告。它把本来我发给裘德考的那盘带子,寄给了你。”

“它为什么这么做?”

“我不清楚,也许它并不希望裘德考成行,它希望有一支由起灵,解连环和你组成的比较单纯的队伍,我也只能推测。不过,这一次解连环用了非常厉害的计谋,阴差阳错地使得我的计划还是成行了。‘它’一定也在判断,我到底是这么多人中的哪一个。”

我揉了揉脸,感觉思路稍微清晰了,问道:“那你到尸变,还有多少时间?我们还来得及吗?”

她握着我得手道:“你别担心我,已经到了这里,我接受命运的一切安排,不管是好是坏。反正,这里是我的终点,也是起灵的终点,更是解连环的终点,你要考虑的是你自己。”

我看着他们,心说你们都不出去了,这怎么可以。这时,就听到我们做的屏障外,忽然有人轻轻地敲了敲石头,一个人咳嗽道:“里面是不是有人?”

我立即警觉起来,闷油瓶靠过去,我立即叫道:“小心,可能是蛇,这里的蛇会说人话!”

外面那声音立即道:“是不是太天真?”

闷油瓶让我放心,蛇不会和你对话,说着撤掉屏障,立即我就看到一张满是瘀泥的脸,原来是胖子。再一看,他后面还有好几个人,都是三叔的伙计,其中还有那个黑眼镜。

胖子一脸的瘀泥,道:“果然你在这儿,咦,小哥你也在,哎,逮住了?”

我心说你别发出那么多象声词了,胖子就问我们是怎么回事,我说我这里事情真是长了,还是问他们怎么了,怎么找到我们?我三叔呢?

胖子“哎”了一声道:“我们看见有一条缝隙里塞着奶罩,我靠,这真是塔木陀奇景,我们撞了进去就发现了里面的缝隙和瘀泥,我教他们保护自己,不过你三叔没赶上,被咬了,第一时间打了血清,在我们后面。我们听到了有说话声就来看看,我还以为是那些蛇。”

虽然文锦说三叔是解连环假扮的,但是一到情急之处,我还是丝毫没有感觉到他是假的。

我回头看了一眼文锦,心说你打算怎么办,文锦朝我点了点头,“走,去看看。”

后面几个伙计都不认识文锦,问我这女的是谁。

我道:“这是三爷的相好。”胖子立即就道:“叫大姐头。”

那几个人也吓蒙了,还真听胖子话,立即叫。文锦瞟了我一眼,让我少废话。

他们就在不远处的一个蓄水池里,这个蓄水池更大,而且几乎没有什么岔口,同样长满了树根一样的菌丝,这一次,人起码少了一半,全部都面如土色。文锦教他们堵住唯一的一个口子,我就道奇怪,难道这个蓄水池已经是这个蓄水系统的终点了?

我去看三叔,看到他的脖子和胳膊上都有血孔,脸色发青,神智有点模糊。

“咬死了三个人后才咬的他,毒液干了,但还是烈。”照顾他的人道。

三叔微微睁开眼睛,我不知道他有没有看见文锦,应该是看到了,我发现他颤抖了一下,又看了看我,什么话也说不出来。

我心中发酸,看着他的脸,我根本无法想象他会是解连环,我懂事之后都是和他相处的,即使他本身是解连环,我脑海里大部分对于三叔的印象都是来自他,这一切也没法改变。

文锦走了过来,坐到他的边上,看着他,也不说话,两个人就这么看着。三叔忽然吃力地朝她伸出了手。

文锦握了上去,轻声道:“小邪知道了,你不用瞒了,我们都不怪你。”

他动了动嘴巴,我看到他的眼泪一下泉涌而出,看了看我,看了看文锦,竭力想说话。

文锦也有些动容,凑了下去,贴着他的嘴巴,听完后紧紧握住他的手:“我知道了,你归队了,这不是你的错。”

他看向我,我也握住他的手,我不知道我应该说什么,这里的事情发生得太快了,昨天我还在和他聊天,三叔长三叔短,现在竟然成了这个样子,想着不由就叫了一声:“三叔。”

听到我叫他三叔,他忽然激动起来,动了一下,慢慢失去了知觉。我以为他不行了,立即叫人。旁边那个人过来看了看,就道:“放心,只是昏过去了。”

我长出一口气,这时候就听到背后有人叫,“这里有道石门!”

我们过去看,三叔的几个伙计,发现这个蓄水池的底部有一个石板,上面有两个铁环。

他们吆喝起来,用力去拉铁环将铁板抬了起来,就发现下面压着一个洞。

黑眼镜和闷油瓶下去探路,不久便返回,黑眼镜说下面别有洞天,完全不是人工开凿的,好像是一个溶洞,四周有很多的石门,好像是在开凿这里的蓄水系统时候被发现利用了起来。里面空气清新,好像没有蛇的踪迹。好像还能通到其他地方去。

我们来时的道路上可能布满了蛇,从原路返回至少也要等到天黑,也许从这下面有路可以出去,胖子说要么下去看看。

一听好像没有蛇,这里的人都要下去,我对他们说情况不明了,不要一窝蜂拟的全部都下去,现在我们待的地方还是比较安全的。下面可能有机关陷阱,到时候比蛇咬还惨。

这么一说又没人肯下去,最后还是我们几个决定先下去看看,其他的人都是乌合之众,下去也帮不上什么忙,就留下照顾伤者,等我们回来。

闷油瓶和黑眼镜再次下去,接着是我和胖子,紧接着我们的是文锦。

下面是一个环形的巨大岩洞,用矿灯照了一圈,可以看到很多的石门,胖子甩下绳子就往一边走去,道:“哟嗬,真的是别有洞天!”

\chapter{记号的终点}

我赶紧把胖子拉住,转头看了看文锦,她正和一个伙计忙着揭开从绳梯上送下来的装备,没有注意到胖子的举动。

我就问那伙计:“你下来干什么?不去照顾我三叔?”

他咧开嘴巴笑道:“三爷有人照顾,我下来看看有什么可以帮忙的。”

我看他的表情,感觉有点不对,心说不妙,这批王八羔子是一群乌合之众,乌合之众最擅长的就是有危险作鸟兽散,有好处就窝里反。这家伙的表情似乎有什么企图。

很快,又有三个人爬了下来,看着这巨大的环形墓室,他们的眼睛里都冒出火来了。三叔在临行前骗过他们,说这里如何如何肥斗,一路过来吓破了胆,但是一看到墓室就什么都忘了,虽然全是新手散盗,但是盗墓贼就是盗墓贼,对于古墓的贪念比我们更甚。文锦从绳梯上爬下来,看到这样的情况,也面有不善,对我轻声说:“让他们去吧,这些人都是亡命之徒,对‘你三叔’只是表面客气,冲的只是财物,他们都有武器,和他们闹翻了对我们非常不利,反正要是有摸到的东西,就给他们,我们现在也不能阻止他们。”

我一想也是,三叔现在行动不便,就算他能威慑这些人现在也没办法,我一个小三爷,到了这批人嘴巴里叫起来就没有一点尊重的感觉,完全成了调侃,一点也奈何不了他们,想想以前在长沙风光的样子,确实都是沾了我三叔的光了。

我心里有点郁闷,反而是我们受制于人,我预感这些人可能会坏我们的大事。

胖子对这些非常敏感,已经紧张了起来,握紧手里的猎枪,对我们使眼色,让我们走快点,摆脱他们。

一路过来这么多危险,到了最后我发现最大的威胁竟然来自自已人,这真是莫大的讽刺。而且这些人要财也就罢了,如果心黑点,甚至可能要了我们的命,对于他们来说,这辈子也没富贵过,什么道义什么积德都是屁话,这实在是一个巨大的后顾之忧。

闷油甁也带着装备,顺着绳梯下来,我们不再理会那些人,开始摸索着向前走。“非”字形的甬道很快就到底了,我们面前出现了一个溶洞,甬道的尽头有阶梯,顺着溶洞的壁修茸,盘旋而下。

矿灯在这里就不够用了,三叔他们有着大量备用装备,胖子立即打起了照明弹。

三叔装备了好几种照明弹,胖子用的是低空照明弹,这是在洞穴专用的,射程不远,火球飞入黑暗中不久就绽放开来,洞穴被照得雪亮。胖子又打了两发,把四周的死角也照亮。这有点奢侈,不过我们从来就没有装备这么充足过,反正也到了最后的关头,不用白不用。

胖子丢下弹壳,还要装弹打一发,文锦把他按住:“家底再厚也不是你这么用的,而且已经够亮了,再亮反而看不见了,小心把我们眼睛烧坏。”

胖子这才作罢,我们等最闪的那一阶段过去,光线收缩,四周的情形才清晰地显现出来。

这确实是塔木陀的城底最深的地方了,岩洞也不是天然形成的,而是被人开挖出来的,上面还有很高,看不清楚岩洞的顶部,却能看到岩洞的四周如体育场的座位一样被人修成了一阶一阶的,每一阶上面全是黑色的一具具造型臃肿的雕像,密密麻麻,一圈又一圈,没有一处是空的。

这些雕像因为是黑色,仍旧看不清楚细节,我感觉在这里从没见过,难道是秘密雕像,或是皇族特有的图腾,外人不能看见,也不得拥有?

我想起了云顶天宫的藏尸阁,也是这样的格局,就感觉这些雕像也许不是石头的,可能是特殊处理过的尸体。这里或许是皇族的藏尸洞,地位不高的皇族就葬在这里自然阴干。

照明弹越落越低,底下有人工活动的痕迹,我看到有一只石头的圆盘放在最下面,四周是好几十只造型奇特、大小不一的青铜器血,一切都十分的筒陃。看四壁山岩,再没有明显可以继续前进的地方,确实我们已经走到了路途的尽头,所有的迹团,应该就在这个地方可以解开。

胖子看得叹为观止,这里有多深,实在说不出来,王母族不如被称呼为鼹鼠族好了,真是太嗜好挖洞了,竟然在皇城底下挖出这么深的一个地方,目的何在呢?

文锦说:“这里可能是王母国的圣地,西王母的皇族进行秘密活动的场所,他们可能在这里举行某些极度机密的仪式,或者进行某种宗教的修炼。”

胖子道:“我操,他娘的这个圣地太破烂了,实在让人失望,这些王母族也是缺心眼,这些青铜器是什么,还有这些石雕,雕的是……我的天!小吴,你看这些石雕都是什么东西!”

胖子一惊一乍的,我给他吓了一跳,此时照明弹落到了地上,还在燃烧,但照明范围已经大幅减小。我抬起矿灯去照着,仔细一看,几乎大叫了出来,原来这些围在洞穴壁上的“石雕”,根本不是石雕,而是成排的王俑!

我不住地倒吸冷气,七星鲁王宫里的记忆如潮水一般涌了出来,同时闷洞甁也发出了一声呻吟,显然是受到了极大的刺激,眉头紧锁起来。

果然,这几个点都是有联系的,这里竟然会出现如此多的玉俑,难道每一具里面,都有一个活尸吗?

胖子胆子大,立即扒着墙壁趴到一处阶梯上。我怕他闯祸,一把把他拉住,对他道,要到下面去看最底层,不需要费力气。

我们收敛心神,快速顺着石头台阶往下,到了最后一阶,胖子跳上去,来到一具玉俑之前,用矿灯一照,就照出了里面的尸体,是完全干化的干尸,因为缝隙太细看不清楚细节,一具一具照过来,每一具玉俑内都有。

“看来,大姐头说得没错。这里真的可能是他们修炼的地方。”胖子道,“妈的,这批干巴巴的东西,难道就是中国神话里西王母座下的众仙?这也差得太远了吧。”

“不过这些玉俑和鲁王宫里的有点不同。”我道,“鲁王宫里的玉俑,里面的尸体还是活的,这些好像都已经成干尸了。”

“那是因为时间,这个岩洞应该是在西王母国鼎盛的时候挖掘的,那应该是在五千年前,经历了如此长的岁月,再有水分的东西也被风干了。”

胖子用手去抚摸黑色的玉俑外壳,闷油甁抓住他的手,让他小心,我道:“这东西少碰为妙,小哥当时不是说过,如果时间不对,玉俑脱壳后就非同小可。”

胖子郁闷道:“我就是摸摸,让我留点回忆行不?”

我说你别身体一好就忘了伤痛,心说说了也没用,就不再理他。一边的文锦已经被其他的东西吸引,往全是青铜器皿的地方走去。

我跟了上去,惊讶地发现这些青铜器巨大无比,站在下面看,比我还高,而且造型奇特,我一只也叫不出来名称。不过,每一只青铜器显然都有自己的作用,我看到上面惊人的腐朽,使用的痕迹明显,显然这里不是一个用来摆设的地方。如果这个洞窟是当年的西王母族用来修炼或者进行宗教仪式的地方,那这些东西应该和修炼及宗教仪式也有关系。

这时候就听文锦喃喃道:“天,这里是西王母的炼丹室,竟然真的存在。”

\chapter{炼丹室}

我朝她看去,见她已经走到了最中心那巨大石磨一样的石磐边上。我们也靠过去,就看到那是一只石头的星盘,上面全是罗列棋布的小点,代表着天上的繁星,而每一小点上,都是由一颗墨绿色的丑陋小石头表示的。

这就是三叔以前给我看的丹药,这里竟然有这么多。

“这是什么?长生不老药吗?”身后传来一个陌生的声音,伴随着一声口哨声,我们回头一看,原来那个拖着带几个伙计已经尾随我们而来。

我立即觉得头大,摇头:“这是吃了会立即挂掉的剧毒,绝对不能动这些丹药,剧毒无比。”

“当然不会去吃咯,只是看看不成么?”

“不成。”我道,“这里什么都不能碰。”

那几个人很有兴趣,听我这么说悻悻然就嘀咕了几句,一个就点起了烟,道:“你算什么东西,这么多规矩。”言语中已经没有之前的客气了。我假装没听见,这时候四周燃烧着的照明弹逐个熄灭了,胖子又打了两个,抬头看了一下,忽然大呼小叫起来。

我们全部抬头看去,只见照明弹在最高处,就照出在这个山洞的最顶上,有好几条铁链悬挂着什么东西,十几条铁链呈发散的形状,犹如一只蜘蛛网,一边镶嵌在石头里,一边连在那个东西上,那东西黑漆漆的,好象是一只巨大的黑球。

照明弹随即落下,山洞上方又陷入了黑暗之中。

“那是什么玩意儿?”旁边有人惊讶地自言自语。

“这是悬空炉。”文锦惊道,“天哪,这个洞,肯定就是大风水万山龙母的穴眼,这是炼丹室的最高境界,丹炉的最高境界,丹炉不着地,尽收整条龙脉的精华。”

胖子换上高空信号弹,道:“看个清楚。”

又是一发,这一次照明弹竟然一下打在了那黑球边缘,炸起来,一下看得无比清楚。只见上面果然是一只雕花的青铜球状器皿,比这里任何一只青铜器都要大三倍以上,从下面看上去,和那些铁链连在一起,犹如伺伏在蜘蛛网中心的巨大狼蛛。

文锦立即让胖子不要再发射了,说丹炉之内不知道会不会有易燃的东西,等一下引起爆炸,我们等于被轰炸机轰炸,这里的人一个也别想活。

胖子叹气道:“可惜没法上去看看,不然也许长生不老药就在上面。咱们吃个一打,也直接上月亮上去,不知道嫦娥最近混得怎么样。”

我拍了一下胖子,叹气道:“你终于露出马脚了,天蓬元帅,难怪我看你的体形这么面熟。”

那拿丹药的人笑起来:“小三爷,你还真以为你是爷啊,时代变了,现在人不讲辈分了。”说着,他就挖出了一颗丹药,用手电照着,仔细去看。我身边闷油瓶的脸色却变了,我听到他轻声叫了一声:“完了。”

话音未落,那石盘忽然失去了平衡,朝一边歪了一下,接着,四周一片寂静。

那几个人也吓了一跳,所有人都不敢动了,全部定在了那里,等待着事态的变化。

等了一会儿,什么都没发生,我们面面相觑,胖子道:“我靠,这石头没放稳当?”

闷油瓶的脸色却更加的苍白,他不去看那石盘,而是把目光投向了四周的玉俑。接着,我们就清晰地听到玉俑之中“哗哗”几声,立即寻声看去,发现一具玉俑身上的俑片竟然散了开来,似乎是一下子玉俑穿着的金丝被抽离了,俑片立刻没了形状,散落下来,露出了里面的古尸。那是一具狰狞无比的马脸古尸。

我顿时结舌,听说玉俑脱落之后,尸体立即尸变,这事情就大条了。想着立即大叫:“快退出去!”

还没说完,就听道洞口处一连串机关锁动的声音,来时的石头门闸已经落下,封住了我们的去路。

\chapter{机关}

这石盘之下设置了一个平衡陷阱,所有的星图星点上的丹药重量都是经过精确计算的,拿的顺序必须严格的遵守,按照固定的顺序去取下丹药,才不会触动机关,否则平衡立即被破坏,机关倾倒牵拉机括,引起连锁反应,四周的玉佣立即脱落,血尸尸变。

这里可以说是王母族最重要的圣地,如果这里被侵入,相当于皇族最核心的机密有暴露的危险,所以这里设置了如此可怕地机关,完全是为了同归于尽。

我们现在的处境可以说是极端的绝望,我们来时的洞口现在已经封住了,所有人都被围在岩洞底部的这片区域内。

三叔的那几个伙计已经吓瘫了,不要说我们,就是胖子和闷油瓶也失了血色,这种阵势可能连我爷爷也没见识过,他的笔记上也没写要是碰上一千只粽子同时尸变,应该怎么来管理和运营,他娘的不知道倒斗这行有没有EMBA读。

当下在干尸群中,突然就发出了一连串的“咯咯咯咯”的声音,接着又是一处,很快到处都是这种声音。同时我看到这些干尸身上的干皮不停地脱落,似乎是真的要起尸了。

那拖把看向我们,大吼了一声:“你们他娘的在看什么,还不想想办法?怎么办?”

胖子骂了一声,捡起地上的枪,道:“怎么办?咱们现在可以比比看谁活得更久一点。”

“你放屁,老子可不想死,快给我想办法,不然我毙了你。”那人把枪指过来。

胖子检查了一下子弹:“你可以投降看看,不过可能不管用,这里这么深,上帝要过来可能也没那么容易。”

说完就朝血尸靠过去,抬头开枪,把最近的几具干尸打得趔趄了一下,那身上的干皮被轰掉,我们就看到了里面青紫色的尸皮,子弹打上去,只能打出一个豁口来。

我看得出胖子已经释然了,虽然还是感到恐惧,但是他心里已经接受了死亡。他连开了三枪,那些伙计才反应过来,立即帮忙,先下手为强,能活一分钟是一分钟。

胖子一边换子弹一边走到身边,掏出信号弹给我,对我道:“保持照明,不要射上面,射到他们脸里去,咱们要学狼牙山五壮士了!”

“上面?”我抬头看了看头顶,忽然有了个灵感,想起了爷爷笔记里刚开始讲述的故事,他是怎么说的?

爷爷当时第一反应,就是这些血尸不会上树!

不会上树,那更不会上墙了,攀岩就更不会了。我想到这里,立即对他们道:“我们得想个办法上去!到悬空炉上边去,他们既然能把炉子修得这么高,而且四周没有阶梯,那肯定有其他办法可以上。”

一下子大家都感觉到有了一线生机,所有人立即行动了起来,胖子大叫不要乱,有枪的做好防守争取时间,没枪的去找。

我立刻冲向边上的一个青铜器,这些东西都有一人高,爬上去之后看得清楚。

但是上去一看,我一下子就发现不对,要是有任何可以上去的办法,我们之前肯定可以看到了,而且我知道一般古人的设计理念,是人不动而形动,这个悬空炉不是修在上面,而可能是被吊上去的,任何的操作还是要在下面进行。那样我们是不可能上去的,因为这炉子下来之后我们没有力气把它再拉上去。

不过我站在这个青铜器上,就发现我们不一定要爬的这么高,只要爬到那些血尸够不到的地方就行了,那这青铜器就足够了。

想到这里我立即大叫,几个人马上反应过来,都往我站的青铜器上爬。

很快所有人都爬了上来。阶梯上,更多的血尸开始站了起来,我一看,发现不对,这些血尸非常魁梧,这高度还不够,但是没有更高的青铜器了。居高临下的射击,只能暂时缓住几只血尸的靠近。矿灯照出去就看到好几只怪脸已经离我们很近了,而矿灯没照到的地方更是不能想象。

就在几乎绝望之际,胖子大叫:“伙计们,要拼命了!”说着抖出了几根雷管,叫道:“我冲过去,一路扔炸药,炸出一条血路来,你们在四周掩护,我们就往前冲。”

我一看大叫:“这玩意你从哪来的?”

“上回我不是说过,没炸药我再也不下斗了。”胖子大叫道:“老子的私藏!”

我一看虽然这方法等于自杀,但是总算也有一线生机,大吼了一声:“拼了!”

胖子大叫道:“只有四根雷管,距离那么远,所有人必须跟上,有一秒落下就救不了了!”

说着拔掉引信,甩出了第一根雷管,我看着冒着烟的雷管甩入干尸群,立即一蹲,顿时一声巨响,冲击波把几具血尸都冲了起来。我们低头让过炸飞的碎石和碎片,青铜炉被打的坑坑洼洼,当当作响。抬头一看,果然前面炸出了一个口子。

胖子跳下去,立即丢出第二根雷管,大叫:“冲啊!”

我们立即跳下青铜炉,那一瞬间,爆炸又起,这一下没有青铜炉做掩护,碎石头如子弹一样朝我们飞来,我们几个立即给掀飞。但是也顾不上剧痛,胖子跳起来又是一根雷管甩出去,有枪的人朝向四周,立即开枪把涌过来的血尸打下去。

我们继续不要命的往前跑,简直和战争片一样,又是一记爆炸,我们扑到在地一秒,等气浪飞过,再次狂奔,所有人的耳朵都震得嗡嗡响。我想上甘岭也就是这种感觉了。

胖子打吼:“最后一根了,冲啊!”

说着雷管甩出,就往石门处扔去,这一根一定要能炸开石门,否则我们就白干了。

我们死命往前,一边毛腰等气浪冲来,可是几乎冲到了,那雷管却没有爆炸。冲在前面的胖子,一下停了下来,回头大叫:“不好意思,判断失误!臭弹!”

身边的血尸立即围了上来,空气中充满了火药和血尸特有的那种辛辣气味。我们围起来,做了一个圈,我大叫:“用枪,打那根雷管!”

胖子道:“被挡住了,看不见。”

只见闷油瓶猛地跳了起来,踩着胖子的肩膀用力一蹬就飞了起来,双膝凌空一压,一下子卡住一具血石的脑袋,用力一拧就连着它的脑袋一起拧了下来,然后用力一脚把无头血尸踢进堆里。那无头血尸翻倒在尸群,露出了后面的雷管。

胖子动作非常快,甩手就是一枪,顿时那雷管就爆炸了。我们此时离雷管十分近,这一下就中了实招了,所有人都炸飞了。

我头晕目眩,爬起来就呕吐,咬牙不让自己晕过去,站起来一看,只见石门竟然没破,上面炸出一个大口子,仔细一看我才发现石门里面竟是青铜。

完了,我爬起来,看着四周的血尸,心说彻底完了。还没站稳,身后突然一声犹如暮鼓晨钟般的巨响,整个洞穴都震了起来,把我们全部都震翻在地,四周的古尸也大面积地翻倒。回头一看,只见刚才看到的巨大悬空炉因为炸药引起的震动,悬挂的铁链终于断裂,从洞穴顶上掉了下来,狠很地摔进洞穴底部。巨大的重量竟然把洞穴底部砸出了一个大洞,炉身深深地嵌了进去,这洞穴底部好似还有空间。

丹炉的蜂鸣声让我头脑发麻,一边的群尸围绕过来,我们有好几个都站不起来。闷油瓶大叫:“退回去!我来引开它们。”

我们看来路因为一路炸过来,血尸还没有完全聚拢起来,只得重新退回去。闷油瓶对胖子大叫:“刀!”

胖子一边开枪一边甩出一把匕首,闷油瓶凌空接住,一下划开自己的手心,对着那些血尸一张,那些血尸顿时好象被他吸引一样,全部都转向了他。他离开我们,就往上走。那些血尸不知道为什么,立即就跟了过去。

我们就趁这一瞬间,迅速往底部退去,我大叫:“你怎么办?”

闷油瓶没理我,胖子就拉着我就往后退。一直到我们退到底部,闷油瓶已经淹没在血尸群里面了,连影子也看不到了。那拖把就道:“他妈的够仗义!”

我抢过他的枪大骂:“够仗仪你妈!”就想冲回去,心说怎么可能让他牺牲掉,胖子将我拉住,对着那边大叫:“小哥,我们到了!”

忽然看到了闷油瓶从血尸群里翻了出来,犹如天神一般踩着一边的几乎垂直的岩壁就蹬了上去,然后一纵跳出了包围,借着冲击力就地滚到血尸稀疏的地方,接着就看他几乎是毛腰贴着地面在跳,从血尸之间迅速穿过,瞬间就退到丹炉边上。

几个三叔的伙计都看呆了。闷油瓶翻过来之后,对我们道:“这些血尸还没有见血,关节还硬,不象在鲁王宫那只浸在血里的,否则我们一个也跑不了,别发呆,看看可以往哪里跑。”

我们这才反应过来,一下就看到丹炉深陷入底下的空洞中,四周圈是裂缝,通往地下,果然下面还有地方。入口应该是被那石盘压住,我们没有发现。

此时没有选择,我们趴到丹炉身上,手挂住它身上的纹路就往下攀爬。

这底下是一个只有半人高的夹层,连蹲着都抬不起头来,下面全是碎石,我们下去之后,立即摸起石头,将那缝隙堵住。直到堵到一点缝隙也看不见,我们才松了口气,全部瘫痪在地,我的耳朵几乎听不到声音了,只觉得天旋地转。

文锦立即撕下衣服给他止血。

胖子用手电观察四周,发现这是一个很小的石腔,而且同样是人工凿出来的,只有六七个平方大,丹炉砸在里面就显得更加狭小,根本不能活动开手脚。

“我靠,现在我们怎么办?那些东西会不会散开?”有一个伙计问。

“一般情况下,有太阳能把他们晒倒,不过这里是没什么指望了,我们得另想出路。”胖子拿着手电乱照,忽然,我们都看到一边的岩石上,有人刻着什么东西,一看,是闷油瓶用的那种文字,却不像是记号,而是一句话。

所有人全部都凑过来,胖子就喜道:“小哥,你看这个,是不是表示还有路下去。”

闷油瓶猫腰过来看了一下,脸色就一变,我们问他这是什么意思,他摇头,但是我看他的表情,显然是看懂了。

但是刻记号的地方是一块山壁,胖子摸了摸,找不出破绽。闷油瓶过来,用他奇长的手指顺着山壁上的纹路摸了一把,就拿起一块石头,开始砸,连砸两下,忽然那石头如粉糜一样裂了,他一撞,就撞出一个只能容纳一人,匍匐着才能勉强通过的洞。

“这里怎么会有盗洞?”胖子惊讶道。

“不是盗洞,这是用来设计机关用的管道,我们上面的机关就是在这里面动。”闷油瓶道,已经大头钻了进去。

我们互相看了看,陆续跟上,匍匐进去之后不到十米,突然转向垂直向下,我们在里面没法掉头,只得头朝下爬。大概爬得脑充血快晕过去了,忽然听到水声。

有水,那就是和渠道相通了,当下立即加速,很快到了尽头,就发现一石块挡住了去路,闷油瓶用力撞了几下,把石头撞出去,石头滚下去,下面传来了水声。

我们探头出去,发现外面是一条宽阔的水道,水流平缓,而且并不深,看着是到腰部,水流清澈,能看到水道底部的石板。

闷油瓶打头,几个人陆续下去,一入水就发现水下一阵骚动,无数的虫子被我们惊扰的散了开来,几个人吓的差点开枪。

我也吓了一跳,见这水道里全是一种没有壳的肉色小虫子,浑身透明,平时伏在水底几乎看不到,好像没有什么攻击性,我们一动他们就四散而逃。

全部下到水道之后,几个人照了照水道的两边,只见水道的上游是一道铁闸,闸外堆满了从上游冲下来的树枝杂物。下游一片黝黑,不知道通向哪里。

我们来到铁闸处摇动了片刻,发现无法撼动,十分的结实。

“这里是什么地方?”三叔的一个伙计问。

“这里的水渠这么深,水流量这么大,可能是通往最下方蓄水湖的主渠道了。”文锦道。话音未落,忽然有人就叫起来,我们转头望去,只见下游的水道中间,竟然立着一只人面鸟的雕像,有两米多高,出现在这里非常突兀。

我们走过去,就看到雕像和我在雨林中看到的几乎一样,正想仔细看,只见闷油瓶吸了口凉气,忽然绕过雕像,往下游走去。我们几个互相打了一下眼色,立即跟了上去。

\chapter{近了}

一路走过,那些没有壳的肉色小虫被我们惊扰,纷纷潜入水底,不知去向。

胖子弯下腰摊入水中,想去抓上几只,被我拦住,这水下情况未明,我们过多的惊扰恐怕会引来麻烦,能不折腾就不折腾。而且这些虫子我从没见过,可能是一些特殊的品种,全世界可能就只有这里生存着,价值连城,被他弄死几只太可惜了。

胖子骂道:“你看这些密密麻麻的,我看这里的水里没十万也有八千的,抓几只带回去有什么关系,这一趟已经基本上白来了,你也不让我弄个纪念品当念想。”

我说:“这肉呼呼的东西,看着就恶心,你怎么下得去手,别琢磨这些旁门左道的东西了,咱们赶紧过去是真。”

这么多虫子在这儿,就没人想休息,我们只好继续顺着这条水道往深处去寻找尽头的地下蓄水湖,这里水流平稳,前面也没有巨大的水声,显然没有大的断崖,我们可以从容向前。

我们继续前行,越走水越凉,能感觉到一股寒气在水中蔓延,身上都起了鸡皮疙瘩。我们在水道的两边看到了无数那种肉色的虫子,大部分都趴在水线上下地方的石壁上,密密麻麻,看着我就开始头皮发麻,水中更是多,不时感到有东西撞到我的脚上。

水道越来越宽,道顶越来越高,呈现一个喇叭状的开口,我知道快到了,立即加快了脚步。走了不到一百米,头顶上一黑,我们就出了水道,周围的空间一下变得空灵而有回音,凭感觉就知道来到了一个大地方,脚下是一片浅滩往前蔓延,矿灯的光柱划过,便看到一片宽阔而平静的水面。

矿灯有弱光和强光选线,为了省电我们一般都选择弱光,这样你能持续是有180小时以上,但是照射距离只有二十多米,现在弱光显然无法达到要求了,几个人纷纷打开枪管,使用百米照明LED灯泡,去照头顶和四周。强光下,这里的大概面目才显露出来,能看到这时一个巨大的地下水洞,但不是喀斯特地貌,而是那种火山岩洞穴。远处洞的深处大量从洞顶垂下来的巨型石柱插入湖中,犹如神庙的巨大廊柱,洞顶只有两三层楼高,整个地方乍一看感觉像淹没在海里的波塞冬神庙大殿,气氛形象之极,不得不说是大自然的鬼斧神工。

水道出口的两边是巨型岩壁,呈现火山岩特有的特征,有岩层的出现,说明我们已经越过了砂土层到达戈壁地质深处的地下山脉之中,这些岩壁肯定是昆仑山渗入地下的部分。回头看水道口感觉是人工开凿出来的。西王母在当时那个年代,能挖掘到这么深的地方,不能不说他们文明有着极度发达的工程能力。

这里应该就是整个西王母古城地下蓄水系统的重点,一个天然的小型地下湖了,因为矿灯光线的照射距离有限,我们无法得知这片蓄水湖到底有多大,中心有多深,也许往湖的中心走,湖底可以深到我们无法想象的地步,但是看不到开阔的湖面也难说有什么被震撼的心情。观察片刻,胖子就问接下来应该怎么办。没有什么新的办法,还是要寻找闷油瓶的记号,之前的记号就是指向这里,再往前就是地下湖的湖心,之后的引路记号不可能刻在水底,我感觉应该会在这些石柱上。

我们分开去寻找,淌水往湖的深处走,照射那些石柱。

走了几步我发现湖水的深度变化不大,偶有深下去水淹到脖子的地方,但是走几步又上来了,显然水底坑坑洼洼,但是平均深度变化不大,很快黑瞎子就打了个呼哨,我们走过去,发现有一根石柱子上果然有清晰的记号,刻得端端正正。

文锦看着闷油瓶问道:“这里的水流基本上平了,没有继续往下走的迹象,我看这里是整个蓄水工程最低的位置了,我们要找的地方肯定就在前方,到了这地步,你还不能想起什么来吗?”

闷油瓶摇头不语,只是看着他刻下的痕迹,眼神中看不出一丝的波澜,胖子就说西王母古城可以说处在一处秘境之中,在全盛时期这片绿洲湖水环绕,外面是无数魔鬼城形成的保护层,绿洲内有终年大雾,只有大雨的时候才能看见。西王母城的居民信奉残酷的蛇崇拜和神秘主义,使得这个沙漠中的政权如同鬼魅,晦涩难窥,而这古城之下犹如迷宫一般的蓄水系统又错综复杂至极。我们现在几乎耗尽了心力到达了这所防御工程的最底层,要是西王母有什么东西要藏的,也应该就是在这个地方了。什么都别说,顺着这些记号继续走应该就能到达目的地。

我觉得有点不妥当,这一路过来,到了后一段几乎太过顺利,在水道中看到的人面怪鸟的雕像让人无法不在意。我们一路过来,已经可以肯定这些人面怪鸟的图腾应该就是西王母国的先民警告外来人的标示,从硅谷外围一路深入,每看到一次遇到的怪事就险恶一分。这次又看到人面怪鸟图腾,说明这蓄水湖必然不会是一个平和之地,现在我们其实都累得只剩半条命,一旦出事,恐怕这次一个也逃不脱了。

我问文锦:“接下来采取何种策略,我们是休息一下,还是先派人探路?”

文锦道:“已经到了这里,如这个胖子说的,我没有理由退缩或者放弃,这是我命里注定要走的路,但是我们没有必要所有人都过去,后面不知道是什么情况,你们在这里休息,我一个人过去就行了。如果我两个小时内不回来,你们可以顺着湖岸寻找其他的出口,再想办法出去,千万不要过来了。”

闷油瓶在一边淡然道:“我也去。”他压根没有看我们,只是看着湖深处的黑暗,似乎完全没有考虑什么危险。

我想了一下,我也必须过去,不说待在这里有多少机会能出去,来路已经被困死了,我历尽千辛万苦到了这里,不就是为了这一刻吗?而且以我的体质,能够到达这里可以说有很多人为我做出了牺牲,包括生死不明的潘子和枉死的阿宁,我如果再没有出息的缩着,当初就真的就不应该来这里,既然是我自己要来的,那么我也应该走完。

胖子就咧嘴:“我靠,你们这不是逼我也去吗?和这批菜鸟在一起还不如和你们在一起安全。”

这一来三叔的几个伙计也不干了,都要跟去,他们确实都没什么经验,搞点小偷小摸可以,把他们留在这里他们肯定不干,而且他们也怕我们通过这种方式结党,偷偷甩下他们跑掉,所以决计要跟在我们后面。为首的那个叫拖把的就道:“你们想的美,他娘的要么留一个下来,要么咱们一起去,别想甩掉我们。”

黑瞎子一直没说话,自个儿在那儿似笑非笑,看这情形就过来搭到我的肩膀上,也不知道是什么意思,可能意思是他也加入,或者是让我留下。

我看着那批人就觉得恶心,这些人实在是个累赘,跟着我们不知道会出什么事情,我们还得防着他们。要是我留下,不给他们折腾死。

胖子道:“小吴你就算了,你还有大好的年华,跟着这些爷们,也许还有条活路,你三叔不是说吗,这是一条不归路,这路由我陪着大姐头和小哥走一趟,来年还多一个人给我们上香。”

我骂道:“你少来这套,到了这份上,横竖都差不离,反正我是去定了。”

我这话是实话,其实到了现在这种地步,谁有信心说一定能出得去?搞不好我们来的那条路就是唯一的通道,这里就是地下岩山中一个完全封闭的水洞,我们不得不困死在这里。这也未尝不是好事,让这些谜团在这里完全画上一个句号。

想到这个我反调侃胖子,拍拍他的肩膀:“到是你,要是有个三长两短,家里的大奶二奶抢你那点压箱底的明器肯定要抢破头了,你还是留下合算。”

胖子道:“你胖爷我是出了名的亮马桥销金客,万花丛中过,不留一点红,钱袋里的银子不放过夜,睡过的女人无数,用过的钱也够本,少有人能活到胖爷我一半潇洒,这一次若是不走运,我也值了。”

我道:“这么说你倒是最适合给人家陪葬,了无牵挂。”

胖子说:“你这话说的欠缺,陪人家送死也要看人,咱们这几个人真叫缘分,你要去,冲着你的面子我也得护着你啊。”说着拉枪上栓,就问那几个伙计要子弹,说你们几个脓包,子弹都放他那里能救命,否则就浪费了。

我呸了一口,一边见文锦拔出匕首甩了下头发试了试刀锋,对我道:“好了,别贫了,既然都要去,那就抓紧时间吧。”

既然要走就不再犹豫,我们抓紧时间各自喝了几口烧酒,把队伍拉开,顺着闷油瓶留记号的方向,开始淌水而行。大概是人多的关系,看着前方深邃的黑暗,我倒不是感觉特别的害怕,只是心中有种难以形容的忐忑。

之后是一段几乎没有任何对话的过程,我们分了几个人,每人都警惕着队伍四周的一个方向,特别注意水面的涟漪,耳边的声音只有我们淌水的破水声,这一路走的不快也不慢,逐渐远离了来时的入口。

好在这里的水清澈的离谱,用矿灯对着水底直射,我们能清晰地看到水下只有高低不平的碎石,并没有什么特别的东西,扫过水面也能大概看到水下的情形。

想着以往的一些,我们并不敢放松哪怕一点注意力。但是,我看着四周水面的时候,已经感到一点奇怪的地方,让我十分的在意。

走了一段,文锦就提了出来道:“这里没有那种虫子。”

胖子点头道:“可能是因为水温,这里的水可他娘的真凉。”话说,这里的水有很大一部分从这个洞形成的时候就囤积在这里了,过了保质期上万年了,大家千万别喝,可能会拉肚子。

我道:“这种水叫老水,自然沉淀富含矿物质,会不会有可能这些水含有有毒的矿物,所以那些虫子不敢游入?”

胖子听了啧了一声:“不会吧?难怪我觉得屁股里有点痒。你们有没有什么特别的感觉?”

没人接话,走在最前面的闷油瓶回头看了我们一眼,我们也只好闭嘴,到了这份上,讨论这些完全没有意义。殿后的黑瞎子就笑,这两个人一个黑,一个白,一个冷面一个傻笑,简直好像黑白无常一样,让人无语。

继续走,我们深入到了蓄水湖的内部,四周手电照去全是平静的水,半个篮球场大小的黑斑,这说明在湖底开始出现起伏很大的深坑,每一个黑斑都极深,矿灯照不到底部,似乎下面连着什么地方。

这种黑斑,隔三差五就会出现一个,形状也不规则,水底全是细碎的石头,这些洞就像是被什么东西挖出来的。我们开始感觉有点不妥当,竭力避开这些深坑。

这么走着,不久我们便找到了第二个刻有记号的石柱。

一行人停下来休息,有人打了个喷嚏,这里的水实在是冷,但是我知道这不是最难受的,这些水怎么说也没到冰点,还在人可以忍受的范围,所以并没有怎么抱怨。

那个记号,指向了另外一个方向。而且符号也不同了,似乎变换了什么意思。

文锦看向闷油瓶,还没开口问,闷油瓶就回答了:“这时最后一个,我们就要到了。”

最后一个——应该是最后一个记号的意思,这说明下一站就是目的地了。

我们心中一震荡,后面就有人下意识的举枪了。二话不说,我们顺着记号马上动身,我心中也不知道是什么感觉,既兴奋,又害怕,又感觉到不祥的气息,同时还有一种事到临头的紧张。

可就在绕过石柱走不到两三步的时候,我的脚下一阵刺疼,不知道踩到了什么东西。

我小时候在长沙,经常和三叔在溪涧中游泳,所以凭着脚底的感觉,我立即就知道脚底肯定破了,而且还比较严重。

我马上停下,让胖子帮我照一下,说着抬脚去看。胖子的矿灯划过水面照到我的脚上,我发现脚后跟被划了一大道口子,显然水下有什么尖锐的东西,我低头去找。这一看,却发现这里的水底,有不寻常之处。

\chapter{终点}

在齐腰深的水下,矿灯光清晰地照出水底,我原本以后我脚下踩的还是那些细碎的石头,然而不知道什么时候却不同了。在我们脚下的碎石中,出现了一些形状奇怪的片状石片,我探手下去摸了一片,发现那竟然是我们在魔鬼城挖出的古沉船上看到的那种陶罐的碎片。

这些陶片被埋在碎石中露出了一小部分,必须仔细看才能和细碎的石头分开来,显然到了这里,出现了古人活动的痕迹。但是看数量,好像不少,都隐在碎石的下面。

所有人开始用脚拨开那些碎石头,很快更多的碎片露了出来。胖子把矿灯举高,把我们站的地方四周照了一个遍,我们得以更加清晰地看水底的情形。

在这里的碎石下面,混杂在大量的陶罐碎片,埋得并不深,从我们站的地方一直往湖底的远处延伸,看不到尽头,而且越往闷油瓶留的记号所指的方向,这些陶片的数目越密集,我看得出这是被什么力量从那边冲过来的。

胖子挖得深了,发现碎石下得深处还有不少,以这样的规模,根本无法统计原先到底有多少罐子埋在这里。水中这些陶罐的碎片棱角分明,十分尖锐,好像一把把刀片,在碎片之中还混杂着人的骨头,已经腐朽得满是孔洞,基本上也是不完全了,有些甚至还粘着一些头发,让人不寒而栗。

这样的场面,看上去很像我在西沙附近看到的海捞瓷铺满海底的场景,当时也是整片海底都是瓷器。但是瓷器是埋在白色的海沙里,显得古老而神密,而这些丑陋的罐子是在碎石中,加上里面的人骨和头发,只让人感觉恶心。

看着那些头骨,我们都有点起鸡皮疙瘩。“这些是什么鬼东西?”胖子就咋舌道。

我和他们说过在雅丹魔鬼城挖掘沉船之后发生的事情,但是他们并不清楚详情,我于是向他们解释这些就是当时发现的陶罐。按照乌老四的说法和浮雕的显示,这应该是一种给蛇的祭品。

“难道这后面也是艘沉船?”胖子一边划动矿灯一边道。

我摇头,估计不可能是船。一来,不可能有沉船会沉在这么深的地下,除非这个湖有水道通往外界。二来,这些罐子属于那些蛇的祭品,应该是放在和祭祀活动有关的场所,我想这里肯定和西王母的宗教有关系,数量这么多,看来这种罐子在当时并不是罕见之物。

乌老四对于这是祭品的说法我还是比较赞同的,不过知道这个也没有什么意义,我脑海里又想起当时乌老四的惨叫声,不由感觉脚底如针刺一般。

想起在魔鬼城的经历,我还是有点后怕,不过这里应该不会出事。看这些罐子的破损程度,里面的虫子必然就不在了,人骨也都糜烂了,一碰就酥,这些东西被水泡了上千年,没有成尘埃已经不错了。而且陶罐是吸水的,如果有密封的陶罐,在水里埋了这么久,水早就一点一点透进去,里面肯定被水充满了,虫子应该淹死了。

“这么多祭品,会不会这后面就是西王母的坟墓所在?”三叔的一个伙计问道。

我想了想,不能说没有这个可能,但这也是完全无根据的猜测。心说最好还是不要。

胖子道:“管他是什么,咱们得小心点,别踩到那些陶片,不知道这些骨头有没有毒,小吴你还是快点洗洗,小心你的伤口感染,等下要截肢可就惨了。而且既然这些是献给蛇得祭品,那这里就可能会有那种野鸡脖子,我们一定要小心。”

“谢谢你的关心。”我没好气地瞪了他一眼。他丝毫不在意,又奇怪道:“说来奇怪,说到那些蛇,好像进了这里之后就没看到过了,那些挂腊肠到哪儿去了?”

扎破我脚的,不知道是这些头骨的骨片,还是有陶片被我踩碎了,反正随便哪一样都不是好东西。

这时黑瞎子潜入水里,从里面挖出来了半块头骨,后脑勺已经没了,可以看到脑腔里面灰色的胶质,像蜂巢一样的组织,这应该就是那些尸鳖王的杰作。为何这头颅之中会有尸鳖王,完全不可考证,不过看这意思来猜,似乎这些陶罐泥封着人头是为了饲养这种恐怖的虫子,这倒是有点像现代人养蜂。如果乌老四的推断是正确的,这种行为可能起源于西王母时期某些诡秘的习俗,不知道他们从哪里抓来这种在人脑子里筑巢的虫子。

我们在碎片中继续往前,特别注意着水下以免被陶片划伤,情形越来越分明,越往里走,脚下的陶罐碎片越多。这样踩着走了不到一公里,我们发现自己来到了一片完全由陶罐碎片堆积成的浅滩上。

整块区域都是陶罐的碎片,大大小小,颜色大部分是暗红色和陶黄色的,而在这些陶罐碎片下面可以看到埋着不少看似完整的鬼头罐,看着好像水底之下还垒了好几层。

我们无法得知碎片下面埋了几层这种东西,不过这场面已经够让人毛骨悚然的了。怕踩破鬼头罐,我们不敢再贸然挺进,于是停下来找路。

胖子对这些破烂不感兴趣,三叔的那几个伙计也不敢碰,都喝着烧酒驱寒。黑瞎子却很有兴趣,一次又一次地潜水下去仔细看这些鬼头罐,胖子就不耐烦道:“四眼,死人你瞧得还少吗?捞那玩意儿干吗?”

找了一圈,四周都是这样,这片区域很大,要想通过,要么原路返回,从边上想办法绕过去,要么就硬着头皮从这些锋利的骨头和陶片上踩过去。

正犹豫呢,我看到文锦看着脚下,若有所思,就问她想到了什么。她忽然道:“会不会我们已经到了?”

“到了?什么意思?”我奇怪,随即就明白了,“你是说,这里就是我们的目的地?”

她点头:“看样子我们已经到达了一个堆祭品的地方,这种地方一般就是祭伺的场所,走了也有一段距离了,你说有没有可能,这个地方是我们的目的地?”

我看着脚下和四周,感觉不太可能,至少我心里无法接受,这算什么地方,这里除了这些鬼头罐什么都没有,那我们千辛万苦到这里来干什么?要看这些罐子我在魔鬼城早看的仔仔细细了。

看向闷油瓶,他还是没有发话,文锦就掏出荧光棒,折了几根让他们亮了起来,甩入四周的水里,把四周照亮。其他人看看,也开始学样打起来荧光棒丢了出去,很快四周的水底亮起了幽绿色的荧光。

我们开始寻找水底任何可疑之处,绿光下的水面鬼魅异常,这一次看的十分仔细,却还是没有我们想发现的任何异样,除了陶片就是陶片。

我们有一些沮丧,我看着水底心说,如果这地方就是目的地,那么唯一的可能性就是有什么东西被埋在这些陶罐下面了。但这应该是不可能的事,这里来过这么多人,如果东西在下面,肯定已经挖了出来了。显然这里不是终点,我们还得继续搜索。

最可恨的是完全不知道我们的目的地是什么样子,闷油瓶又什么都记不起来。

我踢了几脚水来驱散我的寒冷和紧张。就在这个时候,我忽然看到自己的倒影被水波扭曲成了诡异的样子,接着我看到了我的脸和我的下半身重叠在了一起,忽然意识到了什么,抬起头看我们正上方,发现不知道从什么时候起,我们头顶高了很多,看上去一片漆黑。

我拿矿灯往上方照去,灯光照入黑暗之中,看不到顶。这矿灯的弱光照射距离有近四十米,这里的洞顶竟然超过了这个距离。我调节矿灯的照明强度到强光档,一下矿灯光射出一道白炽的光柱。

四周的人都被我突然拧亮的矿灯光吸引了注意力,我没有理会,将矿灯照向洞顶,照出了我们的头顶。

那一刹那我愣住了。我看到,在我们头顶上的洞顶岩石中,镶嵌者一块巨大的无法言语的物体。

这块东西巨大无比,凸出洞顶的部分,呈现球形,完全无法估计其直径,几乎盖住了我们整个视野。看地质似乎也是岩石,但是颜色和四周的四周和洞顶完全不同。奇异的是,这块石头的表面全是柏油桶大小的孔,成千上万,密密麻麻,看上去无比的丑陋,犹如被驻空的莲藕一般。

其他人也顺着我的灯光抬头看天,一下子没人说话,所有人都僵直了,气氛如同凝固。

“什么玩意?”胖子嘀咕了一句。

文锦喃喃道:“天,这……这是一块天石。”

\chapter{天石}

天石是古代人对于陨石的一种称呼,古代人见陨石由天而降,便称呼为天石。天石的种类很多,经常被用作雕刻的材料,最名贵的一种叫天心石。

这确实只可能是陨石,否则无法解释我们看到的现象,人力是不可能在岩层中镶嵌进去如此巨大的一块圆石的。可是这陨石太大了,嵌入岩石中的部分还有多少?简直无法想象。

其他人逐渐反应过来,纷纷拧亮了矿灯往洞顶四周照去,试图寻找陨石和岩顶交接处的边缘,发现这直径足有五六百米,算上岩石内部的大小,估计可能有近一公里的直径。

那些孔洞让这颗陨石看起来丑陋无比,好比一只已经腐烂的巨大的蜂巢。不知为什么,我总觉得这玩意像我们看到的那种丹药,那些孔洞之中漆黑一片,用灯光去照,完全看不出里面的情形,不知道是怎么形成的。看着无数黑漆漆的洞口在你头顶,犹如细小的眼睛,我忽然有一种强烈的被注视的感觉,让人浑身不舒服。文锦道:“这里肯定是我们的目的地了,这里一定是西王母最终的秘密,汪藏海要找的可能就是这东西……”

“他要这东西干吗?这陨石有什么用?”我无法理解。

文锦也摇头:“我还不清楚,可能是这些孔有关系,怎么会有这么多?”

我看着那些窟窿背脊发凉:“会不会是人工挖出来的?他娘的,难道这陨石里面有东西?”

黑瞎子突然道:“不是,这应该是天然的,很多陨石都是蜂窝状的,只不过这些洞的蜂窝难看了一点。”

他突然一本正经地说话,让我很不习惯。三叔的一个伙计道:“你们有没有听说过一种未经证实的说法,柴达木盆地、塔里木盆地都是由一颗分裂的小行星撞击而成的,这玩意也许就是当时的一块陨石碎片,塔木陀这绿洲就是陨石撞击的陨石眼,西王母人在这个陨石坑里修建了西王母城,并且在修建地下畜水池的时候发现了这颗深入地层的陨石,我猜想这东西肯定是西王母神权的象征。”

这是这个伙计第二次说话,我从来没有注意过他,看了他一眼,记不起他叫什么名字,正想问他那个说法的具体内容,却被胖子吸引了注意力。

胖子又无组织无纪律,不知道什么时候和闷油瓶走得非常远,离我们有四五百米,照出的地方我们看不到,那里似乎有什么东西,他吆喝我们去看。我们蹚水过去,到了他们的那个位置,才看到了陨石和洞顶的交接处。这里的情形简直犹如地狱,大量的石柱从上面垂挂下来变成了一大片怪异的巨大石瀑布,坡度很缓能徒步而上,而且大得离谱,简直就是一座小山。

这不是溶洞地貌,这些石瀑布形状狰狞,无比的丑陋,犹如粘在一起的无数巨大的妖怪的触手。这应该是陨石撞击后的高温化岩石形成的奇景,我简直无法用语言来形容。

在其中一条最宽最大的石瀑布上,我们看到了简陋的石阶,石阶的两边放着青铜的灯器,石阶的最上端,就是石瀑布和洞顶连接的部分断裂了,断口被修整成了一个石台。我回头看了一下四周的情况,就明白那一定是祭祀台,在那个台上可以无限接近陨石,又可以一览祭祀的全景。

最关键的是那祭祀台上,能看到放着一只石头的王座,有好几个角,看不清样子,但是个头极大,在王座上,可以看到坐着一个人。

我吸了一口气,心说那是谁,难道是西王母?这么久了她还在这里看守着她的圣地?

\chapter{等待}

远远地看着那个王座上的人影,不是十分分明,是否是西王母的尸身?这种事情我经历的多了,感觉这地方邪气冲天,立即让人准备黑驴蹄子。

胖子说:“不可能是西王母,死了要么埋了,要么趟在棺材里,哪有坐着的道理。我看可能是石头人。”

文锦道:“绝对不可能是石头人,这里不兴人俑,我们一路过来没有看过一个人俑。这里如此隐秘,是西王母的圣地,这个人影在这里肯定非同小可,要千万小心。”

胖子道:“可惜潘子的枪毁了,否则这个距离,老子一枪打他的脑袋,是人是鬼一下就试出来了。”

我心说是鬼你也打不死,是人你就成杀人犯了。

不过无论如何,我们必须过去,因为那个地方是唯一可以接近陨石的地方。我们召集人过来,一边朝石阶漟水而去。

这里肯定不会有机关,因为根本就没有修建机关的条件,石阶都是非常简陋的砸出来的,两边本来可能是用来照明的青铜灯座现在完全绣成了摆设,胖子想装一个进背包里,结果一碰就碎。慢慢的石阶梯脱离出水,觉得身子重的灌了铅一样。休整了片刻,我们才揣着黑驴蹄子,小心翼翼地毛腰走上神台。人多胆子大,几乎没什么犹豫,矿灯光攒动往那人影照去,果然就看到王座上坐着一个人。

走近看,发现那是一具端坐在王座上的女尸。

这具女尸戴着非常繁琐的头冠,如果不是发簪,已经无法分辩出男女,身上穿着金丝裙袍,缀满了玉片。整具女尸端坐如定,栩栩如生。

女尸的脸发青,仔细一看,才发现那是尸脸上覆盖了一层类似于石灰的青色胶质,然后仔细雕塑出来的效果。女尸浑身上下没有露出一丝皮肉来,也不知道衣服中的尸体保存的如何。这么看上去,好像庙里得泥塑菩萨,在矿灯光下显得无比的阴森。

在女尸的身后还站着两具守卫,穿着西域的盔甲。这两具尸体显然没有女尸保护的那么好,能看到脸上的石灰已经脱完,露出了里面糜烂殆尽的骨骸。因为盔甲是黑色的,好似玉俑同样的材料,刚才我们没有看到。

三具古尸都笔直的或立或坐,显然经过了特殊处理。

“这会不会是西王母?”胖子轻声问。

我点头:“看这架势差不离,想不到她还真的在这里,一定是古人将她的尸体处理之后安放在这里。”

胖子看见那些玉片,一下就两眼放光了,道:“总算给胖爷我看到些好东西了,原来这娘们都穿在身上呢。娘们就是娘们,临死也舍不得这点基业。”我一听立刻在他没动前就把他抓住。

闷油瓶让我们不要靠近,他指着王座四周地面雕刻的花纹,是一只大头小身的人面鸟,花纹呈现一个圆盘将王座围在中心,他用奇长的手指摸着圆盘的边缘道:“有细小的缝隙,可能也有平衡机关,不要靠近她。”

我们松了口气,这才想起抬头看头顶,只见陨石的表面几乎就在我们天灵盖上面,跳一下就能碰到。在我们头顶的部分就有几个深深的孔洞,照进去,发现那些洞口直通到陨石的内部,深不见底,而孔壁非常光滑,确实不可能是人工开凿的。

汪藏海找这东西干什么呢?如果按照文锦说的,他是来寻找长生之法的诀窍,那么这颗陨石和长生又有什么关系呢?

我仔细抬头去看,看着看着,忽然发现一个奇怪的现象。

这颗陨石的材质,怎么这么像玉俑?这种颜色,这种光泽,似乎是同一种材料。我跳起来摸了一把,发现陨石温润一点也不凉手,竟然真的好像是玉石。

乖乖,我心说,这该不是一块“陨玉”?

这个世界上有一种宝石叫做陨玉,是一种特殊的陨石,因为材质手感和玉石十分相似,所有被当成玉石,在古代极端珍贵。不过这陨石的颜色比陨玉的颜色要深上许多,会不会是一块含有特殊成分的罕见陨玉?而那些玉俑就是使用这种陨玉做的?

如果是真的,这玩意可值了钱了。这么大一块儿,就是按斤卖我们也发大财了。

我把我的想法一说,众人都感觉很有道理。

“看来,那些血尸的形成,和这块陨石有着相当深的关系。”文锦道,“而古代的西王母发现这种力量,就用陨石来制作那些玉俑。”

我一下发散开去,就想到一件事情:“你们说,从汉开始流行的金缕衣,传说可以防止尸体千年不腐烂,然而现在考古发现的金缕衣往往连玉石都烂了,显然这种传说是不科学的。那么这种传说是从哪来的呢?最开始,会不会是因为那些方士查阅了某些古籍,查到了对于金缕衣千年护尸的描写,却不知道这个玉和普通的玉是不同的。”

“难道是战国锦书!”胖子道,“你是说,汉代的金缕衣是模仿战国锦书上写的玉俑来制作的?”

“有这个可能。”我就点头道,“然后,汪藏海就发现了这个破绽,所以他开始来寻找古籍上制作这种玉俑的真实材料。”

一下我就觉得脑子里的事情变清晰了。“他娘的,如果真是这样,那么汪藏海这么多的盗墓活动,都是在寻找这块陨玉?最后他终于发现了陨玉的所在地,于是带人来这里?”

“不对。”文锦并没有我那么兴奋,“按照你这么说,他既然到了这里,应该已经得手了,可是我们在海底墓里没有看到玉俑,玉俑应该不是汪藏海的目标。”

“那他的目标是什么?”我道,我觉得我的想法十分的合理。

文锦看着那些陨石上的孔洞,对我们道:“不知道,不过我有一种感觉,这个目标,就在这些洞的里面。”

文锦说的语气很玄,我们都给她说得愣了一下,心里有点发毛。抬头看那些洞,心说里面会是什么呢?

看了一会儿,她忽然开始抽出背包里的绳子,对我道:“我要进去看看。”

我一听这怎么行,想阻止,却被闷油瓶拦住了,我和他对视了一下就意识到他是什么意思:我们有选择,但是文锦别无选择,说什么都没有意义。

我长叹一声,有一种无力感,人只有在无法帮助自己想帮助别人的时候才会觉得自己渺小,我总以为这种无奈只有在电视上才有,没想到现实中也会给我碰上,感觉真的不好受。

文锦动作很利索,立即便开始准备,让闷油瓶去帮她连接绳子,自己用矿灯照那些洞口,准备选择一个进去。

我本想找个人替她,发现也不大可能,虽然这一个个洞都有柏油桶那么大,但是孔洞几乎是垂直,进去必须使用膝盖或者脚掌蹬着孔壁往上。我们几个男人都太高了,进去之后无法完全弯曲,几乎都不能用力,胖子就更不用说了,如果里面孔洞直径变小他都可能被卡住。只有文锦身材娇小,可以勉强用上力气。

我有些担心,但是看到文锦身手矫捷的样子,也知道这种担心是无意义的。一边的文锦在腰上系上绳套,被胖子托到了肩膀上,她探身进入孔洞之内,然后用力一蹬胖子,人就进去了。

我叫道“小心点”,她应了一声,低头看了我一眼。我发现她的脸色有些奇怪,有一种说不出的感觉,随即她对我笑了一下,就开始往深处爬去。

我们一边放绳子,一边看着她逐渐往上深入孔洞,动作十分缓慢,显然十分吃力,直到看着她的矿灯光消失,整整过了半个小时,估计进入的距离还不到五十米。

看着整个过程,我觉得毛骨悚然,这就是爬盗洞的感觉,但是这孔洞到底有多深,到达最深处起码也有两三百米的距离,这种好像爬进别人食道的滋味绝对不好受,更何况爬到中途的时候,会出现前后够不着的情况。

又抬头看了洞口十几分钟,脖子就吃不消了,我不忍再看,就和三叔那几个伙计一样坐下来休息,脱掉衣服用烧酒抹身驱寒。绳子一直在往里面放,隔十几米,胖子就和里面的文锦确认一下,打几个信号。

气氛很凝固,我们都不说话,也不知道说什么好,一方面身后的女尸让人毛骨悚然,一方面又担心文锦的安危。

等了大概一小时,忽然就听道胖子“嗯”了一声,我立即站起来问怎么回事,他道:“大姐头没回应了。”

我们凑过去,看道胖子拉扯着绳子,拉了几下,绳子被扯下来一些,没有人把绳子拉回去。

我脑子一紧,心说是不是出事了,示意胖子再试一下。

胖子又拉一下,绳子还是被拉了下来,他的眉头就皱了起来,“不好,绳子很轻,好像那头没系着人。”

闷油瓶一听,脸色一变,立即对胖子道:“把她拉出来!”

胖子马上用力,飞快地拉动绳子。我看着他拉的力气就发现不对,完全不需要用力了,绳子犹如流水一样被他拉了出来,一直拉到垂直,绳子就结成一团整个从孔洞摔了出来,全部打在我身上,把我缠绕进了里面。

我挣脱绳子那起末端一看,发现没有割裂的痕迹,绳子是被她自己解开的。我们面面相觑,我心里忽然就有了一种十分糟糕的感觉,“他娘的,文锦自己解开的绳子?”

闷油瓶脸色凝重一下按住胖子的肩膀,整个人借力踩着胖子的背,接着一跳,就钻进那个洞里,动作之快,根本拦不住。胖子大叫:“绳子!带上绳子!”他也不理会,几下就往上缩了进去。

我一看他不带绳子不行啊,立即对胖子叫道:“蹲一下。”胖子大怒:“他娘的都当老子是马夫啊。”我不去管他,贴着他的身子就歪歪扭扭地爬了上去,他托了我一把,我用力一蹬腿也窜了上去,无奈力气不够,屏住呼吸撑住孔壁想把脚也提上来,结果没几秒就滑了下来,直接摔在胖子身上。再来一次,还是那样,一下明白自己的体质肯定是进不去了。

我站起来揉了揉摔痛的地方,抬头就看到闷油瓶艰难地从洞里面前进。他太高了,膝盖无法着力,只能用小步上,十分消耗体力。我突然产生了一种错觉,这陨石会不会是活的,这些孔洞就是它进食的陷阱,闷油瓶在自投罗网。

但是随即我就意识到这不可能,再想脑子已经一片混乱,无法思考了。我就这么抬头看着闷油瓶爬上去,也不知道过了多久,闷油瓶也完全消失在孔洞的深处。

我再也坐不住了,一直坚持站在洞口往下看,希望能看到有灯光返回,然后他们两个都安全地回来。

时间一分一秒的过去,我心急如焚地等着,从焦虑到冷静,从冷静到麻木,从麻木到脑子一片空白。

十个小时之后,还是什么都没有发生,闷油瓶也没有回来,文锦也没有回来,空洞里没有一点声音。这两个人,好象被这些孔洞带去了另外一个世界。

\chapter{继续等待}

我们在这里什么都没有做,足足等了三天时间。这三天里,我唯一注意的地方,就是那个他们消失的洞口,这是一种多么漫长而又焦虑的过程,我想只有设身处地的人,才能体会。

期间,我曾经不止一次地想进入那个洞口,但都以失败告终。这实在不是普通人力可以攀爬的通道,我最高的一次只爬上去十米,已经完全力尽,小腿抖得如筛糠。

这批人中,三叔的那批伙计必然不敢深入,唯一有可能进去的是黑瞎子,但是他始终没有表现出那个意思,我想他大概是觉得进去也没有把握能出来。营地里气氛沉闷,那个拖把好几次都催着离开,说这两个人可能已经死在里面了,既然我们不可能进去,那么还是省点力气和干粮为出去做准备。

我无法接受,千辛万苦来到这里会是这个结果,我蒙头几乎听不进去这些话,脑子里只想着这里面到底发生了什么事情。

文锦解开了绳子,她是故意的,我想起了她临走前的笑容,我感觉她可能早就计划好了,这么说她知道在里面会遇到什么情况,知道会有这种不出来的情况发生。

文锦一路过来,话都说的很宿命,她这几年来的生活简直无法形容,她有这种想法是有可能的,也许她在里面发现了并没有解决她尸化的办法,所以万念俱灰,选择了结束自己的生命。但是闷油瓶呢,他为什么不出来,这就说不通了,我能肯定这里面一定发生了一些什么。

会是什么呢?简直没有任何的方向去想,他们是否迷路了?我想这里面的孔道蜿蜒曲折,形成了无尽的迷宫,进去之后就无法出来,但是这又无法解释文锦为什么要解开绳子。

我脑子里面是无比焦虑的念头,休息的时候眼前就看到一只深洞,闭上眼睛也是深洞。

之后的情形我实在不愿记述下来。

第四天开始,拖把这批人就开始不停的发牢骚,我心情非常糟糕,几次要和他们打起来,但是那个洞里还是没有任何的动静,一度我甚至怀疑,是否文锦和闷油瓶压根就没有存在过,这一切都是我们的臆想。

不安和焦虑越来越重,我的心里开始承认拖把他们说的可能是正确的,但是我的理智又让我必须和他们争吵。这让我几乎崩溃。

到了第六天,拖把终于带着人走了,在他们看来,这事情已经没有任何疑问了,闷油瓶和文锦就算没死,再过几天也死定了。本来他们希望依靠我们的经验带他们出去,但是现在这种情况他们显然不肯虚耗下去。黑瞎子拍了拍我,意思是让我也走,但是我拒绝了,他叹着气跟着离开,只剩下我和胖子两个人。

他们带走的还有大量的食物,我知道肯定超过平均的分量,但是我实在懒得和他们吵了。

胖子其实也劝过我,但是他知道我的脾气,我经历了这一切,到了这里,就算没有一个完美的句号,也应该有一个残缺的休止符了,但是这样戛然而止,我忽然发现自己蠢得要命,我来这里到底是干什么?难道就是这样,一切都结束了?我绝对无法接受。

胖子没有办法只好陪我,我们俩个人就这么互相看着,等着,我忽然想起一出荒诞剧叫“等待戈多”,不由就想哭,心说我的荒诞剧竟然还是悲剧。

这样的日子一共持续了几天,我也记不清了,不过不会太久,因为我们的干粮并不多,但是当时没有吃完。

拖把他们离开之后,我心里其实已经几乎绝望了,甚至说只差一点我就会崩溃了,我已经完全无法去思考我在这干什么,每天能做的事情就是去看那个洞口。按照胖子的说法,就是一个疯子的行径。

那一天,我睡完浑浑噩噩的起来,胖子要守夜,但是也睡着了,在那里打呼噜。这几天倒是睡舒坦了,身上的伤口都愈合了。

我没有任何的动力去叫醒他。我走到那个空洞下方,不知道多少次往上望去,还是什么都没有,我几乎是呆滞的看了十几分钟,然后就去吃早饭。我和胖子干粮已经所剩无几了,翻出来,找出昨天吃剩下的半截饼干接着吃。吃着吃着,我忽然听到一种奇怪的声音,好像是唱歌,又像是在梦呓。

我以为是胖子在说梦话,压根没在意,几口将饼干吃完,想去叫醒他。就在这个时候,我忽然一个激灵,我看到,在我和胖子之间,竟然躺着一个人。

我一下从恍惚的状态中挣脱了出来,仔细一看,发现那竟然是闷油瓶。

他明显瘦了一圈儿,缩在那里披着毯子,没有任何的动作。

他是什么时候回来的?在我们睡觉的时候?

一开始我以为我在做梦,随即就发现不是,我几乎疯癫了,立即冲过去,拉住他的毯子,大叫道:“你个混蛋,你他娘的上哪儿去了?”

他被我拉了起来,我就想去掐他,可一下我看到他的脸,突然发现不对劲。他的表情很怪,和他平时的样子完全不同,而且目光呆滞,浑身发抖,嘴唇在不停地颤动,好像中了邪一样。

我心中咯噔了一声,立即将胖子踹醒,然后把闷油瓶扶起来,按住他的脖子叫他的名字。可是他没有任何的反应,似乎根本听不到我们的声音,甚至连眼珠都不会转动。

我心中涌起了极度不祥的念头,胖子过来看了看我,问我怎么回事,我说我怎么知道。他按住闷油瓶的太阳穴看了看他的表情,咋舌道:“我操,不会吧,难道小哥傻了?”

“不可能,你他娘的别胡说。”我道,叫了几声:“别装,我知道你在装,你骗不了我!”就听见他一边发抖,一边无神地缩在那里,嘴巴里不时地念叨着什么。

我贴近他的嘴唇去听,就听到他在不停的急促地念着一句话:“没有时间了。”

\chapter{离开}

闷油瓶躺在那里,胖子给他打了一针镇静剂,之后他便睡着了。

我看着他的样子,心中觉得非常的堵,难受的要命。

他一定是在我们睡觉的时候,从那个洞里出来的,可是他怎么会变成这样?

我看着头顶的陨石,青黑的表面丑陋如常,没有任何的变化,无数的孔洞好比眼睛,看得我一阵窒息。

狗日的,这到底是怎么回事?

我郁闷的要死,心说这简直是在耍我。

没有时间了。又是什么意思呢?听上去像是有一件事情马上就要发生了,而且什么措施都已经没有时间去做了,难道这里会发生什么事?

四周安静的犹如宇宙,没有矿灯去照射,看不到任何的东西,这里如果正在发生什么变化,我们也无法得知。

他肯定受了极大地刺激,胖子叹气道:“对于外界的一切都没有反应,听也听不见,看也看不见,他的感觉全部给关闭了,和我的一个朋友一样,医生说,这就像他脑子就停在最后经历的那一刹那,卡住了。”

我沉默不语,闷油瓶是一个怎么样的人我不了解,但是在他的心理承受能力方面我还是可以打保票的,这种人的心理素质已经到达了一种境界,要想让他受到极大地刺激是非常困难的。这陨石之内发生的事情,肯定恐怖的超出了我们能理解的范围。

可是,我实在无法想象,像他这么冷静的人,会被什么东西给吓的崩溃。我能肯定一定不是什么怪物,尸体的恐惧连我都可以克服,就算里面有再可怕的怪物,也不能将他吓成这样。他见到的,一定是极端诡异的情况。这时候又想到文锦,她现在在哪里?难道她也疯了,出不来了?

如果是这样,那我必须进去,我就算摔一千次也要爬进去把她带出来,绝对不能把她留在陨石里。

想着我有点起鸡皮疙瘩,我又站起来,走到洞口,打起手电就往上照,这几乎已经是一种习惯性的动作,这几天都不知道做了多少次了,我随意的往洞里闪了一下,接着就走了回来。

才走了几步,我忽然一愣,发现不对,这一次,洞里不是黑的,那洞里有个东西!

一下我头皮就麻了,立即回去一照,果然就发现在洞穴的深处,出现了什么东西!

我心里叫了起来,立即叫胖子过来,自己打开强光往上一照,一下就看到大概孔洞二三十米的深处,有一张苍白的脸,正在往外窥探。

我一喜,以为是文锦,可再一看,我一下浑身就凉了。这张白脸面无表情,眼睛深凹进眼窝中,脸色冷若冰霜,表情极度的阴森,让我毛骨悚然的是,那竟然是一张我从来没见过的面孔。

这人是谁?我的冷汗瞬间湿透背脊。

胖子看我脸色不对,过来一看,也僵住了,立即就去端枪,我一把拉住他,矿灯光一晃,再一看,那脸就消失了,尽头还是一片漆黑。

我和胖子面面相觑,两个人的冷汗都像下雨一样,隔了良久我才问道:“你刚才也看到了吧?”

他点头,我发现他脸色都吓青了,似乎被吓得够呛。

这事情已经超过我的理解范围了,这陨石中竟然会有一个陌生人,这怎么可能,难道这里面住着人,原来西王母的先民还有活在里面的?

这太离谱了,我又想到文锦,心里哎呀了一声,难道文锦开始尸化了,刚才那张就是她变异中的面孔?

我看向胖子,想问他刚才有没有看出一点和文锦相似的地方,却看到胖子还是脸色发青,只盯着那洞里看,还没有缓过来。

胖子不是如此胆小之人,我心生异样,问他怎么了,他转头问我道:“你没认出来?”

“认出来?”我愣了一下:“你认识这个人?”

胖子指了指我们身后,我转头一看,就看到那具坐在王座上的女尸。胖子把矿灯照向那具女尸的脸,光线一闪,因为阴影效果,那女尸的面孔突然一阵狰狞。

我看的分明,一下就明白了,顿时觉得寒气透心而过,几乎没晕过去。

我的天,刚才我们看到得脸,竟然和这具女尸外面雕刻的样子有些相似!

这是怎么回事,我们刚才看到的脸——是西王母?

这具尸体难道真是具尸壳子?真正的西王母,还活在这颗巨大的石头中心?

不可能,这怎么可能呢?几千年的人怎么可能还活着?就算没老死,在这里也饿死了。

是幻觉?我忽然怀疑自己的感官,精神太过疲惫:我们被这颗陨石搞的神经错乱了,也许刚才那脸就是文锦,只不过因为光线的问题,看起来像这女尸。

胖子顿了顿:“那她为什么不出来?”

我哑然,胖子道:“很少有两个人会一起看错。”

这一下两个人如坐针毡,这地方待不下去了,胖子对我道:“小吴,这地方越来越邪门了,你打算什么时候走?”

“怎么了?文锦还没出来呢。”我看他的脸色问道:“你吓成这样,不像你啊。”

“这是一方面,最重要的是,没吃的了,本来我今天也想和你说,如果你明天不走,我就是打晕了也必须带你走,再等下去,我们就会饿死在这里。我们吃的东西已经剩的不多了。”

我道:“不是还能撑几天吗?”

胖子道:“我算过,剩下的东西,我们省着吃能吃两天,勉强够我们一路顺利的找到口子出去,但是现在多了一个小哥,我们就没有别的办法了,就算能安全到达地面上,我们也必须挨饿穿过雨林。现在水已经下的差不多了,沼泽肯定已经露了出来,穿过去一定是极其艰苦的过程。你如果再坚持等下去,明天我们就要开始挨饿,饿上两天你就不会有力气再出去,我们就等于死在了这里。”

我看了看那个孔洞,摇头道:“不行,我们不能丢下她不管。”

胖子拍了拍我道:“我知道你这个人心软,我早就想好了,我们把能吃的东西都留下来。挨饿出去,到了外面,如果能回到那个营地我们还有补充,实话告诉你,在每一个休息的地方,我临走都埋了一包压缩饼干。只要走对路,我们还是能出去。我看大姐头出来够呛,与其等她出来看到我们饿晕了,不如这个办法好,而且这陨石里面这么邪门,我看……”

我知道胖子想说什么,摆了摆手,发现胖子虽然慢条斯理的这么说,但是他说出来的话斩钉截铁,几乎没有任何可以反驳的地方。可以想象,他一直忍着没有说出来。

“而且,就算你愿意死,小哥不一定愿意,你至少得救一个。”

我看了看闷油瓶,立即妥协了。是啊,我一直想着一个人都不能少,最后可能连闷油瓶都被我害死,而且胖子的方法确实有道理。心说这也许是唯一可以让我们都活下来的办法。看着那孔洞我叹了口气,接着就问他道:“可是现在我们应该怎么回去?”

胖子道:“我们原路走回去,然后顺着河壁走,必然能找到另外的出水口,可以重新回到蓄水工程里去,那么肯定能发现出口。”

“如果没有呢?”

“现在管不了这么多了。”胖子见我答应了,喜出望外,说着就立即开始收拾:“只能听天由命了,不过应该有,否则黑瞎子早回来了。”

胖子动作很快,一个小时候,我们收拾起了装备,留下了我们所有的干粮,写了字条,然后他就催着我开始原路返回。

我还是有点无法割舍,看了几眼,又对着那洞口喊了几声,然后转头离开。

闷油瓶神情恍惚,我们搀扶着他,很快回到来时的那个全是陶片的地方,这时候我就在想黑瞎子他们是往哪个方向走的。忽然胖子停了下来,把矿灯照向水里,我发现在这片堆满了陶片的地方,出现了一个原来没有的深坑。

就和之前我们看到的深坑一样,但是我们可以确定,这个坑我们来的时候是没有的,好像被什么东西拱出来的。

我觉得有些不妙,催促胖子快走,胖子此时却不走了。我问他干嘛?他道:“你没有看到,这坑壁上刚才有什么东西闪了一下光?”

\chapter{陷坑}

“是什么?”我问道。

“不知道,就在坑边上。”胖子看了看我,忽然对我道:“贼不走空,可能有好东西,我得下去看看,你等我几分钟。”

我气得要命,但是现在就我一个人,他不听我的,让我扶着闷油瓶,自己下水翻找。我没有办法,只能让他快点。

不过这并不容易,瓦片大部分埋在碎片的下面,在陶片中翻找,可不像在海里,沙还比较松软,这里的陶片一方面锋利,一方面是在坑口,一动陶片就往坑里滑下去,人也不好保持平衡。表面的还好,挖出几片,再往深挖就非常困难,有时候看到一块陶片想翻开来就是拿不上来,好像长在里面一样。

挖了几下,胖子似乎是发现了目标,浮上水面换气后又潜了下去,用力把手插入挖出的陶片坑里,往外掰,没掰两下,忽然胖子一个哆嗦,猛缩手回来,手上鲜血直流。

“糟了!”我暗叫不好,心说该不是被鳖王咬了。却见胖子并没有中毒的迹象,只是伤口似乎颇深。他用嘴巴吸了一口气,换手又用力一掰,把那根骨头拔了出来,接着就浮上来了。

“怎么回事?”我在一边问道。

“我靠,这骨头里好像有刺,疼死我了。”胖子一边吸着手指,一边甩干捞上来的头骨,招呼我把矿灯照过来。

我嘀咕道:“你看,你自己作孽吧。”走过去给他照明,刚走到他边上,忽然就听到我的身下,传来一连串沉闷的“咕噜”声,接着冒上来一连串的水泡。

胖子和我都愣了一下,那汽泡停了一下,又“咕噜咕噜”冒上来一连串。

“他娘的,真是人不服不行,你这屁放的赶上火箭炮了,还是连发,这动静也太大了。”胖子捂住鼻子道。

我也莫名其妙,看了看四周:“他娘的,我没放屁。”

“你没放屁怎么这么臭?这都是什么味啊,大便都被你熏死了。”胖子皱眉道。

四周的确有了臭味,我闻着却心里一惊,这确实不是屁的味道,虽然一时之间我想不起这是什么味道,但是我潜意识里感觉不妙,似乎是要出事。刚想说快走,突然我一下失去了平衡,水花一炸,好像踩空了一样,整个人猛沉进水里。

那一下极为突然,几乎是在一瞬间我脚下就空了,我的第一反应是我滑倒了,立即就蹬腿想重新站稳,但是紧接着整个水下都起了汽泡,我脚下的陶片动了起来,往一个地方直滑,根本站不稳。

我大惊失色,立即意识到了什么,赶紧缩起腿一个翻身往水下潜入,胖子也潜了下来,我们扎入水里。

扫过矿灯一看,就看到我脚下的水底塌方了,水底塌出一个大坑,和边上的那个坑连在一起,成为一个非常大的深洞,四周的陶片头骨全部往坑底滑去。回头一看,只见闷油瓶顺这坍塌被扯进坑底,脚被裹紧在陶片里拔不出来,好像有什么东西抓着他的脚往下拽,想要把他拖进坑的底部。

刚才没顾到闷油瓶,事实上一直以来都是他在照顾我们,我们还不习惯照顾他,看他的腿陷在碎片中,已经裹到了大腿,显然是刚才坍塌的一霎那被裹进去的。他没有作任何的反抗,呆呆地任由自己顺着瓦片沉下去。

眼看着要被裹到坑里面去了,我和胖子赶紧过去帮忙,一人扯住他的一只手就往上拽。胖子单手用不上力气,咬住矿灯用双手,两个人用力蹬水,把他拔了出来。

这种事情如果他是一个人就死定了,如果有两三个人就不算什么大事故。闷油瓶被提起,开始咳嗽。

胖子就道:“我说你的屁厉害吧,把水底都崩穿了,以后放屁之前记得打招呼,免得误伤别人。”

我大喘气大骂道:“这时候还挤兑我,等会老子和你拼了。”

“你看你这人,一点也不虚心接受教导!”胖子拿矿灯去照水底,下面坍塌慢慢扩大,但有些停止了,很快一个大概有半个篮球场一样大的洞出现在我们面前,黑黝黝的,好比一张大嘴,要将我们吞噬下去。不时有些汽泡从下面冒上来,四周弥漫着一股恶臭。

我记起这是沼气的臭味,这个洞肯定本来就存在了,也许之前有木梁之类的地东西加在上面,腐朽之后,还是维持着脆弱的平衡,没有外力的时候,这种平衡可以延续千年,可一旦有任何的破坏,木梁就崩坏了。那个塌出的坑可能是木梁断裂造成的,胖子又在边缘挖瓦片,结果引起了连锁反应。

“我靠,”胖子道,“这下面好像都是空的?”

下面应该不深,但是水刚才一搅动污浊了起来,看不到底,我道,“这下面可能是之前搭的一个防止鬼头罐的夹层。”看他又往边缘走,就道,“小心点,刚才我踩还结实,忽然就塌了,他娘的可能这块地方下面全是空的,现在踩塌了一块,等下别再来个连锁反应,形成漩涡我们全完蛋。”

“只要你不放屁就没事了。”胖子道,“咦,这是什么?”

我顺着他的手电看去,只见那深坑中竟然有东西浮了上来。

“远点。”胖子提醒了一声,我拉着闷油瓶条件反射地退开了一点距离,胖子就把矿灯聚焦在那东西上。

那些东西上来得很快,很快就浮出了洞口,这时候我们已经看得很清楚,都是一些腐木和树枝,中间还夹着很多没法分辩的棉絮一样的垃圾,这些应该都是被压在下面瘀泥内的沉淀物,被落下去的陶片激起,跟着起来的还有大量污浊的水。一时间,洞口附近的能见度越来越差。

胖子捞起了几个,都是缠绕着垃圾的树枝,弄了他一手的臭泥,他远远地抛开,道:“他娘的,这泥泡子的老泥底子都被我翻出来了,臭死我了,我靠!这该不是以前的粪坑吧?”

我道:“你家才用那么大的粪坑,在这拉屎,脚滑一下就可能直接没命,要是你拉得出来么?”

胖子太会扯了,这要是粪坑那拉屎比蹦极还紧张,我看大象都不敢用,西王母国的先民总不会这么折磨自己吧?

“也许这是因为女王想培养他们的子民居安思危的理念,让他们在拉屎的时候保持十分的警觉。”胖子一本正经道。

我催促说:“快走,这里太危险了!”我们捂住鼻子正想离开,胖子又从水里捞起来一个东西,这个却不是树枝,他“咦”了一声,就举起来:“他娘的,你看这是什么?”

\chapter{水壶}

我朝他看去,就觉得那东西像小一号的人头,但是没有五官,上面沾满了黑泥,四周全是细碎的胡须一样的东西。

“什么鬼东西?”我问。

胖子扔了过来,我凌空接住,发现那东西不大,用水洗了一下,很快外面的黑泥被洗掉,露出里面绿色带锈迹的表面。

我甩了甩,奇怪道:“他娘的,是个军用水壶。”

“水壶?”

“老款式,几十年前的东西,我一看外型就知道了,我家里还有一个。看,这里还有字。”我把水壶翻了过来。

水壶的底上确实有钢印打的一串字,本来就打的不深,现在更看不清楚,可能是生产的地点。

我们面面相觑,都心说怎么回事?这个水壶怎么会从这个洞里漂上来?水底下的空间,应该是碎石和陶片堆积成的河底,虽然不知道几千年前是什么样子,但是近几百年肯定就是这个样子,怎么会有水壶存在?

胖子道:“会不会也是那批逃进这里的反动份子的东西?”

“有可能。”我道,“不过问题不是这个,是这东西怎么会在这下面?”

“也许有个反动份子也到这里来过,碰巧摔死在洞里。”

我摇头道:“不可能,这种平衡结构只能存在一次,如果之前坍塌过,要么会是个洞,要么被后来的泥沙填平,不会再出现后来被陶片覆盖起来的陷坑。”

胖子道:“你怎么知道?”

“老大,这是常识。”我道。

“那也有可能是从其它地方漂到这下面来的,这地方的下面全是空的。”胖子道。

“理论上有可能,但是实际上很难,水壶会浮起来,卡在空洞穹顶上,不是那么容易漂动的。”

话还没说完,忽然感觉脚下动了一下。我立即张开双手保持平衡,对胖子道:“当心当心,又要塌了。”

胖子却骂了一声娘:“你的常识错了。”

我低头看去,只见一团巨大的东西从黑坑里迅速浮上来,反射出一连串鳞片闪烁的光芒,接着出现一只篮球大小的黄色眼睛。

我一下就呆住了,这是什么?

胖子推着我大叫道:“跑跑跑!跑!跑!跑!”我还没反应过来,已被他拉着冲了出去。胖子像疯了一样,扯着我一点都没留力气,我看这一架势真的在逃命,也拉住闷油瓶,奔命而出。在水里其实根本没法跑,阻力太大,非常缓慢,而且脚下都是锐利的瓷片。我只冲出去几步就踩到锋口上,一下摔进了水里,扑腾起来,脚底心阵阵剧痛。

接着胖子和闷油瓶也倒了,胖子背包坏掉了,但是爬起来根本不看,大叫一声别停!就用尽全身力气,跌跌撞撞地继续往前冲去。我就听到我们身后传来了滔天的水声。回头一看,那竟然是一条无比巨大的蟒蛇,从水中腾雾而出,简直犹如青龙出水。

我看那蛇的体型,一下就想了起来。

天!这……不是那条蛇母吗?

这怎么可能?浮雕上的巨蛇居然真的存在,而且到现在还活着!

我心说完了,咬牙继续往前跑,就听着后面简直是惊涛骇浪一样的水声跟来。这可怎么办?只能跑几步是几步了。我几乎是一边跑一边摔,也不知道摔了多少次,脚都崴了,浑身是伤口。

很多人都有经验,遇到危险逃跑的时候,人只凭着最开始那一股劲,在这劲头没用完之前,就算身上给人劈了两刀也感觉不到疼。所以我一路狂奔,摔了爬,爬了摔,脚底都烂了,也不知道划了多少口子。慌乱中根本没有距离感,也不知道跑出去多远,最后猛然脚下一空,踩到一个突然的低洼,一下就滚了下去。下面就是那种深坑,整个人顿时被冲进水里。

我也算反应快,马上稳住身型,但是太突兀了,还是喝了好几口水,怎么踩也踩不上去。

胖子还算注意我,跑出去十几米了,还是冲了回来想把我扯上来,可没等我抓住他的手呢,忽然鳞光一闪,一股无比霸道的力量就带着水流压了过来,一下把他和闷油瓶也压下水来。

这就要命了,三个人扑腾起来,犹如火车一样巨大的蟒身则在水里绕着我们盘起来。胖子拔出了匕首,但是看了看体积差别,那匕首比牙签还不如,不由作罢。

巨大的蟒头探进水里,出现在我们面前,鳞片犹如镜子,太大了,那种气势,我简直像看到一条无爪的青龙。

那水壶是怎么下去的?肯定是有人给他吃了,被他带到了沙土下面。三个人让他当开胃小菜都不够。

我们在水里扑腾,想游出蟒身的包围圈,却发现根本无法控制自己。巨蛇只要一动,水就会奔腾,带着极大的水压把方向打乱。

胖子不认命,一边端起了枪,瞄准了那蛇的眼睛,连开两枪。巨蛇的脑袋动了两下,一点反应也没有,胖子只得把枪扔了。

我让他别白费力气,我们都知道那枪根本不会起任何作用。遇到那种双鳞巨蟒还能拼命,可这玩意儿实在太大了。怎么打啊?任何效果都没有。

胖子就叫道:“不会!大象不吃蚂蚁,我们太小了,他要吃我们也没这么容易。”还没说完,蛇头忽然一缩,猛地朝他咬过来,那种声势根本无法形容,我一下就被冲起的水浪甩了出去。

我爬起来,大叫胖子,却见他拖着闷油瓶也被冲的老远,巨蛇居然没有咬中。巨蛇一击不中,恼羞成怒,蛇身扭动开来,形成巨大的水浪,硕大的鳞片好比无数面镜子,将我手里的矿灯反射出一片瑰影幻境。

胖子朝我大叫:“躲起来!”

我立刻朝一边的石柱后面游,好不容易爬上去,一回头,头皮一麻,竟然看到了犹如恐龙一样的蟒蛇头巧声无息的探到了面前,正直勾勾地盯着我。没法躲,蟒蛇太大了,我游的半死的距离,他一下就探了过来,恐怕两三米内的都是他的直接攻击范围。近距离照着,我发现这蟒蛇更加巨大,不由得腿一软,跪了下来。巨蟒则转动头部,用巨大的蛇眼看着我,没有立即发动进攻,蛇头不时的转动。

我心说死定了,在水面上,他的攻击肯定比在水里准。但是等了几秒,仍不见那蛇来攻击我。我死盯着蛇头看,发现那蛇似乎吃不准什么。

我想了想,忽然看到正照着蛇的矿灯,一下就明白了。矿灯极亮,这条蛇在这里了,可能几百年没见过任何光了。现在给这东西迷了眼睛。

心中想到一个办法,我慢慢的将矿灯放到一边的石柱上,想趁他的注意力被吸引住的工夫溜掉,然而石柱上几乎无法放任何东西,一放就滑下来,我浑身直冒冷汗,放了几次都不行。我一边让自己一定要镇定,一边想办法。真佩服自己这个时候脑子还能转动。要是以前,一定完全吓死了。

突然看到一边的胖子在巨蟒的脑袋后面给我打手势,好像要我把矿灯甩给他。我顿时明白了他的意图,深吸一口气,用力一甩,就将矿灯从那蛇头边上甩了过去。一道弧光飞向胖子,巨蟒被光吸引,马上转过头去。就在这一刻,我猛地潜入到水里。

一边的蟒蛇立即动了,我不管三七二十一,拼命游了开去,直到筋疲力尽才探出头来。发现自己还是没有游出太远,巨蟒就在我的身后,四周横陈着巨大的蛇身。但是他迅速运动,很快竟然消失了,似乎钻入了沙子底下。

不久,看到胖子背着闷油瓶从那边飞快地破水而出。我问他怎么回事?他道:“我把矿灯沉到一个洞里,它追了下去。快走,等它再上来,我们就死定了。”

\chapter{尾声}

之后的经历泛善可陈。

我们喘着气,互相看着,感觉刚才一切都好像在做梦。胖子脸色惨白就让我们快走,一刻也不敢停下来。之后的过程我基本上是非常恍惚的,特别是到了最后,我只能大概的记叙一下经过。

我们几乎没有任何的停留,一路回到了出来的水道口,选了一个方向就顺着石壁开始寻找另外的出口。

在六小时后进入一个水道口,忍着饥饿,三个人干脆闷头走,什么也不说,免得消耗体力。

“不吃东西靠脂肪能支持一到两周,难受的只有前几天,”胖子说,“我经历过这种时候,忍忍就好了。”

我一开始还怀疑我们能否活着出去,同时我也忽然明白了,三叔这一次进来,为什么要称为“不归路”,因为路程实在太长了,一个人背负的食物完全无法满足整个来回,他已经预见到了回程的艰苦卓绝。

在渠道中空腹行军,胖子的计划是一天内走出去,但是往上走比往下走要累得多。饿了两天后,我们实在无法忍受了,开始琢磨办法。这里能吃的东西非常有限,有干枯的树梁,以及很多缝隙里的虫子,探险手册上说,在野外没有食物又摸不准什么能吃的时候,吃虫子是最保险的。我们开始尝试着抓一些来吃,不过这里的虫子也非常的少,并且都很细小,当瓜子还差不多。

闷油瓶一直恍恍忽忽的,后来好了一些,但还是什么都记不起来。我们和他说了好几遍事情的经过他都无法理解,好在不用再搀扶他,他可以自己跟我们走。

靠着虫子又撑了三天,我们终于看到了活的树根出现在井道壁上,胖子判断这里应该是离地面很近了,我们在四处徘徊,终于找到了几个向上的竖井口。胖子爬了上去,发现这是我们当时进入雨林时路过的那片塔林。

这里的孔洞很小,我们没法钻进去,于是胖子用子弹砸出一个小孔,做了一个定向爆破,把几个孔之间的石头炸裂,我们才勉强挤进去。地面上已经面目全非,所有沼泽的水位全部都降到了最低点,露出了瘀泥和狰狞的树根系,此时烈阳高照,所有的毒蛇都在地下,应该是最安全的时候。

雨林里阳光明媚,鸟语花香的景色,很容易让人产生美仑美奂的错觉,以为这里是人间仙境,但是我们深知这片刻安宁绝对是一种假象。越是安宁,越是不能休息。

我们算了一下时间,在天黑前绝对出不了峡谷,最多能进入到峡谷的中端,如果遇到任何的阻击,我们三个筋疲力尽的人肯定会减员。

我们三个都是经历千辛万苦活下来的,我不希望这种关头再有人牺牲,但事到如今,也没有什么更好的办法,只能尽全力了。好在峡谷中鸡冠蛇并不多,而且我们可以涂上瘀泥。这一路,可以说是完全看命了。

接下来是长途跋涉,期间的过程没有必要再赘述了,我也实在不愿意提起,在瘀泥中摸爬滚打,我们都带伤,草蜱子爬满了身上也没有时间处理,入夜之后更是紧张,一有声音就立即加快脚步。

我们用了一天一夜的时间迅速穿过了峡谷,回到了戈壁上,果然看到了在外面等候的定主卓玛他们,那完全是一种如获新生的感觉。胖子一出峡谷,就几乎昏了过去,而定主卓玛他们看到我们,几乎不敢相信自己的眼睛。

在峡谷外,我们休整了三天,所有人都浑浑噩噩,筋疲力尽。这三天我什么都没有想,什么苦恼都没,但是感觉只有睡觉是最重要的,其他的一切都是垃圾。而且我头一次真正感到了释然,似乎那些迷,还未解开的一切,都和我没有了关系。

闷油瓶仍没有起色,要么缩在帐篷中发呆,要么就是靠着岩石看天。我们都叹气,但是毫无办法,谁也没有想到,他追寻到最后,竟然是这样一种结果。

潘子却意外被扎西救了回来,躺在另一个帐篷里时而清醒时而昏迷,我没敢跟他说三叔的事情。扎西说文锦交代过他们一些事,他们知道怎么防蛇,之前信号烟出来的时候,他们也进入营地搜索,在丛林那儿发现了营地,在那里发现了潘子。

我算了一下时间,应该就是我们去抓文锦的后一天,想想只要能熬过那一天晚上,就能碰到扎西,那事情就完全不同了。可惜,那一晚变数太大了。

又休整了两天,扎西就告诉我们应该出发了,按照他的记忆,我们现在处在一个魔鬼城环的中间,魔鬼城设置了蹊跷的机关,我们必须有精确的导航,走出去之后,东西两边可能都会有公路,我们只要到了公路,就可以求救。此刻,我也想知道三叔和黑眼镜的下落,可是却已经没了力气。扎西说,他们可能从另外的入口出去了,也可能根本没有出来,但是我们已经什么都做不了了。

没有车只能步行,我们最缺的是劳力,因为当时的水是三叔大队人马搬过来的,他们出发之后剩下了好多,我们没法全部搬走,而且算一下跋涉的时间旷日持久,我们能带的水坚持不到找到公路的时候。

胖子就道,把食物减半,丢弃帐篷,多出来的空间全部用来带水,少吃点没事,没水坚持不了几天。

于是照办,背着大量的水出发,横渡戈壁,这过程初期免不了艰苦,但是和雨林行军已经属于两个档次了。四天后,我们走出了魔鬼城,又走了一个星期,终于到达了公路。拦到了一辆SUV的驴友,用军车上的电话和裘德考的人取得了联系,大概三十个小时后,阿宁公司的车队赶到,将我们救起。

所有人都瘫倒了,有些人喜极而泣,这是怎样的一次旅程,恐怕只有当事人知道。在回程的路上,胖子靠在车上,忽然唱起来歌:“攀登高峰望故乡,黄沙万里长。何处传来驼铃声,声声敲心坎。”

破锣一样的嗓音倒好听了起来,我忽然觉得一阵感慨与悲凉,一刹那,我泪如泉涌,视线模糊,过往的一切恍如梦幻般从我眼前闪过,仿佛听到了那些个永远失去的声音,在苍茫的戈壁上回荡不止。

回到格尔木后,我权衡了再三,写了一封E-MAIL给我的二叔,将事情的前前后后全部都交代了一遍。半个小时后,二叔就打电话过来了,对我说他知道了,这件事情千万不要对任何人说起,叫我也不要管了,他会处理,让我立即回杭州。

自然不能立即回去,胖子和闷油瓶还有潘子都必须在医院待一段时间。

胖子是疲劳过度,挂了几瓶营养液就缓了过来。潘子命大,我将三叔的情况和他说了一边,他捶胸顿足。我自己筋疲力尽,也无法去和他说什么,他没完全康复就回长沙,说要等三叔的消息。我让他有消息就立即通知我。

最严重的是闷油瓶,住院之后他已经恢复了意识,但是我们发现他什么都记不起来了,过渡的刺激让他的思维非常混乱,医生说要让他静养。

本来他能记起来的不多,现在连我是谁他都不认识了,这种感觉实在让人崩溃,看着他的样子,我实在是不忍心再看下去。

我是最后一个回到家的人,洗了一个热水澡,就百无聊赖地看积下来的信,突然发现其中有一封信竟然是三叔寄过来的。

我心中一动,看了看日期,发现没有邮戳,立即展开,发现这是一封长信。

\[
大侄子:
你看到这封信的时候,我也许踪迹全无,也许已经死了。
我不知道你此时是否已经知道了真相,但是我知道怎么样也欠你一个交代。
现在我即将要去做一件事,这件事是我的宿命,我无法逃避。我感觉这可能是我最后一次了,我为了这件事已经选择毁掉自己的事业,如果这一次我没有找到答案,那么我宁可选择死亡。
你想知道的事情,我写在下面,你可以慢慢看。你大概一直非常奇怪,我为什么一次又一次地骗你,你看完后就明白了,那是因为我自己本身就是一个骗局。
非常抱歉,但是不管你怎么看我,你永远都是我的大侄子。你一定要相信我,你三叔我做的一切,其实都是为了保护你,我从来没有想过害你,也没有想过对你们吴家有任何不利。
也许我其实已经是吴三省了,又或者,这个面具戴得太久,就摘不下来了。
同样抱歉,在这封信里我没有办法说明所有细节,我想说在这件事情上,所有发生的事,都有必然的原因。而我,其实只是一个事故。当时的阴错阳差导致这一切得发生,等我深陷其中的时候,已经没有办法挽回了。在西沙的事情,其实隐藏着一个更大的秘密,文锦他们的背景也远没有那么简单。我在调查他们的时候,发现他们其中几个人完全没有背景,不知道从何而来,也不知道以前他们是干什么的。
再深入调查下去你就会发现,这支考察队背后肯定隐藏着什么,所有的事情都深不可测,所以之后如果你仍旧被卷入在这件事当中,一定要你看看我的下场,就会知道追寻这个秘密,需要付出什么代价。
我更希望这件事情,到了这里就结束了。你知道真相之后,你的生活可以继续下去,不要再陷入其中了。我知道你回想整个事情得经过,还是会发现大量的谜题,但是那些已经和你无关了。
最后,作为临别的最后一句话,你要记好,那是你爷爷留下来的话语:
比鬼神更可怕的东西,是人心。

\hfill ——你的三叔 于敦煌
\] 

下面是很长一段事情经过的描述,和文锦说的几乎相同,我默默地看了下去,看完之后,眼泪就无法抑制地流了下来。

{\fzqiti\hfill (《盗墓笔记》第一季完)}