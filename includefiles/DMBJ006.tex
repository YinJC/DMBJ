%# -*- coding:utf-8 -*-
%%%%%%%%%%%%%%%%%%%%%%%%%%%%%%%%%%%%%%%%%%%%%%%%%%%%%%%%%%%%%%%%%%%%%%%%%%%%%%%%%%%%%
\part{盗墓笔记}
\chapter{后记}

各位,我终于写完了

我很难形容这个时候的心情,不算好,不算差,不算淡定,也不算激动。

真的很难形容

其实我在很久以前就一直在想,如果走到这一刻我的心情会是怎样的。我想过各种可能性,但是唯独没有想到会是现在这种——竟然连最基本的言语都表达不好

我想,也许因为,我对这一刻想的太多了,我的幻想反而超越了现实的感觉

不过,我拉开窗帘,看着北京阴郁的天空,我还是觉得,有一些东西已经改变了。

这是一段长达五年的拉力赛,不折不扣的五年,花费五年时间,写出九本小说,完成一个如此庞大复杂的故事,对于一个业余作者来说,确实有些太吃力了。我写到最后,已经不知道故事好不好,精彩不精彩。我只是想,让里面几个人物,能够实打实地走完他们应该走的旅程。事实上,这也不是由我来控制的。我在最后面临的最大的困境,是主人公已经厌倦了他的生活,我必须在这个故事中寻找让他还能继续往下走的饵料。

就在几分钟前,我让他们走完了,而且很平静。

在写完第四本的时候,我已经想好要写一篇很长的后记,把我写《盗墓笔记》的整个过程,心中的很多疑虑和想法,全部写出来。趁着很多的记忆还没有淡去,趁着所有的人物还在我心中活灵活现,我必须立即动笔。先说一些常规的事情。关于起源:说实话,我真的已经无法记起,当时写这本小说的初衷了。但是我知道,一定不是那种高尚伟大的想法。我从来不是有那种文字理想的人,我从来不想去告诉别人,我是一个什么什么家。我从小追求的东西,说白了是一种认可感,而讲故事恰恰是我比较容易获得认可感的途径。所以,虽然我无法记起,但是我几乎可以肯定的说,当时我落笔写下第一个三千字的时候,应该只是为了赢得一些喝彩而已。

这是一个非常低下的追求。很早之前,我都羞于启齿,因为那是多么世俗,虽然我明白,即使不是一个伟大的人,他也会因为很多人的幸福而去做一番事业,而我因为没有他们那样高尚的口号而变得惶惶不安,觉得自己的动机不纯。

《盗墓笔记》是源自一个民间故事,是我外婆讲给我听的。小时候这个故事给我的印象很深刻。

故事讲的是一个地主买了一个空的宅子,想在宅子的后院里种一些花草,结果发现无论种什么东西都活不下来,便去询问风水大师。风水大师说这院子底下似乎有问题,于是地主找来长工开始挖掘院子,挖到一半就开始见血,也不知道是真的血还是红色的泥水。最后在院子的地底下,挖出了一具雕花大棺材,不知道是谁的。他们把棺材放到了祠堂里,从此这个村子鸡犬不宁。不仅是地里东西不活,而且连地主家的人也快死绝了,四周的邻居家发生了各种奇怪的事情,于是只好继续找风水大师。风水大师看了之后,让他们在院子里继续挖,挖下去几十米,又挖出一具小一点的棺材。原来这是一个合葬墓穴,夫妻两个非常恩爱,但是因为妻子的棺材沉降得比较厉害,两具棺材在底下离得越来越远,怨气就越重。村长重新找了一个风水宝地,在地下铺设了石板,放下了这两具棺材,再次将他们合葬,一切才平息下来。我把这个故事展开了更多的联想,使用了里面的元素写成了《盗墓笔记》的第001章。

我记得故事的第001章有三千多字,我只写了不到半小时,没有任何修改,我把它贴到可大家可以看到的地方,然后用衣领包着头,躲起来竖着耳朵,希望能听到一些喝彩的声音,满足自己的虚荣心。这一听就是五年,五年之中,我经历了改变,是自己之前完全无法想象的。而如今,我再回头去看之前那个自己认为非常低下的追求的时候,却发现那已经变成了当前最高尚的口号。史蒂芬在《黑暗塔》的序里曾经说过:我写这本书,赚了很多的钱,但是写作这本书最初的快乐,和钱一点关系也没有。五年之后我已经成了所谓的畅销书作家,但我很庆幸,我最开心的还是在网络上那个不起眼的地方,听到一些喝彩的声音的时候,而在写完的这一刻,我更加期待那个时候。

关于这本小说:其实,我想说的是,当我写第二本的时候,我已经有一种强烈的感觉,这已经不是一本小说了。

我总觉得有一个世界,已经在其他地方形成。因为我敲动键盘,那个世界慢慢地长大、发展,里面的人物也开始有了自己的灵魂。

在我十三岁的那年,我看了大仲马的传记,里面写到了“人物都活了”。当时大仲马写《三个火枪手》的第三部的时候,里面的一个人物死亡,他边哭边写,把稿纸都哭湿了。我当时觉得特别的奇怪,怎样一种状态,才能让作者可以以这种方式去写自己的人物的死亡呢?

我尝试展开各种想象,都没有结果,一直到我自己开始写这本小说,并且,开始有意识的地赋予小说人物不同的性格赋予他们不同的人生经历。慢慢地,我就发现,故事的情节开始出现一些我自己都无法预测的变化。

很快,这个人应该说什么话,应该做什么动作,我都无法控制了。我发现了一个非常有趣的现象,只要先建立一个场景,比如说大雨,把这些人物放到这个情景中去,他们会走到各自的位子上,做他们应该做的事情。

我无法把其中任意两个人的位置对调,因为那样会出现无法调和的违和感。就算我强行对调了其中两个人物的行为,我也会在日后的到了一个茶话会的现场,谁先说话,谁后说话,谁来活跃气氛,谁在神游天外,一切都已经有了定论。我什么都不用思考,只需要看着他们,就能知道故事情节的走向。他们真的活了。

在后来极长的写作过程中,我从一个作者,变成了一个旁观者。我在上帝的角度,观察每一个人的举动,慢慢地,我甚至能看到他们很多轻微情绪和行为的来历,是他们童年的某一次经历。比如我真的可以通过胖子抖烟灰的时的动作,看到他以往的一切,他的痛苦,他的沧桑,他的一切。一花一世界,一树一如来。我可以把一个场景不停地倒转、反复、在其中任何一个角度去观察,甚至能看到现场所有人的心理活动,几个人的情绪同时在我心中走过。我想很少能有人领略这种快感。

在写“大闹天宫”那一段的时候,我仿佛就在新月饭店的包厢里,我仿佛可以从楼上走到楼下,看着四周的人一片混乱。在飞溅的碎片中,打斗的人群中,我随时让一切停顿,随时倒转一个时间,随时贴着人物的内心,体会他们心中的所有情绪变化。我可以把眼前的一切以一秒一帧的慢速度,慢慢地往前推进,然后蹲在地上,看里面人物表情缓慢变化。这本书中的整个世界,对于我来说,是真实存在的。他的每一个细节都是真实的,是无法改变的。我已经建成的部分,坚固的犹如现实。虽然说我是这本小说的创作者,但是当一切都走上了轨道,我对于这个小说的世界,开始有了极度的敬意。

关于小说的故事:最早发生的事情,是在长沙的镖子岭。

新中国成立初期,几个盗墓贼从战国古墓中盗出了本书中最重要的物件——战国帛书。

这是吴邪爷爷上一代也就是狗五爷年少时候的故事。当时还没有江湖上的排行,比较有名的一共九个人——陈皮阿四、狗五、黑背老六、等等,其中最末的是解阿九,也就是解连环的老爸。

后面也有所谓的十爷、十一爷,那被认可的范围就很小了,都是自己或者手下的人封的,说到外面别人都不知道。有人说陈皮阿四现在九十多了,五十年前他也四十多了,而但还是狗五还不大,如果他当时十七岁,年少成名也得十年,那时候也就二十七,如何能排在年近五十的陈皮阿四后面,成为狗五?如此排下去,解小九当时岂不是还在穿开裆裤?这有点无理取闹。

有点常识的都知道,江湖上排的不是年龄,而是资历和辈分,而且这些都是人家给排的。

吴邪爷爷狗五排的如此高,可见当时他的手腕和魄力是多么厉害,让人不得不服。

第二个故事,同样发生在镖子岭。那是吴邪三叔夜盗血尸墓截了美国人胡的那件事情,是发生在第一个故事后二十到三十年,这件事情可以说完全巧合,而且吴邪三叔也由此知道了当年吴邪爷爷他们第一次盗血尸墓时发生的事情,这一次冒险,三叔上升了若干经验值,得到了一颗奇怪的丹药。

虽然这只是一个插曲,但是这件事情可以说是之后西沙事前的起因。

第三个故事,发生在西沙的外海。这也就是吴邪三叔怒海潜沙的故事了。

张起灵的出现形成了这个故事中最大的谜团,故事中有两个版本,一个是三叔忽悠版本,另一个是三叔经历浩劫后的坦诚版本。

最后的真相是,两个版本都是三叔骗吴邪的。因为在三叔心中,还有一个巨大的秘密,而这个秘密和无邪有关。

第四个故事,发生在山东的七星鲁王宫。这是本作品的第一个故事,也是吴邪第一次下地,经历过这一次后,吴邪从坚定的无神论者变成了神经病患者,参与到这种犯罪活动中实在是好奇心作怪,在这个故事中,靠闷油瓶力挽狂澜吴邪等人最终逃出生天。

由此,之前的三个故事通过这个故事有机会融合到了一起。

战国帛书、西沙事件、莫名的丹药等几条线索聚合,整个故事开始变得极其扑朔迷离。

第五个故事,重新回到西沙。这一次是吴邪自己进入汪藏海的海底墓穴,寻找消失在墓穴中的三叔,此时的三叔,已经从海底墓得到了天宫的线索,开始了云顶天宫计划,而吴邪等人还像傻瓜一样,进入海底古墓。这一次与汪藏海相隔千年的博弈,最后还是王胖子不拐弯的思维,让吴邪等人再次活了下来。

在这个故事中,本作品中的三股力量终于汇聚到一起,谜团开始发展。追求真相的吴邪等人,有着自己计划的三叔以及前几个故事中阴魂不散的海外力量,在这里第一次面对面地开始了较量。在两条主线中,故事顺着汪藏海千年前写好的剧本发展下去,而另一条暂时中断了。

第六个故事,就是秦岭神树。这是诟病最多的一个故事——编辑们认为最好、最有文学性,而读者认为不知所谓的一个故事。

这个故事和主线关系不大,只是引出了山底下巨大的青铜古迹,同时也让主角的能力得到了提升。在这个故事中,吴邪独立带领着心怀不轨的童年好友,深入到秦岭深处。

这个故事对于吴邪来说,有时候想想,好比是一个长长的梦,大有不真实的感觉。第七个故事发生在长白山,永远的云顶天宫。这是最艰难的探险,也是吴邪写的最痛苦的一篇。各路人马带着各自的谜团走上死亡之路,漫天的白雪,狭窄雪域中的痛苦跋涉。在哪里,吴邪等人找到了一千年前汪藏海试图留给后人的终极秘密。然而,这个秘密在地底巨大的青铜门之前戛然而止。

进入地底巨门中的张起灵似乎是唯一一个最贴近这个秘密的人,汪藏害的主线到这里就停止了,铁面生的主线重新开始。

第八个故事,就是蛇沼鬼城故事。由线索拼接成的两个故事,贯穿了整个蛇沼鬼城故事。第一个是汪藏海的传奇。吴邪整理出来之后,发现是绝好的小说题材,用古龙的风格来写,必然是一本奇书,吴邪有生之年一定要把它写出来。第二个是现在慢慢形成的铁面生的故事。现在你可以清晰地看到故事的原点——山中巨大青铜神迹和蛇沼鬼城背后的秘密。

历史上,有两个超越时代的人窥得了这个秘密:一个是战国时代的铁面生,另一个就是明初的汪藏海。从现有的资料来看,吴邪等人并不知道他们之间有没有直接的联系,但可以看到的是,铁面生应该有更加丰富的资料,毕竟他的时代离神话时代十分近。从他们的墓穴中都有那种丹药来判断,两个人应该有共同的地方。最起码,两个人都将自己的经历以某种形式流传了下来——战国帛书和蛇眉铜鱼。

而吴邪等人正是追寻着这两个线索,逐渐揭开了这个扑朔迷离的面纱。关于汪藏海、鲁王宫、格尔木和云顶天宫,是另外一套和张家古墓楼关系非常密切的体系,张家的祖先有关系。而如陈皮阿四倒吊镜儿宫打苗人的故事,那是凑字数的。

关于拖稿:作为一个作者,最大的外来痛苦,一定是出版周期的压力和自己写作质量之间的矛盾,特别是当你已经对赶稿这件事情无比熟悉之后,你知道,这是不可调和的。但是,只要你面临这种痛苦的时间够长,你就会发现,这并不是什么难以忍受的事情。真正难受的,是当你承受完这些痛苦之后,还要承受更多的不理解。

但是我还是在一如既往的拖稿。我是一个慢手,特别是到了后期,写作速度会越来越慢。

倒不是因为不写,而是因为,长篇故事越写到后面,前方的信息就越多,越需要顾虑,等你写到五本之后,前面基本的线索谜题就会变成大山压在你的身上,让你毫无办法每走一步都无比艰难。

在这种情况下,很多时候,我只能选择稳妥的写作速度。然而,因为写作缓慢,我遭到了很多骂名。这些骂名一本书一本书地积累,慢慢地淹没掉了我以前能听到的喝彩声,慢慢地变成了主流。

我不可能违心的说,我的心在面对这些话语的时候,一直是淡定的。任何人,在初期面临那么多非议的时候,都会怀疑自己的价值。“原来有这么多人不喜欢我。”我当时心中的沮丧可想而知,“江郎才尽”“不负责任”,无数责言满天飞舞。

我只为喜欢我的人写,我当时很想撂下这么一句话,但是我做不到。慢慢地,我与这些信息的焦虑开始侵占我的一切。那一年,我不知道是用什么方法,慢慢地静下了自己的心,我要感谢我的朋友们,其中有一位早已成名,早就经历过这一切的朋友,她告诉我,写作就是一种修禅。写作就是一个凝视内心的过程。

我担心失去的那一切,对于以前的我来说,是不存在的。所以,我失去的东西,只是我不应该得到的。

我并没有失去写作之前所拥有的一切,就好像一个孩子从一棵苹果树上摘了十个苹果下来,发现其中三个是腐烂的一样。他不应该为失去了三个苹果而沮丧,而应该看到另外七个的完好。

语言有一些力量,我是慢慢地自己懂得了这个道理:情绪是一种不可以定量的东西,伤心就是伤心,开心就是开心。我写作是为了寻找我最初的快乐,如果因为小小的失去,就拿出自己百分百之百的伤心来,那是很不值当的。

不过,虽然我的心中对于拖稿有着自己的无奈和坚持,但我还是要在这里向我所有的读者道歉。五年的等待,似乎是人生中一个小小的轮回,我为你们在这等待中所有的痛苦道歉。同时,我也希望在这五年的等待中,这套小说能变成一段回忆。

五年是人生中一段不长不短的日子,如果有一个胖子能让那么多人在自己宝贵的人生中纠结五年,这个胖子个算是功德圆满了。所以即使是痛苦的,我道歉的同时,也会暗自窃喜。

我为什么喜欢故事呢?先来说说我的人生吧。一九八二年二月二十日,我生于浙江的一个小镇,子夜出生,出生的时候无论是天空大地还是海洋都没有任何反应。

有事想想,我多少有点埋怨老天爷,因为就算是出生的时候,天上打了个雷,我也能有理由认为自己一定是和其他人不一样的。可惜,回不去了。

我只能作为一个真真正正的普通人,在这个世界上混混日子。我的家庭出身相当复杂。我奶奶是江苏泰兴人,和我的出版商还是老乡。我奶奶是一个船娘,也就是说,她没有产业,她所有的财产九十一艘小木船。我爷爷在我父亲五岁的时候就去世了。我父亲有一个哥哥,一个姐姐。

我并不清楚我爷爷去世的原因,我父亲也不知道,只是隐约知道,我奶奶应该算是我爷爷的童养媳。奶奶其实有很多孩子,当时都没有养活,我的父亲是最小的一个,所以格外疼爱。六十年代的时候,因为饥荒,我奶奶的船从泰兴出发,前往上海,在黄浦江上,他的船因为和大船相撞而沉了。我奶奶带着三个子女,上岸那一刻他们痛哭流涕,他们生活的家没有了,如今来到陆地上,看着茫茫的上海滩,她能感觉到的,只是无比地开具。感谢党和人民,我奶奶得到了安置。

在我父亲的记忆中,有一段特别安宁美好的旧上海的记忆。我算过,如果当时我的父亲没有上岸的话,他也许就不会上学,也许就不会有后面的事情。不知道是什么原因,我父亲后来离开了上海,来到浙江省靠近上海的这一带活动,之后“文化大革命”开始,我父亲跟着铁道兵进大兴安岭支边,在建设兵团度过了自己最宝贵的青春。我的母亲当时也是从南方去北方支边的青年之一。

我的母亲非常漂亮,当时只有十六岁,和另外三个南方姑娘一起被称为大兴安岭的四朵金花,被担任事务长的父亲,用特供的白米饭追到了手。当时他们这一对,应该是相当光彩耀眼的一对。在建设兵团,人们都以地域划分派系,宁波、温州、丽水都有自己的小团体,期间冲突不断。我父亲从小就能打架,尤一寿混不吝的打架功夫。

我母亲说,当时我父亲身上几乎没有一块地方时没有伤疤的。因为能打架而且讲义气,我父亲在所有团体中都有威信。只要有人打架,我父亲一出现,所有人都不再吭声。一直到回到南方以后,有一次我父亲押了一船西瓜,遇到乱民抢西瓜,父亲在船上用一根篙子把几十个乱民全部打落下水,虽然最后寡不敌众只能弃瓜而走,但是他当时的雄风,我想起来就觉得过瘾。加上我母亲是惊人地清秀美丽,两个人在当时还是相当被人嫉妒的。

说道我母亲,他的家族更加有意思了。我外婆是我们老家一个叫做千窑之地的窑主。千窑有一千个窑口,是当时的核心产地。当时我外婆在当地拥有一个大窑,属于非常有地位的阶层。我外公是从国民党的壮丁中逃出来的。一直等到新中国成立以后,经人介绍两个人才成了一对。

我外婆和外公的故事一定也有千千万万。当时我外公天生神力。一米八六的个子,在当时的社会简直犹如巨人一般。我外婆说之所以会嫁给我外公,是因为看到外公一个人抬起三人才能抬起的东西。

当然,似乎这段婚姻之中也有很多插曲。我外公去世的时候,我隐约听到外婆在灵堂里伤感的和我母亲述说我外公以前的风流韵事。我看过我父母当年的照片,我的父亲英俊的让人无法直视,而我的母亲,现在看来都是出水芙蓉一般。他们是那么的美丽优秀,以至于我每次照镜子,都觉得世界是多么的不公平。那么多优良的基因,到了我这里,竟然表现得那么猥琐。

我父母在大兴安岭确立了关系,之后调到了大庆油田,之后又回到了南方。我父亲当时是供销系统的副食品经理,可谓手握物资大权,所以我家算起来还算是不错的。之后,在一个啥特色也没有的夜晚,我就被生了下来。写到这里,很多人会觉得有意思,也有一部分人会觉得无聊,觉得这都是什么跟什么,说这些有意义么?其实是很有意义的。

我是想告诉各位,我的奶奶,我的外婆外公、我的父亲母亲,都是极会讲故事的人。当我作为两个家族的第一个孩子诞生下来,在那个没有电视、没有电影、没有网络、没有小说的年代,我如何度过我的童年的呢?

讲故事。我从小就是在一圈故事达人的看护下长大的。民间故事、战争故事、童话,我的童年充满着这些。有些故事,现在听起来都非常有感染力,好多我都直接用在了《盗墓笔记》中。

我在那个时候已经确定们所有最初的乐趣,只能来源于故事。这也是后来我对故事着迷的最基础的与原因,因为我能百分之一百地享受到故事能够传达的乐趣。之后我的人生,穷极形容就是“无聊”二字,在各方面都失败,用现在的话说,可以被称呼为废柴。有人说,一个人生下来,上天总会给予一些特长让他可以帮助他人。然而,在很长一段时间里,我真的就觉得自己任何特长都没有。在我的朋友圈里,总有这样的现象:成绩好的学生,体育一般都不会太好;如果体育好的学生,成绩一般都不怎么样;成绩和体育都好的学生,一般都长得丑;成绩和体育都好,长得又不丑的同学,一般都会早恋然后被开除;成绩和体育都好,长得不丑,而且特别规矩不早恋的同学,后来都变成了gay了。我想说的是什么呢?我想说的是,我和上面一点关系都没,就是这个社会的悲哀。

从来没有人关系一个体育和成绩都不好,而且长得丑且到处逃课不守纪律的孩子。很多时候午夜梦回,我都觉得上帝是那么不公平,我身边所有的人都有传奇的人生,为何我的人生是这个样子的?

当时我身体不太好自从小学时有一次考试晕倒在考场上之后,每次考试老师都对我重点盯防,会把我安排在通风且温度适宜的地方。这个地方一定是全考场的风水宝地,老师监考的时候,除了巡视之外,都一定会到那个地方休息,且经常顺便来问我的身体状况,生怕我死在考场上,所以作弊这一套也行不通了。而旅游啊,运动啊就更和我没缘分了。我天生长了一对渔民脚——脚趾很长,而且大脚趾最长,懒洋洋游泳的时候特别有用,可是一旦需要爆发力的时候就完全没用了。加上只要太阳稍稍大一点,就很容易忽然到地口吐白沫,体育老师看到我就好像看到了校长儿子一样,呵护备至。所以我的大部分体育课,都是在树荫下,穿着白衬衫手捧小说度过的。对于我自己来说,早期这样的生活还是相当惬意的,除了被球场上的帅哥踢出的香蕉球击中闹大从楼梯上滚下来以外,我还是特别喜欢那些安静的、不出汗看书的日子。我想很多人都有我这样的经历,但是未必有我这样的绝对。

那个时候,我几乎所有的时间都在看小说。我把图书馆掏空之后转向民营的小书店,从书架上的第一本看起。本本都是花钱借,很快钱就不够用了。对于毫无特长的我来说,赚取生活费这种事情简直是天方夜谭,我便开始赖在书店看书,但是通常是看三本借一本,因此老板也不好意思赶我走,因为我初期到底是个大客户,之后虽然借的少了,但频率高啊,总量还是不错的。我觉得我的情商就是在这个时候培养起来的。到初中结束,我已经再没有书可以看了,便开始自己写一些东西。虽然质量都不高,但是在完成一轮正规的小说阅读之后,我忽然有一种很强的欲望——我想自己写一篇小说。当时的这个想法和任何的梦想都没有关系,我压根不想成为一个作家,当时我只是觉得写出一个好看的故事,能让所有人在我背后抢着看,是一件多么拉风的事情啊。

那一年,我开始真正动笔。从最开始的涂鸦写作,到自己去解析那些名家作品,缩写、重列提纲、寻找悬念的设置技巧、寻找小说的基本节奏,仅仅两个月的时间,我便慢慢地发现,我写出来的小说,越来越有样子了。可是,我还是不敢投稿,废材的人生让我很难鼓动自己走出这一步。当时还没有电脑,我使用纸和笔,在稿纸上写作。慢慢地,我就开始沉迷进去了。我荒废了学业(反正也没什么成就了三苏原话),到大学毕业,我写作的总字数超过了两千万字,大部分都是写在各种废弃的作业本上。我是一个换作业本特别勤的人,因为我的作业本前头是作业,后头往往就是我写的小说。这能方便我在上课的时候写作,往往两三节课,我就能把一个本子全部写完,那第二天写作业,织好换一个新的本子了。

说真的,现在回头去看我写的东西没有一部分的水平还是能让我自己咂舌的,不仅仅是能和现在想媲美,很多作品甚至写的比现在的还要好。因为当时我注重文笔和语句,而现在的我已经是个老油条了,知道把意思表述清楚就很足够了,往往懒得在文字上多琢磨。在整个写作过程中,我有一个特别明显的特征,就是只写故事。

那时候的故事种类非常多,我写武侠、写悬疑、写爱情,甚至很早我就开始写一下现在比较流行的类型,比如穿越类型的小说。但是和其他的文学爱好者不同,我只想写故事,我最希望听到的一句话是:“后面呢?后面写了吗?”因为,这是对于我故事的最好的评价。

在出版《盗墓笔记》之后,有很多人问过我一个问题:你是否觉得你的成功有运气的成分?我想说,没有任何一次成功是没有运气的成分。有一些好运气总是好的,虽然人最需要的并不是运气。很多时候我们也知道,运气其实并不能帮你太多,即使你中了彩票,如果你没有能力处理句子,手上的钱也会很快变成大麻烦。人需要的,其实是抓住机会的能力。

决定写《盗墓笔记》的那一刻,我带着一种并不在意的心态,这种不在意能够吸引很多人来看,其中,应该是有那两千万字的功劳。所以,如果真的要说我的运气在哪里的话,我觉得我的运气是来自我不聪明、成绩不够好、体育不够好,但是老天爷偏爱长得丑的。如今的一切,我接受得很坦然,和运气天赋第一没有关系,我只是一直被故事牵着鼻子走而已。

我想说的是,如果这个人很喜欢吃东西,他从童年开始就深陷吃东西之中,吃到三十岁,那她也是可以成功的;如果这个人很喜欢打架,他从童年开始就喜欢打架,打到三十岁,那他也是可以成功的。

喜欢一件事情,坚持做下去,总是可以成功的。说了一些客套话,大概后记该写的东西,现在来说一些外婆真正想说的。翻开这一页,要做一点心理准备。

吴邪:吴邪,是一个很难形容的人。如果一定要说,我想说:他其实,就是一个普通人。但这并不代表他不伟大,正因为是普通人,所经历的这一切,才让人那么佩服。我想,很多朋友在刚刚看到他的时候,一定会厌恶他的软弱,他的犹豫不决。然而,随着故事一步一步推进,喜欢他的人越来越多,他是一个柔弱的像水一样的男孩子,但是请不要忘记,在严酷的寒冬,最没有形态的水,也会变成坚固的冰。吴邪就是这么一个人。他单纯,有一些小小的聪明;他懦弱,珍惜自己的生命;他敏感,害怕伤害身边的人,他是在所有的队伍中,最不适合经历危险的人。然而,我却让他成为了这个故事的主角,去经历一段最可怕的旅途,这可能也是这个故事最最特别的地方。在所有人可以退缩的时候,他恰恰不能退缩;在所有人可以逃避的时候,他却不能逃避。

我很想和他说声对不起,把这个普通人推进了如此复杂的迷局烦恼。有一段时间,我能深深地感觉出他心中对于一切的绝望,当时我很想知道,他这样一个普通人,在面对如此庞杂的绝望时,他会如何做。我没有想到他能撑下来,在故事的发展中,大家都看到了一个普通人如何在挣扎中成为一个他不希望成为的人。而让所有人喜欢的是,在所有可以成为他人生拐点的地方,他都保持了自己的良知,即使他最后带着一张穷凶极恶的面具,他的内心还是吴邪。他可以有很多的小奸小恶,可以有很多的小道德问题,但在他面临最大的抉择的时候,他永远还是那个希望所有人都好的吴邪。

“我希望这一路走来,所有人都能好好地活着,所有人都可以看到各自的结局。我们也许不能长久地活下去,请让我们活完我们应该享有的一生。”吴邪在潘子的弥留之际向天际祈祷,虽然他身处漆黑一片的山洞中。他把所有的责任都归咎于自己,他无法面对自己一路走来的意义。这就是吴邪,在队伍中拥有的“白搭”,铁三角中最废材的领袖,他需要别人的保护,需要别人的帮助,他有无穷的好奇心和欲望,但是只要有一个人受到伤害,他自己的一切就都不重要了。他是一个无论多么恨你,都希望你可以活下去的普通人。因为他不懂杀戮,不懂那超越生命的财富,他只懂得“活着”二字的价值。

闷油瓶:这是一个强大的有如神佛一般的男人。有他在的篇幅中,我总是能写的格外轻松,因为只要他在身边,就能为你挡下一切的灾难和痛苦他没有言语,不会开心,不会悲痛,他总是像一个瓷娃娃一样,默默地站在那里,淡淡地看着一切,然而,你知道他是关心着你的。永远没有任何一个人可以像他那样,给你带来那么多的安全感。然而,不知道为什么,我在书写这个男人的各种举动时,心中总是反着一股深深的伤感。

正如自己所说的,他是一个没有过去和未来的人,他和世界的唯一的联系,似乎并没有多少价值。他不知道自己来自何方,不知道自己将要去往哪里。他只知道,自己在这个世界上,有意见他必须要做的事情。“你能想象么?有一天,当你从一个山洞中醒来,在你什么都不知道,疑惑地望着四周的时候,你的身上已经有了一个你必须肩负的责任,你没有权利去看沿途的风景,不能去享受朋友和爱人,你人生的中所有美好的东西,在你有意识的一刻,已经对你没有了意义。”

张起灵就是这样默默地背负着自己的命运。最让我心痛的是,他只是淡淡地背负着,好像这一切都理所当然,好像这只是一件无关紧要的小事。如果你问他,他只会默默地摇头,和你说:“没关系。”这就是我写出来的这个男人。他背负着世界上最痛苦的命运,甚至比死亡还要痛苦一千倍,然而他不怒不帅,既不逃避也不痛苦。他就在那里,告诉你他所保护的所有人,没关系。

在《盗墓笔记·捌》的结尾,我让他再次沉睡,十年之后,才有再次唤醒他的机会。这也许不是一个很好的结局,对于所有的人来说都不是。但是,对于他来说,我真的想不出更好的结局。

胖子:胖子是一个粗中有细的人,整体来说,我认为他是一个细的人,甚至在很多层面上,他比吴邪更细一些。胖子给人的印象一直是嘻嘻哈哈的,并且总是闯祸。他有自己的臭毛病,但是我还是觉得,他是三个人当中最正常的一个人。也就是说,要选人做老公的话,这三个人当中,只有胖子可以胜任。如果说无邪是那种逃避痛苦的人,小哥是那种无视痛苦的人,那么胖子是唯一可以化解痛苦的那种人。在这些人当中,无疑胖子是承受过最多痛苦的。所谓的承受,是指胖子他能够体会到痛苦对自己的伤害,而不是像小哥那样,无尽的痛苦穿过身,他只是点头致意。一个能够理解痛苦而又承受了那么多痛苦的人,并且将其一一化解,真正地发自内心开心快乐的人,我们几乎可以称之为佛了。是的,胖子就是那个看穿一切的佛。

在某种程度上,在他的谈笑中所蕴含的东西更多。他拍着天真的肩膀,说出那一句“天真无邪”,已经是将吴邪看的通透无比,他能够默契地和小哥点头包抄任何危险,说明他也完全理解小哥内心的那一片空白。然而,在最后,胖子终于承担不了了。云彩死了之后,他强大的内心还可以化解那强烈的悲痛吗?他发现,他的心中不愿意化解了,他不想这段痛苦和他以前那些痛苦一样,最后变成了那一片空灵。胖子选择了让这段痛苦和自己永远在一起。

我写胖子抱着云彩的尸体痛哭流涕,对吴邪道:“我是真的喜欢,我从来没有开过玩笑。”我的眼泪也无法止住地流了下来。我很后悔,没有在前面,为他和云彩多写一些篇幅,让他和云彩可以有更多回忆的东西。对于胖子来说他的爱是简单的,喜欢就是喜欢了,没有那么多理由,不需要那么多相处。

铁三角:我不知道他们之间的感情是说明,是朋友吗?我觉得,他们已经超越了朋友的关系。他们有着各自的目的,到了最后,却又都放弃了各自的目的;是亲人吗?我觉得也不是他们疏离着,互相猜测着,然而这种疏离,又是一种默默的保护。所有的一切,好像都是出于最基本的感情:我希望你能平安,不管是吴邪千里追踪规劝闷油瓶,还是胖子不图金钱帮吴邪涉险,还是闷油瓶屡次解救他们两人而让自己身陷险境。“这是我的朋友,请你们走开,告诉你们老板,如果我的朋友受到任何一点伤害,我一定会杀死他,即使跑到天涯海角,我也能找到他,反正我有的是时间。”闷油瓶淡淡地说出了这句话,身后是不知所措的胖子和吴邪。

“我告诉你们,就是他以后想把我所有的产业全部毁掉,我也不会皱一下眉头。这是我吴家的产业,我想让他败在谁的手上,就败在谁的手上。我今天到这里来,不是来求你们同意这件事情,而是来知会你们一声。谁要再敢对张爷说一句废话,犹如此案!”吴邪用他不完全结实的拳头,砸穿了书桌。那一刻,他的愤怒没有让他感觉到指骨碎裂时的剧烈痛苦。“胖爷我就待在这里,只有两个人可以让我从这里出去,一个是你天真,一个就是小哥。你们一定要好好地活着,不要再发生任何要劳烦胖爷我的事情了,你知道胖爷年纪大了。当然,咱们一起死在斗里,也算是一件美事。如果你们真的有一天,觉得有一个地方非去不可并且凶多吉少的话,一定要叫上我,别让胖爷这辈子再有什么遗憾。”这就是铁三角。

{\fzqiti\hfill (《盗墓笔记》全书完)}
